\documentclass[../main/main.tex]{subfiles}

\raggedbottom

\makeatletter
\renewcommand{\@chapapp}{M\'ecanique -- chapitre}
\makeatother

% \toggletrue{student}

\begin{document}
\setcounter{chapter}{7}

\chapter{M\'ecanique du solide}
On veut généraliser les lois des la mécanique du point matériel à des systèmes
constitués de plusieurs points matériels, ici pour des solides indéformables.

\section{Système de points matériels}
\subsection{Systèmes discret et continu}
Un solide peut être vu comme un ensemble de points matériels auquel on peut
appliquer le PFD. On en distingue deux types~:
\begin{tdefi}{Définition, sidebyside}
    \begin{center}
        \fbox{Système discret}
    \end{center}
    Un ensemble de $n$ points matériels M$_i$ de masses $m_i$
    FIGURE
    \tcblower
    \begin{center}
        \fbox{Système continu}
    \end{center}
    Un ensemble d'éléments de volumes $\dd{V}$ de masse $\dd{m}$, de position
    $\Mr$.
    FIGURE
\end{tdefi}

Sauf cas particuliers, on considèrera des systèmes \textbf{discrets} et
\ul{fermés} (tous les points restent dans le système).

\subsection{Centre d'inertie}
\begin{tdefi}{Définition}
    Le \textbf{centre d'inertie} ou \textbf{centre de gravité} G d'un ensemble
    de points matériels M$_i$ de masses $m_i$ telles que $m\ind{tot} = \sum_i
    m_i$ est défini par~:
    \[
        \boxed{m\ind{tot}\vv{\rm OG} = \sum_i m_i \vv{{\OMr}_i}}
        \Lra
        \boxed{\sum_i m_i\vv{{\rm GM}_i} = \of}
    \]
    Il s'agit du barycentre des points du système, pondéré par leur masse.
\end{tdefi}

% Le centre d'inertie correspond au centre d'équilibre gravitationnel d'un
% ensemble de point. Ainsi, pour mettre une règle en équilibre horizontalement il
% faut la reposer en son milieu. Un marteau, en revanche, a son centre d'inertie
% bien plus proche de la masse que du milieu.
% 
% Il existe une autre formulation à cette définition. En effet,
% \begin{align*}
%     \sum_i m_i\vv{\rm OG} &= \sum_i m_i\vv{\OMr_i}
%     \\
%     \Leftrightarrow
%     \of &= \sum_i m_i \left( \vv{\OMr_i} - \vv{\rm OG} \right)
% \end{align*}
% Or, comme $\vv{\OMr_i} - \vv{\rm OG} = \vv{\rm GO} + \vv{\OMr_i} = \vv{\rm
% GM_i}$, on pourra également retenir~:
% \[\boxed{\sum_i m_i\vv{{\rm GM}_i} = \of}\]
% 
% \begin{rexem}{Exemple}
%     Soient 2 masses $m$ placées en A et en B.
%     \begin{center}
%         \begin{tikzpicture}[]
%             \coordinate (A) at (-2,0);
%             \coordinate (B) at (2,0);
%             \coordinate (G) at (0,0);
%             \node[] at (A) {$\bullet$};
%             \node[below] at (A) {$m$};
%             \node[above] at (A) {A};
%             \node[] at (B) {$\bullet$};
%             \node[below] at (B) {$m$};
%             \node[above] at (B) {B};
%             \node[] at (G) {$\times$};
%             \node[above] at (G) {G};
%         \end{tikzpicture}
%     \end{center}
%     \leftcentersright{On a~:}{$m\vv{\rm GA} + m\vv{\rm GB} = \of
%     \Lra \vv{\rm GA} + \vv{\rm GB} = \of$}{car $m\neq0$}
% 
%     Or, $\vv{\rm GA} + \vv{\rm GB} = \vv{\rm GA} + \vv{\rm GA} + \vv{\rm AB}$,
%     donc
%     \begin{align*}
%         2\vv{\rm GA} + \vv{\rm AB} &= \of
%         \\
%         \Lra
%         2\vv{\rm AG} &= \vv{\AB}
%         \\
%         \Lra
%         \Aboxed{\vv{\rm AG} &= \frac{1}{2}\vv{\rm AB}}
%     \end{align*}
%     Avec $3m$ en A et $m$ en B~:
%     \begin{center}
%         \begin{tikzpicture}[]
%             \coordinate (A) at (-2,0);
%             \coordinate (B) at (2,0);
%             \coordinate (G) at (-1,0);
%             \node[] at (A) {$\bullet$};
%             \node[below] at (A) {$m$};
%             \node[above] at (A) {A};
%             \node[] at (B) {$\bullet$};
%             \node[below] at (B) {$m$};
%             \node[above] at (B) {B};
%             \node[] at (G) {$\times$};
%             \node[above] at (G) {G};
%         \end{tikzpicture}
%     \end{center}
%     \leftcentersright{Cette fois,}{$3m\vv{\rm GA} + m\vv{\rm GB} = \of
%     \Lra 3\vv{\rm GA} + \vv{\rm GB} = \of$}{car $m\neq0$}
% 
%     Or, $3\vv{\rm GA} + \vv{\rm GB} = 3\vv{\rm GA} + \vv{\rm GA} + \vv{\rm AB}$,
%     donc
%     \begin{align*}
%         4\vv{\rm GA} + \vv{\AB} &= \of
%         \\
%         \Lra
%         4\vv{\rm AG} &= \vv{\AB}
%         \\
%         \Lra
%         \Aboxed{\vv{\rm AG} &= \frac{1}{4}\vv{\rm AB}}
%     \end{align*}
% \end{rexem}
% 
% Cette définition peut être étendue aux solides qui peuvent être vus comme un
% ensemble infini de points infiniment proches. Dans ce cas, la somme discrète
% devient une intégrale.

\subsection{Mouvements d'un solide indéformable}

\begin{tdefi}{Définition}
    Un solide \textbf{indéformable} est un ensemble de points tels que la
    distance entre deux points quelconques soit constante~:
    \[\forall (\Mr_1,\Mr_2) \in ({\rm solide}), \quad {\rm M_1M_2} = \cte\]
\end{tdefi} 

Un solide peut avoir un mouvement complexe. Dans le cadre du programme, on se
limite à deux situations.
\begin{itemize}
    \item la translation : tous les points ont le même vecteur vitesse. Dans ce
        cas, la connaissance du mouvement d’un des points permet de connaître le
        mouvement du solide. On a déjà vu le cas de la translation rectiligne
        (véhiculé en mouvement sur une route rectiligne), de la translation
        circulaire (cas de la grande roue), de la translation parabolique (cas
        d’un objet en chute libre sans mouvement de rotation initial).
    \item la rotation : tous les points ont un mouvement circulaire autour d’un
        axe z. C’est le cas d’un rotor de moteur. Attention, dans le cas d’une
        roue de voiture, l’axe de rotation n’est fixe que dans le référentiel de
        la voiture. Dans ce cas, la vitesse de chaque point est 
        \[\vf = r\tp\ut\]
        où r désigne la distance à l’axe de rotation, $\tp$ la vitesse angulaire.
\end{itemize}

\subsubsection{Translation}
\begin{tdefi}{Définition~: translation}
    Un solide en mouvement est en \textbf{translation} si son
    \textbf{orientation est fixe} au cours du mouvement. Ainsi, de manière
    équivalente~:
    \begin{enumerate}
        \item $\forall (\Mr_1,\Mr_2) \in ({\rm solide}), \quad \vv{\rm M_1M_2} =
            \vcte$~;
        \item $\forall t$ et $\forall (\Mr_1,\Mr_2) \in ({\rm solide}), \quad
            \vf(\Mr_1) = \vf(\Mr_2)$.
    \end{enumerate}
    Alors, la connaissance du mouvement d'un point du solide en translation
    permet de connaître le mouvement de tout point du solide~; on prendra
    habituellement le centre d'inertie.
\end{tdefi}

\begin{rexem}{Exemples}
    \begin{enumerate}
        \item Translation quelconque~:
        \item Translation rectiligne~: chaque point décrit une droite.
        \item Translation circulaire~: chaque point décrit un arc de cercle.
    \end{enumerate}
\end{rexem}

\subsubsection{Rotation}
\begin{tdefi}{Définition~: rotation}
    Un solide est dit en \textbf{mouvement de rotation} autour d'un \textbf{axe
    fixe} $\D$ lorsque tous ses points possèdent un \textbf{mouvement
    circulaire} autour de cet axe.
\end{tdefi}
\begin{tcoro}{Conséquence}
    Tous les points du solide $\Sc$ parcourent un cercle à la \textbf{même
    vitesse angulaire} $\w(t) = \tp(t)$~; ils sont donc animés avec les vitesses
    de rotations~:
    \[\vf_{\Mr/\Rc}(t) = r_{\Mr}\tp(t)\ut = r_{\Mr}\w(t)\ut\]
    Ce mouvement est décrit complètement par le \textbf{vecteur rotation}
    \[\Of_{\Sc/\Rc} = \w(t)\vv{u_\D}\]
\end{tcoro}

\begin{rexem}{Exemple}
    Rotation autour de l'axe $\D$ fixe~:
\end{rexem}

\subsubsection{Translation circulaire ou rotation~?}
\begin{table}[!h]
    \centering
    \begin{tabularx}{.8\linewidth}{YY}
        \toprule
        Translation circulaire & Rotation autour d'un axe fixe
        \\\midrule
        Tous les points suivent une trajectoire circulaire de même rayon mais de
        centre différent & Tous les points suivent une trajectoire circulaire de
        même centre mais de rayon différent.
        \\\bottomrule
    \end{tabularx}
\end{table}

\section{Rappel~: TRC}
\subsection{Quantité de mouvement d'un ensemble de points}

On souhaiterait pouvoir étudier un ensemble de points comme le mouvement d'un
point unique, comme le centre d'inertie. Pour cela, il faut étudier la quantité
de mouvement d'un ensemble de points.

\begin{tdefi}{Définition}
    Le vecteur quantité de mouvement d'un ensemble $\Sc$ de points matériels
    M$_i$ de masses $m_i$ est défini par~:
    \[\boxed{\pf(\Sc) = \sum_i\pf(\Mr_i) = \sum_i m_i \vf(\Mr_i)}\]
\end{tdefi}

Pour que les choses soient simples, il faudrait donc que $\pf(\Sc)$ soit relié
au centre d'inertie. Ça tombe bien~:

\begin{tprop}{Propriété~: quantité de mouvement d'un système}
    \centering\bfseries
    La quantité de mouvement d'un ensemble de points est la quantité de
    mouvement d'un point matériel placé en $G$ et de masse $m_{\tot}$~:
    \[\boxed{\pf(\Sc) = m_{\tot}\vf(\Gr)}\]
    Tout se passe comme si la masse était concentrée en G.
\end{tprop}

\begin{tdemo}{Démonstration}
    \begin{gather*}
        m_{\tot}\vf(\Gr) = m_{\tot}\dv{\vv{\rm OG}}{t} =
            \underbrace{\sum_i m_i \dv{\vv{\OMr_i}}{t}}_{\pf(\Sc)}
        \Lra
        \boxed{\pf(\Sc) = m_{\tot}\vf(\Gr)}
        \qed
    \end{gather*}
\end{tdemo}

\subsection{Forces intérieures et extérieures}
Si on peut étudier la cinématique d'un corps par l'étude de son centre de
gravité, comment les forces interviennent-elles sur cet ensemble de points~?
Les forces s'appliquant aux points M$_i$ de $\Sc$ se rangent en deux
catégories~:
\begin{enumerate}
    \item Les forces intérieures $\Ff_{\rint\ra\Mr_i}$ exercées par les autres
        points M$_j$ du système, avec $j\neq i$~;
    \item Les forces extérieures $Ff_{\ext\ra\Mr_i}$ exercées par une origine
        externe au système.
\end{enumerate}
Les forces intérieures ont cependant une propriété remarquable~:
\begin{tprop}{Propriété~: résultante des forces intérieures}
    \centering\bfseries
    La résultante $\Ff_{\rint}$ des forces intérieures d'un système est toujours
    nulle.
\end{tprop}
\begin{tdemo}{Démonstration, breakable}
    La résultante des forces intérieures exercées sur M$_i$ s'écrit
    \[\Ff_{\rint\ra i} = \sum_{j\neq i}\Ff_{j\ra i}\]
    Ainsi la résultante des forces intérieures au système s'écrit
    \[\Ff_{\rint} = \sum_i \Ff_{\rint\to i} = \sum_i\sum_{j\neq i}\Ff_{j\to i}\]
    Or, d'après la troisième loi de \textsc{Newton}, $\forall i\neq j, \quad
    \Ff_{j\to i} = -\Ff{i\to j}$~; ainsi, les termes de la somme précédente
    s'annulent deux à deux, et on a bien
    \begin{gather*}
        \boxed{\Ff_{\rint} = \sum_i\Ff_{\rint\to i} = \of}
        \qed
    \end{gather*}
\end{tdemo}
Rien de remarquable ne se produit pour les forces extérieures, et on aura
simplement
\[\boxed{\Ff_{\ext} = \sum_i \Ff_{\ext\to i}}\]

\subsection{Théorème de la résultante cinétique}
% Considérons pour simplifier un système de deux points M$_1$ et M$_2$ de masses
% $m_1$ et $m_2$ en mouvement dans un référentiel galiléen. On peut appliquer le
% principe fondamental de la dynamique à chacun d'entre eux~:
% \[\dv{\pf(\Mr_1)}{t} = \Ff_{\Mr_2\ra\Mr_1} + \Ff_{\ext\ra\Mr_1}\]
% avec deux types de forces~: les forces intérieures du système, ici celles
% exercées par M$_2$ sur M$_1$, et les forces extérieures, c'est-à-dire toutes les
% autres. En faisant de même pour M$_2$~:
% \[\dv{\pf(\Mr_2)}{t} = \Ff_{\Mr_1\ra\Mr_2} + \Ff_{\ext\ra\Mr_2}\]
% Ainsi, avec la définition de la quantité de mouvement d'un ensemble de points,

\begin{tcbraster}[raster columns=2, raster equal height=rows, raster
    valign=top]%
    \begin{tprop}{Théorème de la résultante cinétique, heart}
        Le principe fondamental de la dynamique pour un point matériel
        s'applique à un ensemble de point en prenant pour point matériel le
        centre d'inertie G affecté de la masse totale $m_{\tot}$ du système, en
        ne considérant que les forces extérieurs s'appliquant à l'ensemble~:
        \[\boxed{\dv{\pf\Rg(\Sc)}{t} = m_{\tot}\dv{\vf(\Gr)}{t} = \Ff_{\ext}}\]
    \end{tprop}%
    \begin{tdemo}{Démonstration}
        \begin{align*}
            \dv{\pf\Rg(\Sc)}{t} &= \sum_i \dv{\pf\Rg(\Mr_i)}{t}
            \\
                                &= \underbracket[1pt]{
                                       \sum_i \Ff_{\rint\to i}
                                   }_{=\of\text{ par 3\ieme\ loi}} +
                                   \underbracket[1pt]{
                                       \sum_i \Ff_{\ext\to i}
                                   }_{=\Ff_{\ext}\text{ par déf.}}
            \\\Lra
            \Aboxed{\dv{\pf\Rg(\Sc)}{t} &= m_{\tot}\dv{\vf(\Gr)}{t} = \Ff_{\ext}}
            \qed
        \end{align*}
    \end{tdemo}
\end{tcbraster}
\begin{tror}{, hand}
    Le mouvement du centre de gravité n’est affecté que par les forces
    extérieures au système. Ainsi, dans la suite, on étudiera le mouvement du
    centre de gravité, de masse $m_{\tot}$, soumis aux forces extérieures au
    système.
\end{tror}

\section{Énergétique des systèmes de points}
On l'a vu dans les chapitres précédents, différentes approches sont possibles en
mécanique selon le résultat désiré. Si le PFD permet d'avoir l'information
dynamique sur le centre d'inertie, on cherche à établir les résultats de
l'approche énergétique aux solides. Commençons par le plus simple~:
\subsection{Cinétique}
\begin{tdefi}{Définition}
    Comme pour la masse ou la quantité de mouvement, l'énergie cinétique d'un
    solide est la \textbf{somme des énergies cinétiques de chaque point le
    constituant}~:
    \[
        \boxed{\Ec_{c/\Rc}(\Sc) = \sum_i\Ec_{c/\Rc}(\Mr_i) = \sum_i
        \frac{1}{2}m_iv_{i/\Rc}{}^2}
    \]
\end{tdefi}

\subsection{Puissance intérieure}
Pour pouvoir appliquer les théorèmes énergétiques, il faut détailler les
puissances des forces s'appliquant au solide, et notamment les forces
intérieures. TBD.

\subsection{Théorèmes}

\section{Moments pour un système de points}
\subsection{Moment cinétique et moment d'inertie}
Mélange Olivier/Schweitzer

\subsection{Moments intérieurs}
Olivier

\subsection{TMC}
Olivier

Analogie Schweitzer

\subsection{Énergie cinétique de rotation}
Schweitzer

\section{Cas particuliers et application}
Schweitzer~: couple, pivot~; aucun moment = pas de frottements

Schweitzer pendule pesant

\end{document}
