\documentclass[a4paper, 12pt, final, garamond]{book}
\usepackage{cours-preambule}

\makeatletter
\renewcommand{\@chapapp}{Devoir maison -- chapitre}
\makeatother

\begin{document}
\setcounter{chapter}{-1}

\chapter{Commentaires sur le DM}

\begin{itemize}
    \item Utilisez des petits carreaux
    \item Ne rendez pas des feuilles simples même pas agrafées entre elles et
        même pas numérotées
    \item Numérotez les copies (pas les pages) dans le coin inférieur droit
    \item Pas de pochettes svp
    \item Gardez les sujets
    \item Écrivez de manière lisible
    \item \textit{N'écrivez surtout pas au crayon} (sauf les schémas)
    \item On n'écrit pas «~exo~» sur un rendu
    \item Sachez écrire les lettres grecques~: $\rho$ n'est pas $P$
    \item Séparez bien les exercices en soulignant les titres
    \item Soulignez ou encadrez vos résultats
    \item Si vous inventez une notation, introduisez-la
    \item Tout calcul est d'abord sous forme littérale puis une application
        numérique
    \item ÉCRIVEZ VOS APPLICATIONS NUMÉRIQUES AVE UNE UNITÉ
    \item N'écrivez rien d'inutile, et n'écrivez pas trop (ça vous évite de
        faire des fautes… dures à lire)
    \item Utilisez des mots de liaison
    \item On n'écrit pas «~$\times$~» ou «~X~» pour écrire «~fois~» dans les
        expressions littérales (et on n'écrit pas d'expressions numériques)
    \item Laissez une marge, $\leq \SI{4}{cm}$ pas plus
    \item On ne rend \textbf{jamais} un brouillon. S'il est rendu, il ne sera
        pas lu.
    \item Écrivez votre nom, prénom, date sur chaque copie
    \item Il fallait déterminer la relation entre $Q$, $c$, $m$ et $\Delta T$
        par analyse des unités
    \item Ne confondez pas l'analyse dimensionnelle avec les équations entre
        grandeurs
\end{itemize}

\begin{ror}[label=impo:règles, hand]{À retenir}
    Avant d'encadrer un résultat :
    \begin{enumerate}
        \item Vérifer la cohérence mathématique avec la ligne précédente : les
            signes devant les grandeurs, le nombre de grandeurs, ne pas oublier
            les fonctions inverses… ;
        \item Vérifier l'homogénéité de part et d'autre de l'équation pour les
            résultats littéraux ;
        \item Vérifier la cohérence physique de la valeur numérique, notamment à
            l'aide d'un schéma.
    \end{enumerate}
\end{ror}

\end{document}
