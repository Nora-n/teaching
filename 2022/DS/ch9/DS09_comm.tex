\documentclass[a4paper, 11pt, final, garamond]{book}
\usepackage{cours-preambule}

\titleformat{\item}{}{\arabic{item})}{.5em}{}{}
\titleformat{\subitem}{}{\arabic{item}) \alph{subitem} --}{.5em}
{}{}

\makeatletter
\renewcommand{\@chapapp}{Devoir surveill\'e -- num\'ero}
\makeatother

\begin{document}
\setcounter{chapter}{8}

\chapter{Commentaires sur le DS n\degree9}

\begin{NCprop}[width=\linewidth]{\centering\bfseries\ Rappel des malus}
    Chacune des lettres suivantes sur vos copies sont des malus de \num{1}
    point.\smallbreak
    \begin{minipage}{0.50\linewidth}
        \begin{itemize}
            \item A~: application numérique mal faite~;
            \item Q~: question mal ou pas indiquée~;
            \item P~: nom/prénom non indiqué~;
        \end{itemize}
    \end{minipage}
    \begin{minipage}{0.50\linewidth}
        \begin{itemize}
            \item U~: unité manquante ou mauvaise~;
            \item H~: homogénéité non respectée~;
            \item N~: numéro de copie non indiqué~;
            \item $\f$~: loi fondamentale de la physique brisée.
        \end{itemize}
    \end{minipage}
\end{NCprop}

\section{Commentaires généraux}

DS à 43, moyen. Certaimes ont compris, d'autres pas du tout, mais tout le monde
connaît 2-3 choses. Note moyenne à 10/20. Pas tant de malus, beaucoup de
non-malus, bravo. Plus grand gain de place par rapport au DS08~: \textbf{31}
(également 26). Plus grande perte de place~: -19 places. Vous noterez que tous
les exercices sont tirés de vrais sujet de 2022, voire \textbf{2023}~: ce que
vous venez de faire est proche de ce que vous aurez dans \SI{10}{mois}.

\begin{center}
  \begin{framed}
    \huge
    Adiabatique n'est pas isotherme~!!
  \end{framed}
\end{center}

\begin{center}
    \includegraphics[width=\linewidth]{res_DS09.pdf}
\end{center}
\vspace*{-20pt}

\section{Exercice 1 \hfill \textcolor{red}{/40}}
\begin{enumerate}
  \item Très aléatoire. Pour $T_1$, une compression réduit le volume. Pour
    $T_2$, diminuer la température diminue le volume à pression constante.
    \hfill \textcolor{ForestGreen}{/5}
  \item Citez les lois de \textsc{Laplace}. Très bien réussi dans l'ensemble.
    \hfill \textcolor{ForestGreen}{/8}
  \item Idem.
    \hfill \textcolor{ForestGreen}{/4}
  \item Quelques variations, sinon très bien. Pensez à citer le 1er principe en
    entier.
    \hfill \textcolor{ForestGreen}{/5}
  \item RAS.
    \hfill \textcolor{ForestGreen}{/5}
  \item $\Delta{U}$ n'est pas nulle que pour un cycle~: sur une isotherme aussi.
    Encore une fois, adiabatique n'est pas isotherme~!
    \hfill \textcolor{ForestGreen}{/7}
  \item RAS.
    \hfill \textcolor{ForestGreen}{/6}
\end{enumerate}

\section{Exercice 2 \hfill \textcolor{red}{/50}}
\begin{enumerate}
  \item RAS.
    \hfill \textcolor{ForestGreen}{/2}
  \item $\abs{W} =$ aire du cycle.
    \hfill \textcolor{ForestGreen}{/4}
  \item Toute une variété de courbes, c'est fantastique. Des isothermes
    concaves, des isochores penchées (trouvez-vous une règle bon sang), des
    points aléatoires…
    \hfill \textcolor{ForestGreen}{/5}
  \item Citez \textbf{tout le développement} pour $W$~! Ça n'est pas pour rien
    que c'est resté en démonstration pendant 4 semaines de khôlles. $P_{\rm ext}
    \neq P$ si pas réversible ou QS, et $P \neq P_1$~: $V_1$ n'est \textbf{pas}
    une variable. Pensez également à justifier, par une expression ou la loi de
    \textsc{Joule}, que $\Delta{U} = 0$ pour une isoT.
    \hfill \textcolor{ForestGreen}{/12}
  \item RAS de 5 à 7.
    \hfill \textcolor{ForestGreen}{/4+3+2}
\end{enumerate}
\begin{enumerate}[start=8]
  \item Définir le rendement avant son expression. Pour les moteur, $Q_c$ = tout
    ce qui est positif.
    \hfill \textcolor{ForestGreen}{/5}
  \item RAS.
    \hfill \textcolor{ForestGreen}{/2}
  \item Une rendement de moteur \textbf{ne peut être égal à 1} (et \ul{encore
    moins l'infini})~!
    \hfill \textcolor{ForestGreen}{/4}
  \item Quelques bonnes pistes. Faire le lien entre énergie et puissance…
    ATTENTION~: la température doit \textbf{toujours} être en kelvins.
    $\frac{T_1}{T_3}$ ne donne pas la même chose en celsius… \hfill
    \textcolor{ForestGreen}{/7}
\end{enumerate}

\section{Exercice 3 \hfill \textcolor{red}{/56}}
\begin{enumerate}
  \item RAS.
    \hfill \textcolor{ForestGreen}{/4}
  \item Hypothèse $\ra$ test $\ra$ conclusion.
    \hfill \textcolor{ForestGreen}{/6}
  \item Pas de théorème des moments sans graphique… il fallait repartir de la
    source. Négliger $V_L$ devant $V_v$ ne permet pas de négliger $V$ devant
    $V_V$…
    \hfill \textcolor{ForestGreen}{/7}
  \item {\Large Vos courbes de rosée doivent être plus penchées~!!} Elle est
    entre l'isotherme et l'adiabatique. La courbe d'ébullition est pratiquement
    verticale. Expliquez votre démarche pour la transformation.
    \hfill \textcolor{ForestGreen}{/8}
  \item Question… très compliquée. Revoir le découpage d'étapes de
    transformations, en vous appuyant sur le diagramme de \textsc{Clapeyron}.
    Plein de $\Delta{H}_{\rm calo}$ sortis de nulle part~?
    \hfill \textcolor{ForestGreen}{/12}
  \item Idem, compliqué. \textbf{Attention}~: $H = U + PV \Ra \Delta{H} =
    \Delta{U} + \Delta{PV} = \Delta{U} + P \Delta{V} + V \Delta{P}$. Bien penser
    à la valeur de pression pour un équilibre diphasé vs. vapeur sèche.
    \hfill \textcolor{ForestGreen}{/12}
  \item Inhomogénéité de $T \Ra$ irréversible $\Ra \Sc_c > 0$.
    \hfill \textcolor{ForestGreen}{/7}
\end{enumerate}

\section{Exercice 4 \hfill \textcolor{red}{/60}}
\begin{enumerate}
  \item Très peu de points sur une question si simple. N'inversez pas $P
    \lessgtr P_{\rm sat}$.
    \hfill \textcolor{ForestGreen}{/1}
  \item Très peu faite, pourtant simple \textsc{Laplace}.
    \hfill \textcolor{ForestGreen}{/6}
  \item RAS.
    \hfill \textcolor{ForestGreen}{/4}
  \item Beaucoup de confusion.
    \hfill \textcolor{ForestGreen}{/7}
  \item Quelques excellentes réponses.
    \hfill \textcolor{ForestGreen}{/8}
  \item Pareil, revoir le lien entre puissance et énergie, et utiliser à bon
    escient les unités pour déterminer une relation.
    \hfill \textcolor{ForestGreen}{/8}
  \item RAS.
    \hfill \textcolor{ForestGreen}{/6}
  \item Souvent bien, souvent tout confondu. Attention, les éléments physiques
    (compresseur, etc) ne \textbf{sont} jamais les sources chaudes ou froides,
    mais sont situés \textbf{au niveau} des sources.
    \hfill \textcolor{ForestGreen}{/6}
  \item RAS.
    \hfill \textcolor{ForestGreen}{/4}
  \item[10 et 11)] \textbf{ÉTABLIR} ou \textbf{MONTRER}. Pas de point pour des
    réponses brutes.
    \hfill \textcolor{ForestGreen}{/5+5}
\end{enumerate}

\section{Exercice 5 \hfill \textcolor{red}{/12}}
Exercice à la portée de tout le monde. De bonnes pistes dans l'ensemble, une
seule finalisation. Entraînez-vous à estimer des valeurs du réel.

\end{document}
