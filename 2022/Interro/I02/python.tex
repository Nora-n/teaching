\documentclass[a4paper, 12pt, final, garamond]{book}
\usepackage{cours-preambule}

\raggedbottom

\makeatletter
\renewcommand{\@chapapp}{Contr\^ole de connaissances}
\makeatother

\HideSolutionstrue

\begin{document}
\setcounter{chapter}{1}

\chapter{R\'egressions lin\'eaires en \texttt{Python}}

Dans tout cet exercice, on suppose les importations suivantes~:
\begin{python}
import numpy as np
import matplotlib.pyplot as plt
\end{python}
Pour chaque question ci-après, écrire les commandes \texttt{python} demandées~:

\begin{enumerate}[label=\sqenumi]
    \item Donnez 2 manières de créer \texttt{X} le tableau \texttt{numpy}
        comportant les données \texttt{[1, 2, 3]}. \hfill (3 lignes) \smallbreak
        \begin{solution}
            \begin{python}
X = np.array([1, 2, 3])
# ou
X = [1, 2, 3]
X = np.asarray(X)
            \end{python}
\end{solution}
    \item Calculer \texttt{a}, \texttt{b} le coefficient directeur et l'ordonnée
        à l'origine de la régression linéaire entre \texttt{X} et \texttt{Y}.
        \hfill (1 ligne)
        \begin{solution}
            \begin{python}
a, b = np.polyfit(X, Y, 1)
            \end{python}
\end{solution}

    \item Créer la \textbf{fonction} \texttt{yfunc} permettant de tracer les valeurs
        d'une régression. \hfill (2 lignes)
        \begin{solution}
            \begin{python}
def yfunc(x, a, b):
    return a*x+b
            \end{python}
\end{solution}

    \item Créer la liste \texttt{xliste} découpant l'intervalle entre la plus petite
        valeur de \texttt{X} et la plus grande valeur de \texttt{X} en 1000
        points. \hfill (1 ligne)
        \begin{solution}
            \begin{python}
xliste = np.linspace(min(X), max(X), 1000)
            \end{python}
\end{solution}

    \item Créer la liste \texttt{yliste} des valeurs de la régression calculées sur
        \texttt{xliste}. \hfill (1 ligne)
        \begin{solution}
            \begin{python}
yliste = yfunc(xliste, a, b)
            \end{python}
\end{solution}

    \item Tracer le nuage de points de \texttt{X} et \texttt{Y}. \hfill (1
        ligne)
        \begin{solution}
            \begin{python}
plt.scatter(X, Y)
            \end{python}
\end{solution}

    \item Tracer le nuage de points avec barres d'erreurs, avec \texttt{uX}
        l'incertitude sur \texttt{X}, \texttt{uY} l'incertitude sur \texttt{Y}.
        (1 ligne ou 2)
        \begin{solution}
            \begin{python}
plt.errorbar(X, Y, xerr=uX, yerr=uY,
             linestyle='None') # bonus
            \end{python}
\end{solution}

    \item Tracer la régression linéaire entre \texttt{X} et \texttt{Y}. \hfill
        (1 ligne)
        \begin{solution}
            \begin{python}
plt.plot(xliste, yliste)
            \end{python}
\end{solution}

    \item Comment valider une régression~? \hfill (2 éléments)
        \begin{solution}
            \begin{enumerate}
                \item Pour être valide, un régression doit passer par les points
                    et les barres d'erreurs. 
                \item Pour s'assurer de sa qualité, on peut ensuite regarder la
                    valeur du coefficient de corrélation $r^2$~: on doit avoir
                    $r^2 > \num{0.99}$. $r^2 = \num{0.98}$ n'est pas
                    satisfaisant.
            \end{enumerate}
\end{solution}

    \item Citer les 4 étapes pour obtenir la meilleure estimation de $a$ et $b$
        ainsi que leurs incertitudes grâce à une méthode \textsc{Monte-Carlo}.
        \begin{solution}
            Pour obtenir des incertitudes sur $a$ et $b$, nous allons~:
            \begin{enumerate}[label=\arabic*)]
                \item faire varier aléatoirement les $n$ couples de valeurs
                    mesurées $(x_i, y_i)$ selon des **lois de probabilité
                    uniformes rectangulaires** de demi-largeur la précision
                    $\Delta(x_i) = \sqrt{3}u(x_i)$, simulée grâce à la
                    fonction \texttt{np.random.uniform()} (même chose pour $y_i$)~;
                \item Pour chaque série de mesures simulée, faire la régression 
                    linéaire, et obtenir des valeurs de pente $a_k$ et
                    d'ordonnée à l'origine $b_k$~;
                \item La meilleur estimation de $a$ et $b$ sera la moyenne des
                    valeurs calculées~;
                \item Les incertitude $u(a)$ et $u(b)$ sur ces moyennes seront
                    l'écart-type expérimental des valeurs calculées.
            \end{enumerate}
\end{solution}
\end{enumerate}

\newpage
~

\end{document}
