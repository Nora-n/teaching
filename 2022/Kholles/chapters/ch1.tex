\documentclass[a4paper, 12pt, final, garamond]{book}
\usepackage{cours-preambule}

\raggedbottom

\makeatletter
\renewcommand{\@chapapp}{Programme de kh\^olle -- semaine}
\makeatother

\begin{document}
\setcounter{chapter}{0}

\chapter{Du 12 au 16 septembre}

\section{Cours et exercices}

\section*{Optique chapitre 1 -- Propagation de la lumière}
\begin{enumerate}[label=\Roman*]
    \item \textbf{L'onde lumineuse}~: nature ondulatoire, célérité dans le vide,
        définition milieu TLHI, indice optique et ordres de grandeur, célérité
        dans un milieu, onde monochromatique caractérisée par fréquence ou
        longueur d'onde dans le vide, longueur d'onde dans un milieu TLHI.
    \item \textbf{Sources lumineuses primaires}~: notion de source primaire, de
        spectre d'émission, caractéristiques des sources thermiques, spectrales
        et LASER.
    \item \textbf{Diffraction de la lumière}~: principe (situation de
        \textsc{Fraunhofer} seulement), loi de la diffraction.
\end{enumerate}

\section*{Optique chapitre 2 -- Base de l'optique géométrique}
\begin{enumerate}[label=\Roman*]
    \item \textbf{Propriétés générales}~: approximation de l'optique
        géométrique, notion de rayon lumineux, propagation rectiligne, retour
        inverse de la lumière, indépendance des rayons lumineux.
    \item \textbf{Lois de Snell-Descartes}~: définition dioptre, rayons
        réfléchis et réfractés, lois de Snell-Descartes pour la réflexion et la
        réfraction, phénomène de réflexion totale.
\end{enumerate}

\section{Cours seulement}

\section*{Optique chapitre 2 -- Base de l'optique géométrique}
\begin{enumerate}[label=\Roman*, start=3]
    \item \textbf{Généralités sur les systèmes optiques}~: définition S.O.,
        S.O.\ centré, rayons incidents/émergents, faisceau lumineux convergent
        ou divergent, points objets et images, objets et images réelles ou
        virtuelles, conjugaison et schématisation $A \opto{S}{}A'$, objet étendu
        et grandissement transversal, foyers principaux et secondaire d'un S.O.\
        et propriétés associées.
    \item \textbf{Approximation de Gauss}~: définition stigmatisme, aplanétisme,
        rigoureux ou approché, rayons paraxiaux, conditions et approximation de
        Gauss.
\end{enumerate}

\section*{Optique chapitre 3 -- Miroir plan et lentilles minces}
\begin{enumerate}[label=\Roman*]
    \item \textbf{Miroir plan}~: définition, stigmatisme et aplanétisme
        rigoureux, construction pour objet réel et virtuel, relation de
        conjugaison (démonstration), grandissement transversal (démonstration).
    \item \textbf{Lentilles minces}~: définition lentille, minces, convergentes
        et divergentes, stigmatisme et aplanétisme, centre optique et propriété,
        distance focale image, vergence, construction rayons parallèles à l'axe
        optique pour divergente et convergente, règles primaires des
        constructions géométriques, cas simple pour lentille convergente et
        divergente.
\end{enumerate}

\section{Questions de cours possibles}
\begin{enumerate}
    \item Démontrer l'expression de la longueur d'onde dans un milieu d'indice
        $n$ d'une onde monochromatique de longueur d'onde dans le vide
        $\lambda_0$~;
    \item Tracer schématiquement les spectres d'émission des sources thermiques
        (à deux températures différentes), spectrales et LASER~;
    \item Définir un rayon lumineux et énoncer les propriétés liées à leur
        propagation~;
    \item Énoncer les lois de Snell-Descartes pour la réflexion et la réfraction
        \textit{avec un schéma}~;
    \item Énoncer les conditions de réflexion totale \textit{avec un schéma},
        donner et démontrer la valeur de l'angle limite $i_{\rm lim}$ en
        fonction de $n_2$ et $n_1$~;
    \item Toute définition des systèmes optiques \textit{avec un schéma}
        (principalement objets et images réelles avec exemples de situations et
        foyers principaux d'un S.O.)~: plusieurs définitions peuvent être
        demandées~;
    \item Définir la notion de stigmatisme et d'aplanétisme, les conditions de
        Gauss et leur conséquence. \textit{Schéma non demandé} pour
        l'aplanétisme.
    \item Construire l'image d'un objet (point ou étendu, réel ou virtuel) par
        un miroir plan~;
    \item Donner et démontrer la relation de conjugaison d'un miroir plan~;
    \item Définir le grandissement transversal, donner et démontrer
        schématiquement au moins sa valeur pour un miroir plan.
\end{enumerate}

\section{Consignes}
\begin{enumerate}
    \item Une question de cours non connue entraîne un 0 à cette partie (note
        maximale 10/20 si exercice parfait)~;
    \item Les schémas des questions de cours sont obligatoires~: s'ils manquent,
        la question ne saurait être notée au-dessus de 5~;
    \item Chacune des règles suivantes qui ne serait respectée enlèvera un demi
        point~:
        \begin{enumerate}
            \item Les schémas optiques doivent comporter le sens de comptage
                algébrique des distances et des angles~;
            \item Les rayons lumineux doivent avoir un sens de propagation~;
            \item Les angles doivent être orientés~;
        \end{enumerate}
\end{enumerate}

\end{document}
