\documentclass[a4paper, 12pt, final, garamond]{book}
\usepackage{cours-preambule}

\raggedbottom

\makeatletter
\renewcommand{\@chapapp}{Programme de kh\^olle -- semaine}
\makeatother

\begin{document}
\setcounter{chapter}{10}

\chapter{Du 05 novembre au 09 d\'ecembre}

\section{Cours et exercices}
\section*{Chimie chapitre 3 -- Cinétique chimique}
\begin{enumerate}[label=\Roman*]
    \item \textbf{Introduction}~: réactions lentes et rapides, méthodes de
        suivi, exemple de suivi cinétique, facteurs cinétiques.
    \item \textbf{Vitesse(s) de réaction}~: hypothèses de travail, vitesse de
        réaction, vitesses de formation/disparition.
    \item \textbf{Concentration et ordre de réaction}~: ordre d'une réaction,
        ordre initial et courant, cas particulier des réactions simples loi de
        \textsc{Van't Hoff}, cas particulier dégénérescence de l'ordre et
        proportions stœchiométriques.
    \item \textbf{Méthodes de résolution}~: temps de demi-réaction, ordres 0, 1
        et 2 par rapport à un réactif~: hypothèse de départ, unité de $k$,
        équation différentielle, résolution et $t_{1/2}$~; résumé méthodes en
        pratique et résumé.
    \item \textbf{Température et loi d'\textsc{Arrhénius}}~: expression de
        $k(T)$, exemple d'utilisation pour $k(T_1)$ et $k(T_2)$.
    \item \textbf{Méthodes de suivi cinétique expérimental}~: dosage par
        titrage et trempe chimique, dosage par étalonnage~: loi de
        \textsc{Beer-Lambert} et loi de \textsc{Kohlrausch}.
\end{enumerate}

\section*{Électrocinétique chapitre 5 -- Circuits électriques en RSF}
\begin{enumerate}[label=\Roman*]
    \item \textbf{Circuit RC série en RSF}~: présentation, réponse d'un système
        en RSF, passage en complexes.
    \item \textbf{Circuits électriques en RSF}~: lois de l'électrocinétique (loi
        des nœuds, loi des mailles), impédance et admittance complexes
        (définition, impédances de bases, comportements limites), associations
        d'impédances et ponts diviseurs.
    \item \textbf{Mesure de déphasages}~: définition, valeurs particulières,
        lecture d'un déphasage, déphasage des impédances.
\end{enumerate}

\section{Cours uniquement}
\section*{Électrocinétique chapitre 6 -- Oscillateurs en RSF}
\begin{enumerate}[label=\Roman*]
    \item \textbf{Introduction}~: rappel oscillateurs, méthode des complexes,
        notion de résonance et bande passante.
    \item \textbf{Exemple électrique~: circuit RLC série en RSF}~: présentation,
        étude de l'intensité (amplitude complexe, amplitude réelle et maximum,
        phase, influence de $Q$), étude de la tension (amplitude complexe,
        amplitude réelle et condition de résonance, phase).
    \item \textbf{Exemple mécanique~: ressort horizontal en RSF}~: présentation,
        étude de l'élongation (amplitude d'élongation complexe, amplitude réelle
        et condition de résonance), résonance en vitesse.
\end{enumerate}

\section{Questions de cours possibles}
\begin{enumerate}
    \item Donner la loi de vitesse d'une réaction $a\rm{A} + b\rm{B} = c\rm{C}
        + d\rm{D}$ admettant un ordre, la loi de vitesse de la même réaction si
        elle est simple, montrer l'intérêt de la dégénérescence de l'ordre et
        des proportions stœchiométriques \textbf{par le calcul}.
    \item À partir d'une loi de vitesse d'ordre 0 par rapport à un unique
        réactif $[{\rm A}]$, donner l'unité de $k$, démontrez l'équation
        différentielle vérifiée par $[{\rm A}]$ et la solution associée,
        indiquer quelle régression linéaire pourrait permettre de vérifier cette
        loi et donner le temps de demi-réaction.
    \item À partir d'une loi de vitesse d'ordre 1 par rapport à un unique
        réactif $[{\rm A}]$, donner l'unité de $k$, démontrez l'équation
        différentielle vérifiée par $[{\rm A}]$ et la solution associée,
        indiquer quelle régression linéaire pourrait permettre de vérifier cette
        loi et donner le temps de demi-réaction.
    \item À partir d'une loi de vitesse d'ordre 2 par rapport à un unique
        réactif $[{\rm A}]$, donner l'unité de $k$, démontrez l'équation
        différentielle vérifiée par $[{\rm A}]$ et la solution associée,
        indiquer quelle régression linéaire pourrait permettre de vérifier cette
    \item Énoncer la loi d'\textsc{Arrhénius}, indiquer une manière d'utiliser
        deux constantes de vitesse à deux températures différentes pour
        déterminer l'énergie d'activation, et une autre manière d'utiliser
        plusieurs constantes de vitesse à différentes températures pour
        déterminer l'énergie d'activation.
    \item Méthode des complexes en RSF~: donner la forme de réponse d'un système
        en RSF, les relations entre les grandeurs réelle et complexe associée,
        l'intérêt pour la dérivation et le lien entre une équation
        différentielle réelle et l'équation algébrique complexe associée.
    \item Rappeler comment se définit une impédance complexe, puis donner
        \textbf{et démontrer} les impédances complexes d'une résistance, d'une
        bobine et d'un condensateur, indiquer et justifier leurs comportements
        limites si elles en ont.
    \item Donner \textbf{et démontrer} les associations en série et en parallèle
        d'impédances complexes, et déterminer l'impédance équivalente d'une
        association donnée par l'examinataire.
    \item Donner \textbf{et démontrer} les relations des ponts diviseur de tension
        et diviseur de courant.
    \item Circuit RC série~: présenter le système réel, le système en RSF
        complexe, déterminer l'amplitude complexe sur la tension du condensateur
        ainsi que son amplitude réelle et sa phase.
    \item Étude de la résonance en intensité pour le circuit RLC série en RSF~:
        établir et tracer l'allure de l'amplitude réelle $I(\w)$ sous forme
        canonique.
\end{enumerate}
\end{document}
