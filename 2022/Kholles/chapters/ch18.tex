\documentclass[a4paper, 12pt, final, garamond]{book}
\usepackage{cours-preambule}

\raggedbottom

\makeatletter
\renewcommand{\@chapapp}{Programme de kh\^olle -- semaine}
\makeatother

\begin{document}
\setcounter{chapter}{17}

\chapter{Du 06 au 10 f\'evrier}

\section{Cours et exercices}
\section*{Mécanique chapitre 4 -- Approche énergétique}
\begin{enumerate}[label=\Roman*]
    \item \textbf{Notions énergétiques}~: énergie, conservation, puissance.
    \item \textbf{Énergie cinétique et travail force constante}~: définitions,
        exemples, travail du poids, théorème de l'énergie cinétique, approche
        énergétique ou PFD~?
    \item \textbf{Puissance d'une force et TPC}~: définition, TPC, TPC ou PFD~?
    \item \textbf{Travail élémentaire}~: définition, propriété, exemples,
        démonstration TEC.
    \item \textbf{Énergies potentielle et mécanique}~: forces conservatives ou
        non, énergie potentielle, gradient d'un scalaire, opérateur
        différentiel, lien à l'énergie potentielle, énergie mécanique, TEM et
        TPM.
    \item \textbf{Énergie potentielle et équilibres}~: notion d'équilibre, lien
        avec $\Ec_p$, équilibres stables et instables, lien avec
        $\dv[2]{\Ec_p}{x}$, étude générale autour d'un point d'équilibre
        stable~: oscillateur harmonique.
    \item \textbf{Énergie potentielle et trajectoire}~: détermination
        qualitative d'une trajectoire, état lié et diffusion~; cas du pendule
        simple, étude mouvement selon $\Ec_p$ et $\Ec_m$.
\end{enumerate}

\section{Cours uniquement}
\section*{Mécanique chapitre 5 -- Mouvement de particules chargées}
\begin{enumerate}[label=\Roman*]
    \litem{Champs électrique et magnétique}~: définitions, exemples condensateur
        et bobine.
    \litem{Force de \textsc{Lorentz}}~: définition, comparaison au poids,
        remarque produit vectoriel, puissance de la force de \textsc{Lorentz},
        potentiel électrostatique.
    \litem{Mouvement dans un champ électrique}~: situation générale, accélération
        pour $\vfo\parr\Ef$, déviation pour $\vfo\perp\Ef$, angle de déviation,
        applications (accélérateur linéaire, oscilloscope analogique).
    \litem{Mouvement dans un champ magnétique}~: mise en équation, cas
        $\vfo\parr\Bf$, cas $\vfo\perp\Bf$~: trajectoire et équations horaires
        cyclotron~; cas général (mouvement hélicoïdal), applications
        (spectromètre de masse, cyclotron, effet \textsc{Hall})
\end{enumerate}

\section{Questions de cours possibles}
\begin{enumerate}[label=\sqenumi]
    \item Définir la puissance d'une \textbf{force} (pas la définition
        introductive), son travail élémentaire, ainsi que
        son travail sur un chemin entre A et B. Définir ce qu'est une force
        conservative, en français et mathématiquement, et son lien avec
        l'énergie potentielle associée sous forme différentielle ($\dd{\Ec_p}$)
        et gradient ($\gd\Ec_p$).
    \item Retrouver les énergies potentielles de forces classiques (poids,
        rappel élastique, force newtonienne en $K/r^2$) à partir du gradient et
        à partir du travail élémentaire (les deux approches doivent apparaître
        au moins une fois, au choix), et trouver l'expression
        d'une force à partir d'une énergie potentielle proposée par
        l'interrogataire.
    \item Énoncer et démontrer les théorèmes de la puissance cinétique et de
        l'énergie cinétique.
    \item Énoncer et démontrer les théorèmes de la puissance mécanique et de
        l'énergie mécanique.
    \item Retrouver l'équation différentielle sur $\tt$ du pendule simple non
        amorti à l'aide~: soit du TPC, soit du TPM\footnote{Non fait
            intégralement en cours, guidage minimum pour exprimer $z(\tt)$
        possible.}, au choix de l'interrogataire.
    \item Savoir discuter le mouvement d'une particule en comparant son profil
        d'énergie potentielle et son énergie mécanique~; état lié ou de
        diffusion. Expliquer l'obtention des positions d'équilibre et leur
        stabilité sur un graphique $\Ec_p(x)$. Traduire l'équilibre et sa
        stabilité en terme de conditions sur la dérivée première et seconde de
        l'énergie potentielle.
    \item Savoir réaliser l'approximation harmonique d'une cuvette de potentiel
        par développement limité. En déduire que tout système décrit par une
        énergie potentielle présentant un minimum local est assimilable à un
        oscillateur harmonique.
    \item Définir la force de \textsc{Lorentz}~; comparer les ordres de
        grandeurs des forces électriques et magnétiques au poids~; déterminer la
        puissance de la force de \textsc{Lorentz} et discuter des conséquences.
        Démontrer qu'elle est conservative et déterminer l'expression de
        l'énergie potentielle associée.
    \item Action de $\Ef$ uniforme entre deux grilles chargées sur une particule
        chargée avec $\vfo\parr\Ef$~: présenter la situation, faire un bilan
        énergétique pour calculer la vitesse de sortie en fonction de la
        différence de potentiel $U$.
    \item Action de $\Bf$ uniforme sur une particule chargée avec
        $\vfo\perp\Bf$~: présenter la situation, et prouver que le mouvement est
        uniforme, plan et circulaire. On déterminera l'équation de la
        trajectoire en introduisant le rayon et la pulsation cyclotron, ainsi
        que les équations scalaires.
\end{enumerate}

\end{document}
