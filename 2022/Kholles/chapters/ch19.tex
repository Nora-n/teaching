\documentclass[a4paper, 12pt, final, garamond]{book}
\usepackage{cours-preambule}

\raggedbottom

\makeatletter
\renewcommand{\@chapapp}{Programme de kh\^olle -- semaine}
\makeatother

\begin{document}
\setcounter{chapter}{18}

\chapter{Du 27 f\'evrier au 03 mars}

\section{Exercices uniquement}
\section*{Mécanique chapitre 4 -- Approche énergétique}
\begin{enumerate}[label=\Roman*]
    \item \textbf{Notions énergétiques}~: énergie, conservation, puissance.
    \item \textbf{Énergie cinétique et travail force constante}~: définitions,
        exemples, travail du poids, théorème de l'énergie cinétique, approche
        énergétique ou PFD~?
    \item \textbf{Puissance d'une force et TPC}~: définition, TPC, TPC ou PFD~?
    \item \textbf{Travail élémentaire}~: définition, propriété, exemples,
        démonstration TEC.
    \item \textbf{Énergies potentielle et mécanique}~: forces conservatives ou
        non, énergie potentielle, gradient d'un scalaire, opérateur
        différentiel, lien à l'énergie potentielle, énergie mécanique, TEM et
        TPM.
    \item \textbf{Énergie potentielle et équilibres}~: notion d'équilibre, lien
        avec $\Ec_p$, équilibres stables et instables, lien avec
        $\dv[2]{\Ec_p}{x}$, étude générale autour d'un point d'équilibre
        stable~: oscillateur harmonique.
    \item \textbf{Énergie potentielle et trajectoire}~: détermination
        qualitative d'une trajectoire, état lié et diffusion~; cas du pendule
        simple, étude mouvement selon $\Ec_p$ et $\Ec_m$.
\end{enumerate}

\section{Cours et exercices}
\section*{Mécanique chapitre 5 -- Mouvement de particules chargées}
\begin{enumerate}[label=\Roman*]
    \litem{Champs électrique et magnétique}~: définitions, exemples condensateur
        et bobine.
    \litem{Force de \textsc{Lorentz}}~: définition, comparaison au poids,
        remarque produit vectoriel, puissance de la force de \textsc{Lorentz},
        potentiel électrostatique.
    \litem{Mouvement dans un champ électrique}~: situation générale, accélération
        pour $\vfo\parr\Ef$, déviation pour $\vfo\perp\Ef$, angle de déviation,
        applications (accélérateur linéaire, oscilloscope analogique).
    \litem{Mouvement dans un champ magnétique}~: mise en équation, cas
        $\vfo\parr\Bf$, cas $\vfo\perp\Bf$~: trajectoire et équations horaires
        cyclotron~; cas général (mouvement hélicoïdal), applications
        (spectromètre de masse, cyclotron, effet \textsc{Hall})
\end{enumerate}

\section{Cours uniquement}
\section*{Architecture matière ch.\ 1~: structure des entités chimiques}
\begin{enumerate}[label=\Roman*]
    \litem{Niveaux d'énergie d'un électron dans un atome}~: nombres quantiques et
        orbitales atomiques, niveaux d'énergie, électrons de cœur et de valence.
    \litem{Tableau périodique}~: construction et blocs, analyse par période,
        analyse par famille.
    \litem{Structure électronique des molécules}~: représentation de
        \textsc{Lewis} des atomes, liaison covalente, notation de \textsc{Lewis}
        des molécules, écarts à la règle de l'octet.
    \litem{Géométrie et polarité des entités chimiques}~: modèle VSEPR, polarité
        des liaisons et des molécules, polarisabilité.
\end{enumerate}

\section*{AM ch.\ 2~: forces intermoléculaires et pptés.\ macroscopiques}
\begin{enumerate}[label=\Roman*]
    \litem{Interactions de \textsc{Van der Waals}}~: \textsc{Keesom}
        permanent/permanent, \textsc{Debye} permanent/induit, \textsc{London}
        induit/induit, bilan et remarque répulsion.
    \litem{Températures de changement d'état}~: influence du moment dipolaire,
        influence de la polarisabilité.
    \litem{Liaison hydrogène}~: introduction expérimentale, définition et
        exemples.
    \litem{Solubilité, miscibilité}~: classement des solvants, solubilité, mise
        en solution d'espèces ioniques, miscibilité.
\end{enumerate}

\section{Questions de cours possibles}
\begin{enumerate}[label=\sqenumi]
    \item Définir la force de \textsc{Lorentz}~; comparer les ordres de
        grandeurs des forces électriques et magnétiques au poids~; déterminer la
        puissance de la force de \textsc{Lorentz} et discuter des conséquences.
        Démontrer qu'elle est conservative et déterminer l'expression de
        l'énergie potentielle associée.
    \item Action de $\Ef$ uniforme entre deux grilles chargées sur une particule
        chargée avec $\vfo\parr\Ef$~: présenter la situation, faire un bilan
        énergétique pour calculer la vitesse de sortie en fonction de la
        différence de potentiel $U$.
    \item Action de $\Bf$ uniforme sur une particule chargée avec
        $\vfo\perp\Bf$~: présenter la situation, et prouver que le mouvement est
        uniforme, plan et circulaire. On déterminera l'équation de la
        trajectoire en introduisant le rayon et la pulsation cyclotron, ainsi
        que les équations scalaires.
    \item Savoir comment construire (pas connaître par cœur) les 4 premières
        lignes du tableau périodique. Définir et placer les blocs $s$, $p$ et
        $d$. Préciser les colonnes des familles des gaz rares, des halogènes et
        des métaux alcalins. Placer un élément ($Z \leq 36$) sur le tableau à
        partir de son numéro atomique \textbf{\ul{et/ou}} déterminer son numéro
        atomique à partir de sa position~; dans tous les cas établir sa
        configuration de valence et son schéma de \textsc{Lewis} (bloc $s$ ou
        $p$).
    \item Établir (pas «~juste~» donner) les représentations de \textsc{Lewis}
        de molécules simples (\ce{CO2}, \ce{CH4}, \ce{H2O}, \ce{NH3}…) et
        indiquer leurs représentations spatiales liées à la méthode VSEPR en
        donnant un ordre de grandeur des angles.
    \item Établir les représentations de \textsc{Lewis} et les charges formelles
        de $\ce{HO^-, CN^-, NO3^-}$.
    \item Définir l'électronégativité d'un élément et donner (en le justifiant)
        son évolution par colonne, par famille et globalement dans le tableau.
        Définir le moment dipolaire d'une liaison, d'une molécule et la
        polarisabilité, et déterminer le moment dipolaire de \ce{H2O}
        connaissant $p_{\ce{HO}} = \SI{1.51}{D}$ et $\widehat{({\rm HOH})} =
        \ang{104.45}$.
    \item Définir ce que sont les interactions de \textsc{Van der Waals} et en
        donner l'énergie potentielle générale. Présenter les 3 interactions que
        ce terme regroupe~: nature, énergie potentielle, énergie de liaison.
        Donner la forme de l'énergie potentielle des interactions répulsives, la
        forme de l'énergie potentielle totale et indiquer sur le schéma comment
        se trouve la distance de liaison et l'énergie de liaison. On donnera un
        ordre de grandeur des distances d'interaction de \textsc{VdW}.
    \item Définir ce qu'est la liaison hydrogène, donner un ordre de grandeur de
        l'énergie d'une LH, les représenter sur les molécules d'eau et indiquer,
        avec 2 valeurs numériques, l'impact de la LH sur la température
        d'ébullition de l'eau. Indiquer (sans valeur numérique nécessaire) et
        justifier l'évolution des températures d'ébullition des composés
        hydrogénés de la 14\ieme colonne (\ce{CH4}, \ce{SiH4}, \ce{GeH4} et
        \ce{SnH4}).
    \item Définir ce qu'est un solvant polaire, protique, et dispersant.
        Déterminer, à partir de la représentation d'une molécule de solvant et
        de sa valeur de permittivité relative, s'il est polaire, protique et
        dispersant ou non. Indiquer comment choisir un solvant connaissant le
        soluté à dissoudre.
\end{enumerate}
\vspace{-5pt}
\begin{framed}
    \centering\bfseries\large
    Les fiches doivent être \ul{succinctes} et ne pas faire 3 copies doubles.
    Synthétisez l'information. Il est interdit de copier-coller le cours.
\end{framed}

\end{document}
