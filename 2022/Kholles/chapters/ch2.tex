\documentclass[a4paper, 12pt, final, garamond]{book}
\usepackage{cours-preambule}

\raggedbottom

\makeatletter
\renewcommand{\@chapapp}{Programme de kh\^olle -- semaine}
\makeatother

\begin{document}
\setcounter{chapter}{1}

\chapter{Du 19 au 23 septembre}

\section{Cours et exercices}

\section*{Optique chapitre 2 -- Base de l'optique géométrique}
\begin{enumerate}[label=\Roman*]
    \item \textbf{Propriétés générales}~: approximation de l'optique
        géométrique, notion de rayon lumineux, propagation rectiligne, retour
        inverse de la lumière, indépendance des rayons lumineux.
    \item \textbf{Lois de Snell-Descartes}~: définition dioptre, rayons
        réfléchis et réfractés, lois de Snell-Descartes pour la réflexion et la
        réfraction, phénomène de réflexion totale.
    \item \textbf{Généralités sur les systèmes optiques}~: définition S.O.,
        S.O.\ centré, rayons incidents/émergents, faisceau lumineux convergent
        ou divergent, points objets et images, objets et images réelles ou
        virtuelles, conjugaison et schématisation $A \opto{S}{}A'$, objet étendu
        et grandissement transversal, foyers principaux et secondaire d'un S.O.\
        et propriétés associées.
    \item \textbf{Approximation de Gauss}~: définition stigmatisme, aplanétisme,
        rigoureux ou approché, rayons paraxiaux, conditions et approximation de
        Gauss.
\end{enumerate}

\section*{Optique chapitre 3 -- Miroir plan et lentilles minces}
\begin{enumerate}[label=\Roman*]
    \item \textbf{Miroir plan}~: définition, stigmatisme et aplanétisme
        rigoureux, construction pour objet réel et virtuel, relation de
        conjugaison (démonstration), grandissement transversal (démonstration).
    \item \textbf{Lentilles minces}~: définition lentille, minces, convergentes
        et divergentes, stigmatisme et aplanétisme, centre optique et propriété,
        distance focale image, vergence, construction rayons parallèles à l'axe
        optique pour divergente et convergente, règles primaires et secondaires
        des constructions géométriques, tous les cas pour lentilles convergentes
        et divergentes, relations de conjugaison + démonstration, grandissement
        transversal.
    \item \textbf{Quelques applications}~: condition de netteté (méthode de
        Bessel, $D \geq 4f'$), champ de vision à travers un miroir plan et
        hauteur d'un arbre.
\end{enumerate}

\section{Cours seulement}

\section*{Optique chapitre 4 -- Dispositifs optiques}
\begin{enumerate}[label=\Roman*]
    \item \textbf{L'œil}~: présentation et modélisation, accommodation et
        focales minimales et maximales, réglage d'un instrument optique,
        résolution angulaire et vocabulaire sur les défauts.
    \item \textbf{La loupe}~: présentation de l'effet loupe, définition
        grossissement général et propriété $G = d_m/f'$ pour la loupe avec
        démonstration.
    \item \textbf{Appareil photo}~: description, modélisation simple, champ et
        influence de la focale et de la taille du capteur, distance de mise au
        point, profondeur de champ et influence de la distance de mise au point,
        de la focale et de l'ouverture.
    \item \textbf{Systèmes optiques à plusieurs lentilles}~: association
        quelconque, convergente+convergente en cours, notion de microscope,
        définition lunettes astronomiques Kepler et Galilée, définition système
        afocal, calcul d'encombrement, grossissement $G=-f'_1/f'_2$ et
        démonstration.
\end{enumerate}

\section{Questions de cours possibles}
\begin{enumerate}
    \item Énoncer les lois de Snell-Descartes pour la réflexion et la réfraction
        \textit{avec un schéma}~;
    \item Énoncer les conditions de réflexion totale \textit{avec un schéma},
        donner et démontrer la valeur de l'angle limite $i_{\rm lim}$ en
        fonction de $n_2$ et $n_1$~;
    \item Donner et démontrer la relation de conjugaison d'un miroir plan~;
    \item Définir le grandissement transversal, donner et démontrer
        schématiquement au moins sa valeur pour un miroir plan, donner son
        expression pour une lentille~;
    \item Plusieurs tracés peuvent être demandés parmi~:
        \begin{enumerate}
            \item Construire l'image d'un objet étendu réel ou virtuel par une
                lentille quelconque en présentant les règles primaires et en
                précisant la nature de l'objet et de l'image~;
            \item Construire le rayon émergent d'un rayon quelconque en
                présentant les règles de construction secondaires et nommant
                tous les points d'intérêt.
        \end{enumerate}
    \item Savoir utiliser les relations de conjugaison pour trouver la position
        et la taille de l'image d'un objet par une lentille mince (accompagné
        d'un schéma)~;
    \item Savoir refaire la démonstration de la condition de netteté pour
        l'image réelle d'un objet réel d'une lentille convergente ($D \geq
        4f'$)~; les conditions du système seront redonnées~;
    \item Savoir refaire l'exercice «~champ de vision à travers un miroir
        plan~» dont l'énoncé sera redonné~;
    \item Décrire un modèle simple de l'œil et de l'appareil photographique, et
        décrire les analogies et différences~;
    \item Décrire l'effet loupe, définir le grossissement et démontrer sa
        formule pour une loupe.
\end{enumerate}

\section{Consignes}
\begin{enumerate}
    \item \textbf{Les relations de conjugaison ne sont pas à connaître}. Le
        grandissement pour une lentille est à connaître.
    \item \textbf{Pas d'association de lentilles en cours ou exercice cette
        semaine}. Association lentille-miroir éventuellement.
    \item Une question de cours non connue entraîne un 0 à cette partie (note
        maximale 10/20 si exercice parfait)~;
    \item Les schémas des questions de cours sont obligatoires~: s'ils manquent,
        la question ne saurait être notée au-dessus de 5~;
    \item Chacune des règles suivantes qui ne serait respectée enlèvera
        \textbf{un  point}~:
        \begin{enumerate}
            \item Les schémas optiques doivent comporter le sens de comptage
                algébrique des distances et des angles~;
            \item Les rayons lumineux doivent avoir un sens de propagation~;
            \item Les angles doivent être orientés.
        \end{enumerate}
\end{enumerate}

\end{document}

