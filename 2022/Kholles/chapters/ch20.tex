\documentclass[a4paper, 12pt, final, garamond]{book}
\usepackage{cours-preambule}

\raggedbottom

\makeatletter
\renewcommand{\@chapapp}{Programme de kh\^olle -- semaine}
\makeatother

\begin{document}
\setcounter{chapter}{19}

\chapter{Du 06 au 10 mars}

\section{Cours et exercices}
\section*{Architecture matière ch.\ 1 -- structure des entités chimiques}
\begin{enumerate}[label=\Roman*]
    \litem{Niveaux d'énergie d'un électron dans un atome}~: nombres quantiques et
        orbitales atomiques, niveaux d'énergie, électrons de cœur et de valence.
    \litem{Tableau périodique}~: construction et blocs, analyse par période,
        analyse par famille.
    \litem{Structure électronique des molécules}~: représentation de
        \textsc{Lewis} des atomes, liaison covalente, notation de \textsc{Lewis}
        des molécules, écarts à la règle de l'octet.
    \litem{Géométrie et polarité des entités chimiques}~: modèle VSEPR, polarité
        des liaisons et des molécules, polarisabilité.
\end{enumerate}

\section*{AM ch.\ 2 -- forces intermoléculaires et pptés.\ macroscopiques}
\begin{enumerate}[label=\Roman*]
    \litem{Interactions de \textsc{Van der Waals}}~: \textsc{Keesom}
        permanent/permanent, \textsc{Debye} permanent/induit, \textsc{London}
        induit/induit, bilan et remarque répulsion.
    \litem{Températures de changement d'état}~: influence du moment dipolaire,
        influence de la polarisabilité.
    \litem{Liaison hydrogène}~: introduction expérimentale, définition et
        exemples.
    \litem{Solvants}~: classement des solvants, pouvoir dispersant, exemples~; 
        solubilité et miscibilité, mise en solution d'espèces ioniques.
\end{enumerate}

\section{Cours uniquement}
\section*{Mécanique ch. 6 -- Moment cinétique d'un point matériel}
\begin{enumerate}[label=\Roman*]
    \litem{Moment d'une force}~: par rapport à un point, définition et exemples~;
        par rapport à un axe orienté~: définition et exemples~; bras de levier
        d'une force~: propriété, méthode et application~; exemples de calcul de
        moments.
    \litem{Moment cinétique}~: par rapport à un point, définition et exemples~;
        par rapport à un axe orienté, définition et exemples.
    \litem{Théorème du moment cinétique}~: par rapport à un point fixe, énoncé et
        démonstration~; par rapport à un axe orienté fixe~: énoncé et
        démonstration.
    \litem{Exemple du pendule simple}~: équation du mouvement par TMC.
\end{enumerate}

\section*{Mécanique ch. 7 -- Mouvement à force centrale conservative}
\begin{enumerate}[label=\Roman*]
    \litem{Forces centrales conservatives}~: définition force centrale,
        définition force centrale conservative et exemples.
    \litem{Quantités conservées}~: moment cinétique, loi des aires, énergie
        mécanique et énergie potentielle effective.
    \litem{Champs de force newtoniens}~: définition, cas attractif, cas répulsif.
    \litem{Mécanique céleste}~: lois des \textsc{Kepler}, mouvement circulaire.
    % \litem{Satellite en orbite terrestre}~: vitesses cosmiques, satellite
    %     géostationnaire.
\end{enumerate}

\section{Questions de cours possibles}
\begin{enumerate}[label=\sqenumi]
    \item Savoir comment construire (pas connaître par cœur) les 4 premières
        lignes du tableau périodique. Définir et placer les blocs $s$, $p$ et
        $d$. Préciser les colonnes des familles des gaz rares, des halogènes et
        des métaux alcalins. Placer un élément ($Z \leq 36$) sur le tableau à
        partir de son numéro atomique \textbf{\ul{et/ou}} déterminer son numéro
        atomique à partir de sa position~; dans tous les cas établir sa
        configuration de valence et son schéma de \textsc{Lewis} (bloc $s$ ou
        $p$).
    \item Établir (pas «~juste~» donner) les représentations de \textsc{Lewis}
        de molécules simples (\ce{CO2}, \ce{CH4}, \ce{H2O}, \ce{NH3}…) et
        indiquer leurs représentations spatiales liées à la méthode VSEPR en
        donnant un ordre de grandeur des angles.
    \item Définir l'électronégativité d'un élément et donner (en le justifiant)
        son évolution par colonne, par famille et globalement dans le tableau.
        Définir le moment dipolaire d'une liaison, d'une molécule et la
        polarisabilité, et déterminer le moment dipolaire de \ce{H2O}
        connaissant $p_{\ce{HO}} = \SI{1.51}{D}$ et $\widehat{({\rm HOH})} =
        \ang{104.45}$.
    \item Définir ce qu'est la liaison hydrogène, donner un ordre de grandeur de
        l'énergie d'une LH, les représenter sur les molécules d'eau et indiquer,
        avec 2 valeurs numériques, l'impact de la LH sur la température
        d'ébullition de l'eau. Indiquer (sans valeur numérique nécessaire) et
        justifier l'évolution des températures d'ébullition des composés
        hydrogénés de la 14\ieme\ colonne (\ce{CH4}, \ce{SiH4}, \ce{GeH4} et
        \ce{SnH4}).
    \item Définir ce qu'est un solvant polaire, protique, et dispersant.
        Déterminer, à partir de la représentation d'une molécule de solvant et
        de sa valeur de permittivité relative, s'il est polaire, protique et
        dispersant ou non. Indiquer comment choisir un solvant connaissant le
        soluté à dissoudre.

    \item Définir le moment cinétique d'un point matériel par rapport à un point
        et à un axe, et le moment d'une force par rapport à un point et à un
        axe. Expliquer ce qu'est le bras de levier \textbf{avec un schéma}, et
        énoncer le lien entre moment d'une force et bras de levier.
        Démonstration \textbf{pour $\Ff \perp$ à l'axe}.
    \item Énoncer et démontrer le théorème du moment cinétique par rapport à un
        point et à un axe~; application au pendule simple pour retrouver
        l'équation du mouvement.

    \item Présenter ce qu'est une force centrale, démontrer que le moment
        cinétique se conserve, en déduire l'expression de la constante des
        aires, prouver que le mouvement est donc plan, et démontrer la loi des
        aires.
    \item En utilisant la constante des aires, déterminer l'expression de
        l'énergie potentielle effective pour un mouvement à force centrale
        conservative. Donner $\Ec_p$ pour un champ de force newtonien, 
        représenter $\Ec_{p,\rm eff}$ et discuter de la nature du mouvement en
        fonction de l'énergie mécanique totale (cas attractif \textbf{et}
        répulsif).
    \item Énoncer les trois lois de \textsc{Kepler}, démontrer la troisième loi
        de \textsc{Kepler} pour le cas spécifique de l'orbite circulaire.
\end{enumerate}
\vspace{-5pt}
\begin{framed}
    \centering\bfseries\large
    Les fiches doivent être \ul{succinctes} et ne pas faire 3 copies doubles.
    Synthétisez l'information. Il est interdit de copier-coller le cours.
    \bigbreak
    \Huge
    Les fiches de plus de 2 copies doubles impliqueront un malus de 1 point sur
    la question de cours.
\end{framed}

\end{document}
