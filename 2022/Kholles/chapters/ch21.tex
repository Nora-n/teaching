\documentclass[a4paper, 12pt, final, garamond]{book}
\usepackage{cours-preambule}

\raggedbottom

\makeatletter
\renewcommand{\@chapapp}{Programme de kh\^olle -- semaine}
\makeatother

\begin{document}
\setcounter{chapter}{20}

\chapter{Du 13 au 17 mars}

\section{Cours et exercices}
\section*{Mécanique ch. 6 -- Moment cinétique d'un point matériel}
\begin{enumerate}[label=\Roman*]
    \litem{Moment d'une force}~: par rapport à un point, définition et exemples~;
        par rapport à un axe orienté~: définition et exemples~; bras de levier
        d'une force~: propriété, méthode et application~; exemples de calcul de
        moments.
    \litem{Moment cinétique}~: par rapport à un point, définition et exemples~;
        par rapport à un axe orienté, définition et exemples.
    \litem{Théorème du moment cinétique}~: par rapport à un point fixe, énoncé et
        démonstration~; par rapport à un axe orienté fixe~: énoncé et
        démonstration.
    \litem{Exemple du pendule simple}~: équation du mouvement par TMC.
\end{enumerate}

\section*{Mécanique ch. 7 -- Mouvement à force centrale conservative}
\begin{enumerate}[label=\Roman*]
    \litem{Forces centrales conservatives}~: définition force centrale,
        définition force centrale conservative et exemples.
    \litem{Quantités conservées}~: moment cinétique, loi des aires, énergie
        mécanique et énergie potentielle effective.
    \litem{Champs de force newtoniens}~: définition, cas attractif, cas répulsif.
    \litem{Mécanique céleste}~: lois des \textsc{Kepler}, mouvement circulaire.
    \litem{Satellite en orbite terrestre}~: vitesses cosmiques, satellite
        géostationnaire.
\end{enumerate}

\section{Cours uniquement}
\section*{Mécanique ch. 8 -- Mécanique du solide}
\begin{enumerate}[label=\Roman*]
    \litem{Système de points matériels}~: Systèmes discret et continu, centre
    d'inertie, mouvements d'un solide indéformable~: translation, rotation.
    \litem{Rappel~: TRC}~: quantité de mouvement d'un ensemble de points, forces
    intérieures et extérieures, théorème de la résultante cinétique.
    \litem{Énergétique des systèmes de points}~: énergie cinétique, puissances
    intérieures et extérieures, théorèmes énergétiques.
    \litem{Moments pour un système de points}~: moment cinétique et moment
    d'inertie, moments intérieurs et extérieurs, théorème du moment cinétique,
    énergétique d'un solide en rotation.
    \litem{Cas particuliers et application}~: notion de couple, liaison pivot,
    pendule pesant.
\end{enumerate}

\section{Questions de cours possibles}
\begin{enumerate}[label=\sqenumi]
    \item Définir le moment cinétique d'un point matériel par rapport à un point
        et à un axe, et le moment d'une force par rapport à un point et à un
        axe. Expliquer ce qu'est le bras de levier \textbf{avec un schéma}, et
        énoncer le lien entre moment d'une force et bras de levier.
        Démonstration \textbf{pour $\Ff \perp$ à l'axe}.
    \item Énoncer et démontrer le théorème du moment cinétique par rapport à un
        point et à un axe~; application au pendule simple pour retrouver
        l'équation du mouvement.
    \item Présenter ce qu'est une force centrale, démontrer que le moment
        cinétique se conserve, en déduire l'expression de la constante des
        aires, prouver que le mouvement est donc plan, et démontrer la loi des
        aires.
    \item En utilisant la constante des aires, déterminer l'expression de
        l'énergie potentielle effective pour un mouvement à force centrale
        conservative. Donner $\Ec_p$ pour un champ de force newtonien, 
        représenter $\Ec_{p,\rm eff}$ et discuter de la nature du mouvement en
        fonction de l'énergie mécanique totale (cas attractif \textbf{et}
        répulsif).
    \item Énoncer les trois lois de \textsc{Kepler}, démontrer la troisième loi
        de \textsc{Kepler} pour le cas spécifique de l'orbite circulaire~:
        vitesse, période, et énergie mécanique.
    \item Définir et démontrer les expressions des vitesses cosmiques en
        justifiant les valeurs d'énergie mécanique à atteindre à l'aide du
        schéma de l'énergie potentielle effective.

    \item Donner le lien entre quantité de mouvement d'un système et le centre
        d'inertie d'un solide. Démontrer que la résultante des forces
        intérieures d'un solide est nulle, et démontrer le théorème de la
        résultante cinétique.
    \item Définir le moment d'inertie d'un solide, donner et démontrer la
        relation entre moment cinétique scalaire et moment d'inertie d'un
        solide. Retrouver le TMC pour un solide en rotation, en supposant acquis
        que la somme des moments intérieurs est nulle. Définir un couple, une
        liaison pivot et une liaison pivot parfaite.
    \item Établir l'équation différentielle du mouvement pour le pendule
        \textbf{pesant} grâce au TMC scalaire.
    \item Donner l'expression de l'énergie cinétique d'un solide en translation
        \textbf{et} dans le cas particulier d'un solide en rotation autour d'un
        axe fixe. Exprimer les théorèmes énergétiques pour les solides. Donner
        l'expression de la puissance des forces extérieures pour un solide en
        rotation en fonction du moment des forces extérieures. Démonstration
        pour une force $\Ff$ dans le sens de $\ut$.
\end{enumerate}
\vspace{-5pt}

\begin{framed}
    \centering\bfseries\large
    Les fiches doivent être \ul{succinctes} et ne pas faire 3 copies doubles.
    Synthétisez l'information. Il est interdit de copier-coller le cours.
    \bigbreak \Huge
    Les fiches de plus de 2 copies doubles impliqueront un malus de 1 point sur
    la question de cours.
\end{framed}

\end{document}
