\documentclass[a4paper, 12pt, final, garamond]{book}
\usepackage{cours-preambule}

\raggedbottom

\makeatletter
\renewcommand{\@chapapp}{Programme de kh\^olle -- semaine}
\makeatother

\begin{document}
\setcounter{chapter}{22}

\chapter{Du 27 au 21 mars}

\section{Exercices uniquement}
\section*{Mécanique ch. 7 -- Mouvement à force centrale conservative}
\section*{Mécanique ch. 8 -- Mécanique du solide}

\section{Cours et exercices}
\section*{Chimie chapitre 4 -- Réactions acido-basiques}
\begin{enumerate}[label=\Roman*]
    \litem{Acides et bases}~: définitions, pH.
    \litem{Rappel état d'équilibre}~: introduction, transformations totales et
        limitées, quotient de réaction et constante d'équilibre, évolution d'un
        système chimique.
    \litem{Réactions acido-basiques}~: autoprotolyse de l'eau, constantes
        d'acidité, calcul de constantes de réactions.
    \litem{Distribution des espèces d'un couple}~: lien pH et concentration
        (relation de \textsc{Henderson}), diagramme de prédominance, diagramme
        de distribution.
    \litem{Prédiction des réactions et des équilibres}~: sens d'échange des
        protons et diagramme de pKa, pH et composition à l'équilibre.
    \litem{Titrages acido-basiques}~: définition et exemple, méthodes de suivi.
\end{enumerate}

\section{Cours uniquement}
\section*{Chimie chapitre 5 -- Réactions de précipitation}
\begin{enumerate}[label=\Roman*]
    \litem{Observations expérimentales}~: exemple et définition précipité.
    \litem{Produit de solubilité}~: définition et exemples.
    \litem{Condition d'existence}~: existence en fonction de $K_s$.
    \litem{Solubilité}~: définition, dans l'eau pure, paramètres d'influence~:
        température, ions communs, pH.
\end{enumerate}

\section*{Chimie chapitre 6 -- Réactions d'oxydoréduction}
\begin{enumerate}[label=\Roman*]
    \litem{Oxydants et réducteurs}~: introduction, définition, réactions
        d'oxydoréduction, équilibrage des demi-équations et couples à connaître,
        équilibrage des réactions rédox~; nombre d'oxydation, introduction,
        règles de calcul, interprétation, lien avec la position dans la
        classification périodique.
    \litem{Piles}~: introduction, vocabulaire, potentiel d'électrode, application
        calcul f.é.m., capacité d'une pile.
    \litem{Réactions d'oxydoréduction}~: diagramme de prédominance, sens de
        réaction et diagramme en potentiel standard, calcul des constantes
        d'équilibre, et application, dismutation et médiamutation
\end{enumerate}

\section{Questions de cours possibles}
\begin{center}
    \begin{framed}
        Plusieurs questions \textbf{simples} peuvent être posées.
    \end{framed}
\end{center}

\begin{enumerate}[label=\sqenumi]
    \item Définir le pH, la constante d'acidité d'un couple acide/base,
        l'autoprotolyse de l'eau et le produit ionique de l'eau. Écrire la
        réaction associée à la constante d'acidité du couple \ce{H3O+/H2O},
        exprimer la constante d'acidité en fonction de \ce{[H3O+]} et en déduire
        pK$_a$(\ce{H3O+/H2O}) = 0. Faire de même avec la réaction associée à la
        constante d'acidité du couple \ce{H2O/HO-}, et en déduire
        pK$_a$(\ce{H2O/HO-}) = pK$_e$.
    \item Connaître nom, formule et équation entre acide et base des couples
        contenant~: acide sulfurique, acide nitrique, acide chlorhydrique, acide
        phosphorique, acide éthanoïque, acide carbonique, ion ammonium, ion
        hydroxyde. À partir du lien entre pH et pK$_a$ d'un couple acide-base,
        justifier et tracer un diagramme de prédominance.
    \item Tracer qualitativement le diagramme de distribution de l'acide
        carbonique \ce{H2CO3}. Identifier les espèces sur le schéma, indiquer
        comment lire le pK$_a$ des couples, et le lien entre les concentrations
        des espèces des couples quand pH = pK$_a$.

    \item Définir le produit de solubilité avec un exemple. Déterminer la
        condition d'existence d'un précipité. \textbf{Un diagramme d'existence
        n'est pas demandé} (fait en TD, et revu chapitre 9).
    \item Définir la solubilité. Calculer la solubilité de \ce{PbI2}, sachant
        que p$K_s(\ce{PbI2}) = 8$.
    \item Donnez les paramètres influençant la solubilité. Donner un exemple
        d'application chacun d'eux. En particulier, connaissant p$K_s(\ce{AgCl})
        = 9.8$, déterminer la solubilité de $\ce{AgCl\sol{}}$ dans une solution
        aqueuse contenant déjà $c = \SI{0.1}{mol.L^{-1}}$ de \ce{Cl-}. On
        supposera $s \ll c$.
    \item Donner les couples et les demi-équations redox des couples contenant~:
        ions thiosulfate, ion permanganate, ion hypochlorite. Donner le nombre
        d'oxydation des éléments. Équilibrer la réaction entre
        $\ce{Fe^{2+}\aqu{}}$ et $\ce{MnO4^-\aqu{}}$.
    \item Présenter ce qu'est une pile avec l'exemple de la pile
        \textsc{Daniell} (\ce{Cu^{2+}/Zn})~: schéma, vocabulaire, explication.
    \item Pour une demi-réaction rédox générale, donner la formule de
        \textsc{Nernst}. Application pour le couple
        $(\ce{MnO4^{-}\aqu{}/Mn^{2+}\aqu{}})$.
    \item Établir l'expression de la capacité d'une pile en fonction du nombre
        d'électrons échangés, de l'avancement à l'équilibre et du nombre de
        \textsc{Faraday} à partir de l'exemple de la pile \textsc{Daniell}.
\end{enumerate}
\vspace{-5pt}

\begin{framed}
    \centering\bfseries\large
    Les fiches doivent être \ul{succinctes} et ne pas faire 3 copies doubles.
    Synthétisez l'information. Il est interdit de copier-coller le cours.
    \bigbreak \Huge
    Les fiches de plus de 2 copies doubles impliqueront un malus de 1 point sur
    la question de cours.
\end{framed}

\end{document}
