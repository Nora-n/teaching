\documentclass[a4paper, 12pt, final, garamond]{book}
\usepackage{cours-preambule}

\raggedbottom

\makeatletter
\renewcommand{\@chapapp}{Programme de kh\^olle -- semaine}
\makeatother

\begin{document}
\setcounter{chapter}{4}

\chapter{Du 10 au 14 octobre}

\section{Cours et exercices}

\section*{Électrocinétique chapitre 1 -- Circuits électriques dans l'ARQS}
\begin{enumerate}[label=\Roman*]
    \item \textbf{Courant électrique et intensité}~: charge électrique, courant
        électrique, sens conventionnel.
    \item \textbf{Tension et potentiel}~: définition, additivité, masse,
        analogie électro-hydraulique.
    \item \textbf{Vocabulaire des circuits électriques}~: circuit, schéma,
        dipôle, nœud, branche, maille~; conventions générateur et récepteur,
        dipôles en série ou dérivation, mesures de tensions et d'intensités.
    \item \textbf{Lois fondamentales des circuits électriques dans l'ARQS}~:
        approximation, application, loi des branches et nœuds, loi des mailles,
        puissance électrocinétique, fonctionnement générateur et récepteur, et
        conservation de l'énergie.
\end{enumerate}

\section*{Électrocinétique chapitre 2 -- Résistances et sources}
\begin{enumerate}[label=\Roman*]
    \item \textbf{Généralité sur les dipôles}~: caractéristique courant-tension,
        vocabulaire associé.
    \item \textbf{Résistance}~: définition et schéma, association en série
        \textbf{et démonstration}, association en parallèle \textbf{et
        démonstration}, pont diviseur de tension \textbf{et démonstration}, pont
        diviseur de courant \textbf{et démonstration}.
    \item \textbf{Sources}~: sources idéale et réelle de tension, sources idéale
        et réelle de courant, résistances de sortie.
\end{enumerate}

\section{Cours uniquement}

\section*{Électrocinétique chapitre 3 -- Condensateurs et bobines}
\begin{enumerate}[label=\Roman*]
    \item \textbf{Condensateur idéal}~: présentation et lien $q=Cu$,
        caractéristique, continuité et régime permanent, énergie stockée
        \textbf{et démonstration}.
    \item \textbf{Bobine idéale}~: présentation, caractéristique, continuité et
        régime permanent, énergie stockée \textbf{et démonstration}.
    \item \textbf{Circuit RC série~: charge}~: présentation, équation
        différentielle, résolution avec méthode, représentation graphique et
        constante de temps + régimes transitoire, permanent, évolution de
        l'intensité, bilans de puissance et d'énergie.
    \item \textbf{Circuit RC série~: décharge}~: présentation, équation
        différentielle, résolution avec méthode, représentation graphique et
        constante de temps + régimes transitoire, permanent, évolution de
        l'intensité.
    \item \textbf{Circuit RL série~: échelon montant}~: présentation, équation
        différentielle, résolution avec méthode, représentation graphique et
        constante de temps + régimes transitoire, permanent, évolution de
        la tension, bilan de puissance.
\end{enumerate}

\section{Questions de cours possibles}
\begin{framed}
    \begin{center}
        \huge Il est possible de demander plusieurs questions de cours
    \end{center}
\end{framed}
\begin{enumerate}
    \item Énoncer et expliquer les conditions de l'ARQS, donner des exemples
        d'application et non-application~;
    \item Démontrer puis utiliser la loi des mailles pour trouver l'intensité
        dans un circuit simple (deux mailles possible)~;
    \item Démontrer les relations des associations séries et parallèles
        \textbf{et} déterminer la résistance équivalente d'une portion de
        circuit donné par l'examinataire~;
    \item Démontrer les relations des ponts diviseurs de tension et de courant
        et en utiliser sur un schéma donné par l'examinataire~;
    \item Présenter et démontrer les caractéristiques d'un condensateur et d'une
        bobine~: relation courant-tension (sans démonstration pour la bobine),
        continuité, régime permanent, énergie stockée.
    \item Présenter le circuit RC en charge sous un échelon de tension $E$
        (schéma et condition initiale), donner et démontrer l'équation
        différentielle sur $u_C$, donner la solution et la tracer. Indiquer sans
        le démontrer comment trouver la constante de temps et le régime
        permanent.
    \item Présenter le circuit RC en décharge depuis une tension $E$ aux bornes
        du condensateur (schéma et condition initiale), donner et démontrer
        l'équation différentielle sur $u_C$, \textbf{démontrer} la solution et
        la tracer. Indiquer sans le démontrer comment trouver la constante de
        temps et le régime permanent.
    \item Présenter le circuit RL soumis à un échelon de tension $E$ (schéma et
        condition initiale), donner et démontrer l'équation différentielle sur
        $i$, donner la solution et la tracer. Indiquer sans le démontrer comment
        trouver la constante de temps et le régime permanent.
\end{enumerate}

% \section{Consignes}
% \begin{enumerate}
% 
%     \item \textbf{Les relations de conjugaison \underline{sont} à connaître}.
%     \item Une question de cours non connue entraîne un 0 à cette partie (note
%         maximale 10/20 si exercice parfait)~;
%     \item \textbf{Les schémas des questions de cours sont obligatoires~: s'ils
%         manquent, la question ne saurait être notée au-dessus de 5}~;
%     \item Chacune des règles suivantes qui ne serait respectée enlèvera
%         \textbf{un  point}~:
%         \begin{enumerate}
%             \item Les schémas optiques doivent comporter le sens de comptage
%                 algébrique des distances et des angles~;
%             \item Les rayons lumineux doivent avoir un sens de propagation~;
%             \item Les angles doivent être orientés.
%             \item Tous les dipôles doivent être fléchés en courant et tension
%                 sur les schémas.
%         \end{enumerate}
% \end{enumerate}

\end{document}
