\documentclass[a4paper, 12pt, final, garamond]{book}
\usepackage{cours-preambule}

\raggedbottom

\makeatletter
\renewcommand{\@chapapp}{Programme de kh\^olle -- semaine}
\makeatother

\begin{document}
\setcounter{chapter}{8}

\chapter{Du 21 au 25 novembre}

\section{Exercices seulement}
\section*{Électrocinétique chapitre 4 -- Oscillateurs harmonique et amorti}

\begin{enumerate}[label=\Roman*, start=5]
    \item \textbf{Introduction amorti}~: exemple RLC différents régimes ($Q =$
        13, 3, 0.5 et 0.2) et analyse, équation différentielle générale et
        analyse $Q$, définition équation caractéristique, discriminant et
        différents régimes, solutions générales.
    \item \textbf{Oscillateur amorti électrique RLC libre}~: présentation,
        \textbf{bilan énergétique} et analyse, équation différentielle et
        conditions initiales, solution, démonstrations, régimes transitoires à
        95\% et visualisation dans l'espace des phases \textbf{pour tous les
        régimes}, limite $Q \rightarrow \infty$.
    \item \textbf{Oscillateur amorti mécanique ressort frottements fluides}~:
        présentation et force de frottement fluide, équation différentielle et
        solution pour $\ell$ et $x$, analogie RLC-ressort amorti, sur toutes les
        grandeurs ($x$, $v$, $m$, $k$, $\alpha$, $\w_0$, $Q$), et résumé complet
        oscillateurs amortis.
\end{enumerate}

\section{Cours et exercices}
\section*{Chimie chapitre 1 -- Introduction}
\begin{enumerate}[label=\Roman*]
    \item \textbf{Vocabulaire général}~: atomes et molécules, classification par
        composition, états de la matière et systèmes physico-chimiques,
        transformations de la matière.
    \item \textbf{Quantification des systèmes}~: mole, masse molaire, fractions
        molaire et massique, masse volumique, concentrations molaire et
        massique, dilution~; pression d'un gaz, modèle du gaz parfait, volume
        molaire, pression partielle et loi de \textsc{Dalton}, intensivité et
        extensivité, activité d'un élément chimique.
\end{enumerate}

\section*{Chimie chapitre 2 -- Transformation et équilibre chimique}
\begin{enumerate}[label=\Roman*]
    \item \textbf{Avancement d'une réaction}~: présentation, avancements molaire
        et volumique, tableau d'avancement, coefficients stœchiométriques
        algébriques.
    \item \textbf{États final et d'équilibre d'un système chimique}~: réactions
        totales et limitées et exercice d'application, quantifications de
        l'avancement~: taux de conversion, coefficient de dissociation,
        rendement~; quotient de réaction et exercice d'application, constante
        d'équilibre et exercice d'application, réactions quasi-nulles et
        quasi-totales.
    \item \textbf{Évolution d'un système chimique}~: quotient réactionnel et
        évolution et exercice d'application, cas des ruptures d'équilibre,
        résumé pratique de résolution.
\end{enumerate}

\section{Questions de cours possibles}
\begin{enumerate}
    \item Définir la quantité de matière, la masse molaire et son lien avec la
        quantité de matière, les fractions molaire et massique, les
        concentrations molaire et massique et le lien entre les deux.

    \item Refaire l'exercice sur les fractions molaire et massique de dioxygène
        et diazote, l'exemple sur la concentration molaire de $\ce{Na+}$~:
\end{enumerate}
\begin{tcbraster}[raster columns=2, raster equal height=rows]
    \begin{NCexem}[width=\linewidth]{Exercice}
        L'air est constitué, en quantité de matière, à 80\% de diazote N$_2$ et
        à 20\% de dioxygène O$_2$. On a $M({\rm N_2}) = \SI{28.0}{g.mol^{-1}}$
        et $M({\rm O_2}) = \SI{32.0}{g.mol^{-1}}$. En déduire les fractions
        molaires puis les fractions massiques.
    \end{NCexem}
    \begin{NCexem}[]{Exercice}
        On dissout une masse $m = \SI{2.00}{g}$ de sel NaCl$\sol$ dans $V =
        \SI{100}{mL}$ d'eau. \textbf{Déterminer la concentration en
        Na$\plus{}$ dans la solution}. ($M(\ce{NaCl}) = \SI{58.44}{g.mol^{-1}}$)
    \end{NCexem}
\end{tcbraster}
\begin{enumerate}[resume]
    \item Savoir ajuster l'équation suivante \textbf{\underline{et}} une autre
        équation proposée par l'interrogataire~:
        \begin{gather*}
            \ce{\ldots I^-\aqu{} + \ldots Cr2O7\moin{2}\aqu{} + \ldots H\plus{}\aqu}
            =
            \ce{\ldots Cr\plus{3}\aqu{} + \ldots I2\gaz{} + \ldots H2O\liq{}}
        \end{gather*}

    \item Réaction et avancement~: \textbf{définir le taux de conversion,
        le coefficient de dissociation et le rendement} et refaire l'exemple du
        cours sur la combustion totale du méthane $\ce{CH4\gaz{} + 2O2\gaz{}
        \rightarrow CO2\gaz{} + 2H2O\gaz{}}$ avec $n_{\ce{CH4}}^0 = \SI{2}{mol}$
        et $n_{\ce{O2}}^0 = \SI{3}{mol}$.

    \item Donner les différentes expressions de l'activité d'un constituant
        selon sa nature, exprimer le quotient de réaction d'une équation-bilan
        générale $0=\sum_i \nu_i{\rm X}_i$ ou $\alpha_1{\rm R}_1 + \alpha_2{\rm
        R}_2 + … = \beta_1{\rm P}_1 + \beta_2{\rm P}_2 + …$ et la constante
        d'équilibre associée, et exprimer $Q_r$ pour les réactions~:
        \begin{enumerate}
            \item $\ce{2I^-\aqu{} + S2O8^{2-}\aqu{} = I2\aqu{} +2SO4^{2-}\aqu}$
            \item $\ce{Ag+\aqu{} + Cl^-\aqu{} = AgCl\sol}$
            \item $\ce{2FeCl3\gaz{} = Fe2Cl6\gaz{}}$
        \end{enumerate}
    \item Refaire l'exercice d'application de la réaction de l'acide éthanoïque
        avec l'eau~:
\end{enumerate}
\begin{NCexem}[width=\linewidth, breakable]{Exercice}
    Soit la réaction de l'acide éthanoïque avec l'eau~:
    \[\ce{CH3COOH\aqu{} + H2O\liq{} = CH3COO\moin{}\aqu{} + H3O\plus{}\aqu{}}\]
    de constante $K = \num{1.78e-5}$. On introduit $c = \SI{1.0e-1}{mol.L^{-1}}$
    d'acide éthanoïque et on note $V$ le volume de solution. \textbf{Déterminer
    la composition à l'état final}.
\end{NCexem}

\begin{enumerate}[resume]
    \item Indiquer comment prévoir le sens d'évolution d'un système, et refaire
        l'exercice~:
\end{enumerate}
\begin{NCexem}[width=\linewidth, breakable]{Exercice}
    Soit la synthèse de l'ammoniac~:

    \centersright{$\ce{N2\gaz{} + 3H2\gaz{} = 2NH3\gaz{}}$}{$K = \num{0.5}$}

    On introduit \SI{3}{mol} de diazote, \SI{5}{mol} de dihydrogène et
    \SI{2}{mol} d'ammoniac sous une pression de \SI{200}{bars}.
    \textbf{Déterminer les pressions partielles des gaz} et \textbf{indiquer
    dans quel sens se produit la réaction}.
\end{NCexem}
\begin{enumerate}[resume]
    \item Rupture d'équilibre~: refaire l'exercice d'application~:
\end{enumerate}
\begin{NCexem}[width=\linewidth]{Exercice}

    Considérons la dissolution du chlorure de sodium, de masse molaire
    $M(\ce{NaCl}) = \SI{58.44}{g.mol^{-1}}$~:

    \centersright{$\ce{NaCl\sol{} = Na\plus{}\aqu{} + Cl\moin{}\aqu{}}$}{$K=33$}

    On introduit \SI{2.0}{g} de sel dans \SI{100}{mL} d'eau. \textbf{Déterminer
    l'état d'équilibre}.
\end{NCexem}

\end{document}
