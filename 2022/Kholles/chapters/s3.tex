\documentclass[a4paper, 11pt]{book}
\usepackage{/home/nicolas/Documents/Enseignement/Prepa/bpep/fichiers_utiles/preambule}

\newcommand{\dsNB}{3}
\makeatletter
\renewcommand{\@chapapp}{Kh\^olles MPSI3 -- semaine \dsNB}
\makeatother

\toggletrue{corrige}  % décommenter pour passer en mode corrigé

\begin{document}

\resetQ
\newpage

\chapter{Sujet 1\siCorrige{\!\!-- corrig\'e}}
\section{Question de cours}

Présenter le défaut d'un œil hypermétrope \textbf{avec un schéma}, comment
corriger ce défaut et les points caractéristique du verre correcteur et de l'œil
qui doivent être confondus pour corriger la vision de loin. Une schématisation
optique (du type $AB \opto{\Lc}{O} A'B'$) et un schéma sont nécessaires~;

\subimport{/home/nicolas/Documents/Enseignement/Prepa/bpep/exercices/Colle/doublet_focal/}{sujet.tex}


\resetQ
\newpage

\chapter{Sujet 2\siCorrige{\!\!-- corrigé}}
\section{Question de cours}

Savoir comment se modélise un microscope et construire le chemin de deux rayons
parallèles quelconques. Les positions des points d'intérêt nécessaires au tracé
seront données par l'examinataire. Définir alors le grossissement \textbf{sans}
donner ou démontrer son expression, en donner un ordre de grandeur et commenter
son signe~;

\subimport{/home/nicolas/Documents/Enseignement/Prepa/bpep/exercices/Colle/doublet_focal_2/}{sujet.tex}


\resetQ
\newpage

\chapter{Sujet 3\siCorrige{\!\!-- corrigé}}
\section{Question de cours}

Démontrer le théorème des vergences pour les lentilles accolées, et démontrer la
relation du grandissement d’une association de lentilles en fonction du
grandissement de chacune des lentilles.

\subimport{/home/nicolas/Documents/Enseignement/Prepa/bpep/exercices/Colle/lunette_astronomique/}{sujet.tex}


\resetQ
\newpage

\chapter{Sujet 4\siCorrige{\!\!-- corrigé}}
\section{Question de cours}

Savoir comment se modélise une lunette de \textbf{Kepler} et construire le
chemin de deux rayons parallèles quelconques. Les positions des points d'intérêt
nécessaires au tracé seront données par l'examinataire. Définir alors le
grossissement, \textbf{donner et démontrer} son expression, en donner un ordre
de grandeur et commenter son signe.

\subimport{/home/nicolas/Documents/Enseignement/Prepa/bpep/exercices/TD/systeme_catadioptrique/}{sujet.tex}

\end{document}
