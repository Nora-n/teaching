\documentclass[a4paper, 11pt]{book}
\usepackage{/home/nicolas/Documents/Enseignement/Prepa/bpep/fichiers_utiles/preambule}

\newcommand{\dsNB}{7}
\makeatletter
\renewcommand{\@chapapp}{Kh\^olles MPSI3 -- semaine \dsNB}
\makeatother

\toggletrue{corrige}  % décommenter pour passer en mode corrigé

\begin{document}

\resetQ
\newpage

\chapter{Sujet 1\siCorrige{\!\!-- corrigé}}

\section{Question de cours}

Faire l'analogie complète entre les deux systèmes harmoniques LC libre et
ressort sans frottement~: présentation, conditions initiales, équations
différentielles \textbf{sans démonstration}, correspondance entre les grandeurs,
tracer de la solution \textbf{dans l'espace des phases} sans résolution et
commenter sur la conservation de l'énergie visible dans le graphique.

\subimport{/home/nicolas/Documents/Enseignement/Prepa/bpep/exercices/Colle/oscillateur_harmonique_ressort_decole/}{sujet.tex}

\resetQ
\newpage

\chapter{Sujet 2\siCorrige{\!\!-- corrigé}}
\section{Question de cours}

Faire un bilan d'énergie pour le circuit LC libre, démontrer la conservation de
l'énergie totale, tracer la forme du graphique, et faire un bilan d'énergie pour
le ressort horizontal sans frottement, démontrer la conservation de l'énergie
mécanique, tracer le graphique correspondant.

\subimport{/home/nicolas/Documents/Enseignement/Prepa/bpep/exercices/Colle/vibration_molecule/}{sujet.tex}

\resetQ
\newpage

\chapter{Sujet 3\siCorrige{\!\!-- corrigé}}
\section{Question de cours}

Présenter le circuit RLC libre (schéma et conditions initiales), donner et
\textbf{démontrer} l'équation différentielle sur $u_C$ sous forme canonique
\textbf{qu'on ne cherchera pas à résoudre}, vérifier son homogénéité, présenter
les graphiques des solutions selon les valeurs de $Q$ dans l'espace temporel
\textbf{et} dans l'espace des phases ($u_C$, $i$) en donnant un approximation de
la durée du régime transitoire à 95\%.

\subimport{/home/nicolas/Documents/Enseignement/Prepa/bpep/exercices/Colle/energie_oscillateur_harmonique/}{sujet.tex}

\resetQ
\newpage

\chapter{Sujet 4\siCorrige{\!\!-- corrigé}}
\section{Question de cours}

Présenter le circuit LC libre (schéma et conditions initiales), donner et
\textbf{démontrer} l'équation différentielle sur $u_C$, vérifier son
homogénéité, donner et \textbf{démontrer} la solution et la tracer en espace
temporel \textbf{et} dans l'espace des phases ($u_C$, $i$).

\subimport{/home/nicolas/Documents/Enseignement/Prepa/bpep/exercices/Colle/charge_bobine/}{sujet.tex}

\end{document}
