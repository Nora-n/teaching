\documentclass[11pt, a4paper, garamond]{book}

\usepackage[utf8x]{inputenc}
\usepackage[french]{babel}
\frenchbsetup{StandardLists=true}
\usepackage[T1]{fontenc}

\usepackage[top=1.5cm, bottom=1cm, left=1.5cm, right=1.5cm]{geometry}
\setlength{\parindent}{0pt}
\usepackage{multicol}
\usepackage{fancyhdr}

\usepackage{tabularx}
\newcolumntype{Y}{>{\centering\arraybackslash}X}
\renewcommand\tabularxcolumn[1]{m{#1}}

%%% En-tête et pied de page

\pagestyle{fancy}
\fancyhead[L]{Nora NICOLAS}
\fancyhead[C]{MPSI3 -- Physique-chimie}
\fancyhead[R]{Rapport de khôlle S1}
\fancyfoot[C]{}

\begin{document}

\vfill
\begin{tabularx}{\linewidth}{|Y*{4}{|Y}}\hline
    \multicolumn{3}{|l|}{
    NOM Prénom~: IDIR Bruno, sujet 1
    } & \multicolumn{1}{r|}{
    Note~: 10/20
    }\\\hline
    Connaissances~: Moyenne et brouillon.
    &
    S'approprier, analyser~: Pas trop mal
    &
    Réaliser et valider~: --
    &
    Communication~: à travailler
    \\

    \hline

    \multicolumn{2}{|>{\hsize = 2\hsize}X|}{
    \underline{Cours}~: Conjugaison miroir plan.
    } & \multicolumn{2}{>{\hsize = 2\hsize}X|}{\vspace{.5em}

    \underline{Commentaires}~: Se confond dans le vocabulaire. Très bon schéma,
    peut-être même trop détaillé pour ce que c'est, sauf que rayons
    émergents en pointillés et pas fléchés. Très bien sur le nommage des points.
    Présentation pêche~: beaucoup de blancs et de pauses, trop de détails
    inutiles c'est difficule à suivre. Sens algébrique non indiqué. Ne trouve
    pas comment correctement expliquer la relation de conjugaison même s'il l'a
    écrite. \vspace{.5em}}\\
          
    \hline

    \multicolumn{2}{|>{\hsize = 2\hsize}X|}{
    \underline{Exercice}~: Grenouille intelligente
    } & \multicolumn{2}{>{\hsize = 2\hsize}X|}{\vspace{.5em}
    \underline{Commentaires}~: Schéma qui fait presque sens. Le rayon passant le
    dioptre n'est pas dévié. Il fallait voir la réflexion totale.
    \vspace{.5em}}\\
        
    \hline

    \multicolumn{2}{|>{\hsize = 2\hsize}X|}{
    \underline{Tableau}~: Propre et large, mais vide.
    } & \multicolumn{2}{>{\hsize = 2\hsize}X|}{\vspace{.5em}
    \underline{Oral}~: Fluidité à travailler. Ne pas s'arrêter sur du
    vocabulaire inconnu.
    \vspace{.5em}}\\

    \hline
\end{tabularx}

\smallbreak

\begin{tabularx}{\linewidth}{|Y*{4}{|Y}}\hline
    \multicolumn{3}{|l|}{
    NOM Prénom~: JULIEN Alexandre, sujet 4
    } & \multicolumn{1}{r|}{
    Note~: 12/20
    }\\\hline
    Connaissances~: Bonnes
    &
    S'approprier, analyser~: Ok
    &
    Réaliser et valider~: Un peu lent
    &
    Communication~: Très bien
    \\

    \hline

    \multicolumn{2}{|>{\hsize = 2\hsize}X|}{
    \underline{Cours}~: Définitions systèmes.
    } & \multicolumn{2}{>{\hsize = 2\hsize}X|}{\vspace{.5em}
    \underline{Commentaires}~: Très bon rappel du sujet. Centré ok. Point objet
    et image~: attention à «~en fonction de~», ce sont les intersections des
    rayons. Réel et virtuel à revoir aussi, mais bonne réponse aux questions.
    \vspace{.5em}}\\
          
    \hline

    \multicolumn{2}{|>{\hsize = 2\hsize}X|}{
    \underline{Exercice}~: Prisme rectangle.
    } & \multicolumn{2}{>{\hsize = 2\hsize}X|}{\vspace{.5em}
    \underline{Commentaires}~: Sens algébriques ok. Angles par rapport à la
    noremale ok. Bonne appropriation de l'énoncé, mais schéma avec angles peu
    réalistes. Avant de tracer un rayon réfracté, si $n_2 < n_1$, il faut
    regarder s'il y a réflexion totale.
    \vspace{.5em}}\\
        
    \hline

    \multicolumn{2}{|>{\hsize = 2\hsize}X|}{
    \underline{Tableau}~: Très bien géré.
    } & \multicolumn{2}{>{\hsize = 2\hsize}X|}{\vspace{.5em}
    \underline{Oral}~: Super.
    \vspace{.5em}}\\

    \hline
\end{tabularx}

\smallbreak

\begin{tabularx}{\linewidth}{|Y*{4}{|Y}}\hline
    \multicolumn{3}{|l|}{
    NOM Prénom~: JOFFRE Arsène, sujet 5
    } & \multicolumn{1}{r|}{
    Note~: 15/20
    }\\\hline
    Connaissances~: Très bien
    &
    S'approprier, analyser~: Très bien
    &
    Réaliser et valider~: Bien
    &
    Communication~: Très bien
    \\

    \hline

    \multicolumn{2}{|>{\hsize = 2\hsize}X|}{
    \underline{Cours}~: Réflexion totale.
    } & \multicolumn{2}{>{\hsize = 2\hsize}X|}{\vspace{.5em}
    \underline{Commentaires}~: Très bon rappel du sujet. Malheureusement, «~moins
    réfringent~» ok mais ordre des indices incorrect. Bonne traduction du
    phénomène, très bon schéma, déroulé bien présenté.  S'est repris sur
    l'ordre des indices à la fin après guidage sur les propriétés de la fonction
    $\sin$.
    \vspace{.5em}}\\
          
    \hline

    \multicolumn{2}{|>{\hsize = 2\hsize}X|}{
    \underline{Exercice}~: Détecteur de pluie.
    } & \multicolumn{2}{>{\hsize = 2\hsize}X|}{\vspace{.5em}
    \underline{Commentaires}~: Schéma impeccable. Réponses impeccables. A
    cependant choisi l'exercice qui l'arrangeait plutôt que de faire
    \textsc{Brewster} directement.
    \vspace{.5em}}\\
        
    \hline

    \multicolumn{2}{|>{\hsize = 2\hsize}X|}{
    \underline{Tableau}~: Superbe.
    } & \multicolumn{2}{>{\hsize = 2\hsize}X|}{\vspace{.5em}
    \underline{Oral}~: Super.
    \vspace{.5em}}\\

    \hline
\end{tabularx}

\end{document}
