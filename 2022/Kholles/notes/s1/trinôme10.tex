\documentclass[11pt, a4paper, garamond]{book}

\usepackage[utf8x]{inputenc}
\usepackage[french]{babel}
\frenchbsetup{StandardLists=true}
\usepackage[T1]{fontenc}

\usepackage[top=1.5cm, bottom=1cm, left=1.5cm, right=1.5cm]{geometry}
\setlength{\parindent}{0pt}
\usepackage{multicol}
\usepackage{fancyhdr}

\usepackage{tabularx}
\newcolumntype{Y}{>{\centering\arraybackslash}X}
\renewcommand\tabularxcolumn[1]{m{#1}}

%%% En-tête et pied de page

\pagestyle{fancy}
\fancyhead[L]{Nora NICOLAS}
\fancyhead[C]{MPSI3 -- Physique-chimie}
\fancyhead[R]{Rapport de khôlle S1}
\fancyfoot[C]{}

\begin{document}

\begin{tabularx}{\linewidth}{|Y*{4}{|Y}}\hline
    \multicolumn{3}{|l|}{
    NOM Prénom~: UGOLINI Samy, sujet 2
    } & \multicolumn{1}{r|}{
    Note~: 15/20
    }\\\hline
    Connaissances~: Super
    &
    S'approprier, analyser~: Très bien
    &
    Réaliser et valider~: Oui
    &
    Communication~: OK
    \\

    \hline

    \multicolumn{2}{|>{\hsize = 2\hsize}X|}{
    \underline{Cours}~: définition rayon lumineux et propriétés.
    } & \multicolumn{2}{>{\hsize = 2\hsize}X|}{\vspace{.5em}
    \underline{Commentaires}~: Parfait.
    \vspace{.5em}}\\
          
    \hline

    \multicolumn{2}{|>{\hsize = 2\hsize}X|}{
        \underline{Exercice}~: Réfractomètre d'\textsc{Abbe}
    } & \multicolumn{2}{>{\hsize = 2\hsize}X|}{\vspace{.5em}
    \underline{Commentaires}~: Schéma avec sens algébriques. Couleur OK. Bonne
    réflexion pour la question 1. Manque de précision sur les conditions de
    parallélisme mais phénomène bien compris. Pareil pour question 2 et question
    4. Dommage pour la lenteur alors que la question de cours était courte et
    parfaitement maîtrisée.
    \vspace{.5em}}\\
        
    \hline

    \multicolumn{2}{|>{\hsize = 2\hsize}X|}{
    \underline{Tableau}~: Extrêmement propre.
    } & \multicolumn{2}{>{\hsize = 2\hsize}X|}{\vspace{.5em}
    \underline{Oral}~: Très bonne présence.
    \vspace{.5em}}\\

    \hline
\end{tabularx}

\smallbreak

\begin{tabularx}{\linewidth}{|Y*{4}{|Y}}\hline
    \multicolumn{3}{|l|}{
    NOM Prénom~: COLIN Thibault, sujet 1
    } & \multicolumn{1}{r|}{
    Note~: 10/20
    }\\\hline
    Connaissances~: Moyen
    &
    S'approprier, analyser~: Faible
    &
    Réaliser et valider~: --
    &
    Communication~: presque aucune
    \\

    \hline

    \multicolumn{2}{|>{\hsize = 2\hsize}X|}{
    \underline{Cours}~: Relation de conjugaison miroir plan.
    } & \multicolumn{2}{>{\hsize = 2\hsize}X|}{\vspace{.5em}
    \underline{Commentaires}~: Commence par écrire la relation de conjugaison
    des lentilles minces (sans flèches algébriques !). En le dirigeant vers le
    fait qu'on a un miroir donc que l'image est un symétrique, on avance. Note
    $O$ le projeté orthogonal de $A$ sur le miroir. N'a pas dessiné de flèche
    sur son rayon, mais sens algébriques OK.
    \vspace{.5em}}\\
          
    \hline

    \multicolumn{2}{|>{\hsize = 2\hsize}X|}{
    \underline{Exercice}~: Grenouille intelligente.
    } & \multicolumn{2}{>{\hsize = 2\hsize}X|}{\vspace{.5em}
    \underline{Commentaires}~: N'a pas su partir. Il faut réussir à schématiser
    ce qu'il se passe. Schéma avec grandeurs algébriques mais les rayons
    manquent de flèches. Justification du calcul de l'angle limite de réfraction
    faible, mais une fois guidé il a réussi a trouver le raisonnement pour la
    taille du nénuphar~; n'a cependant pas réussi à dire comment trouver
    l'équation pour la taille en fonction de $h$ puisque les angles d'incidence
    n'étaient pas représentés sur le schéma.
    \vspace{.5em}}\\
        
    \hline

    \multicolumn{2}{|>{\hsize = 2\hsize}X|}{
    \underline{Tableau}~: Très propre. Bonne gestion des couleurs et séparation
    du tableau.
    } & \multicolumn{2}{>{\hsize = 2\hsize}X|}{\vspace{.5em}
    \underline{Oral}~: À l'aise dans sa présentation et dans les réponses.
    \vspace{.5em}}\\

    \hline
\end{tabularx}

\smallbreak

\begin{tabularx}{\linewidth}{|Y*{4}{|Y}}\hline
    \multicolumn{3}{|l|}{
    NOM Prénom~: Siméon RESCOUSSIER, sujet 5
    } & \multicolumn{1}{r|}{
    Note~: 14/20
    }\\\hline
    Connaissances~:
    &
    S'approprier, analyser~:
    &
    Réaliser et valider~:
    &
    Communication~:
    \\

    \hline

    \multicolumn{2}{|>{\hsize = 2\hsize}X|}{
    \underline{Cours}~: Réflexion totale.
    } & \multicolumn{2}{>{\hsize = 2\hsize}X|}{\vspace{.5em}
    \underline{Commentaires}~: N'a pas mentionné que $n_2 < n_1$. Pas évident
    les domaines de définitions de la fonction $\sin$. Schéma OK. Calcul OK.
    Rayons avec couleur et avec flèches OK. Sens algébrique OK.
    \vspace{.5em}}\\
          
    \hline

    \multicolumn{2}{|>{\hsize = 2\hsize}X|}{
    \underline{Exercice}~: Incidence de \textsc{Brewster}
    } & \multicolumn{2}{>{\hsize = 2\hsize}X|}{\vspace{.5em}
    \underline{Commentaires}~: Très bon schéma. Comprend bien l'énoncé et sait
    le traduire. Bloqué sur $\sin(\pi/2 - x) = \cos(x)$, mais bien après.
    Attention aux fonctions mathématiques~: arctan n'est pas $1/\tan$ !!
    \vspace{.5em}}\\
        
    \hline

    \multicolumn{2}{|>{\hsize = 2\hsize}X|}{
    \underline{Tableau}~: Très bien. Encadrement, application numérique propre.
    } & \multicolumn{2}{>{\hsize = 2\hsize}X|}{\vspace{.5em}
    \underline{Oral}~: Bonne présentation, bonne présence et réactivité aux
    questions.
    \vspace{.5em}}\\

    \hline
\end{tabularx}

\end{document}
