\documentclass[11pt, a4paper, garamond]{book}

\usepackage[utf8x]{inputenc}
\usepackage[french]{babel}
\frenchbsetup{StandardLists=true}
\usepackage[T1]{fontenc}

\usepackage[top=1.5cm, bottom=1cm, left=1.5cm, right=1.5cm]{geometry}
\setlength{\parindent}{0pt}
\usepackage{multicol}
\usepackage{fancyhdr}

\usepackage{tabularx}
\newcolumntype{Y}{>{\centering\arraybackslash}X}
\renewcommand\tabularxcolumn[1]{m{#1}}

%%% En-tête et pied de page

\pagestyle{fancy}
\fancyhead[L]{Nora \textsc{NICOLAS}}
\fancyhead[C]{MPSI3 -- Physique-chimie}
\fancyhead[R]{Rapport de khôlle SEMAINE}
\fancyfoot[C]{}

\begin{document}

\begin{tabularx}{\linewidth}{|Y*{4}{|Y}}\hline
    \multicolumn{3}{|l|}{
    NOM Prénom~: CHAPELET Séréna, sujet 4
    } & \multicolumn{1}{r|}{
    Note~: 13/20
    }\\\hline
    Connaissances~: Bien
    &
    S'approprier, analyser~: Bien
    &
    Réaliser et valider~: Bien
    &
    Communication~: Bien
    \\

    \hline

    \multicolumn{2}{|>{\hsize = 2\hsize}X|}{
    \underline{Cours}~: Définitions diverses avec schéma
    } & \multicolumn{2}{>{\hsize = 2\hsize}X|}{\vspace{.5em}

    \underline{Commentaires}~: SO ok mais pas centré, dit «~rotatif~». Point
    objet et image ok. Très beaux schémas, fléchés etc. Réel et virtuel OK.
    Presque parfait.

    \vspace{.5em}}\\
          
    \hline

    \multicolumn{2}{|>{\hsize = 2\hsize}X|}{
    \underline{Exercice}~: Prisme rectangle
    } & \multicolumn{2}{>{\hsize = 2\hsize}X|}{\vspace{.5em}
    \underline{Commentaires}~: Bon tracé du rayon et nommage des points. Angle
    de réflexion bien trouvé, mais pas démontré qu'il y avait réflexion totale~:
    vite rattrapé nonobstant. Pas beaucoup plus d'avancée, difficile de trouver
    $i_3$.
    \vspace{.5em}}\\
        
    \hline

    \multicolumn{2}{|>{\hsize = 2\hsize}X|}{
    \underline{Tableau}~: Très beau tableau, bien organisé, très clair, et très
    bons schémas. Angles et distances positives tout le temps ok.
    } & \multicolumn{2}{>{\hsize = 2\hsize}X|}{\vspace{.5em}
    \underline{Oral}~: Bonne présence.
    \vspace{.5em}}\\

    \hline
\end{tabularx}

\smallbreak

\begin{tabularx}{\linewidth}{|Y*{4}{|Y}}\hline
    \multicolumn{3}{|l|}{
    NOM Prénom~: LEDENKO Ferdinand
    } & \multicolumn{1}{r|}{
    Note~: 10/20
    }\\\hline
    Connaissances~: Faible
    &
    S'approprier, analyser~: Faible
    &
    Réaliser et valider~: Faible
    &
    Communication~: Faible
    \\

    \hline

    \multicolumn{2}{|>{\hsize = 2\hsize}X|}{
    \underline{Cours}~: Relation de conjugaison d'un miroir
    } & \multicolumn{2}{>{\hsize = 2\hsize}X|}{\vspace{.5em}

    \underline{Commentaires}~: Relation connue, vocabulaire méconnu
    («~symétrique~», très beau schéma avec distances et angles positifs, mais
    démonstration presque impossible. Au départ, angles aux mauvais endroits
    (pas sur la normale au dioptre), et même en le donnant la trigonométrie ne
    vient que difficilement. Question non finie.

    \vspace{.5em}}\\
          
    \hline

    \multicolumn{2}{|>{\hsize = 2\hsize}X|}{
    \underline{Exercice}~: Détecteur de pluie sur un pare-brise
    } & \multicolumn{2}{>{\hsize = 2\hsize}X|}{\vspace{.5em}
    \underline{Commentaires}~: Exercice facile vu en TD et avec corrigé en
    ligne, mais pas maîtrisé. Réflexion totale esquivée. Bon sens de réfraction
    nonobstant. Encore une fois trigonométrie faible~: $\cos$ au lieu de $\tan$…
    Dommage. Beau schéma.
    \vspace{.5em}}\\
        
    \hline

    \multicolumn{2}{|>{\hsize = 2\hsize}X|}{
    \underline{Tableau}~: Propre. Bien vide, mais propre.
    } & \multicolumn{2}{>{\hsize = 2\hsize}X|}{\vspace{.5em}
    \underline{Oral}~: Pas assez de confiance, persévérez.
    \vspace{.5em}}\\

    \hline
\end{tabularx}

\smallbreak

\begin{tabularx}{\linewidth}{|Y*{4}{|Y}}\hline
    \multicolumn{3}{|l|}{
    NOM Prénom~: DUPRÉ Erwan
    } & \multicolumn{1}{r|}{
    Note~: 11/20
    }\\\hline
    Connaissances~: Bien
    &
    S'approprier, analyser~: Bien
    &
    Réaliser et valider~: Moyen
    &
    Communication~: Bof
    \\

    \hline

    \multicolumn{2}{|>{\hsize = 2\hsize}X|}{
    \underline{Cours}~: Aplanétisme, stigmatisme et conditions de Gauss
    } & \multicolumn{2}{>{\hsize = 2\hsize}X|}{\vspace{.5em}

    \underline{Commentaires}~: Aplanétisme ok, schéma bien. Stigmatisme ok.
    Paraxiaux que défini comme peu inclinés. Réponse ok pour le détecteur.

    \vspace{.5em}}\\
          
    \hline

    \multicolumn{2}{|>{\hsize = 2\hsize}X|}{
    \underline{Exercice}~: Détecteur de niveau d'eau
    } & \multicolumn{2}{>{\hsize = 2\hsize}X|}{\vspace{.5em}
    \underline{Commentaires}~: Bonne intuition, mais réalisation moyenne. C'est
    bien de ne pas trop écrire au tableau pour présenter à l'oral, mais encore
    faut-il le faire au moment utile. Justifiez vos schémas. Pareil, réflexion
    totale à démontrer. Ok pour la justification de $i_1 = 45°$.
    \vspace{.5em}}\\
        
    \hline

    \multicolumn{2}{|>{\hsize = 2\hsize}X|}{
    \underline{Tableau}~: Propre. Assez vide, mais propre.
    } & \multicolumn{2}{>{\hsize = 2\hsize}X|}{\vspace{.5em}
    \underline{Oral}~: Se justifie trop (soirée la veille, pas de calculatrice).
    Soyez plus scientifique dans vos présentations au tableau.
    \vspace{.5em}}\\

    \hline
\end{tabularx}

\end{document}
