\documentclass[12pt, a5paper, garamond, landscape]{book}

\usepackage[utf8x]{inputenc}
\usepackage[french]{babel}
\frenchbsetup{StandardLists=true}
\usepackage[T1]{fontenc}

\usepackage[top=1.5cm, bottom=1cm, left=1.5cm, right=1.5cm]{geometry}
\setlength{\parindent}{0pt}
\usepackage{multicol}
\usepackage{fancyhdr}

\usepackage{tabularx}
\newcolumntype{Y}{>{\centering\arraybackslash}X}
\renewcommand\tabularxcolumn[1]{m{#1}}

%%% En-tête et pied de page

\pagestyle{fancy}
\fancyhead[L]{Nora \textsc{NICOLAS}}
\fancyhead[C]{MPSI1 -- Physique-chimie}
\fancyhead[R]{Rapport de khôlle semaine 1}
\fancyfoot[C]{}

\begin{document}

\begin{tabularx}{\linewidth}{|Y*{4}{|Y}}\hline
    \multicolumn{3}{|l|}{
    NOM Prénom~: GOURDET Mathieu
    } & \multicolumn{1}{r|}{
    Note~: 14/20
    }\\\hline
    Connaissances~: Très bien
    &
    S'approprier, analyser~: Bien
    &
    Réaliser et valider~: Bien
    &
    Communication~: Bien
    \\

    \hline

    \multicolumn{2}{|>{\hsize = 2\hsize}X|}{
    \underline{Cours}~: Snell-Descartes et réflexion totale.
    } & \multicolumn{2}{>{\hsize = 2\hsize}X|}{\vspace{.5em}
    \underline{Commentaires}~: Note bien le sens des angles et distances
    positifve, mais angles pas mis par rapport à la normale (qui n'est pas en
    pointillés)~: s'est rattrapé ensuite, notamment dans le texte.
    Plein de fautes d'orthographe par contre, mais bonne présentation et tout à
    fait correct sur $r=-i$ et l'angle limite pour $n_2 < n_1$.
    \vspace{.5em}}\\
          
    \hline

    \multicolumn{2}{|>{\hsize = 2\hsize}X|}{
    \underline{Exercice}~: Réfractomètre d'Abbe.
    } & \multicolumn{2}{>{\hsize = 2\hsize}X|}{\vspace{.5em}
    \underline{Commentaires}~: Schéma brouillon, mais bonne appropriation de
    l'exercice. Principe du retour inverse de la lumière ok. Confusion sur la
    notation des angles donc mauvaise relation entre eux, mais sinon presque
    arrivé au bout de l'exercice, et a bien compris les limites du dispositif.
    Fausse valeur de $n_0 = 0,\!36$ bien commentée en disant que c'était
    impossible.
    \vspace{.5em}}\\
        
    \hline

    \multicolumn{2}{|>{\hsize = 2\hsize}X|}{
    \underline{Tableau}~: Beaucoup d'écriture et schémas rapides, mais résultats
    bien encadrés.
    } & \multicolumn{2}{>{\hsize = 2\hsize}X|}{\vspace{.5em}
    \underline{Oral}~: Bonne présence dans les réponses aux questions.
    \vspace{.5em}}\\

    \hline
\end{tabularx}

%\newpage

\begin{tabularx}{\linewidth}{|Y*{4}{|Y}}\hline
    \multicolumn{3}{|l|}{
    NOM Prénom~: GARANDEL Marine
    } & \multicolumn{1}{r|}{
    Note~: 12/20
    }\\\hline
    Connaissances~: Ok
    &
    S'approprier, analyser~: Moyen
    &
    Réaliser et valider~: Moyen
    &
    Communication~: RAS
    \\

    \hline

    \multicolumn{2}{|>{\hsize = 2\hsize}X|}{
    \underline{Cours}~: Image d'un point et d'un objet par le miroir plan~;
    propriétés du MP~; stigmatisme rigoureux~; réel/virtuel.
    } & \multicolumn{2}{>{\hsize = 2\hsize}X|}{\vspace{.5em}

    \underline{Commentaires}~: Met des axes optiques sur tous les miroirs. Pour
    le point, n'a pas orienté les angles directement mais fait correctement
    quand demandé. Loi de Snell-Descartes de réflexion ok. Construction d'un
    objet avec $F$ et $F'$ comme pour une lentille, donc une certaine
    confusion, mais construction correcte. Propriétés de stigmatisme
    rigoureux ok. Relation de conjugaison du miroir plan~: n'a pas réussi à
    dire «~symétrique~» mais une fois guidée a réussi à dire $\overline{OA'} =
    -\overline{OA}$ (même si ça devrait être $H$ ou $S$).

    \vspace{.5em}}\\
          
    \hline

    \multicolumn{2}{|>{\hsize = 2\hsize}X|}{
    \underline{Exercice}~: Prisme rectangle et chemin à l'intérieur (3
    réflexions totales)
    } & \multicolumn{2}{>{\hsize = 2\hsize}X|}{\vspace{.5em}

    \underline{Commentaires}~: N'a pas bien lu l'énoncé et a fait arriver le
    rayon par la mauvaise face. A trouvé par intuition le premier angle
    d'incidence mais a du mal pour aller plus loin. Peu d'essai montré, il ne
    faut pas hésiter à aller plus vite.

    \vspace{.5em}}\\
        
    \hline

    \multicolumn{2}{|>{\hsize = 2\hsize}X|}{
    \underline{Tableau}~: Beaucoup d'écriture aussi, fait perdre un peu de
    temps. Tableau bien géré et schémas très propres.
    } & \multicolumn{2}{>{\hsize = 2\hsize}X|}{\vspace{.5em}
    \underline{Oral}~: Réservée.
    \vspace{.5em}}\\

    \hline
\end{tabularx}

%\newpage

\begin{tabularx}{\linewidth}{|Y*{4}{|Y}}\hline
    \multicolumn{3}{|l|}{
    NOM Prénom~: FRACSO Yawir
    } & \multicolumn{1}{r|}{
    Note~: 14/20
    }\\\hline
    Connaissances~: Bien
    &
    S'approprier, analyser~: Très bien
    &
    Réaliser et valider~: Bien
    &
    Communication~: Bien
    \\

    \hline

    \multicolumn{2}{|>{\hsize = 2\hsize}X|}{
    \underline{Cours}~: Conditions de Gauss, stigmatisme approché et détecteur.
    } & \multicolumn{2}{>{\hsize = 2\hsize}X|}{\vspace{.5em}

    \underline{Commentaires}~: Faute à «~Gauss~». Présentation des conditions
    OK, mais sans schéma~: fournit un schéma de rayons paraxiaux quand demandé.
    Connaît le terme «~paraxial~». N'a pas directement expliqué qu'on a
    stigmatisme approché si la tâche image d'un point objet est plus petite que
    la cellule photosensible, mais l'a énoncé avec une question supplémentaire
    (parlait sinon d'un diaphragme permettant d'être dans les conditions de
    Gauss).

    \vspace{.5em}}\\
          
    \hline

    \multicolumn{2}{|>{\hsize = 2\hsize}X|}{
    \underline{Exercice}~: Détecteur de niveau d'eau (cuve avec coins à
    45°, réflexion totale dans l'air et plus quand plongée dans l'eau).
    } & \multicolumn{2}{>{\hsize = 2\hsize}X|}{\vspace{.5em}
    \underline{Commentaires}~: Très bien, clair sur la réflexion totale et sur
    la justification du retour en $D$ avec les deux déviations. Redémontre
    même l'angle limite tout en finissant l'exercice. Très bonne appropriation,
    attention aux applications numériques.
    \vspace{.5em}}\\
        
    \hline

    \multicolumn{2}{|>{\hsize = 2\hsize}X|}{
    \underline{Tableau}~: Bien géré.
    } & \multicolumn{2}{>{\hsize = 2\hsize}X|}{\vspace{.5em}
    \underline{Oral}~: À l'aise.
    \vspace{.5em}}\\

    \hline
\end{tabularx}

\end{document}
