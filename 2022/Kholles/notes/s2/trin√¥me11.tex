\documentclass[11pt, a4paper, garamond]{book}

\usepackage[utf8x]{inputenc}
\usepackage[french]{babel}
\frenchbsetup{StandardLists=true}
\usepackage[T1]{fontenc}

\usepackage[top=1.5cm, bottom=1cm, left=1.5cm, right=1.5cm]{geometry}
\setlength{\parindent}{0pt}
\usepackage{multicol}
\usepackage{fancyhdr}

\usepackage{tabularx}
\newcolumntype{Y}{>{\centering\arraybackslash}X}
\renewcommand\tabularxcolumn[1]{m{#1}}

%%% En-tête et pied de page

\pagestyle{fancy}
\fancyhead[L]{Nora \textsc{Nicolas}}
\fancyhead[C]{MPSI3 -- Physique-chimie}
\fancyhead[R]{Rapport de khôlle S2}
\fancyfoot[C]{}

\begin{document}

\begin{tabularx}{\linewidth}{|Y*{4}{|Y}}\hline
    \multicolumn{3}{|l|}{
    NOM Prénom~: HATTON Victor, sujet 3
    } & \multicolumn{1}{r|}{
    Note~: 13/20
    }\\\hline
    Connaissances~: Bien
    &
    S'approprier, analyser~: Moyen
    &
    Réaliser et valider~: Pas mal
    &
    Communication~: Ok
    \\

    \hline

    \multicolumn{2}{|>{\hsize = 2\hsize}X|}{
    \underline{Cours}~: Vidéoprojecteur
    } & \multicolumn{2}{>{\hsize = 2\hsize}X|}{\vspace{.5em}
    \underline{Commentaires}~: Conversion de toutes les grandeurs
    en mètres inutile au début, avec répétition de surcroit ($h = \overline{AB}$)…
    Un peu lent mais sinon bonne maîtrise de l'exercice, bonne technique
    appliquée de l'extraction des données, du résultat attendu, de l'outil et de
    l'application. Par contre, ne mélangez pas littéral et numérique.
    \vspace{.5em}}\\
          
    \hline

    \multicolumn{2}{|>{\hsize = 2\hsize}X|}{
    \underline{Exercice}~: Prisme rectangle
    } & \multicolumn{2}{>{\hsize = 2\hsize}X|}{\vspace{.5em}
    \underline{Commentaires}~: Appropriation ratée, malheureusement. Incidence
    normale = pas de déviation. Loupé la réflexion totale, mais très bonne
    détermination de l'angle de 40°.
    \vspace{.5em}}\\
        
    \hline

    \multicolumn{2}{|>{\hsize = 2\hsize}X|}{
    \underline{Tableau}~: Propre
    } & \multicolumn{2}{>{\hsize = 2\hsize}X|}{\vspace{.5em}
    \underline{Oral}~: Ok assez à l'aise
    \vspace{.5em}}\\

    \hline
\end{tabularx}

\smallbreak

\begin{tabularx}{\linewidth}{|Y*{4}{|Y}}\hline
    \multicolumn{3}{|l|}{
    NOM Prénom~: LOT Benoit, sujet 1
    } & \multicolumn{1}{r|}{
    Note~: 16/20
    }\\\hline
    Connaissances~: Très bien
    &
    S'approprier, analyser~: Ok
    &
    Réaliser et valider~: Bien
    &
    Communication~: Clair
    \\

    \hline

    \multicolumn{2}{|>{\hsize = 2\hsize}X|}{
    \underline{Cours}~: Objet virtuel convergente, rayon quelconque divergente
    } & \multicolumn{2}{>{\hsize = 2\hsize}X|}{\vspace{.5em}
    \underline{Commentaires}~: Parfait. Manque juste les sens de comptage
    algébrique.
    \vspace{.5em}}\\
          
    \hline

    \multicolumn{2}{|>{\hsize = 2\hsize}X|}{
    \underline{Exercice}~: Rétroprojecteur
    } & \multicolumn{2}{>{\hsize = 2\hsize}X|}{\vspace{.5em}
    \underline{Commentaires}~: Appropriation super. Par contre ne pas mélanger
    les expressions littérales et numériques. Décomposez bien votre réflexion
    pour ne pas vous fier qu'à votre intuition. Attention à bien maîtriser la
    construction de l'image d'un objet par un miroir plan.
    \vspace{.5em}}\\
        
    \hline

    \multicolumn{2}{|>{\hsize = 2\hsize}X|}{
    \underline{Tableau}~: Propre
    } & \multicolumn{2}{>{\hsize = 2\hsize}X|}{\vspace{.5em}
    \underline{Oral}~: Net et précis
    \vspace{.5em}}\\

    \hline
\end{tabularx}

\smallbreak

\begin{tabularx}{\linewidth}{|Y*{4}{|Y}}\hline
    \multicolumn{3}{|l|}{
    NOM Prénom~: SICCA Pierrick, sujet 4
    } & \multicolumn{1}{r|}{
    Note~: 14/20
    }\\\hline
    Connaissances~: Bien
    &
    S'approprier, analyser~: Moyen
    &
    Réaliser et valider~: --
    &
    Communication~: RAS
    \\

    \hline

    \multicolumn{2}{|>{\hsize = 2\hsize}X|}{
    \underline{Cours}~: Champ de vision à travers un miroir plan
    } & \multicolumn{2}{>{\hsize = 2\hsize}X|}{\vspace{.5em}
    \underline{Commentaires}~: Parfait. Un peu de doute mais maîtrisé.
    \vspace{.5em}}\\
          
    \hline

    \multicolumn{2}{|>{\hsize = 2\hsize}X|}{
    \underline{Exercice}~: Réfractomètre d'Abbe
    } & \multicolumn{2}{>{\hsize = 2\hsize}X|}{\vspace{.5em}
    \underline{Commentaires}~: Justification moyenne de l'angle de sortie, et
    non-perception de la réflexion totale~: c'est dommage. Bien sur l'expression
    de l'angle limite, mais \textbf{attention} à la relation $n < n_{\rm air}$
    !!
    \vspace{.5em}}\\
        
    \hline

    \multicolumn{2}{|>{\hsize = 2\hsize}X|}{
    \underline{Tableau}~: Ok
    } & \multicolumn{2}{>{\hsize = 2\hsize}X|}{\vspace{.5em}
    \underline{Oral}~: RAS
    \vspace{.5em}}\\

    \hline
\end{tabularx}

\end{document}
