\documentclass[11pt, a4paper, garamond]{book}

\usepackage[utf8x]{inputenc}
\usepackage[french]{babel}
\frenchbsetup{StandardLists=true}
\usepackage[T1]{fontenc}

\usepackage[top=1.5cm, bottom=1cm, left=1.5cm, right=1.5cm]{geometry}
\setlength{\parindent}{0pt}
\usepackage{multicol}
\usepackage{fancyhdr}

\usepackage{tabularx}
\newcolumntype{Y}{>{\centering\arraybackslash}X}
\renewcommand\tabularxcolumn[1]{m{#1}}

%%% En-tête et pied de page

\pagestyle{fancy}
\fancyhead[L]{Nora \textsc{Nicolas}}
\fancyhead[C]{MPSI3 -- Physique-chimie}
\fancyhead[R]{Rapport de khôlle S2}
\fancyfoot[C]{}

\begin{document}

\begin{tabularx}{\linewidth}{|Y*{4}{|Y}}\hline
    \multicolumn{3}{|l|}{
    NOM Prénom~: LILA Huggo, sujet 2
    } & \multicolumn{1}{r|}{
    Note~: 17/20
    }\\\hline
    Connaissances~: Très bien
    &
    S'approprier, analyser~: Bien
    &
    Réaliser et valider~: Ok
    &
    Communication~: À continuer
    \\

    \hline

    \multicolumn{2}{|>{\hsize = 2\hsize}X|}{
    \underline{Cours}~: Bessel
    } & \multicolumn{2}{>{\hsize = 2\hsize}X|}{\vspace{.5em}
    \underline{Commentaires}~: Super, sens algébriques ok, bonne présentation.
    \vspace{.5em}}\\
          
    \hline

    \multicolumn{2}{|>{\hsize = 2\hsize}X|}{
    \underline{Exercice}~: grenouille intelligente
    } & \multicolumn{2}{>{\hsize = 2\hsize}X|}{\vspace{.5em}
    \underline{Commentaires}~: Un peu perdu sur le départ. Mais avec un guidage
    a réussi à comprendre, s'approprier et presque réaliser. Un peu lent.
    \vspace{.5em}}\\
        
    \hline

    \multicolumn{2}{|>{\hsize = 2\hsize}X|}{
    \underline{Tableau}~: Propre
    } & \multicolumn{2}{>{\hsize = 2\hsize}X|}{\vspace{.5em}
    \underline{Oral}~: Un peu réservé mais très bonne présentation quand elle
    est faite.
    \vspace{.5em}}\\

    \hline
\end{tabularx}

\smallbreak

\begin{tabularx}{\linewidth}{|Y*{4}{|Y}}\hline
    \multicolumn{3}{|l|}{
    NOM Prénom~: LAPARRA Thomas, sujet 1
    } & \multicolumn{1}{r|}{
    Note~: 11/20
    }\\\hline
    Connaissances~: Moyen
    &
    S'approprier, analyser~: Moyen
    &
    Réaliser et valider~: Moyen
    &
    Communication~: Moyen
    \\

    \hline

    \multicolumn{2}{|>{\hsize = 2\hsize}X|}{
    \underline{Cours}~: Objet virtuel convergente et quelconque divergente
    } & \multicolumn{2}{>{\hsize = 2\hsize}X|}{\vspace{.5em}
    \underline{Commentaires}~: Ok, un peu lent sur la présentation. Sens
    algébriques bien. Attention lentille divergente : $F'$ à gauche ! N'a pas su
    énoncer les règles de constructions secondaires. Un peu de panique à la fin
    alors que tracés super quand la lentille était convergente. 7/10
    \vspace{.5em}}\\
          
    \hline

    \multicolumn{2}{|>{\hsize = 2\hsize}X|}{
    \underline{Exercice}~: rétroprojecteur
    } & \multicolumn{2}{>{\hsize = 2\hsize}X|}{\vspace{.5em}
    \underline{Commentaires}~: Pas de schéma~!! Bonne intuition du phénomène
    mais approche peu rigoureuse. On ne mélange pas expressions littérales et
    numériques. Et surtout… on ne parlera pas de cette affrosité au tableau.
    \vspace{.5em}}\\
        
    \hline

    \multicolumn{2}{|>{\hsize = 2\hsize}X|}{
    \underline{Tableau}~: Propre.
    } & \multicolumn{2}{>{\hsize = 2\hsize}X|}{\vspace{.5em}
    \underline{Oral}~: Hésitant mais bonne utilisation du vocabulaire.
    \vspace{.5em}}\\

    \hline
\end{tabularx}

\smallbreak

\begin{tabularx}{\linewidth}{|Y*{4}{|Y}}\hline
    \multicolumn{3}{|l|}{
    NOM Prénom~: OUECHTATI Yahia
    } & \multicolumn{1}{r|}{
    Note~: 05/20
    }\\\hline
    Connaissances~: --
    &
    S'approprier, analyser~: --
    &
    Réaliser et valider~: --
    &
    Communication~: --
    \\

    \hline

    \multicolumn{2}{|>{\hsize = 2\hsize}X|}{
    \underline{Cours}~: Champ de vision à travers un miroir plan.
    } & \multicolumn{2}{>{\hsize = 2\hsize}X|}{\vspace{.5em}
    \underline{Commentaires}~: Aucune réponse sans guidage intense. Donnez de la
    matière à donner des points à la personne qui vous note.
    \vspace{.5em}}\\
          
    \hline

    \multicolumn{2}{|>{\hsize = 2\hsize}X|}{
    \underline{Exercice}~: Réfractomètre d'Abbe
    } & \multicolumn{2}{>{\hsize = 2\hsize}X|}{\vspace{.5em}
    \underline{Commentaires}~: --
    \vspace{.5em}}\\
        
    \hline

    \multicolumn{2}{|>{\hsize = 2\hsize}X|}{
    \underline{Tableau}~: --
    } & \multicolumn{2}{>{\hsize = 2\hsize}X|}{\vspace{.5em}
    \underline{Oral}~: Ne se démonte pas, assez positif.
    \vspace{.5em}}\\

    \hline
\end{tabularx}

\end{document}
