\documentclass[11pt, a4paper, garamond]{book}

\usepackage[utf8x]{inputenc}
\usepackage[french]{babel}
\frenchbsetup{StandardLists=true}
\usepackage[T1]{fontenc}

\usepackage[top=1.5cm, bottom=1cm, left=1.5cm, right=1.5cm]{geometry}
\setlength{\parindent}{0pt}
\usepackage{multicol}
\usepackage{fancyhdr}

\usepackage{tabularx}
\newcolumntype{Y}{>{\centering\arraybackslash}X}
\renewcommand\tabularxcolumn[1]{m{#1}}

%%% En-tête et pied de page

\pagestyle{fancy}
\fancyhead[L]{Nora \textsc{NICOLAS}}
\fancyhead[C]{MPSI3 -- Physique-chimie}
\fancyhead[R]{Rapport de khôlle S2}
\fancyfoot[C]{}

\begin{document}

\begin{tabularx}{\linewidth}{|Y*{4}{|Y}}\hline
    \multicolumn{3}{|l|}{
    NOM Prénom~: TRAVERS Alizé, sujet 1
    } & \multicolumn{1}{r|}{
    Note~: 11.5/20
    }\\\hline
    Connaissances~: Moyen
    &
    S'approprier, analyser~: Bof
    &
    Réaliser et valider~: --
    &
    Communication~: Hésitante.
    \\

    \hline

    \multicolumn{2}{|>{\hsize = 2\hsize}X|}{
    \underline{Cours}~: Tracés objet virtuel lentille convergente et rayon
    quelconque divergente.
    } & \multicolumn{2}{>{\hsize = 2\hsize}X|}{\vspace{.5em}
    \underline{Commentaires}~: Attention à bien comprendre l'énoncé~: l'objet
    \textit{après} le centre optique doit bien être après $O$. Trop de
    difficulté avec l'objet virtuel, \textbf{à réviser} absolument.
    \vspace{.5em}}\\
          
    \hline

    \multicolumn{2}{|>{\hsize = 2\hsize}X|}{
    \underline{Exercice}~: Grenouille intelligente.
    } & \multicolumn{2}{>{\hsize = 2\hsize}X|}{\vspace{.5em}
    \underline{Commentaires}~: Pas de réelle appropriation. Faites travailler
    votre sens physique ! Écrivez ce que vous savez~: d'un milieu plus à moins
    réfringent, je peux avoir réflexion totale… etc. Super sur la fin pour la
    relation trigonométrique et le calcul de l'angle limite.
    \vspace{.5em}}\\
        
    \hline

    \multicolumn{2}{|>{\hsize = 2\hsize}X|}{
    \underline{Tableau}~: Super sur les couleurs, les grandeurs algébriques, les
    rayons.
    } & \multicolumn{2}{>{\hsize = 2\hsize}X|}{\vspace{.5em}
    \underline{Oral}~: Réservée.
    \vspace{.5em}}\\

    \hline
\end{tabularx}

\smallbreak

\begin{tabularx}{\linewidth}{|Y*{4}{|Y}}\hline
    \multicolumn{3}{|l|}{
    NOM Prénom~: NDOIZINE-WANGO Lionel, sujet 3
    } & \multicolumn{1}{r|}{
    Note~: 11.5/20
    }\\\hline
    Connaissances~: Faible
    &
    S'approprier, analyser~: Moyen
    &
    Réaliser et valider~: Faible
    &
    Communication~: Bof
    \\

    \hline

    \multicolumn{2}{|>{\hsize = 2\hsize}X|}{
    \underline{Cours}~: Vidéoprojecteur.
    } & \multicolumn{2}{>{\hsize = 2\hsize}X|}{\vspace{.5em}
    \underline{Commentaires}~: Difficulté à partir. Appelle l'objet BA et pas
    AB, curieux. Même avec relation de conjugaison calculs non réussis.
    Grandissement pas connu. Pas de retour critique sur la valeur et sur le sens
    physique. \textbf{Attention à bien mettre les traits au-dessus des grandeurs
    algébriques}.
    \vspace{.5em}}\\
          
    \hline

    \multicolumn{2}{|>{\hsize = 2\hsize}X|}{
    \underline{Exercice}~: Prisme rectangle.
    } & \multicolumn{2}{>{\hsize = 2\hsize}X|}{\vspace{.5em}
    \underline{Commentaires}~: Pas beaucoup de temps et attention à l'incidence
    normale… Sinon calculs ok.
    \vspace{.5em}}\\
        
    \hline

    \multicolumn{2}{|>{\hsize = 2\hsize}X|}{
    \underline{Tableau}~: Super sur les couleurs. Bien pour les grandeurs
    algébriques.
    } & \multicolumn{2}{>{\hsize = 2\hsize}X|}{\vspace{.5em}
    \underline{Oral}~: RAS
    \vspace{.5em}}\\

    \hline
\end{tabularx}

\smallbreak

\begin{tabularx}{\linewidth}{|Y*{4}{|Y}}\hline
    \multicolumn{3}{|l|}{
    NOM Prénom~: NOGIER Guillaume, sujet 4
    } & \multicolumn{1}{r|}{
    Note~: 13/20
    }\\\hline
    Connaissances~: Très bien
    &
    S'approprier, analyser~: Moyen
    &
    Réaliser et valider~: Moyen
    &
    Communication~: Bien
    \\

    \hline

    \multicolumn{2}{|>{\hsize = 2\hsize}X|}{
    \underline{Cours}~: Champ de vision à travers un miroir plan.
    } & \multicolumn{2}{>{\hsize = 2\hsize}X|}{\vspace{.5em}
    \underline{Commentaires}~: Super sur tout. Bravo. Encadrez bien vos
    résultats littéraux et numériques.
    \vspace{.5em}}\\
          
    \hline

    \multicolumn{2}{|>{\hsize = 2\hsize}X|}{
    \underline{Exercice}~: Réfractomètre d'Abbe.
    } & \multicolumn{2}{>{\hsize = 2\hsize}X|}{\vspace{.5em}
    \underline{Commentaires}~: Intuition ok mais peu d'explications. Pas de
    schéma pour la 2 donc forcément faux, et appropriation de la question 4
    manquante.
    \vspace{.5em}}\\
        
    \hline

    \multicolumn{2}{|>{\hsize = 2\hsize}X|}{
    \underline{Tableau}~: Très bonne utilisation des couleurs. Rayons fléchés
    ok. Construction impeccable.
    } & \multicolumn{2}{>{\hsize = 2\hsize}X|}{\vspace{.5em}
    \underline{Oral}~: Ok à l'aise.
    \vspace{.5em}}\\

    \hline
\end{tabularx}

\end{document}
