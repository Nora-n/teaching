\documentclass[12pt, a5paper, garamond, landscape]{book}

\usepackage[utf8x]{inputenc}
\usepackage[french]{babel}
\frenchbsetup{StandardLists=true}
\usepackage[T1]{fontenc}

\usepackage[top=1.5cm, bottom=1cm, left=1.5cm, right=1.5cm]{geometry}
\setlength{\parindent}{0pt}
\usepackage{multicol}
\usepackage{fancyhdr}

\usepackage{tabularx}
\newcolumntype{Y}{>{\centering\arraybackslash}X}
\renewcommand\tabularxcolumn[1]{m{#1}}

%%% En-tête et pied de page

\pagestyle{fancy}
\fancyhead[L]{Nora \textsc{NICOLAS}}
\fancyhead[C]{MPSI2 -- Physique-chimie}
\fancyhead[R]{Rapport de khôlle S2}
\fancyfoot[C]{}

\begin{document}

\begin{tabularx}{\linewidth}{|Y*{4}{|Y}}\hline
    \multicolumn{3}{|l|}{
    NOM Prénom~: THOMAS Mathis
    } & \multicolumn{1}{r|}{
    Note~: 13/20
    }\\\hline
    Connaissances~: Super
    &
    S'approprier, analyser~: Faible
    &
    Réaliser et valider~: --
    &
    Communication~: Ok
    \\

    \hline

    \multicolumn{2}{|>{\hsize = 2\hsize}X|}{
    \underline{Cours}~: Relations de Descartes.
    } & \multicolumn{2}{>{\hsize = 2\hsize}X|}{\vspace{.5em}
    \underline{Commentaires}~: Super schémas. Parfait.
    \vspace{.5em}}\\
          
    \hline

    \multicolumn{2}{|>{\hsize = 2\hsize}X|}{
    \underline{Exercice}~: Grenouille intelligente.
    } & \multicolumn{2}{>{\hsize = 2\hsize}X|}{\vspace{.5em}
    \underline{Commentaires}~: Pratiquement rien de fait. S'emmêle les pinceaux
    sur les conditions de réfraction et réflexion totale. Ok pour la formule de
    l'angle limite, mais mauvaise application. Dommage.
    \vspace{.5em}}\\
        
    \hline

    \multicolumn{2}{|>{\hsize = 2\hsize}X|}{
    \underline{Tableau}~: Extrêmement propre.
    } & \multicolumn{2}{>{\hsize = 2\hsize}X|}{\vspace{.5em}
    \underline{Oral}~: Super présence.
    \vspace{.5em}}\\

    \hline
\end{tabularx}

\newpage

\begin{tabularx}{\linewidth}{|Y*{4}{|Y}}\hline
    \multicolumn{3}{|l|}{
    NOM Prénom~: JOLLY Perceval
    } & \multicolumn{1}{r|}{
    Note~: 13/20
    }\\\hline
    Connaissances~: Bien -
    &
    S'approprier, analyser~: Moyen
    &
    Réaliser et valider~: Ok
    &
    Communication~: RAS
    \\

    \hline

    \multicolumn{2}{|>{\hsize = 2\hsize}X|}{
        \underline{Cours}~: Méthode de \textsc{Bessel}.
    } & \multicolumn{2}{>{\hsize = 2\hsize}X|}{\vspace{.5em}
    \underline{Commentaires}~: Connaît la pratique, pas la technique. Guidage
    nécessaire. Faute de calcul ($ \frac{1}{a} - \frac{1}{b} = \frac{a-b}{ab}$),
    mais après discriminant ok.
    \vspace{.5em}}\\
          
    \hline

    \multicolumn{2}{|>{\hsize = 2\hsize}X|}{
    \underline{Exercice}~: Champ de vision à travers un miroir plan.
    } & \multicolumn{2}{>{\hsize = 2\hsize}X|}{\vspace{.5em}
    \underline{Commentaires}~: Schéma ok, mais pas compris la formation de
    l'image de la personne. ATTENTION les angles sont par rapport à la normale
    au dioptre. Les rayons doivent être fléchés. Bien sur l'arbre, valeur finale
    trouvée et Thalès bien appliqué~; assez confus mais réponse trouvée en
    autonomie.
    \vspace{.5em}}\\
        
    \hline

    \multicolumn{2}{|>{\hsize = 2\hsize}X|}{
    \underline{Tableau}~: Petit mais plein de liens logiques. Bien sur les
    couleurs.
    } & \multicolumn{2}{>{\hsize = 2\hsize}X|}{\vspace{.5em}
    \underline{Oral}~: Ok.
    \vspace{.5em}}\\

    \hline
\end{tabularx}

\newpage

\begin{tabularx}{\linewidth}{|Y*{4}{|Y}}\hline
    \multicolumn{3}{|l|}{
    NOM Prénom~: HAMANI Rayan
    } & \multicolumn{1}{r|}{
    Note~: 11/20
    }\\\hline
    Connaissances~: Bof
    &
    S'approprier, analyser~: Bof
    &
    Réaliser et valider~: Bof
    &
    Communication~: Faible
    \\

    \hline

    \multicolumn{2}{|>{\hsize = 2\hsize}X|}{
    \underline{Cours}~: Relations de \textsc{Newton}.
    } & \multicolumn{2}{>{\hsize = 2\hsize}X|}{\vspace{.5em}
    \underline{Commentaires}~: Ne sait pas comment partir. Ok pour la formule,
    attention au $f'_2$ (c'était bien $f'^2$). Schéma trop vide, c'est votre
    support~: utilisez-le. A réussi avec du temps, bravo quand même.
    \vspace{.5em}}\\
          
    \hline

    \multicolumn{2}{|>{\hsize = 2\hsize}X|}{
    \underline{Exercice}~: Rétroprojecteur.
    } & \multicolumn{2}{>{\hsize = 2\hsize}X|}{\vspace{.5em}
    \underline{Commentaires}~: Commence par un schéma retourné à -90° de celui
    de l'énoncé~: dangereux. Schéma plus vide que celui de l'énoncé~: c'est
    votre support, utilisez-le. Aucune schématisation optique du système, il
    avance à l'aveugle.
    \vspace{.5em}}\\
        
    \hline

    \multicolumn{2}{|>{\hsize = 2\hsize}X|}{
    \underline{Tableau}~: Schémas petits mais tableau propre et écriture grande
    et lisible.
    } & \multicolumn{2}{>{\hsize = 2\hsize}X|}{\vspace{.5em}
    \underline{Oral}~: RAS (mais plutôt moyen).
    \vspace{.5em}}\\

    \hline
\end{tabularx}

\end{document}
