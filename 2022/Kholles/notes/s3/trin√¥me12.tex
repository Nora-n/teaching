\documentclass[11pt, a4paper, garamond]{book}

\usepackage[utf8x]{inputenc}
\usepackage[french]{babel}
\frenchbsetup{StandardLists=true}
\usepackage[T1]{fontenc}

\usepackage[top=1.5cm, bottom=1cm, left=1.5cm, right=1.5cm]{geometry}
\setlength{\parindent}{0pt}
\usepackage{multicol}
\usepackage{fancyhdr}

\usepackage{tabularx}
\newcolumntype{Y}{>{\centering\arraybackslash}X}
\renewcommand\tabularxcolumn[1]{m{#1}}

%%% En-tête et pied de page

\pagestyle{fancy}
\fancyhead[L]{Nora \textsc{NICOLAS}}
\fancyhead[C]{MPSI3 -- Physique-chimie}
\fancyhead[R]{Rapport de khôlle S3}
\fancyfoot[C]{}

\begin{document}

\begin{tabularx}{\linewidth}{|Y*{4}{|Y}}\hline
    \multicolumn{3}{|l|}{
    NOM Prénom~: MALHERBE Ylan, sujet 2
    } & \multicolumn{1}{r|}{
    Note~: 13/20
    }\\\hline
    Connaissances~: Bien
    &
    S'approprier, analyser~: Bien
    &
    Réaliser et valider~: Bien
    &
    Communication~: Ok
    \\

    \hline

    \multicolumn{2}{|>{\hsize = 2\hsize}X|}{
    \underline{Cours}~: Microscope : image d'un objet, grossissement, ordre de
    grandeur.
    } & \multicolumn{2}{>{\hsize = 2\hsize}X|}{\vspace{.5em}
    \underline{Commentaires}~: Ok sur le tracé et le principe. Schématisation
    optique ok, schéma pas très propre. Confusion sur définition du
    grossissement, oral par pareil que l'écrit. Écrit un angle négatif pour un
    angle qui ne l'est pas. Ordre de grandeur et signe ok.
    \vspace{.5em}}\\
          
    \hline

    \multicolumn{2}{|>{\hsize = 2\hsize}X|}{
    \underline{Exercice}~: Doublet tout court
    } & \multicolumn{2}{>{\hsize = 2\hsize}X|}{\vspace{.5em}
    \underline{Commentaires}~: Très bien sur la construction, position bien
    trouvée. Brouillon dans les relations de conjugaison, pas cohérent entre
    schématisation et schéma. Calcul bien effectué cependant, mais on regrette
    une certaine lenteur.
    \vspace{.5em}}\\
        
    \hline

    \multicolumn{2}{|>{\hsize = 2\hsize}X|}{
    \underline{Tableau}~: Un peu sale mais bien géré.
    } & \multicolumn{2}{>{\hsize = 2\hsize}X|}{\vspace{.5em}
    \underline{Oral}~: Ok, un peu hésitant mais impliqué.
    \vspace{.5em}}\\

    \hline
\end{tabularx}

\bigbreak

\begin{tabularx}{\linewidth}{|Y*{4}{|Y}}\hline
    \multicolumn{3}{|l|}{
    NOM Prénom~: PETITGAS Benjamin, sujet 3
    } & \multicolumn{1}{r|}{
    Note~: 11/20
    }\\\hline
    Connaissances~: Ok
    &
    S'approprier, analyser~: Moyen
    &
    Réaliser et valider~: --
    &
    Communication~: RAS
    \\

    \hline

    \multicolumn{2}{|>{\hsize = 2\hsize}X|}{
    \underline{Cours}~: Théorème des vergences.
    } & \multicolumn{2}{>{\hsize = 2\hsize}X|}{\vspace{.5em}
    \underline{Commentaires}~: Très bien. Attention aux notations ! $F'_2$ n'est
    pas $f'_2$. Ne pas oublier de barre sur les grandeurs algébriques.
    \vspace{.5em}}\\
          
    \hline

    \multicolumn{2}{|>{\hsize = 2\hsize}X|}{
    \underline{Exercice}~: Lunette astronomique
    } & \multicolumn{2}{>{\hsize = 2\hsize}X|}{\vspace{.5em}
    \underline{Commentaires}~: Définition hasardeuse~: est afocal tout système
    (pas que 2 associations) qui fait d'un objet \textbf{à l'infini} une image à
    l'infini. Ensuite, l'objet sur le schéma n'était pas à l'infini et les rayons
    pas fléchés. Le grossissement est défini correctement mais $\alpha'$ n'a pas
    le même signe que sur le schéma, et il manque de barres. Physiquement
    grossissement < 0 ok quand même, et calcul pas mal. Par contre fini bien
    trop tôt, ces questions c'était juste le cours…
    \vspace{.5em}}\\
        
    \hline

    \multicolumn{2}{|>{\hsize = 2\hsize}X|}{
    \underline{Tableau}~: Ok, assez propre.
    } & \multicolumn{2}{>{\hsize = 2\hsize}X|}{\vspace{.5em}
    \underline{Oral}~: RAS.
    \vspace{.5em}}\\

    \hline
\end{tabularx}

\bigbreak

\begin{tabularx}{\linewidth}{|Y*{4}{|Y}}\hline
    \multicolumn{3}{|l|}{
    NOM Prénom~: MOHAMED ELHAD Ambdaloi, sujet 4
    } & \multicolumn{1}{r|}{
    Note~: 07/20
    }\\\hline
    Connaissances~: Faible
    &
    S'approprier, analyser~: --
    &
    Réaliser et valider~: --
    &
    Communication~: RAS
    \\

    \hline

    \multicolumn{2}{|>{\hsize = 2\hsize}X|}{
    \underline{Cours}~: Lunette de \textsc{Kepler}
    } & \multicolumn{2}{>{\hsize = 2\hsize}X|}{\vspace{.5em}
    \underline{Commentaires}~: Présentation bancale. Lunette afocale
    certes, mais pour voir des objets à l'infini. Rayons faits un peu au hasard.
    Ne sait pas énoncer les règles de construction secondaire.
    \vspace{.5em}}\\
          
    \hline

    \multicolumn{2}{|>{\hsize = 2\hsize}X|}{
    \underline{Exercice}~: Catadioptre
    } & \multicolumn{2}{>{\hsize = 2\hsize}X|}{\vspace{.5em}
    \underline{Commentaires}~: Peu de temps mais pas d'appropriation.
    \vspace{.5em}}\\
        
    \hline

    \multicolumn{2}{|>{\hsize = 2\hsize}X|}{
    \underline{Tableau}~: Correct, propre mais vide.
    } & \multicolumn{2}{>{\hsize = 2\hsize}X|}{\vspace{.5em}
    \underline{Oral}~: Maîtrisez le vocabulaire.
    \vspace{.5em}}\\

    \hline
\end{tabularx}

\end{document}
