\documentclass[11pt, a4paper, garamond]{book}

\usepackage[utf8x]{inputenc}
\usepackage[french]{babel}
\frenchbsetup{StandardLists=true}
\usepackage[T1]{fontenc}

\usepackage[top=1.5cm, bottom=1cm, left=1.5cm, right=1.5cm]{geometry}
\setlength{\parindent}{0pt}
\usepackage{multicol}
\usepackage{fancyhdr}

\usepackage{tabularx}
\newcolumntype{Y}{>{\centering\arraybackslash}X}
\renewcommand\tabularxcolumn[1]{m{#1}}

%%% En-tête et pied de page

\pagestyle{fancy}
\fancyhead[L]{Nora \textsc{Nicolas}}
\fancyhead[C]{MPSI3 -- Physique-chimie}
\fancyhead[R]{Rapport de khôlle S3}
\fancyfoot[C]{}

\begin{document}

\begin{tabularx}{\linewidth}{|Y*{4}{|Y}}\hline
    \multicolumn{3}{|l|}{
    NOM Prénom~: SIGAL Quentin, sujet 4
    } & \multicolumn{1}{r|}{
    Note~: 14/20
    }\\\hline
    Connaissances~: Bien
    &
    S'approprier, analyser~: Super
    &
    Réaliser et valider~: Ok
    &
    Communication~: Bien
    \\

    \hline

    \multicolumn{2}{|>{\hsize = 2\hsize}X|}{
    \underline{Cours}~: Lunette Kepler
    } & \multicolumn{2}{>{\hsize = 2\hsize}X|}{\vspace{.5em}
    \underline{Commentaires}~: Rayons pas fléchés. Angles pas orientés.
    Grandeurs algébriques pas définies. Schéma très beau, très propre, très
    grand. Schématisation optique super. Grossissement bien défini mais
    expression et démonstration compliquées. Ordre de grandeur ok.
    \vspace{.5em}}\\
          
    \hline

    \multicolumn{2}{|>{\hsize = 2\hsize}X|}{
    \underline{Exercice}~: Système catadioptrique
    } & \multicolumn{2}{>{\hsize = 2\hsize}X|}{\vspace{.5em}
    \underline{Commentaires}~: Très bien, effet loupe, image miroir, puis
    intuition d'inverser les points focaux. Calcul ok.
    \vspace{.5em}}\\
        
    \hline

    \multicolumn{2}{|>{\hsize = 2\hsize}X|}{
    \underline{Tableau}~: Très bien géré.
    } & \multicolumn{2}{>{\hsize = 2\hsize}X|}{\vspace{.5em}
    \underline{Oral}~:
    \vspace{.5em}}\\

    \hline
\end{tabularx}

\bigbreak

\begin{tabularx}{\linewidth}{|Y*{4}{|Y}}\hline
    \multicolumn{3}{|l|}{
    NOM Prénom~: PICHON-LEGROUX Gaspard, sujet 2
    } & \multicolumn{1}{r|}{
    Note~: 10/20
    }\\\hline
    Connaissances~: Moyen
    &
    S'approprier, analyser~: Moyen
    &
    Réaliser et valider~: Moyen
    &
    Communication~: RAS
    \\

    \hline

    \multicolumn{2}{|>{\hsize = 2\hsize}X|}{
    \underline{Cours}~: Microscope
    } & \multicolumn{2}{>{\hsize = 2\hsize}X|}{\vspace{.5em}
    \underline{Commentaires}~: Ok, schéma pas incroyablement grand mais
    ok. Intuition physique des rayons émergent parallèles, mais pas
    d'utilisation de la schématisation optique pour corriger le schéma. Rayons
    émergent faits parallèles sans logique (passe par $F'_2$). Grossissement
    défini ok, mais pas sur le schéma~: pour le microscope, $\theta$ c'est pour
    l'objet à $d$ = 25\,cm de l'œil.
    \vspace{.5em}}\\
          
    \hline

    \multicolumn{2}{|>{\hsize = 2\hsize}X|}{
    \underline{Exercice}~: Doublet tout court
    } & \multicolumn{2}{>{\hsize = 2\hsize}X|}{\vspace{.5em}
    \underline{Commentaires}~: Ne sait pas faire l'image finale pour la
    divergente… en fait avec de l'aide si totalement ! Pour les calculs, manque
    de rigueur.
    \vspace{.5em}}\\
        
    \hline

    \multicolumn{2}{|>{\hsize = 2\hsize}X|}{
    \underline{Tableau}~: RAS
    } & \multicolumn{2}{>{\hsize = 2\hsize}X|}{\vspace{.5em}
    \underline{Oral}~: Développer l'autonomie.
    \vspace{.5em}}\\

    \hline
\end{tabularx}

\end{document}
