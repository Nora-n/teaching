\documentclass[12pt, a5paper, garamond, landscape]{book}

\usepackage[utf8x]{inputenc}
\usepackage[french]{babel}
\frenchbsetup{StandardLists=true}
\usepackage[T1]{fontenc}

\usepackage[top=1.5cm, bottom=1cm, left=1.5cm, right=1.5cm]{geometry}
\setlength{\parindent}{0pt}
\usepackage{multicol}
\usepackage{fancyhdr}

\usepackage{tabularx}
\newcolumntype{Y}{>{\centering\arraybackslash}X}
\renewcommand\tabularxcolumn[1]{m{#1}}

%%% En-tête et pied de page

\pagestyle{fancy}
\fancyhead[L]{Nora \textsc{Nicolas}}
\fancyhead[C]{MPSI1 -- Physique-chimie}
\fancyhead[R]{Rapport de khôlle S3}
\fancyfoot[C]{}

\begin{document}

\begin{tabularx}{\linewidth}{|Y*{4}{|Y}}\hline
    \multicolumn{3}{|l|}{
    NOM Prénom~: LE GRAVIER Lucie
    } & \multicolumn{1}{r|}{
    Note~: 16/20
    }\\\hline
    Connaissances~: Super
    &
    S'approprier, analyser~: Très bien
    &
    Réaliser et valider~: Ok
    &
    Communication~: Super
    \\

    \hline

    \multicolumn{2}{|>{\hsize = 2\hsize}X|}{
    \underline{Cours}~: Lunette astronomique et grossissement
    } & \multicolumn{2}{>{\hsize = 2\hsize}X|}{\vspace{.5em}
    \underline{Commentaires}~: Pratiquement parfait. Attention au sens de
    comptage algébrique des angles.
    \vspace{.5em}}\\
          
    \hline

    \multicolumn{2}{|>{\hsize = 2\hsize}X|}{
    \underline{Exercice}~: Doublet focal.
    } & \multicolumn{2}{>{\hsize = 2\hsize}X|}{\vspace{.5em}
    \underline{Commentaires}~: Très bonne intuition pour le foyer objet. Ok pour
    le foyer image. Dommage, pas eu le temps de faire les calculs.
    \vspace{.5em}}\\
        
    \hline

    \multicolumn{2}{|>{\hsize = 2\hsize}X|}{
    \underline{Tableau}~: Très bien géré et très propre.
    } & \multicolumn{2}{>{\hsize = 2\hsize}X|}{\vspace{.5em}
    \underline{Oral}~: Très bonne présentation.
    \vspace{.5em}}\\

    \hline
\end{tabularx}

\newpage

\begin{tabularx}{\linewidth}{|Y*{4}{|Y}}\hline
    \multicolumn{3}{|l|}{
    NOM Prénom~: LE COCGUEN Guillain
    } & \multicolumn{1}{r|}{
    Note~: 16/20
    }\\\hline
    Connaissances~: Très bien
    &
    S'approprier, analyser~: Ok
    &
    Réaliser et valider~: RAS
    &
    Communication~: Bien
    \\

    \hline

    \multicolumn{2}{|>{\hsize = 2\hsize}X|}{
    \underline{Cours}~: $D \geq 4f'$
    } & \multicolumn{2}{>{\hsize = 2\hsize}X|}{\vspace{.5em}
    \underline{Commentaires}~: Attention à bien faire un schéma pour présenter
    le système et les notations. N'oubliez pas les barres sur les grandeurs
    algébriques.
    \vspace{.5em}}\\
          
    \hline

    \multicolumn{2}{|>{\hsize = 2\hsize}X|}{
        \underline{Exercice}~: Lunette astronomique avec $\mathcal{L}_3$ au
        milieu.
    } & \multicolumn{2}{>{\hsize = 2\hsize}X|}{\vspace{.5em}
    \underline{Commentaires}~: Ok sur le début, attention à bien noter les
    angles correctement. Attention également au vocabulaire~: $\mathcal{L}_3$
    conjugue $F_2$ et $F'_1$.
    \vspace{.5em}}\\
        
    \hline

    \multicolumn{2}{|>{\hsize = 2\hsize}X|}{
    \underline{Tableau}~: Bien, mais petits schémas.
    } & \multicolumn{2}{>{\hsize = 2\hsize}X|}{\vspace{.5em}
    \underline{Oral}~: Ok
    \vspace{.5em}}\\

    \hline
\end{tabularx}

\newpage

\begin{tabularx}{\linewidth}{|Y*{4}{|Y}}\hline
    \multicolumn{3}{|l|}{
    NOM Prénom~: MALRIEU Mewen
    } & \multicolumn{1}{r|}{
    Note~: 15/20
    }\\\hline
    Connaissances~: Très bien
    &
    S'approprier, analyser~: Bien
    &
    Réaliser et valider~: Ok
    &
    Communication~: RAS
    \\

    \hline

    \multicolumn{2}{|>{\hsize = 2\hsize}X|}{
    \underline{Cours}~: Relations de conjugaison de Descartes
    } & \multicolumn{2}{>{\hsize = 2\hsize}X|}{\vspace{.5em}
    \underline{Commentaires}~: Super présentation, beau schéma bien fléché.
    \vspace{.5em}}\\
          
    \hline

    \multicolumn{2}{|>{\hsize = 2\hsize}X|}{
    \underline{Exercice}~: Catadioptre
    } & \multicolumn{2}{>{\hsize = 2\hsize}X|}{\vspace{.5em}
    \underline{Commentaires}~: Pas évident sur la conjugaison des points $O$ et
    $F'$, et n'a pas eu l'autonomie de détailler la construction avec les images
    intermédiaires. Pas de schématisation optique non plus, mais calcul amorcé
    et bien effectué.
    \vspace{.5em}}\\
        
    \hline

    \multicolumn{2}{|>{\hsize = 2\hsize}X|}{
    \underline{Tableau}~: Très beau, propre, grands schémas.
    } & \multicolumn{2}{>{\hsize = 2\hsize}X|}{\vspace{.5em}
    \underline{Oral}~: Ok
    \vspace{.5em}}\\

    \hline
\end{tabularx}

\end{document}
