\documentclass[11pt, a4paper, garamond]{book}

\usepackage[utf8x]{inputenc}
\usepackage[french]{babel}
\frenchbsetup{StandardLists=true}
\usepackage[T1]{fontenc}

\usepackage[top=1.5cm, bottom=1cm, left=1.5cm, right=1.5cm]{geometry}
\setlength{\parindent}{0pt}
\usepackage{multicol}
\usepackage{fancyhdr}

\usepackage{tabularx}
\newcolumntype{Y}{>{\centering\arraybackslash}X}
\renewcommand\tabularxcolumn[1]{m{#1}}

%%% En-tête et pied de page

\pagestyle{fancy}
\fancyhead[L]{Nora \textsc{Nicolas}}
\fancyhead[C]{MPSI3 -- Physique-chimie}
\fancyhead[R]{Rapport de khôlle S4}
\fancyfoot[C]{}

\begin{document}

\begin{tabularx}{\linewidth}{|Y*{4}{|Y}}\hline
    \multicolumn{3}{|l|}{
    NOM Prénom~: DAUTRY Eliott, sujet 3
    } & \multicolumn{1}{r|}{
    Note~: 05/20
    }\\\hline
    Connaissances~: Moyen
    &
    S'approprier, analyser~: Faible
    &
    Réaliser et valider~: Faible
    &
    Communication~: RAS
    \\

    \hline

    \multicolumn{2}{|>{\hsize = 2\hsize}X|}{
    \underline{Cours}~: Condensateur et bobine
    } & \multicolumn{2}{>{\hsize = 2\hsize}X|}{\vspace{.5em}
    \underline{Commentaires}~: Que de maigres connaissances sur les deux.
    Énergie connue cependant. 2/10
    \vspace{.5em}}\\
          
    \hline

    \multicolumn{2}{|>{\hsize = 2\hsize}X|}{
    \underline{Exercice}~: Lunette 3 lentilles
    } & \multicolumn{2}{>{\hsize = 2\hsize}X|}{\vspace{.5em}
    \underline{Commentaires}~: Pas grand chose. Explication $F'_1 = F_2$
    absente, tracé aléatoire des rayons de la lunette, grossissement pas calculé
    juste donné et rien de pertinent à partir de la troisième lentille. 3/10
    \vspace{.5em}}\\
        
    \hline

    \multicolumn{2}{|>{\hsize = 2\hsize}X|}{
    \underline{Tableau}~: Petit mais propre
    } & \multicolumn{2}{>{\hsize = 2\hsize}X|}{\vspace{.5em}
    \underline{Oral}~: RAS
    \vspace{.5em}}\\

    \hline
\end{tabularx}

\bigbreak

\begin{tabularx}{\linewidth}{|Y*{4}{|Y}}\hline
    \multicolumn{3}{|l|}{
    NOM Prénom~: MAURICE Jules, sujet 1
    } & \multicolumn{1}{r|}{
    Note~: 05/20
    }\\\hline
    Connaissances~: Faible
    &
    S'approprier, analyser~: Faible
    &
    Réaliser et valider~: Faible
    &
    Communication~: Faible
    \\

    \hline

    \multicolumn{2}{|>{\hsize = 2\hsize}X|}{
    \underline{Cours}~: Ponts diviseurs
    } & \multicolumn{2}{>{\hsize = 2\hsize}X|}{\vspace{.5em}
    \underline{Commentaires}~: Extrêmement confus. Diviseur de courant non
    connu. Réponse tension trouvée avec beaucoup d'aide. 3/10
    \vspace{.5em}}\\
          
    \hline

    \multicolumn{2}{|>{\hsize = 2\hsize}X|}{
    \underline{Exercice}~: Doublet Huygens
    } & \multicolumn{2}{>{\hsize = 2\hsize}X|}{\vspace{.5em}
    \underline{Commentaires}~: Régles de construction secondaires non connues,
    incompréhension complète de l'exercice même quand je l'explique à l'oral.
    2/10
    \vspace{.5em}}\\
        
    \hline

    \multicolumn{2}{|>{\hsize = 2\hsize}X|}{
    \underline{Tableau}~: RAS
    } & \multicolumn{2}{>{\hsize = 2\hsize}X|}{\vspace{.5em}
    \underline{Oral}~: Confus
    \vspace{.5em}}\\

    \hline
\end{tabularx}

\bigbreak

\begin{tabularx}{\linewidth}{|Y*{4}{|Y}}\hline
    \multicolumn{3}{|l|}{
    NOM Prénom~: WATROBA Maxime, sujet 2
    } & \multicolumn{1}{r|}{
    Note~: 20/20
    }\\\hline
    Connaissances~: Très bien
    &
    S'approprier, analyser~: Très bien
    &
    Réaliser et valider~: Très bien
    &
    Communication~: Très bien
    \\

    \hline

    \multicolumn{2}{|>{\hsize = 2\hsize}X|}{
    \underline{Cours}~: Association résistances série parallèle et exercice
    } & \multicolumn{2}{>{\hsize = 2\hsize}X|}{\vspace{.5em}
    \underline{Commentaires}~: Super. Un peu lent nonobstant, mais super. 10/10
    \vspace{.5em}}\\
          
    \hline

    \multicolumn{2}{|>{\hsize = 2\hsize}X|}{
    \underline{Exercice}~: Doublet focal
    } & \multicolumn{2}{>{\hsize = 2\hsize}X|}{\vspace{.5em}
    \underline{Commentaires}~: Parfait.
    \vspace{.5em}}\\
        
    \hline

    \multicolumn{2}{|>{\hsize = 2\hsize}X|}{
    \underline{Tableau}~: Super
    } & \multicolumn{2}{>{\hsize = 2\hsize}X|}{\vspace{.5em}
    \underline{Oral}~: Très bien
    \vspace{.5em}}\\

    \hline
\end{tabularx}

\end{document}
