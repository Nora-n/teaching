\documentclass[12pt, a5paper, garamond, landscape]{book}

\usepackage[utf8x]{inputenc}
\usepackage[french]{babel}
\frenchbsetup{StandardLists=true}
\usepackage[T1]{fontenc}

\usepackage[top=1.5cm, bottom=1cm, left=1.5cm, right=1.5cm]{geometry}
\setlength{\parindent}{0pt}
\usepackage{multicol}
\usepackage{fancyhdr}

\usepackage{tabularx}
\newcolumntype{Y}{>{\centering\arraybackslash}X}
\renewcommand\tabularxcolumn[1]{m{#1}}

%%% En-tête et pied de page

\pagestyle{fancy}
\fancyhead[L]{Nora \textsc{Nicolas}}
\fancyhead[C]{MPSI1 -- Physique-chimie}
\fancyhead[R]{Rapport de khôlle S4}
\fancyfoot[C]{}

\begin{document}

\begin{tabularx}{\linewidth}{|Y*{4}{|Y}}\hline
    \multicolumn{3}{|l|}{
    NOM Prénom~: PEYTHIEU Quentin
    } & \multicolumn{1}{r|}{
    Note~: 18/20
    }\\\hline
    Connaissances~: Bien
    &
    S'approprier, analyser~: Ok
    &
    Réaliser et valider~: Bien
    &
    Communication~: Ok
    \\

    \hline

    \multicolumn{2}{|>{\hsize = 2\hsize}X|}{
    \underline{Cours}~: Condensateur idéal
    } & \multicolumn{2}{>{\hsize = 2\hsize}X|}{\vspace{.5em}
    \underline{Commentaires}~: Bien sur les connaissances mais de la confusion.
    Un condensateur n'est pas équivalent à un interrupteur ouvert dans la
    majorité des cas, mais en régime permanent. Difficulté à voir le lien
    mathématique. Pareil avec la démonstration de $W$. 7/9
    \vspace{.5em}}\\
          
    \hline

    \multicolumn{2}{|>{\hsize = 2\hsize}X|}{
    \underline{Exercice}~: Doublet focal
    } & \multicolumn{2}{>{\hsize = 2\hsize}X|}{\vspace{.5em}
    \underline{Commentaires}~: Très bon schéma même sans couleur, calculs bien
    gérés, pas de confusion entre $O$ et $O_1$ ou $O_2$. Rapide et efficace, a
    même commencé un autre exercice. 11/11
    \vspace{.5em}}\\
        
    \hline

    \multicolumn{2}{|>{\hsize = 2\hsize}X|}{
    \underline{Tableau}~: Bien, un peu chaotique mais bien
    } & \multicolumn{2}{>{\hsize = 2\hsize}X|}{\vspace{.5em}
    \underline{Oral}~: Plus d'appropriation serait apprécié
    \vspace{.5em}}\\

    \hline
\end{tabularx}

\newpage

\begin{tabularx}{\linewidth}{|Y*{4}{|Y}}\hline
    \multicolumn{3}{|l|}{
    NOM Prénom~: ONDET Lucas
    } & \multicolumn{1}{r|}{
    Note~: 18/20
    }\\\hline
    Connaissances~: Très bien
    &
    S'approprier, analyser~: Ok
    &
    Réaliser et valider~: Bien
    &
    Communication~: Ok
    \\

    \hline

    \multicolumn{2}{|>{\hsize = 2\hsize}X|}{
    \underline{Cours}~: ARQS
    } & \multicolumn{2}{>{\hsize = 2\hsize}X|}{\vspace{.5em}
    \underline{Commentaires}~: Très bien, bon vocabulaire et bonne
    compréhension. 9/9
    \vspace{.5em}}\\
          
    \hline

    \multicolumn{2}{|>{\hsize = 2\hsize}X|}{
    \underline{Exercice}~: Lunette avec 3 lentiles
    } & \multicolumn{2}{>{\hsize = 2\hsize}X|}{\vspace{.5em}
    \underline{Commentaires}~: Manque des fléchages sur les rayons et sens de
    comptage algébrique des angles et orientation des angles. Très clair sur la
    base de l'exercice avec 2 lentilles.
    \vspace{.5em}}\\
        
    \hline

    \multicolumn{2}{|>{\hsize = 2\hsize}X|}{
    \underline{Tableau}~: Petit mais propre.
    } & \multicolumn{2}{>{\hsize = 2\hsize}X|}{\vspace{.5em}
    \underline{Oral}~: N'hésite pas à dire qu'il ne sait pas, super.
    \vspace{.5em}}\\

    \hline
\end{tabularx}

\newpage

\begin{tabularx}{\linewidth}{|Y*{4}{|Y}}\hline
    \multicolumn{3}{|l|}{
    NOM Prénom~: MARCELLIN Marie
    } & \multicolumn{1}{r|}{
    Note~: 18/20
    }\\\hline
    Connaissances~: Super
    &
    S'approprier, analyser~: Très bien
    &
    Réaliser et valider~: Très bien
    &
    Communication~: Très bien
    \\

    \hline

    \multicolumn{2}{|>{\hsize = 2\hsize}X|}{
    \underline{Cours}~: Bobine idéale
    } & \multicolumn{2}{>{\hsize = 2\hsize}X|}{\vspace{.5em}
    \underline{Commentaires}~: Super, parfait 9/9
    \vspace{.5em}}\\
          
    \hline

    \multicolumn{2}{|>{\hsize = 2\hsize}X|}{
    \underline{Exercice}~: Téléobjectif
    } & \multicolumn{2}{>{\hsize = 2\hsize}X|}{\vspace{.5em}
    \underline{Commentaires}~: Très bonne gestion, pas coincée par la réflexion.
    \vspace{.5em}}\\
        
    \hline

    \multicolumn{2}{|>{\hsize = 2\hsize}X|}{
    \underline{Tableau}~: Grand et propre, mais \textbf{très} chaotique. À
    améliorer prestement.
    } & \multicolumn{2}{>{\hsize = 2\hsize}X|}{\vspace{.5em}
    \underline{Oral}~: Bien
    \vspace{.5em}}\\

    \hline
\end{tabularx}

\end{document}
