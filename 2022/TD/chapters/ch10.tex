\documentclass[a4paper, 12pt, final, garamond]{book}
\usepackage{cours-preambule}

\raggedbottom

\makeatletter
\renewcommand{\@chapapp}{Chimie -- chapitre}
\makeatother

\begin{document}
\setcounter{chapter}{2}

\chapter{TD~: Cin\'etique chimique}

\section{Pour s'échauffer}
\subsection{Énergie d'activation et constante de vitesse}
\begin{enumerate}
    \item Calculer l'énergie d'activation de la conversion du cyclopropane en
        propène à partir des données suivantes~:
        \begin{center}
            \begin{tabular}{lcccc}
                \toprule 
                $T(\si{K})$  &
                750          & 800          & 850          & 900\\
                \midrule
                $k(\si{s^{-1}})$ &
                \num{1.8e-4} & \num{2.7e-3} & \num{3.0e-2} & \num{0.26}\\
                \bottomrule
            \end{tabular}
        \end{center}
    \item Quelle est la valeur de la constante de vitesse à
        \SI{500}{\degreeCelsius}~?
\end{enumerate}

\subsection{Utilisation du temps de demi-réaction}
Soit la réaction
\[{\rm A} \rightarrow {\rm B} + {\rm C}\]
Déterminer son ordre sachant que lorsqu'on multiplie par 10 la concentration
initiale de A, on divise le temps de demi-réaction par 10.

\section{Utilisation de la méthode intégrale}
À température élevée et en phase gazeuse, le buta-1,3-diène se dimérise en
4-vinylcyclohexène suivant la réaction totale d'équation
\[\ce{2C4H6\gaz{} = C8H12\gaz{}}\]
Afin d'étudier cette réaction, une certaine quantité de buta-1,3-diène est
introduite dans un récipient de volume $V$ constant, maintenu à température
constante $T = \SI{326}{K}$. On mesure alors la pression partielle en butadiène
$p_B$ dans le récipient en fonction du temps~:
\begin{center}
    \begin{tabular}{lcccccccccccc}
        \toprule 
        $t(\si{min})$ &
        0 & \num{3.25} & \num{8.02} & \num{12.18} & \num{17.3} & \num{24.55} &
        \num{33.0} & \num{43.0} & \num{55.08} & \num{68.05} & \num{90.1} &
        \num{119}\\
        \midrule
        $p_B(\si{bar})$ &
        \num{0.843} & \num{0.807} & \num{0.756} & \num{0.715} & \num{0.670} &
        \num{0.615} & \num{0.565} & \num{0.520} & \num{0.465} & \num{0.423} &
        \num{0.366} & \num{0.311}\\
        \bottomrule
    \end{tabular}
\end{center}
\begin{enumerate}
    \item Montrer, en utilisant la loi des gaz parfaits, que la connaissance de
        la pression initiale $p_B$ et de la température $T$ suffit pour calculer
        la concentration initiale $c_B$ en buta-1,3-diène.
    \item Montrer que les résultats sont compatibles avec une cinétique d'ordre
        2.
    \item Déterminer la valeur de la constante de vitesse à cette température.
    \item Déterminer le temps de demi-réaction du système précédent.
    \item On admet souvent qu'une réaction est pratiquement terminée lorsque au
        moins 99\% du réactif limitant a été consommé. Déterminer la durée
        d'évolution du système précédent~; exprimer cette durée en fonction du
        temps de demi-réaction.
\end{enumerate}

\section{Utilisation de la méthode différentielle}
La réaction étudiée est l'oxydation des ions iodure par les ions ferriques
Fe(III). Les couples d'oxydoréduction mis en jeu sont les couples
$\ce{I2}/\ce{I-}$ et $\ce{Fe\plus{3}}/\ce{Fe\plus{2}}$, toutes les espèces étant
dissoutes dans l'eau.

\begin{enumerate}
    \item Écrire l'équation-bilan de l'oxydation des ions iodure par les ions
        fer (III), en affectant les espèces du fer du nombre stœchiométrique 1.
        Si la concentration d'ions iodure passe de $c_0$ à $c_0 - x$ entre 0 et
        $t$, comment définit-on par rapport à $x$ la vitesse volumique de la
        réaction~?
    \item On suppose une cinétique avec ordre, de constante de vitesse $k$~; on
        note $a$ l'ordre partiel par rapport aux ions fer (III) et $b$ l'ordre
        partiel par rapport aux ions iodure. Comment s'écrit la vitesse $v$~?
        Quelle est alors l'unité usuelle de $k$ (au besoin en fonction de $a$ et
        de $b$)~?
    \item À la date $t$ après le mélange d'une solution d'iodure de potassium
        avec une solution ferrique, on prélève à la pipette \SI{5}{mL} de
        solution et on dilue 10 fois avant de procéder à un dosage de la
        quantité d'iode formée. Justifier l'intérêt cinétique de cette dilution.
    \item Les résultats d'une série de mesures sont présentés ci-dessous, $x$ se
        rapportant à la quantité d'ions iodure qui ont été oxydés dans le milieu
        réactionnel à la date du prélèvement.
        \begin{center}
            \begin{tabular}{lccccc}
                \toprule 
                $t(\si{s})$ &
                60 & 120 & 180 & 240 & 300\\
                \midrule
                $x(\si{\micro mol.L^{-1}})$ &
                13 & 25 & 36 & 46 & 55\\
                \bottomrule
            \end{tabular}
        \end{center}
        Que représente la grandeur $x(t)/t$~? Pourquoi diminue-t-elle en cours
        de réaction~? Représenter graphiquement cette grandeur en fonction de
        $t$ à partir du tableau ci-dessus, avec en abscisse $t \in
        \SIrange{0}{300}{s}$~; en déduire une estimation de la valeur initiale
        $\left.\dv{x}{t}\right|_0$.
    \item Grâce à la méthode précédente, on détermine les valeurs initiales de
        $\dv{x}{t}$ pour différentes concentrations initiales des deux réactifs.
        Quelques résultats sont présentés ci-dessous~:
        \begin{center}
            \begin{tabular}{lccccccc}
                \toprule 
                $c_0 = [\ce{I-}]_0$ &
                $(\si{\micro mol.L^{-1}})$ &
                2 & 2 & 2 & 6 & 6 & 8\\
                \midrule
                $[\ce{Fe\plus{3}}]_0$ &
                $(\si{\micro mol.L^{-1}})$ &
                2 & 4 & 8 & 2 & 4 & 8\\
                \midrule
                $\DS \left. \dv{x}{t} \right|_{0}$ &
                $(\si{\micro mol.L^{-1}.s^{-1}})$ &
                \num{5.7} & \num{11.1} & \num{22.5} & 52 & 99 & 354\\
                \bottomrule
            \end{tabular}
        \end{center}
        En déduire les valeurs de $a$ et $b$, supposées entières.
    \item Déterminer la constante de vitesse $k$ définie à la question 2)~; on
        précisera la méthode suivie pour utiliser au mieux les données.
    \item Dans l'hypothèse d'un état initial ne contenant que les deux réactifs
        à la même concentration $c_0$, établier la relation littérale donnant
        $x(t)$ sous la forme~:
        \begin{center}
            «~expression en $(x,c_0)$ = expression en $(k,t)$~»
        \end{center}
        En déduire la dépendance entre le temps de demi-réaction $\tau$ et la
        concentration $c_0$.
\end{enumerate}

\section{Étude d'un mélange stœchiométrique}
On étudie à \SI{25}{\degreeCelsius} l'action d'une solution de soude diluée sur
le bromoéthane~; la réaction totale a pour équation~:
\[\ce{CH3CH2Br + HO\moin{} \leftrightharpoons CH3CH2OH + Br\moin{}}\]
On utilise des mélanges stœchiométriques en bromoéthane et en ion hydroxyde.
Soit $c_0$ la concentration initiale commune des deux réactifs. Le tableau
ci-dessous donne les temps de demi-réaction pour différentes valeurs de $c_0$.
\begin{center}
    \begin{tabular}{lccccc}
        \toprule 
        $c_0 (\si{mmol.L^{-1}})$ &
        10 & 25 & 50 & 75 & 100\\
        \midrule
        $\tau_{1/2} (\si{min})$ &
        1100 & 445 & 220 & 150 & 110\\
        \bottomrule
    \end{tabular}
\end{center}
\begin{enumerate}
    \item Démontrer que ces données sont compatibles avec une réaction d'ordre
        partiel 1 par rapport à chacun des réactifs.
    \item Déterminer la constante de vitesse de la réaction.
\end{enumerate}

\section{Méthode des vitesses initiales}
Le chlorure d'hydrogène (B) réagit sur le cyclohexène (A) avec formation de
chlorocyclohexane (C), selon la réaction~:
\[\ce{C6H10 + HCl \longrightarrow C6H11Cl}
    \quad
    \text{schématisée par}
    \quad
    {\rm A} + {\rm B} \longrightarrow {\rm C}
\]
On réalise une série d'expériences à \SI{25}{\degreeCelsius}, où l'on mesure la
vitesse initiale $v_0$ de la réaction en fonction des concentrations molaires
initiales $[{\rm A}]_0$ en cyclohexène et $[{\rm B}]_0$ en chlorure d'hydrogène
dans le milieu réactionnel. Le volume du mélange est constant et égal à
\SI{1}{L}. Les résultats sont rassemblés dans le tableau ci-dessous~:
\begin{center}
    \begin{tabular}{lcccc}
        \toprule 
        Expérience &
        1 & 2 & 3 & 4\\
        \midrule
        $[{\rm A}]_0$ ($\si{mol.L^{-1}}$) &
        \num{0.470} & \num{0.470} & \num{0.470} & \num{0.313}\\
        $[{\rm B}]_0$ (\si{mol.L^{-1}}) &
        \num{0.235} & \num{0.328} & \num{0.448} & \num{0.448}\\
        $v_0$ (\SI{e-9}{mol.s^{-1}}) &
        \num{15.7} & \num{30.6} & \num{57.1} & \num{38.0}\\
        \bottomrule
    \end{tabular}
\end{center}
\begin{enumerate}
    \item On désigne par $p$ et $q$ les ordres partiels initiaux de la réaction
        par rapport au cyclohexène (A) et au chlorure d'hydrogène (B). Exprimer
        la loi de vitesse initiale de cette réaction en fonction de $p$ et $q$.
    \item Déterminer $p$.
    \item Détermine $q$~; en déduire l'ordre global de la réaction.
    \item Calculer la constante cinétique de la réaction.
    \item Dans le cas d'un mélange stœchiométrique en A et B, déterminer la loi
        de vitesse de la réaction en fonction de [A]. En déduire l'équation
        différentielle satisfaite par [A]$(t)$.
\end{enumerate}

\section{Intérêt de la dégénérescence de l'ordre}
On considère la réaction suivante~:
\[\ce{2Hg\plus{2} + 2Fe\plus{2} \longrightarrow Hg2\plus{2} + 2Fe\plus{3}}\]
On suit deux expériences à \SI{80}{\degreeCelsius} par spectrophotométrie. On
définit $\alpha = \DS \frac{[\ce{Hg\plus{2}}]}{[\ce{Hg\plus{2}}]_0}$.
\begin{description}
    \item[Expérience 1] : $[\ce{Fe\plus{2}}]_0 = \SI{0.100}{mol.L^{-1}}$ et
        $[\ce{Hg\plus{2}}]_0 = \SI{0.100}{mol.L^{-1}}$
        \begin{center}
            \begin{tabular}{lccccc}
                \toprule 
                $t (\SI{e5}{s})$ &
                \num{0.0} & \num{1.0} & \num{2.0} & \num{3.0} & $\infty$\\
                \midrule
                $\alpha(t)$ &
                \num{1.000} & \num{0.500} & \num{0.333} & \num{0.250} &
                \num{0.000}\\
                \bottomrule
            \end{tabular}
        \end{center}
    \item[Expérience 2] : $[\ce{Fe\plus{2}}]_0 = \SI{0.100}{mol.L^{-1}}$ et
        $[\ce{Hg\plus{2}}]_0 = \SI{0.001}{mol.L^{-1}}$
        \begin{center}
            \begin{tabular}{lcccccc}
                \toprule 
                $t (\SI{e5}{s})$ &
                \num{0.0} & \num{0.5} & \num{1.0} & \num{1.5} & \num{2.0} &
                $\infty$\\
                \midrule
                $\alpha(t)$ &
                \num{1.000} & \num{0.585} & \num{0.348} & \num{0.205} &
                \num{0.122} & \num{0.000}\\
                \bottomrule
            \end{tabular}
        \end{center}
\end{description}
\begin{enumerate}
    \item On considère que la réaction est d'ordre partiel $p$ par rapport à
        $\ce{Fe\plus{2}}$ et $q$ par rapport à $\ce{Hg\plus{2}}$. Écrire
        l'expression de la vitesse de réaction.
    \item Déterminer l'ordre global de la réaction à l'aide de l'expérience 1.
    \item Déterminer $q$ à l'aide de l'expérience 2. En déduire $p$.
    \item Déterminer la constante de vitesse de la réaction.
\end{enumerate}

\end{document}
