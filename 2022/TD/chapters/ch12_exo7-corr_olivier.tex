\section{Déphasage, pulsation et impédance}

Pour exprimer simplement $i$, il nous fait une seule maille avec une seule
impédance équivalente $\ul{Z}_{\eq}$~: de cette manière, la loi des mailles nous
donnera $\ul{E} = \ul{Z}_{\eq}\ul{I}$ et on pourra facilement déterminer le
déphasage entre $i$ et $e$. \bigbreak
On calcule l'impédance équivalente de l'association en série de $R_2$ et $L$~:
\[\ul{Z}_{\eq,1} = R_2 + \jj L\w\]
Cette association est en parallèle avec $C$~:
\begin{gather*}
    \ul{Z}_{\eq, 2}
        = \frac{\ul{Z}_C\times\ul{Z}_{\eq, 1}}{\ul{Z}_C + \ul{Z}_{\eq,1}}
        = \frac{\dfrac{1}{\jj C\w}(R_2 + \jj L\w)}{\dfrac{1}{\jj C\w} + R_2 +
            \jj L\w}\\
    \Leftrightarrow
    \ul{Z}_{\eq,2} = \frac{R_2 + \jj L\w}{1+\jj R_2C\w - LC\w^2}
\end{gather*}
On a donc comme prévu avec la loi des mailles~:
\[\boxed{\ul{I} = \frac{\ul{E}}{R_1 + \ul{Z}_{\eq,2}}}\]
L'intensité est en phase avec la tension si $\arg(R_1 + \ul{Z}_{\eq,2}) = 0$,
c'est-à-dire si
\begin{align*}
    \arg\left(R_1 + \frac{R_2 + \jj L\w}{1 + \jj R_2C\w - LC\w^2}\right)
        &= 0\\
    \Leftrightarrow
    \arg \left( \frac{R_1 + \jj R_1R_2C\w - LCR_1\w^2 + R_2 + \jj L\w}
        {1 + \jj R_2 C\w - LC\w^2} \right)
        &= 0\\
    \Leftrightarrow
    \arg \left( (R_1 + R_2 - LCR_1\w^2) + \jj(R_1R_2C\w + L\w) \right)
        & = \arg \left( (1-LC\w^2) + \jj R_2C\w \right) = 0\\
    \Leftrightarrow
    \frac{R_1R_2C\w + L\w}{R_1 + R_2 - LCR_1\w^2}
        &= \frac{R_2C\w}{1-LC\w}\\
    \Leftrightarrow
    \frac{R_1 + \dfrac{L\cancel{\w}}{R_2C\cancel{\w}}}{R_1 + R_2 - LCR_1\w^2}
        &= \frac{1}{1-LC\w^2}\\
    \Leftrightarrow
    \left( R_1 + \frac{L}{R_2C} \right) \left( 1-LC\w^2 \right)
        &= R_1 + R_2 - LCR_1\w^2\\
    \Leftrightarrow
    \cancel{R_1} - \bcancel{LCR_1\w^2} + \frac{L}{R_2C} - \frac{L^2\w^2}{R_2}
        &= \cancel{R_1} + R_2 - \bcancel{LCR_1\w^2}\\
    \Leftrightarrow
    L
        &= R_2{}^2 C + L^2C\w^2\\
    \Leftrightarrow
    \w^2
        &= \frac{1}{LC} - \frac{R_2{}^2}{L^2}\\
    \Leftrightarrow
    \Aboxed{\w^2
        &= \frac{1}{LC} \left( 1 - \frac{R_2{}^2C}{L} \right)}
\end{align*}
