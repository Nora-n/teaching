% Options for packages loaded elsewhere
\PassOptionsToPackage{unicode}{hyperref}
\PassOptionsToPackage{hyphens}{url}
\PassOptionsToPackage{dvipsnames,svgnames,x11names}{xcolor}
%
\documentclass[
  10pt,
  a4paper,
  DIV=18]{scrartcl}
\usepackage{amsmath,amssymb}
\usepackage{lmodern}
\usepackage{iftex}
\ifPDFTeX
  \usepackage[T1]{fontenc}
  \usepackage[utf8]{inputenc}
  \usepackage{textcomp} % provide euro and other symbols
\else % if luatex or xetex
  \usepackage{unicode-math}
  \defaultfontfeatures{Scale=MatchLowercase}
  \defaultfontfeatures[\rmfamily]{Ligatures=TeX,Scale=1}
  \setmonofont[Scale = MatchLowercase]{DejaVu Sans Mono}
  \setmathfont[]{latinmodern-math.otf}
\fi
% Use upquote if available, for straight quotes in verbatim environments
\IfFileExists{upquote.sty}{\usepackage{upquote}}{}
\IfFileExists{microtype.sty}{% use microtype if available
  \usepackage[]{microtype}
  \UseMicrotypeSet[protrusion]{basicmath} % disable protrusion for tt fonts
}{}
\makeatletter
\@ifundefined{KOMAClassName}{% if non-KOMA class
  \IfFileExists{parskip.sty}{%
    \usepackage{parskip}
  }{% else
    \setlength{\parindent}{0pt}
    \setlength{\parskip}{6pt plus 2pt minus 1pt}}
}{% if KOMA class
  \KOMAoptions{parskip=half}}
\makeatother
\usepackage{xcolor}
\usepackage{graphicx}
\makeatletter
\def\maxwidth{\ifdim\Gin@nat@width>\linewidth\linewidth\else\Gin@nat@width\fi}
\def\maxheight{\ifdim\Gin@nat@height>\textheight\textheight\else\Gin@nat@height\fi}
\makeatother
% Scale images if necessary, so that they will not overflow the page
% margins by default, and it is still possible to overwrite the defaults
% using explicit options in \includegraphics[width, height, ...]{}
\setkeys{Gin}{width=\maxwidth,height=\maxheight,keepaspectratio}
% Set default figure placement to htbp
\makeatletter
\def\fps@figure{htbp}
\makeatother
\setlength{\emergencystretch}{3em} % prevent overfull lines
\providecommand{\tightlist}{%
  \setlength{\itemsep}{0pt}\setlength{\parskip}{0pt}}
\setcounter{secnumdepth}{5}
\ifLuaTeX
\usepackage[bidi=basic]{babel}
\else
\usepackage[bidi=default]{babel}
\fi
\babelprovide[main,import]{french}
% get rid of language-specific shorthands (see #6817):
\let\LanguageShortHands\languageshorthands
\def\languageshorthands#1{}
\ifLuaTeX
  \usepackage{selnolig}  % disable illegal ligatures
\fi
\IfFileExists{bookmark.sty}{\usepackage{bookmark}}{\usepackage{hyperref}}
\IfFileExists{xurl.sty}{\usepackage{xurl}}{} % add URL line breaks if available
\urlstyle{same} % disable monospaced font for URLs
\hypersetup{
  pdftitle={TD C7 - Oscillateurs linéaires en régime sinusoïdal forcé},
  pdflang={fr-FR},
  colorlinks=true,
  linkcolor={Maroon},
  filecolor={Maroon},
  citecolor={Blue},
  urlcolor={Blue},
  pdfcreator={LaTeX via pandoc}}

\title{TD C7 - Oscillateurs linéaires en régime sinusoïdal forcé}
\author{false}
\date{2022-2023}

\begin{document}
\maketitle

\hypertarget{capacituxe9s-exigibles}{%
\subsection*{Capacités exigibles}\label{capacituxe9s-exigibles}}
\addcontentsline{toc}{subsection}{Capacités exigibles}

\begin{itemize}
\item
  \emph{Manipuler des signaux complexes en régime sinusoïdal forcé:}
  tous les exercices!
\item
  \emph{Exploiter des courbes d'amplitude et de phase en fonction de la
  fréquence d'excitation:}
  {[}@sec:wien;@sec:haut-parleur;@sec:bouchon{]}.
\item
  \emph{Déterminer l'impédance équivalente d'un circuit en régime
  sinusoïdal forcé:} {[}@sec:bouchon;@sec:antenne{]}.
\item
  \emph{Relier l'acuité d'une résonance au facteur de qualité:}
  {[}@sec:bouchon;@sec:antenne{]}.
\end{itemize}

\hypertarget{sec:notation}{%
\section{Notation complexe}\label{sec:notation}}

Écrire, sous forme complexe, les équations différentielles suivantes :
\[\tau \derive{u}{t} + u(t) = E_0\sin\omega t \qquad \qquad \ddot{x} +  2\lambda \dot x + \omega_0^2 x(t) =F_0\cos\omega t\]

\hypertarget{sec:wien}{%
\section{Filtre de Wien}\label{sec:wien}}

\begin{multicols}{2}

On considère le circuit ci-contre avec \(e(t) = E_m \cos(\omega t)\). On
note \(u(t) = U_m \cos(\omega t + \varphi)\) et on pose
\(H_m = U_m / E_m\) .

\begin{enumerate}
\def\labelenumi{\arabic{enumi}.}
\tightlist
\item
  Déterminer les valeurs limites de \(u(t)\) à basse et haute
  fréquences.
\end{enumerate}

Les courbes représentatives de \(H_m (\omega)\) et \(\varphi(\omega)\)
sont fournies par les figures ci-dessous.

\begin{center}

\includegraphics[width=7cm,height=\textheight]{images-C7/exo-wien-1.png}

\end{center}

\end{multicols}

\begin{center}

\includegraphics[width=9.5cm,height=\textheight]{images-C7/exo-wien-2.png}
\includegraphics[width=9.5cm,height=\textheight]{images-C7/exo-wien-3.png}

\end{center}

\begin{enumerate}
\def\labelenumi{\arabic{enumi}.}
\setcounter{enumi}{1}
\tightlist
\item
  Observe-t-on un phénomène de résonance en tension ? Justifier.
\item
  Déterminer graphiquement la pulsation de résonance, les pulsations de
  coupure et la bande passante du filtre.
\item
  Après avoir associé certaines impédances entre elles, établir
  l'expression de \(\underline{H} = \underline{u} / \underline{e}\). La
  mettre sous la forme :
  \[\underline{H} = \dfrac{H_0}{1 + jQ\left(x - \dfrac{1}{x}\right)} \qquad \text{avec } x = \dfrac{\omega}{\omega_0}\]
  avec \(H_0,\omega_0\) et \(Q\) des constantes à exprimer en fonction
  (éventuellement) de \(R\) et \(C\).
\item
  Déterminer graphiquement la valeur du produit \(RC\).
\end{enumerate}

\hypertarget{sec:haut-parleur}{%
\section{Modélisation d'un haut-parleur}\label{sec:haut-parleur}}

\begin{multicols}{2}

On modélise la partie mécanique d'un haut-parleur comme une masse \(m\),
se déplaçant horizontalement le long d'un axe \((Ox)\). Cette masse est
reliée à un ressort de longueur à vide \(\ell_0\) et de raideur \(k\) et
subit une force de frottement fluide : \(\vec{f} = -\alpha \vec{v}\).
Elle est par ailleurs soumise à une force \(\vec{F}(t)\), imposée par le
courant \(i(t)\) entrant dans le haut-parleur, qui vaut :
\(\vec{F}(t) = K i(t) \vec{u}_x\) où \(K\) est une constante. On
travaille dans le référentiel du laboratoire
\((O, \vec{u}_x , \vec{u}_y)\). On suppose que le courant est de la
forme \(i(t) = I_m \cos(\omega t)\).

\begin{center}

\includegraphics[width=9cm,height=\textheight]{images-C7/exo-HP-1.png}

\end{center}

\end{multicols}

\emph{Données:} \(m = 10\ \mathrm{g}\), \(K = 200\ \mathrm{N.A^{-1}}\)
et \(I_m = 1,0\ \mathrm{A}\).

\begin{enumerate}
\def\labelenumi{\arabic{enumi}.}
\tightlist
\item
  Écrire l'équation différentielle vérifiée par \(x(t)\), la position de
  la masse \(m\).
\item
  La mettre sous forme canonique et identifier les expressions de la
  pulsation propre \(\omega_0\) et du facteur de qualité \(Q\).
\item
  Justifier qu'en régime permanent: \(x(t) = X_m \cos(\omega t + \phi)\)
\item
  On pose \(\underline{x}(t) = \underline{X}e^{i\omega t}\). Déterminer
  l'expression de l'amplitude complexe \(\underline{X}\).
\item
  Exprimer \(X_m(\omega)\). Existe-t-il toujours une résonance?
\end{enumerate}

On a tracé ci-dessous les courbes de \(X_m (\omega)\) et de
\(\phi(\omega)\). L'axe des abscisses est en échelle logarithmique.

\begin{center}

\includegraphics[width=17cm,height=\textheight]{images-C7/exo-HP-2.png}

\end{center}

\begin{enumerate}
\def\labelenumi{\arabic{enumi}.}
\setcounter{enumi}{5}
\tightlist
\item
  Pour quelle pulsation le déplacement est-il en quadrature de phase
  avec la force excitatrice ? Déterminer alors graphiquement la
  pulsation propre \(\omega_0\).
\end{enumerate}

\hypertarget{sec:bouchon}{%
\section{Résonance d'un circuit bouchon}\label{sec:bouchon}}

\begin{multicols}{2}

On considère le circuit \(RLC\) représenté ci-contre, composé d'un
résistor, de résistance \(R\), d'une bobine idéale d'inductance \(L\),
d'un condensateur idéal, de capacité \(C\), alimenté par une source
idéale de tension, de f.e.m. \(e(t)=E_0\cos(\omega t)\). On se place en
régime sinusoïdal forcé.

\begin{center}
\begin{circuitikz}
\draw (0,0) to[V=$e(t)$] (0,2) to[R=$R$] (2,2) to[C=$C$] (2,0) to[short] (0,0);
\draw (2,0) to[short] (4,0) to[L,l=$L$,v=$u(t)$] (4,2) to[short] (2,2);
\end{circuitikz}
\end{center}

\end{multicols}

\begin{enumerate}
\def\labelenumi{\arabic{enumi}.}
\item
  Exprimer l'amplitude complexe \(\underline{U}\) de \(u(t)\) en
  fonction de \(E_0,R,L,C\) et \(\omega\).
\item
  Établir qu'il existe un phénomène de résonance pour la tension
  \(u(t)\). Préciser la pulsation \(\omega_0\) à laquelle ce phénomène
  se produit et la valeur de l'amplitude réelle de \(u(t)\) à cette
  pulsation.
\item
  Mettre l'amplitude réelle \(U\) de \(u(t)\) sous la forme:
  \[U = \dfrac{E_0}{\sqrt{1 + Q^2\left(\dfrac{\omega}{\omega_0} - \dfrac{\omega_0}{\omega}\right)^2}}\]
  avec \(Q\) un facteur sans dimension à exprimer en fonction de \(R,L\)
  et \(C\).
\item
  Exprimer la bande passante \(\Delta\omega\) de cette résonance en
  fonction de \(Q\) et \(\omega_0\).
\item
  En déduire les valeurs numériques de \(C\) et \(E_0\) à l'aide du
  graphe ci-dessous représentant l'amplitude réelle de \(u(t)\) en
  fonction de la fréquence \(f=\omega/2\pi\), sachant que \(L=1\) mH et
  \(R=1\) k\(\Omega\).
\end{enumerate}

\begin{center}

\includegraphics{images-C7/exo-bouchon-figure.pdf}

\end{center}

\hypertarget{sec:2-ressorts}{%
\section{Système à deux ressorts}\label{sec:2-ressorts}}

\begin{multicols}{2}

Un point matériel \(M\), de masse \(m\), peut se déplacer sur une tige
\emph{horizontale} parallèle à l'axe \(Ox\) au sein d'un fluide visqueux
qui exerce sur lui la force de frottement \(\vec{f} = - h\vec{v}\) avec
\(\vec{v}\) le vecteur vitesse de \(M\) dans le référentiel galiléen
\(\mathcal{R}\) du laboratoire. Les frottements entre \(M\) et l'axe
horizontal sont négligeables. On repère \(M\) par son abscisse \(x(t)\).

\begin{center}

\includegraphics[width=5.5cm,height=\textheight]{images-C7/exo-2ressorts.pdf}

\end{center}

\end{multicols}

\(M\) est relié à deux parois verticales par deux ressorts de raideurs
\(k_1\) et \(k_2\), de longueurs à vide \(l_{10}\) et \(l_{20}\). Celle
de droite est immobile en \(x = L\), celle de gauche, d'abscisse
\(x_0 (t)\), est animée d'un mouvement d'équation horaire
\(x_0 (t) = X_{0m} \cos(\omega t)\). On supposera que
\(L = l_{10} + l_{20}\).

\begin{enumerate}
\def\labelenumi{\arabic{enumi}.}
\tightlist
\item
  Identifier les différentes forces s'exerçant sur \(M\).
\item
  Déterminer la position d'équilibre \(x_\text{eq}\) de \(M\) lorsque la
  paroi de gauche est immobile en \(x = 0\).
\item
  On introduit \(X = x - x_\text{eq}\). Établir l'équation
  différentielle vérifiée par \(X\) lorsque la paroi bouge. Pour étudier
  le régime sinusoïdal forcé, on introduit les grandeurs complexes
  \(\underline{x}_0(t) = X_{0m} \exp(j\omega t)\),
  \(X(t) = X_m \exp(j(\omega t + \varphi))\) et
  \(v(t) = V_m \exp(j(\omega t + \phi))\) associées à \(x_0(t)\),
  \(X(t)\) et \(v(t) = \dot X(t)\).
\item
  Définir les amplitudes complexes \(\underline{X}_0\) ,
  \(\underline{X}\) et \(\underline{V}\) de \(x_0(t)\), \(X(t)\) et
  \(v(t)\).
\item
  En exprimant \(\omega_0\), \(Q\) et \(\alpha\) en fonction des données
  du problème, établir la relation :
  \[\underline{V} = \dfrac{\alpha}{1 + jQ\left(\dfrac{\omega}{\omega_0} - \dfrac{\omega_0}{\omega}\right)} \underline{X}_0\]
\item
  Mettre en évidence l'existence d'une résonance de vitesse.
\end{enumerate}

\hypertarget{sec:antenne}{%
\section{Résonance d'intensité dans un circuit RLC
parallèle}\label{sec:antenne}}

\begin{multicols}{2}

L'antenne d'un émetteur radio peut être modélisée par un circuit
électrique équivalent composé de l'association en parallèle d'une
résistance \(R\), d'une bobine d'inductance \(L\) et d'un condensateur
de capacité \(C\).

L'antenne est alimentée par une source idéale de courant dont
l'intensité caractéristique varie de manière sinusoïdale dans le temps:
\(i(t) = I_0 \cos(\omega t)\).

\begin{center}
\begin{circuitikz}
\draw (0,0) to[I=$i(t)$] (0,2) to[short] (4.5,2);
\draw (1.5,2) to[C=$C$,i=$i_C$] (1.5,0);
\draw (3,2) to[L=$L$,i=$i_L$] (3,0);
\draw (4.5,2) to[R,l_=$R$,i=$i_R$,v^<=$u$] (4.5,0);
\draw (4.5,0) to[short] (0,0);
\end{circuitikz}
\end{center}

\end{multicols}

On s'intéresse à la manière dont l'amplitude de la tension \(u(t)\) aux
bornes de l'antenne, qui correspond au signal envoyé, dépend de
\(\omega\).

\begin{enumerate}
\def\labelenumi{\arabic{enumi}.}
\tightlist
\item
  Déterminer l'impédance complexe de l'association des dipôles \(R,L\)
  et \(C\).
\item
  En déduire l'amplitude complexe \(\underline{U}\) de la tension \(u\)
  en fonction de \(\omega,I_0,R,L\) et \(C\).
\item
  Pour quelle pulsation l'amplitude réelle \(U\) de \(u\) prend-elle sa
  valeur maximale notée \(U_\text{max}\)? Conclure sur la fréquence à
  utiliser.
\item
  Représenter le graphe donnant \(U\) en fonction de la pulsation
  réduite \(x=\omega/\omega_0\) avec \(\omega_0=1/\sqrt{LC}\).
\item
  Exprimer la largeur de la bande passante \(\Delta\omega\).
\item
  On se place dans le cas \(R = 37\ \Omega\),
  \(L = 1,2 \times 10^{-8}\ \mathrm{H}\) et
  \(C = 2,3 \times 10^{-10}\ \mathrm{F}\). Calculer la valeur de
  l'acuité \(A_c = \omega_0/\Delta\omega\) de la résonance. Interpréter
  sa dépendance en \(R\).
\end{enumerate}

\hypertarget{Sec:conditions}{%
\section{Condition de résonance}\label{Sec:conditions}}

\begin{multicols}{2}

Le circuit ci-contre est alimenté par une source de tension sinusoïdale
de f.é.m. \(e(t) = E_0 \cos( \omega t)\). On s'intéresse à la tension
\(u(t)\) aux bornes du résistor et de la capacité montés en parallèle.

On pose : \(\omega_0 =\dfrac{1}{\sqrt{LC}}\),
\(\xi = \dfrac{R}{2}\sqrt{\dfrac{C}{L}}\) et
\(x = \dfrac{\omega}{\omega_0}\).

\begin{center}

\includegraphics[width=4.5cm,height=\textheight]{images-C7/exo-conditions.pdf}

\end{center}

\end{multicols}

\begin{enumerate}
\def\labelenumi{\arabic{enumi}.}
\tightlist
\item
  Établir l'expression du signal complexe \(\underline{u}\) associé à
  \(u(t)\) en régime sinusoïdal forcé, en fonction de \(E_0\), \(x\) et
  \(\xi\).
\item
  Étudier l'existence éventuelle d'une résonance pour la tension
  \(u(t)\).
\end{enumerate}

\end{document}
