\documentclass[a4paper, 12pt, final, garamond]{book}
\usepackage{cours-preambule}

\raggedbottom

\makeatletter
\renewcommand{\@chapapp}{\'Electrocin\'etique -- chapitre}
\makeatother

\begin{document}
\setcounter{chapter}{6}

\chapter{Correction du TD}

\section{Filtrage et spectres}
\begin{enumerate}
    \item Sur la figure deux, les basses fréquences sont globalement conservées,
        et les fréquences à partir de \SI{3}{kHz} sont fortement atténuées voire
        coupées~: \textbf{c'est un passe-bas}.
    \item Sur la figure trois, seules les fréquences entre 3 et \SI{4}{kHz} sont
        gardées, les fréquences supérieures ou inférieures sont coupées~:
        \textbf{c'est un passe-bande}.
    \item Sur la figure quatre, on ne distingue pas de relation simple vue en
        cours~; on remarque de plus que de nouvelles fréquences apparaissent, ce
        qui n'est pas le cas dans le filtrage linéaire~: c'est un \textbf{filtre
        non-linéaire}.
\end{enumerate}

\section{Filtre avec une bobine}
\begin{enumerate}
    \item À basses fréquences, $\abs{\ul{Z}_L} \xrightarrow[\w\rightarrow0]{} 0$, donc
        $H(\w) \xrightarrow[\w\rightarrow0]{} 0$. À hautes fréquences,
        $\abs{\ul{Z}_L} \xrightarrow[\w\rightarrow\infty]{} \infty$, donc
        $H(\w) \xrightarrow[\w\rightarrow\infty]{} 1$~: \textbf{c'est un
        passe-haut}.
    \item On fait un pont diviseur de tension~:
        \begin{gather*}
            \ul{u_s} = \frac{\jlw}{R + \jlw}\ul{u_e}
            \Leftrightarrow
            \frac{\ul{u_s}}{\ul{u_e}} = \ul{H} =
            \cancel{\frac{R}{R}}\times\frac{\jj\w\frac{L}{R}}{1 + \jj\w
            \frac{L}{R}}\\
            \Leftrightarrow
            \boxed{\ul{H}(\jj\w) = H_0 \frac{\jj\frac{\w}{\w_c}}{1 +
            \jj\frac{\w}{\w_c}}}
            \qavec
            \boxed{H_0 = 1
            \qet
            \w_c = \frac{R}{L} = \SI{1e5}{rad.s^{-1}}}
        \end{gather*}
    \item $1 + \jj \frac{\w}{\w_c} \underset{\w \ll \w_c}{\sim} 1$, et donc
        \begin{gather*}
            \ul{H}(\jj\w) \underset{\w\ll\w_c}{\sim} \jj \frac{\w}{\w_c}
            \Leftrightarrow
            \boxed{G_{\rm dB} = 20\log \left( \abs{\ul{H}} \right)
            \underset{\w\ll\w_c}{\sim} 20\log x}
        \end{gather*}
        d'où la pente de \SI{20}{dB/décade}.
    \item On trouve le spectre de sortie en multipliant chaque amplitude
        d'entrée par le module de la fonction de transfert pour avoir
        l'amplitude de sortie. \textbf{Attention, pulsation $\neq$ fréquence}.
        On a
        \[\abs{\ul{H}(f)} = \frac{\dfrac{2\pi f}{\w_c}}{\sqrt{1 +
                    \left( \dfrac{2\pi f}{\w_c} \right)^2}}\]
        \begin{itemize}
            \item $H(f_1) \approx \num{6.3e-3}$~: le fondamental est
                complètement atténué, il ne reste que \num{0.6}\% de son
                amplitude initiale~;
            \item $H(f_2) \approx \num{6.3e-2}$~: l'harmonique $f_2$ est
                fortement atténué, il n'en reste que 6\%~;
            \item $H(f_3) \approx \num{0.99}$~: l'harmonique $f_3$ est
                pratiquement entièrement conservé.
        \end{itemize}
\end{enumerate}

\section{Lecture de diagrammes de \textsc{Bode}}

Pour faciliter la rédaction on note $e(t) = e_0 + e_1 (t) + e_{10} (t) + e_{100}
(t)$, et de même pour le signal de sortie $s$. Ainsi, par linéarité, chaque
composante $e_n$ du signal d'entrée donne une composante $s_n$ au signal de
sortie.
\begin{itemize}
    \item Filtre 1~: d'après l'allure du diagramme de \textsc{Bode}, il s'agit d'un
        filtre passe-haut, de fréquence de coupure $f_c$ de l'ordre de
        \SI{10}{kHz}. Reconstruisons le signal de sortie~:
        \begin{itemize}
            \item Le terme constant $e_0$ est complètement coupé par le filtre,
                donc $s_0 = 0$.
            \item L'harmonique de fréquence $f$ est atténuée de \SI{40}{dB} et
                peut donc être négligée dans le signal de sortie (\SI{40}{dB}
                correspond à une division de l'amplitude par 100), soit $s_1 (t)
                \ll$ autres harmoniques de $s(t)$.
            \item L'harmonique de fréquence $10f$ est atténuée de \SI{10}{dB},
                soit \[S = 10^{-10/20} E = 10^{-1/2} E \approx \num{0,3}E\] et
                elle est également déphasée d'environ $+\pi/2$. Ainsi \[s_{10}
                (t) \approx \num{0,3}E \cos(10\wt + \pi/4 + \pi/2)\]
            \item L'harmonique de fréquence $100f$ n'est presque pas atténuée ni
                déphasée, donc $s_{100} (t) \approx e_{100} (t)$. Au final, on
                obtient le signal de sortie $s(t)$ suivant~:
        \end{itemize}
\end{itemize}

\[\boxed{
    s(t) \approx
    \num{0,3}E_0 \cos \left(10\wt + \frac{3\pi}{4}\right) +
    E_0 \cos \left(100\wt - \frac{\pi}{3}\right)
}\]

\begin{itemize}
    \item Filtre 2~: d'après l'allure du diagramme de \textsc{Bode}, il s'agit d'un
        filtre \textbf{passe-haut}, de fréquence de coupure $f_c$ de l'ordre de
        \SI{0.1}{kHz}. De la même manière que pour le fitre 1, on détermine que~:
\end{itemize}

\[\boxed{
    s(t) \approx
    E_0 \cos (\wt) + E_0 \cos \left(10\wt + \frac{\pi}{4}\right) +
    E_0 \cos \left(100\wt - \frac{\pi}{3}\right)
}\]
\begin{itemize}
    \item Filtre 3~: d'après l'allure du diagramme de \textsc{Bode}, il s'agit d'un
        filtre \textbf{coupe-bande}, la bande coupée étant proche de
        \SI{1}{kHz}. Ainsi seule l'harmonique $e_1(t)$ est coupée (soit $s_1 =
        0$). Les autres composantes harmoniques du signal d'entrée, y compris la
        composante continue, sont de fréquences suffisamment différentes de la
        fréquence coupée pour n'être ni atténuée ni déphasée. La signal de
        sortie $s(t)$ s'écrit donc sous la forme~:
\end{itemize}

\[\boxed{
    s(t) =
    E_0 +
    E_0 \cos \left(10\wt - \frac{\pi}{4}\right) +
    E_0 \cos \left(100\wt - \frac{\pi}{3}\right)
}\]
\begin{itemize}
    \item Filtre 4~: d'après l'allure du diagramme de \textsc{Bode}, il s'agit
        d'un filtre \textbf{passe-bas}, de fréquence de coupure $f_c$ de l'ordre
        de \SI{0.1}{kHz}. Le terme constant $e_0$ passe au travers du filtre
        sans être modifié. Les termes suivants sont de fréquence suffisamment
        supérieure à la fréquence de coupure pour que le diagramme de
        \textsc{Bode} puisse être approximé par son asymptote. On peut alors
        déterminer le signal de sortie comme dans le cas du premier filtre, mais
        il y a plus simple~! Comme le filtre est d'ordre 1 (une seule asymptote
        de pente \SI{-20}{dB/décade}, alors il se comporte comme un intégrateur
        pour les signaux de fréquence supérieure à sa fréquence de coupure. En
        déduire le signal de sortie est donc très simple~:
\end{itemize}

\[s(t) =
    E_0 +
    \frac{\w_c}{\w}E_0\sin(\wt) +
    \frac{\w_c}{10\w}\sin \left( 10\wt + \frac{\pi}{4} \right) +
    \frac{\w_c}{100\w} \sin \left( 100\wt - \frac{\pi}{3} \right)
\]
\begin{itemize}
    \item{}
    En écrivant le signal en termes de cosinus, on obtient~:
\end{itemize}
\[\boxed{
    s(t) = 
    E_0 +
    \frac{\w_c}{\w}E_0\cos\left(\wt - \frac{\pi}{2}\right) +
    \frac{\w_c}{10\w}\cos \left( 10\wt - \frac{\pi}{4} \right) +
    \frac{\w_c}{100\w} \cos \left( 100\wt - \frac{5\pi}{6} \right)
}\]

\section{Filtre de \textsc{Wien}}
\begin{enumerate}
    \item Dans la limite très hautes fréquences, les condensateurs sont
        équivalents à des fils, donc $\ul{S} = 0$. Dans la limite très basses
        fréquences, les condensateurs sont cette fois équivalents à des
        interrupteurs ouverts. Aucun courant ne circule dans les résistances, et
        on a donc également $\ul{S} = 0$. Selon toute vraisemblance, c'est donc
        un filtre \textbf{passe-bande}.
    \item Notons $\ul{Z}$ l'impédance et $\ul{Y}$ l'admittance de l'association
        RC parallèle. En utilisant cette impédance, on reconnaît un pont
        diviseur de tension~:
        \begin{gather*}
            \ul{H} = \frac{\ul{S}}{\ul{E}} = \frac{\ul{Z}}{R + \dfrac{1}{\jj
            C\w} + \ul{Z}}
            \Leftrightarrow
            \ul{H}
                = \frac{1}{1 + \left(R + \dfrac{1}{\jj C\w}\right)\ul{Y}}
                = \frac{1}{1 + \left(R + \dfrac{1}{\jj C\w}\right)\left(
                    \dfrac{1}{R} + \jj C\w\right)}\\
            \Leftrightarrow
            \boxed{\ul{H} = \frac{1}{3+\jj\left(RC\w - \dfrac{1}{RC\w}\right)}}
        \end{gather*}
    \item En factorisant par 3 et en utilisant les notations introduites dans
        l'énoncé, on trouve
        \begin{gather*}
            \ul{H}
                = \frac{1/3}{1 + \dfrac{\jj}{3} \left( x - \dfrac{1}{x} \right)}
            \Leftrightarrow
            \boxed{
            \ul{H} = \frac{H_0}{1 + \jj Q \left( x - \dfrac{1}{x} \right)}}
            \qavec
            \boxed{
            \left\{
                \begin{array}{rcl}
                    H_0 & = & 1/3\\
                    Q & = & 1/3
                \end{array}
            \right.
            }
        \end{gather*}
    \item Le gain en amplitude du filtre est défini par
        \[G = \abs{\ul{H}} = \frac{H_0}{\sqrt{1 + Q^2 \left( x - \dfrac{1}{x}
        \right)^2}}\]
        Il est maximal lorsque le dénominateur est minimal, c'est-à-dire lorsque
        le terme entre parenthèses s'annule. Cela correspond à $x=1$, d'où le
        gain maximal $\mathbf{G_{\max} = 1/3}$. \bigbreak
        Le gain \textbf{en décibels} du filtre est défini par
        \[G_{\rm dB} = 20\log(\abs{\ul{H}})\]
        et on trouve donc $\mathbf{G_{\rm dB, max} = 20\log(1/3) =
        \SI{-9.5}{dB}}$. De plus, en $x = 1$ la fonction de transfert est
        réelle, donc son argument est nul~: à la pulsation $\w_0$, la sortie et
        l'entrée ne sont donc pas déphasées.
    \item Dans la limite très basses fréquences, la fonction de transfert est
        équivalente à
        \[\ul{H}
            \underset{x\rightarrow0}{\sim} \frac{H_0}{-\jj Q/x}
            = \jj \frac{H_0}{Q} x
            \qdonc
            \left\{
                \begin{array}{rcl}
                    G_{\dB} & = & 20\log\abs{\ul{H}}
                    \underset{x\rightarrow0}{\sim} 20 \log x\\
                    \f & = & \arg \ul{H} = \frac{\pi}{2}
                \end{array}
            \right.
        \]
        De même, dans la limite très hautes fréquences, on a
        \[\ul{H}
            \underset{x\rightarrow\infty}{\sim} \frac{H_0}{\jj Qx}
            = -\jj \frac{H_0}{Q} \frac{1}{x}
            \qdonc
            \left\{
                \begin{array}{rcl}
                    G_{\dB} & = & 20\log\abs{\ul{H}}
                    \underset{x\rightarrow0}{\sim} -20 \log x\\
                    \f & = & \arg \ul{H} = -\frac{\pi}{2}
                \end{array}
            \right.
        \]

        Ainsi, le diagramme de \textsc{Bode} asymptotique en gain compte
        \textbf{deux asymptotes de pentes $\mathbf{\pm\SI{20}{dB/décade}}$
        passant par $\mathbf{G_{\dB} = 0}$ pour $\mathbf{x=1}$}, alors que le
        diagramme asymptotique en phase compte \textbf{deux asymptotes
        horizontales de hauteurs $\mathbf{\pm \pi/2}$}. \bigbreak
        Pour tracer l'allure du diagramme réel, on utilise en plus les résultats
        de la question précédente qui indique que la courbe réelle passe par
        $G_{\dB} = \SI{-9.5}{dB}$ en $x=1$, alors que la courbe de phase réelle
        passe par $0$ en $x=1$~; d'où les diagrammes ci-dessous.

        \begin{center}
            \includegraphics[width=\linewidth]{wien_bode}
        \end{center}
    \item Numériquement, on trouve $\w_0 = \SI{2.0e3}{rad.s^{-1}}$. Comme le
        diagramme de \textsc{Bode} réel n'est pas donné dans l'énoncé, on peut
        au choix utiliser la fonction de transfert ou raisonner sur le diagramme
        asymptotique. Étudions le signal de sortie du filtre associé à chaque
        composante du signal d'entrée~:
        \begin{itemize}
            \item Le terme continu est complètement coupé par le filtre~;
            \item Le terme de pulsation $\w = \w_0/10$ se trouve une décade
                en-dessous de la pulsation propre~: avec le diagramme
                asymptotique il est donc atténué de \SI{20}{dB}, ce qui
                correspond à un facteur 10 en amplitude, et déphasé d'environ
                \SI{1.2}{rad} si le diagramme réel tracé~;
            \item Le terme de pulsation $10\w = \w_0$ est à la pulsation propre
                du filtre~: il n'est pas déphasé mais seulement atténué d'un
                facteur 1/3 (gain maximal)~;
            \item Le terme à la pulsation $100\w = 10\w_0$ est une décade
                au-dessus de la pulsation propre~: il est atténué comme le
                premier terme d'un facteur 10 en amplitude, et déphasé d'environ
                \SI{-1.2}{rad}. Ainsi,
        \end{itemize}
\end{enumerate}
\[\boxed{
    s(t) =
        \frac{E_0}{10}\cos (\wt - \num{1.2}) +
        \frac{E_0}{3}\cos(10\wt) +
        \frac{E_0}{10}\cos (100\wt + \num{1.2})
}\]

\section{Filtre ADSL}
\begin{enumerate}
    \item On isole les signaux téléphoniques avec un \textbf{filtre passe-bas},
        et les signaux informatiques avec un \textbf{filtre passe-haut}. La
        fréquence de coupure doit être à la fois nettement supérieure aux
        fréquences téléphoniques et nettement plus faible que les fréquences
        informatiques~: on prendra donc \fbox{$f_0 = \SI{10}{kHz}$}.
    \item ~

        \vspace{-24pt}
        \begin{minipage}{0.45\linewidth}
            En basses fréquences ($\w \longrightarrow 0$), les bobines se
            comportent comme des fils, soit
            \begin{center}
                \includegraphics[width=\linewidth]{adsl_bf}
            \end{center}
        \end{minipage}
        \hfill
        \vrule
        \hfill
        \begin{minipage}{0.45\linewidth}
            En hautes fréquences ($\w \longrightarrow \infty$), les bobines se
            comportent comme des interrupteurs ouverts, soit
            \begin{center}
                \includegraphics[width=\linewidth]{adsl_hf}
            \end{center}
        \end{minipage}
        Ainsi, le signal de sortie est non nul pour les hautes fréquences, et
        négligeable pour les basses fréquences~: c'est un \textbf{filtre
        passe-haut}. Il permettra d'obtenir les signaux informatiques.
    \item Pour exprimer $u_s$ en fonction de $u_e$, on peut faire un premier
        pont diviseur de tension pour exprimer $u_s$ en fonction de $u_{AB}$ du
        milieu~; puis avec une impédance équivalente à l'ensemble des 3 dipôles
        de droite, on refait un pont diviseur de tension pour avoir $u_{AB}$ en
        fonction de $u_e$, et on combine.
        \begin{center}
            \includegraphics[width=0.8\linewidth]{adsl_equiv}
        \end{center}
        On a donc d'abord~:
        \begin{gather*}
            \Uu_s = \frac{\Zu_L}{\Zu_L + \Zu_R}\Uu_{AB}
            \Leftrightarrow
            \Uu_s = \frac{\jlw}{\jlw + R}\Uu_{AB}
        \end{gather*}
        On aura donc ensuite~:
        \begin{gather*}
            \Uu_{AB} = \frac{\Zu_{\eq}}{\Zu_{\eq} + \Zu_R}\Uu_e
            \Leftrightarrow
            \Uu_{AB} = \frac{1}{1+\Zu_R\Yu_{\eq}}\Uu_e
        \end{gather*}
        On calcule alors $\Yu_{\eq}$~:
        \begin{gather*}
            \Yu_{\eq} = \frac{1}{\jlw} + \frac{1}{R+\jlw}
        \end{gather*}
        Et on combine~:
        \begin{gather*}
            \Uu_s =
                \frac{\jlw}{R+\jlw}\times\frac{1}{1+\Zu_R\Yu_{\eq}}\Uu_e
            \Leftrightarrow
            \Uu_s =
                \frac{\jlw}{R + \jlw + R \left( \dfrac{R+\jlw}{\jlw} + 1
                \right)}\times {\color{Red} \frac{\jlw}{\jlw}}\Uu_e\\
            \Leftrightarrow
            \Uu_s =
                \frac{-(L\w)^2}{R^2 + 3\jj RL\w - (L\w)^2}\Uu_e
            \Leftrightarrow
            \Uu_s =
                {\color{red}\cancel{\frac{R^2}{R^2}}}
                \frac{- \left( \dfrac{L}{R}\w \right)^2}{1 + 3\jj \dfrac{L}{R}\w
                - \left( \dfrac{L}{R}\w \right)^2}
                \Uu_e
        \end{gather*}
        Ainsi, en divisant par $\Uu_e$ pour avoir la fonction de transfert, on
        a~:
        \begin{gather*}
            \boxed{\Hu = \frac{-x^2}{1-x^2+3\jj x}}
            \qavec
            \boxed{\w_0 = \frac{R}{L}}
        \end{gather*}
    \item Pour $x \gg 1$, les termes en $x^2$ l'emportent sur les autres termes
        au numérateur et au dénominateur, et la fonction de transfert devient
        $\Hu \underset{x\rightarrow\infty}{\sim} 1$, donc \fbox{$G_{\dB} = 0$}
        et \fbox{$\f = 0$} (réel positif). \bigbreak
        Pour $x \ll 1$, les termes en $x$ sont négligeables devant $1$ au
        dénominateur, et on garde le numérateur~: la fonction de transfert
        devient donc $\Hu \underset{x\rightarrow0}{\sim} -x^2$, donc
        \fbox{$G_{\dB} \underset{x\rightarrow0}{\sim} 40\log(x)$} (pente de
        40\,\dB/décade), et \fbox{$\f = -\pi$} (réel négatif). \bigbreak
        Pour $x = 1$, on trouve $\Hu(1) = \jj/3$ donc \fbox{$G_{\dB}(1) =
        20\log(1/3) = \SI{-9.5}{dB}$}, et \fbox{$\f(1) = \pi/2$} (imaginaire pur).
\end{enumerate}
\begin{center}
    \includegraphics[width=.48\linewidth]{adsl_bode-gain}
    \includegraphics[width=.48\linewidth]{adsl_bode-phase}
\end{center}
\begin{enumerate}[resume]
    \item{} Il n'y a pas de pic de résonance car le facteur de qualité $Q$ est
        plus petit que $1/\sqrt{2}$.
    \item La fréquence de coupure est $f_0 = \DS \frac{\w_0}{2\pi} =
        \frac{R}{2\pi L}$~; on doit donc prendre
        \begin{gather*}
            \boxed{L = \frac{R}{2\pi f_0}}
            \qavec
            \left\{
                \begin{array}{rcl}
                    R & = & \SI{100}{\Omega}\\
                    f_0 & = & \SI{10}{kHz}
                \end{array}
            \right.\\
            \mathrm{A.N.~:}\quad
            \boxed{L = \SI{1.6}{mH}}
        \end{gather*}
\end{enumerate}

\section{Filtre de \textsc{Colpitts}}
\begin{enumerate}
    \item ~

        \vspace{-18pt}
        \begin{minipage}{0.48\linewidth}
            En basses fréquences ($\w \rightarrow 0$), les condensateurs se
            comportent comme des interrupteurs ouverts, la bobine comme un fil~: la
            tension $u_s$ est donc nulle.
            \begin{center}
                \includegraphics[width=\linewidth]{colpitts_bf}
            \end{center}
        \end{minipage}
        \hfill
        \vrule
        \hfill
        \begin{minipage}{0.48\linewidth}
            En hautes fréquences ($\w \rightarrow \infty$), les condensateurs se
            comportent comme des fils, la bobine comme un interrupteur ouvert~: la
            tension $u_s$ est donc nulle.
            \begin{center}
                \includegraphics[width=\linewidth]{colpitts_hf}
            \end{center}
        \end{minipage}

        Comme la tension est nulle aux extrêmes, c'est un \textbf{passe-bande}.
        Si elle était égale à la tension d'entrée aux extrêmes, ça serait un
        coupe-bande.
    \item On effectue deux diviseurs de tension successifs~: un pour déterminer
        $u_s$ en fonction de $u_L$, puis avec une impédance équivalente des
        trois dipôles de droite, on détermine $u_L$ en fonction de $u_e$ et on
        combine. C'est le même fonctionnement que pour l'exercice sur l'ADSL,
        question 3.
        \begin{center}
            \includegraphics[width=0.8\linewidth]{colpitts_equiv}
        \end{center}
        On a ainsi en premier lieu
        \begin{gather*}
            \Uu_s = \frac{\Zu_{3C}}{\Zu_{3C} + \Zu_C}\Uu_{AB}
            \Leftrightarrow
            \Uu_s = \frac{1/\jj3C\w}{1/\jj3C\w + 1/\jj C\w}\Uu_{AB}
            \Leftrightarrow
            \Uu_s = \frac{1}{1+3}\Uu_{AB}
            \Leftrightarrow
            \boxed{\Uu_s = \frac{\Uu_{AB}}{4}}
        \end{gather*}
        On aura donc ensuite~:
        \begin{gather*}
            \Uu_{AB} = \frac{\Zu_{\eq}}{\Zu_{\eq} + \Zu_R}\Uu_e
            \Leftrightarrow
            \boxed{\Uu_{AB} = \frac{1}{1+\Zu_R\Yu_{\eq}}\Uu_e}
        \end{gather*}
        On calcule alors $\Yu_{\eq}$ de l'association en parallèle de $L$ et
        $C$ en série avec $3C$. \textbf{Attention} à l'association en série de
        capacités~:
        \begin{gather*}
            Z_{C+3C} = \frac{1}{\jj3C\w} + \frac{1}{\jcw}{\color{orange}\times 
                \frac{3}{3}} = \frac{4}{\jj3C\w}\\
            \Yu_{\eq} = \Yu_L + \Yu_{C+3C}
            \Leftrightarrow
            \boxed{\Yu_{\eq} = \frac{1}{\jlw} + \jj3C\w/4}
        \end{gather*}
        Et on combine~:
        \begin{gather*}
            \Uu_s =
                \frac{1}{4}
                \frac{1}{1+R \left( \dfrac{1}{\jlw} + \dfrac{\jj3C\w}{4} \right)}
                \Uu_e
            \Leftrightarrow
            \Uu_s =
                \frac{1}{4}
                \frac{1}{1+\jj \left( - \dfrac{R}{L\w} + \dfrac{3RC\w}{4} \right)}
                \Uu_e
                \\
        \end{gather*}
        Ainsi, en divisant par $\Uu_e$ pour avoir la fonction de transfert, on
        a~:
        \begin{gather*}
            \Hu = \frac{1/4} {1+\jj\left(\dfrac{3RC\w}{4}-\dfrac{R}{L\w}\right)}
            \Leftrightarrow
            \boxed{\Hu = \frac{A}{1+\jj Q \left( \dfrac{\w}{\w_0} -
                \dfrac{\w_0}{\w} \right)}}
            \qavec
            \boxed{A = \frac{1}{4}}
        \end{gather*}
        Reste à trouver $Q$ et $\w_0$. Pour cela, on identifie membre à membre~:
        \begin{align*}
            \frac{Q}{\w_0} = \frac{3RC}{4} \quad (1)
            &\qet
            Q\w_0 = \frac{R}{L} \quad (2)\\
            \Leftrightarrow
            \boxed{Q = \frac{R\sqrt{3C}}{2\sqrt{L}}}
            &\qet
            \boxed{\w_0 = \frac{2}{\sqrt{3LC}}}
        \end{align*}
        où l'on obtient $Q$ et $\w_0$ en multipliant les équations (1) et (2)
        d'une part puis en en prenant la racine carrée, et en divisant (2) par
        (1) en en prenant la racine carrée, respectivement.
    \item Les parties rectilignes du diagramme correspondent aux limites
        asymptotiques du gain en décibels, c'est-à-dire pour $\w \ll \w_0$ et
        $\w \gg \w_0$. En effet,
        \begin{align*}
            \Hu \underset{\w \ll \w_0}{\sim} \jj \frac{A}{Q} \frac{\w}{\w_0}
            &\qet
            \Hu \underset{\w \gg \w_0}{\sim} -\jj \frac{A}{Q} \frac{\w_0}{\w}
            \\
            \Leftrightarrow
            G_{\dB} \underset{\w \ll \w_0}{\sim}
                20 \log \frac{A}{Q} + 20 \log \frac{\w}{\w_0}
            &\qet
            G_{\dB} \underset{\w \gg \w_0}{\sim}
                20 \log \frac{A}{Q} - 20 \log \frac{\w}{\w_0}
            \\
            \Leftrightarrow
            \f \underset{\w \ll \w_0}{\sim} \frac{\pi}{2}
            &\qet
            \f \underset{\w \gg \w_0}{\sim} - \frac{\pi}{2}
        \end{align*}
        Pour $\w = \w_0$, on trouve simplement $\Hu = A$ donc $G_{\dB}(\w_0) =
        \SI{-12}{dB}$ et $\f = 0$. La fréquence de résonance (ou fréquence d'accord)
        correspond au pic du diagramme de \textsc{Bode} (ou à l'intersection des
        asymptotes du gain en décibels) d'une part, ou correspond à la fréquence
        pour laquelle la phase est nulle~: on lit simplement \fbox{$f_0 =
        \SI{1}{kHz}$}. \bigbreak
        On trouve les fréquences de coupure en trouvant les fréquences $f_1$ et
        $f_2$ telles que $G_{\dB} = G_{\max} - \SI{3}{dB}$, soit $G_{\dB} =
        \SI{-15}{dB}$~: on lit approximativement \fbox{$f_1 = \SI{950}{Hz}$} et
        \fbox{$f_2 = \SI{1050}{Hz}$}.
\end{enumerate}

\section{Filtre de \textsc{Butterworth} d'ordre 3}
\begin{enumerate}
    \item Il suffit pour cette question de développer les puissances sur les
        $\jj$, de calculer le module et de développer~:
        \begin{gather*}
            \Hu = \left( 1 + 2\jj x - 2x^2 -\jj x^3 \right)^{-1}
            \Leftrightarrow
            \abs{\Hu}
                = \left( (1-2x^2)^2 + (2x - x^3)^2 \right)^{-1/2}\\
            \Leftrightarrow
            \abs{\Hu}
                = \left( 1-\bcancel{4x^2} + \cancel{4x^4}
                    + \bcancel{4x^2} - \cancel{4x^4} + x^6 \right)^{-1/2}
                = \left( 1+x^6 \right)^{-1/2}
        \end{gather*}
        ce qui correspond bien à un filtre de \textsc{Butterworth} d'ordre 3.
    \item ~

        \vspace{-24pt}
        \begin{minipage}{0.48\linewidth}
            Pour étudier le diagramme de \textsc{Bode} asymptotique, on définit
            d'abord le gain en décibels~: $G_{\dB} = 20\log(\abs{\Hu}) =
            20\log(\left( 1+x^6 \right)^{-1/2}) = -10\log(1+x^6)$. Ensuite, on
            étudie son comportement asymptotique pour $x \ll 1$ et $x \gg 1$~:
            on trouve
            \begin{gather*}
                \boxed{G_{\dB} \underset{x \ll 1}{\sim} 0}
                \qet
                \boxed{G_{\dB} \underset{x \gg 1}{\sim} -60\log(x)}
            \end{gather*}
            d'où le diagramme de \textsc{Bode} asymptotique ci-contre. Par
            rapport à de l'ordre 1 (\SI{-20}{dB/décade}) ou de l'ordre 2
            (\SI{-40}{dB/décade}), l'atténuation des hautes fréquences est
            encore plus prononcé~: une fréquence 10 fois supérieure à $f_0$
            serait atténuée d'un facteur 1000 au lieu d'un facteur 10.
        \end{minipage}
        \hfill
        \begin{minipage}{0.48\linewidth}
            \begin{center}
                \includegraphics[width=\linewidth]{butterworth_bode-gain}
            \end{center}
        \end{minipage}
    \item Ici encore, on utilise deux ponts diviseurs de tension successifs~: on
        calcule $u_s$ en fonction de $u_{AB}$, puis $u_{AB}$ en fonction de
        $u_e$ après avoir déterminé l'impédance équivalente de l'ensemble des
        dipôles de droite.
        \begin{center}
            \includegraphics[width=0.8\linewidth]{butterworth_equiv}
        \end{center}
        On aura donc en premier lieu
        \begin{gather*}
            \Uu_s = \frac{\Zu_R}{\Zu_R + \Zu_{L_2}}\Uu_{AB}
            \Leftrightarrow
            \boxed{\Uu_s = \frac{R}{R+\jj L_2\w}\Uu_{AB}}
        \end{gather*}
        Et ensuite, on aura
        \begin{gather*}
            \Uu_{AB} = \frac{\Zu_{\eq}}{\Zu_{\eq} + \Zu_{L_1}}\Uu_e
            \Leftrightarrow
            \boxed{\Uu_{AB} = \frac{1}{1+\Zu_{L_1}\Yu_{\eq}}\Uu_e}
        \end{gather*}
        On calcule alors $\Yu_{\eq}$ de l'association en parallèle de $C$ et
        $L_2$ en série avec $R$~:
        \begin{gather*}
            Z_{L_2+R} = \jj L_2\w + R\\
            \Yu_{\eq} = \Yu_C + \Yu_{L_2+R}
            \Leftrightarrow
            \Yu_{\eq} = \jcw + \frac{1}{\jj L_2\w + R}
            \Leftrightarrow
            \Yu_{\eq} = \frac{\jcw (\jj L_2\w + R)+1}{\jj L_2\w + R}\\
            \Leftrightarrow
            \boxed{\Yu_{\eq} = \frac{1-L_2C\w^2 + \jj RC\w}{R + \jj L_2\w}}
        \end{gather*}
        Et on combine~:
        \begin{align*}
            \Uu_s &=
                \frac{R}{R+\jj L_2\w} \times
                \frac{1}{1+\jj L_1\w
                    \left( \dfrac{1-L_2C\w^2 + \jj RC\w}{R + \jj L_2\w} \right)}
                \Uu_e\\
            \Leftrightarrow
            \Uu_s &=
                \frac{R}{R+\jj L_2\w}\times
                \frac{1}{1+\dfrac{\jj L_1\w - \jj L_1L_2C\w^3 + (\jj\w)^2RCL_1}
                    {R + \jj L_2\w}}
                \Uu_e\\
            \Leftrightarrow
            \Uu_s &= 
                \frac{R}
                    {R+\jj L_2\w + \jj L_1\w - \jj L_1L_2C\w^3 + (\jj\w)^2RCL_1}
                \Uu_e\\
            \Leftrightarrow
            \Uu_s &= \frac{1}{1+ \jj\w \dfrac{L_1+L_2}{R} + (\jj\w)^2L_1C +
                (\jj\w)^3 \dfrac{L_1L_2C}{R}}
                \Uu_e
        \end{align*}
        en utilisant que $-\jj = \jj^3$. Ainsi, en divisant par $\Uu_e$ pour
        avoir la fonction de transfert, on a bien
        \begin{gather*}
            \boxed{\Hu = \frac{1}{1+2\jj x + 2(\jj x)^2 + (\jj x)^3}}
            \qavec
            \left\{
                \begin{array}{rcl}
                    \dfrac{2}{\w_0}     & = & \dfrac{L_1+L_2}{R}\\[12pt]
                    \dfrac{2}{\w_0{}^2} & = & L_1C\\[12pt]
                    \dfrac{1}{\w_0{}^3} & = & \dfrac{L_1L_2C}{R}
                \end{array}
            \right.
            \Leftrightarrow
            \boxed{
            \left\{
                \begin{array}{rcl}
                    L_1 & = & \dfrac{3R}{2\w_0}\\[12pt]
                    L_2 & = & \dfrac{R}{2\w_0}\\[12pt]
                    C   & = & \dfrac{4}{3Rw_0}
                \end{array}
        \right.}
        \end{gather*}
    \item Pour mettre des filtres en cascade et avoir $\Hu = \Hu_1\Hu_2$, il
        faut que l'impédance de sortie du filtre 1 soit faible devant
        l'impédance d'entrée du filtre 2. Dans ce cas, on utilise un filtre
        d'ordre 1 avec un numérateur constant (donc un passe-bas de la forme
        $\Hu_1 = \frac{H_1}{1 + \jj x}$), et un filtre
        d'ordre 2 avec un numérateur lui aussi constant~: soit un passe-bas soit
        un passe-bande. Le passe-bande fait intervenir $\jj Q \left( x -
        \frac{1}{x} \right)$ au dénominateur, donc il est plus simple d'utiliser
        une passe-bas d'ordre 2 avec $1 + \jj/Q x + (\jj x)^2$ au dénominateur~:
        \begin{gather*}
            \Hu
                = \frac{H_1}{1+ \jj x}\times \frac{H_2}{1 + \dfrac{\jj}{Q}x +
                    (\jj x)^2}
                = \frac{H_1H_2}{1 + \jj x \left( 1 + \dfrac{1}{Q} \right)
                    + (\jj x)^2 \left( 1 + \dfrac{1}{Q^2} \right) + (\jj x)^3}
        \end{gather*}
        Pour trouver un filtre de \textsc{Butterworth} d'ordre 3 de cette
        manière, il faut donc $H_1 = H_2 = 1$ et \fbox{$Q = 1$}.
\end{enumerate}

\end{document}
