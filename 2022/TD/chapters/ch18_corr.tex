\documentclass[a4paper, 12pt, final, garamond]{book}
\usepackage{cours-preambule}

\raggedbottom

\makeatletter
\renewcommand{\@chapapp}{M\'ecanique -- chapitre}
\makeatother

\begin{document}
\setcounter{chapter}{1}

\chapter{Correction du TD}

\section{Collision entre deux voitures}
\begin{enumerate}
    \item Notons M$_1$ et M$_2$ les points matériels représentant chacun une des
        deux voitures. On se limite au mouvement unidimensionnel selon l'axe $x$
        et on notera $x_1(t)$ et $x_2(t)$ les positions respectives de M$_1$ et
        M$_2$ selon cet axe. Initialement, $x_1(t=0) = d = \SI{20}{m}$ et
        $x_2(t=0) = 0$. \bigbreak
	
        La voiture M$_1$ de Xari subit l'accélération (qui est négative donc
        c'est une décélération) constante $a_1$. Ainsi, par intégration
        successive,
        \[
            x_1(t) = \frac{1}{2}a_1 t^2 + \alpha t + \beta
        \]
        Avec $\alpha$ et $\beta$ deux constantes d'intégration. En considérant
        par ailleurs une vitesse initiale $v_0$ et une position initiale $d$, on
        obtient~:
        \[
            x_1(t) = \frac{1}{2}a_1 t^2 + v_0 t + d
        \]
        Pour le second véhicule, il faut décomposer le mouvement en deux étapes
        successives~:
        \begin{itemize}
            \item pour $t\in \SIrange{0}{1}{s}$, $a = 0$. La position initiale
                étant par ailleurs nulle et la vitesse initiale étant égale à
                $v_0$, il vient, pour $t\in \SIrange{0}{1}{s}$~:
                \[
                    x_2(t) = v_0t
                \]
            \item 
                pour $t>1$, l'accélération vaut $a_2$ constante. Notons par
                ailleurs $t_2 = \SI{1}{s}$. On a par intégration~:
                \[
                    v_2(t) = a_2 t + \gamma
                \]
                Avec $\gamma$ une constante à déterminer. Or, par continuité de
                la vitesse, $v_2(t=t_2) = v_0$. Ainsi,
                \[
                    v_2(t) = a_2 (t-t_2) + v_0
                \]
                Intégrons une nouvelle fois, avec $\delta$ une nouvelle
                constante d'intégration~:
                \[
                    x_2(t) = \frac{1}{2} a_2 (t-t_2)^2 + v_0t + \delta
                \]
	            En utilisant le fait que $x(t_2) = v_0t_2$, il vient finalement
                \[
                    x_2(t) = \frac{1}{2} a_2 (t-t_2)^2 + v_0t
                \]
        \end{itemize}
    \item Il y a contact à l'instant $t_c$ tel que
        \[
            x_1(t_c) = x_2(t_c)
        \]
        Supposons d'abord le contact sur l'intervalle $t\in\SIrange{0}{1}{s}$.
        Il faut alors résoudre~:
        \begin{gather*}
            \frac{1}{2}a_1 {t_c}^2 + \cancel{v_0t_c} + d = \cancel{v_0t_c}
            \\
            \Leftrightarrow
            \boxed{t_c = \sqrt{\frac{-2d}{a_1}}}
            \qavec
            \left\{
                \begin{array}{rcl}
                    d & = & \SI{20}{m}\\
                    a_1 & = & -\SI{30.0}{m.s^{-2}}
                \end{array}
            \right.\\
            \AN
            \boxed{t_c = \SI{1.41}{s} > \SI{1}{s}}
        \end{gather*}
	
        Cette solution est donc exclue puisqu'elle n'est pas en accord avec
        notre hypothèse initiale $t\in\SIrange{0}{1}{s}$. \bigbreak
	
	    Supposons maintenant $t_c>\SI{1}{s}$. Il faut résoudre :
        \begin{align*}
            \frac{1}{2}a_1 {t_c}^2 + \cancel{v_0t_c} + d
                &= \frac{1}{2} a_2 (t_c-t_2)^2 + \cancel{v_0t_c}
            \\
            \Leftrightarrow
            \frac{1}{2}a_1t_c{}^2 + d
                &= \frac{1}{2}a_2 \left( t_c{}^2 - 2t_2t_c + t_2{}^2\right)
            \\
            \Leftrightarrow
            \frac{1}{2} \left( a_1 - a_2 \right)t_c{}^2 + a_2t_2t_c + d -
            \frac{1}{2}a_2t_2{}^2
                &= 0
        \end{align*}

        C'est un polynôme de degré 2 dont le discriminant $\Delta$ est tel que
        \begin{gather*}
            \boxed{\D = (a_2t_2)^2 - 2(a_1-a_2)\left(d -
                \frac{1}{2}a_2t_2{}^2\right)}
            \qavec
            \left\{
                \begin{array}{rcl}
                    d & = & \SI{20}{m}\\
                    a_1 & = & -\SI{30.0}{m.s^{-2}}\\
                    a_2 & = & -\SI{20.0}{m.s^{-2}}\\
                    t_2 & = & \SI{1}{s}
                \end{array}
            \right.\\
            \AN
            \boxed{\D = \SI{600}{m.s^{-2}}}\\
            \text{D'où}\quad
            t_{c,\pm} = \frac{-a_2t_2 \pm \sqrt{\D}}{(a_1-a_2)}\\
            \Leftrightarrow
            t_{c,+} = \SI{-3.45}{s}
            \qou
            t_{c,-} = \SI{1.45}{s}
        \end{gather*}
	    La solution négative étant exclue, on trouve finalement
        \[
            \boxed{t_c = \SI{1.45}{s}}
            \qet
            \boxed{x_1(t_c) = \SI{42.5}{m}}
        \]
        Il était donc pratiquement impossible que Pierre esquive Xari, étant
        donné qu'en freinant au plus tôt il n'a eu que \SI{0.45}{s} avant de
        rentrer en collision avec lui, laissant peu de marge à un autre temps de
        réaction et à une autre manœuvre évasive.
\end{enumerate}

\section{Masse attachée à 2 ressorts}
\begin{enumerate}
    \item On étudie ici le point matériel M de masse $m$, dans le référentiel du
        laboratoire supposé galiléen avec le repère (O,$\uz$), $\uz$ vertical
        ascendant. On repère le point M par son altitude $\OMr = z(t)$.
        On effectue le \textbf{bilan des forces}~:
        \[
            \begin{array}{ll}
                \textbf{Poids} & \Pf = m\gf = -mg\uz\\
                \textbf{Ressort 1} & \Ff\ind{ressort 1}
                    = -k(\OMr - \ell_0)\uz = -k(z-\ell_0)\uz\\
                \textbf{Ressort 2} & \Ff\ind{ressort 2}
                    = +k({\rm O'M} - \ell_0)\uz = +k(L-z-\ell_0)\uz
            \end{array}
        \]
        avec le ressort 1 celui d'en-dessous, le ressort 2 celui d'au-dessus. On
        notera simplement $\Ff_1$ et $\Ff_2$ dans la suite.
        Avec le PFD, on a
        \begin{gather*}
            m\af = \Pf + \Ff_1 + \Ff_2\\
            \Lra
            m\zpp = -mg -k(z - \cancel{\ell_0}) + k(L - z -\cancel{\ell_0})\\
            \Lra
            \boxed{\zpp + \frac{2k}{m}z = \frac{k}{m}L -g}
        \end{gather*}
        À l'équilibre, le ressort ne bouge plus~; on a donc $\zp = \zpp = 0$, et
        on trouve ainsi $z_{\eq}$~:
        \[\boxed{z_{\eq} = \frac{L}{2} - \frac{mg}{2k}}\]
        Sans la pesanteur, la masse sera à l'équilibre entre les deux ressorts,
        en toute logique. La gravité diminue cette altitude. On remarque que
        cette association de ressort est équivalente à avoir un seul ressort de
        raideur $2k$.
    \item On a commencé la détermination de l'équation différentielle dans la
        question 2. On peut simplifier son expression en remarquant qu'à droite
        du signe égal, on doit trouver quelque chose homogène à $\w_0{}^2z$. On
        commence par identifier $\w_0$ avec la forme canonique~:
        \[\w_0 = \sqrt{\frac{2k}{m}}
            \qdonc
            \frac{k}{m}L - g = \w_0{}^2z_{\eq}
        \]
        et finalement,
        \[\boxed{\zpp + \w_0{}^2z = \w_0{}^2z_{\eq}}\]
    \item La solution complète $z(t)$ et la somme de la solution particulière
        constante $z_p$ et de la solution homogène $z_h$. La solution
        particulière est, par définition, $z_{\eq}$ (on l'a montré question 1).
        La solution homogène est celle d'un oscillateur harmonique, à savoir
        \[z_h = A\cos(\w_0t) + B\sin(\w_0t)\]
        Ainsi,
        \[z(t) = z_{\eq} + A\cos(\w_0t) + B\sin(\w_0t)\]
        On trouve $A$ et $B$ avec les conditions initiales~:
        \begin{itemize}
            \item $z(0) = z_{\eq} + a$ (masse lâchée d'une hauteur $a$ par
                rapport à la position d'équilibre), or $z(0) = A + z_{\eq}$,
                donc
                \[A = a\]
            \item $\zp(0) = 0$ (masse lâchée sans vitesse initiale), or $\zp(0)
                = B\w_0$ donc
                \[B = 0\]
        \end{itemize}
        \leftcenters{Ainsi,}{$\boxed{z(t) = z_{\eq} + a\cos(\w_0t)}$}
\end{enumerate}

\section{Plan incliné et frottements solides}
\begin{enumerate}
    \item
        \begin{enumerate}[leftmargin=20pt]
            \item 
                \begin{itemize}[label=$\diamond$, leftmargin=10pt]
                    \litem{Système~:} \{brique\}
                    \litem{Référentiel~:} galiléen $(\Or, \ux, \uy)$ (voir schéma)
                    \litem{O et $t$ initial~:} tels que $\OM(0) = \of$
                    \litem{Vitesse initiale~:} $\vf(0) = v_0\ux$
                    \litem{Bilan des forces~:}
                        \[
                            \begin{array}{ll}
                                \textbf{Poids} & \Pf = -mg\cos\a\uy - mg\sin\a\ux\\
                                \textbf{Réaction} & \Rf = R\uy
                            \end{array}
                        \]
                    \litem{PFD~:}
                        \begin{gather*}
                            m\af = \Pf + \Rf
                            \Lra
                            \left\{
                                \begin{array}{rcl}
                                    \cancel{m}\xpp & = & -\cancel{m}g\sin\a\\
                                    \underbrace{\cancel{m\ypp}}_{=0}
                                                   & = & -mg\cos\a + R
                                \end{array}
                            \right.
                        \end{gather*}
                \end{itemize}
                Il n'y a pas de mouvement sur $\uy$ étant donné que le mouvement
                se fait selon $\ux$~; ainsi $\boxed{y = \yp = \ypp = 0}$, et la
                seconde équation donne
                \[R = mg\cos\a\]
                On intègre la première pour avoir l'équation horaire sur $x(t)$~:
                \begin{gather*}
                    \xp(t) = -gt\sin\a+v_0
                    \Ra
                    \boxed{x(t) = -\frac{1}{2}gt^2\sin\a + v_0t}
                \end{gather*}
                avec les conditions initiales $\xp(0) = v_0$ et $x(0) = 0$.
            \item On trouve le temps d'arrêt quand la vitesse est nulle. Soit
                $t_s$ ce temps d'arrêt~:
                \begin{gather*}
                    \xp(t_s) = 0
                    \Lra
                    v_0 = gt_s\sin\a
                    \Lra
                    \boxed{t_s = \frac{v_0}{g\sin\a}}
                \end{gather*}
                On remarque alors que si $\a = 0$, $t_s \rightarrow +\infty$, ce
                qui est logique puisque sans frottement la brique ne
                s'arrêterait jamais. On obtient la distance d'arrêt en injectant
                ce temps dans $x(t)$~:
                \begin{gather*}
                    x(t_s) = -\frac{1}{2}\cancel{g}
                    \frac{v_0{}^2}{g^{\cancel{2}}\sin^{\bcancel{2}}\a}
                    \bcancel{\sin\a} + v_0 \frac{v_0}{g\sin\a}
                    \Lra
                    \boxed{x(t_s) = \frac{1}{2}\frac{v_0{}^2}{g\sin\a}}
                \end{gather*}
        \end{enumerate}
    \item 
        \begin{enumerate}[leftmargin=20pt]
            \item On reprend le même système, mais le bilan des forces change~:
                \begin{itemize}[label=$\diamond$, leftmargin=10pt]
                    \litem{Bilan des forces~:}
                        \[
                            \begin{array}{ll}
                                \textbf{Poids} & \Pf = -mg\cos\a\uy - mg\sin\a\ux\\
                                \textbf{Réaction} & \Rf = R_N\uy -R_T\ux
                            \end{array}
                        \]
                        En effet, sur la montée de la brique, sa vitesse est
                        dirigée vers $+\ux$, donc la force de frottement (qui
                        est une force de freinage et donc opposée à la vitesse)
                        est dirigée vers $-\ux$. De plus, avec les lois du
                        frottement de \textsc{Coulomb}, sur la montée la brique
                        glisse sur le support, on a donc
                        \[\boxed{R_T = fR_N}\]
                    \litem{PFD~:}
                        \begin{gather*}
                            m\af = \Pf + \Rf
                            \Lra
                            \left\{
                                \begin{array}{rcl}
                                    m\xpp & = & -mg\sin\a - fR_N\\
                                    \underbrace{\cancel{m\ypp}}_{=0}
                                          & = & -mg\cos\a + R_N
                                \end{array}
                            \right.
                        \end{gather*}
                \end{itemize}
                Il n'y a pas de mouvement sur $\uy$ étant donné que le mouvement
                se fait selon $\ux$~; ainsi $\boxed{y = \yp = \ypp = 0}$, et la
                seconde équation donne
                \[R_N = mg\cos\a\]
                Que l'on réinjecte dans la première~:
                \[\xpp = -g\sin\a - fg\cos\a\]
                On intègre cette dernière pour avoir l'équation horaire sur $x(t)$~:
                \begin{gather*}
                    \xp(t) = -g(\sin\a + f\cos\a)t+v_0
                    \Ra
                    \boxed{x(t) = -\frac{1}{2}g(\sin\a + f\cos\a)t^2 + v_0t}
                \end{gather*}
                avec les conditions initiales $\xp(0) = v_0$ et $x(0) = 0$. On
                retrouve le résultat précédent en posant $f=0$.
            \item On trouve le temps d'arrêt quand la vitesse est nulle. Soit
                $t_s$ ce temps d'arrêt~:
                \begin{gather*}
                    \xp(t_s) = 0
                    \Lra
                    v_0 = gt_s(\sin\a + f\cos\a)
                    \Lra
                    \boxed{t_s = \frac{v_0}{g(\sin\a + f\cos\a)}}
                \end{gather*}
                Ce temps est plus \textbf{court} que sans frottements. On
                obtient la distance d'arrêt en injectant ce temps dans $x(t)$~:
                \begin{gather*}
                    x(t_s) = -\frac{1}{2}\cancel{g(\sin\a + f\cos\a)}
                    \frac{v_0{}^2}{\left(g(\sin\a + f\cos\a)\right)^{\cancel{2}}}
                    + v_0 \frac{v_0}{g(\sin\a + f\cos\a)}\\
                    \Lra
                    \boxed{x(t_s) = \frac{1}{2}\frac{v_0{}^2}{g(\sin\a + f\cos\a)}}
                \end{gather*}
        \end{enumerate}
    \item 
        \begin{enumerate}[leftmargin=20pt]
            \item Cette fois, la brique est initialement à l'arrêt, soit $\af(0)
                = \of$, et la brique ne glisse pas donc $R_T < fR_N$. On aura
                mouvement quand il y aura glissement, c'est-à-dire quand $R_T =
                fR_N$. On reprend donc le système précédent avec $\af = \of$~:
                \begin{gather*}
                    \underbrace{\bcancel{m\af}}_{=\of} = \Pf + \Rf
                    \Lra
                    \left\{
                        \begin{array}{rcl}
                            0 & = & -mg\sin\a - fR_N\\
                            0 & = & -mg\cos\a + R_N
                        \end{array}
                    \right.
                    \Lra
                    \left\{
                        \begin{array}{rcl}
                            \sin\a & = & f\cos\a\\
                            R_N    & = & mg\cos\a
                        \end{array}
                    \right.\\
                    \Lra
                    f = \tan\a
                    \Lra
                    \boxed{\alpha = \atan(f)}
                \end{gather*}
            \item \leftcenters{On trouve}{
                    $\boxed{\a\ind{fer/chêne} = \ang{14}}
                    \qet
                    \boxed{\a\ind{chêne/chêne} = \ang{19}}
                    $}
        \end{enumerate}
    \item 
        \begin{enumerate}[leftmargin=20pt]
            \item 
                \begin{itemize}[label=$\diamond$, leftmargin=10pt]
                    \litem{Système~:} \{armoire\}
                    \litem{Référentiel~:} $(\Or, \ux, \uy)$ avec $\uy$ vertical
                        asendant
                    \litem{Repère~:} On suppose la force de traction dirigée
                        vers $+\ux$, et donc la vitesse de l'armoire selon $+\ux$
                    \litem{Bilan des forces}~:
                        \[
                            \begin{array}{ll}
                                \textbf{Poids} & \Pf = m\gf = -mg\uy\\
                                \textbf{Réaction normale} & \Rf_N = R_N\uy\\
                                \textbf{Réaction tangentielle} & \Rf_T =
                                -R_T\ux\\
                                \textbf{Traction} & \Ff = F\ux
                            \end{array}
                        \]
                        À la limite du glissement, on a $R_T = fR_N$.
                    \litem{PDF~:} quand le mouvement est lancé, l'accélération est
                        nulle.
                        \begin{gather*}
                            \underbrace{\bcancel{m\af}}_{=0}
                                = \Pf + \Rf_N + \Rf_T + \Ff
                            \Lra
                            \left\{
                                \begin{array}{rcl}
                                    0 & = & -mg + R_N\\
                                    0 & = & F - fR_N
                                \end{array}
                            \right.\\
                            \Lra
                            \left\{
                                \begin{aligned}
                                    R_N       & = mg\\
                                    \Aboxed{F & = fmg}
                                \end{aligned}
                            \right.
                            \qavec
                            \left\{
                                \begin{array}{rcl}
                                    m & = & \SI{100}{kg}\\
                                    g & = & \SI{10}{m.s^{-2}}\\
                                    f & = & \num{0.25}
                                \end{array}
                            \right.\\
                            \AN
                            \boxed{F = \SI{250}{N}}
                            \Lra
                            \boxed{\frac{F}{g} = \SI{25}{kg}}
                        \end{gather*}
                \end{itemize}
                Ainsi, il suffit de fournir une force égale à un quart du poids.
            \item Mettre des patins permet de diminuer le coefficient de
                frottement, et donc de diminuer la force de traction nécessaire
                pour déplacer le meuble.
        \end{enumerate}
\end{enumerate}

\section{Coup franc et frottements fluides}
\begin{enumerate}
    \item 
        \begin{enumerate}[leftmargin=20pt]
            \item
                \begin{itemize}[label=$\diamond$, leftmargin=10pt]
                    \litem{Système~:} \{ballon\}
                    \litem{Référentiel~:} terrestre galiléen
                    \litem{Repère~:} cartésien $(\Or,\ux,\uy)$, $\uy$ vertical
                        ascendant, $\ux$ vers le but
                    \litem{Origine et instant initial~:} $\OM(0) = \of$
                    \litem{Vitesse initiale~:} $\vf(0) = v_0\cos\a\ux +
                        v_0\sin\a\uy$
                    \litem{BDF~:}
                        \[
                            \begin{array}{ll}
                                \textbf{Poids} & \Pf = m\gf = -mg\uy
                            \end{array}
                        \]
                    \litem{PFD~:}
                        \begin{gather}
                            \label{eq:foot1pfd}
                            \cancel{m}\af = -\cancel{m}g\uy
                            \Lra
                            \left\{
                                \begin{array}{rcl}
                                    \xpp & = & 0\\
                                    \ypp & = & -g
                                \end{array}
                            \right.
                        \end{gather}
                \end{itemize}
                Ainsi,
                \begin{gather}
                    \label{eq:foot1eqho}
                    \eqref{eq:foot1pfd}
                    \Ra
                    \left\{
                        \begin{array}{rcl}
                            \xp & = & v_0\cos\a\\
                            \yp & = & -gt + v_0\sin\a
                        \end{array}
                    \right.
                    \Ra
                    \left\{
                        \boxed{
                        \begin{aligned}
                            x(t) &= v_0t\cos\a\\
                            y(t) &= -\frac{1}{2}gt^2 + v_0t\sin\a
                        \end{aligned}
                        }
                    \right.
                \end{gather}
                étant donné les conditions initiales. On trouve la
                trajectoire en isolant $t(x)$ pour avoir $y(x)$~:
                \begin{gather*}
                    \eqref{eq:foot1eqho}
                    \Ra
                    \left\{
                        \begin{aligned}
                            t(x) &= \frac{x}{v_0\cos\a}\\
                            \Aboxed{
                                y(x) &= - \frac{g}{2v_0{}^2\cos^2\a}x^2 +
                                x\tan\a}
                        \end{aligned}
                    \right.
                \end{gather*}
            \item Le ballon passe au-dessus du mur si $y(x\ind{mur}) \geq
                h\ind{mur}$ avec $h\ind{mur}$ la hauteur du mur et $x\ind{mur}$
                sa position horizontale. Avec une application numérique, on
                obtient
                \[y(x\ind{mur}) = \SI{2.17}{m} > h\ind{mur} = \SI{1.90}{m}\]
                donc le ballon \underline{passe bien au-dessus du mur}.
            \item Le tir est cadré si $y(x\ind{but}) \leq h\ind{but}$. Or,
                \[y(x\ind{but}) = \SI{1.73}{m}\]
                donc \underline{le tir est bien cadré}.
        \end{enumerate}
    \item 
        \begin{enumerate}[leftmargin=20pt]
            \item Avec le même système, seul le bilan des forces est modifié (et
                donc le PFD)~:
                \begin{itemize}[label=$\diamond$, leftmargin=10pt]
                    \litem{BDF~:}
                        \[
                            \begin{array}{ll}
                                \textbf{Poids} & \Pf = -mg\uy\\
                                \textbf{Frottements} & \Ff = -h\vf = -h\xp\ux
                                -h\yp\uy
                            \end{array}
                        \]
                    \item \leftcenters{\bfseries
                        PFD~:}{$m\af = -mg\uy -h\xp\ux -h\yp\uy$}
                \end{itemize}
                \vspace{-10pt}
                \begin{align*}
                    \Lra
                    \left\{
                        \begin{aligned}
                            m\xpp &= -h\xp\\
                            m\ypp &= -mg -h\yp
                        \end{aligned}
                    \right.
                    & \Lra
                    \left\{
                        \begin{aligned}
                            \xpp + \frac{h}{m}\xp &= 0\\
                            \ypp + \frac{h}{m}\yp &= -g
                        \end{aligned}
                    \right.
                    \\\Lra
                    \left\{
                        \begin{aligned}
                            \dot{v_x} + \frac{v_x}{\tau} &= 0\\
                            \dot{v_y} + \frac{v_y}{\tau} &= -g
                        \end{aligned}
                    \right.
                    & \Lra
                    \left\{
                        \begin{aligned}
                            v_x(t) &= A\exr^{-t/\tau}\\
                            v_y(t) &= -g\tau + B\exr^{-t/\tau}
                        \end{aligned}
                    \right.
                    \\
                    \text{Or,}\quad
                    \left\{
                        \begin{aligned}
                            v_x(0) &= v_0\cos\a\\
                            v_y(0) &= v_0\sin\a
                        \end{aligned}
                    \right.
                    & \Ra
                    \left\{
                        \begin{aligned}
                            A &= v_0\cos\a\\
                            B &= v_0\sin\a + g\tau
                        \end{aligned}
                    \right.\\
                    \text{donc}\quad
                    \left\{
                        \begin{aligned}
                            v_x(t) &= v_0\cos\a\exr^{-t/\tau}\\
                            v_y(t) &= \left(v_0\sin\a
                                + g\tau\right)\exr^{-t/\tau}
                                -g\tau
                        \end{aligned}
                    \right.
                    & \Ra
                    \left\{
                        \begin{aligned}
                            x(t) &= -v_0\tau\cos\a\exr^{-t/\tau}
                                + C\\
                            y(t) &= -\left(v_0\tau\sin\a
                                + g\tau^2\right)\exr^{-t/\tau}
                                -g\tau t + D
                        \end{aligned}
                    \right.\\
                    \qor
                    \left\{
                        \begin{aligned}
                            x(0) &= 0\\
                            y(0) &= 0
                        \end{aligned}
                    \right.
                    & \Ra
                    \left\{
                        \begin{aligned}
                            C &= v_0\tau\cos\a\\
                            D &= -\left(v_0\tau\sin\a + g\tau^2\right)
                        \end{aligned}
                    \right.
                \end{align*}
                Finalement,
                \begin{empheq}[box=\fbox, left=\empheqlbrace]{align}
                    \label{eq:foot2x}
                    x(t) &= v_0\tau\cos\a\left(1 - \exr^{-t/\tau}\right)\\
                    \label{eq:foot2y}
                    y(t) &= \left(v_0\tau\sin\a + g\tau^2\right)
                        \left(1-\exr^{-t/\tau}\right) -g\tau t
                \end{empheq}
            \item On isole $t(x)$ de \eqref{eq:foot2x} pour l'injecter dans
                \eqref{eq:foot2y}~:
                \begin{empheq}[box=\fbox, left=\empheqlbrace]{align*}
                    % \label{eq:foot2t}
                    t(x) &= -\tau\ln(1- \frac{x}{v_0\tau\cos\a})\\
                    % \label{eq:foot2traj}
                    y(x) &= \left(\tan\a + \frac{g\tau}{v_0\cos\a}\right)x
                        +g\tau^2\ln(1- \frac{x}{v_0\tau\cos\a})
                \end{empheq}
            \item On calcule~:
                \[y(x\ind{mur}) = \SI{2.17}{m}\]
                donc \underline{le ballon passe au-dessus du mur}.
            \item On calcule~:
                \[y(x\ind{but}) \approx \SI{1.73}{m}\]
                donc \underline{le tir est bien cadré}. On constate que les
                frottements n'ont eu que peu d'influence sur ce mouvement~; il
                n'est en effet pas très rapide, donc la force de frottements est
                restée assez faible.
        \end{enumerate}
\end{enumerate}

\section{Charge soulevée par une grue}
\begin{enumerate}
    \item 
        \begin{itemize}[label=$\diamond$, leftmargin=10pt]
            \litem{Système~:} \{masse $m$\} repérée par son centre d'inertie $M$.
            \litem{Référentiel~:} relié au sol, galiléen.
            \litem{Coordonnées~:} cartésiennes, $(\Or,\ux,\uy,\uz)$ avec $\uz$
                vertical ascendant, O au pieds de la grue.
            \litem{BDF~:} avant qu'elle ne décolle, il y a la réaction du sol~; on
                s'intéresse au décollage, donc au moment où elle s'annule. On
                aura donc
                \[
                    \begin{array}{ll}
                        \textbf{Poids} & \Pf = m\gf = -mg\uz\\
                        \textbf{Tension} & \Tf = T\uz
                    \end{array}
                \]

            \litem{PFD~:} au moment où la masse décolle, son accélération est
                positive et selon $\uz$, soit $\af = \zpp\uz$~; en supposant un
                décollage en douceur, $\zpp \approx 0$, soit
                \[
                    m\af = \Pf + \Tf
                    \Lra
                    0 = -mg + T
                    \Lra
                    \boxed{T = mg}
                \]
                On a donc la tension égale au poids.
        \end{itemize}
    \item \leftcenters{Dans ce cas, on a explicitement}{$\boxed{T = m(a_v+g)}$}
        \smallbreak
        La tension est supérieure au poids, et fonction affine de $a_v$~: si
        l'accélération est trop forte, le câble peut rompre.
\end{enumerate}
\hspace*{-0.75cm}
\begin{minipage}{0.80\linewidth}
    \begin{enumerate}
        \item
            \begin{enumerate}[leftmargin=20pt]
                \item L'accélération de M est $\af_M = \dv[2]{\OM}{t}$. Or, $\OM =
                    \vec{\rm OA} + \vec{\rm AM}$ avec $\vec{\rm AM}$ constant~:
                    ainsi
                    \[\boxed{\af_M = \dv[2]{\OM}{t} = \dv[2]{\vec{\rm OA}}{t} =
                    \af_h}\]
                \item On a alors le PFD~:
                    \[
                        m\af_h = m\gf + \Tf
                        \Lra
                        ma_h\ux = -mg\uz + T\cos\a\uz + T\sin\a\ux
                    \]
                    \leftcenters{Ainsi,}{
                        $\DS \left\{
                            \begin{aligned}
                                ma_h &= T\sin\a\\
                                mg   &= T\cos\a
                            \end{aligned}
                        \right.
                        \Lra
                        \boxed{
                        \left\{
                            \begin{aligned}
                                \tan\a &= \frac{a_h}{g}\\
                                T      &= m\sqrt{a_h{}^2 + g^2}
                            \end{aligned}
                        \right.}$}
            \end{enumerate}
    \end{enumerate}
\end{minipage}
\hfill
\begin{minipage}{0.15\linewidth}
    \begin{center}
        \includegraphics[width=\linewidth]{grue_corr}
    \end{center}
\end{minipage}

\section{Étude d'un volant de badminton}
\begin{enumerate}
    \item 
        \begin{itemize}[label=$\diamond$, leftmargin=10pt]
            \litem{Système~:} \{volant\} assimilé à un point matériel M de masse
                $m$
            \litem{Référentiel~:} terrestre supposé galiléen
            \litem{Repère~:} $(\Or, \uz)$ avec O départ de chute, $\uz$ vertical
                \textit{descendant} (voir schéma)
            \litem{Repérage~:} $\OM = z(t)\uz$, $\vf = \zp(t)\uz$, $\af =
                \zpp(t)\uz$
            \litem{Origine et instant initial~:} $\OM(0) = z(0)\uz = \of$
        \end{itemize}\smallbreak
        \begin{minipage}{0.70\linewidth}
            \begin{itemize}[label=$\diamond$, leftmargin=10pt]
                \litem{BFD~:}
                    \[
                        \begin{array}{ll}
                            \textbf{Poids} & \Pf = m\gf = mg\uz\\
                            \textbf{Frottements} & \Ff = -\lambda v\vf =
                            -\lb\zp^2\uz
                        \end{array}
                    \]
                \litem{PFD~:}
                    \[m\af = \Pf + \Ff \Lra m\zpp = mg -\lb\zp^2 \Lra \zpp +
                    \frac{\lb}{m}\zp^2 = g\]
            \end{itemize}
        \end{minipage}
        \begin{minipage}{0.25\linewidth}
            \hfill
            \begin{center}
                \includegraphics[width=\linewidth]{volant_corr}
            \end{center}
            \vspace{-30pt}
            \hfill~
        \end{minipage}\smallbreak
        \leftcenters{Ainsi,}{$\DS\boxed{\dv{v}{t} + \frac{\lb}{m}v^2 = g}$}
    \item Lorsqu'on lâche M sans vitesse initiale d'une hauteur $h$, la vitesse
        est faible au départ et la force principale est le poids, accélérant le
        mobile vers le bas. Quand la vitesse augmente, les frottements
        s'intensifient jusqu'à ce qu'ils compensent le poids, donnant $\af =
        \of$~: la vitesse n'évolue plus et reste à sa valeur avant compensation,
        la vitesse limite $v_l$. $v_l$ étant constante, $\vp_l = 0$, donc 
        l'équation différentielle donne
        \[ 
            \frac{\lb}{m}v_l{}^2 = g
            \Lra
            \boxed{v_l = \sqrt{\frac{mg}{\lb}}}
        \]
    \item $v^*$ est le rapport de deux vitesses, donc est forcément sans
        dimension. Ensuite,
        \begin{gather*}
            [\tau] = \left[\frac{v_l}{g}\right] =
                \frac{\si{m.s^{-1}}}{\si{m.s^{-2}}} = \si{s}
            \\
            [L] = [v_l][\tau] = \si{m.s^{-1}}\times\si{s} = \si{m}
        \end{gather*}
        donc $\tau$ est bien un temps et $L$ une longueur~; ce faisant, $t^*$ et
        $z^*$ sont évidemment adimensionnées.
    \item On réécrit l'équation avec $v = v_lv^*$ et $t=\tau t^*$~:
        \begin{gather*}
            \dv{v}{t} + \frac{\lb}{m}v = g
            \Lra
            \dv{(v_lv^*)}{(\tau t^*)} + \frac{\lb}{m} \left(v_lv^*\right)^2 = g
            \Lra
            \frac{v_l}{\tau} \dv{v^*}{t^*} + \frac{\lb v_l{}^2}{m}(v^*)^2 = g
        \end{gather*}
        Or,
        \begin{gather*}
            \frac{v_l}{\tau} = g
            \qet
            \frac{\lb v_l{}^2}{m} = g
            \LRa
            \boxed{\dv{v^*}{t^*} + (v^*)^2 = 1}
        \end{gather*}
    \item Ces courbes montrent que la vitesse augmente pendant 2 à 3$\tau$,
        avant de se stabiliser à $v_l$. Le mouvement est ensuite rectiligne
        uniforme, et $z$ est une fonction affine du temps.
    \item Le courbe représentant $v^*(t^*)$ montre que $v^* = \num{0.95}$ pour
        $t^* = \num{1.8}$. La durée de l'expérience pour arriver à cette valeur
        est donc $\SI{1.8}{\tau}$, et la hauteur $z^*$ à ce temps est $z^* =
        \num{1.2}$, ce qui correspond à $z = \SI{1.2}{L}$~; ainsi
        \[
            \boxed{\Dt = \SI{1.8}{\tau}}
            \qet
            \boxed{h = \SI{1.2}{L}}
        \]
    \item En supposant $v_l$ connue, on a
        \begin{gather*}
            \tau = \frac{v_l}{g}
            \qet
            L = v_l\tau
            \qavec
            \left\{
                \begin{array}{rcl}
                    v_l & = & \SI{25}{km.h^{-1}} = \SI{7.0}{m.s^{-1}}\\
                    g & = & \SI{9.81}{m.s^{-2}}
                \end{array}
            \right.\\
            \AN
            \boxed{\tau = \SI{7.1e-1}{s}}
            \qet
            \boxed{L = \SI{4.9}{m}}
        \end{gather*}
        Ainsi, \vspace*{-24pt}
        \begin{gather*}
            \Dt = \SI{1.8}{\tau}
            \qet
            h = \SI{1.2}{L}
            \\\Ra
            \boxed{\Dt = \SI{1.3}{s}}
            \qet
            \boxed{h = \SI{5.9}{m}}
        \end{gather*}
\end{enumerate}

\section{Étude d'une skieuse}
\begin{enumerate}
    \item
        \begin{minipage}[t]{0.60\linewidth}
            \begin{itemize}[label=$\diamond$, leftmargin=10pt]
                \litem{Système~:} \{skieuse\} assimilée à son centre de gravité
                \litem{Référentiel~:} $\Rc\ind{sol}$ supposé galiléen
                \litem{Repère~:} $(\Or, \ux, \uy)$ (voir schéma)
                \litem{Repérage~:} $\OM = x(t)\ux$~; $\vf = \xp(t)\ux$~; $\af =
                    \xpp(t)\ux$.
                \litem{Origine et instant initial~:} $\OM(0) = \of$
                \litem{Vitesse initiale~:} $\vf(0) = \of$
            \end{itemize}
        \end{minipage}
        \hfill
        \begin{minipage}{0.35\linewidth}
            \begin{center}
                \includegraphics[width=.9\linewidth]{ski_corr}
            \end{center}
            \vspace{-3cm}
        \end{minipage}\smallbreak
        \begin{itemize}[label=$\diamond$, leftmargin=10pt]
                \litem{BDF~:}
                    \[
                        \begin{array}{ll}
                            \textbf{Poids} & m\gf = mg(\sin\a\ux - \cos\a\uy)\\
                            \textbf{Réaction normale} & \Nf = N\uy\\
                            \textbf{Réaction tangentielle} & \Tf = -T\ux =
                            -fN\ux\\
                            \textbf{Frottements} & \Ff = -\lb\vf = -\lb\xp\ux
                        \end{array}
                    \]
                    Comme la skieuse glisse sur la piste, avec les lois du
                    frottement de \textsc{Coulomb}, on a
                    \[T = fN\]
            \litem{PFD~:}
                \[
                    m\af = \Pf + \Nf + \Tf + \Ff
                    \Lra
                    \left\{
                        \begin{aligned}
                            m\xpp &= mg\sin\a - fN - \lb\xp\\
                            m\ypp &= -mg\cos\a + N
                        \end{aligned}
                    \right.
                \]
        \end{itemize}
        Ainsi, comme il n'y a pas de mouvement sur $\uy$, $\ypp = 0$ et
        \[
            \boxed{N = mg\cos\a}
            \Ra
            \boxed{T = fN = fmg\cos\a}
        \]
    \item On réutilise la première équation en y injectant l'expression de $T$
        pour avoir~:
        \[
            \xpp + \frac{\lb}{m}\xp = g(\sin\a - f\cos\a)
        \]
        Avec $\vf = \xp(t)\ux$, on obtient une équation différentielle sur
        $v(t)$ que l'on résout en posant $\tau = m/\lb$ avec la solution
        homogène $A\exr^{-t/\tau}$ et la solution particulière $v_p$~:
        \[
            \vp(t) + \frac{v}{\tau} = g(\sin\a - f\cos\a)
            \Ra
            v = A\exr^{-t/\tau} + v_p
        \]
        et on trouve $v_p$ directement en remarquant que, par construction,
        $\vp_p = 0$ donc $v_p = g\tau(\sin\a - f\cos\a)$. En combinant on peut
        utiliser la condition initiale sur la vitesse~:
        \begin{gather*}
            v(t) = A\exr^{-t/\tau} + g\tau(\sin\a - f\cos\a)
            \shortintertext{Or,}
            \left.
            \begin{aligned}
                v(0) &= 0
                \\\Lra 
                0 &= A + g\tau(\sin\a - f\cos\a) 
                \\\Lra
                A &= -g\tau(\sin\a - f\cos\a)
            \end{aligned}
            \right\}
            \quad\Ra\quad
            \boxed{v(t) = g\tau(\sin\a - f\cos\a)\left(1-\exr^{-t/\tau}\right)}
        \end{gather*}
        On trouve la position $x(t)$ en intégrant $v(t)$~:
        \begin{gather*}
            x(t) = g\tau(\sin\a - f\cos\a)\left(t + \tau\exr^{-t/\tau}\right) +
            B
            \shortintertext{Or,}
            \left.
            \begin{aligned}
                x(0) &= 0
                \\\Lra
                0 &= g\tau(\sin\a - f\cos\a)\left(0 + \tau\right) + B 
                \\\Lra
                B &= -g\tau^2(\sin\a - f\cos\a)
            \end{aligned}
            \right\}
            \quad\Ra\quad
            \boxed{x(t) =
                g\tau(\sin\a-f\cos\a)\left(t+\tau\left(\exr^{-t/\tau}-1\right)\right)}
        \end{gather*}
    \item La vitesse limite est la solution particulière $v_p$~:
        \[\boxed{\vf_l = g\tau(\sin\a-f\cos\a)\ux}\]
        En effet, la présence de la force de frottements fluides dont la norme
        augmente avec la vitesse fait que la vitesse ne peut pas augmenter
        indéfiniment. La skieuse atteint une vitesse limite lorsque les
        frottement compensent la force motrice du mouvement. Ainsi,
        \[
            \boxed{\vf(t) = v_l\left(1-\exr^{-t/\tau}\right)\ux}
            \qet
            \boxed{\OM(t) =
                v_l\left(t+\tau\left(\exr^{-t/\tau}-1\right)\right)\ux}
        \]
    \item 
        \begin{gather*}
            \boxed{v_l = \frac{mg}{\lb}(\sin\a-f\cos\a)}
            \qavec
            \left\{
                \begin{array}{rcl}
                    m   & = & \SI{65}{kg}\\
                    g   & = & \SI{10}{m.s^{-2}}\\
                    \lb & = & \SI{1}{kg.s^{-1}}\\
                    \a  & = & \ang{45}\\
                    f   & = & \num{0.9}
                \end{array}
            \right.\\
            \AN
            \boxed{v_l = \SI{46}{m.s^{-1}}}
        \end{gather*}
        On remarque que la vitesse limite est une fonction affine du poids.
        Ainsi, le manque de représentation des femmes dans les sports d'hiver,
        souvent justifié par une moins bonne performance pure, est biaisé par la
        répartition moyenne de leurs tailles (et donc de leurs poids) plus
        faible que la répartition moyenne des tailles (et donc poids) des
        hommes, rendant \underline{pour certains} leurs records moins
        impressionnants.
    \item 
        \begin{gather*}
            \begin{aligned}
                v(t_1) &= \frac{v_l}{2}
                \\\Lra
                \frac{\cancel{v_l}}{2} &= \cancel{v_l}(1-\exr^{-t_t/\tau})
                \\\Lra
                \frac{1}{2} &= 1-\exr^{-t_1/\tau}
                \\\Lra
                \exr^{-t_1/\tau} &= \frac{1}{2}
            \end{aligned}
            \\\Lra
            \boxed{t_1 = \tau\ln2}
            \qavec
            \tau = \frac{m}{\lb}
            \qet
            \left\{
                \begin{array}{rcl}
                    m & = & \SI{65}{kg}\\
                    \lb & = & \SI{1}{kg.s^{-1}}
                \end{array}
            \right.\\
            \AN
            \boxed{t_1 = \SI{45}{s}}
        \end{gather*}
    \item En tombant à $t=t_1$, la skieuse a pour vitesse $v_l/2$. L'équation du
        mouvement sur $\uy$ ne change pas de forme, mais on multiplie $f$ par
        10, donc $T=10fmg$. Ainsi, en posant $t'=t-t_1$, en projection sur $\ux$
        et en négligeant $\lb$,
        \begin{gather*}
            \xpp(t') = g(\sin\a-10f\cos\a)
            \Ra
            \xp(t') = gt'(\sin\a-10f\cos\a) + v_l/2
        \end{gather*}
        On trouve le temps d'arrêt $t'_a$ quand $\xp(t'_a) = 0$, soit
        \[t'_a = \frac{-v_l}{2g(\sin\a-10f\cos\a)}\]
        et la distance d'arrêt depuis le point de chute en intégrant $\xp(t')$
        puis en prenant $x(t'_a)$~:
        \begin{gather*}
            x(t') = \frac{1}{2}gt'^2(\sin\a-10f\cos\a) + \frac{v_lt'}{2}
            \\\Lra
            \boxed{x(t'_a) = - \frac{v_l{}^2}{8g(\sin\a-10f\cos\a)}}
            \qavec
            \left\{
                \begin{array}{rcl}
                    v_l & = & \SI{46}{m.s^{-1}}\\
                    g   & = & \SI{10}{m.s^{-1}}\\
                    \a  & = & \ang{45}\\
                    f   & = & \num{0.9}
                \end{array}
            \right.\\
            \AN
            \boxed{x(t'_a) = \SI{4.7}{m}}
        \end{gather*}
\end{enumerate}

\end{document}
