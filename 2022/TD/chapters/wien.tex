\documentclass[a4paper, 12pt, final, garamond]{book}
%\usepackage{cours-preambule}
\usepackage{bpep_full}
\usepackage{chngcntr}

\counterwithin*{equation}{section}
\counterwithin*{equation}{subsection}

\raggedbottom

\makeatletter
\renewcommand{\@chapapp}{\'Electrocin\'etique -- chapitre 4}
\makeatother

\begin{document}
\setcounter{chapter}{3}

\toggletrue{corrige}  % décommenter pour passer en mode corrigé

\chapter{\sujetUniquement{TD~: oscillateurs harmonique et amorti}\siCorrige{Correction du TD}}

\resetQ
\section{Circuit de \textsc{Wien}}

\begin{minipage}{0.60\linewidth}
    On réalise le montage suivant. On ferme l'interrupteur à l'instant $t =0$,
    $C$ traversé par $i'$ étant initialement chargé et $C$ traversé par $i$
    étant initialement déchargé.\smallbreak
    On pose $\tau = RC$. Données~: $R = \SI{10}{k\ohm}$ et $C = \SI{0.1}{\micro
    F}$.
\end{minipage}
\begin{minipage}{0.40\linewidth}
    \centering
    \includegraphics[width=\linewidth]{../figures/ch8/wien}
\end{minipage}

\QR{À partir de considérations physiques, préciser les valeurs de la tension $v$
    lorsque $t = 0$ et $t = \infty$.
}{
    Le condensateur de tension $v$ est indiqué être initialement déchargé, on a
    donc $v(0^-) = 0$. Comme un condensateur est de tension continue, on a donc
    \fbox{$v(0^+) = 0$}. De plus, à $t \longrightarrow \infty$, les deux
    condensateurs seront forcément déchargés à cause des résistances dissipant
    l'énergie, il ne peut y avoir conservation~: il seront donc équivalent à des
    interrupteurs ouverts, et on aura donc notamment \fbox{$v(\infty) = 0$}.
}

\QR{
    Établir l'équation différentielle du second ordre dont la tension $v$ est
    solution.
}{
    Avec une loi des mailles, on a
    \[ u = v+Ri'\]
    Or, la RCT du condensateur de gauche \textbf{en convention générateur} est
    \[i' = -C \dv{u}{t} \Rightarrow i' = -C \dv{v}{t} - RC \dv{i'}{t}\]
    On a donc une équation avec $\dv{v}{t}$. On cherche donc à exprimer $i'$ en
    fonction de $v$, ce que l'on fait avec la loi des nœuds et les RCT du
    condensateur de droite $i = C \dv{v}{t}$ et de la résistance $R(i'-i) = v$~:
    \begin{equation}\label{eq:wieni}
        i' = i + \frac{v}{R} \Leftrightarrow i' = C \dv{v}{t} + \frac{v}{R}
    \end{equation}
    En combinant les deux, on a
    \begin{gather*}
        C \dv{v}{t} + \frac{v}{R} =
            -C \dv{v}{t} - RC \dv{}{t} \left( C \dv{v}{t} + \frac{v}{R} \right)
        \Leftrightarrow
        C \dv{v}{t} + \frac{v}{R} =
            -C \dv{v}{t} - RC^2 \dv[2]{v}{t} - C \dv{v}{t}\\
        \Leftrightarrow
        \dv[2]{v}{t} + \frac{3}{RC} \dv{v}{t} + \frac{v}{(RC)^2} = 0
        \Leftrightarrow
        \boxed{\dv[2]{v}{t} + \frac{3}{\tau} \dv{v}{t} + \frac{v}{\tau^2} = 0}
    \end{gather*}
}

\QR{
    En déduire l'expression de $v(t)$ sans chercher à déterminer les
    constantes d'intégration.
    
}{
    On écrit l'équation caractéristique de discriminant $\Delta$~:
    \begin{gather*}
        r^2 + \frac{3}{\tau}r + \frac{1}{\tau^2} = 0 \Rightarrow \Delta =
        \frac{9}{\tau^2} - \frac{4}{\tau^2} = \frac{5}{\tau^2} > 0\\
        \Longrightarrow r_\pm = - \frac{3}{2\tau} \pm \frac{\sqrt{5}}{2\tau} < 0
    \end{gather*}
    On a donc un régime apériodique, dont les solutions générales sont
    \[\boxed{v(t) = A\exr^{r_+t} + B\exr^{r_-t}}\]
}

\QR{Donner l'allure du graphe correspondant à $v(t)$. 
}{
    \begin{minipage}{0.49\linewidth}

        Le condensateur est initialement chargé. Soit $E$ sa tension initiale.
        On utilise l'équation~\ref{eq:wieni} pour trouver que $\dv{v}{t} (0) =
        \frac{i' (0)}{C}$, sachant qu'à $t = 0$ le circuit est équivalent à un
        circuit $RC$ en décharge et qu'on a donc $i'(0) = E/R$. On trouve ainsi
        \[\boxed{\dv{v}{t} (0) = \frac{E}{\tau}}\]
        En finissant la détermination des constantes d'intégration, on trouve
        \begin{equation*}
            \boxed{v(t) = \frac{E}{\tau(r_+ - r_-)}
                          \left[ \exr^{r_+t} - \exr^{r_-t} \right]}
        \end{equation*}
        \hfill
    \end{minipage}
    \begin{minipage}{0.49\linewidth}
        \begin{center}
            \includegraphics[width=\linewidth]{../figures/ch8/wien_carac}
        \end{center}
    \end{minipage}
}

\end{document}
