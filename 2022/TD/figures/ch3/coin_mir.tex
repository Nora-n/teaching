\documentclass{standalone}
\usepackage{mintikz}

\begin{document}
\begin{tikzpicture}[use optics]
    % Miroir
    \node[mirror, rotate=-90,
    name path=Mpath1]
    (M1) at (0,0) {};
    \node[below] at (M1.center) {$M_1$};
    % Miroir
    \def\angg{45}
    \node[mirror, rotate=\angg,
    name path=Mpath2, anchor=south]
    (M2) at (M1.north) {};
    \node[above right] at (M2.center) {$M_2$};
    % Points
    \node[below] at (M1.south) {$A$};
    \coordinate (A) at (M1.south);
    \node[below] at (M1.north) {$B$};
    \coordinate (B) at (M1.north);
    \node[above left] at (M2.north) {$C$};
    \coordinate (C) at (M2.north);
    % Rayon 1
    \coordinate (O) at (-2,1);
    \path[name path=AB] (O) --++ (3,0);
    % Define intersection
    \path[name intersections={of=AB and Mpath2, by=I}];
    \node[] at (I) {$\times$};
    \node[above] at (I) {I};
    \draw[brandeisblue, simple] (O) -- (I);
    % Perpendicular
    \draw[dashed, gray] (I)
        -- ($(I)!1cm!-90:(B)$)
        coordinate (Idown);
    \path[name path=Jpath] (I) --++ ({-90}:1.2cm);
    % Define intersection
    \path[name intersections={of=Jpath and Mpath1, by=J}];
    \node[] at (J) {$\times$};
    \node[above left] at (J) {J};
    \draw[ForestGreen, simple] (I) -- (J);
    % Perpendicular
    \draw[dashed, gray] (J) --++ (0,1cm);
    \draw[transform canvas={xshift=.5mm}, orange!70, simple] (J) -- (I);
    \draw[transform canvas={yshift=.5mm}, Red!70, simple] (I) -- (O);
    % Draw angles
    \pic[angle radius=.5cm, angle eccentricity=1.3,
    draw, <-, black,
    "$\alpha$" alias=alpha]
    {angle=C--B--A};
    % Draw angles
    \pic[angle radius=.5cm, angle eccentricity=1.3,
    draw, <-, black,
    "$\alpha$" alias=alpha]
    {angle=C--I--O};
    % Draw angles
    \pic[angle radius=.5cm, angle eccentricity=1.3,
    draw, <-, brandeisblue,
    "$i_1$" alias=i1]
    {angle=O--I--Idown};
    % Draw angles
    \pic[angle radius=.4cm, angle eccentricity=1.3,
    draw, ->, ForestGreen,
    "$i_2$" alias=i2]
    {angle=Idown--I--J};
    
\end{tikzpicture}
\end{document}
