\documentclass[a4paper, 12pt, final, garamond]{book}
\usepackage{cours-preambule}

\raggedbottom

\makeatletter
\renewcommand{\@chapapp}{Travaux pratiques -- TP}
\makeatother

\let\SavedIndent\indent
\protected\def\indent{%
  \begingroup
    \parindent=\the\parindent
    \SavedIndent
  \endgroup
}
\setlength{\parindent}{0pt}

\begin{document}
\setcounter{chapter}{-1}

\chapter{Utiliser la calculatrice pour analyser des donn\'ees}

\section{Objectifs}

\begin{itemize}
    \item Utiliser les fonctions usuelles de sa calculatrice~;
    \item Comprendre l'intérêt et savoir mettre en place une régression linéaire
        pour vérifier un modèle.
\end{itemize}

\section{Fonctions de base}

\subsection{Calcul num\'erique}

1. Calculer les fractions suivantes à l'aide de la calculatrice (attention à vos
parenthèses)~: 

\[\frac{7}{3\times 5}+4 \quad\text{et}\quad \frac{2}{9\times8 + 5}\]

\vspace{2cm}

2. Le rayon de Bohr $a_0$ (caractéristique de la taille d'un atome) est donné
par la formule~: 

\[a_0 = \frac{4\pi \epsilon_0 \hbar^2}{m_ee^2}\] 

avec $e = \SI{1,60e19}{C}$, $1/(4\pi\epsilon_0) = \SI{9,00e9}{USI}$,
$\hbar = \SI{1.05e-34}{J.s{^-1}}$ et $m_e = \SI{9,11e-31}{kg}$. Déterminer
numériquement la valeur de $a_0$.

\vspace{2cm}

\subsection{Utilisation des angles}

Lorsque l'on travaille avec les fonctions trigonométriques (cos, sin, tan...),
il faut être très vigilant-e à l'unité des angles utilisée par votre
calculatrice. 


\underline{Remarque} : Dans la suite, on distinguera les degrés par \si{\degree}
et les radians par \si{rad}, mais ça ne sera pas à écrire dans la calculatrice~!

\begin{instruc}[]{Casio}
    Dans le mode \texttt{RUN}, on choisit entre degrés et radians en allant dans
    le mode \texttt{SET UP}.
\end{instruc}

\begin{instruc}{TI}
    Dans le menu \texttt{MODE}, on choisit entre degrés et radians en allant sur
    la troisième ligne.
\end{instruc}

1. Mettez-vous en radians. Vérifier alors que

\[\cos(\SI{\pi}{rad}) = -1 \quad\text{et}\quad \cos(\ang{180}) \approx
-\num{0,5985}\]

\medskip

2. Mettez-vous en degrés. Vérifier alors que

\[\cos(\SI{\pi}{rad}) \approx \num{0,9985} \quad\text{et}\quad
\cos(\ang{180}) =-1\]

\medskip

3. Faire les applications numériques suivantes~:

\[\tan(\SI{2}{rad}) \quad\text{et}\quad \cos^2\left(\frac{\pi}{3}\,\si{rad}\right)
-\sin(\ang{10;;}) \]

\vspace{2cm}

\leftcenters{4. Calculer $n$ pour }{$\DS n =
\frac{\sin\left(\frac{\DS D_m+A}{2}\right)}{\sin\left(\frac{A}{2}\right)}$}

avec $D_m = \ang{54.85}$ et $A = \pi/3 \si{rad}$.

\subsection{R\'esolution des équations d'ordre 2}

\begin{instruc}{Casio}
    
    Dans le mode \texttt{EQUA}, on sélectionne le type de l'équation à résoudre.
    Il y a~: 
    
    \begin{itemize}
        \item \texttt{SIML} (bouton \texttt{F1}) pour résoudre un système
            d'équations à plusieurs inconnues~;
        \item \texttt{POLY} (bouton \texttt{F2}) pour résoudre une équation
            polynômiale~;
        \item \texttt{SOLV} (bouton \texttt{F3}) pour résoudre une équation plus
            complexe.
    \end{itemize}

Choisir ici le mode \texttt{POLY}. On peut ensuite choisir le degré de
l'équation (2 ou 3). Dans notre cas, choisir 2. On est alors invité-e à rentrer
les coefficients $a$, $b$ et $c$ de l'équation $ax^2+bx+c=0$. Une fois cela
fait, presser \texttt{SOLV} (bouton \texttt{F1}). On obtient alors les deux
solutions de l'équation. 
\end{instruc}

\begin{instruc}{TI}
    Aller dans \texttt{Apps}. Choisir \texttt{PlySmlt2} (bouton 4) puis
    \texttt{Poly Root Finder} (bouton 1). Choisir ensuite l'ordre 2 et presser
    \texttt{ENTER} et \texttt{NEXT}. L'équation s'affiche alors sous la forme
    $a2*x^2+a1*x+a0=0$. Renseigner alors les coefficients $a2$, $a1$ et $a0$.
    Presser \texttt{SOLVE} pour obtenir les deux solutions de l'équation.
\end{instruc}

Donner les solutions des équations suivantes~: 

\[2x^2+3 = 0 \qquad ; \qquad 3x^2 = 2x+1 \qquad;\qquad x^2+x+1 = 0\]

\vspace{2cm}

Pour résoudre une équation aux racines complexes~:

\begin{instruc}{Casio}
    Dans le menu polynômial \texttt{POLY}, degré $? \rightarrow
    2$ puis $\texttt{shift} \rightarrow \texttt{setup} \rightarrow
    \texttt{complex mode a+ib}$.
\end{instruc}

\begin{instruc}{TI}
    Dans le menu où l'on choisit l'ordre, sélectionner $\texttt{a+ib}$.
\end{instruc}

\section{Régression linéaire}

\underline{Note} : Pour cette partie, s'aider de la fiche pratique «~régression
linéaire~».

\subsection{Exemple 1 : la loi d'Ohm}

La tension est l'intensité est mesurée au travers d'une résistance de $R=
\SI{1}{k\Omega}$ d'après le constructeur. 

\centers{
\begin{tabular}{| c || c | c | c | c | c | c |}
    \hline

    $I$ (en A) & \num{0,010} & \num{0,020} & \num{0,030} & \num{0,040} &
    \num{0,050} & \num{0,060} \\

    $U$ (en V) & \num{10,1} & \num{20,0} & \num{29,8} & \num{40,2} &
    \num{50,0} & \num{60,1}\\

    \hline
 \end{tabular}}
 
1. Tracer le graphe de la tension en fonction de l'intensité sur votre
calculatrice. 

\medskip

2. Réaliser la régression linéaire. Relever les valeurs des coefficients de
régression $a$ et $b$ ainsi que le coefficient de corrélation linéaire
$r$ et le coefficient de détermination $r^2$.

\vspace{2cm}

3. La loi d'Ohm est-elle vérifiée ? Vérifier la valeur de la résistance. 

\vspace{3cm}

4. Conclure quant à la valeur constructeur. Quel est l'écart relatif entre la
valeur constructeur et la valeur expérimentale de $R$ ?

\vspace{3cm}


\subsection{Exemple 2 : Cinétique chimique}

\subsubsection{Position du problème}

Les relations que l'on étudie en science ne sont pas toujours linéaires.
Pourtant, il est tout de même possible d'exploiter la méthode de
régression linéaire pour s'assurer de la validité du modèle et
déterminer des coefficients numériques inconnus. À titre d'exemple, on va
étudier l'évolution de la concentration $c(t)$ d'une espèce chimique en
solution lors d'une réaction chimique. La forme de cette évolution peut
être de deux types :

\begin{enumerate}
\item Avec une cinétique d'ordre 1, la concentration évolue selon

\[c(t) = c_0 e^{-kt}\]

\item Avec une cinétique d'ordre 2, la concentration évolue selon

\[\frac{1}{c(t)} = \frac{1}{c_0} + kt\]
\end{enumerate}

Ces deux modèles sont non linéaires. Afin de vérifier que les
données expérimentales valident (ou non) l'un de ces modèles à
l'aide d'une régression linéaire, il convient de «~linéariser~»
(rendre linéaire) les données. 

\begin{enumerate}
\item Dans le cas d'une cinétique d'ordre 1, on remarque que

\[\ln(c(t)) = \ln(c_0)-kt\]

Ainsi, si ce modèle est vérifié le tracé de $\ln(c(t))$ en fonction
de $t$ doit aboutir à une droite de coefficient directeur $a=-k$ et
d'ordonnée à l'origine $b = \ln(c_0)$.
 
 \item Dans le cas d'une cinétique d'ordre 2, on remarque que, si ce modèle
     est vérifié le tracé de $1/c(t)$ en fonction de $t$ doit aboutir
     à une droite de coefficient directeur $a=k$ et d'ordonnée à
     l'origine $b = 1/c_0$.
\end{enumerate}

\underline{Remarque} : Il est possible d'effectuer directement des opérations
sur les listes avec les calculatrices (voir la fiche pratique «~régression
linéaire~») pour avoir de l'aide. 

\subsubsection{Application}
La concentration $c(t)$ d'une espèce chimique est mesurée dans la solution au
cours du temps. On obtient les données suivantes : 

\centers{
\begin{tabular}{| c || c | c | c | c | c | c |}
\hline
   $t$ (en s) & 20 & 40 & 60 & 80 & 100 & 120 \\
   $c$ (en $\si{\micro mol.L^{-1}}$) & 278 & 192 & 147 & 119 & 100 & 86 \\
   \hline
 \end{tabular}}
 
1. Réaliser les régressions linéaires suivantes et donner l'équation de la
droite (coefficients $a$ et $b$) ainsi que la valeur des coefficients de
corrélation : 

\begin{enumerate}
\item $c$ en fonction de $t$ ;
\item $\ln(c)$ en fonction de $t$ ;
\item $1/c$ en fonction de $t$.
\end{enumerate}

Avec quel modèle les résultats expérimentaux s'accordent-ils le mieux ? Conclure.

\vfill

Résultats attendus :

\begin{instruc}[tikz={rotate=180, transform shape}]{Aide}
    \begin{enumerate}
        \item $c = f(t)$ : $c = -\num{1,8}.10^{-6} t + \num{0,0003}$; $r^2 =
            \num{0,89}$ et $r = - \num{0,94}$.
        \item $\ln(c) = f(t)$ : $\ln(c) = -\num{0,0115} t - \num{8,0597}$; $r^2 =
            \num{0,97}$ et $r = - \num{0,98}$.
        \item $1/c = f(t)$ : $1/c = \num{80,185} t + \num{1993,6}$; $r^2 =
            \num{0,999991}$ et $r = - \num{0,999995}$.
    \end{enumerate}
\end{instruc}

\end{document}
