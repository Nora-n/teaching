\documentclass[a4paper, 11pt, final, garamond]{book}
\usepackage{cours-preambule}
\usepackage{pdfpages}

\raggedbottom

\makeatletter
\renewcommand{\@chapapp}{Travaux pratiques -- TP}
\makeatother

\let\SavedIndent\indent
\protected\def\indent{%
  \begingroup
    \parindent=\the\parindent
    \SavedIndent
  \endgroup
}
\setlength{\parindent}{0pt}

\begin{document}
\setcounter{chapter}{15}

\chapter{Ondes ultrasonores~: mesure de caract\'eristiques}

\section{Objectifs}

\begin{itemize}
    \item Se familiariser avec les logiciels \texttt{Oscillo 5} et
        \texttt{LatisPro}.
    \item Utiliser l'interface Sysam.
    \item Mesurer la fréquence et la longueur d'onde pour une onde ultrasonore
    \item Mesurer la vitesse de propagation des ultrasons dans l'air à la
        température de la salle.
    \item Reconnaître une avance ou un retard entre deux courbes visualisées sur
        un oscilloscope.
    \item Repérer le passage par un déphasage $0$ ou $\pi$ en mode \texttt{XY}.
    \item Évaluer une incertitude de type A.
    \item Simuler un processus aléatoire de variation des valeurs 
        expérimentales de l'une des grandeurs – simulation \textsc{Monte-Carlo}
        – pour évaluer l'incertitude sur les paramètres du modèle. 
\end{itemize}

\section{S'approprier~: Outils théoriques et matériel disponible}

Un émetteur d'ultrasons émet des ondes ultrasonores (fréquence supérieure à
$\SI{20}{kHz}$ donc non audibles) en continu ou en salves (appelées trains
d'ondes). Les ondes ultrasonores sont des ondes mécaniques, longitudinales
(lorsqu'elles se propagent dans les fluides), de compression-dilatation à trois
dimensions. \bigbreak

\begin{enumerate}[label=\clenumi]
    \item Rappeler la différence entre ondes longitudinales et ondes
        transversales.
\end{enumerate} \bigbreak

Selon les expériences, on disposera en plus d'un ou de deux récepteurs adaptés.
On observe les ondes qui sont émises par l'émetteur et celles qui sont
éventuellement reçues par le ou les récepteur(s) sur l'écran du logiciel
\texttt{Oscillo 5}.

\section{Mesure de la période $T$ des ondes}

\subsection{Réaliser}
\subsubsection{Connexion à l'interface Sysam}

\begin{itemize}
    \item Relier l'émetteur d'ultrasons à la sortie analogique SA1 de
        l'interface Sysam et les masses des appareils entre elles. Cette sorte
        remplace un GBF et alimente le générateur d'ultrasons.
    \item Relier l'émetteur d'ultrasons à la voie EA0 du canal 0 de
        l'interface pour visualiser le signal de l'émetteur sur \texttt{Oscillo
        5} et les masses des appareils entre elles.
    \item Allumer l'ordinateur sur votre session.
\end{itemize}

\subsubsection{Démarrer Oscillo 5 et régler le GBF}
\begin{itemize}
    \item Ouvrir Oscillo5 (Programmes Physique-chimie $\rightarrow$ Eurosmart
        $\rightarrow$ Oscillo5)~;
    \item Cliquer sur \texttt{Voir GBF 1} dans le panneau de contrôle (permet
        d'accéder à un menu de réglage du GBF).
    \item Cliquer sur \texttt{marche} du bouton marche/arrêt.
    \item Sélectionner \texttt{sinusoïde}.
    \item Régler la fréquence à \SI{40}{kHz} avec les curseurs et l'amplitude à
        \SI{10}{V}.
    \item Les réglages sont terminés. Vous pouvez cacher le panneau de contrôle
        du GBF.
\end{itemize}

\subsubsection{Visualiser la voie EA0 et vérifier la fréquence}
\begin{itemize}
    \item Activer la voie.
    \item Choisir la base de temps (menu balayage) et l'amplitude du signal de
        façon à pouvoir mesurer la période du signal.
    \item En utilisant les curseurs, dans le menu mesures en bas à droite,
        déplacer les curseurs pour mesurer de nouveau la période.
    \item En déduire la fréquence et la comparer à celle du constructeur~:
        \SI{40}{kHz}.
\end{itemize}

\subsection{Valider}
\begin{enumerate}[label=\sqenumi, resume]
    \item Mettre en commun vos résultats de mesure de $T$ entre les différents
        groupes. Vous ferez un tableau sur \texttt{LatisPro}, recopié sur vos
        comptes-rendus. En déduire le résultat du mesurage de $T$ en calculant
        l'incertitude de type A par la méthode de votre choix (code
        \texttt{Python}, tableau \texttt{LatisPro}…).
\end{enumerate}

\begin{timpo}{Rappel~: incertitude de type A}
    Considérons la grandeur physique $x$. L'évaluation de l'incertitude de type
    A permet de donner un résultat de mesurage $M(x)$ par une \textbf{analyse
    statistique} réalisées sur plusieurs mesures $m_i(x)$ réalisées dans des
    \textbf{conditions de répétabilité} (les grandeurs d'influence n'ont pas
    changé). L'avantage est qu'il n'est \textbf{pas nécessaire de connaître
    les caractéristiques techniques des dispositifs de mesure employés}. Pour un
    ensemble $m_i(x)$ de $n$ mesures de $x$, 
    \[
        \boxed{M(x) = \obar{m}(x) \pm \frac{1}{\sqrt{n}} \sigma(x)}
    \]
    % Avec\vspace*{-24pt}
    \[
        \obar{m}(x) = \frac{1}{n} \, \sum_{i=1}^{n} m_i(x)
        \qqet
        \sigma(x) = \sqrt{\frac{1}{n-1} \, \sum_{i=1}^{n} (m_i(x)-\obar{m}(x))^2}
    \]
\end{timpo}


\section{Mesure de la vitesse de propagation $c$}

\subsection{Analyser~: proposer un protocole}

Vous disposez d'un émetteur, d'un récepteur, d'une règle graduée au millimètre.
\bigbreak

\begin{enumerate}[label=\clenumi, resume]
    \item Proposer un protocole expérimental vous permettant de mesurer la
        vitesse de propagation des ultrasons dans l'air. Pour plus de précision,
        vous prendrez plusieurs mesures vous permettant de réaliser une
        régression linéaire. 
\end{enumerate}

\begin{instruc}{Aide}
    Comment allez-vous pouvoir procéder à votre mesure en mode continu~? Ne
    serait-il pas plus pertinent d'utiliser le mode salve et pourquoi~? 
\end{instruc}

\bigskip

\begin{itemize}
    \item Appeler læ professeurx à cette étape avec le protocole rédigé pour
        vérification. 
\end{itemize}

\subsection{Réaliser}

Réaliser le protocole précédemment proposé. Quelques indications~: 

\begin{itemize}
    \item Même montage que précédemment
    \item Relier en plus le récepteur sur la voie EA1 de l'interface (et laisser
        l'émetteur sur EA0).
    \item Accéder au menu GBF sur \texttt{Oscillo 5}
    \item Sélectionner \texttt{salve}. Laisser $\SI{10}{ms}$ de durée et
        $\SI{10}{V}$ d'amplitude. 
    \item Les nouveaux réglages sont terminés. Cacher le panneau GBF.
    \item Visualiser les deux voies EA0 et EA1. Synchro auto et préacquisition
        $10 \%$.
    \item Cliquer sur \texttt{monocoup} pour réaliser l'acquisition.
    \item La fonction curseur permet de réaliser la mesure souhaitée. 
\end{itemize}

\subsection{Valider}

\begin{enumerate}[label=\clenumi, resume]
    \item Proposer (en vous aidant de la fiche pratique «~Régression linéaire~»)
        un script \texttt{Python} permettant de réaliser la régression linéaire
        dont $c$ est le coefficient directeur. \bigbreak
\end{enumerate}\vspace{-12pt}
\begin{enumerate}[label=\sqenumi, resume]
    \item Évaluer l'incertitude-type sur la mesure de $c$ par méthode
        \textsc{Monte-Carlo}, après avoir préalablement proposé une
        incertitude-type (de type B) sur vos mesures de temps et de distance.
    \item Imprimer \textbf{en noir et blanc} le graphique de la régression
        linéaire. Rappel~: un graphique comporte des axes nommés, des points de
        mesures clairement visibles, et on fera apparaître les valeurs de la
        régression linéaire clairement sur le graphique.
    \item Comparer la valeur de célérité des ondes ultrasonores dans l'air avec
        la formule empirique donnant $c$ dans l'air en fonction de la
        température
        \footnote{\url{https://hypertextbook.com/facts/2000/CheukWong.shtml}}~:
        \[
            c = \num{331,5} + \num{0,60} \, \theta
        \]
        avec $c$ la vitesse en $\si{m.s^{-1}}$ et $\theta$ la température de l'air en
        $\degreeCelsius$. \bigbreak
        Vous utiliserez pour cela l'écart \textbf{normalisé} $E_N$, tel que
        \[E_N = \frac{\abs{m_1-m_2}}{\sqrt{u(m_1)^2 + u(m_2)^2}}\]
        avec $m_1$ et $m_2$ les valeurs de $c$ obtenues, et $u(m_1)$ et $u(m_2)$
        leurs incertitudes-types. Les valeurs sont dites \textbf{compatibles} si
        $E_N \lesssim 2$.
\end{enumerate}

\section{Détermination de la longueur d'onde $\lambda$}

Faites cette partie si vous avez pu terminer \textbf{proprement} les deux mesures précédentes. 

\subsection{Réaliser}

\begin{itemize}
    \item L'émetteur d'ultrasons émet maintenant en direction des deux
        récepteurs adaptés situés côte à côte de chaque côté de la règle.
    \item On observe sur l'écran de l'oscilloscope deux signaux correspondants
        aux ondes reçues par chacun des deux récepteurs. Positionner de nouveau
        le générateur en position continue sinusoïde comme au III.
    \item Relier les deux récepteurs aux voies EA1 et EA2 de l'interface Sysam.
    \item Superposer la ligne de «~zéro~» de chacune des deux voies. 
    \item Régler correctement les sensibilités horizontales et verticales
        permettant d'observer les sinusoïdes des voies EA1 et EA2.
    \item[\fbox{8}] Mesurer la plus petite distance $d$ entre récepteurs pour
        laquelle les deux courbes se retrouvent en phase. La mesure de $d$
        est-elle précise~? Quel est le lien entre $d$ et $\lambda$~?
    \item Vérifier que les deux courbes se retrouvent en phase pour une distance
        entre les récepteurs égale à un multiple entier de d.
    \item[\fbox{9}] Mesurer la distance entre les deux récepteurs correspondant
        à $10$ longueurs d'onde et en déduire une valeur de $\lambda$.
    \item[\fbox{10}] Dans le menu horizontal d'\texttt{Oscillo 5}, sélectionner
        \texttt{X-Y}. Observer l'allure des courbes quand elles sont en phase.
        Que remarquez-vous~? Cette méthode est-elle plus précise que la
        précédente~? Quelle est l'allure en mode \texttt{X-Y} lorsque les
        courbes sont en phase, en opposition de phase ou en quadrature de
        phase~? Comment différencier des signaux en phase et des signaux en
        opposition de phase~?
\end{itemize}

\subsection{Valider}

\begin{enumerate}[label=\sqenumi, start=11]
    \item Vos trois mesures indépendantes de $\lambda$, $c$ et $T$
        respectent-elles la relation~: 
        \[
            c = \frac{\lambda}{T}
        \]
\end{enumerate}

\end{document}
