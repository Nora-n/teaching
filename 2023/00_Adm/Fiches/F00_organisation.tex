\documentclass[a4paper, 10pt, garamond]{book}
\usepackage{cours-preambule}

\makeatletter
\renewcommand{\@chapapp}{Fiches -- numéro}
\makeatother

\begin{document}
\setcounter{chapter}{-1}

\chapter{Bienvenue en classes pr\'eparatoires}

\section{Objectif}

\begin{tcbraster}[raster columns=4, raster equal height=rows]
	\begin{tcb}[width=.5\linewidth, cnt](impo){Important}
		\textbf{Le retard pris en MPSI ne se rattrape pas~!}
	\end{tcb}
	\begin{tcb}*[raster multicolumn=3](prop)<non>{}
		La formation que vous allez suivre vous prépare aux concours d'entrée
		aux grandes écoles d'ingénieurs, qui se divisent entre écrits et oraux.
		Les \underline{écrits} seront fin \textbf{avril 2025}, donc dans
		\underline{moins de deux ans}. Les épreuves durent de 2h à 6h selon les
		écoles visées, et peuvent être avec ou \textit{sans} calculatrice~; les
		oraux se déroulent entre mi-juin et fin juillet selon les écoles. Ces
		épreuves \textbf{portent sur les deux années} de CPGE, de un tiers à la
		moitié sur la première~; sachant que la seconde est courte (les cours se
		terminent en avril), il vous faut vite intégrer la chose suivante~: dès
		aujourd'hui nous travaillons pour avril 2025.
	\end{tcb}
\end{tcbraster}

Le premier semestre s'attarde à reprendre les bases de sujets déjà connus en les
renforçant puis en les prolongeant, le second à introduire de nouveaux concepts
plus approfondis. La chimie fait partie \textbf{intégrante} de cet
enseignement~: si vous êtes venu-es en MPSI pour éviter la chimie, en plus
d'être un non-sens scientifique c'est une réalité à laquelle vous ne pourrez pas
échapper~: les sciences naturelles sont profondément physico-\textit{chimiques}.

\section{Organisation}
\begin{tcb}*[cnt](defi)"info"{Documents}
	\bfseries
	Tous les documents et toutes les informations importantes seront mises en
	ligne sur le site de la
	classe\footnote{\url{https://cahier-de-prepa.fr/mpsi3-pothier/}}.
\end{tcb}

\subsection{Matériel}
\noindent
\begin{minipage}[t]{.48\linewidth}
	\begin{itemize}[label=$\diamond$, leftmargin=10pt]
		\item Trieur 12 emplacements~;
		\item Pochettes cartonnées ou plastifiées~;
		\item Bloc-notes avec feuilles détachables (recommandé)~;
		      \bitem{Calculatrice scientifique} avec piles de rechange~;
	\end{itemize}
\end{minipage}
\hfill
\begin{minipage}[t]{.48\linewidth}
	\begin{itemize}[label=$\diamond$, leftmargin=10pt]
		\bitem{Règle}~;
		\item Agrafeuse (recommandé)~;
		\item Copies doubles \textbf{petits carreaux} pour les évaluations~;
		\item 1 cahier 90+ pages pour les TPs.
	\end{itemize}
\end{minipage}

\subsection{Cours}
La semaine de cours se découpe en 2x2h+1h. Ce sont des moments importants
d'interaction avec moi~; notamment à chaque début de séance vous aurez un temps
pour me poser vos questions. Il faut donc y participer \ul{activement} et ne pas
juste faire acte de présence.

Il est attendu que vous preniez en note les cours, mais \ul{dans la mesure du
	possible}, je vous donnerai des polycopiés à compléter pour alléger la surcharge
cognitive. \textbf{Ce ne seront en aucun cas vos seules notes}~: il est
\textit{obligatoire} que vous ayez des feuilles de notes \textit{personnelles}
pour noter tout ce que je dis à l'oral, que vous notiez les questions qui vous
passent par la tête, pour faire des calculs au brouillon avant que je ne donne
les démonstrations.

% Je vérifierai vos classeurs de temps en temps.

% Dans la mesure du possible, on prendra à chaque fois un peu de temps à la fin de
% la séance pour des exercices de travaux dirigés.

\subsection{Travaux dirigés (TD)}
Il n'y a officiellement qu'une heure de TD en demi-groupe par semaine. C'est
bien trop peu pour s'entraîner en direct~: les sujets de TDs seront mis en ligne
bien en avance pour que vous les traitiez en amont de la séance. \textbf{Ceci
	est indispensable}. On profitera de la courte heure du mardi midi pour faire les
exercices les plus simples, et les séances de TD à proprement parler seront
dédiées aux exercices plus difficiles.

% Je vérifierai également vos TDs.

\subsection{Travaux pratiques (TP)}
Vous avez une séance de 2h de TP par semaine en demi-classe. C'est \textbf{le
	moment de mise en pratique} et est essentiel à votre formation~: avant d'être
sur des pages de calculs, les sciences physiques sont dans la nature. Ce sont
d'autres moments pour augmenter votre compréhension des notions abordées par la
manipulation. Ici aussi, les sujets des TPs seront donnés en avance, et devront
être \textbf{lus, analysés et traités} avant la séance. En aucun cas vous ne
rentrerez dans la salle de TP sans connaître le contenu de la manipulation.
\textbf{Il y a une épreuve pratique aux concours}, n'espérer pas y échapper.

Pour tirer partie de ces séances, vous devrez vous munir d'un \textbf{cahier de
	manipulations}, l'outil de base de tout-e expérimentateur-ice. Il vous servira à
renseigner vos recherches, les valeurs des paramètres testés et relevés, ainsi
que toutes vos remarques et astuces personnelles concernant les manipulations.
Certains TPs ne sont fait qu'en début de première année, mais tombent pourtant
au concours~: il faut vous y préparer, vous ne retiendrez pas magiquement toutes
les pratiques.

% Comme pour les précédents items, je vérifierai vos cahiers.

\subsection{Évaluations}
Le cursus de CPGE est très, très encadré, notamment \textit{via} des
évaluations. Il y en a de plusieurs types~:
\begin{itemize}[label=$\diamond$, leftmargin=10pt]
	\bitem{Diagnostique}~: elles ne servent pas à vous mettre une note juste pour
	vous mettre une note, mais pour que vous \textit{et moi} puissions réagir à
	votre attention, votre compréhension et éventuellement revenir sur des
	notions.
	\bitem{Formative}~: elles ont pour but d'établir un échange avec vous, de vous
	faire plus activement participer au processus scientifique et visent
	l'apprentissage.
	\bitem{Sommative}~: on y teste la qualité globale de vos connaissances, de
	votre compréhension et de vos pratiques sur une plus grande échelle.
\end{itemize}

\subsubsection{Interrogations écrites}
Il y aura chaque lundi une interrogation d'une dizaine ou quinzaine de minutes,
sur un polycopié, dont on fera la correction en suivant. Il vous faudra être
\textbf{en avance} pour ces interrogations étant donné que le temps nous est
très précieux~: le chronomètre débutera à la sonnerie. Elles seront toutes
notées, avec un coefficient adapté.

% \begin{tcb}*(prop)"divi"{Notation~: interrogations}
% 	Les interrogations sont notées sur 10, et leur moyenne au semestre est
% 	coefficient 1.
% \end{tcb}

\subsubsection{Interrogations orales}
Vous aurez \textbf{toutes les deux semaines} une interrogation orale d'une
heure, qu'on appelle «~colle~» (ou «~khôlle~» dans une orthographe
alternative\footnote{À l'origine, les citoyens appelés au service militaire qui
	avaient les genoux cagneux (déviation osseuse) étaient envoyés à la
	bureaucratie~; le terme s'est détourné pour désigner les universitaires, et pour
	s'en jouer ont utiliser une orthographe nullement compliquée donnant
	«~khâgneux~».}) en \textbf{physique-chimie}. Elles se déroulent en trinôme
devant un-e professeur-e. Elles sont également notées, mais ça n'est pas le cœur
de cette pratique~: ce sont là aussi des moments privilégiés avec une personne
de connaissance en effectif réduit, pour quoi vous puissiez travailler vos
connaissances, votre intuition et vos réflexes. \textbf{Le but premier d'une
	khôlle est que vous ayez appris quelque chose à la fin}.

Vous y serez testé-es sur une question de cours parmi une liste de 10
questions et sur un ou plusieurs exercices.
% \begin{tcb}*(impo)"rema"{Notation}
\begin{tcb}*(prop)"divi"{Notation~: khôlles}
	Elles sont notées sur 20, et leur moyenne au semestre est coefficient 1.
	\smallbreak
	Une question de cours \textit{complètement ratée} entraîne automatiquement une
	note en-dessous de 10. Le cas échéant, vous devrez obligatoirement me fournir
	une \textbf{fiche personnelle des 10 questions} de cours pour le lundi qui
	suit, sur votre table avant l'interrogation.
\end{tcb}
Le programme de khôlle avec les questions possibles sera
toujours disponible au plus vite après un cours, avec un affichage direct sur le
site. Pour suivre vos avancées, vous devez \textbf{impérativement} avoir sur
vous votre feuille de khôlle.

\subsubsection{Travaux pratiques}
En plus du cahier de manipulation, les travaux pratiques sont aussi sujet à une
évaluation, sous forme de \textbf{compte-rendu}. Vous devez travailler en
binôme, libre à chaque séance, et rendre une copie pour deux. Le compte-rendu
doit comporter~:
\begin{enumerate}
	\item L'intitulé du TP~;
	\item Les objectifs du TP~;
	\item Les réponses aux questions à préparer à l'avance\footnote{Elles sont
		      repérées par \circled{1}}~;
	\item Les réponses aux questions explicitement posées pendant la
	      séance\footnote{Elles sont repérées par \fbox{1}}.
\end{enumerate}

% \begin{tcb}*(prop)"divi"{Notation~: travaux pratiques}
% 	Certains comptes-rendus seront relevés chaque semaine, \textit{aléatoirement}.
% 	Ils sont notés par une lettre, de A à F.
% \end{tcb}

\subsubsection{Devoirs maisons (DM)}
Il y aura régulièrement des devoirs maisons, avec un délai de une ou deux
semaines, obligatoires ou facultatifs. Ce sont des exercices complets à traiter
sérieusement.

% \begin{tcb}*(prop)"divi"{Notation~: devoirs maisons}
% 	Ils ne sont pas notés numériquement, mais comptent grandement dans mon avis
% 	global sur votre bulletin. Vous pouvez les chercher à 2, avec une copie pour
% 	2, mais \textbf{jamais plus}. Tout plagiat honteux sera relevé et marqué.
% \end{tcb}

\subsubsection{Devoirs surveillés (DS)}
Il y a un DS de physique-chimie toutes les 3 semaines, de 3h chacun. Vous saurez
le programme concerné en avance. Toutes les connaissances de l'année écoulées
peuvent être requises.

% \begin{tcb}*(prop)"divi"{Notation~: devoirs surveillés}
% 	Les DS sont notés en ramenant la moyenne de la classe à $\approx 10$. La
% 	moyenne des DS au semestre est au coefficient 4.
% \end{tcb}

\subsection{Résumé}
\begin{tcb}*(ror)"rapp"{}
	\begin{itemize}[label=$\diamond$, leftmargin=10pt]
		\bitem{Lundi 13\string:00}~: 2h en classe entière, \ul{interrogation} de
		cours, puis cours~;
		\bitem{Mardi 12\string:10}~: 1h en classe entière, distribution
		interrogations corrigées, \ul{correction d'exercices} donnés la semaine
		précédente~;
		\bitem{Mardi 14\string:00}~: 1h en demi-groupe, travaux dirigés (TD)~;
		\bitem{Jeudi 08\string:00}~: 2h en classe entière~;
		\bitem{Jeudi 10\string:00}~: Potentiellement soutien (premier semestre), 1
		ou 2h~;
		\bitem{Jeudi 14\string:00}~: 2h en demi-groupe, travaux pratiques (TP)~;
		\bitem{Vendredi 08\string:00}~: (second semestre)~: 2h de travaux
		d'initiative personnelle encadrés (TIPE)~;
		\bitem{Vendredi 14\string:00}~: devoir surveillé de 3h.
	\end{itemize}
\end{tcb}

\section{Fonctionnement du groupe-classe}
\subsection{Rapports interpersonnels}

% Je considère que dans des relations interpersonnelles, nous ne respectons pas un
% titre ou une profession mais bien une \textbf{personne}. Ainsi je ne me fais pas
% figure hiérarchique absolue, j'attends de vous de me traiter avec respect comme
% j'interagirai avec vous avec respect.

Je m'exprime à vous en employant le \textbf{vouvoiement} et, au choix,
votre prénom ou nom de famille. Ceci a pour volonté de vous faire sentir la
responsabilité partagée qui nous incombe à vous comme à moi~: si je vais mener
la plupart des cours et transmettre la plupart des informations, il est de votre
responsabilité en tant que scientifiques en essence de travailler en ce sens. Je
porterai une grande valeur à vos propos, questions, interrogations, suggestions,
réflexions et propositions sur tout ce qui concerne la science.

Ces consignes fondamentales sont également applicables à vos relations entre
vous. Aucun harcèlement, aucune moquerie, et aucune discrimination ne seront
tolérées dans cette classe. Si vous venez en cours dans l'idée d'arriver au
sommet en marchant sur la tête des autres, \textbf{vous n'êtes pas les
	bienvenu-es}. Si vous venez en cours sans respecter le travail de vos camarades,
\textbf{vous n'êtes pas les bienvenu-es}. Si vous venez en cours sans respecter
l'institution qui vous accueille ou la science que je vous transmets,
\textbf{vous n'êtes pas les bienvenu-es}.

En revanche, venez avec l'attitude et la rigueur qui est attendue de vous, et
vous profiterez du meilleur accompagnement vers la réussite que vous puissiez
espérer.

\subsection{Règles de vie}
\noindent
\begin{minipage}[t]{.48\linewidth}
	\begin{enumerate}
		\item Soyez à l'heure\footnote{À défaut rentrez \textit{discrètement} par la
			      porte du fond}~;
		\item Soyez attentif-ves en cours~;
		\item Soyez silencieux-ses~;
	\end{enumerate}
\end{minipage}
\hfill
\begin{minipage}[t]{.48\linewidth}
	\begin{enumerate}[resume]
		\item Aucun téléphone toléré~;
		\item Nourriture interdite, eau autorisée~;
		\item Ne me demandez pas d'aller aux toilettes. Levez-vous et allez-y.
	\end{enumerate}
\end{minipage}

\section{Conseils d'apprentissage}
\subsection{En classe}

\begin{tcb}[cnt](impo){Important}
	\bfseries
	Posez vos questions si vous en avez~: il y a 95\% de chance qu'une autre
	personne se la pose aussi.
\end{tcb}
\begin{itemize}
	\item Soyez \textbf{attentif-ves} (plus d'attention en classe = moins de
	      travail seul-e dans son coin après)~;
	\item Soyez organisé-es~: bloc-notes, trieur, pochettes plastifiées avec
	      code couleur pour avoir les cours et TDs pertinents et séparés…
	\item Soyez efficaces dans votre prise de note~: établissez des codes
	      couleurs, des abréviations, ne faites pas tous les schémas à la règle du
	      premier coup… Écouter et écrire divise l'attention. Je n'assure
	      absolument pas de vous fournir tous les documents de cours.
\end{itemize}

\subsection{En dehors}
\begin{itemize}
	\item \textbf{Relisez votre cours le soir-même}, ajoutez des annotations,
	      refaites les schémas, commencez à mémoriser~;
	\item Travaillez avec vos camarades pour préparer les TDs, réviser les
	      khôlles, posez-vous des questions entre vous~;
	\item Soyez actif-ves pendant vos séances, cherchez à comprendre, testez
	      votre compréhension~;
	\item Faites \underline{vos propres fiches pour chaque chapitre}~:
	      nécessaire et obligatoire pour réviser les concours~;
	\item \textbf{Dormez suffisamment (8 heures) et levez-vous suffisamment tôt
		      !!} Ne surestimez pas votre capacité à faire des courtes nuits et à
	      rester efficaces le lendemain. Ça ne sert à rien de venir en cours si
	      c'est pour dormir…
\end{itemize}

\begin{tcb}[cnt](ror){Motto}
	\Large\bfseries
	Le seul échec, c'est de ne pas essayer.
\end{tcb}

\vfill

Si le besoin s'en fait sentir, vous pouvez m'écrire à
\href{nora.nicolas@ac-orleans-tours.fr}{nora.nicolas@ac-orleans-tours.fr}.

\vfill

\begin{tcb}[cnt, boxrule=3pt, sharp corners, valign=center](black){}
	\vspace{-5pt} \Large\bfseries Bonne rentrée~!
\end{tcb}

\vfill

\end{document}
