\documentclass[a4paper, 12pt, garamond]{book}
\usepackage{cours-preambule}

\makeatletter
\renewcommand{\@chapapp}{Fiches -- numéro}
\makeatother

\begin{document}
\setcounter{chapter}{0}

\chapter{Motivation, procrastination et distractions}

% AsapSCIENCE
% https://www.youtube.com/watch?v=vcjQ5JkEE_0

% author: Jeffrey Kaplan
% https://www.youtube.com/watch?v=i2EEnJedcYU

% Fabien Olicard
% https://www.youtube.com/watch?v=ogMSk0-iHa4

Ce sujet est crucial pour votre réussite académique~: comment vaincre la
procrastination. Avant de plonger dans les détails, voici un aperçu rapide de ce
que nous allons explorer.

La procrastination, ce phénomène qui nous empêche de faire ce que nous devrions,
n'est pas simplement un problème de gestion du temps, mais plutôt de gestion des
émotions. Nous connaissons toustes ce sentiment de remettre à plus tard des
tâches importantes. Cela se traduit souvent par l'utilisation excessive de nos
téléphones, comme lorsqu'on commence à envoyer un message important et qu'on
finit par passer des heures sur TikTok sans avoir envoyé le message en question,
et qui conduit bien souvent à un sentiment de culpabilité.

La première étape consiste à comprendre que la procrastination est liée à des
émotions négatives, comme la peur de l'échec ou l'ennui, et à des systèmes de
récompenses déviés de leur utilité primaire principalement dû à une utilisation
excessive des \textit{smartphones}.

\section{Utilisation du téléphone}

À l'ère numérique actuelle, les \textit{smartphones} sont devenus une partie
intégrante de nos vies, nous permettant de nous connecter, d'apprendre et
d'accéder à l'information comme jamais auparavant. Cependant, il est essentiel
de reconnaître les éventuels inconvénients d'une utilisation excessive du
téléphone, en particulier en ce qui concerne son impact sur notre motivation. Ce
document vise à mettre en lumière comment votre utilisation du téléphone
pourrait affecter votre motivation et à fournir des informations pour atténuer
ces effets.

\subsection{Les statistiques écrasantes}

Il y a à peine 15 ans, seulement 20 \% des personnes accédaient à Internet
depuis leur téléphone. Aujourd'hui, ce nombre a grimpé en flèche à 91 \%.
L'adulte moyen passe environ 11 heures par jour à interagir avec les médias. Ces
statistiques indiquent un changement significatif dans la manière dont nous
consommons de l'information et nous connectons avec les autres, mais cela
soulève également des préoccupations quant à la manière dont cette connectivité
constante affecte notre bien-être.

\subsection{Le rôle de la dopamine}

La dopamine est un neurotransmetteur souvent associé au plaisir et à la
récompense. Elle est essentielle pour motiver des comportements bénéfiques tels
que manger, socialiser et rechercher de nouvelles expériences. Votre
\textit{smartphone} déclenche la libération de dopamine dans votre cerveau, vous
faisant ressentir une récompense lorsque vous recevez des notifications, des
\textit{likes} ou que vous interagissez avec du contenu divertissant. Avec le
temps, ces activités renforcent les voies neuronales associées à la libération
de dopamine, entraînant des comportements compulsifs et réduisant votre
motivation à participer à d'autres activités nécessitant effort et persévérance.

\subsection{Le dilemme de la motivation}

La recherche montre qu'une utilisation excessive du téléphone peut entraîner une
diminution de l'attention, des difficultés à se concentrer et une capacité
réduite à différer la gratification. Ce phénomène, appelé «~décote de délai~»,
peut entraîner une diminution de la motivation pour les tâches nécessitant un
effort soutenu ou offrant des récompenses différées. En réalité, passer des
heures prolongées devant des écrans a été lié à un risque accru de dépression et
d'anxiété, en particulier chez les jeunes adultes. Les adolescents qui passent
trop de temps sur leurs appareils mobiles ont 71 \% de risques en plus de
développer des facteurs de risque de suicide.

\subsection{Reconnaître l'addiction au Téléphone}

Comprendre si vous êtes dépendant de votre téléphone est essentiel pour atténuer
son impact sur votre motivation. Posez-vous les questions suivantes~:
\begin{tcb}*[breakable](impo)"rema"{Dépendance au téléphone}
	\begin{enumerate}
		\item Avez-vous des envies de vérifier votre téléphone au détriment d'autres
		      activités~?
		\item Votre humeur change-t-elle en fonction des notifications et des
		      interactions sur les réseaux sociaux~?
		\item Trouvez-vous nécessaire de passer plus de temps sur votre téléphone pour
		      être satisfait~?
		\item Vous sentez-vous mal à l'aise ou stressé-e lorsque vous n'utilisez pas
		      votre téléphone~?
		\item Vos tentatives pour réduire l'utilisation du téléphone ont-elles
		      entraîné des rechutes~?
	\end{enumerate}
\end{tcb}

Si vous avez répondu «~oui~» à ces questions, vous pourriez faire face à une
dépendance au téléphone.

\subsection{Prendre le contrôle}

La bonne nouvelle est que vous pouvez reprendre le contrôle de votre utilisation
du téléphone et de votre motivation. Voici trois stratégies soutenues par la
science à envisager~:
\begin{tcb}*(prop)"bulb"{Astuces}
	\begin{enumerate}
		\bitem{Restriction du temps}~: Allouez des plages horaires spécifiques
		pour l'utilisation du téléphone, comme une heure par jour. Cette
		approche empêche les vérifications compulsives et permet aux systèmes de
		dopamine de votre cerveau de se réajuster progressivement~;
		\bitem{Contrainte physique}~: Déconnectez-vous des applications
		déclencheuses, confiez vos mots de passe à un-e ami-e de confiance ou placez
		physiquement votre téléphone hors de portée à certains moments, comme
		l'éteindre à 21 heures~;
		\bitem{Contrainte catégorique}~: Rendez votre téléphone moins attractif
		en utilisant le mode gris, en vérifiant les applications à forte teneur en
		dopamine uniquement sur un ordinateur et en supprimant les applications
		inutiles qui gaspillent votre temps.
	\end{enumerate}
\end{tcb}

\subsection{Conclusion}

Dans un monde où les \textit{smartphones} sont omniprésents, comprendre l'impact
d'une utilisation excessive du téléphone sur votre motivation est crucial. En
reconnaissant le rôle de la dopamine, en étant conscient des signes de
dépendance au téléphone et en mettant en œuvre des stratégies pour contrôler
votre utilisation du téléphone, vous pouvez retrouver la concentration, la
motivation et une relation plus saine avec la technologie.

\section{Méthodes pour combattre la procrastination}
Certes, l'utilisation du téléphone est prouvé comme impactant grandement les
circuits de récompenses, et s'en détacher ne pourra que vous faire du bien. Il
reste que les schémas neuronaux créés par cette utilisation sont forts, et que
la procrastination peut passer par d'autres mécanismes. Je vous présente dans ce
qui suit trois pistes pour y remédier.

\subsection{Listes et récompenses}

Une première méthode consiste à découper la tâche en petites parties et à vous
récompenser à chaque étape accomplie. Vous avez peut-être entendu parler de
l'effet de la carotte et du bâton. Dans ce cas, la carotte est une petite
récompense que vous vous offrez à chaque progression. Notez chaque étape sur une
liste physique, pas sur votre téléphone. Lorsque vous cochez une tâche
accomplie, votre cerveau libère de la dopamine, l'hormone du bonheur. Cette
sensation positive renforce votre motivation à continuer. Classez les tâches par
ordre d'importance et d'urgence pour une meilleure organisation, et à la fin de
chaque section terminée, accordez-vous une pause pour savourer une friandise,
prendre une pause ou faire ce qui vous détend, mais \textbf{qui n'est pas votre
	téléphone}. Cette approche réduit le stress et augmente votre sentiment
d'accomplissement à chaque étape franchie.

\subsection{Éliminer les tentations}

La deuxième méthode est de supprimer les distractions et les tentations qui nous
éloignent de nos tâches. Imaginez que vous travaillez sur un devoir important,
mais votre téléphone est à portée de main, prêt à vous distraire avec des
notifications et des divertissements en ligne. En éliminant ces tentations, vous
supprimez les obstacles qui vous empêchent de vous concentrer. Si elles ont une
utilité certaine, par exemple écouter de la musique qui vous aide à travailler,
remplacer chaque fonction \textbf{essentielle} par un outil spécifique. Besoin
de prendre des notes~: prenez un carnet~!

Ne négligez pas la valeur de l'ennui et du temps libre. Dans notre société
actuelle, nous cherchons constamment à combler chaque minute de notre temps
libre avec des distractions numériques. Nous sommes devenus réticent-es à
laisser notre esprit vagabonder et à profiter de moments de calme. Pourtant, en
n'ayant pas de téléphone, vous découvrirez que l'ennui peut être bénéfique. Il
nous permet de laisser notre esprit vagabonder, de laisser place à la créativité
et à la réflexion. Cela nous donne également l'opportunité d'observer le monde
qui nous entoure, de remarquer des détails que nous aurions autrement manqués.

\subsection{Récolter la motivation}

Passons à la méthode que j'appelle «~la récolte de motivation~»~: organisez-vous
avec un groupe de travail pour vous engagez à accomplir des tâches ensemble.
Ce concept a été éprouvé et testé. Par exemple, lorsque j'étais en études
supérieures, je faisais partie d'un groupe d'étude où nous nous retrouvions
régulièrement pour travailler ensemble, que ce soit en commun ou en silence. À
la fin de chaque période de travail, nous partagions nos réalisations. Cette
pression positive de ne pas être l'élément déconcentré du groupe nous motivait à
rester concentré-es et productif-ves~: qui sème la motivation récolte la
motivation~!

\begin{tcb}(ror){Points clés à retenir}
	\begin{itemize}
		\item La procrastination découle souvent d'émotions négatives liées à la
		      tâche à accomplir.
		\item Les récompenses progressives, l'élimination des distractions et la
		      récolte de motivation sont des méthodes efficaces pour lutter contre
		      la procrastination.
		\item Travailler en groupe peut augmenter votre motivation et votre
		      productivité.
	\end{itemize}

\end{tcb}

Il faut tout de même savoir se réserver un moment à part pour vos projets
personnels. Bloquer deux heures dans la semaine dédiées à ce qui vous tient à
cœur est un excellent moyen de garantir que vous progresserez régulièrement.
Cette habitude vous libère mentalement et vous permet de vous consacrer
pleinement à vos aspirations.

% \subsection{Exemples concrets et anecdotes}
%
% Avant d'entrer en CPGE, je n'avais pas \textit{vraiment} travaillé un seul jour
% de ma vie. Faire des fiches m'était inconnu, et je faisais mes exercices dans le
% bus le matin même, comme nombre d'entre vous sous doute. Aussi je suis arrivée
% dans la chambre que j'habitais avec mon ordinateur fixe, et je continuais ma vie
% comme au lycée à passer mes weekends et soirées à jouer. En réalité j'avais une
% peur bien ancrée~: celle d'être \textbf{bête} et d'être vouée à me rendre compte
% que je n'étais pas une petite génie. Je m'étais créé, a moitié consciemment,
% tout un arsenal de comportements «~destructeurs~»~: refuser de travailler trop
% par peur de me rendre compte que je n'y arriverais pas, et me complaire des
% mauvaises notes sous couvert de «~si j'ai eu 08 sans effort, j'aurais sans doute
% eu 15 en travaillant~».
%
% J'ai tenu 2 mois, puis les notes en-dessous de 08/20 ont commencé à sérieusement
% m'impacter. Ramener mon ordinateur chez mes parents a été la deuxième meilleure
% décision de ma vie, académiquement parlant. La meilleure a été de remplacer mon
% \textit{smartphone} par un téléphone à clavier physique pendant ma troisième
% année (clairement l'ordi seul c'était pas suffisant~!), mais l'élément le plus
% impactant a été de trouver un groupe de khôlle et des amitiés qui créaient une
% synergie de travail. Ne sous-estimez absolument jamais vos propres capacités~:
% vous êtes tous et toutes très capables, les erreurs font partie intégrante du
% chemin vers la réussite, et le seul échec qui puisse vraiment exister c'est de
% ne pas essayer.

\subsection{Conclusion et récapitulation}

En résumé, la procrastination n'est pas simplement une question de gestion du
temps, mais elle est liée à nos émotions. Nous avons discuté de trois méthodes
pour lutter contre la procrastination~: les listes et récompenses, l'élimination
des tentations et la récolte de motivation. En appliquant ces techniques, vous
pouvez transformer vos émotions négatives en moteurs de productivité.

De nombreuses personnes cherchent à trouver un équilibre entre les avantages de
la technologie et leur bien-être global. En prenant des mesures pour limiter
l'utilisation de votre téléphone, vous investissez dans votre succès futur et
votre bien-être.

N'oubliez pas que vous n'êtes pas seul-es dans ce combat contre la
procrastination. Trouvez des partenaires d'étude ou des collègues pour
travailler ensemble, et vous verrez que la motivation peut se propager.

Je vous encourage à mettre en pratique ces méthodes et à observer les
changements positifs dans votre façon de gérer votre temps et vos tâches. En fin
de compte, la clé réside dans la gestion de vos émotions pour devenir plus
productif-ves et atteindre vos objectifs académiques.

% \chapter{Arrêtez de perdre votre temps (votre cerveau a un antidote)}
%
% Aujourd'hui, nous allons aborder un sujet essentiel pour notre productivité~:
% la procrastination. Vous savez, ce sentiment qui nous pousse à remettre à plus
% tard ce que nous pourrions accomplir aujourd'hui. Eh bien, ne vous inquiétez
% pas, car aujourd'hui je vais vous présenter des stratégies simples mais
% efficaces pour combattre la procrastination et optimiser votre temps.
%
% \section{Comprendre la Procrastination}
%
% Avant de plonger dans les astuces, il est crucial de comprendre ce qu'est la
% procrastination. Procrastiner, c'est remettre à demain ce que nous pourrions
% réaliser aujourd'hui.
%
% \section{La Méthode des Petits Pas}
%
% L'une des raisons principales de la procrastination est le stress et l'angoisse
% associés à la taille apparente d'une tâche à accomplir. Une excellente
% stratégie consiste à utiliser la méthode des petits pas. Cette méthode implique
% de découper une grande tâche en de petites étapes réalisables. Par exemple, si
% vous devez préparer une recette, commencez par dresser une liste de courses.
% Ensuite, accomplissez chaque étape une par une. 
%
% \section{L'Art des Listes}
%
% Les listes sont vos meilleures amies pour lutter contre la procrastination.
% Lorsque vous utilisez la méthode des petits pas, 
%
% \section{Savoir Dire Non}
%
% Apprendre à dire non est une compétence clé pour éviter la surcharge de travail.
% Avant d'accepter une nouvelle tâche, prenez le temps de considérer si elle est
% vraiment essentielle. Dire non vous permettra de préserver votre temps pour des
% activités qui vous tiennent réellement à cœur.
%
% \section{Bloquer du temps et aller au bout}
%
% \section{Points Clés}
%
% Avant de conclure, voici quelques points clés à retenir:
% \begin{itemize}
% 	\item La procrastination est le report de tâches à plus tard.
% 	\item La méthode des petits pas consiste à découper les tâches en étapes
% 	      réalisables.
% 	\item Utilisez des listes physiques pour visualiser vos progrès et stimuler la
% 	      dopamine.
% 	\item Apprenez à dire non pour préserver votre temps.
% 	\item Bloquez du temps chaque semaine pour vos projets personnels.
% \end{itemize}
%
% N'oubliez pas, chers étudiants, que même les plus grands esprits ont dû lutter
% contre la procrastination. En utilisant ces astuces, vous pouvez forger un
% chemin vers la réalisation de vos ambitions. Je vous encourage à intégrer ces
% pratiques dans votre quotidien. En fin de compte, la procrastination peut être
% surmontée par une combinaison de stratégies simples et de compréhension de nos
% processus mentaux. Rappelez-vous que vous avez le pouvoir de transformer vos
% intentions en actions concrètes.
%
% \chapter{I Stopped Using My Phone. The results were shocking}
%
% % Nate O'Brien
% % https://www.youtube.com/watch?v=C-Iewo7zUFo
%
% Chers étudiants,
%
% Aujourd'hui, nous allons explorer les effets de l'utilisation excessive des
% téléphones portables sur nos vies. Dans la société moderne, les smartphones sont
% omniprésents et nous passons en moyenne près de six heures par jour à les
% utiliser. Si vous faites le calcul, cela équivaut à trois mois par an, soit
% quinze années de notre vie si l'on considère une durée de soixante ans. C'est
% une perspective saisissante, n'est-ce pas~? Eh bien, j'ai récemment entrepris
% une expérience audacieuse~: j'ai décidé de cesser totalement d'utiliser mon
% téléphone portable pendant un mois et de le remplacer par un carnet. Cette idée
% peut sembler radicale, mais elle m'a permis d'en apprendre davantage sur
% l'impact de nos appareils sur notre vie quotidienne.
%
% Tout d'abord, il est important de noter que les téléphones portables, bien
% qu'extrêmement utiles, peuvent parfois susciter un sentiment de culpabilité.
% Utiliser le téléphone pendant de longues heures sans véritable raison peut
% laisser un arrière-goût désagréable. Je suis sûr que vous avez tous ressenti
% cette sensation à un moment donné. Pourtant, lors de cette expérience, j'ai
% remplacé mon téléphone par un simple carnet, et j'ai découvert quelque chose
% d'étonnant~: cette absence de téléphone a été l'une des expériences les plus
% libératrices de ma vie. Normalement, nous sommes constamment à la recherche de
% notre téléphone, ressentant une forme d'anxiété si nous ne le trouvons pas.
% Pourtant, dès que j'ai pris la décision consciente de ne pas avoir mon téléphone
% sur moi, j'ai ressenti un allègement, un déchargement de ce poids. Cela nous
% rappelle à quel point nous sommes souvent esclaves de ces appareils.
%
%
%
% L'une des conséquences notables de cette expérience a été l'amélioration de mes
% compétences sociales. Sans téléphone pour me distraire, j'ai été plus enclin à
% engager des conversations avec les autres et à nouer des interactions sociales.
% Cela m'a rappelé que nos téléphones peuvent parfois nous éloigner de
% l'expérience humaine directe. En somme, l'expérience a renforcé ma conviction
% que les moments de pause, de réflexion et d'interaction humaine sont précieux et
% nécessaires à notre bien-être.
%
% Cependant, il est également important de mentionner les obstacles que j'ai
% rencontrés lors de cette expérience. En ne disposant pas de mon téléphone, je me
% suis souvent retrouvé perdu, devant mémoriser les directions ou demander de
% l'aide aux passants pour me repérer. Cela a engendré des moments amusants, mais
% a également révélé à quel point notre dépendance aux téléphones est profondément
% enracinée dans nos habitudes quotidiennes. De plus, il est probable que dans les
% années à venir, l'utilisation du téléphone sera indispensable pour des tâches
% courantes, rendant ainsi difficile la réalisation d'une expérience similaire.
%
% En conclusion, cette expérience m'a permis de prendre du recul par rapport à
% l'utilisation excessive des téléphones portables. Si je n'ai pas réussi à tenir
% un mois complet sans téléphone, j'ai tout de même retiré des leçons
% significatives. Il est essentiel de trouver un équilibre entre l'utilisation des
% technologies et des moments de déconnexion. Je vous encourage à réfléchir à vos
% propres habitudes liées aux téléphones et à envisager comment vous pourriez
% mieux gérer votre temps et votre relation avec les téléphones. Merci d'avoir
% participé à cette discussion enrichissante.
%
% Bien à vous,
% Professeur de Physique
%
% Points clés à retenir~:
% \begin{enumerate}
% 	\item L'utilisation excessive des téléphones peut engendrer un sentiment de
% 	      culpabilité.
% 	\item Remplacer le téléphone par un carnet peut être libérateur et permettre de réduire l'anxiété.
% 	\item L'ennui et le temps libre sont importants pour la créativité et la
% 	      réflexion.
% 	\item Les interactions sociales peuvent s'améliorer en l'absence de téléphone.
% 	\item Les obstacles rencontrés incluent la perte de repères et la dépendance
% 	      aux directions.
% 	\item L'utilisation croissante des téléphones dans la société rendra difficile
% 	      de mener de telles expériences à l'avenir.
% 	\item Trouver un équilibre entre l'utilisation des technologies et la
% 	      déconnexion est essentiel pour le bien-être mental.
% \end{enumerate}

\end{document}
