\documentclass[a4paper, 12pt, garamond]{book}
\usepackage{cours-preambule}

\dominitoc
\faketableofcontents

\makeatletter
\renewcommand{\@chapapp}{Fiches -- numéro}
\makeatother

\begin{document}
\setcounter{chapter}{6}

\chapter{\'Equations diff\'erentielles en sciences physiques}

\section{Introduction}
\begin{tcb}(defi){Définition}
	Une équation différentielle\ftn{Notée ED dans la suite.} sur la fonction $y(t)$
	est une équation liant $y(t)$ à ses \textbf{dérivées}. Elle est dite
	\textit{linéaire}\ftn{Notée donc EDL dans la suite.} (on ne verra pas le cas
	non-linéaire cette année) si elle peut s’écrire comme~:
	\[
		f_0(t) \cdot y(t) + f_1(t) \cdot \dv{y(t)}{t} + \ldots + f_N(t) \cdot
		\dv[N]{y(t)}{t} = g(t)
		\Lra
		\sum_{i=0}^{N} f_i(t) \dv[i]{y(t)}{t} = g(t)
	\]
  Avec $f_{i \in [1,N]}(t)$ et $g(t)$ des fonctions dépendantes de la variable
	$t$\ftn{Attention, cette variable n'est pas \textit{que} le temps. Ça peut
		être une dimension d'espace, par exemple $x$.}.
\end{tcb}

\begin{tcb}[sidebyside, sidebyside align=top](exem){Exemples}
	\tcbsubtitle{\fatbox{Exemple}}
	\begin{itemize}
		\item $f_0(t) = t$, $f_1(t) = 1$ et $f_{i \geq 2}(t) = 0$~;
		\item $y(t)$ à résoudre~;
		\item $g(t) = A$~;
	\end{itemize}
	On a
	\[
		t \cdot y(t) + \dv{y(t)}{t} = A
	\]
	\tcblower
	\tcbsubtitle{\fatbox{Contre-exemple}}
	\[
		4\xunderbracket{y^{2}(t)}_{\mathclap{\text{non-linéaire}}} +
		\dv{y(t)}{t} = 0
	\]
	ou
	\[
		4y(t) +
		\xunderbracket{y(t) \cdot \dv{y(t)}{t}}_{\mathclap{\text{non-linéaire}}} = 0
	\]
\end{tcb}

Dans la pratique, on rencontrera des équations différentielles très
particulières en sciences physiques. On les caractérise par~:
\begin{itemize}
	\bitem{Leur ordre}~: c'est le rang maximum de la dérivation portant sur
	$y(t)$. Cette année, on s'intéresse à des équations d'ordre 1 et 2 seulement.
	\bitem{Leurs coefficients}~: les $f_i(t)$ sont des fonctions, mais la plupart
	du temps ce seront des constantes. On dit alors que l'équation différentielle
	est à \textit{coefficients constants}\ftn{Notée EDC dans la suite.}.
	\bitem{Leur second membre}~: dans l'écriture d'une ED, on met tous les termes
	qui dépendent de $y$ à gauche du signe «~$=$~», et tout le reste à droite,
	c'est le $g(t)$ de la définition. On l'appelle \textit{second membre}.
\end{itemize}

\begin{tcb}(defi){Équation homogène}
	On appelle \textbf{équation homogène} une équation différentielle \textbf{sans
		second membre}~: $g(t) = 0$
	\[
		\sum_{i} f_i(t) \dv[i]{y(t)}{t} = \fbox{0}
	\]
\end{tcb}

\begin{tcb*}[label=prop:lin](prop){Linéarité pour une EDL homogène}
	Soient $y_1$ et $y_2$ deux solutions d'une même EDL homogène\ftn{notée EDLH
		dans la suite.}. Alors, toute combinaison de ces deux solutions est également
	solution de l'EDLH~:
	\[
		\boxed{
			\forall (y_1,y_2)
			\text{~solutions d'une EDLH}
			\Ra
			\forall (\alpha_1,\alpha_2) \in \Rb^{2},
			\quad \alpha_1 y_1(t) + \alpha_2 y_2(t)
			\text{~est solution de l'EDLH}
		}
	\]
\end{tcb*}

\begin{tcb}(ror){Solution générale avec second membre}
	Pour une EDL avec second membre (non homogène), \textit{une} \textbf{solution
		générale} s'obtient par somme de la \textbf{solution homogène} $y_h$ et d'une
	\textbf{solution particulière} $y_p$~:
	\[
		\boxed{y(t) = y_h(t) + y_p(t)}
	\]
\end{tcb}

\begin{tcb}(impo){Attention}
	\begin{enumerate}
		\item Ici, c'est bien la \textbf{somme} de $y_h$ et $y_p$ qui est solution,
		      et pas toute autre combinaison $\alpha_1 y_h + \alpha_2 y_p$.
		\item Avec ceci, on obtient \textbf{une} solution de l'EDL, pas \textbf{la}
		      solution. Pour obtenir l'\textbf{unique} solution, on aura besoin de
		      \textbf{condition initiales}.
	\end{enumerate}
\end{tcb}

\begin{tcb}(tool){Solution particulière pour une EDLC à second membre
			constant}
	Dans le cas où $f_i(t) = a_i$ et $g(t) = b$ avec $a_i$ et $b$ des
	constantes, on cherche $y_p (t) = \lambda$ également une constante. Ainsi,
	\begin{align*}
		\sum_{i} f_i(t) \dv[i]{y_p(t)}{t} & = b
		\\\Lra
		a_0 \lambda +
		\underbracket[1pt]{
			\sum_{i \geq 1} f_i(t) \dv[i]{\lambda}{t}
		}_{ =0}                           & = b
		\\\Lra
		\Aboxed{y_p (t) = \lambda         & = \frac{b}{a_0}}
	\end{align*}
\end{tcb}

\begin{tcb}(rema)<lfnt>{}
	Une autre manière de tenir compte d'un second membre constant est de réaliser
	un changement de variable pour se ramener à une EDLH.
\end{tcb}

\section{Résolution d'EDLHC}
\subsection{Premier ordre}
On s'intéresse à une EDLHC\ftn{Équation Différentielle Linéaire Homogène à
	coefficients Constants.} d'ordre 1. Elle peut se mettre sous la forme~:
\begin{equation}
	\dv{y_h(t)}{t} + \frac{1}{\tau} y_h(t) = 0
	\label{eq:edlhc1}
\end{equation}
avec $\tau \in \Rb$ une constante homogène à $t$.
\begin{tcb}(nota){Vocabulaire}
	Si $t$ désigne bien le temps, comme souvent en physique, alors par rapide
	analyse d'homogénéité, $\tau$ s'exprime en secondes et s'appelle la
	\textbf{constante de temps}.
\end{tcb}

\begin{tcb}(ror){Solution d'EDLHC d'ordre 1}
	La solution \textbf{générale} de \eqref{eq:edlhc1} s'écrit, avec $A \in \Rb$
	la constante d'intégration~:
	\[
		\boxed{y_h(t) = A\exr^{-t/\tau}}
	\]
	On obtient \textbf{la} solution spécifique à l'aide des \textbf{conditions
		initiales}\ftn{Notées CI dans la suite.} du système étudié, qui fixera $A$.
\end{tcb}

\subsection{Second ordre}
Soit une EDLHC d'ordre 2. Elle peut se mettre sous la forme suivante, avec
$(\w_0,Q) \in \Rb^{2}$ des constantes~:
\begin{equation}
	\dv[2]{y_h(t)}{t} + \frac{\w_0}{Q}\dv{y_h(t)}{t} + \w_0{}^{2} y_h(t) = 0
	\label{eq:edlhc2}
\end{equation}
\begin{tcb}(nota){Vocabulaire}
	Si $t$ désigne bien le temps, alors~:
	\begin{itemize}
		\item $\w_0$ est l'\textbf{inverse} d'un temps et s'appelle la
		      \textbf{pulsation propre}, exprimée en $\si{rad.s^{-1}}$~;
		\item $Q$ est \textbf{adimensionné} et s'appelle le \textbf{facteur de
			      qualité}.
	\end{itemize}
\end{tcb}
Le fonctionnement complet derrière la résolution sera extensivement développé en
mathématiques. Cependant, on peut retenir l'idée suivante~: les fonction
exponentielles sont des fonctions clés pour les ED, et on peut logiquement
chercher $y_h(t) = K\exr^{rt}$ avec $r \in \Cb$ à déterminer. C'est ce qu'on
injecte dans~\eqref{eq:edlhc2}~:
\begin{tcb}(tool){Méthode pour EDLHC d'ordre 2}
	Pour résoudre~\eqref{eq:edlhc2}, on introduit le \textbf{polynôme
		caractéristique}~:
	\begin{equation}
		r^{2} + \frac{\w_0}{Q}r + \w_0^{2} = 0
		\label{eq:polcar}
	\end{equation}
	dont le discriminant $\Delta$ s'écrit
	\[
		\Delta =
		\left( \frac{\w_0}{Q} \right)^{2} - 4\w_0{}^{2} =
		\frac{\w_0{}^{2}}{Q^{2}} \left( 1 -4Q^{2} \right)
	\]
\end{tcb}
La forme des solutions dépend donc des valeurs possibles de $\Delta$, et on
distinguera trois cas, tous à connaître.

\subsubsection{$\Delta > 0 \Lra Q <1/2$}
\begin{tcb}(ror){EDLHC d'ordre 2 avec $\Delta > 0 \Lra Q <1/2$}
	Si $\Delta > 0 \Lra Q < 1/2$, alors~\eqref{eq:polcar} a deux racines~:
	\[
		r_{\pm} = -\frac{\w_0}{2Q} \pm \frac{\w_0}{2Q} \sqrt{1-4Q^{2}}
	\]
	et la solution \textbf{générale} de~\eqref{eq:edlhc2} s'écrit, avec $(A,B) \in
		\Rb^{2}$ les constantes d'intégrations~:
	\[
		\boxed{y_h(t) = A\exr^{r_+t} + B\exr^{r_-t}}
	\]
	par linéarité (propriété~\ref{prop:lin}).
\end{tcb}

\subsubsection{$\Delta = 0 \Lra Q=1/2$}
\begin{tcb}(ror){EDLHC d'ordre 2 avec $\Delta = 0 \Lra Q=1/2$}
	Si $\Delta = 0 \Lra Q = 1/2$, alors~\eqref{eq:polcar} a une racine double~:
	\[
		r = -\frac{\w_0}{2Q} = -\w_0
	\]
	et la solution \textbf{générale} de~\eqref{eq:edlhc2} s'écrit, avec $(A,B) \in
		\Rb^{2}$ les constantes d'intégrations~:
	\[
		\boxed{y_h(t) = (At + B)\exr^{-\w_0t}}
	\]
\end{tcb}

\subsubsection{$\Delta < 0 \Lra Q>1/2$}
\begin{tcb}(ror){EDLHC d'ordre 2 avec $\Delta < 0 \Lra Q>1/2$}
	Si $\Delta < 0 \Lra Q > 1/2$, alors~\eqref{eq:polcar} a deux racines complexes
	conjugées~:
	\[
		r_{\pm} =
		-\frac{\w_0}{2Q} \pm \jj \frac{w_0}{2Q} \sqrt{4Q^{2}-1} =
		-\frac{\w_0}{2Q} \pm \jj \Omega
	\]
	et la solution \textbf{générale} de~\eqref{eq:edlhc2} s'écrit, avec $(A,B) \in
		\Rb^{2}$ les constantes d'intégrations~:
	\[
		\boxed{
			y_h(t) = \exp\left( -\frac{\w_0}{2Q}t \right)
			\left[A\cos(\Omega t) + B\sin(\Omega t)\right]
		}
	\]
	par linéarité (propriété~\ref{prop:lin}).
\end{tcb}

\subsubsection{Cas particulier~: $Q \ra +\infty$}
Si $Q \ra +\infty$, alors l'Équation~\eqref{eq:edlhc2} se simplifie en~:
\[
	\dv[2]{y_h(t)}{t} + \w_0{}^{2}y_h(t) = 0
\]
Et la solution est celle pour $\Delta < 0 \Lra Q>1/2$ avec $Q \ra +\infty$, soit
\[
	\boxed{y_h(t) = A \cos(\w_0t) + B\sin(\w_0t)}
\]

\section{Résolution d'EDLC (avec second membre)}
Comme introduit dans la première section, une EDL se résout en prenant la somme
de $y_h(t)$ et de $y_p(t)$. Ça n'en donne cependant pas \textbf{la} solution,
mais une famille de solutions. Pour que la solution représente le système
étudiée, il faut trouver les valeurs des constantes d'intégrations. On
s'intéresse ici aux EDLC avec second membre constant.

\begin{tcb}[cnt](impo){Important}
	Rien de cette section n'est à retenir par cœur~! Seule la \textbf{méthode} est
	à comprendre.
\end{tcb}
\begin{tcb}*[cnt, bld, fontupper=\Large](prop)"bomb"{Attention}
	La détermination des constantes d'intégrations se fait bien sur la
	\textbf{solution générale}, pas uniquement sur l'équation homogène~!
\end{tcb}

\subsection{EDLC d'ordre 1}
Soit une EDLC dont le second membre vaut $K$, une constante connue.
\begin{equation}
	\dv{y_h(t)}{t} + \frac{1}{\tau} y_h(t) = \frac{1}{\tau}K
	\label{eq:edlc1}
\end{equation}
On obtient $y_p = K$ une solution particulière constante, et donc la solution
générale de~\eqref{eq:edlc1} ~:
\[
	y(t) = y_h(t) + y_p
	\Lra
	y(t) = A\exr^{-t/\tau} + K
\]
\textbf{La} solution s'obtient en étudiant cette forme de solution à un temps
donné, où l'on connaît la valeur que doit prendre la solution~: c'est une
\textbf{condition initiale}. Typiquement, on suppose que $y(0) = y_0$ une valeur
connue. Alors, \textbf{la} solution est telle que
\[
	y(0) = y_0 = A\exr^{0} + K
	\Lra
	A = y_0 - K
\]
ainsi,
\[
	\boxed{
		y(t) =
		(y_0-K)\exr^{-t/\tau}+K =
		K\left(1-\exr^{-t/\tau}\right)+y_0 \exr^{-t/\tau}}
\]

\subsection{EDLC d'ordre 2}
Soit une EDLC dont le second membre vaut $K$, une constante connue.
\begin{equation}
	\dv[2]{y_h(t)}{t} + \frac{\w_0}{Q}\dv{y_h(t)}{t} + \w_0{}^{2} y_h(t) =
	\w_0{}^{2}K
	\label{eq:edlc2}
\end{equation}
On obtient $y_p = K$ une solution particulière constante. Supposons $Q>1/2 \Lra
	\Delta<0$~: la solution générale de~\eqref{eq:edlc2} s'écrit
\[
	y(t) = y_h(t) + y_p
	\Lra
	y(t) = K + \exp\left( -\frac{\w_0}{2Q}t \right)
	\left[A\cos(\Omega t) + B\sin(\Omega t)\right]
\]
Les deux valeurs à déterminer pour connaître \textbf{la} solution sont $A$ et
$B$, les constantes d'intégrations. Il faut \textbf{2 conditions} initiales pour
connaître les deux constantes~; soit $y(0) = y_0$ et $\dv{y}{t}\/(0) = v_0$. On
obtient
\begin{align*}
	y(0) = y_0 = K + A
	\quad & \text{et} \quad
	\dv{y}{t}\/(0) = v_0 = -\frac{\w_0}{2Q}A + B\Omega
	\\\Lra
	A = y_0 - K
	\quad & \text{et} \quad
	B = \frac{v_0}{\Omega} + \frac{\w_0}{2Q}\frac{y_0-K}{\Omega}
\end{align*}
D'où la solution
\[
	\boxed{
		y(t) = K + \exp(-\frac{\w_0}{2Q})
		\left[
			(y_0-K)\cos(\Omega t) +
			\left( \frac{v_0}{\Omega} + \frac{\w_0}{2Q}\frac{y_0-K}{\Omega} \right)
			\sin(\Omega t)
			\right]
	}
\]
\begin{tcb}(rema)<lfnt>{Remarque}
	Pas d'inquiétude face à cette solution, encore une fois seule la
	\textbf{méthode} est à comprendre. Dans la pratique, les systèmes ont des
	propriétés élégantes et simples $y_0 = 0, v_0 = 0$ par exemple.
\end{tcb}

\section{Conclusion~: méthode de résolution}
\begin{tcb}(ror){Résolution EDLC second membre constant}
	Pour résoudre une équation différentielle linéaire à
	coefficients constants et second membre constant~:
	\begin{enumerate}[label=\sqenumi]
		\item On écrit l'\textbf{équation homogène} associée à
		      l'équation différentielle obtenue.
		\item On écrit la \textbf{forme générale de la solution de
			      l'équation homogène} $y_h(t)$.
		\item On recherche une \textbf{solution particulière constante de l'équation
			      générale}, de la forme $y_p(t) = \lambda$.
		\item On écrit la \textbf{solution générale}, somme de la
		      solution particulière et de la forme générale~: $y(t) = y_h(t) + y_p$
		\item On détermine la constante à l'aide des \textbf{conditions initiales}.
	\end{enumerate}
\end{tcb}

Toute tentative de détermination de CI d'une EDLC sur $y_h$ et non sur $y$
causera la mort instantanée d'un innocent et mignon petit chaton.

\section{Complément~: démonstrations}
\subsection{Linéarité pour EDLH}
\begin{tcb}(demo)<lfnt>{}
	Soit une EDLH
	\[
		\sum_{i} f_i(t) \dv[i]{y(t)}{t} = 0
	\]
	Alors, si
	\begin{gather*}
		\sum_{i} f_i(t) \dv[i]{y_1(t)}{t} = 0
		\qqet
		\sum_{i} f_i(t) \dv[i]{y_2(t)}{t} = 0
		\intertext{on a par linéarité de l'EDHL et de la dérivée}
		\sum_{i} f_i(t) \dv[i]{(y_1(t) + y_2(t))}{t} =
		\sum_{i} f_i(t) \dv[i]{y_1(t)}{t} +
		\sum_{i} f_i(t) \dv[i]{y_2(t)}{t} = 0
		\qed
	\end{gather*}
\end{tcb}

\subsection{$y_h(t)$ pour $\Delta < 0 \Lra Q > 1/2$}
\begin{tcb}[breakable](demo)<lfnt>{}
	En effet, avec ces deux solutions on trouve
	\[
		y_+(t) = \exr^{\left(-\frac{\w_0}{2Q} + \jj \Omega\right)t}
		\qqet
		y_-(t) = \exr^{\left(-\frac{\w_0}{2Q} - \jj \Omega\right)t}
	\]
	Pour avoir des solutions réelles, on utilise la formule d'\textsc{Euler}~:
	\[
		\exr^{\pm \Ir\tt} = \cos(\theta) \pm \Ir \sin(\theta)
	\]
	Ainsi,
	\begin{align*}
		y_+(t) & =
		\exr^{-\frac{\w_0}{2Q}t}\exr^{\jj \Omega t} =
		\exr^{-\frac{\w_0}{2Q}t}\left( \cos(\Omega t) + \jj\sin(\Omega t) \right)
		\\\text{et} \quad
		y_-(t) & =
		\exr^{-\frac{\w_0}{2Q}t}\exr^{-\jj \Omega t} =
		\exr^{-\frac{\w_0}{2Q}t}\left( \cos(\Omega t) - \jj\sin(\Omega t) \right)
	\end{align*}
	Ce qui ne rend pas encore les solutions réelles, mais on peut les combiner par
	linéarité. Par exemple,
	\begin{align*}
		y_{c} & =
		\frac{y_+ + y_-}{2} =
		\frac{\exr^{-\frac{\w_0}{2Q}t}}{2}
		\left(
		\cos(\Omega t) + \cancel{\jj \sin(\Omega t)} +
		\cos(\Omega t) - \cancel{\jj \sin(\Omega t)}
		\right) =
		\exr^{-\frac{\w_0}{2Q}t}\cos(\Omega t)
		\\
		\text{et} \quad
		y_{s} & = \frac{y_+ - y_-}{2\jj} =
		\frac{\exr^{-\frac{\w_0}{2Q}t}}{2 \jj}
		\left(
		\cancel{\cos(\Omega t)} + \jj \sin(\Omega t) -
		\cancel{\cos(\Omega t)} + \jj \sin(\Omega t)
		\right) =
		\exr^{-\frac{\w_0}{2Q}t}\sin(\Omega t)
	\end{align*}
	D'où une forme réelle et pratique pour $y_h(t) = Ay_c + By_s$, c'est-à-dire
	\begin{align*}
		y_h(t) & = A\exr^{-\frac{\w_0}{2Q}t}\cos(\Omega t) +
		B\exr^{-\frac{\w_0}{2Q}t}\sin(\Omega t)
		\notag
		\\\Lra
		\Aboxed{
		y_h(t) & = \exp\left( -\frac{\w_0}{2Q}t \right)
			\left[A\cos(\Omega t) + B\sin(\Omega t)\right]
		}
		\qed
	\end{align*}
\end{tcb}

\end{document}
