\documentclass[a4paper, 10pt, final, garamond]{book}

\usepackage{cours-preambule}
\setlength{\columnseprule}{0pt}
\setlength{\columnsep}{-2.5cm}

\begin{document}
\setcounter{chapter}{-1}

\chapter*{Planning de Physique-Chimie}

\section{Premier semestre}
\subsection{Optique géométrique}

\begin{multicols}{2}
	\subsubsection{Chapitres}

	\begin{itemize}[label=$\diamond$]
		\item[O1]: Propagation de la lumière
		\item[O2]: Base de l'optique géométrique
		\item[O3]: Miroirs plans et lentilles minces
		\item[O4]: Dispositifs optiques
	\end{itemize}
	\columnbreak
	\subsubsection{Travaux pratiques}

	\begin{itemize}[label=$\diamond$]
		\item[TP1]: Détermination de focales de lentilles
		\item[TP2]: Formation et observation d'images à distance finie~: mesures de
		distances
		\item[TP3]: Formation et observation d'images à l'infini~: lunette
		autocollimatrice
		\item[TP4]: Spectrométrie de la lumière visible~: goniomètre à réseau
	\end{itemize}
\end{multicols}

\subsection{Électrocinétique, partie 1}

\begin{multicols}{2}
	\subsubsection{Chapitres}

	\begin{itemize}[label=$\diamond$]
		\item[E1]: Circuits électriques dans l'ARQS
		\item[E2]: Résistances et sources
		\item[E3]: Capacités et inductances
		\item[E4]: Oscillateurs harmoniques et amortis
	\end{itemize}

	\columnbreak

	\subsubsection{Travaux pratiques}

	\begin{itemize}[label=$\diamond$]
		\item[TP5]: Dipôles en régime permanent
		\item[TP6]: Oscilloscope et tracé de caractéristiques
		\item[TP7]: Circtuis du premier ordre en régime transitoire
		\item[TP7]: Oscillateurs amortis en électricité et mécanique
	\end{itemize}
\end{multicols}

\begin{center}
	\textit{\Large Vacances}
\end{center}

\subsection{Transformations chimiques, partie 1}

\begin{multicols}{2}
	\subsubsection{Chapitres}

	\begin{itemize}[label=$\diamond$]
		\item[C1]: Introduction à la chimie
		\item[C2]: Transformations et équilibres chimiques
		\item[C3]: Cinétique chimique
	\end{itemize}

	\columnbreak

	\subsubsection{Travaux pratiques}

	\begin{itemize}[label=$\diamond$]
		\item[TP9]: Dosage par étalonnage~: spectrophotométrie et conductimétrie
		\item[TP10]: Suivi cinétique par spectrophotométrie~: cristal violet
		\item[TP11]: Suivi cinétique par conductimétrie~: saponification
	\end{itemize}

\end{multicols}

\subsection{Électrocinétique, partie 2}

\begin{multicols}{2}
	\subsubsection{Chapitres}

	\begin{itemize}[label=$\diamond$]
		\item[E5]: Circuits électriques en régime sinusoïdal forcé
		\item[E6]: Oscillateurs en régime sinusoïdal forcé
		\item[E7]: Filtrage linéaire
	\end{itemize}

	\columnbreak

	\subsubsection{Travaux pratiques}

	\begin{itemize}[label=$\diamond$]
		\item[TP12]: Étude des oscillations forcées d'un oscillateur électrique amorti
		\item[TP13]: Étude d'un filtre passe-bas du premier ordre
		\item[TP14]: Étude d'un filtre actif du second ordre
		\item[TP15]: Analyses spectrales de signaux électriques
	\end{itemize}
\end{multicols}

\begin{center}
	\textit{\Large Vacances}
\end{center}

\newpage

\section{Second semestre}

\subsection{Signal et onde}

\begin{multicols}{2}
	\subsubsection{Chapitres}

	\begin{itemize}[label=$\diamond$]
		\item[SO1]: Ondes progressives
		\item[SO2]: Interférences à deux ondes
	\end{itemize}

	\columnbreak

	\subsubsection{Travaux pratiques}

	\begin{itemize}[label=$\diamond$]
		\item[TP16]: Ondes ultrasonores~: mesure de caractéristiques
	\end{itemize}
\end{multicols}

\subsection{Mécanique, partie 1}

\begin{multicols}{2}
	\subsubsection{Chapitres}

	\begin{itemize}[label=$\diamond$]
		\item[M1]: Cinétique du point
		\item[M2]: Dynamique du point
		\item[M3]: Mouvement courbe
		\item[M4]: Approche énergétique du mouvement
		\item[M5]: Mouvement de particules chargées
	\end{itemize}

	\columnbreak

	\subsubsection{Travaux pratiques}

	\begin{itemize}[label=$\diamond$]
		\item[TP17]: Étude du pendule simple
		\item[TP18]: Étude de la chute d'une bille en fluide visqueux
		\item[TP19]: Mesure du coefficient adiabatique de l'air
		\item[TP20]: Étude des oscillations forcées d'un oscillateur mécanique amorti
	\end{itemize}
\end{multicols}

\setlength{\columnsep}{0cm}

\begin{multicols}{2}
	\subsection{Architecture de la matière, prt. 1}
	\begin{itemize}[label=$\diamond$]
		\item[AM1]: Structure des entités chimiques
		\item[AM2]: Propriétés physico-chimiques macroscopiques
	\end{itemize}

	\begin{center}
		\textit{\Large Vacances}
	\end{center}

	\columnbreak

	\subsection{Mécanique, partie 2}
	\begin{itemize}[label=$\diamond$]
		\item[M6]: Moment cinétique du point
		\item[M7]: Mouvement à force centrale
		\item[M8]: Mécanique du solide
	\end{itemize}
\end{multicols}

\setlength{\columnsep}{-2.5cm}

\subsection{Chimie, partie 2}

\begin{multicols}{2}
	\subsubsection{Chapitres}

	\begin{itemize}[label=$\diamond$]
		\item[C4]: Réactions acido-basiques
		\item[C5]: Réactions de précipitation
		\item[C6]: Réactions d'oxydoréduction
		\item[C7]: Diagrammes potentiel-pH
	\end{itemize}

	\columnbreak

	\subsubsection{Travaux pratiques}

	\begin{itemize}[label=$\diamond$]
		\item[TP21]: Dosage par titrage acido-basique avec suivi pH-métrique et
		conductimétrique du vinaige
		\item[TP22]: Détermination de $\mathrm{p}K_s(\ce{AgCl})$ par colorimétrie et
		potentiométrie
		\item[TP23]: Titrage du sulfate ferreux~: potentiométrie à intensité nulle et
		colorimétrie
		\item[TP24]: Dosage indirect de la vitamine C dans un comprimé
		\item[TP25]: Exploitation d'un diagramme potentiel-pH~: méthode de
		\textsc{Winkler}
	\end{itemize}
\end{multicols}

\subsection{Thermodynamique}

\begin{multicols}{2}
	\subsubsection{Chapitres}

	\begin{itemize}[label=$\diamond$]
		\item[T1]: Description d'un système à l'équilibre
		\item[T2]: Premier principe de la thermodynamique
	\end{itemize}

	\begin{center}
		\textit{\Large Vacances}
	\end{center}

	\begin{itemize}
		\item[T3]: Second principe et machines thermiques
		\item[T4]: Changements d'états
	\end{itemize}

	\columnbreak

	\subsubsection{Travaux pratiques}

	\begin{itemize}[label=$\diamond$]
		\item[TP26]: Mesures de capacités thermiques~: calorimétrie et méthode de
		\textsc{Regnault}
		\item[TP27]: Équilibre liquide-vapeur de l'eau
		\item[TP28]: Mesures d'une enthalpie de changement d'état
	\end{itemize}
\end{multicols}

\newpage

\subsection{Architecture de la matière, partie 2}

\begin{multicols}{2}
	\subsubsection{Chapitres}

	\begin{itemize}[label=$\diamond$]
		\item[AM3]: Solides cristallins
	\end{itemize}

	\columnbreak

	\subsubsection{Travaux pratiques}

	\begin{itemize}[label=$\diamond$]
		\item[TP29]: Observation numérique de cristaux
	\end{itemize}

\end{multicols}

\subsection{Induction}

\begin{multicols}{2}
	\subsubsection{Chapitres}

	\begin{itemize}[label=$\diamond$]
		\item[I1]: Champs magnétiques
		\item[I2]: Actions mécaniques des champs mag.
		\item[I3]: Lois de l'induction et induction de \textsc{Neumann}
		\item[I4]: Conversion électromécanique
	\end{itemize}

	\columnbreak

	\subsubsection{Travaux pratiques}

	\begin{itemize}[label=$\diamond$]
		\item[TP30]: Mesures de champs magnétiques~: Terre et solénoïde
		\item[TP31]: Mesures de champs magnétiques~: \textsc{Helmoltz} et
		\textsc{Cotton}
	\end{itemize}

\end{multicols}

\subsection{Mécanique quantique}
\begin{itemize}[label=$\diamond$]
	\item[MQ1]: Introduction à la mécanique quantique
\end{itemize}

\end{document}
