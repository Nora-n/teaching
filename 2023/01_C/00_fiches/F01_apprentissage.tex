Dans le long chemin pour devenir expert-e dans une pratique, nous pouvons
distinguer trois éléments primordiaux~:
\begin{enumerate}
	\item la répétition avec retour~;
	\item la présence d'un environnement sain (et prédictif\footnote{Sans
		      détailler dans le corps de texte, faire de nombreux paris à la
		      roulette permet la répétition et le retour (est-ce que je gagne ou
		      perd), mais l'environnement est aléatoire et ne permet pas d'affiner
		      nos pratiques. Dans nos domaines une même action amène à un même
		      résultat.})~;
	\item la recherche personnelle.
\end{enumerate}

Ces trois éléments permettent d'atteindre la capacité de repérage de motifs et
de schémas propres à une discipline, trait caractéristique des expert-es. En
effet, un-e Grand-e Maître-sse d'échecs a une performance bien supérieure qu'une
personne lambda pour mémoriser la position des pièces d'un échiquier si elles
sont dans une position réaliste, mais aucune différence notable n'existe entre
les deux quand elles sont placées aléatoirement sur celui-ci. Je vous propose de
détailler ces points-là.

\subsection{Répétition avec retour}

Si vous pratiquez un instrument, un art ou un sport (e-sport compris), vous
savez bien qu'il ne suffit pas de regarder une vidéo d'une personne pratiquant
la même chose que ce que vous voulez faire pour vous réveiller le lendemain et
être aussi performant-e. Vous savez qu'avant de faire proprement du ski, faire
une figure en skate ou jouer dans un concert, il vous faut pratiquer avec
assiduité et répéter les actions~: c'est pareil avec le corps qu'avec l'esprit.
En physique-chimie et en mathématiques, il arrive un moment dans la pratique où
la répétition de raisonnements et de calculs forment un tout qui permet de
rapidement identifier ce qu'il se passe, de supposer les hypothèses desquelles
nous partons, et même prévoir le résultat sous une certaine forme (et donc
savoir à l'avance quand quelque chose ne fonctionne pas). D'une manière
générale, la mémoire repose sur la répétition~: il est estimé (\textit{via} la
«~courbe d'oubli~»\footnote{\textit{forgetting curve} en anglais, cf. les
	travaux de Hermann \textsc{Ebbinghaus}.}) qu'une heure après l'acquisition
d'une connaissance il ne nous en reste que 50\%, et 30\% après un jour. Le
secret de la mémoire tient dans la répétition espacée, comme un muscle se forme
par l'utilisation répétée et espacée.

Cependant, répéter la même action en boucle sans analyse et confrontation avec
un avis extérieur ne suffit pas à développer une expertise. Dans le cas de la
physique, nous pouvons heureusement voir si le résultat est correct ou non,
notamment avec l'aide des professeur-es et qui permettent la rétroaction
nécessaire à votre compréhension. Mais cette répétition cache quelque chose dont
nous ne parlons que trop rarement~: répéter implique de se tromper. Il y a une
grande, trop grande crainte au fait de se tromper quand il est question de
compétences intellectuelles. Cette crainte est le principal frein à un
apprentissage sain. En réalité, qui n'a jamais échoué n'a jamais essayé
suffisamment. Et pour cela l'environnement d'apprentissage est primordial.

\subsection{Environnement d'apprentissage}

Plus que normal, il est nécessaire de faire des fautes. Il est important
d'essayer~: la connaissance même du monde se base sur l'échec, je dirais même
qu'il en est principalement constitué. Chaque pratique qui ne fonctionne pas
nous permet de l'analyser pour intégrer la raison de cet échec et affiner cette
pratique dans une direction de plus en plus polie. Par exemple, je vais sans
doute me tromper en disant quelque chose, en écrivant au tableau ou en faisant
un calcul, et à chaque fois nous prendrons l'opportunité d'étudier comment il
est possible de voir pourquoi telle chose était fausse et comment la corriger~:
il en va de même pour vous, chacune de vos tentative est valide et vous mènera
vers la réussite.

Dans ce contexte, je considère la zone devant le tableau non pas celle du ou de
la professeur-e, mais comme une zone géographique de création de savoir, peu
importe qui l'occupe. J'attends de tout le monde ici présent d'avoir la même
attitude, et de respecter les efforts de chaque individu s'essayant à la
pratique de la physique et de la chimie. Ne soyez pas condescendant-es, déjà
dans vos pensées mais absolument jamais dans vos actes ou paroles si un-e de vos
camarades ne répond pas correctement à une question ou fait une faute au
tableau. Je ne le ferai jamais dans mon cas.

\subsection{Recherche personnelle}

Comme énoncé plus haut, il ne suffit pas d'écouter du \textsc{Chopin} pour jouer
du \textsc{Chopin}, mais même en répétant à l'infini le même morceau dans un
environnement sain avec un retour sur votre performance vous ne deviendrez pas
un-e expert-e~: il faut pour cela vous attaquer à différents morceaux,
différents styles, faire varier vos conditions de travail et développer vos
capacités de reconnaissance de motifs musicaux.

Il en va de même avec la pratique scientifique. Vous devez pratiquer avec effort
et vous approprier la connaissance que nous créons ensemble en étant partie
active de cet apprentissage. Le système des classes préparatoires vous aide dans
ce sens avec les khôlles hebdomadaires que vous allez effectuer, en vous
plongeant dans le rôle du ou de la transmetteur-ice, ou avec les devoirs maison
que je vous demanderai d'effectuer. Mais le plus important se passe en-dehors de
ces moments-là, lors de vos temps de révision.

Il est indispensable que vous relisiez vos cours, prépariez vos TDs et vos
khôlles. Si ceci est théoriquement faisable seul-e, la meilleure manière
d'apprendre est encore d'enseigner. Je vous invite donc fortement à vous
rapprocher de vos camarades (le principe du groupe de khôlle), peu importe leur
niveau, pour échanger avec elleux sur ce qui n'est pas bien compris. Posez-vous
des questions entre vous, regardez comment les ressources dont vous disposez
vous permettent de partir d'un point A de la réflexion à un point B, notamment
sur les démonstrations, regardez des vidéos sur la science en essayant de
prédire le phénomène mis en jeu ou encore expliquez à votre chat le cours sur
lequel vous serez interrogé-es en khôlle.

Il vous faut sortir de votre zone de confort pour explorer et affiner les
compétences que vous apprenez. Alors, et seulement alors, la pratique que vous
travaillez pourra être agréable et la connaissance accumulée source de fierté et
de plaisir.

\subsection{Résumé}
\subsubsection{En classe}
\begin{itemize}
	\item Posez des questions si vous ne comprenez pas~;
	\item Soyez \textbf{attentif-ves} (plus d'attention en classe = moins de
	      travail seul-e dans son coin après)~;
	\item Soyez organisé-es~: bloc-notes, trieur, pochettes plastifiées avec
	      code couleur pour avoir les cours et TDs pertinents et séparés…
	\item Soyez efficaces dans votre prise de note~: établissez des codes
	      couleurs, des abréviations, ne faites pas tous les schémas à la règle du
	      premier coup… Écouter et écrire divise l'attention. Je n'assure
	      absolument pas de vous fournir tous les documents de cours\footnote{Par
		      contre, tous les documents distribués en cours seront sur le site
		      \href{https://cahier-de-prepa.fr/mpsi3-pothier/}
		      {https://cahier-de-prepa.fr/mpsi3-pothier/}.}.
\end{itemize}

\subsubsection{En dehors}
\begin{itemize}
	\item \textbf{Relisez votre cours le soir-même}, ajoutez des annotations,
	      refaites les schémas, commencez à mémoriser~;
	\item Travaillez avec vos camarades pour préparer les TDs, réviser les
	      khôlles, posez-vous des questions entre vous~;
	\item Soyez actif-ves pendant vos séances, cherchez à comprendre, testez
	      votre compréhension~;
	\item Faites \underline{vos propres fiches pour chaque chapitre}~:
	      nécessaire et obligatoire pour réviser les concours~;
	\item \textbf{Dormez suffisamment (8 heures) et levez-vous suffisamment tôt
		      !!} Ne surestimez pas votre capacité à faire des courtes nuits et à
	      rester efficaces le lendemain. Ça ne sert à rien de venir en cours si
	      c'est pour dormir…
\end{itemize}

\subsection{En supplément}

\begin{itemize}
	\item Plein de ressources sur YouTube~;
	\item Appli \texttt{Qmax} sur Android~;
	\item Le site et l'application \texttt{Brilliant} (en anglais).
\end{itemize}

