\documentclass[a4paper, 10pt, garamond]{book}
\usepackage{cours-preambule}

\begin{document}

\chapter{Motivation, procrastination et distractions}

% author: Jeffrey Kaplan
% https://www.youtube.com/watch?v=i2EEnJedcYU

Ce sujet est crucial pour votre réussite académique~: comment vaincre la
procrastination. Avant de plonger dans les détails, voici un aperçu rapide de ce
que nous allons explorer.

\section{La procrastination et les émotions}

La procrastination, ce phénomène qui nous empêche de faire ce que nous devrions,
n'est pas simplement un problème de gestion du temps, mais plutôt de gestion des
émotions. Nous connaissons tous ce sentiment de remettre à plus tard des tâches
importantes. Aujourd'hui, je présente trois méthodes pour surmonter ce problème.
La première étape consiste à comprendre que la procrastination est liée à des
émotions négatives, comme la peur de l'échec ou l'ennui.

\section{Méthodes pour combattre la procrastination}

\subsection{Récompenses progressives}

Une première méthode consiste à découper la tâche en petites parties et à vous
récompenser à chaque étape accomplie. Vous avez peut-être entendu parler de
l'effet de la carotte et du bâton. Dans ce cas, la carotte est une petite
récompense que vous vous offrez à chaque progression. Par exemple, à la fin de
chaque section terminée, accordez-vous une pause pour savourer une friandise,
prendre une pause ou faire ce qui vous détend.

\subsection{Éliminer les tentations}

La deuxième méthode est de supprimer les distractions et les tentations qui nous
éloignent de nos tâches. Imaginez que vous travaillez sur un devoir important,
mais votre téléphone est à portée de main, prêt à vous distraire avec des
notifications et des divertissements en ligne. En éliminant ces tentations, vous
supprimez les obstacles qui vous empêchent de vous concentrer.

\subsection{Récolter la motivation}

Passons à la méthode que j'appelle «~la récolte de motivation~». Imaginez-vous
dans un groupe de travail où vous vous engagez à accomplir des tâches ensemble.
Ce concept a été éprouvé et testé. Par exemple, lorsque j'étais en études
supérieures, je faisais partie d'un groupe d'étude où nous nous retrouvions
régulièrement pour travailler en silence. À la fin de chaque période de travail,
nous partagions nos réalisations. Cette pression positive de ne pas être
l'élément déconcentré du groupe nous motivait à rester concentré-es et
productif-ves.

\section{Exemples concrets et anecdotes}

Laissez-moi illustrer ces idées avec une anecdote personnelle. Durant ma
première année d'université, je me suis rendu compte que si je rentrais dans le
dortoir où se trouvait la console de jeu avec le jeu «~Mario Kart~», j'allais
finir par y jouer toute la journée au lieu de travailler. C'est pourquoi j'ai
choisi de ne pas retourner au dortoir pendant la journée. Cette décision a eu un
impact positif sur ma réussite académique.

\section{Conclusion et récapitulation}

En résumé, la procrastination n'est pas simplement une question de gestion du
temps, mais elle est liée à nos émotions. Nous avons discuté de trois méthodes
pour lutter contre la procrastination~: les récompenses progressives,
l'élimination des tentations et la récolte de motivation. En appliquant ces
techniques, vous pouvez transformer vos émotions négatives en moteurs de
productivité. N'oubliez pas que vous n'êtes pas seul dans ce combat contre la
procrastination. Trouvez des partenaires d'étude ou des collègues pour
travailler ensemble, et vous verrez que la motivation peut se propager.

\section{Points clés à retenir}

\begin{itemize}
	\item La procrastination découle souvent d'émotions négatives liées à la
	      tâche à accomplir.
	\item Les récompenses progressives, l'élimination des distractions et la
	      récolte de motivation sont des méthodes efficaces pour lutter contre la
	      procrastination.
	\item Travailler en groupe peut augmenter votre motivation et votre
	      productivité.
\end{itemize}

Je vous encourage à mettre en pratique ces méthodes et à observer les changements
positifs dans votre façon de gérer votre temps et vos tâches. En fin de compte,
la clé réside dans la gestion de vos émotions pour devenir plus productifs et
atteindre vos objectifs académiques.


\chapter{Arrêtez de perdre votre temps (votre cerveau a un antidote)}

% Fabien Olicard
% https://www.youtube.com/watch?v=ogMSk0-iHa4

Aujourd'hui, nous allons aborder un sujet essentiel pour notre productivité :
la procrastination. Vous savez, ce sentiment qui nous pousse à remettre à plus
tard ce que nous pourrions accomplir aujourd'hui. Eh bien, ne vous inquiétez
pas, car aujourd'hui je vais vous présenter des stratégies simples mais
efficaces pour combattre la procrastination et optimiser votre temps.

\section{Comprendre la Procrastination}

Avant de plonger dans les astuces, il est crucial de comprendre ce qu'est la
procrastination. Procrastiner, c'est remettre à demain ce que nous pourrions
réaliser aujourd'hui. Cela se traduit souvent par l'utilisation excessive de
nos téléphones, comme lorsqu'on commence à envoyer un message important et
finit par passer des heures sur TikTok sans avoir envoyé le message en
question.

\section{La Méthode des Petits Pas}

L'une des raisons principales de la procrastination est le stress et l'angoisse
associés à la taille apparente d'une tâche à accomplir. Une excellente
stratégie consiste à utiliser la méthode des petits pas. Cette méthode implique
de découper une grande tâche en de petites étapes réalisables. Par exemple, si
vous devez préparer une recette, commencez par dresser une liste de courses.
Ensuite, accomplissez chaque étape une par une. Cette approche réduit le stress
et augmente votre sentiment d'accomplissement à chaque étape franchie.

\section{L'Art des Listes}

Les listes sont vos meilleures amies pour lutter contre la procrastination.
Lorsque vous utilisez la méthode des petits pas, notez chaque étape sur une
liste physique, pas sur votre téléphone. Lorsque vous cochez une tâche
accomplie, votre cerveau libère de la dopamine, l'hormone du bonheur. Cette
sensation positive renforce votre motivation à continuer. Classez les tâches par
ordre d'importance et d'urgence pour une meilleure organisation.

\section{Savoir Dire Non}

Apprendre à dire non est une compétence clé pour éviter la surcharge de travail.
Avant d'accepter une nouvelle tâche, prenez le temps de considérer si elle est
vraiment essentielle. Dire non vous permettra de préserver votre temps pour des
activités qui vous tiennent réellement à cœur.

\section{Bloquer du temps et aller au bout}

Il faut savoir se réserver un moment à part pour nos projets personnels. Bloquer
deux heures dans la semaine dédiées à ce qui vous tient à cœur est un excellent
moyen de garantir que vous progresserez régulièrement. Cette habitude vous
libère mentalement et vous permet de vous consacrer pleinement à vos
aspirations.

\section{Points Clés}

Avant de conclure, voici quelques points clés à retenir:
\begin{itemize}
	\item La procrastination est le report de tâches à plus tard.
	\item La méthode des petits pas consiste à découper les tâches en étapes
	      réalisables.
	\item Utilisez des listes physiques pour visualiser vos progrès et stimuler la
	      dopamine.
	\item Apprenez à dire non pour préserver votre temps.
	\item Bloquez du temps chaque semaine pour vos projets personnels.
\end{itemize}

N'oubliez pas, chers étudiants, que même les plus grands esprits ont dû lutter
contre la procrastination. En utilisant ces astuces, vous pouvez forger un
chemin vers la réalisation de vos ambitions. Je vous encourage à intégrer ces
pratiques dans votre quotidien. En fin de compte, la procrastination peut être
surmontée par une combinaison de stratégies simples et de compréhension de nos
processus mentaux. Rappelez-vous que vous avez le pouvoir de transformer vos
intentions en actions concrètes.

\chapter{Le vrai ennemi des études}

% AsapSCIENCE
% https://www.youtube.com/watch?v=vcjQ5JkEE_0

À l'ère numérique actuelle, les smartphones sont devenus une partie intégrante
de nos vies, nous permettant de nous connecter, d'apprendre et d'accéder à
l'information comme jamais auparavant. Cependant, il est essentiel de
reconnaître les éventuels inconvénients d'une utilisation excessive du
téléphone, en particulier en ce qui concerne son impact sur notre motivation. Ce
document vise à mettre en lumière comment votre utilisation du téléphone
pourrait affecter votre motivation et à fournir des informations pour atténuer
ces effets.

\section{Les Statistiques Écrasantes}

Il y a à peine 15 ans, seulement 20 \% des personnes accédaient à Internet
depuis leur téléphone. Aujourd'hui, ce nombre a grimpé en flèche à 91 \%.
L'adulte moyen passe environ 11 heures par jour à interagir avec les médias. Ces
statistiques indiquent un changement significatif dans la manière dont nous
consommons de l'information et nous connectons avec les autres, mais cela
soulève également des préoccupations quant à la manière dont cette connectivité
constante affecte notre bien-être.

\section{Le Rôle de la Dopamine}

La dopamine est un neurotransmetteur souvent associé au plaisir et à la
récompense. Elle est essentielle pour motiver des comportements bénéfiques tels
que manger, socialiser et rechercher de nouvelles expériences. Votre smartphone
déclenche la libération de dopamine dans votre cerveau, vous faisant ressentir
une récompense lorsque vous recevez des notifications, des likes ou que vous
interagissez avec du contenu divertissant. Avec le temps, ces activités
renforcent les voies neuronales associées à la libération de dopamine,
entraînant des comportements compulsifs et réduisant votre motivation à
participer à d'autres activités nécessitant effort et persévérance.

\section{Le Dilemme de la Motivation}

La recherche montre qu'une utilisation excessive du téléphone peut entraîner une
diminution de l'attention, des difficultés à se concentrer et une capacité
réduite à différer la gratification. Ce phénomène, appelé « décote de délai »,
peut entraîner une diminution de la motivation pour les tâches nécessitant un
effort soutenu ou offrant des récompenses différées. En réalité, passer des
heures prolongées devant des écrans a été lié à un risque accru de dépression et
d'anxiété, en particulier chez les jeunes adultes. Les adolescents qui passent
trop de temps sur leurs appareils mobiles ont 71 \% de risques en plus de
développer des facteurs de risque de suicide.

\section{Reconnaître l'addiction au Téléphone}

Comprendre si vous êtes dépendant de votre téléphone est essentiel pour atténuer
son impact sur votre motivation. Posez-vous les questions suivantes :
\begin{enumerate}
	\item Avez-vous des envies de vérifier votre téléphone au détriment d'autres
	      activités ?
	\item Votre humeur change-t-elle en fonction des notifications et des
	      interactions sur les réseaux sociaux ?
	\item Trouvez-vous nécessaire de passer plus de temps sur votre téléphone pour
	      être satisfait ?
	\item Vous sentez-vous mal à l'aise ou stressé lorsque vous n'utilisez pas
	      votre téléphone ?
	\item Vos tentatives pour réduire l'utilisation du téléphone ont-elles
	      entraîné des rechutes ?
\end{enumerate}

Si vous avez répondu « oui » à ces questions, vous pourriez faire face à une
dépendance au téléphone.

\section{Prendre le Contrôle}

La bonne nouvelle est que vous pouvez reprendre le contrôle de votre utilisation
du téléphone et de votre motivation. Voici trois stratégies soutenues par la
science à envisager :
\begin{enumerate}
	\item \textbf{Restriction du Temps :} Allouez des plages horaires spécifiques
	      pour l'utilisation du téléphone, comme une heure par jour. Cette
	      approche empêche les vérifications compulsives et permet aux systèmes de
	      dopamine de votre cerveau de se réajuster progressivement.
	\item \textbf{Contrainte Physique :} Déconnectez-vous des applications
	      déclencheuses, confiez vos mots de passe à un ami de confiance ou placez
	      physiquement votre téléphone hors de portée à certains moments, comme
	      l'éteindre à 21 heures.
	\item \textbf{Contrainte Catégorique :} Rendez votre téléphone moins attractif
	      en utilisant le mode gris, en vérifiant les applications à forte teneur en
	      dopamine uniquement sur un ordinateur et en supprimant les applications
	      inutiles qui gaspillent votre temps.
\end{enumerate}

\section{Conclusion}

Dans un monde où les smartphones sont omniprésents, comprendre l'impact d'une
utilisation excessive du téléphone sur votre motivation est crucial. En
reconnaissant le rôle de la dopamine, en étant conscient des signes de
dépendance au téléphone et en mettant en œuvre des stratégies pour contrôler
votre utilisation du téléphone, vous pouvez retrouver la concentration, la
motivation et une relation plus saine avec la technologie.

N'oubliez pas que vous n'êtes pas seul dans ce voyage. De nombreuses personnes
cherchent à trouver un équilibre entre les avantages de la technologie et leur
bien-être global. En prenant des mesures pour limiter l'utilisation de votre
téléphone, vous investissez dans votre succès futur et votre bien-être.



\end{document}
