\documentclass[../main/main.tex]{subfiles}

\makeatletter
\renewcommand{\@chapapp}{Introduction -- chapitre}
\makeatother

% \toggletrue{student}
% \HideSolutionstrue

\begin{document}
\setcounter{chapter}{-1}

\chapter{Unit\'es et analyse dimensionnelle}

\begin{prgm}
	\begin{tcb}*(ror)"how"{Savoir-faire}
		\begin{itemize}[label=$\diamond$, leftmargin=10pt]
			\item Conduire une analyse dimensionnelle.
		\end{itemize}
	\end{tcb}
\end{prgm}

En sciences physiques, il faut opérer la distinction entre~:
\begin{enumerate}
	\item \textbf{Le phénomène}~: \psw{
		      du domaine de la sensation («~il fait chaud~»)
		      ou de l'observation («~une lumière blanche traversant un prisme sort en
		      arc-en-ciel~»)~;
	      }
	\item \textbf{La grandeur physico-chimique}~: \psw{
		      quantité mesurable (directement ou
		      indirectement) et rendant compte du phénomène (par exemple, la
		      température). Elle est représentée par un \textbf{symbole} ($T$ dans le
		      même exemple) qui intervient dans les équations mathématiques
		      caractérisant le phénomène physique, et est caractérisée par sa
		      \textbf{dimension} traduisant sa nature~;
	      }
	\item \textbf{La valeur} de la grandeur~: \psw{
		      résultant d'une mesure et associée
		      à une \textbf{unité} ($K$ ou $\degreeCelsius$)~; la valeur change selon
		      l'unité utilisée et contient un certain nombre de \textbf{chiffres
			      significatifs}.
	      }
\end{enumerate}

Ces notions sont la fondation de tout raisonnement scientifique qui repose sur
la précision et l'objectivité.

\section{Systèmes d'unités}
\subsection{Grandeurs de base}

Les grandeurs physiques sont \textbf{reliées} entre elles, soit par des
\textbf{définitions} (surface d'un carré = carré d'un côté) soit par des
\textbf{lois} physiques ($U = RI$ en électronique). Par souci de concision, il
est pratique de choisir des grandeurs de base à partir desquelles nous
exprimerons toutes les autres~: en mécanique par exemple, nous utilisons la
longueur, la masse et le temps. Ce choix n'est pas unique mais pratique.

À partir de grandeurs de base choisies, nous leur associons donc des unités «~de
base~». Le bureau international des poids et mesures
(BIPM\footnote{\href{https://www.bipm.org/fr/measurement-units/}
	{https://www.bipm.org/fr/measurement-units/}}) a défini le \textbf{système
	international (SI)}, et se réunit tous les 4 ans pour discuter de leurs
définitions et de leurs choix.

\subsection{Définition du SI}

\begin{tcb}[label=def:si](defi){Grandeurs, dimensions et unités de base du SI}
	\begin{center}
		\begin{tabular}{lclc}
			\toprule
			Grandeur             & Dimension      & Unité            & Symbole de l'unité \\
			\midrule
			Longueur             & \psw{L}        & \psw{mètre}      & \psw{m}            \\
			Masse                & \psw{M}        & \psw{kilogramme} & \psw{kg}           \\
			Temps                & \psw{T}        & \psw{seconde}    & \psw{s}            \\
			Intensité électrique & \psw{I}        & \psw{Ampère\fnm} & \psw{A}            \\
			Température          & \psw{$\Theta$} & \psw{Kelvin\fnm} & \psw{K}            \\
			Quantité de matière  & \psw{N}        & \psw{mole}       & \psw{mol}          \\
			Intensité lumineuse  & \psw{J\fnm}    & \psw{candela}    & \psw{cd}           \\
			\bottomrule
		\end{tabular}
	\end{center}
\end{tcb}

\footnotetext[2]{Du nom du physicien André-Marie \textsc{Ampère}
	(\textsc{xviii-xix}\ieme), précurseur de la mathématisation de la physique
	et créateur du vocabulaire tenant à l'électricité.\label{fn:amp}}
\footnotetext[3]{Du nom du physicien William
	\textsc{Thomson} (\textsc{xix}\ieme), anobli en Lord \textsc{Kelvin}, à
	l'origine de la thermodynamique.\label{fn:kel}}
\footnotetext[4]{Ne pas confondre avec l'unité des énergies en joules…\label{fn:can}}
On remarquera que les unités provenant d'un nom propre s'écrivent avec une
majuscule, et leur symbole l'est également.

\begin{tcb}(nota){Notation}
	On utilisera $\dim{X}$ pour dénoter la dimension de $X$, et $[X]$ son unité.
\end{tcb}

\subsection{Opérations sur les grandeurs}
D'une manière générale, vous étudierez les dimensions de vos équations
directement \textit{via} les opérateurs qui la composent. Il faut donc savoir
déduire les dimensions dans les cas suivants~:
% \begin{enumerate}
% 	\bitem{Somme et différence}~: \psw{
% 		tous les termes d'une somme ou d'une différence ont la même dimension, et le
% 		résultat également
% 	}
% 	\bitem{Produit et quotient}~: \psw{
% 		on peut multiplier ou diviser des grandeurs de n'importe quelles dimensions.
% 		La dimension d'un produit est le produit des dimensions, de même pour les
% 		quotients et puissances.
% 	}
% 	\bitem{Dérivation}~: \psw{
% 		la dimension d'une dérivée est le rapport des dimensions de la grandeur
% 		dérivée et de la grandeur dérivante.
% 	}
% 	\bitem{Intégration}~: \psw{
% 		la dimension d'une intégrale est le produit des dimensions de la grandeur
% 		intégrée et de la grandeur intégrante.
% 	}
% 	\bitem{Fonction transcendantes}\ftn{Fonctions type exponentielle, logarithme,
% 		cosinus.}~: \psw{
% 		une fonction transcendante est adimensionnée, donc son argument également
% 	}
% \end{enumerate}
\begin{tcbraster}[raster equal height=rows]
	\begin{tcb}(prop){Opérations}
		\begin{enumerate}
			\bitem{Dérivation}~: \psw{
				la dimension d'une dérivée est le rapport des dimensions de la grandeur
				dérivée et de la grandeur dérivante.
			}
			\bitem{Intégration}~: \psw{
				la dimension d'une intégrale est le produit des dimensions de la grandeur
				intégrée et de la grandeur intégrante.
			}
			\bitem{Fonction transcendantes}\ftn{Fonctions type exponentielle, logarithme,
				cosinus.}~: \psw{
				une fonction transcendante est adimensionnée, donc son argument également
			}
		\end{enumerate}
	\end{tcb}
	\begin{tcb}(exem)'r'{Exemples}
		\begin{enumerate}
			\bitem{Dérivation}~:
			\psw{
				\[
					v_z =
					\dv{z}{t} \Ra \dim{v_z} =
					\frac{\dim{z}}{\dim{t}} =
					\rm L \cdot T^{-1}
				\]
			}
			\vspace*{-10pt}
			\bitem{Intégration}~:
			\psw{
				\[
					\Ec = \int_{t_1}^{t_2} \Pc(t)\dd{t} \Ra \dim{\Ec} = \dim{\Pc} \times \dim{t}
				\]
			}
			\vspace*{-10pt}
			\bitem{Fonctions transcendantes}~:
			\psw{
				\[
					[\exp(-t/\tau)] = 1 \Lra \dim{\tau} = \dim{t}
				\]
			}
			\vspace*{-10pt}
		\end{enumerate}
	\end{tcb}
\end{tcbraster}

\subsection{Grandeurs dérivées}

Les grandeurs exprimées à partir des grandeurs de bases \textit{via} des
équations physiques sont appelées «~grandeurs dérivées~». Leurs dimensions sont
écrites sous la forme de produits de puissances des dimensions de base~: d'une
manière générique, une grandeur $G$ a pour dimension

\psw{
	\[
		\boxed{\dim{G} = \rm L^\alpha M^\beta T^\gamma I^\delta \Theta^\epsilon N^\xi
			J^\eta}
	\]
}
où les lettres grecques sont les \textbf{exposants dimensionnels}, qui peuvent
être nuls. S'ils sont tous nuls, la grandeur et dite \textbf{adimensionnée}.

\begin{tcb}[label=exem:grandeurs](exem){Grandeurs dérivées}
	\begin{center}
		\begin{tabular}{lcll}
			\toprule
			Grandeurs dérivées & Symbole  & Équation aux dimensions                  & Unités SI dérivées                      \\
			\midrule
			Surface            & $S$      & $\dim{S}      = \psw{\si{L^2}}$          & $ [S]      = \psw{\si{m^2}}           $      \\
			Volume             & $V$      & $\dim{V}      = \psw{\si{L^3}}$          & $ [V]      = \psw{\si{m^3}}           $      \\
			Angle              & $\alpha$ & $\dim{\alpha} = \psw{\si{1}}$            & $ [\alpha] = \psw{\si{rad}}           $ \\
			Vitesse            & $\vf$    & $\dim{v}      = \psw{\si{L.T^{-1}}}$     & $ [v]      = \psw{\si{m.s^{-1}}}      $      \\
			Accélération       & $\af$    & $\dim{a}      = \psw{\si{L.T^{-2}}}$     & $ [a]      = \psw{\si{m.s^{-2}}}      $      \\
			Masse volumique    & $\rho$   & $\dim{\rho}   = \psw{\si{M.L^{-3}}}$     & $ [\rho]   = \psw{\si{kg.m^{-3}}}     $   \\
			Force              & $\Ff$    & $\dim{F}      = \psw{\si{M.L.T^{-2}}}$  & $ [F]      = \psw{\si{kg.m.s^{-2}}}   $      \\
			Charge électrique  & $q$      & $\dim{q}      = \psw{\si{I.T}}$          & $ [q]      = \psw{\si{A.s}}           $      \\
			Énergie            & $\Ec$    & $\dim{E}      = \psw{\si{M.L^2.T^{-2}}}$ & $ [E]      = \psw{\si{kg.m^2.s^{-2}}} $      \\
			\bottomrule
		\end{tabular}
	\end{center}
\end{tcb}
\begin{tcb}(rema){Remarque}
	Certaines de ces unités dérivées portent des noms usuels~: le newton N
	($\SI{1}{N} = \SI{1}{kg.m.s^{-2}}$) pour la force, le coulomb C ($\SI{1}{C} =
	\SI{1}{A.s}$) pour la charge électrique, ou l'énergie en joules\footnote{Du
		physicien James \textsc{Joule} (\textsc{XIX}\ieme), contemporain de
		\textsc{Kelvin}.} J ($\SI{1}{J} = \SI{1}{kg.m^2.s^{-2}}$)
\end{tcb}

% \subsection{Préfixes multiplicatifs}
%
% Suivant la valeur d'une grandeur, il est commode de l'exprimer \textit{via}
% l'ajout d'un préfixe à l'unité. Ils s'expriment en puissances de 10 et ont
% également un symbole et un nom~:
% \begin{center}
% 	\begin{tabular}{lcclcc}
% 		\toprule
% 		\multicolumn{3}{c}{Sous-multiples} & \multicolumn{3}{c}{Multiples}                                               \\
% 		\cmidrule(lr){1-3} \cmidrule(lr){4-6}
% 		Préfixe                            & Puissance                     & Symbole     & Préfixe & Puissance & Symbole \\
% 		\midrule
% 		yocto                              & \num{e-24}                    & y           & déca    & \num{e1}  & da      \\
% 		zepto                              & \num{e-21}                    & z           & hecto   & \num{e2}  & h       \\
% 		atto                               & \num{e-18}                    & a           & kilo    & \num{e3}  & k       \\
% 		femto                              & \num{e-15}                    & f           & méga    & \num{e6}  & M       \\
% 		pico                               & \num{e-12}                    & p           & giga    & \num{e9}  & G       \\
% 		nano                               & \num{e-9}                     & n           & téra    & \num{e12} & T       \\
% 		micro                              & \num{e-6}                     & \si{\micro} & péta    & \num{e15} & P       \\
% 		milli                              & \num{e-3}                     & m           & exa     & \num{e18} & E       \\
% 		centi                              & \num{e-2}                     & c           & zetta   & \num{e21} & Z       \\
% 		déci                               & \num{e-1}                     & d           & yotta   & \num{e24} & Y       \\
% 		\bottomrule
% 	\end{tabular}
% \end{center}

\section{Analyse dimensionnelle}

À l'aide de ces outils, nous pouvons effectuer des actions sur les
équations-mêmes pour en extraire les dimensions. Pour qu'une équation
mathématique ait un sens physique, elle doit suivre un principe fondamental et
naturel~: le \textbf{principe d'homogénéité}.

\subsection{Homogénéité}

\begin{tcb}(prop){Homogénéité}
	Dans une équation ou dans l'expression d'une loi physique, les deux
	membres de chaque côté du signe égal doivent être de \textbf{même
		nature}\footnote{Scalaire, vecteur, matrice, tenseur…} et avoir la
	\textbf{même dimension}, quel que soit le système d'unités. Une telle
	formule est alors dite \textbf{homogène}.
\end{tcb}
\begin{tcb}[cnt](coro){Corollaire~: natures des équations}
	\bfseries Il serait ainsi \textit{barbare} d'égaliser un vecteur d'un côté
	avec un scalaire de l'autre, ou d'additionner ou soustraire
	des mètres à des secondes, etc.
\end{tcb}

\subsection{Écrire un résultat}

Un objectif récurrent des sujets de physique-chimie est d'obtenir la
\textbf{valeur numérique} d'une grandeur physico-chimique. Elle découle alors
d'une équation, forcément homogène, mais doit également être calculée avec les
bonnes unités au sein des dimensions. Ainsi, \textbf{tout résultat numérique}
devra être rédigé sous la forme suivante~:
\begin{tcb}(impo){Règle d'application numérique}
	\vspace*{-10pt}
	\begin{minipage}{0.45\linewidth}
		\begin{gather*}
			\boxed{n = \frac{PV}{RT}}
			\qav
			\left\{
			\begin{array}{rcl}
				p & = & \SI{1.0e5}{Pa}                \\
				V & = & \SI{1.0e-3}{m^3}              \\
				R & = & \SI{8.314}{J.mol^{-1}.K^{-1}} \\
				T & = & \SI{300}{K}
			\end{array}
			\right.\\
			\mathrm{A.N.~:}\quad
			\xul{n = \SI{5.6e-4}{mol}}
		\end{gather*}
	\end{minipage}
	\hfill
	\cancel{\bcancel{
			\begin{minipage}{0.45\linewidth}
				\begin{gather*}
					n = \frac{PV}{RT} = \frac{\num{e5}\cdot\num{1}}{8.32\cdot300}
					= 0.56
				\end{gather*}
			\end{minipage}
		}}
	\smallbreak
	Avec ces règles de mise en page doivent venir des réflexes~:
	\smallbreak
	\begin{isd}
		\tcbsubtitle{\fatbox{Encadrer}}
		Encadrer implique d'avoir vérifié~:
		\begin{enumerate}
			\item La cohérence mathématique~;
			\item L'homogénéité de la formule proposée.
		\end{enumerate}
		\tcblower
		\tcbsubtitle{\fatbox{Souligner}}
		Souligner implique d'avoir vérifié~:
		\begin{enumerate}
			\item La cohérence physique de la grandeur~;
			\item Les chiffres significatifs à utiliser.
		\end{enumerate}
	\end{isd}
\end{tcb}

\begin{tcb}*(coro)"bulb"{Effectuer un changement d'unités}
	Il est très commun de se tromper d'unité lors d'une conversion, et ce pour
	deux raisons~: à cause d'une unité mise à une puissance, ou à cause d'un
	rapport de deux grandeurs. Il suffit d'appliquer le processus suivant~:
	\begin{enumerate}
		\item Écrire la valeur numérique actuelle de la grandeur avec son unité sous
		      forme de \textbf{fraction explicite}~;
		\item Convertir les unités concernées \textbf{en y mettant des
			      parenthèses}~;
		\item Recondenser le calcul.
	\end{enumerate}
\end{tcb}

\begin{tcb}[sidebyside](exem){Exemples}
	\vspace*{-10pt}
	\psw{
		\begin{gather*}
			\SI{1}{m.s^{-1}} =
			1 \frac{\si{m}}{\si{s}} =
			1 \frac{\SI{e-3}{km}}{\SI[parse-numbers=false]{\frac{1}{3600}}{h}} =
			\SI{3.6e-3}{km.h^{-1}}
		\end{gather*}
	}
	\vspace*{-10pt}
	\tcblower
	\psw{
		\begin{gather*}
			\SI{1}{L} = 1\,\si{dm}^{3} = 1\,(\num{e-1}\si{m})^{3} = \SI{1e-3}{m^{3}}
		\end{gather*}
	}
	\vspace*{-10pt}
\end{tcb}

\subsection{Application}

Le principe d'homogénéité permet alors une analyse des dimensions des grandeurs
mises en jeu dans une loi ou une équation. C'est un outil particulièrement
puissant à bien des égards, que nous voyons ci-après.

\subsubsection{Rechercher des unités}

En connaissant une expression que l'on sait vraie, nous pouvons déduire les
unités d'autres grandeurs (cf.\ les unités usuelles comme le Newton).

\begin{tcb}(exem){Recherche d'unités}
	La force de rappel élastique exercée par un ressort s'écrit
	\[\Ff_{\rm el} = -k(\ell-\ell_0)\ux\]
	avec $k$ la constante de raideur du ressort, et $\ux$ un vecteur adimensionné.
	Quelle est la dimension de $k$~? Quelle serait une manière simple d'exprimer
	son unité~?
	\tcblower
	\vspace*{-9pt}
	\begin{isd}
		\psw{
			\begin{gather*}
				\dim{k} =
				\frac{\dim{F}}{\dim(\ell-\ell_0)\dim{u_x}} =
				\frac{\rm M\cancel{\rm L}T^{-2}}{\cancel{\rm L}}
				\\ \Lra
				\boxed{\dim{k} = \si{M.T^{-2}}}
			\end{gather*}
		}
		\tcblower
		\psw{
			Ainsi, on peut écrire
			\begin{gather*}
				[k] = \si{kg.s^{-2}}
				\\\Lra
				[k] = \si{N.m^{-1}}
			\end{gather*}
		}
	\end{isd}
	\vspace*{-20pt}
\end{tcb}

\subsubsection{Détecter des erreurs}

Par simple analyse dimensionnelle, il est aisé d'affirmer qu'un résultat est
nécessairement faux~: si les deux parties mises en jeu n'ont pas la même
dimension, elle ne peuvent être égales entre elles~!

\begin{tcb}(exem){Détecter des erreurs}
	En résolvant un exercice, vous trouvez l'expression suivante pour l'énergie
	potentielle d'une masse $m$ accrochée à un ressort vertical de raideur $k$
	et sous pesanteur $g$~:
	\[\Ec_{\rm p} (z) = \frac{1}{2}kz^2 + mgz^2\]
	avec $z$ la hauteur de la masse. Cette expression est-elle homogène~?
	\tcblower
	\vspace*{-9pt}
	\begin{isd}[lefthand ratio=.4]
		\psw{
			\begin{gather*}
				[kz^2] = \si{N.m^{-1}.m^2} = \si{N.m}
				\\
				[mgz^2] = \si{N.m^2}
			\end{gather*}
		}
		\tcblower
		\psw{
			Cette expression n'est pas homogène~! Elle est forcément fausse. Ici,
			c'est le terme d'énergie potentielle gravitationnelle, qui vaut $mgz$.
		}
	\end{isd}
	\vspace*{-20pt}
\end{tcb}

\subsubsection{Rechercher des lois physiques}
D'autre part, à partir de phénomènes que nous voudrions relier entre eux, il est
possible d'établir des lois les reliant entre eux grâce au principe
d'homogénéité.

\begin{tcb}(exem){Recherche de loi}
	Donnez, par analyse dimensionnelle, la période $T$ des oscillations d'un
	pendule simple.
	\tcblower
	\psw{
		Il faut commencer par identifier les variables propices d'intervenir dans le
		problème. Une pendule simple est constitué d'une masse plongée dans un champ
		de pesanteur et reliée par un fil de longueur $\ell$ à un point pivot. En
		négligeant les frottements, on en déduit que les variables possibles sont
		$\ell$, $g$ et $m$~; ainsi, il nous faudrait avoir
	}
	\begin{isd}
		\psw{
			\begin{gather*}
				T = \ell^{\alpha}g^{\beta}m^{\gamma}
				\\\Lra
				[T] = [\ell]^{\alpha}[g]^{\beta}[m]^{\gamma}
				\\\Lra
				\si{s} = \si{m^{\alpha}.m^{\beta}.s^{-2\beta}.kg^{\gamma}}
				\\\Lra
				\left\{
				\begin{array}{rcl}
					\si{s} & = & \si{s}^{-2 \beta}
					\\
					1      & = & \si{m}^{\alpha+\beta}
					\\
					1      & = & \si{kg}^{\gamma}
				\end{array}
				\right.
				\Lra
				\left\{
				\begin{array}{rcl}
					\beta  & = & - \frac{1}{2}
					\\
					\alpha & = & \frac{1}{2}
					\\
					\gamma & = & 0
				\end{array}
				\right.
			\end{gather*}
		}
		\tcblower
		\psw{
			Autrement dit, on aurait tendance à écrire
			\[
				\boxed{T = \sqrt{\frac{\ell}{g}}}
			\]
			Ce qui serait effectivement homogène~! Mais pourtant faux… L'étude
			complète du système donne un \textbf{facteur} $\mathbf{2\pi}$ avant la
			racine.
		}
	\end{isd}
	\vspace*{-20pt}
\end{tcb}

\begin{tcb}[cnt, bld](impo){Limites implicites}
	Une loi trouvée par analyse dimensionnelle ne saurait permettre de donner les
	bons termes multiplicatifs~!
\end{tcb}

\end{document}
