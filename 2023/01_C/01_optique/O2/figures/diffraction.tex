% https://tex.stackexchange.com/a/599002

\documentclass[border=2mm]{standalone}
\usepackage    {tikz}
\usetikzlibrary{3d}    % For "canvas is..." options
\usetikzlibrary{babel} % There are conflicts between tikz and some babel packages
\usetikzlibrary{angles}

% isometric axes
\pgfmathsetmacro\yx{1/sqrt(2)}
\pgfmathsetmacro\yy{1/sqrt(6)}
\pgfmathsetmacro\xy{sqrt(2/3)}

\newcommand{\rectangle}[6]% position (z), x,y,z dimensions, color, label 
{%
  \draw[canvas is xy plane at z=#1,fill=#5,fill opacity=0.8] (-#2,-#3)   rectangle (#2,#3);
  \draw[canvas is xz plane at y=#3,fill=#5]                  (-#2,#1-#4) rectangle (#2,#1);
  \draw[canvas is yz plane at x=#2,fill=#5]                  (-#3,#1-#4) rectangle (#3,#1);
  \node at (#2,-#3,#1) [above left,rotate=30] {#6};
}
\begin{document}
\begin{tikzpicture}[line cap=round,line join=round,%
		x={(0 cm,\xy cm)},y={(-\yx cm,-\yy cm)},z={(\yx cm,-\yy cm)}]
	% Dimensions
	\def\pr{0.2} % LASER, radius
	\def\ph{2}   % LASER, height
	\def\rz{5}   % Obstacle, position (z)
	\def\ra{1.5} % Obstacle, semi-dimension x
	\def\rb{1.7} % Obstacle, semi-dimension y
	\def\rc{0.1} % Obstacle, dimension z
	\def\pz{10}  % Écran, position (z)
	\def\pa{\ra} % Écran, semi-dimension x
	\def\pb{3}   % Écran, semi-dimension y
	\def\pc{0.1} % Écran, dimension z
	% LASER
	\draw[top color=gray] (-45:\pr)     --++ (0,0,\ph)  node[sloped,midway,above]
	{LASER}
	arc (-45:135:\pr) --++ (0,0,-\ph) arc (135:-45:\pr);
	\draw[canvas is xy plane at z=\ph,fill=white] (0,0) circle (\pr);
	% Single ray
	\draw[red,ultra thick] (0,0,\ph) -- (0,0,\rz);
	% Obstacle
	\rectangle{\rz}{\ra}{\rb}{\rc}{brown!30}{Obstacle}
	% Triple rays
	% \foreach\i in {-1,0,1}
	% 	{%
	% 		\draw[red,thick] (0,0,\rz) -- (0,0.5*\pb*\i,\pz);
	% 	}
	\draw[red,]
	(0,0,\rz) coordinate (O) --
	(0,0.5*\pb*-1,\pz) coordinate (T);
	\draw[red,]
	(0,0,\rz) --
	(0,0.5*\pb*0,\pz) coordinate (M);
	\draw[red,] (0,0,\rz) -- (0,0.5*\pb*1,\pz);
	% Écran
	\rectangle{\pz}{\pa}{\pb}{\pc}{white}{Écran}
	\begin{scope}[canvas is xy plane at z=\pz]
		\foreach\i in {-1,0,1}
			{%
				\draw[dashed] (0,0.5*\pb*\i) --++   (-1.5*\pa,0);
				\fill[red]    (0,0.5*\pb*\i) circle (2pt);
				\ifnum \i < 1
					\draw[<->] (-1.5*\pa,0.5*\pb*\i) --++ (0,0.5*\pb) node [midway,below] {$\frac{L}{2}$};
				\fi
			}
	\end{scope}
	\draw[dashed] (-\ra,\rb,\rz-0.5*\rc)   -- (-\ra,\pb+1,\rz-0.5*\rc);
	\draw[dashed] (-\pa,\pb,\pz-0.5*\rc)   -- (-\pa,\pb+1,\pz-0.5*\pc);
	\draw[<->]    (-\ra,\pb+1,\rz-0.5*\rc) -- (-\pa,\pb+1,\pz-0.5*\pc) node[midway,below] {$D$};
	\node at (0,0,\pz) [above] {$O$};

	\begin{scope}[canvas is yz plane at x=\ra]
		\pic[draw, ->, pic text=\huge$\theta$, pic text options={rotate=110},
			angle radius=2cm, angle eccentricity=1.2,
			transform shape] {angle=M--O--T};
	\end{scope}

\end{tikzpicture}

\end{document}
