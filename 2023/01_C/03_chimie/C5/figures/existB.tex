\documentclass{standalone}
\usepackage{mintikz}

\begin{document}
\begin{tikzpicture}[scale=1, declare function={
				wdth = 14;
				hgth = 1.2;
			}]
	% Scale
	\draw[very thick, -{Stealth[scale=1.3]}]
	(0,0) --
	node[at end, below] {$\prm \ce{B}\ind{lim} \neq \pk[s]$}
	node[midway, below] {Équilibre atteint}
	(wdth/2,0) coordinate (M) --
	node[midway, below] {Rupture d'équilibre}
	node[at end, right] {$\prm \ce{B}$}
	(wdth,0);
	% Equality
	\draw[very thick]
	($(M)+(0,-.2)$) -- (wdth/2,hgth);
	% Space of prédom
	\draw[<->]
	(0,hgth/2) --
	node[midway, align=center, fill=white]
	{\textbf{Existence de $\ce{A_pB_q\sol{}}$}}
	(wdth/2,hgth/2);
	\draw[<->]
	(wdth/2,hgth/2) --
	node[midway, align=center, fill=white]
	{\textbf{Domaine de $\ce{A+\aqu{}}$}}
	(wdth,hgth/2);
\end{tikzpicture}

\end{document}
