\documentclass[../../main/main.tex]{subfiles}
\graphicspath{{./figures/}}

\dominitoc
\faketableofcontents

% \renewcommand{\mtcSfont}{\small\bfseries}
% \renewcommand{\mtcSSfont}{\footnotesize}

\makeatletter
\renewcommand{\@chapapp}{Chimie -- chapitre}
\makeatother

% \toggletrue{student}
% \toggletrue{corrige}
% \renewcommand{\mycol}{black}
% \renewcommand{\mycol}{gray}

\hfuzz=5.002pt

\begin{document}
\setcounter{chapter}{5}

\settype{prof}
\settype{stud}
\settype{book}

\chapter{R\'eactions d'oxydo-r\'eduction}

\vspace*{\fill}

\begin{prgm}
	% \footnotesize
	\begin{tcb}*(ror)"know"{Savoirs}
		\begin{itemize}
			\item Constante de l'équation de dissolution, produit de solubilité $K_s$
			\item Solubilité et condition de précipitation, domaine d'existence,
			      facteurs influençant la solubilité.
		\end{itemize}
	\end{tcb}
	\begin{tcb}*(ror)"how"{Savoir-faire}
		\begin{itemize}
			\item Déterminer la valeur de la constante d'équilibre pour une équation
			      de réaction, combinaison linéaire d'équations dont les constantes
			      thermodynamiques sont connues.
			\item Déterminer la composition chimique du système dans l'état final, en
			      distinguant les cas d'équilibre chimique et de transformation totale,
			      pour une transformation modélisée par une réaction chimique unique.
			\item Prévoir l'état de saturation ou de non saturation d'une solution.
			\item Utiliser les diagrammes de prédominance ou d'existence pour prévoir
			      les espèces incompatibles ou la nature des espèces majoritaires.
			\item Exploiter les courbes d'évolution de la solubilité d'un solide en
			      fonction d'une variable.
			\item Illustrer un procédé de retraitement, de recyclage, de séparation
			      en solution aqueuse.
		\end{itemize}
	\end{tcb}
\end{prgm}

\vspace*{\fill}
\minitoc
\vspace*{\fill}

\newpage

\vspace*{\fill}
% {
\begin{boxes}
	% \footnotesize
	\begin{tcb}(defi)<lftt>{Liste des définitions}
		\tcblistof[\paragraph*]{defi}{\hspace*{6pt}}
	\end{tcb}
	\begin{tcb}(rapp)<lftt>{Liste des rappels}
		\tcblistof[\paragraph*]{rapp}{\hspace*{6pt}}
	\end{tcb}
	\begin{tcb}(prop)<lftt>{Liste des propriétés}
		\tcblistof[\paragraph*]{prop}{\hspace*{6pt}}
		% \tcblistof[\paragraph*]{loi}{\hspace*{6pt}}
		\tcblistof[\paragraph*]{theo}{\hspace*{6pt}}
	\end{tcb}
	% \begin{tcb}(coro)<lftt>{Liste des corollaires}
	% 	\tcblistof[\paragraph*]{coro}{\hspace*{6pt}}
	% \end{tcb}
	% \begin{tcb}(demo)<lftt>{Liste des démonstrations}
	% 	\tcblistof[\paragraph*]{demo}{\hspace*{6pt}}
	% 	\tcblistof[\paragraph*]{prev}{\hspace*{6pt}}
	% \end{tcb}
	% \begin{tcb}(inte)<lftt>{Liste des interprétations}
	% 	\tcblistof[\paragraph*]{inte}{\hspace*{6pt}}
	% \end{tcb}
	% \begin{tcb}(tool)<lftt>{Liste des outils}
	% 	\tcblistof[\paragraph*]{tool}{\hspace*{6pt}}
	% \end{tcb}
	% \begin{tcb}(nota)<lftt>{Liste des notations}
	% 	\tcblistof[\paragraph*]{nota}{\hspace*{6pt}}
	% \end{tcb}
	\begin{tcb}(appl)<lftt>{Liste des applications}
		\tcblistof[\paragraph*]{appl}{\hspace*{6pt}}
	\end{tcb}
	\begin{tcb}(rema)<lftt>{Liste des remarques}
		\tcblistof[\paragraph*]{rema}{\hspace*{6pt}}
	\end{tcb}
	\begin{tcb}(exem)<lftt>{Liste des exemples}
		\tcblistof[\paragraph*]{exem}{\hspace*{6pt}}
	\end{tcb}
	\begin{tcb}(ror)<lftt>{Liste des points importants}
		\tcblistof[\paragraph*]{ror}{\hspace*{6pt}}
	\end{tcb}
	\begin{tcb}(impo)<lftt>{Liste des erreurs communes}
		\tcblistof[\paragraph*]{impo}{\hspace*{6pt}}
	\end{tcb}
\end{boxes}
% }
\vspace*{\fill}
\newpage

\section{Oxydants et réducteurs}



\end{document}
