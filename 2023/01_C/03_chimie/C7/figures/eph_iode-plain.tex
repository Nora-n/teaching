\documentclass{standalone}
\usepackage{mintikz}

\begin{document}
\begin{tikzpicture}[xscale=.5, yscale=3]
	\tikzmath{
		\wdth = 14;
		\phmax = 14;
		\hgth = 4;
		\emin = -0.85;
		\emax = 1.4;
		\tick = 0.025;
		\ct = 1;
		\pt = 1;
		\pke = 14;
		\estanda = 1.1625;
		\estandb = 0.60;
		\estandc = 1.10;
		\phlim = 50/6;
	}
	\tikzset{declare function={
				efront1(\x) = \estanda - 0.0675*(\x);
				efront2(\x) = \estandb;
				efront3(\x) = \estandc - 0.06*(\x);
			}}
	\pgfmathparse{efront1(\phmax)}\let\elima\pgfmathresult
	\pgfmathparse{efront2(\phmax)}\let\elimb\pgfmathresult
	\pgfmathparse{efront3(\phmax)}\let\elimc\pgfmathresult
	% 0
	\coordinate (O) at (0,0);
	\coordinate (Z) at (\wdth,0);
	% Scale
	\draw[very thick, -{Stealth}]
	(-.5,0) --
	node[at end, right] {$\pH$}
	(\wdth+1,0);
	\foreach \x in {2, 4, 6, 10, 12, 14}{
			\draw
			(\x,-\tick)
			node[below] {\num{\x}}
			--++ (0,2*\tick)
			;
		}
	\draw[very thick, -{Stealth}]
	(0,\emin) --
	node[at end, left] {$E~(\si{V})$}
	(0,\emax);
	\foreach \y in {-0.8, -0.4, ..., 1.3}{
			\draw
			(-6*\tick,\y)
			node[left=.3cm] {\num{\fpeval{round(\y,1)}}}
			--++ (12*\tick,0);
		}
\end{tikzpicture}

\end{document}
