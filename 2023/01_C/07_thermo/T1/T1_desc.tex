\documentclass[../../main/main.tex]{subfiles}
\graphicspath{{./figures/}}

\dominitoc
\faketableofcontents

% \renewcommand{\mtcSfont}{\small\bfseries}
% \renewcommand{\mtcSSfont}{\footnotesize}
\mtcsettitle{minitoc}{}
\mtcsetrules{*}{off}

\makeatletter
\renewcommand{\@chapapp}{Thermodynamique -- chapitre}
\makeatother

% \toggletrue{student}
% \toggletrue{corrige}
% \renewcommand{\mycol}{black}
% \renewcommand{\mycol}{gray}

\hfuzz=5.002pt

\begin{document}
\setcounter{chapter}{6}

\settype{book}
\settype{prof}
\settype{stud}

\chapter{Description d'un système à l'équilibre}

% \vspace*{\fill}

\begin{tcb}*(ror)<ctc>{\iconsomm~Sommaire}
	\vspace{-15pt}
	\minitoc
	\vspace{-25pt}
\end{tcb}

% \vspace*{\fill}
% \begin{prgm}
% \small
% \begin{tcb}*(ror)"know"{Savoirs}
% 	\begin{itemize}
% 		\item Principe de construction, lecture et utilisation d'un diagramme
% 		      potentiel-pH.
% 		\item Diagramme potentiel-pH de l'eau.
% 	\end{itemize}
% \end{tcb}
\begin{tcb}*[sidebyside, sidebyside align=top](ror)<ctc>""{\iconhow~Capacités exigibles}
	\small
	\begin{itemize}[label=\rcheck]
		\item Définir l’échelle mésoscopique et en expliquer la nécessité.
		\item Citer quelques ordres de grandeur de libres parcours moyens.

		\item Préciser les paramètres nécessaires à la description d’un état
		      microscopique et d’un état macroscopique sur un exemple.

		\item Calculer l’ordre de grandeur d’une vitesse quadratique moyenne dans un
		      gaz parfait.

		\item Identifier un système ouvert, un système fermé, un système isolé.

		\item Calculer une pression à partir d’une condition d’équilibre mécanique.

		\item Déduire une température d’une condition d’équilibre thermique.

		\item Citer quelques ordres de grandeur de volumes molaires ou massiques
		      dans les conditions usuelles de pression et de température.

		\item Citer et utiliser l’équation d’état des gaz parfaits.

	\end{itemize}
	\tcblower
	\begin{itemize}[label=\rcheck]

		\item Exprimer l’énergie interne d’un gaz parfait monoatomique à partir de
		      l’interprétation microscopique de la température.

		\item Exploiter la propriété $U_m = U_m(T)$ pour un gaz parfait, d'une part,
		      et une phase incrompressible et indilatable d'autre part.

		\item Interpréter graphiquement la différence de compressibilité entre un
		      liquide et un gaz à partir d'isothermes expérimentales.

		\item Comparer le comportement d'un gaz réel au modèle du gaz parfait sur
		      des réseaux d'isothermes expérimentales en coordonnées de
		      \textsc{Clapeyron} ou d'\textsc{Amagat}.

		\item Analyser un diagramme de phase expérimental $(P,T)$.

		\item Proposer un jeu de variables d'état suffisant pour caractériser l'état
		      d'équilibre d'un corps pur diphasé soumis aux seules forces de pression.

		\item Positionner les phases dans les diagrammes $(P,T)$ et $(P,v)$.

		\item Déterminer la composition d'un mélangé diphasé en un point d'un
		      diagramme $(P,v)$.
	\end{itemize}
\end{tcb}
% \end{prgm}
\vspace{-15pt}

% \vspace*{\fill}

\newpage

\vspace*{\fill}
% {
% \begin{boxes}
\begin{tcb}*[sidebyside](ror)<ctc>{\iconchek~L'essentiel}
	\small
	\begin{tcb}*(defi)<ctc>{\icondefi~Définitions}
		\tcblistof[\paragraph*]{defi}{\hspace*{4.8pt}}
	\end{tcb}
	% \begin{tcb}(rapp)<ctc>{Rappels}
	%   \tcblistof[\paragraph*]{rapp}{\hspace*{4.8pt}}
	% \end{tcb}
	\begin{tcb}*(prop)<ctc>{\iconprop~Propriétés}
		\tcblistof[\paragraph*]{prop}{\hspace*{4.8pt}}
		% \tcblistof[\paragraph*]{loi}{\hspace*{4.8pt}}
		\tcblistof[\paragraph*]{theo}{\hspace*{4.8pt}}
	\end{tcb}
	% \begin{tcb}*(coro)<ctc>{Corollaires}
	%   \tcblistof[\paragraph*]{coro}{\hspace*{4.8pt}}
	% \end{tcb}
	% \begin{tcb}*(demo)<ctc>{Démonstrations}
	%   \tcblistof[\paragraph*]{demo}{\hspace*{4.8pt}}
	%   \tcblistof[\paragraph*]{prev}{\hspace*{4.8pt}}
	% \end{tcb}
	% \begin{tcb}*(inte)<ctc>{Interprétations}
	%   \tcblistof[\paragraph*]{inte}{\hspace*{4.8pt}}
	% \end{tcb}
	\begin{tcb}*(tool)<ctc>{\icontool~Outils}
		\tcblistof[\paragraph*]{tool}{\hspace*{4.8pt}}
	\end{tcb}
	% \begin{tcb}*(nota)<ctc>{Notations}
	%   \tcblistof[\paragraph*]{nota}{\hspace*{4.8pt}}
	% \end{tcb}
	% \begin{tcb}*(appl)<ctc>{Applications}
	%   \tcblistof[\paragraph*]{appl}{\hspace*{4.8pt}}
	% \end{tcb}
	% \begin{tcb}*(rema)<ctc>{Remarques}
	%   \tcblistof[\paragraph*]{rema}{\hspace*{4.8pt}}
	% \end{tcb}
	% \begin{tcb}*(exem)<ctc>{Exemples}
	%   \tcblistof[\paragraph*]{exem}{\hspace*{4.8pt}}
	% \end{tcb}
	% \begin{tcb}*(ror)<ctc>{Points importants}
	%   \tcblistof[\paragraph*]{ror}{\hspace*{4.8pt}}
	% \end{tcb}
	% \begin{tcb}*(impo)<ctc>{Erreurs communes}
	%   \tcblistof[\paragraph*]{impo}{\hspace*{4.8pt}}
	% \end{tcb}
	\tcblower
	\small
	% \begin{tcb}*(defi)<ctc>{Définitions}
	%   \tcblistof[\paragraph*]{defi}{\hspace*{4.8pt}}
	% \end{tcb}
	% \begin{tcb}*(rapp)<ctc>{Rappels}
	%   \tcblistof[\paragraph*]{rapp}{\hspace*{4.8pt}}
	% \end{tcb}
	% \begin{tcb}*(prop)<ctc>{Propriétés}
	%   \tcblistof[\paragraph*]{prop}{\hspace*{4.8pt}}
	% > \tcblistof[\paragraph*]{loi}{\hspace*{4.8pt}}
	%   \tcblistof[\paragraph*]{theo}{\hspace*{4.8pt}}
	% \end{tcb}
	% \begin{tcb}*(coro)<ctc>{Corollaires}
	%   \tcblistof[\paragraph*]{coro}{\hspace*{4.8pt}}
	% \end{tcb}
	% \begin{tcb}*(demo)<ctc>{Démonstrations}
	%   \tcblistof[\paragraph*]{demo}{\hspace*{4.8pt}}
	%   \tcblistof[\paragraph*]{prev}{\hspace*{4.8pt}}
	% \end{tcb}
	% \begin{tcb}*(inte)<ctc>{Interprétations}
	%   \tcblistof[\paragraph*]{inte}{\hspace*{4.8pt}}
	% \end{tcb}
	% \begin{tcb}*(tool)<ctc>{Outils}
	%   \tcblistof[\paragraph*]{tool}{\hspace*{4.8pt}}
	% \end{tcb}
	% \begin{tcb}*(nota)<ctc>{Notations}
	%   \tcblistof[\paragraph*]{nota}{\hspace*{4.8pt}}
	% \end{tcb}
	\begin{tcb}*(appl)<ctc>{\iconappl~Applications}
		\tcblistof[\paragraph*]{appl}{\hspace*{4.8pt}}
	\end{tcb}
	% \begin{tcb}*(rema)<ctc>{Remarques}
	%   \tcblistof[\paragraph*]{rema}{\hspace*{4.8pt}}
	% \end{tcb}
	% \begin{tcb}*(exem)<ctc>{Exemples}
	%   \tcblistof[\paragraph*]{exem}{\hspace*{4.8pt}}
	% \end{tcb}
	\begin{tcb}*(ror)<ctc>{\iconhart~Points importants}
		\tcblistof[\paragraph*]{ror}{\hspace*{4.8pt}}
	\end{tcb}
	\begin{tcb}*(impo)<ctc>{\iconimpo~Erreurs communes}
		\tcblistof[\paragraph*]{impo}{\hspace*{4.8pt}}
	\end{tcb}
\end{tcb}
% \end{boxes}
% }

\vspace*{\fill}
\newpage

Comme son nom l’indique, la thermo/dynamique est le domaine de la physique qui
s’intéresse au lien entre les aspects thermiques (le «~chaud~» et le «~froid~»)
et le mouvement. C’est son émergence entre la fin du \textsc{xviii}\ieme{} et le
début du \textsc{xix}\ieme{} siècle qui a permis la révolution industrielle,
dont la «~machine à vapeur~» est emblématique.
\bigbreak
Aujourd’hui encore, le fonctionnement de multiples systèmes exploite les lois de
la thermodynamique. Citons par exemple les réfrigérateurs ou les moteurs des
voitures à essence. La production d’électricité dans une centrale thermique,
qu’elle soit nucléaire, géothermique ou à charbon, commence également par une
conversion d’énergie thermique en énergie mécanique, avant que cette énergie
mécanique ne soit convertie en énergie électrique grâce aux phénomènes
d’induction.



\end{document}
