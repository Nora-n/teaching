% \documentclass[../../main/main.tex]{subfiles}
% \graphicspath{{./figures/}}

\dominitoc
\faketableofcontents

\renewcommand{\mtcSfont}{\small\bfseries}
\renewcommand{\mtcSSfont}{\footnotesize}
\mtcsettitle{minitoc}{}
\mtcsetrules{*}{off}

\makeatletter
\renewcommand{\@chapapp}{Thermodynamique -- chapitre}
\makeatother

% \toggletrue{student}
% \toggletrue{corrige}
% \renewcommand{\mycol}{black}
% \renewcommand{\mycol}{gray}

\hfuzz=5.002pt

\begin{document}
\setcounter{chapter}{2}

% \settype{book}
% \settype{prof}
% \settype{stud}

\chapter{Premier principe de la thermodynamique}

% \vspace*{\fill}

\begin{tcn}*(appl)<ctc>{\iconsomm~Sommaire}
	\vspace{-15pt}
	\minitoc
	\vspace{-25pt}
\end{tcn}

\begin{tcn}*[sidebyside, sidebyside align=top,
		fontupper=\small, fontlower=\small](appl)<ctc>""{\iconhow~Capacités exigibles}
	\begin{itemize}[label=\rcheck]
		\item Définir un système adapté à une problématique donnée.

		\item Exploiter les conditions imposées par le milieu extérieur pour
		      déterminer l'état d'équilibre final.

		\item Évaluer un travail par découpage en travaux élémentaires et sommation
		      sur un chemin donné dans le cas d'une seule variable.

		\item Interpréter géométriquement le travail des forces de pression dans un
		      diagramme de \textsc{Watt}.

		\item Distinguer qualitativement les trois types de transferts thermiques~:
		      conduction, convection et rayonnement.

		\item Identifier dans une situation expérimentale le ou les systèmes
		      modélisables par un thermostat.

		\item Définir un système fermé et établir pour ce système un bilan énergétique
		      faisant intervenir travail et transfert thermique.

		\item Utiliser le premier principe de la thermodynamique entre deux états
		      voisins.
	\end{itemize}
	\tcblower
	\begin{itemize}[label=\rcheck]
		\item Exploiter l'extensivité de l'énergie interne.

		\item Distinguer le statut de la variation de l'énergie interne du statut des
		      termes d'échange.
		\item Calculer le transfert thermique sur un chemin donné connaissant le
		      travail et la variation de l'énergie interne.

		\item Exprimer le premier principe sous forme de bilan d'enthalpie dans le cas
		      d'une transformation monobare avec équilibre mécanique dans l'état initial
		      et dans l'état final.

		\item Exprimer l'enthalpie $H_m(T)$ du gaz parfait à partir de l'énergie
		      interne.

		\item Justifier que l'enthalpie $H_m$ d'une phase condensée peu compressible
		      et peu dilatable peut être considérée comme une fonction de l'unique
		      variable T.

		\item Citer l'ordre de grandeur de la capacité thermique massique de l'eau
		      liquide.

		\item Exploiter l'extensivité de l'enthalpie et réaliser des bilans
		      énergétiques en prenant en compte des transitions de phases.
	\end{itemize}
\end{tcn}

\newpage

\vspace*{\fill}
% {
% \begin{boxes}
\begin{tcn}*[sidebyside, fontupper=\small, fontlower=\small](appl)<ctc>{\iconchek~L'essentiel}
	\begin{tcn}*(defi)<ctc>{\icondefi~Définitions}
		\tcblistof[\paragraph*]{defi}{\hspace*{4.8pt}}
	\end{tcn}
	% \begin{tcn}(rapp)<ctc>{\iconrapp~Rappels}
	%   \tcblistof[\paragraph*]{rapp}{\hspace*{4.8pt}}
	% \end{tcn}
	\begin{tcn}*(prop)<ctc>{\iconprop~Propriétés}
		\tcblistof[\paragraph*]{prop}{\hspace*{4.8pt}}
		% \tcblistof[\paragraph*]{loi}{\hspace*{4.8pt}}
		% \tcblistof[\paragraph*]{theo}{\hspace*{4.8pt}}
	\end{tcn}
	% \begin{tcn}*(coro)<ctc>{\iconcoro~Corollaires}
	%   \tcblistof[\paragraph*]{coro}{\hspace*{4.8pt}}
	% \end{tcn}
	\begin{tcn}*(demo)<ctc>{\icondemo~Démonstrations}
		\tcblistof[\paragraph*]{demo}{\hspace*{4.8pt}}
		\tcblistof[\paragraph*]{prev}{\hspace*{4.8pt}}
	\end{tcn}
	% \begin{tcn}*(inte)<ctc>{\iconinte~Interprétations}
	%   \tcblistof[\paragraph*]{inte}{\hspace*{4.8pt}}
	% \end{tcn}
	% \begin{tcn}*(tool)<ctc>{\icontool~Outils}
	% 	\tcblistof[\paragraph*]{tool}{\hspace*{4.8pt}}
	% \end{tcn}
	% \begin{tcn}*(nota)<ctc>{\iconnota~Notations}
	%   \tcblistof[\paragraph*]{nota}{\hspace*{4.8pt}}
	% \end{tcn}
	% \begin{tcn}*(appl)<ctc>{\iconappl~Applications}
	%   \tcblistof[\paragraph*]{appl}{\hspace*{4.8pt}}
	% \end{tcn}
	% \begin{tcn}*(rema)<ctc>{\iconrema~Remarques}
	%   \tcblistof[\paragraph*]{rema}{\hspace*{4.8pt}}
	% \end{tcn}
	% \begin{tcn}*(exem)<ctc>{\iconexem~Exemples}
	%   \tcblistof[\paragraph*]{exem}{\hspace*{4.8pt}}
	% \end{tcn}
	% \begin{tcn}*(ror)<ctc>{\iconhart~Points importants}
	%   \tcblistof[\paragraph*]{ror}{\hspace*{4.8pt}}
	% \end{tcn}
	% \begin{tcn}*(impo)<ctc>{\iconimpo~Erreurs communes}
	%   \tcblistof[\paragraph*]{impo}{\hspace*{4.8pt}}
	% \end{tcn}
	\tcblower
	% \begin{tcn}*(defi)<ctc>{\icondefi~Définitions}
	%   \tcblistof[\paragraph*]{defi}{\hspace*{4.8pt}}
	% \end{tcn}
	% \begin{tcn}*(rapp)<ctc>{\iconrapp~Rappels}
	%   \tcblistof[\paragraph*]{rapp}{\hspace*{4.8pt}}
	% \end{tcn}
	% \begin{tcn}*(prop)<ctc>{\iconprop~Propriétés}
	%   \tcblistof[\paragraph*]{prop}{\hspace*{4.8pt}}
	%   \tcblistof[\paragraph*]{loi}{\hspace*{4.8pt}}
	%   \tcblistof[\paragraph*]{theo}{\hspace*{4.8pt}}
	% \end{tcn}
	% \begin{tcn}*(coro)<ctc>{\iconcoro~Corollaires}
	%   \tcblistof[\paragraph*]{coro}{\hspace*{4.8pt}}
	% \end{tcn}
	% \begin{tcn}*(demo)<ctc>{\icondemo~Démonstrations}
	% 	\tcblistof[\paragraph*]{demo}{\hspace*{4.8pt}}
	% 	\tcblistof[\paragraph*]{prev}{\hspace*{4.8pt}}
	% \end{tcn}
	\begin{tcn}*(inte)<ctc>{\iconinte~Interprétations}
		\tcblistof[\paragraph*]{inte}{\hspace*{4.8pt}}
	\end{tcn}
	% \begin{tcn}*(tool)<ctc>{\icontool~Outils}
	%   \tcblistof[\paragraph*]{tool}{\hspace*{4.8pt}}
	% \end{tcn}
	% \begin{tcn}*(nota)<ctc>{\iconnotaNotations}
	%   \tcblistof[\paragraph*]{nota}{\hspace*{4.8pt}}
	% \end{tcn}
	\begin{tcn}*(appl)<ctc>{\iconappl~Applications}
		\tcblistof[\paragraph*]{appl}{\hspace*{4.8pt}}
	\end{tcn}
	% \begin{tcn}*(rema)<ctc>{\iconrema~Remarques}
	%   \tcblistof[\paragraph*]{rema}{\hspace*{4.8pt}}
	% \end{tcn}
	% \begin{tcn}*(exem)<ctc>{\iconexem~Exemples}
	% 	\tcblistof[\paragraph*]{exem}{\hspace*{4.8pt}}
	% \end{tcn}
	\begin{tcn}*(ror)<ctc>{\iconhart~Points importants}
		\tcblistof[\paragraph*]{ror}{\hspace*{4.8pt}}
	\end{tcn}
	\begin{tcn}*(impo)<ctc>{\iconimpo~Erreurs communes}
		\tcblistof[\paragraph*]{impo}{\hspace*{4.8pt}}
	\end{tcn}
\end{tcn}
% \end{boxes}
% }%

\vspace*{\fill}
\newpage

\end{document}
