\documentclass[../../main/main.tex]{subfiles}
\graphicspath{{./figures/}}
\usepackage{custikz}

\dominitoc
\faketableofcontents

\renewcommand{\mtcSfont}{\small\bfseries}
\renewcommand{\mtcSSfont}{\footnotesize}
\mtcsettitle{minitoc}{}
\mtcsetrules{*}{off}

\makeatletter
\renewcommand{\@chapapp}{Thermodynamique -- chapitre}
\makeatother

% \toggletrue{student}
% \toggletrue{corrige}
% \renewcommand{\mycol}{black}
% \renewcommand{\mycol}{gray}

\hfuzz=5.003pt

\begin{document}
\setcounter{chapter}{3}

% \settype{book}
% \settype{prof}
% \settype{stud}

\chapter{Machines thermiques}

\vspace*{\fill}

\begin{tcn}*(appl)<ctc>{\iconsomm~Sommaire}
	\let\item\olditem
	\vspace{-15pt}
	\minitoc
	\vspace{-25pt}
\end{tcn}

\begin{tcn}*[fontupper=\small](appl)<ctb>"how"'t'{Capacités exigibles}
	\begin{itemize}[label=\rcheck]
		\item Donner le sens des échanges énergétiques pour un moteur ou un
		      récepteur thermique ditherme.

		\item Analyser un dispositif concret et le modéliser par une machine
		      cyclique ditherme.

		\item Définir un rendement ou une efficacité et les relier aux énergies
		      échangées au cours d'un cycle.

		\item Justifier et utiliser le théorème de Carnot.

		\item Citer quelques ordres de grandeur des rendements des machines
		      thermiques réelles actuelles.

		\item Expliquer le principe de la cogénération.
	\end{itemize}
\end{tcn}

\vspace*{\fill}

\newpage

\vspace*{\fill}
% {
% \begin{boxes}
\begin{tcn}[sidebyside, fontupper=\small, fontlower=\small](appl)<ctb>"check"'t'{L'essentiel}
	\begin{tcn}(defi)<ctc>'t'{Définitions}
		\tcblistof[\paragraph*]{defi}{\hspace*{4.8pt}}
	\end{tcn}
	% \begin{tcn}(rapp)<ctc>'t'{Rappels}
	% 	\tcblistof[\paragraph*]{rapp}{\hspace*{4.8pt}}
	% \end{tcn}
	\begin{tcn}(prop)<ctc>'t'{Propriétés}
		\tcblistof[\paragraph*]{prop}{\hspace*{4.8pt}}
		\tcblistof[\paragraph*]{loi}{\hspace*{4.8pt}}
		% \tcblistof[\paragraph*]{theo}{\hspace*{4.8pt}}
	\end{tcn}
	% \begin{tcn}(coro)<ctc>'t'{Corollaires}
	%   \tcblistof[\paragraph*]{coro}{\hspace*{4.8pt}}
	% \end{tcn}
	\begin{tcn}(demo)<ctc>'t'{Démonstrations}
		\tcblistof[\paragraph*]{demo}{\hspace*{4.8pt}}
		\tcblistof[\paragraph*]{prev}{\hspace*{4.8pt}}
	\end{tcn}
	\begin{tcn}(inte)<ctc>'t'{Interprétations}
		\tcblistof[\paragraph*]{inte}{\hspace*{4.8pt}}
	\end{tcn}
	% \begin{tcn}(impl)<ctc>'t'{Implications}
	% 	\tcblistof[\paragraph*]{impl}{\hspace*{4.8pt}}
	% \end{tcn}
	% \begin{tcn}(tool)<ctc>'t'{Outils}
	% 	\tcblistof[\paragraph*]{tool}{\hspace*{4.8pt}}
	% \end{tcn}
	% \begin{tcn}(nota)<ctc>'t'{Notations}
	%   \tcblistof[\paragraph*]{nota}{\hspace*{4.8pt}}
	% \end{tcn}
	% \begin{tcn}(appl)<ctc>'t'{Applications}
	%   \tcblistof[\paragraph*]{appl}{\hspace*{4.8pt}}
	% \end{tcn}
	% \begin{tcn}(rema)<ctc>'t'{Remarques}
	%   \tcblistof[\paragraph*]{rema}{\hspace*{4.8pt}}
	% \end{tcn}
	% \begin{tcn}(exem)<ctc>'t'{Exemples}
	%   \tcblistof[\paragraph*]{exem}{\hspace*{4.8pt}}
	% \end{tcn}
	% \begin{tcn}*(ror)<ctc>"hart"'t'{Points importants}
	%   \tcblistof[\paragraph*]{ror}{\hspace*{4.8pt}}
	% \end{tcn}
	% \begin{tcn}(impo)<ctc>'t'{Erreurs communes}
	%   \tcblistof[\paragraph*]{impo}{\hspace*{4.8pt}}
	% \end{tcn}
	\tcblower
	% \begin{tcn}(defi)<ctc>'t'{Définitions}
	%   \tcblistof[\paragraph*]{defi}{\hspace*{4.8pt}}
	% \end{tcn}
	% \begin{tcn}(rapp)<ctc>'t'{Rappels}
	%   \tcblistof[\paragraph*]{rapp}{\hspace*{4.8pt}}
	% \end{tcn}
	% \begin{tcn}(prop)<ctc>'t'{Propriétés}
	%   \tcblistof[\paragraph*]{prop}{\hspace*{4.8pt}}
	%   \tcblistof[\paragraph*]{loi}{\hspace*{4.8pt}}
	%   \tcblistof[\paragraph*]{theo}{\hspace*{4.8pt}}
	% \end{tcn}
	% \begin{tcn}(coro)<ctc>'t'{Corollaires}
	%   \tcblistof[\paragraph*]{coro}{\hspace*{4.8pt}}
	% \end{tcn}
	% \begin{tcn}(demo)<ctc>'t'{Démonstrations}
	% 	\tcblistof[\paragraph*]{demo}{\hspace*{4.8pt}}
	% 	\tcblistof[\paragraph*]{prev}{\hspace*{4.8pt}}
	% \end{tcn}
	% \begin{tcn}(inte)<ctc>'t'{Interprétations}
	% 	\tcblistof[\paragraph*]{inte}{\hspace*{4.8pt}}
	% \end{tcn}
	\begin{tcn}(impl)<ctc>'t'{Implications}
		\tcblistof[\paragraph*]{impl}{\hspace*{4.8pt}}
	\end{tcn}
	% \begin{tcn}(tool)<ctc>'t'{Outils}
	%   \tcblistof[\paragraph*]{tool}{\hspace*{4.8pt}}
	% \end{tcn}
	% \begin{tcn}(nota)<ctc>'t'{Notations}
	%   \tcblistof[\paragraph*]{nota}{\hspace*{4.8pt}}
	% \end{tcn}
	\begin{tcn}(appl)<ctc>'t'{Applications}
		\tcblistof[\paragraph*]{appl}{\hspace*{4.8pt}}
	\end{tcn}
	% \begin{tcn}(rema)<ctc>'t'{Remarques}
	%   \tcblistof[\paragraph*]{rema}{\hspace*{4.8pt}}
	% \end{tcn}
	% \begin{tcn}(exem)<ctc>'t'{Exemples}
	% 	\tcblistof[\paragraph*]{exem}{\hspace*{4.8pt}}
	% \end{tcn}
	\begin{tcn}*(ror)<ctc>"hart"'t'{Points importants}
		\tcblistof[\paragraph*]{ror}{\hspace*{4.8pt}}
	\end{tcn}
	\begin{tcn}(impo)<ctc>'t'{Erreurs communes}
		\tcblistof[\paragraph*]{impo}{\hspace*{4.8pt}}
	\end{tcn}
\end{tcn}
% \end{boxes}
% }%

\vspace*{\fill}
\newpage


\begin{center}
	\begin{tikzpicture}
		\draw (0,0) rectangle (4,2);
		\draw (2,1) node{fluide};

		\draw (-3.5,1) circle (1);
		\draw (-3.5,1) node[text centered,text width=2.5cm]{Source \\ froide \\ $T_f$};


		\draw (7.5,1) circle (1);
		\draw (7.5,1) node[text centered,text width=2.5cm]{Source \\ chaude \\ $T_c$};

		\draw[simplef] (-2.5,1)--(0,1)node[midway,above]{$Q_f$};
		\draw[simplef] (6.5,1)--(4,1)node[midway,above]{$Q_c$};
		\draw[simplef] (2,-1)--(2,0)node[midway,right]{$W$};

		\draw [red,->,>=latex,very thick] (-2,0.75)--(-1,0.75);
		\draw [red,->,>=latex,very thick] (5,0.75)--(6,0.75);
		\draw [red,->,>=latex,very thick] (1.75,-0.95)--(1.75,-0.25);

	\end{tikzpicture}
\end{center}

\end{document}

% \begin{center}
% 	\begin{tikzpicture}[scale=1.3]
% 		\draw [->](0,0) -- (0,3.5) node[anchor=west]{$P$};
% 		\draw [->](0,0) -- (5.5,0) node[anchor=west]{$V$};
%
% 		\draw[dashed] plot [domain=0.8:5] (\x,3/\x) node[below]{$T_f$};
% 		\draw[very thick, red,simplef] plot [domain=1.5:3] (\x,3/\x);
% 		\draw[very thick, red,simplef] plot [domain=1.0203:1.5] (\x,{3.5282/(\x^(1.4))});
% 		\draw[dashed] plot [domain=0.8:5] (\x,3.5/\x) node[above]{$T_c$};
% 		\draw[very thick, red,simplef] plot [domain=2.0406:1.0203] (\x,3.5/\x);
% 		\draw[very thick, red,simplef] plot [domain=3:2.0406] (\x,{4.6555/(\x^(1.4))});
%
% 		\fill (3,1) circle(0.04);
% 		\fill (1.5,2) circle(0.04);
% 		\fill (1.0203,3.43) circle(0.04);
% 		\fill (2.0406,1.72) circle(0.04);
%
% 		\node[above] at (3,1) {A};
% 		\node[left] at (1.5,2) {D};
% 		\node[right] at (1.0203,3.43) {C};
% 		\node[right] at (2.0406,1.72) {B};
% 	\end{tikzpicture}
% \end{center}
