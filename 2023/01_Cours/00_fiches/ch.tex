\documentclass[a4paper, 10pt, final, garamond]{book}

\usepackage{cours-preambule}

\begin{document}
\setcounter{chapter}{-1}

\chapter*{Bienvenue en classes pr\'eparatoires}

Bienvenue au lycée Pothier, et bravo d'être arrivé-es ici. Nous allons passer
cette année ensemble pour vous faire passer de lycéen-nes à étudiant-es. La
première année est particulièrement importante à cet effet~: elle vise à ancrer
les bases de la réflexion, de la méthode et de la pratique nécessaires à obtenir
une réelle expertise. Pour la passer dans les meilleures conditions possibles,
je vous présente dans ce document quelques aspect de l'apprentissage tel que
nous allons le partager.

\section{Objectif}

\begin{tcbraster}[raster columns=4, raster equal height=rows]
	\begin{tcolorbox}[blankest, raster multicolumn=3]

		La formation que vous allez suivre vous prépare aux concours d'entrée
		aux grandes écoles d'ingénieurs, qui se divisent entre écrits et oraux.
		Les \underline{écrits} seront fin \textbf{avril 2024}, donc dans
		\underline{moins de deux ans}. Les épreuves durent de 2h à 6h selon les
		écoles visées, et peuvent être avec ou \textit{sans} calculatrice~; les
		oraux se déroulent entre mi-juin et fin juillet selon les écoles. Ces
		épreuves \textbf{portent sur les deux années} de CPGE, de un tiers à la
		moitié sur la première ; sachant que la seconde est courte (les cours se
		terminent en avril), il vous faut vite intégrer la chose suivante~: dès
		aujourd'hui nous travaillons pour avril 2024.

	\end{tcolorbox}
	\begin{tcb}[fil](ror)<width=.5\linewidth, halign=center>{IMPORTANT}
		\textbf{Le retard pris en MPSI ne se rattrape pas~!}
	\end{tcb}
\end{tcbraster}

\section{Parlons hiérarchie}

Il va de soi que pendant cette année je vais être votre professeure, et vous
suivre tout au long de l'année. Cependant, il m'est à cœur de parler de ma
position au sein de la classe. Notamment, je considère que dans des relations
interpersonnelles, nous ne respectons pas un titre ou une profession mais bien
une personne. Ainsi je ne me fais pas figure hiérarchique absolue, j'attends de
vous de me traiter avec respect comme j'interagirai avec vous avec respect.

À cet égard, je m'exprime à vous en employant le vouvoiement et, au choix, votre
prénom ou nom de famille. Ceci a pour volonté de vous faire sentir la
responsabilité partagée qui nous incombe à vous comme à moi~: si je vais mener
la plupart des cours et transmettre la plupart des informations, il est de votre
responsabilité en tant que scientifiques en essence de travailler en ce sens. Je
porterai une grande valeur à vos propos, questions, interrogations, suggestions,
réflexions et propositions sur tout ce qui concerne la science.

\section{Parlons apprentissage}

Dans le long chemin pour devenir expert-e dans une pratique, nous pouvons
distinguer trois éléments primordiaux~:
\begin{enumerate}
	\item la répétition avec retour~;
	\item la présence d'un environnement sain (et prédictif\footnote{Sans
		      détailler dans le corps de texte, faire de nombreux paris à la
		      roulette permet la répétition et le retour (est-ce que je gagne ou
		      perd), mais l'environnement est aléatoire et ne permet pas d'affiner
		      nos pratiques. Dans nos domaines une même action amène à un même
		      résultat.})~;
	\item la recherche personnelle.
\end{enumerate}

Ces trois éléments permettent d'atteindre la capacité de repérage de motifs et
de schémas propres à une discipline, trait caractéristique des expert-es. En
effet, un-e Grand-e Maître-sse d'échecs a une performance bien supérieure qu'une
personne lambda pour mémoriser la position des pièces d'un échiquier si elles
sont dans une position réaliste, mais aucune différence notable n'existe entre
les deux quand elles sont placées aléatoirement sur celui-ci. Je vous propose de
détailler ces points-là.

\subsection{Répétition avec retour}

Si vous pratiquez un instrument, un art ou un sport (e-sport compris), vous
savez bien qu'il ne suffit pas de regarder une vidéo d'une personne pratiquant
la même chose que ce que vous voulez faire pour vous réveiller le lendemain et
être aussi performant-e. Vous savez qu'avant de faire proprement du ski, faire
une figure en skate ou jouer dans un concert, il vous faut pratiquer avec
assiduité et répéter les actions~: c'est pareil avec le corps qu'avec l'esprit.
En physique-chimie et en mathématiques, il arrive un moment dans la pratique où
la répétition de raisonnements et de calculs forment un tout qui permet de
rapidement identifier ce qu'il se passe, de supposer les hypothèses desquelles
nous partons, et même prévoir le résultat sous une certaine forme (et donc
savoir à l'avance quand quelque chose ne fonctionne pas). D'une manière
générale, la mémoire repose sur la répétition~: il est estimé (\textit{via} la
«~courbe d'oubli~»\footnote{\textit{forgetting curve} en anglais, cf. les
	travaux de Hermann \textsc{Ebbinghaus}.}) qu'une heure après l'acquisition
d'une connaissance il ne nous en reste que 50\%, et 30\% après un jour. Le
secret de la mémoire tient dans la répétition espacée, comme un muscle se forme
par l'utilisation répétée et espacée.

Cependant, répéter la même action en boucle sans analyse et confrontation avec
un avis extérieur ne suffit pas à développer une expertise. Dans le cas de la
physique, nous pouvons heureusement voir si le résultat est correct ou non,
notamment avec l'aide des professeur-es et qui permettent la rétroaction
nécessaire à votre compréhension. Mais cette répétition cache quelque chose dont
nous ne parlons que trop rarement~: répéter implique de se tromper. Il y a une
grande, trop grande crainte au fait de se tromper quand il est question de
compétences intellectuelles. Cette crainte est le principal frein à un
apprentissage sain. En réalité, qui n'a jamais échoué n'a jamais essayé
suffisamment. Et pour cela l'environnement d'apprentissage est primordial.

\subsection{Environnement d'apprentissage}

Plus que normal, il est nécessaire de faire des fautes. Il est important
d'essayer~: la connaissance même du monde se base sur l'échec, je dirais même
qu'il en est principalement constitué. Chaque pratique qui ne fonctionne pas
nous permet de l'analyser pour intégrer la raison de cet échec et affiner cette
pratique dans une direction de plus en plus polie. Par exemple, je vais sans
doute me tromper en disant quelque chose, en écrivant au tableau ou en faisant
un calcul, et à chaque fois nous prendrons l'opportunité d'étudier comment il
est possible de voir pourquoi telle chose était fausse et comment la corriger~:
il en va de même pour vous, chacune de vos tentative est valide et vous mènera
vers la réussite.

Dans ce contexte, je considère la zone devant le tableau non pas celle du ou de
la professeur-e, mais comme une zone géographique de création de savoir, peu
importe qui l'occupe. J'attends de tout le monde ici présent d'avoir la même
attitude, et de respecter les efforts de chaque individu s'essayant à la
pratique de la physique et de la chimie. Ne soyez pas condescendant-es, déjà
dans vos pensées mais absolument jamais dans vos actes ou paroles si un-e de vos
camarades ne répond pas correctement à une question ou fait une faute au
tableau. Je ne le ferai jamais dans mon cas.

\subsection{Recherche personnelle}

Comme énoncé plus haut, il ne suffit pas d'écouter du \textsc{Chopin} pour jouer
du \textsc{Chopin}, mais même en répétant à l'infini le même morceau dans un
environnement sain avec un retour sur votre performance vous ne deviendrez pas
un-e expert-e~: il faut pour cela vous attaquer à différents morceaux,
différents styles, faire varier vos conditions de travail et développer vos
capacités de reconnaissance de motifs musicaux.

Il en va de même avec la pratique scientifique. Vous devez pratiquer avec effort
et vous approprier la connaissance que nous créons ensemble en étant partie
active de cet apprentissage. Le système des classes préparatoires vous aide dans
ce sens avec les khôlles hebdomadaires que vous allez effectuer, en vous
plongeant dans le rôle du ou de la transmetteur-ice, ou avec les devoirs maison
que je vous demanderai d'effectuer. Mais le plus important se passe en-dehors de
ces moments-là, lors de vos temps de révision.

Il est indispensable que vous relisiez vos cours, prépariez vos TDs et vos
khôlles. Si ceci est théoriquement faisable seul-e, la meilleure manière
d'apprendre est encore d'enseigner. Je vous invite donc fortement à vous
rapprocher de vos camarades (le principe du groupe de khôlle), peu importe leur
niveau, pour échanger avec elleux sur ce qui n'est pas bien compris. Posez-vous
des questions entre vous, regardez comment les ressources dont vous disposez
vous permettent de partir d'un point A de la réflexion à un point B, notamment
sur les démonstrations, regardez des vidéos sur la science en essayant de
prédire le phénomène mis en jeu ou encore expliquez à votre chat le cours sur
lequel vous serez interrogé-es en khôlle.

Il vous faut sortir de votre zone de confort pour explorer et affiner les
compétences que vous apprenez. Alors, et seulement alors, la pratique que vous
travaillez pourra être agréable et la connaissance accumulée source de fierté et
de plaisir.

\subsection{Résumé}
\subsubsection{En classe}
\begin{itemize}
	\item Posez des questions si vous ne comprenez pas~;
	\item Soyez \textbf{attentif-ves} (plus d'attention en classe = moins de
	      travail seul-e dans son coin après)~;
	\item Soyez organisé-es~: bloc-notes, trieur, pochettes plastifiées avec
	      code couleur pour avoir les cours et TDs pertinents et séparés…
	\item Soyez efficaces dans votre prise de note~: établissez des codes
	      couleurs, des abréviations, ne faites pas tous les schémas à la règle du
	      premier coup… Écouter et écrire divise l'attention. Je n'assure
	      absolument pas de vous fournir tous les documents de cours\footnote{Par
		      contre, tous les documents distribués en cours seront sur le site
		      \href{https://cahier-de-prepa.fr/mpsi3-pothier/}
		      {https://cahier-de-prepa.fr/mpsi3-pothier/}.}.
\end{itemize}

\subsubsection{En dehors}
\begin{itemize}
	\item \textbf{Relisez votre cours le soir-même}, ajoutez des annotations,
	      refaites les schémas, commencez à mémoriser~;
	\item Travaillez avec vos camarades pour préparer les TDs, réviser les
	      khôlles, posez-vous des questions entre vous~;
	\item Soyez actif-ves pendant vos séances, cherchez à comprendre, testez
	      votre compréhension~;
	\item Faites \underline{vos propres fiches pour chaque chapitre}~:
	      nécessaire et obligatoire pour réviser les concours~;
	\item \textbf{Dormez suffisamment (8 heures) et levez-vous suffisamment tôt
		      !!} Ne surestimez pas votre capacité à faire des courtes nuits et à
	      rester efficaces le lendemain. Ça ne sert à rien de venir en cours si
	      c'est pour dormir…
\end{itemize}

\subsection{En supplément}

\begin{itemize}
	\item Plein de ressources sur YouTube~;
	\item Appli \texttt{Qmax} sur Android~;
	\item Le site et l'application \texttt{Brilliant} (en anglais).
\end{itemize}

\section{Le vrai ennemi des études}

À l'ère numérique actuelle, les smartphones sont devenus une partie intégrante
de nos vies, nous permettant de nous connecter, d'apprendre et d'accéder à
l'information comme jamais auparavant. Cependant, il est essentiel de
reconnaître les éventuels inconvénients d'une utilisation excessive du
téléphone, en particulier en ce qui concerne son impact sur notre motivation. Ce
document vise à mettre en lumière comment votre utilisation du téléphone
pourrait affecter votre motivation et à fournir des informations pour atténuer
ces effets.

\subsection{Les Statistiques Écrasantes}

Il y a à peine 15 ans, seulement 20 \% des personnes accédaient à Internet
depuis leur téléphone. Aujourd'hui, ce nombre a grimpé en flèche à 91 \%.
L'adulte moyen passe environ 11 heures par jour à interagir avec les médias. Ces
statistiques indiquent un changement significatif dans la manière dont nous
consommons de l'information et nous connectons avec les autres, mais cela
soulève également des préoccupations quant à la manière dont cette connectivité
constante affecte notre bien-être.

\subsection{Le Rôle de la Dopamine}

La dopamine est un neurotransmetteur souvent associé au plaisir et à la
récompense. Elle est essentielle pour motiver des comportements bénéfiques tels
que manger, socialiser et rechercher de nouvelles expériences. Votre smartphone
déclenche la libération de dopamine dans votre cerveau, vous faisant ressentir
une récompense lorsque vous recevez des notifications, des likes ou que vous
interagissez avec du contenu divertissant. Avec le temps, ces activités
renforcent les voies neuronales associées à la libération de dopamine,
entraînant des comportements compulsifs et réduisant votre motivation à
participer à d'autres activités nécessitant effort et persévérance.

\subsection{Le Dilemme de la Motivation}

La recherche montre qu'une utilisation excessive du téléphone peut entraîner une
diminution de l'attention, des difficultés à se concentrer et une capacité
réduite à différer la gratification. Ce phénomène, appelé « décote de délai »,
peut entraîner une diminution de la motivation pour les tâches nécessitant un
effort soutenu ou offrant des récompenses différées. En réalité, passer des
heures prolongées devant des écrans a été lié à un risque accru de dépression et
d'anxiété, en particulier chez les jeunes adultes. Les adolescents qui passent
trop de temps sur leurs appareils mobiles ont 71 \% de risques en plus de
développer des facteurs de risque de suicide.

\subsection{Reconnaître l'Addiction au Téléphone}

Comprendre si vous êtes dépendant de votre téléphone est essentiel pour atténuer
son impact sur votre motivation. Posez-vous les questions suivantes :
\begin{enumerate}
	\item Avez-vous des envies de vérifier votre téléphone au détriment d'autres
	      activités ?
	\item Votre humeur change-t-elle en fonction des notifications et des
	      interactions sur les réseaux sociaux ?
	\item Trouvez-vous nécessaire de passer plus de temps sur votre téléphone pour
	      être satisfait ?
	\item Vous sentez-vous mal à l'aise ou stressé lorsque vous n'utilisez pas
	      votre téléphone ?
	\item Vos tentatives pour réduire l'utilisation du téléphone ont-elles
	      entraîné des rechutes ?
\end{enumerate}

Si vous avez répondu « oui » à ces questions, vous pourriez faire face à une
dépendance au téléphone.

\subsection{Prendre le Contrôle}

La bonne nouvelle est que vous pouvez reprendre le contrôle de votre utilisation
du téléphone et de votre motivation. Voici trois stratégies soutenues par la
science à envisager :
\begin{enumerate}
	\item \textbf{Restriction du Temps :} Allouez des plages horaires spécifiques
	      pour l'utilisation du téléphone, comme une heure par jour. Cette
	      approche empêche les vérifications compulsives et permet aux systèmes de
	      dopamine de votre cerveau de se réajuster progressivement.
	\item \textbf{Contrainte Physique :} Déconnectez-vous des applications
	      déclencheuses, confiez vos mots de passe à un ami de confiance ou placez
	      physiquement votre téléphone hors de portée à certains moments, comme
	      l'éteindre à 21 heures.
	\item \textbf{Contrainte Catégorique :} Rendez votre téléphone moins attractif
	      en utilisant le mode gris, en vérifiant les applications à forte teneur en
	      dopamine uniquement sur un ordinateur et en supprimant les applications
	      inutiles qui gaspillent votre temps.
\end{enumerate}

\subsection{Conclusion}

Dans un monde où les smartphones sont omniprésents, comprendre l'impact d'une
utilisation excessive du téléphone sur votre motivation est crucial. En
reconnaissant le rôle de la dopamine, en étant conscient des signes de
dépendance au téléphone et en mettant en œuvre des stratégies pour contrôler
votre utilisation du téléphone, vous pouvez retrouver la concentration, la
motivation et une relation plus saine avec la technologie.

N'oubliez pas que vous n'êtes pas seul dans ce voyage. De nombreuses personnes
cherchent à trouver un équilibre entre les avantages de la technologie et leur
bien-être global. En prenant des mesures pour limiter l'utilisation de votre
téléphone, vous investissez dans votre succès futur et votre bien-être.

\section{Parlons devoirs}

Nous allons traverser cette année en tant que groupe-classe. À cet effet, j'ai
un rôle particulier parmi toutes lersonnes qui le constituent, mais aucun et
aucune d'entre nous n'est exempt de devoirs (je ne parle pas des devoirs
surveillés).

\subsection{Vos devoirs}

En effet, pour assurer le bon fonctionnement et le bon déroulé des cours, il est
attendu de vous (et vous avez dû l'entendre maintes et maintes fois)~:

\begin{enumerate}
	\item que vous soyez à l'heure~;
	\item que vous soyez attentif-ves en cours~;
	\item que vous soyez assidu-es dans vos pratiques, notamment dans la
	      préparation des TDs…
\end{enumerate}

La liste n'est pas exhaustive, mais le respect d'autrui en fait évidemment
partie.

\subsection{Mes devoirs}

Cependant, trop peu souvent sont explicités les devoirs de la
personne-professeur-e. Il m'incombe pourtant~:

\begin{enumerate}
	\item d'être également à l'heure~;
	\item d'être attentive à et disponible en cours~;
	\item de préparer les TDs (et TPs, DS, DMs et interrogations)~;
\end{enumerate}
c'est-à-dire les mêmes devoirs que vous, mais j'ai quelques devoirs
supplémentaires~:
\begin{enumerate}[resume]
	\item je dois transmettre l'information de manière efficace et organisée~;
	\item je dois écouter vos incompréhensions~;
	\item je dois savoir faire évoluer ma transmission d'information pour
	      correspondre à vos manières d'apprendre, comprendre ou percevoir le
	      monde~;
	\item je dois gérer la durée des cours~;
	\item et je dois gérer tous les aspects annexes d'organisation (programme de
	      khôlle, correction des écrits, installation des expériences…).
\end{enumerate}

\subsection{Devoirs partagés}

De la même manière, nous avons des devoirs communs, pour le bien-être du
groupe-classe. Notamment, il ne m'incombe pas d'être l'unique personne à faire
respecter le silence dans la salle, mais à l'ensemble des personnes prenant part
au cours. Les discussions entre vous sont par exemple à limiter au maximum, nous
sommes presque une cinquantaine dans la salle et les voix partant du tableau
doivent déjà porter loin pour espérer atteindre les personnes du fond, inutile
d'ajouter de la difficulté. Si vous avez une question très basique («~quel est
ce mot au tableau ?~») je comprends que vous n'interrompiez pas la séance même
s'il est intéressant de signaler quand quelque chose est mal écrit par exemple.
Par contre, si votre question est plus complexe et demande un engagement verbal
certain («~j'ai pas compris comment on passe de la ligne 3 à la ligne 4~»),
posez-la moi directement, il y aura de toute manière 95\% de chance qu'une autre
personne n'ait pas compris.

Au niveau des téléphones, les vôtres comme le mien doivent être au minimum en
silencieux, si ça n'est éteints, mais surtout \textbf{dans vos sacs}. Toute
distraction venant interrompre la réflexion peut réduire à néant un long effort
de transmission de la part de quiconque se trouve au tableau à ce moment-là.

Concernant l'usage des toilettes, il est très simple~: il suffit de se lever et
d'y aller. Ne me demandez pas si vous avez le droit, c'est une évidence. Si j'ai
besoin de sortir de la classe également à un moment (travail de groupe en TD par
exemple), je ferai de même.

Concernant la nourriture, elle est interdite dans les salles. En plus de
probablement gêner la prise de notes elle risque de salir inutilement la
salle, que nous ne nettoyons pas~: c'est donc également une marque de respect
pour les personnes s'en occupant que de ne pas manger en classe. L'eau est par
contre tout à fait acceptée.

\subsection{Gestion des non-respects à ces devoirs}

Ceci ayant été exposé, il arrive à tout le monde de ne pas respecter toutes ces
attentes, et dans le cadre de respect et bienveillance nécessaire à notre
cheminement conjoint je préfère ne pas avoir de comportement punitif. Ainsi, si
vous arrivez en retard en cours je ne vous demanderai ni d'aller chercher un
mot, ni de repartir sur vos pas~: si vous venez en cours, c'est que vous en avez
envie et je respecte les possibles difficultés de chacun et chacune à être à
l'heure. Ceci n'est cependant applicable que dans le cas où ces retards ne sont
pas répétés et excessifs, et ne se font pas dans l'irrespect.

S'il y a trop de bruit dans un moment inopportun (pas en TD en groupe où la
discussion est fortement suggérée), je peux avoir différentes réaction en
fonction de la nature du bruit, mais la plus probable est que je m'asseye et
reste en silence en attendant que l'ensemble de la classe œuvre à faire
remarquer au plus grand nombre qu'il serait souhaitable pour tout le monde que
le bruit cesse.

De la même manière, un téléphone qui sonne en cours peut arriver, et je ne vous
en tiendrai pas rigueur si tant est que ça n'est pas abusif. Son utilisation par
contre, si elle ne dérange pourtant pas le groupe-classe, relève de l'irrespect
envers l'assiduité de la personne présentant au tableau. Vous ne me verrez
notamment jamais regarder mon téléphone lorsque vous serez au tableau. Si vous
avez un événement important ou un appel nécessaire n'hésitez cependant pas à me
le dire au début de l'heure où à sortir de la salle le temps de son utilisation.

\vfill

Pour toute question, vous pouvez m'écrire à
\href{nora.nicolas@ac-orleans-tours.fr}{nora.nicolas@ac-orleans-tours.fr}. Merci
pour votre attention et bonne rentrée à vous !

\vfill

\end{document}
