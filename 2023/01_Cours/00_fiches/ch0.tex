\documentclass[../main/main.tex]{subfiles}

\raggedbottom

\makeatletter
\renewcommand{\@chapapp}{Introduction -- chapitre}
\makeatother

\begin{document}
\setcounter{chapter}{-1}

\chapter{Unit\'es et analyse dimensionnelle}

En sciences physiques, il faut opérer la distinction entre~:
\begin{enumerate}
    \item \textbf{Le phénomène}, du domaine de la sensation («~il fait chaud~»)
        ou de l'observation («~une lumière blanche traversant un prisme sort en
        arc-en-ciel~»)~;
    \item \textbf{La grandeur physique}, quantité mesurable (directement ou
        indirectement) et rendant compte du phénomène (par exemple, la
        température). Elle est représentée par un \textbf{symbole} ($T$ dans le
        même exemple) qui intervient dans les équations mathématiques
        caractérisant le phénomène physique, et est caractérisée par sa
        \textbf{dimension} traduisant sa nature~;
    \item \textbf{La valeur} de la grandeur, résultant d'une mesure et associée
        à une \textbf{unité} ($K$ ou $\degreeCelsius$)~; la valeur change selon
        l'unité utilisée et contient un certain nombre de \textbf{chiffres
        significatifs}.
\end{enumerate}

Ces notions sont la fondation de tout raisonnement scientifique qui repose sur
la précision et l'objectivité.

\section{Systèmes d'unités}
\subsection{Grandeurs de base}

Les grandeurs physiques sont \textbf{reliées} entre elles, soit par des
\textbf{définitions} (surface d'un carré = carré d'un côté) soit par des
\textbf{lois} physiques ($U = RI$ en électronique). Par souci de concision, il
est pratique de choisir des grandeurs de base à partir desquelles nous
exprimerons toutes les autres~: en mécanique par exemple, nous utilisons la
longueur, la masse et le temps. Ce choix n'est pas unique mais pratique.

À partir de grandeurs de base choisies, nous leur associons donc des unités «~de
base~». Le bureau international des poids et mesures
(BIPM\footnote{\href{https://www.bipm.org/fr/measurement-units/}
{https://www.bipm.org/fr/measurement-units/}}) a défini le \textbf{système
international (SI)}, et se réuni tous les 4 ans pour discuter de leur définition
et de leur choix.

\subsection{Définition du SI}

\begin{defi}[label=def:si]{{Grandeurs, dimensions et unités de base du SI}}
    \begin{center}
        \begin{tabular}{lclc}
            \toprule
            Grandeur             & Dimension & Unité      & Symbole de l'unité\\
            \midrule
            Longueur             & L         & mètre      & m\\
            Masse                & M         & kilogramme & kg\\
            Temps                & T         & seconde    & s\\
            Intensité électrique & I         & Ampère
            \footnote{Du nom du physicien
                André-Marie \textsc{Ampère} (XVIII-XIX\ieme), précurseur de la
                mathématisation de la physique et créateur du vocabulaire tenant à
                l'électricité.}
                                                          & A\\
            Température          & $\Theta$  & Kelvin
            \footnote{Du nom du physicien William \textsc{Thomson} (XIX\ieme),
        anobli en Lord \textsc{Kelvin}, à l'origine de la thermodynamique.}
                                                          & K\\
            Quantité de matière  & N         & mole       & mol\\
            Intensité lumineuse  & J\footnote{Ne pas confondre avec l'unité des
            énergies en joules…}
                                             & candela    & cd\\
            \bottomrule
        \end{tabular}
    \end{center}
\end{defi}

On remarquera que les unités provenant d'un nom propre s'écrivent avec une
majuscule, et leur symbole l'est également.

\subsection{Grandeurs dérivées}

Les grandeurs exprimées à partir des grandeurs de bases \textit{via} des
équations physiques sont appelées «~grandeurs dérivées~». Leurs dimensions sont
écrites sous la forme de produits de puissances des dimensions de base~: d'une
manière générique, une grandeur $G$ a pour dimension

\[\boxed{[G] = \rm L^\alpha M^\beta T^\gamma I^\delta \Theta^\epsilon N^\xi
J^\eta}\]

où les lettres grecques sont les \textbf{exposants dimensionnels}, qui peuvent
être nuls. S'ils sont tous nuls, la grandeur et dite \textbf{adimensionnée}.

\begin{exem}[label=exem:grandeurs]{Grandeurs dérivées}
    \begin{center}
        \begin{tabular}{lccc}
            \toprule
            Grandeurs dérivées & Symbole  & Équation aux dimensions  & Unités SI dérivées\\
            \midrule
            Surface            & S        & $\rm [S] = L^2$          & \si{m^2}\\
            Volume             & V        & $\rm [V] = L^3$          & \si{m^3}\\
            Angle              & $\alpha$ & $\rm [\alpha] = 1$       & \si{rad}\\
            Vitesse            & $\vf$    & $\rm [v] = L⋅T^{-1}$     & \si{m.s^{-1}}\\
            Accélération       & $\af$    & $\rm [a] = L⋅T^{-2}$     & \si{m.s^{-2}}\\
            Masse volumique    & $\rho$   & $\rm [\rho] = M⋅L^{-3}$  & \si{kg.m^{-3}}\\
            Force              & $\Ff$    & $\rm [F] = M⋅L⋅T^{-2}$   & \si{kg.m.s^{-2}}\\
            Charge électrique  & $q$      & $\rm [q] = I⋅T$          & \si{A.s}\\
            Énergie            & E        & $\rm [E] = M⋅L^2⋅T^{-2}$ & \si{kg.m^2.s^{-2}}\\
            \bottomrule
        \end{tabular}
    \end{center}
    \tcbsubtitle[before skip=\baselineskip, colback=gray!50!black, colframe =
    gray!50!black]{Remarque}
    Certaines de ces unités dérivées portent des noms usuels~: le Newton N
    (\SI{1}{N} = \SI{1}{kg.m.s^{-2}}) pour la force, le Coulomb C (\SI{1}{C} =
    \SI{1}{A.s}) pour la charge électrique, ou l'énergie en Joules\footnote{Du
    physicien James \textsc{Joule} (XIX\ieme), contemporain de \textsc{Kelvin}}
    J (\SI{1}{J} = \SI{1}{kg.m^2.s^{-2}})
\end{exem}

\subsection{Préfixes multiplicatifs}

Suivant la valeur d'une grandeur, il est commode de l'exprimer \textit{via}
l'ajout d'un préfixe à l'unité. Ils s'expriment en puissances de 10 et ont
également un symbole et un nom~:
\begin{center}
    \begin{tabular}{lcclcc}
        \toprule
        \multicolumn{3}{c}{Sous-multiples} & \multicolumn{3}{c}{Multiples}\\
        \cmidrule(lr){1-3} \cmidrule(lr){4-6}
        Préfixe & Puissance  & Symbole     & Préfixe & Puissance & Symbole\\
        \midrule
        yocto   & \num{e-24} & y           & déca    & \num{e1}  & da\\
        zepto   & \num{e-21} & z           & hecto   & \num{e2}  & h\\
        atto    & \num{e-18} & a           & kilo    & \num{e3}  & k\\
        femto   & \num{e-15} & f           & méga    & \num{e6}  & M\\
        pico    & \num{e-12} & p           & giga    & \num{e9}  & G\\
        nano    & \num{e-9}  & n           & téra    & \num{e12} & T\\
        micro   & \num{e-6}  & \si{\micro} & péta    & \num{e15} & P\\
        milli   & \num{e-3}  & m           & exa     & \num{e18} & E\\
        centi   & \num{e-2}  & c           & zetta   & \num{e21} & Z\\
        déci    & \num{e-1}  & d           & yotta   & \num{e24} & Y\\
        \bottomrule
    \end{tabular}
\end{center}

\section{Analyse dimensionnelle}

À l'aide de ces outils, nous pouvons effectuer des actions sur les
équations-mêmes pour en extraire les dimensions. Pour qu'une équation
mathématique ait un sens physique, elle doit suivre un principe fondamental et
naturel~: le principe d'homogénéité.

\subsection{Homogénéité}

\begin{tcbraster}[raster columns=2, raster equal height=rows]
    \begin{prop}[label=prop:homo, hand]{homogénéité}

        Dans une équation ou dans l'expression d'une loi physique, les deux
        membres de chaque côté du signe égal doivent être de même
        nature\footnote{Scalaire, vecteur, matrice, tenseur…} et avoir la
        \textbf{même dimension}, quel que soit le système d'unités. Une telle
        formule est alors dite \textbf{homogène}.

    \end{prop}
    \begin{impl}[label=impl:homo]{dans la pratique}

        Nous ne pouvons donc égaliser un vecteur d'un côté avec un scalaire de
        l'autre, et ne pouvons égaliser, additionner ou soustraire des mètres à
        des secondes\footnote{Ou «~des patates avec des carottes~»…
        l'appréciation de l'analogie est laissée à votre appréciation.}, etc.

    \end{impl}
\end{tcbraster}

\subsection{Application}

Le principe d'homogénéité permet alors une analyse des dimensions des grandeurs
mises en jeu dans une loi ou une équation. C'est un outil particulièrement
puissant à bien des égards, que nous voyons ci-après.

\subsubsection{Rechercher des unités}

En connaissant une expression que l'on sait vraie, nous pouvons déduire les
unités d'autres grandeurs (cf.\ les unités usuelles comme le Newton).

\begin{exem}[label=exem:homounit]{recherche d'unités}
    La force de rappel élastique exercée par un ressort s'écrit
    \[\Ff_{\rm el} = -k(l-l_0)\ux\]
    avec $k$ la constante de raideur du ressort. Quelle est la dimension de $k$
    ? Comment exprimer son unité ?
    \tcblower
    \vspace{3cm}
\end{exem}

\subsubsection{Détecter des erreurs}

Par simple analyse dimensionnelle, il est aisé d'affirmer qu'un résultat est
nécessairement faux~: si les deux parties mises en jeu n'ont pas la même
dimension, elle ne peuvent être égales entre elles~!

\begin{exem}[label=exem:homoerr]{détecter des erreurs}

    En résolvant un exercice, vous trouvez l'expression suivante pour l'énergie
    potentielle d'une masse $m$ accrochée à un ressort vertical de raideur $k$
    et sous pesanteur $g$~:
    \[E_{\rm p} (z) = \frac{1}{2}kz^2 + mgz^2\]
    avec $z$ la hauteur de la masse. Cette expression est-elle homogène~?
    \tcblower
    \vspace{6cm}
\end{exem}

\subsubsection{Rechercher des lois physiques}
D'autre part, à partir de phénomènes que nous voudrions relier entre eux, il est
possible d'établir des lois les reliant entre eux grâce au principe
d'homogénéité.

\begin{exem}[label=exem:homoloi]{recherche de loi}

    Donnez, par analyse dimensionnelle, la période $T$ des oscillations d'un
    pendule simple.
    \tcblower
    \vspace{6cm}
\end{exem}

\newpage
\section{Exercices}
\subsection{Vitesse du son}

Donner l'expression de la célérité $c$ du son dans un fluide en fonction de la
masse volumique du $\rho$ du fluide et du coefficient d'incompressibilité
$\chi$, homogène à l'inverse d'une pression.

\subsection{Faire cuire des pâtes}
Sur une facture d'électricité, on peut lire sa consommation d'énergie électrique
exprimée en \si{kWh} (kilowatt-heure).
\begin{enumerate}
    \item Quelle est l'unité SI associée~? Que vaut \SI{1}{kWh} dans cette unité
        SI~?
    \item Sachant que la capacité thermique massique\endnote{capacité thermique
            massique~: énergie à apporter pour augmenter de
        \SI{1}{\degreeCelsius} la température de l'unité de masse d'une
    substance.} de l'eau est $c = \SI{4.18}{J.g^{-1}.K^{-1}}$ et que le prix du
    kilowatt-heure est de \SI{0.16}{\EUR}, évaluer le coût du chauffage électrique
    permettant de faire passer \SI{1}{L} d'eau de \SI{20}{\degreeCelsius} à
    \SI{100}{\degreeCelsius}.
\item Si la plaque chauffe avec une puissance de $P = \SI{1200}{W}$, combien de
    temps faudra-t-il pour chauffer ce litre d'eau~?
\end{enumerate}

\subsection{\textsc{Taylor} mieux que James \textsc{Bond}~?}

À l'aide d'un film sur bande magnétique et en utilisant l'analyse
dimensionnelle, le physicien Geoffrey \textsc{Taylor} a réussi en 1950 à estimer
l'énergie $E$ dégagée par une explosion nucléaire, valeur pourtant évidemment
classifiée. Le film permet d'avoir accès à l'évolution du rayon $R(t)$ du
«~nuage~» de l'explosion au cours du temps. Nous supposons que les grandeurs
influant sur ce rayon sont le temps $t$, l'énergie $E$ de l'explosion et la
masse volumique $\rho$ de l'air.
\begin{enumerate}
    \item Quelles sont les dimensions de ces grandeurs~?
    \item Chercher une expression de $R$ sous la forme $R = k\times
        E^{\alpha}t^\beta\rho^\gamma$, avec $k$ une constante adimensionnée.
    \item L'analyse du film montre que le rayon augmente au cours du temps comme
        $t^{2/5}$. Exprimer alors $E$ en fonction de $R$, $\rho$ et $t$.
    \item En estimant que $R\approx \SI{70}{m}$ après $t = \SI{1}{ms}$, sachant
        que la masse volumique de l'air vaut $\rho\approx \SI{1.0}{kg.m^{-3}}$
        et en prenant $K\approx 1$, calculer la valeur de $E$ en joules puis en
        kilotonnes de TNT (une tonne de TNT libère \SI{4.18e9}{J}).
\end{enumerate}

\newpage
\section{Correction}
\subsection{Vitesse du son}

\begin{tcbraster}[raster columns=3, raster equal height=rows]
    \begin{NCdefi}[]{Données}
    
        $c$ est une vitesse, $\rho$ une masse volumique et $\chi$ une grandeur
        relative à la pression. On nous donne $[\chi] = [P]^{-1}$ avec $P$ une
        pression.
    
    \end{NCdefi}
    \begin{NCprop}[]{Résultat attendu}
    
        On cherche $c$ en fonction de $\rho$ et $\chi$, soit
        \[\boxed{c = \rho^\alpha\chi^\beta}\]
        avec $\alpha$ et $\beta$ à déterminer.
    
    \end{NCprop}
    \begin{NCrapp}[]{Outil}

        Une pression est une force surfacique, c'est-à-dire une force répartie
        sur une surface. On a donc

        \[[P] = \frac{[F]}{\rm L^2}\]

        De plus, la force de pesanteur s'exprime $F = mg$, avec $g$
        l'accélération de la pesanteur~: ainsi,

        \[[F] = [m]⋅[g] = \rm M⋅L⋅T^{-2}\]
    \end{NCrapp}
\end{tcbraster}
~
\begin{NCexem}[sidebyside]{Application}
    On commence par déterminer la dimension de $c$. En tant que vitesse, on a
    \[[c] = \rm L⋅T^{-1}\]
    On exprime ensuite les dimensions de $\rho$ et $\chi$. D'une part,
    \[[\rho] = \rm M⋅L^{-3}\]
    D'autre part,
    \begin{align*}
        [\chi] & = \DS\frac{\rm L^2}{[F]}\\
        [\chi] & = \DS\frac{\rm L^{\cancel{2}}}{M⋅\cancel{L}⋅T^{-2}}\\
        [\chi] & = \rm L⋅M^{-1}⋅T^2\\
    \end{align*}
    \tcblower
    L'expression recherchée revient à résoudre
    \[\rm L⋅T^{-1} = (M⋅L^{-1})^\alpha(L⋅M^{-1}⋅T^2)^\beta\]
    En développant, on trouve un système de 3 équations à 2 inconnues~:
    \[ \left\{
            \begin{array}{rcl}
                1  & = & -3\alpha + \beta\\
                -1 & = & 2\beta\\
                0  & = & \alpha - \beta\\
            \end{array}
        \right. \Longleftrightarrow \left\{
            \begin{array}{rcl}
                \beta  & = & - \frac{1}{2}\\
                \alpha & = & - \frac{1}{2}
            \end{array}
    \right.\]
    Ainsi, on peut exprimer $c$ tel que
    \begin{empheq}[box=\fbox]{equation*}
        c = \frac{1}{\sqrt{\rho\chi}}
    \end{empheq}
\end{NCexem}

\newpage

\subsection{Cuisson des pâtes}

\begin{enumerate}
    \item ~
        \begin{tcbraster}[raster columns=3, raster equal height=rows]
            \begin{tcolorbox}[blankest, space to=\myspace]
                \begin{tcbraster}[raster columns=1]
                    \begin{NCdefi}[]{Donnée}
                        Consommation électrique en \si{kWh}.
                    \end{NCdefi}
                    \begin{NCprop}[add to natural height=\myspace]{Résultat attendu}
                        Unité associée en unités SI et grandeurs usuelles.
                    \end{NCprop}
                \end{tcbraster}
            \end{tcolorbox}
        \begin{NCrapp}[raster multicolumn=1]{Outil}
            Toute énergie s'exprime en joules (J), et les \textbf{puissances} sont des
            \textbf{énergies par unité de temps}. Notamment pour les
            watts on a $\SI{1}{W} = \SI{1}{J.s^{-1}}$.
        \end{NCrapp}
        \begin{NCexem}[raster multicolumn=1]{Application}
            On a directement
            \[ \SI{1}{kWh} = \SI{1e3}{J.s^{-1}.h}\]
            Avec l'évidence que $ \SI{1}{h} = \SI{3600}{s}$, finalement
            \[\boxed{\SI{1}{kWh} = \SI{3.6e6}{J}}\]
        \end{NCexem}
        \end{tcbraster}
    \item ~
        \begin{tcbraster}[raster columns=2, raster equal height=rows]
            \begin{NCdefi}[]{Données}
                Notre objet d'étude est l'eau. On a~:
                \begin{itemize}
                    \item $V_{\rm eau} = \SI{1}{L}$~;
                    \item $T_{\rm i} = \SI{20}{\degreeCelsius}$~;
                    \item $T_{\rm f} = \SI{100}{\degreeCelsius}$~;
                    \item $c = \SI{4.18}{J.g^{-1}.K^{-1}}$
                \end{itemize}
                De plus, on nous donne
                \begin{itemize}
                    \item $ \SI{1}{kWh} = \SI{1}{\EUR}$.
                \end{itemize}
            \end{NCdefi}
            \begin{tcolorbox}[blankest, raster multicolumn=1, space to=\myspace]
                \begin{tcbraster}[raster columns=1]
                    \begin{NCprop}[add to natural height=\myspace]{Résultat attendu}

                        On cherche à monter \SI{1}{L} d'eau de 20 à
                        \SI{100}{\degreeCelsius} et d'en calculer le coût en
                        euros.

                    \end{NCprop}
                    \begin{NCrapp}[]{Outil}

                        On doit donc trouver le coût en énergie et le convertir
                        en euro. On cherche pour ça une loi reliant l'énergie
                        consommée avec les données du problème, sachant que
                        \textbf{pour l'eau}, \SI{1}{L} = \SI{1}{kg}.

                    \end{NCrapp}
                \end{tcbraster}
            \end{tcolorbox}
        \end{tcbraster}
        \begin{NCexem}[]{Application}
           L'énergie à apporter $Q$ se déduit de la dimension de la capacité
           thermique massique~: $[c] = [Q]\rm⋅M^{-1}⋅\Theta^{-1}$. En appelant
           $m$ la masse du volume d'eau, par cette analyse dimensionnelle on a
           \[\boxed{Q = mc\Delta T}\]
           On a donc
           \[Q = \SI{3.3e5}{J}\quad\text{avec}\quad \left\{
                   \begin{array}{rcl}
                       m & = & \SI{1}{kg}\\
                       c & = & \SI{4.18}{J.g^{-1}.K^{-1}}\\
                       c & = & \SI{4.18e3}{J.kg^{-1}.K^{-1}}\\
                       \Delta T & = & \SI{80}{K}
                   \end{array}
           \right.\]
           et pour utiliser le coût en euros, on la converti en \si{kWh}~:
           \[Q = \SI{9.3e-2}{kWh} = \SI{1.5e-2}{\EUR}\]
        \end{NCexem}
    \item ~
        \begin{tcbraster}[raster columns=2, raster equal height=rows]
            \begin{tcolorbox}[blankest, raster multicolumn=1, space to=\myspace]
                \begin{tcbraster}[raster columns=1]
                    \begin{NCdefi}[]{Données}

                        On utilise une plaque chauffante de puissance $P =
                        \SI{1200}{W}$.

                    \end{NCdefi}
                    \begin{NCprop}[]{Résultat attendu}

                        On cherche la durée que cette plaque prendrait pour
                        transférer l'énergie calculée précédemment.

                    \end{NCprop}
                    \begin{NCrapp}[]{Outil}

                        Une puissance est une énergie par unité de temps, et
                        \SI{1}{W} = \SI{1}{J.s^{-1}}.

                    \end{NCrapp}
                \end{tcbraster}
            \end{tcolorbox}
            \begin{NCexem}[raster multicolumn=1]{Application}
                On en déduit
                \[P = \frac{Q}{\Delta t}
                    \quad\text{d'où}\quad
                \boxed{\Delta t = \frac{Q}{P} = \SI{280}{s}}\]
                avec
                \[\left\{
                    \begin{array}{rcl}
                        Q & = & \SI{3.3e5}{J}\\
                        P & = & \SI{1200}{J⋅s^{-1}}
                    \end{array}
                \right.\]
            \end{NCexem}
        \end{tcbraster}
\end{enumerate}

\subsection{\textsc{Taylor} meilleur que James \textsc{Bond}~?}

\begin{enumerate}
    \item On a directement \fbox{$[R] = \rm L$}, \fbox{$[t] = \rm T$},
        \fbox{$[\rho] = \rm M⋅L^{-3}$} et \fbox{$[E] = \rm M⋅L^2⋅T^{-2}$}.
    \item ~
        \begin{tcbraster}[raster columns=2, raster equal height=rows]
            \begin{tcolorbox}[blankest, raster multicolumn=1, space to=\myspace]
                \begin{tcbraster}[raster columns=1]
                    \begin{NCdefi}[]{Données}

                        On nous donne la formule $R = k\times
                        E^{\alpha}t^\beta\rho^\gamma$ et que $[k] = 1$.

                    \end{NCdefi}
                    \begin{NCprop}[]{Résultat attendu}

                        On cherche $\alpha$, $\beta$ et $\gamma$ tels que $R =
                        k\times E^{\alpha}t^\beta\rho^\gamma$

                    \end{NCprop}
                    \begin{NCrapp}[add to natural height=\myspace]{Outils}

                        \begin{itemize}
                            \item $[E] = \rm M⋅L^2⋅T^{-2}$~;
                            \item $[t] = \rm T$~;
                            \item $[\rho] = \rm M⋅L^{-3}$.
                        \end{itemize}

                    \end{NCrapp}
                \end{tcbraster}
            \end{tcolorbox}
            \begin{NCexem}[]{Application}
                $[R] = L$, donc on a 
                \[ L = \left( M⋅L^2⋅T^{-2} \right)^\alpha T^\beta \left(
                M⋅L^{-3} \right)^\gamma\]
                Soit
                \[\left\{
                    \begin{array}{rcl}
                        1 & = & 2\alpha - 3\gamma\\
                        0 & = & -2\alpha + \beta\\
                        0 & = & \alpha + \gamma
                    \end{array}
                \right.\Longleftrightarrow
                \left\{
                \begin{array}{rcl}
                    \alpha & = & -\gamma\\
                    \alpha & = & \beta/2\\
                    \alpha & = & 1/5
                \end{array}
                \right.\]
                Ainsi,
                \[\left\{
                    \begin{array}{rcl}
                        \alpha & = & 1/5\\
                        \gamma & = & -1/5\\
                        \beta  & = & 2/5
                    \end{array}
                \right.\]
                Soit
                \[R = K\times E^{1/5}t^{2/5}\rho^{-1/5}\]
            \end{NCexem}
        \end{tcbraster}

    \item On isole simplement en mettant la relation à la puissance 5~: \fbox{$E
        = K^{-5}R^5 t^{-2}\rho$}.

    \item On fait une simple application numérique~:
        \[E = \SI{1.7e15}{J}\quad\text{avec}\quad \left\{
                \begin{array}{rcl}
                    K & = & 1\\
                    R & = & \SI{70}{m}\\
                    t & = & \SI{1e-3}{s}\\
                    \rho & = & \SI{1.0}{kg.m^{-3}}
                \end{array}
        \right.\]
        En équivalent tonne de TNT, on trouve~:
        \[\boxed{E = \SI{40}{kT}\text{ de TNT}}\]
\end{enumerate}
\end{document}
