\documentclass[../main/main.tex]{subfiles}

\makeatletter
\renewcommand{\@chapapp}{Introduction -- chapitre}
\makeatother

\toggletrue{student}
\HideSolutionstrue

\begin{document}
\setcounter{chapter}{-1}

\chapter{\switch{TD~: Unit\'es et analyse dimensionnelle}{Correction du TD}}
\section{Vitesse du son}

\switch{
	Donner l'expression de la célérité $c$ du son dans un fluide en fonction de la
	masse volumique du $\rho$ du fluide et du coefficient d'incompressibilité
	$\chi$, homogène à l'inverse d'une pression.
}{
	\begin{tcb}[](data){Données}
		$c$ est une vitesse, $\rho$ une masse volumique et $\chi$ une grandeur
		relative à la pression. On nous donne $\dim{\chi} = \dim{P}^{-1}$ avec $P$
		une pression.
	\end{tcb}
	\begin{tcb}[](ques){Résultat attendu}
		On cherche $c$ en fonction de $\rho$ et $\chi$, soit
		\[
			\boxed{c = \rho^\alpha\chi^\beta}
		\]
		avec $\alpha$ et $\beta$ à déterminer.
	\end{tcb}
	\begin{tcb}[](tool){Outil}
		Une pression est une force surfacique, c'est-à-dire une force répartie
		sur une surface. On a donc
		\[
			\dim{P} = \frac{\dim{F}}{\rm L^2}
		\]
		De plus, la force de pesanteur s'exprime $F = mg$, avec $g$
		l'accélération de la pesanteur~: ainsi,
		\[
			\dim{F} = \dim{m}\cdot \dim{g} = \rm M\cdot L\cdot T^{-2}
		\]
	\end{tcb}
	\begin{tcb}[sidebyside](appl){Application}
		On commence par déterminer la dimension de $c$. En tant que vitesse, on a
		\[\dim{c} = \rm L\cdot T^{-1}\]
		On exprime ensuite les dimensions de $\rho$ et $\chi$. D'une part,
		\[\dim{\rho} = \rm M\cdot L^{-3}\]
		D'autre part,
		\begin{align*}
			\dim{\chi} & = \DS\frac{\rm L^2}{\dim{F}}   \\
			\dim{\chi} & = \DS\frac{\rm L^{\cancel{2}}}
			{\rm M\cdot \cancel{\rm L}\cdot T^{-2}}     \\
			\dim{\chi} & = \rm L\cdot M^{-1}\cdot T^2   \\
		\end{align*}
		\tcblower
		L'expression recherchée revient à résoudre
		\[\rm L\cdot T^{-1} = (M\cdot L^{-1})^\alpha(L\cdot M^{-1}\cdot T^2)^\beta\]
		En développant, on trouve un système de 3 équations à 2 inconnues~:
		\[ \left\{
			\begin{array}{rcl}
				1  & = & -3\alpha + \beta \\
				-1 & = & 2\beta           \\
				0  & = & \alpha - \beta   \\
			\end{array}
			\right. \Longleftrightarrow \left\{
			\begin{array}{rcl}
				\beta  & = & - \frac{1}{2} \\
				\alpha & = & - \frac{1}{2}
			\end{array}
			\right.\]
		Ainsi, on peut exprimer $c$ tel que
		\begin{empheq}[box=\fbox]{equation*}
			c = \frac{1}{\sqrt{\rho\chi}}
		\end{empheq}
	\end{tcb}
}

\section{Faire cuire des pâtes}
\switch{
	Sur une facture d'électricité, on peut lire sa consommation d'énergie
	électrique exprimée en \si{kWh} (kilowatt-heure).
	\begin{enumerate}
		\item Quelle est l'unité SI associée~? Que vaut \SI{1}{kWh} dans cette unité
		      SI~?
		\item Sachant que la capacité thermique massique de l'eau est $c =
			      \SI{4.18}{J.g^{-1}.K^{-1}}$ et que le prix du kilowatt-heure est de
		      \SI{0.16}{\EUR}, évaluer le coût du chauffage électrique permettant de
		      faire passer \SI{1}{L} d'eau de \SI{20}{\degreeCelsius} à
		      \SI{100}{\degreeCelsius}.
		\item Si la plaque chauffe avec une puissance de $P = \SI{1200}{W}$, combien
		      de temps faudra-t-il pour chauffer ce litre d'eau~?
	\end{enumerate}
}
{
	\begin{enumerate}
		\item ~
		      \begin{tcbraster}[raster columns=2]
			      \begin{tcb}[](data){Donnée}
				      Consommation électrique en \si{kWh}.
			      \end{tcb}
			      \begin{tcb}(ques)'r'{Résultat attendu}
				      Unité associée en unités SI et grandeurs usuelles.
			      \end{tcb}
		      \end{tcbraster}
		      \begin{tcb}(tool){Outil}
			      Toute énergie s'exprime en joules (J), et les \textbf{puissances}
			      sont des \textbf{énergies par unité de temps}. Notamment pour les
			      watts on a $\SI{1}{W} = \SI{1}{J.s^{-1}}$.
		      \end{tcb}
		      \begin{tcb}(appl){Application}
			      On a directement
			      \[ \SI{1}{kWh} = \SI{1e3}{J.s^{-1}.h}\]
			      Avec l'évidence que $ \SI{1}{h} = \SI{3600}{s}$, finalement
			      \[\xul{}{\SI{1}{kWh} = \SI{3.6e6}{J}}\]
		      \end{tcb}
		\item ~
		      \begin{tcbraster}[raster columns=2, raster equal height=rows]
			      \begin{tcb}[](data){Données}
				      Notre objet d'étude est l'eau. On a~:
				      \begin{itemize}
					      \item $V_{\rm eau} = \SI{1}{L}$~;
					      \item $T_{\rm i} = \SI{20}{\degreeCelsius}$~;
					      \item $T_{\rm f} = \SI{100}{\degreeCelsius}$~;
					      \item $c = \SI{4.18}{J.g^{-1}.K^{-1}}$
				      \end{itemize}
				      De plus, on nous donne
				      \begin{itemize}
					      \item $ \SI{1}{kWh} = \SI{1}{\EUR}$.
				      \end{itemize}
			      \end{tcb}
			      \begin{tcolorbox}[blankest, raster multicolumn=1, space to=\myspace]
				      \begin{tcbraster}[raster columns=1]
					      \begin{tcb}[add to natural height=\myspace](ques)'r'{Résultat
						      attendu}
						      On cherche à monter \SI{1}{L} d'eau de 20 à
						      \SI{100}{\degreeCelsius} et d'en calculer le coût en
						      euros.
					      \end{tcb}
					      \begin{tcb}[](tool)'r'{Outil}
						      On doit donc trouver le coût en énergie et le convertir
						      en euro. On cherche pour ça une loi reliant l'énergie
						      consommée avec les données du problème, sachant que
						      \textbf{pour l'eau}, \SI{1}{L} = \SI{1}{kg}.
					      \end{tcb}
				      \end{tcbraster}
			      \end{tcolorbox}
		      \end{tcbraster}
		      \begin{tcb}[](appl){Application}
			      L'énergie à apporter $Q$ se déduit de la dimension de la capacité
			      thermique massique~: $\dim{c} = \dim{Q}\rm\cdot M^{-1}\cdot
				      \Theta^{-1}$. En appelant $m$ la masse du volume d'eau, par cette
			      analyse dimensionnelle on a
			      \[\boxed{Q = mc\Delta T}\]
			      On a donc
			      \[Q = \SI{3.3e5}{J}\quad\text{avec}\quad \left\{
				      \begin{array}{rcl}
					      m        & = & \SI{1}{kg}                    \\
					      c        & = & \SI{4.18}{J.g^{-1}.K^{-1}}    \\
					      c        & = & \SI{4.18e3}{J.kg^{-1}.K^{-1}} \\
					      \Delta T & = & \SI{80}{K}
				      \end{array}
				      \right.\]
			      et pour utiliser le coût en euros, on la converti en \si{kWh}~:
			      \[\xul{Q = \SI{9.3e-2}{kWh} = \SI{1.5e-2}{\EUR}}\]
		      \end{tcb}
		\item ~
		      \begin{tcbraster}[raster columns=2, raster equal height=rows]
			      \begin{tcb}[](data){Données}

				      On utilise une plaque chauffante de puissance $P =
					      \SI{1200}{W}$.

			      \end{tcb}
			      \begin{tcb}[](ques)'r'{Résultat attendu}

				      On cherche la durée que cette plaque prendrait pour
				      transférer l'énergie calculée précédemment.

			      \end{tcb}
		      \end{tcbraster}
		      \begin{tcbraster}[raster columns=2, raster equal height=rows]
			      \begin{tcb}[](tool){Outil}

				      Une puissance est une énergie par unité de temps, et
				      \SI{1}{W} = \SI{1}{J.s^{-1}}.

			      \end{tcb}
			      \begin{tcb}(appl)'r'{Application}
				      On en déduit
				      \begin{gather*}
					      P = \frac{Q}{\Delta t}
					      \qav
					      \left\{
					      \begin{array}{rcl}
						      Q & = & \SI{3.3e5}{J}            \\
						      P & = & \SI{1200}{J\cdot s^{-1}}
					      \end{array}
					      \right.
					      \\
					      \text{d'où}\quad
					      \boxed{\Delta t = \frac{Q}{P} = \xul{\SI{280}{s}}}
				      \end{gather*}
			      \end{tcb}
		      \end{tcbraster}
	\end{enumerate}
}

\section{\textsc{Taylor} mieux que James \textsc{Bond}~?}
\switch{
	À l'aide d'un film sur bande magnétique et en utilisant l'analyse
	dimensionnelle, le physicien Geoffrey \textsc{Taylor} a réussi en 1950 à
	estimer l'énergie $E$ dégagée par une explosion nucléaire, valeur pourtant
	évidemment classifiée. Le film permet d'avoir accès à l'évolution du rayon
	$R(t)$ du «~nuage~» de l'explosion au cours du temps. Nous supposons que les
	grandeurs influant sur ce rayon sont le temps $t$, l'énergie $E$ de
	l'explosion et la masse volumique $\rho$ de l'air.
	\begin{enumerate}
		\item Quelles sont les dimensions de ces grandeurs~?
		\item Chercher une expression de $R$ sous la forme $R = k\times
			      E^{\alpha}t^\beta\rho^\gamma$, avec $k$ une constante adimensionnée.
		\item L'analyse du film montre que le rayon augmente au cours du temps comme
		      $t^{2/5}$. Exprimer alors $E$ en fonction de $R$, $\rho$ et $t$.
		\item En estimant que $R\approx \SI{70}{m}$ après $t = \SI{1}{ms}$, sachant
		      que la masse volumique de l'air vaut $\rho\approx \SI{1.0}{kg.m^{-3}}$
		      et en prenant $K\approx 1$, calculer la valeur de $E$ en joules puis en
		      kilotonnes de TNT (une tonne de TNT libère \SI{4.18e9}{J}).
	\end{enumerate}
}{
	\begin{enumerate}
		\item On a directement \fbox{$\dim{R} = \rm L$}, \fbox{$\dim{t} = \rm T$},
		      \fbox{$\dim{\rho} = \rm M\cdot L^{-3}$} et \fbox{$\dim{E} = \rm M\cdot
				      L^2\cdot T^{-2}$}.
		\item ~
		      \begin{tcbraster}[raster columns=2, raster equal height=rows]
			      \begin{tcb}[](data){Données}

				      On nous donne la formule $R = k\times
					      E^{\alpha}t^\beta\rho^\gamma$ et que $\dim{k} = 1$.

			      \end{tcb}
			      \begin{tcb}[](ques)'r'{Résultat attendu}

				      On cherche $\alpha$, $\beta$ et $\gamma$ tels que $R =
					      k\times E^{\alpha}t^\beta\rho^\gamma$

			      \end{tcb}
		      \end{tcbraster}
		      \begin{tcb}(tool){Outils}

			      \begin{itemize}
				      \item $\dim{E} = \rm M\cdot L^2\cdot T^{-2}$~;
				      \item $\dim{t} = \rm T$~;
				      \item $\dim{\rho} = \rm M\cdot L^{-3}$.
			      \end{itemize}

		      \end{tcb}
		      \begin{tcb}[](appl){Application}
			      $\dim{R} = L$, donc on a
			      \[ L = \left( M\cdot L^2\cdot T^{-2} \right)^\alpha T^\beta \left(
				      M\cdot L^{-3} \right)^\gamma\]
			      Soit
			      \[\left\{
				      \begin{array}{rcl}
					      1 & = & 2\alpha - 3\gamma \\
					      0 & = & -2\alpha + \beta  \\
					      0 & = & \alpha + \gamma
				      \end{array}
				      \right.\Longleftrightarrow
				      \left\{
				      \begin{array}{rcl}
					      \alpha & = & -\gamma \\
					      \alpha & = & \beta/2 \\
					      \alpha & = & 1/5
				      \end{array}
				      \right.\]
			      Ainsi,
			      \[\left\{
				      \begin{array}{rcl}
					      \alpha & = & 1/5  \\
					      \gamma & = & -1/5 \\
					      \beta  & = & 2/5
				      \end{array}
				      \right.\]
			      Soit
			      \[R = K\times E^{1/5}t^{2/5}\rho^{-1/5}\]
		      \end{tcb}

		\item On isole simplement en mettant la relation à la puissance 5~: \fbox{$E
				      = K^{-5}R^5 t^{-2}\rho$}.

		\item On fait une simple application numérique~:
		      \[E = \SI{1.7e15}{J}\quad\text{avec}\quad \left\{
			      \begin{array}{rcl}
				      K    & = & 1                   \\
				      R    & = & \SI{70}{m}          \\
				      t    & = & \SI{1e-3}{s}        \\
				      \rho & = & \SI{1.0}{kg.m^{-3}}
			      \end{array}
			      \right.\]
		      En équivalent tonne de TNT, on trouve~:
		      \[\xul{E = \SI{40}{kT}\text{ de TNT}}\]
	\end{enumerate}
}

\end{document}
