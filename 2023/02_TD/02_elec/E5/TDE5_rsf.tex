\documentclass[a4paper, 12pt, final, garamond]{book}
\usepackage{cours-preambule}

\raggedbottom

\makeatletter
\renewcommand{\@chapapp}{\'Electrocin\'etique -- chapitre}
\makeatother

\toggletrue{student}
\HideSolutionstrue
\toggletrue{corrige}

\begin{document}
\setcounter{chapter}{4}

\chapter{\cswitch{Correction du TD}{TD~: circuits \'electriques en RSF}}

\section{Impédance équivalente}
\iftoggle{student}{ % pour version student
	\iftoggle{corrige}{ % version avec corrigé : affiche juste la correction
		\begin{enumerate}
			\item On commence par convertir le circuit avec les impédances complexes~:
			      \begin{itemize}
				      \item $\ul{Z}_{C_1} = \frac{1}{\jj C_1\w}$~;
				      \item $\ul{Z}_L = \jj L\w$~;
				      \item $\ul{Z}_{C_2} = \frac{1}{\jj C_2\w}$.
			      \end{itemize}
			      On peut ensuite déterminer l'impédance équivalente à l'association en
			      parallèle de $L$ et $C_2$. Avec les admittances, on a
			      \begin{gather*}
				      \ul{Z}\ind{eq,1}
				      = \frac{1}{\frac{1}{\ul{Z}_{C_2}} + \frac{1}{\ul{Z}_L}}
				      = \frac{1}{\jj C_2\w + \frac{1}{\jj L\w}}
				      = \frac{\jj L\w}{1 - \w^2LC_2}
			      \end{gather*}
			      Il suffit alors de faire l'association en série de $\ul{Z}_{C_1}$ et de
			      $\ul{Z}\ind{eq,1}$~:
			      \begin{gather*}
				      \boxed{
					      \ul{Z}\ind{eq} = \jj C_1\w + \frac{\jj L\w}{1 - \w^2LC_2}}
			      \end{gather*}
			      Il n'est ici pas nécessaire d'aller plus loin dans le calcul.

			\item Ici, on utilise que $\ul{Z}_R = R$ et comme précédemment, on effectue
			      l'association en parallèle des $R$ et $C$ de droite avant de faire
			      l'association en série de $R$ et $C$ de gauche avec cette impédance
			      équivalente~:
			      \begin{gather*}
				      \ul{Z}\ind{eq, 1}
				      = \frac{1}{\frac{1}{\ul{Z}_1} + \frac{1}{\ul{Z}_C}}
				      = \frac{1}{\frac{1}{R} + \jj C\w}
				      = \frac{R}{1 + \jj RC\w}
			      \end{gather*}
			      Et on a donc finalement
			      \begin{gather*}
				      \boxed{
					      \ul{Z}\ind{eq} = R + \frac{1}{\jj C\w} + \frac{R}{1 + \jj RC\w}}
			      \end{gather*}
		\end{enumerate}
	}{ % énoncé
		Déterminer l'impédance complexe équivalente de chacun des dipôles ci-dessous
		en RSF.
		\begin{center}
			\includegraphics[width=.8\linewidth]{exo1_plain}
		\end{center}
	}%
}{% pour version prof
	\iftoggle{corrige}{% pour version avec corrigé : question ET réponse
		Déterminer l'impédance complexe équivalente de chacun des dipôles ci-dessous
		en RSF.
		\begin{center}
			\includegraphics[width=.8\linewidth]{exo1_plain}
		\end{center}
		\begin{answ}%
			\begin{enumerate}
				\item On commence par convertir le circuit avec les impédances complexes~:
				      \begin{itemize}
					      \item $\ul{Z}_{C_1} = \frac{1}{\jj C_1\w}$~;
					      \item $\ul{Z}_L = \jj L\w$~;
					      \item $\ul{Z}_{C_2} = \frac{1}{\jj C_2\w}$.
				      \end{itemize}
				      On peut ensuite déterminer l'impédance équivalente à l'association
				      en parallèle de $L$ et $C_2$. Avec les admittances, on a
				      \begin{gather*}
					      \ul{Z}\ind{eq,1}
					      = \frac{1}{\frac{1}{\ul{Z}_{C_2}} + \frac{1}{\ul{Z}_L}}
					      = \frac{1}{\jj C_2\w + \frac{1}{\jj L\w}}
					      = \frac{\jj L\w}{1 - \w^2LC_2}
				      \end{gather*}
				      Il suffit alors de faire l'association en série de $\ul{Z}_{C_1}$
				      et de $\ul{Z}\ind{eq,1}$~:
				      \begin{gather*}
					      \boxed{
						      \ul{Z}\ind{eq} = \jj C_1\w + \frac{\jj L\w}{1 - \w^2LC_2}}
				      \end{gather*}
				      Il n'est ici pas nécessaire d'aller plus loin dans le calcul.

				\item Ici, on utilise que $\ul{Z}_R = R$ et comme précédemment, on effectue
				      l'association en parallèle des $R$ et $C$ de droite avant de faire
				      l'association en série de $R$ et $C$ de gauche avec cette impédance
				      équivalente~:
				      \begin{gather*}
					      \ul{Z}\ind{eq, 1}
					      = \frac{1}{\frac{1}{\ul{Z}_1} + \frac{1}{\ul{Z}_C}}
					      = \frac{1}{\frac{1}{R} + \jj C\w}
					      = \frac{R}{1 + \jj RC\w}
				      \end{gather*}
				      Et on a donc finalement
				      \begin{gather*}
					      \boxed{
						      \ul{Z}\ind{eq} = R + \frac{1}{\jj C\w} + \frac{R}{1 + \jj RC\w}}
				      \end{gather*}
			\end{enumerate}
		\end{answ}%
	}{ % énoncé uniquement
		Déterminer l'impédance complexe équivalente de chacun des dipôles ci-dessous
		en RSF.
		\begin{center}
			\includegraphics[width=.8\linewidth]{exo1_plain}
		\end{center}
	}
}

\section{Circuit RL série en RSF}
On considère le circuit ci-contre en régime sinusoïdal forcé, où la source de
tension impose $e(t) = E\cos(\wt)$ avec $E > 0$.

\begin{minipage}{0.60\linewidth}
	\begin{enumerate}
		\item Déterminer l'amplitude de $u$ à «~très haute~» ($\w\rightarrow\infty$)
		      et «~très basse~» ($\w\rightarrow0$) fréquence.
		\item Exprimer l'amplitude complexe $\ul{U}$ de $u(t)$ en fonction de $E$,
		      $R$, $L$ et $\w$.
		\item Les tensions $e$ et $u$ peuvent-elles être en phase~? En opposition de
		      phase~? En quadrature de phase~? Préciser le cas échéant pour quelle(s)
		      pulsation(s).
	\end{enumerate}
\end{minipage}
\hfill
\begin{minipage}{0.35\linewidth}
	\begin{center}
		\includegraphics[width=\linewidth]{exo2_plain}
	\end{center}
\end{minipage}

\section{Exploitation d'un oscillogramme en RSF}
On considère le circuit ci-dessous. On pose $e(t) = E_m\cos(\wt)$ et $u(t) =
	U_m\cos(\wt+\f)$. La figure ci-dessous représente un oscillogramme réalisé à la
fréquence $f = \SI{1.2e3}{Hz}$, avec $R = \SI{1.0}{k\Omega}$ et $C =
	\SI{0.10}{\micro F}$.
\begin{center}
	\includegraphics[width=\linewidth]{exo3_plain}
\end{center}
\begin{enumerate}
	\item Déduire de cet oscillogramme les valeurs expérimentales de $E_m$,
	      $U_m$ et $\f$.
	\item Exprimer $U_m$ et $\f$ en fonction des composants du circuit.
	\item En déduire la valeur numérique de l'inductance $L$ de la bobine.
\end{enumerate}

\section{Comportement d'un circuit à haute et basse fréquence}
On considère le circuit ci-contre. On pose $e(t) = E_m\cos(\wt)$ et $u(t) =
	U_m\cos(\wt+\f)$.

\begin{minipage}{0.60\linewidth}
	\begin{enumerate}
		\item Définir les signaux complexes $\ul{e}(t)$ et $\ul{u}(t)$ puis les
		      amplitudes complexes $\ul{E}$ et $\ul{U}$ associées aux tensions
		      $e(t)$ et $u(t)$, respectivement.
		\item Établir l'expression de $\ul{U}$ en fonction de $E_m$, $R$, $L$,
		      $C$ et $\w$.
		\item En déduire les expressions de $U_m$ et de $\f$ en fonction de
		      $E_m$, $R$, $L$, $C$ et $\w$.
		\item Déterminer les valeurs limites de $U_m$ à très basse et très haute
		      fréquence. Ces résultats étaient-ils prévisibles par une analyse
		      qualitative du montage~?
	\end{enumerate}
\end{minipage}
\hfill
\begin{minipage}{0.35\linewidth}
	\begin{center}
		\includegraphics[width=\linewidth]{exo4_plain}
	\end{center}
\end{minipage}

\section{Dipôle inconnu}

\begin{minipage}{0.60\linewidth}
	Dans le montage ci-contre, le GBF délivre une tension $e(t)$ sinusoïdale de
	pulsation $\w$, $R$ est une résistance et $D$ un dipôle inconnu. On note
	$u(t) = U_m\cos(\wt)$ et $v(t) = V_m\cos(\wt+\F)$ les tensions aux bornes
	respectivement de $R$ et $D$. On visualise à l'oscilloscope $v(t)$ et
	$u(t)$, et on obtient le graphe ci-dessous.
\end{minipage}
\begin{minipage}{0.35\linewidth}
	\begin{center}
		\includegraphics[width=\linewidth]{exo5_plain}
	\end{center}
\end{minipage}

\begin{center}
	\includegraphics[width=\linewidth]{exo5_plain-2}
\end{center}

L'unité de l'axe des temps est $\SI{e-2}{s}$, et celle de l'axe des tensions est
\SI{1}{V}. On utilise ces résultats graphiques pour déterminer les
caractéristiques de $D$, sachant que $R = \SI{100}{\Omega}$.

\begin{enumerate}
	\item Déterminer $V_m$, $U_m$ ainsi que la pulsation $\w$ des signaux
	      utilisés.
	\item La tension $v$ est-elle en avance ou en retard sur la tension $u$~? En
	      déduire le signe de $\F$. Déterminer la valeur de $\F$ à partir du
	      graphe.
	\item On note $\ul{Z} = X + \jj Y$ l'impédance complexe du dipôle $D$.
	      \begin{enumerate}
		      \item Déterminer les valeurs de $X$ et $Y$ à partir des résultats
		            précédents.
		      \item Par quel dipôle (condensateur, bobine, résistance) peut-on
		            modéliser $D$~?
	      \end{enumerate}
\end{enumerate}

\section{Obtention d'une équation différentielle}
\begin{minipage}{0.60\linewidth}
	En utilisant les complexes, montrer que la tension $u(t)$ est solution de
	l'équation différentielle
	\[4\tau^2 \dv[2]{u}{t} + R\tau \dv{u}{t} + u(t) = e(t)
		\qavec
		\tau = RC
	\]
\end{minipage}
\hfill
\begin{minipage}{0.35\linewidth}
	\begin{center}
		\includegraphics[width=\linewidth]{exo6_plain}
	\end{center}
\end{minipage}

\section{Déphasage, pulsation et impédance}
\begin{minipage}{0.55\linewidth}
	On considère le circuit en RSF. Déterminer l'expression de la pulsation $w$
	de la tension sinusoïdale $e(t) = E\cos(\wt)$ pour que le courant $i(t)$
	soit en phase avec $e(t)$. \bigbreak
	\textit{Indication}~: utiliser l'impédance équivalente constituée de $C$,
	$L$ et $R_2$.
\end{minipage}
\hfill
\begin{minipage}{0.40\linewidth}
	\begin{center}
		\includegraphics[width=\linewidth]{exo7_plain}
	\end{center}
\end{minipage}

\section{Oscillateur à quartz}
\begin{minipage}{0.60\linewidth}
	Un quartz piézo-électrique se modélise par un condensateur (de capacité
	$C_0$) placé en parallèle avec un condensateur (de capacité $C$) en série
	avec une inductance $L$. On se place en régime sinusoïdal forcé de pulsation
	$\w$.
\end{minipage}
\hfill
\begin{minipage}{0.35\linewidth}
	\begin{center}
		\includegraphics[width=\linewidth]{quartz_plain}
	\end{center}
\end{minipage}

\begin{enumerate}
	\item Donner l'impédance équivalente $\ul{Z}$ de l'oscillateur.
	\item Trouver la pulsation pour laquelle l'impédance de l'ensemble est
	      nulle, puis celle pour laquelle elle est infinie.
	\item Tracer l'allure de $|\ul{Z}(\w)|$.
	\item Comment la courbe précédente serait-elle modifiée si on prenant en
	      compte les résistances de chacun des composants~?
\end{enumerate}

\end{document}
