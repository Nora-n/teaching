\documentclass[a4paper, 12pt, final, garamond]{book}
\usepackage{cours-preambule}

\raggedbottom

\makeatletter
\renewcommand{\@chapapp}{Chimie -- chapitre}
\makeatother

\begin{document}
\setcounter{chapter}{1}

\chapter{TD~: Transformation et \'equilibre chimique}

\section{Transformations}
Identifier la nature des transformations suivantes~:\\

\begin{minipage}{0.40\linewidth}
	\begin{enumerate}
		\item $\ce{CH4 + 2O2 = CO2 + 2H2O}$
		\item $\ce{C\sol{} + OS\gaz{} = CO2\gaz{}}$
		\item $\ce{^{14}_{7}N + ^{1}_{0}n \rightarrow ^{14}_{6}C + ^1_1p}$
		\item $\ce{^14C + O2 \rightarrow 14^CO2}$
	\end{enumerate}
\end{minipage}
\hfill
\begin{minipage}{0.60\linewidth}
	\begin{enumerate}[start=5]
		\item $\ce{Fe\sol{} = Fe\liq{}}$
		\item $\ce{CH3COOH + CH3CH2OH = CH3COOCH2CH3 + H2O}$
		\item $\ce{Zn + Cu\plus{2} = Zn\plus{2} + Cu}$
		\item $\ce{CH3COOH + HO\moin{} = H2O + CH3COO\moin{}}$
	\end{enumerate}
\end{minipage}

\section{Transformations totales}
Compléter les tableaux suivants. Les gaz seront supposés parfaits. Dans la ligne
$t$, on demande d'exprimer la quantité de matière en fonction de l'avancement
molaire $\xi(t)$ à un instant $t$ quelconque.
\begin{enumerate}
	\item Réaction de l'oxydation du monoxyde d'azote en phase gazeuse, à $T =
		      \SI{25}{\degreeCelsius}$ dans un volume $V = \SI{10.0}{L}$~:
\end{enumerate}
\begin{center}
	%\renewcommand{\arraystretch}{1.3}
	\def\mystrut{\rule[-.5em]{0ex}{1.5em}}
	\centering
	\begin{tabularx}{\linewidth}{|!{\mystrut}p{3cm}||
		Y @{$+$} Y @{$\rightarrow$} Y || *2{p{2.5cm}|}}\hline
		Équation                      &
		$\ce{2NO\gaz{}} $             &
		$\ce{O2\gaz{}}$               &
		$\ce{2NO2\gaz{}}$             &
		$n_{\rm tot, gaz}$ (\si{mol}) &
		$ P_{\rm tot}$ (\si{bar})
	\end{tabularx}
	\par\vspace{-\lineskip}%
	\def\mystrut{\rule[-1em]{0ex}{2.5em}}
	%\renewcommand{\arraystretch}{4}
	\begin{tabularx}{\linewidth}{|!{\mystrut}p{3cm}|| *3{Y|} |*2{p{2.5cm}|}}\hline
		$t = 0$ (\si{mol}) &
		$\num{1.00} $      &
		$\num{2.00} $      &
		$\num{0.00} $      &
		$ $                &
		$ $                  \\
		\hline
		$t$ (\si{mol})     &
		$ $                &
		$ $                &
		$ $                &
		$ $                &
		$ $                  \\
		\hline
		État final (\si{mol})\smallbreak
		$\xi_f = ………$      &
		$ $                &
		$ $                &
		$ $                &
		$ $                &
		$ $                  \\
		\hline
	\end{tabularx}
\end{center}

\begin{enumerate}[resume]
	\item Réaction de combustion de l'éthanol dans l'air. Les réactifs sont
	      introduits dans les proportions stœchiométriques. Le dioxygène provient
	      de l'air, qui contient 20\% de $\ce{O2}$ et 80\% de $\ce{N2}$ en
	      fraction molaire.
\end{enumerate}
\begin{center}
	%\renewcommand{\arraystretch}{1.3}
	\def\mystrut{\rule[-.5em]{0ex}{1.5em}}
	\centering
	\begin{tabularx}{\linewidth}{|!{\mystrut}p{3cm}||
		Y @{$+$} Y @{$\rightarrow$} Y @{$+$} Y || *2{p{2.5cm}|}}\hline
		Équation                      &
		$\ce{C2H2OH\liq{}} $          &
		$\ce{3O2\gaz{}}$              &
		$\ce{2CO2\gaz{}}$             &
		$\ce{3H2O\gaz{}}$             &
		$n_{\mathrm{N_2}}$ (\si{mol}) &
		$n_{\rm tot, gaz}$ (\si{mol})
	\end{tabularx}
	\par\vspace{-\lineskip}%
	\def\mystrut{\rule[-1em]{0ex}{2.5em}}
	%\renewcommand{\arraystretch}{4}
	\begin{tabularx}{\linewidth}{|!{\mystrut}p{3cm}||
		*4{Y|} |*2{p{2.5cm}|}}\hline
		$t = 0$ (\si{mol}) &
		$\num{2.00} $      &
		$ $                &
		$ $                &
		$ $                &
		$ $                &
		$ $                  \\
		\hline
		$t$ (\si{mol})     &
		$ $                &
		$ $                &
		$ $                &
		$ $                &
		$ $                &
		$ $                  \\
		\hline
		État final (\si{mol})\smallbreak
		$\xi_f = ………$      &
		$ $                &
		$ $                &
		$ $                &
		$ $                &
		$ $                &
		$ $                  \\
		\hline
	\end{tabularx}
\end{center}

\section{Équilibre… ou pas~!}
La dissociation du peroxyde de baryum sert à l'obtention de dioxygène avant la
mise au point de la liquéfaction de l'air, selon l'équation

\centersright{$\ce{2BaO2\sol{} \rightleftharpoons{} 2BaO\sol{} + O2\gaz{}}$}
{$K(\SI{795}{\degreeCelsius}) = \num{0.50}$}

Le volume de l'enceinte, initialement vide de tout gaz, vaut $V = \SI{10}{L}$.
On rappelle que $R = \SI{8.314}{J.K^{-1}.mol^{-1}}$.

\begin{enumerate}
	\item
	      \begin{enumerate}
		      \item Exprimer la constante d'équilibre $K$ en fonction de la pression
		            partielle à l'équilibre $p_{\ce{O2}, \rm eq}$.
		      \item En déduire la valeur numérique de $p_{\ce{O2}, \rm eq}$.
		      \item Calculer le nombre de moles de dioxygène qui permet d'atteindre
		            cette pression dans l'enceinte.
	      \end{enumerate}
	\item Cas 1~:
\end{enumerate}
\begin{center}
	% \renewcommand{\arraystretch}{1.3}
	\def\mystrut{\rule[-.5em]{0ex}{1.5em}}
	\centering
	\begin{tabularx}{\linewidth}{|!{\mystrut}p{3cm}||
		Y @{$\rightleftharpoons$} Y @{$+$} Y || p{3cm}|}\hline
		Équation            &
		$\ce{2BaO2\sol{}} $ &
		$\ce{2BaO\sol{}}$   &
		$\ce{O2\gaz{}}$     &
		$n_{\rm tot, gaz}$ (\si{mol})
	\end{tabularx}
	\par\vspace{-\lineskip}%
	\def\mystrut{\rule[-1em]{0ex}{2.5em}}
	\begin{tabularx}{\linewidth}{|!{\mystrut}p{3cm}||
		*3{Y|} |M{3cm}|}\hline
		État initial (\si{mol}) &
		$\num{0.20} $           &
		$0 $                    &
		$0 $                    &
		$0 $                      \\
		\hline
		En cours (\si{mol})     &
		$ $                     &
		$ $                     &
		$ $                     &
		$ $                       \\
		\hline
		État final (\si{mol})   &
		$ $                     &
		$ $                     &
		$ $                     &
		$ $                       \\
		\hline
	\end{tabularx}
\end{center}
\begin{enumerate}[resume]
	\item[]
		\begin{enumerate}
			\item Calculer le quotient de réaction initial $Q_{r,i}$ et en déduire
			      le sens d'évolution du système.
			\item Remplir le tableau d'avancement et remplir la ligne $t$ dans le
			      tableau en fonction de $\xi(t)$.
			\item Déterminer $\xi_f$ en précisant si l'équilibre est atteint ou pas.
			      On rappelle que l'équilibre correspond à la coexistence de toutes
			      les espèces.
			\item Remplir la dernière ligne du tableau d'avancement.
		\end{enumerate}
	\item Mêmes questions dans le cas 2~:
\end{enumerate}
\begin{center}
	% \renewcommand{\arraystretch}{1.3}
	\def\mystrut{\rule[-.5em]{0ex}{1.5em}}
	\centering
	\begin{tabularx}{\linewidth}{|!{\mystrut}p{3cm}||
		Y @{$\rightleftharpoons$} Y @{$+$} Y || p{3cm}|}\hline
		Équation            &
		$\ce{2BaO2\sol{}} $ &
		$\ce{2BaO\sol{}}$   &
		$\ce{O2\gaz{}}$     &
		$n_{\rm tot, gaz}$ (\si{mol})
	\end{tabularx}
	\par\vspace{-\lineskip}%
	\def\mystrut{\rule[-1em]{0ex}{2.5em}}
	\begin{tabularx}{\linewidth}{|!{\mystrut}p{3cm}||
		*3{Y|} |M{3cm}|}\hline
		État initial (\si{mol}) &
		$\num{0.10} $           &
		$0 $                    &
		$0 $                    &
		$0 $                      \\
		\hline
		En cours (\si{mol})     &
		$ $                     &
		$ $                     &
		$ $                     &
		$ $                       \\
		\hline
		État final (\si{mol})   &
		$ $                     &
		$ $                     &
		$ $                     &
		$ $                       \\
		\hline
	\end{tabularx}
\end{center}
\begin{enumerate}[resume]
	\item Mêmes questions dans le cas 3~:
\end{enumerate}
\begin{center}
	% \renewcommand{\arraystretch}{1.3}
	\def\mystrut{\rule[-.5em]{0ex}{1.5em}}
	\centering
	\begin{tabularx}{\linewidth}{|!{\mystrut}p{3cm}||
		Y @{$\rightleftharpoons$} Y @{$+$} Y || p{3cm}|}\hline
		Équation            &
		$\ce{2BaO2\sol{}} $ &
		$\ce{2BaO\sol{}}$   &
		$\ce{O2\gaz{}}$     &
		$n_{\rm tot, gaz}$ (\si{mol})
	\end{tabularx}
	\par\vspace{-\lineskip}%
	\def\mystrut{\rule[-1em]{0ex}{2.5em}}
	\begin{tabularx}{\linewidth}{|!{\mystrut}p{3cm}||
		*3{Y|} |M{3cm}|}\hline
		État initial (\si{mol}) &
		$\num{0.10} $           &
		$\num{0.050} $          &
		$\num{0.10} $           &
		$\num{0.01} $             \\
		\hline
		En cours (\si{mol})     &
		$ $                     &
		$ $                     &
		$ $                     &
		$ $                       \\
		\hline
		État final (\si{mol})   &
		$ $                     &
		$ $                     &
		$ $                     &
		$ $                       \\
		\hline
	\end{tabularx}
\end{center}

\section{Combinaisons de réactions et constantes d'équilibre}
On considère les réactions numérotées $(1)$ et $(2)$ ci-dessous~:
\[\ce{4Cu\sol{} + O2\gaz{} = 2Cu2O\sol{}}\quad K\degree_1
	\qquad
	\ce{2Cu2O\sol{} + O2\gaz{} = 4CuO\sol{}}\quad K\degree_2\]
Exprimer les constantes d'équilibre des trois réactions ci-dessous en fonction
de $K\degree_1$ et $K\degree_2$~:
\[
	\ce{2Cu\sol{} + O2\gaz{} = 2CuO\sol{}}\quad
	\ce{8Cu\sol{} + 2O2\gaz{} = 3Cu2O\sol{}}\quad
	\ce{2CuO\sol{} = 4Cu\sol{}+ O2\gaz{}}
\]

\section{Équilibre avec des solides}

La chaux vive, solide blanc de formule $\ce{CaO\sol{}}$, est un des produits de
chimie industrielle les plus communs. Utilisée depuis l'Antiquité, notamment
dans le domaine de la construction, elle est aujourd'hui utilisée comme
intermédiaire en métallurgie. Elle est obtenue industriellement par dissociation
thermique du calcaire dans un four à $T = \SI{1100}{K}$. On modélise cette
transformation par la réaction d'équation~:

\centersright{$\ce{CaCO3\sol{} = CaO\sol{} + CO2 \gaz{}}$}{$K(\SI{1100}{K}) =
		\num{0.358}$}

\begin{enumerate}
	\item Dans un récipient de volume $V = \SI{10}{L}$ préalablement vide, on
	      introduit \SI{10}{mmol} de calcaire à température constante $T =
		      \SI{1100}{K}$. Déterminer le sens d'évolution du système chimique.
	\item Supposons que l'état final est un état d'équilibre. Déterminer la
	      quantité de matière de calcaire qui devrait avoir réagi. Conclure sur
	      l'hypothèse faite.
	\item Quelle quantité maximale de calcaire peut-on transformer en chaux dans
	      ces conditions~?
\end{enumerate}

\section{Équilibre en solution aqueuse}
Considérons un système de volume \SI{20}{mL} évoluant selon la réaction
d'équation bilan~:

\centersright{$\ce{CH3COOH\aqu{} + F\moin{}\aqu{} \rightleftarrows CH3CHH\moin{}\aqu{} +
			HF\aqu{}}$}{$K(\SI{25}{\degreeCelsius}) = \num{e-1.60}$}

Déterminer le sens d'évolution du système et l'avancement à l'équilibre en
partant des deux situations initiales suivantes~:

\begin{enumerate}
	\item $[\ce{CH3COOH}]_0 = [\ce{F\moin{}}]_0 = c = \SI{0.1}{mol.L^{-1}}$ et
	      $[\ce{CH3COO\moin{}}]_0 = [\ce{HF}]_0 = 0$
	\item $[\ce{CH3COOH}]_0 = [\ce{F\moin{}}]_0 = [\ce{CH3COO\moin{}}]_0 =
		      [\ce{HF}]_0 = 0 = c = \SI{0.1}{mol.L^{-1}}$
\end{enumerate}

\section{Équilibre en phase gazeuse}

On étudie en phase gazeuse l'équilibre de dimérisation de \ce{FeCl3}, de constante
d'équilibre $K\degree(T)$ à une température $T$ donnée et d'équation-bilan
\[\ce{2FeCl3\gaz{} = Fe2Cl6\gaz{}}\]

La réaction se déroule sous une pression totale constante $p_{\rm tot} =
	2p\degree = \SI{2}{bars}$. À la température $T_1 = \SI{750}{K}$, la constante
d'équilibre vaut $K\degree(T_1) = \num{20.8}$. Le système est maintenu à la
température $T_1 = \SI{750}{K}$. Initialement le système contient $n_0$ moles de
\ce{FeCl3} et de \ce{Fe2Cl6}. Soit $n_{\rm tot}$ la quantité totale de matière
d'espèces dans le système.

\begin{enumerate}
	\item Exprimer la constante d'équilibre en fonction des pressions partielles
	      des constituants à l'équilibre et de $p\degree$.

	\item Exprimer le quotient de réaction $Q_r$ en fonction de la quantité de
	      matière de chacun des constituants, de la pression totale $p_{\rm tot}$
	      et de $p\degree$. Calculer la valeur initial $Q_{r,0}$ du quotient de
	      réaction.

	\item Le système est-il initialement à l'équilibre thermodynamique~?
	      Justifier la réponse. Si le système n'est pas à l'équilibre, dans quel
	      sens se produira l'évolution~?

	\item On considère désormais une enceinte indéformable, de température
	      constante $T_1 = \SI{750}{K}$, initialement vide. On y introduit une
	      quantité $n$ de \ce{FeCl3} gazeux et on laisse le système évoluer de
	      telle sorte que la pression soit maintenu constante et égale à $p =
		      2p\degree = \SI{2}{bars}$. On désigne par $\xi$ l'avancement de la
	      réaction. Calculer à l'équilibre la valeur du rapport $\xi/n$.
\end{enumerate}

\section{Transformations de gaz}
\begin{enumerate}
	\item On considère l'équilibre suivant~:
\end{enumerate}

\centersright{$\DS\ce{H2S\gaz{} + \frac{3}{2}O2\gaz{} = H2O\gaz{} + SO2\gaz{}}$}
{$K\degree_1$}

\begin{enumerate}[resume]
	\item[] Donner l'expression de la constante d'équilibre $K\degree_1$. En
		supposant les réactifs introduits dans les proportions stœchiométriques,
		faire un bilan de matière à l'équilibre. Exprimer $K\degree_1$ en
		fonction de $\xi_{\rm eq}$.
	\item On considère l'équilibre suivant~:
\end{enumerate}

\centersright{$\DS\ce{2H2S\gaz{} + SO2\gaz{} = 2H2O\gaz{} + 3S\liq{}}$}
{$K\degree_2$}

\begin{enumerate}[resume]
	\item[] Donner l'expression de la constante d'équilibre $K\degree_2$. On
		introduit les réactifs avec des quantités quelconques. Faire un bilan de
		matière à l'équilibre. Exprimer $K\degree_2$ en fonction de $\xi_{\rm
				eq}$.
	\item On fait brûler du méthane dans de l'oxygène~:
\end{enumerate}
\[\ce{\ldots CH4\gaz{} + \ldots O2\gaz{} \rightarrow \ldots CO2\gaz{} + \ldots H2O\liq{}}\]
\begin{enumerate}[resume]
	\item[] Équilibrer l'équation de la réaction. Elle peut être considérée
		comme totale. On introduit les réactifs de façon à consommer la moitié
		du dioxygène. Décrire l'état final du système.
\end{enumerate}

\section{Coefficient de dissociation} On considère l'équilibre de l'eau en phase
gazeuse~: \[\ce{2H2O = 2H2\gaz{} + O2\gaz{}}\]
\begin{enumerate}
	\item On se place à \SI{400}{K} sous une pression constante $P =
		      \SI{1.00}{bar}$. Sous quelle forme se trouve l'eau~?
	\item La valeur de la constante vaut $K(\SI{400}{K}) = \num{3.12e-59}$.
	      Conclure sur la stabilité de l'eau dans ces conditions.
	\item On supposant que l'on introduit de l'eau pure, calculer le coefficient
	      de dissociation de l'eau.\\
	      \textit{Rappel}~: le coefficient de dissociation $\alpha$ est
	      égal au rapport de la quantité ayant été dissociée sur la quantité
	      initiale.
	\item A 3000 K, toujours sous une pression de 1 bar, le coefficient de
	      dissociation vaut $\alpha = \num{0.30}$ Calculer $K(\SI{3000}{K})$.
	      Conclure sur la stabilité de l'eau dans ces conditions.
\end{enumerate}

\section{Ions mercure}
Les ions mercure (II) $\ce{Hg\plus{2}}$ peuvent réagir avec le métal liquide
(insoluble dans l'eau) mercure \ce{Hg} pour donner les ions mercure (I)
$\ce{Hg2\plus{2}}$ selon l'équilibre chimique ci-dessous~:

\centersright{$\ce{Hg\plus{2}\aqu{} + Hg\liq{} = Hg2\plus{2}\aqu{}}$}
{$K\degree(\SI{25}{\degreeCelsius}) = 91$}

\begin{enumerate}
	\item Dans quel sens évolue un système obtenu en mélangeant du mercure
	      liquide en large excès avec $V_1 = \SI{40.0}{mL}$ d'une solution de
	      chlorure de mercure (I) à $c_1 = \SI{1.0e-3}{mol.L^{-1}}$ et $V_2 =
		      \SI{10.0}{mL}$ d'une solution de chlorure de mercure (II) à $c_2 =
		      \SI{2.0e-3}{mol.L^{-1}}$~?
	\item Déterminer la composition finale de la solution.
\end{enumerate}

\end{document}
