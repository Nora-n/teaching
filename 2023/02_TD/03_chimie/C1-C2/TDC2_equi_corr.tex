\documentclass[a4paper, 12pt, final, garamond]{book}
\usepackage{cours-preambule}

\raggedbottom

\makeatletter
\renewcommand{\@chapapp}{Chimie -- chapitre}
\makeatother

\begin{document}
\setcounter{chapter}{1}

\chapter{Correction du TD}

\section{Transformations}

\begin{minipage}{0.40\linewidth}
    \begin{enumerate}
        \item Chimique
        \item Chimique
        \item Nucléaire
        \item Chimique
    \end{enumerate}
\end{minipage}
\hfill
\begin{minipage}{0.60\linewidth}
    \begin{enumerate}[start=5]
        \item Physique
        \item Chimique
        \item Chimique
        \item Chimique
    \end{enumerate}
\end{minipage}

\section{Transformations totales}
\begin{enumerate}
    \item Pour la quantité totale de gaz, il suffit de sommer les quantité de
        matière de chacun des gaz~: ici, initialement on a
        $\num{1.00}+\num{2.00} = \SI{3.00}{mol}$ de gaz. Ensuite, pour la
        pression totale on utilise l'équation d'état des gaz parfaits~:
        \begin{NCrapp}[]{Rappel~: gaz parfait}
            \begin{gather*}
                \boxed{pV = nRT}
                \qavec
                \left\{
                    \begin{array}{ll}
                        p & \text{en Pa }\\
                        V & \text{en m}^3\\
                        n & \text{en mol}\\
                        T & \text{en \textbf{Kelvin} (K)}
                    \end{array}
                \right.\\
                \text{et }
                \boxed{R = \SI{8.314}{J.mol^{-1}.K^{-1}}}\\
                \text{est la constante des gaz parfaits}
            \end{gather*}
        \end{NCrapp}
        Il faut donc convertir le volume en $\si{m^3}$. Pour cela, il suffit
        d'écrire
        \[\SI{10.0}{L} = \SI{10.0}{dm^3} = \num{10.0}(\SI{e-1}{m})^3 =
        \SI{10.0e-3}{m^3}\]
        Il est très courant d'oublier les puissances sur les conversions du
        genre~: n'oubliez pas les parenthèses. Il nous faut de plus convertir la
        température en Kelvins, attention à ne pas vous tromper de sens~: il
        faut ici \textbf{ajouter} \SI{273.15}{K} à la température en degrés
        Celsius, ce qui donne $T = \SI{298.15}{K}$. On peut donc faire
        l'application numérique pour $P_{\tot}$ initial.\bigbreak

        On remplit la deuxième ligne du tableau avec les coefficients
        stœchiométriques algébriques des constituants en facteur de chaque
        $\xi$, et on somme les quantités de matière de gaz pour $n_{\rm tot,
        gaz}$. En réalité, il est plus simple de partir de la valeur totale de
        la première ligne et de compter algébriquement le nombre de $\xi$~: on
        en perd 3 avec les réactifs pour en gagner 2 avec les produits, donc en
        tout la quantité de matière totale de gaz décroit de 1$\xi$. On ne peut
        pas calculer précisément la valeur de $P_{\tot}$ ici, il faudrait
        l'exprimer en fonction de $\xi$ (ça viendra dans d'autres
        exercices).\bigbreak

        Enfin, pour trouver le réactif limitant, on résout~:
        \begin{gather*}
            \left\{
                \begin{array}{rcl}
                    \num{1.00}-2\xi_f & = & 0\\
                    \num{2.00}-\xi_f & = & 0
                \end{array}
            \right.
            \Longleftrightarrow
            \left\{
                \begin{array}{rcl}
                    \xi_f & = & \num{0.50}\\
                    \xi_f & = & \num{1.00}
                \end{array}
            \right.
        \end{gather*}
        La seule valeur possible est la plus petite, $\xi_f = \SI{0.50}{mol}$~:
        si on prenait \SI{1.00}{mol} on trouverait une quantité négative de
        \ce{NO} à l'état final, ce qui, vous en conviendrez, est une absurdité.
        Même travail qu'initialement pour $n_{\rm tot, gaz}$ et $P_{\tot}$.
        D'où le tableau~:
\end{enumerate}
\begin{center}
    %\renewcommand{\arraystretch}{1.3}
    \def\mystrut{\rule[-.5em]{0ex}{1.5em}}
    \centering
    \begin{tabularx}{\linewidth}{|!{\mystrut}p{3cm}||
        Y @{$+$} Y @{$\rightarrow$} Y || *2{p{2.5cm}|}}\hline
        Équation                      &
        $\ce{2NO\gaz{}} $             &
        $\ce{O2\gaz{}}$               &
        $\ce{2NO2\gaz{}}$             &
        $n_{\rm tot, gaz}$ (\si{mol}) &
        $ P_{\tot}$ (\si{bar})
    \end{tabularx}
    \par\vspace{-\lineskip}%
    \def\mystrut{\rule[-1em]{0ex}{2.5em}}
    %\renewcommand{\arraystretch}{4}
    \begin{tabularx}{\linewidth}{|!{\mystrut}p{3cm}|| *3{Y|} |*2{M{2.5cm}|}}\hline
        $t = 0$ (\si{mol}) &
        $\num{1.00} $ &
        $\num{2.00} $ &
        $\num{0.00} $ &
        $\num{3.00} $ &
        $\num{7.40} $\\
        \hline
        $t$ (\si{mol})     &
        $\num{1.00}-2\xi $ &
        $\num{2.00}-\xi $  &
        $2\xi $            &
        $\num{3.00}-\xi $  &
        ---\\
        \hline
        État final (\si{mol})\smallbreak
        $\xi_f = \num{0.5}$ &
        $\num{0.00}$ &
        $\num{1.50}$ &
        $\num{1.00}$ &
        $\num{2.50}$ &
        $\num{6.20}$\\
        \hline
    \end{tabularx}
\end{center}

\begin{enumerate}[resume]
    \item Pour une réaction $a{\rm A} + b{\rm B} = c{\rm C} + d{\rm D}$, le fait
        que les réactifs soient introduits dans les proportions stœchiométriques
        se traduit par
        \[ \frac{n_{\rm A}^0}{a} = \frac{n_{\rm B}^0}{b}
            \Longleftrightarrow
            n_{\rm B}^0 = \frac{b}{a}n_{\rm A}^0
        \]
        Ici, on a donc $n_{\ce{O2}}^0 = 3n_{\ce{C2H2OH}}^0$, c'est-à-dire
        $n_{\ce{O2}}^0 = \SI{6.00}{mol}$. On peut donc remplir cette
        case.\bigbreak

        On suppose qu'on commence sans $\ce{CO2}$ ou $\ce{H2O}$ initialement,
        puisque rien n'est indiqué~; en revanche, on sait qu'il y a déjà du
        diazote dans le milieu puis que le dioxygène vient de l'air, comme c'est
        indiqué. Comme il y a 80\% de \ce{N2} pour 20\% de \ce{O2}, cela veut
        dire qu'il y a 4 fois plus de diazote que de dioxygène, donc
        \SI{24.00}{mol}. Ici, la colonne $n_{\rm tot, gaz}$ n'a pas grande
        utilité puisqu'il n'y a qu'un gaz, mais c'est une bonne pratique à ne
        pas oublier.\bigbreak

        Le reste du remplissage est le même que pour la question 1. On trouve
        évidemment $\xi_f = \SI{2.00}{mol}$ avec les deux réactifs limitant,
        c'est le principe des proportions stœchiométriques.
\end{enumerate}
\begin{center}
    %\renewcommand{\arraystretch}{1.3}
    \def\mystrut{\rule[-.5em]{0ex}{1.5em}}
    \centering
    \begin{tabularx}{\linewidth}{|!{\mystrut}p{3cm}||
        Y @{$+$} Y @{$\rightarrow$} Y @{$+$} Y || *2{p{2.5cm}|}}\hline
        Équation                 &
        $\ce{C2H2OH\liq{}} $     &
        $\ce{3O2\gaz{}}$         &
        $\ce{2CO2\gaz{}}$        &
        $\ce{3H2O\gaz{}}$         &
        $n_{\mathrm{N_2}}$ (\si{mol}) &
        $n_{\rm tot, gaz}$ (\si{mol})
    \end{tabularx}
    \par\vspace{-\lineskip}%
    \def\mystrut{\rule[-1em]{0ex}{2.5em}}
    %\renewcommand{\arraystretch}{4}
    \begin{tabularx}{\linewidth}{|!{\mystrut}p{3cm}||
        *4{Y|} |*2{M{2.5cm}|}}\hline
        $t = 0$ (\si{mol}) &
        $\num{2.00} $ &
        $\num{6.00} $ &
        $\num{0.00} $ &
        $\num{0.00} $ &
        $\num{24.00} $ &
        $\num{30.00} $\\
        \hline
        $t$ (\si{mol}) &
        $\num{2.00} -\xi $ &
        $\num{6.00} -3\xi $ &
        $2\xi $ &
        $3\xi $ &
        $\num{24.00}$ &
        $\num{30.00}+2\xi$\\
        \hline
        État final (\si{mol})\smallbreak
        $\xi_f = \num{2.00}$ &
        $\num{0.00}$ &
        $\num{0.00}$ &
        $\num{4.00}$ &
        $\num{6.00}$ &
        $\num{24.00}$ &
        $\num{34.00}$\\
        \hline
    \end{tabularx}
\end{center}

\section{Équilibre… ou pas~!}
\begin{enumerate}
    \item 
        \begin{enumerate}
            \item Par définition, $K = Q_{r, \eq}$. On exprime donc le
                quotient de réaction avec les activités à l'équilibre~:
                \[ K = \frac{p_{\ce{O2,\eq}}}{p\degree}\]
            \item On a la valeur de $K$ et la valeur de $p\degree$~: de
                l'équation précédente on isole $p_{\ce{O2}, \eq}$~:
                \begin{gather*}
                    p_{\ce{O2},\eq} = Kp\degree
                    \qavec
                    \left\{
                        \begin{array}{rcl}
                            K & = & \num{0.50}\\
                            p\degree & = & \SI{1.00}{bar}
                        \end{array}
                    \right.\\
                    \mathrm{A.N.~:}\quad
                    \boxed{p_{\ce{O2},\eq} = \SI{0.50}{bar} = \SI{5.0e4}{Pa}}
                \end{gather*}
            \item Avec la loi des gaz parfaits, on a
                \begin{gather*}
                    p_{\ce{O2},\eq}V = n_{\ce{O2}, \eq}RT
                    \Leftrightarrow
                    n_{\ce{O2}, \eq} = \frac{p_{\ce{O2},\eq}V}{RT}
                    \qavec
                    \left\{
                        \begin{array}{rcl}
                            V & = & \SI{10e-3}{m^3}\\
                            T & = & \SI{1068.15}{K}
                        \end{array}
                    \right.\\
                    \mathrm{A.N.~:}\quad
                    \boxed{n_{\ce{O2}, \eq} = \SI{0.056}{mol}}
                \end{gather*}
        \end{enumerate}
    \item Cas 1~:
    \begin{enumerate}
        \item On change juste $p_{\ce{O2},\eq}$ de la première question en
            $p_{\ce{O2}, 0}$~; sachant qu'on commence sans gaz dans l'enceinte,
            cette pression est nulle~:
            \[\boxed{Q_{r,i} = \frac{p_{\ce{O2}, 0}}{p\degree} = 0}\]
            On a donc $Q_{r,i} < K$, et l'évolution se fait en sens direct.
        \item Voir tableau.

        \item Pour trouver l'état final dans cette situation, \textbf{on
                détermine $\xi_{\eq}$ s'il y avait équilibre, et on regarde
                si c'est compatible avec $\xi_{\max}$ si la réaction était
            totale}.\bigbreak

            S'il y a équilibre, ça veut dire que $n_{\ce{O2}, \eq} =
            \SI{0.056}{mol}$ comme déterminé au début. Or, le tableau nous
            indique que $n_{\ce{O2}, f} = \xi_f$, donc si c'est un équilibre
            \fbox{$\xi_{\eq} = \SI{0.056}{mol}$}.\bigbreak

            L'avancement est maximal si \ce{BaO2} est limitant~: on trouve donc
            $\xi_{\max}$ en résolvant $\num{0.20} - 2\xi_{\max} = 0$,
            c'est-à-dire \fbox{$\xi_{\max} = \SI{0.1}{mol}$}.\bigbreak

            La valeur est finale $\xi_f$ est la plus petite valeur (en valeur
            absolue) de $\xi_{\eq}$ et $\xi_{\max}$~; or ici $\xi_{\eq} <
            \xi_{\max}$~: il y a donc bien équilibre, et on a
            \[\boxed{\xi_f = \xi_{\eq} = \SI{0.056}{mol}}\]

        \item Voir tableau.
    \end{enumerate}
\end{enumerate}
\begin{center}
    % \renewcommand{\arraystretch}{1.3}
    \def\mystrut{\rule[-.5em]{0ex}{1.5em}}
    \centering
    \begin{tabularx}{\linewidth}{|!{\mystrut}p{3cm}||
        Y @{$\rightleftharpoons$} Y @{$+$} Y || p{3cm}|}\hline
        Équation           &
        $\ce{2BaO2\sol{}} $ &
        $\ce{2BaO\sol{}}$  &
        $\ce{O2\gaz{}}$    &
        $n_{\rm tot, gaz}$ (\si{mol})
    \end{tabularx}
    \par\vspace{-\lineskip}%
    \def\mystrut{\rule[-1em]{0ex}{2.5em}}
    \begin{tabularx}{\linewidth}{|!{\mystrut}p{3cm}||
        *3{Y|} |M{3cm}|}\hline
        État initial (\si{mol}) &
        $\num{0.20} $&
        $\num{0.00} $&
        $\num{0.00} $&
        $\num{0.00} $\\
        \hline
        En cours (\si{mol}) &
        $\num{0.20}-2\xi(t)$&
        $2\xi(t)$&
        $\xi(t)$&
        $\xi(t)$\\
        \hline
        État final (\si{mol}) &
        $\num{0.088}$&
        $\num{0.112}$&
        $\num{0.056}$&
        $\num{0.056}$\\
        \hline
    \end{tabularx}
\end{center}
\begin{enumerate}[resume]
    \item Cas 2~:
    \begin{enumerate}
        \item On a toujours aucun gaz au départ, donc ici aussi 
            \[\boxed{Q_{r,i} = \frac{p_{\ce{O2}, 0}}{p\degree} = 0}\]
            et la réaction est en sens direct.
        \item Voir tableau.
        \item Même procédé~: \textbf{on détermine $\xi_{\eq}$ s'il y avait
                équilibre, et on regarde si c'est compatible avec $\xi_{\max}$
            si la réaction était totale}.\bigbreak

            S'il y a équilibre, ça veut dire que $n_{\ce{O2}, \eq} =
            \SI{0.056}{mol}$ comme déterminé au début. Or, le tableau nous
            indique que $n_{\ce{O2}, f} = \xi_f$, donc si c'est un équilibre
            \fbox{$\xi_{\eq} = \SI{0.056}{mol}$}.\bigbreak

            L'avancement est maximal si \ce{BaO2} est limitant~: on trouve donc
            $\xi_{\max}$ en résolvant $\num{0.10} - 2\xi_{\max} = 0$,
            c'est-à-dire \fbox{$\xi_{\max} = \SI{0.050}{mol}$}.\bigbreak

            La valeur est finale $\xi_f$ est la plus petite valeur (en valeur
            absolue) de $\xi_{\eq}$ et $\xi_{\max}$~; or ici $\xi_{\eq} >
            \xi_{\max}$~: il n'y a donc \textbf{pas équilibre}, et on a
            \[\boxed{\xi_f = \xi_{\max} = \SI{0.050}{mol}}\]

        \item Voir tableau.
    \end{enumerate}
\end{enumerate}
\begin{center}
    % \renewcommand{\arraystretch}{1.3}
    \def\mystrut{\rule[-.5em]{0ex}{1.5em}}
    \centering
    \begin{tabularx}{\linewidth}{|!{\mystrut}p{3cm}||
        Y @{$\rightleftharpoons$} Y @{$+$} Y || p{3cm}|}\hline
        Équation            &
        $\ce{2BaO2\sol{}} $ &
        $\ce{2BaO\sol{}}$   &
        $\ce{O2\gaz{}}$     &
        $n_{\rm tot, gaz}$ (\si{mol})
    \end{tabularx}
    \par\vspace{-\lineskip}%
    \def\mystrut{\rule[-1em]{0ex}{2.5em}}
    \begin{tabularx}{\linewidth}{|!{\mystrut}p{3cm}||
        *3{Y|} |M{3cm}|}\hline
        État initial (\si{mol}) &
        $\num{0.10} $&
        $\num{0.00} $&
        $\num{0.00} $&
        $\num{0.00} $\\
        \hline
        En cours (\si{mol})  &
        $\num{0.10}-2\xi(t)$ &
        $2\xi(t)$            &
        $\xi(t)$             &
        $\xi(t)$\\
        \hline
        État final (\si{mol}) &
        $\num{0.00}$&
        $\num{0.10}$&
        $\num{0.05}$&
        $\num{0.05}$\\
        \hline
    \end{tabularx}
\end{center}

\begin{enumerate}[resume]
    \item Cas 3~:
        \begin{enumerate}
            \item On a cette fois du gaz au départ, donc ici
                \begin{gather*}
                    Q_{r,i}
                    = \frac{p_{\ce{O2}, 0}}{p\degree}
                    = \frac{n_{\ce{O2}, 0}RT}{Vp\degree}
                    \qavec
                    \left\{
                        \begin{array}{rcl}
                            n_{\ce{O2}, 0} & = & \SI{0.10}{mol}\\
                            T & = & \SI{1069.15}{K}\\
                            V & = & \SI{10e-3}{m^3}
                        \end{array}
                    \right.\\
                    \mathrm{A.N.~:}\quad
                    \boxed{Q_{r,i} = \num{0.89}}
                \end{gather*}
                Cette fois, $Q_{r,i} > K$ donc la réaction se fait en sens
                indirect.
            \item Voir tableau.
        \end{enumerate}
\end{enumerate}
\begin{NCror}[width=\textwidth]{Important}

    Le procédé de remplissage du tableau \textbf{ne doit pas changer} même si la
    réaction se fait dans le sens indirect~: les coefficients stœchiométriques
    de la réaction n'ont pas changé, donc les facteurs devant des $\xi(t)$ non
    plus. Certes, on aura $\xi < 0$ mais il est plus naturel et moins perturbant
    de garder la forme de base du remplissage du tableau plutôt que de s'embêter
    à repenser l'écriture du tableau.

\end{NCror}
\begin{enumerate}[resume]
    \item[]
        \begin{enumerate}[start=3]

            \item Même procédé~: \textbf{on détermine $\xi_{\eq}$ s'il y
                    avait équilibre, et on regarde si c'est compatible avec
                $\xi_{\max}$ si la réaction était totale}.\bigbreak

                S'il y a équilibre, ça veut dire que $n_{\ce{O2}, \eq} =
                \SI{0.056}{mol}$ comme déterminé au début. Or, le tableau nous
                indique que $n_{\ce{O2}, f} = \num{0.10} + \xi_f$, donc si c'est
                un équilibre \fbox{$\xi_{\eq} = -\SI{0.044}{mol}$}.\bigbreak

                L'avancement est maximal si \ce{BaO} ou \ce{O2} sont limitant~:
                on résout donc
                \begin{gather*}
                    \left\{
                        \begin{array}{rcl}
                            \num{0.05}+2\xi_{\max} & = & 0\\
                            \num{0.10}+\xi_{\max} & = & 0
                        \end{array}
                    \right.
                    \Longleftrightarrow
                    \left\{
                        \begin{array}{rcl}
                            \xi_{\max} & = & -\num{0.025}\\
                            \xi_{\max} & = & -\num{0.050}
                        \end{array}
                    \right.
                \end{gather*}

                Le seul $\xi_{\max}$ possible est le plus petit \textbf{en
                valeur absolue}, c'est-à-dire \fbox{$\xi_{\max} =
                -\SI{0.025}{mol}$}.\bigbreak

                La valeur est finale $\xi_f$ est la plus petite valeur
                \textbf{en valeur absolue} de $\xi_{\eq}$ et $\xi_{\max}$~;
                or ici $\left|\xi_{\eq}\right| > \left|\xi_{\max}\right|$~:
                il n'y a donc \textbf{pas équilibre}, et on a

            \[\boxed{\xi_f = \xi_{\max} = -\SI{0.025}{mol}}\]

        \item Voir tableau.
    \end{enumerate}
\end{enumerate}
\begin{center}
    % \renewcommand{\arraystretch}{1.3}
    \def\mystrut{\rule[-.5em]{0ex}{1.5em}}
    \centering
    \begin{tabularx}{\linewidth}{|!{\mystrut}p{3cm}||
        Y @{$\rightleftharpoons$} Y @{$+$} Y || p{3cm}|}\hline
        Équation            &
        $\ce{2BaO2\sol{}} $ &
        $\ce{2BaO\sol{}}$   &
        $\ce{O2\gaz{}}$     &
        $n_{\rm tot, gaz}$ (\si{mol})
    \end{tabularx}
    \par\vspace{-\lineskip}%
    \def\mystrut{\rule[-1em]{0ex}{2.5em}}
    \begin{tabularx}{\linewidth}{|!{\mystrut}p{3cm}||
        *3{Y|} |M{3cm}|}\hline
        État initial (\si{mol}) &
        $\num{0.10} $&
        $\num{0.05} $&
        $\num{0.10} $&
        $\num{0.10} $\\
        \hline
        En cours (\si{mol})  &
        $\num{0.10}-2\xi(t)$ &
        $\num{0.05}+2\xi(t)$ &
        $\num{0.10}+\xi(t)$  &
        $\num{0.10}+\xi(t)$\\
        \hline
        État final (\si{mol}) &
        $\num{0.15}$&
        $\num{0.00}$&
        $\num{0.075}$&
        $\num{0.075}$\\
        \hline
    \end{tabularx}
\end{center}

\section{Combinaisons de réactions et constantes d'équilibre}
Dans cet exercice, on introduit le lien entre relation sur les équations-bilan
et les constantes d'équilibre associées. En effet, on a vu dans le cours que
\[a{\rm A} + b{\rm B} = c{\rm C} + d{\rm D}\]
a pour constante d'équilibre
\[K_1 = \prod_i a({\rm X}_i)^{\nu_{i, 1}}\]
Si on inverse la réaction pour avoir
\[c{\rm C} + d{\rm D} = a{\rm A} + b{\rm B}\]
alors on prend l'opposé de chaque coefficient stœchiométrique~: $\nu_{i, 2} = -
\nu_{i,1}$, ce qui fait que
cette réaction a pour constante d'équilibre
\[K_2
    = \prod_i a({\rm X}_i)^{\nu_{i,2}}
    = \prod_i a({\rm X}_i)^{-\nu_{i,1}}
    = \left(\prod_i a({\rm X}_i)^{\nu_{i,1}}\right)^{-1}
    = \left( K_1 \right)^{-1}
\]

Le même raisonnement tient pour montrer que
\[2a{\rm A} + 2b{\rm B} = 2c{\rm C} + 2d{\rm D}\]
a pour constante d'équilibre
\[K_3 = K_1{}^2\]

On étend le raisonnement pour montrer que si on ajoute deux réactions (1) et (2)
pour avoir une équation (3), alors on aura $K_3 = K_1\times K_2$, et que si on a
$(3) = \alpha(1) + \beta(2)$, alors $K_3 = K_1{}^{\alpha}\times
K_2{}^{\beta}$.\bigbreak

Ainsi, dans cet exercice il suffit de trouver les relations entre les équations
(3), (4), (5) et les équations (1) et (2) de constantes respectives $K_1\degree$
et $K_2\degree$. On trouve alors~:

\begin{enumerate}
    \item \[
            (3)
            = \frac{(1)+(2)}{2}
            \Longrightarrow
            K_3\degree
            = \left(K_1\degree\times K_2\degree\right)^{1/2}
            = \sqrt{K_1\degree K_2\degree}
        \]
    \item \[
            (4) = 2(1)
            \Longrightarrow
            K_4\degree
            = \left( K_1\degree \right)^2
        \]
    \item \[
            (5)
            = -(1)
            \Longrightarrow
            K_5\degree
            = \left( K_1\degree \right)^{-1}
        \]
\end{enumerate}

Tout ceci se vérifie bien sûr en écrivant les constantes de chacune des
réactions~:

\[
    K_1\degree = \frac{p\degree}{p_{\ce{O2}}}
    \qquad
    K_2\degree = \frac{p\degree}{p_{\ce{O2}}}
    \qquad
    K_3\degree = \frac{p\degree}{p_{\ce{O2}}}
    \qquad
    K_4\degree = \frac{p\degree^2}{p_{\ce{O2}}{}^2}
    \qquad
    K_5\degree = \frac{p_{\ce{O2}}}{p\degree}
\]

\section{Équilibre avec des solides}
\begin{enumerate}
    \item Comme on ne part que de calcaire, la réaction \textbf{ne peut avoir
        lieu que dans le sens direct}. On vérifie cette intuition en calculant
        $Q_{r,i}$ pour le comparer à $K$, sachant qu'on part d'un récipient vide
        de gaz au début~:
        \[\boxed{Q_{r,i} = \frac{p_{\ce{CO2}, 0}}{p\degree} = 0 < K}\]
        La réaction se fait bien dans le sens direct.
    \item Si l'état final est un état d'équilibre, alors avec l'équation
        précédente on aura
        \begin{gather*}
            p_{\ce{CO2}, \eq}
                = Kp\degree\\
            \Longleftrightarrow
            n_{\ce{O2}, \eq}
                = \frac{p_{\ce{O2},\eq}V}{RT}
                = \frac{Kp\degree V}{RT}
        \end{gather*}
        Or, un tableau d'avancement donne que $n_{\ce{O2}, \eq} = \xi_{\rm
        eq}$~; on trouve donc
        \begin{gather*}
            \xi_{\eq} = \frac{Kp\degree V}{RT}
            \qavec
            \left\{
                \begin{array}{rcl}
                    K & = & \num{0.358}\\
                    V & = & \SI{10e-3}{m^3}\\
                    T & = & \SI{1100}{K}\\
                    p\degree & = & \SI{1}{bar}\\
                    R & = & \SI{8.314}{J.mol.K^{-1}}
                \end{array}
            \right.\\
            \mathrm{A.N.~:}\quad
            \boxed{\xi_{\eq} = \SI{39}{mmol}}
        \end{gather*}
        Pour savoir si cette valeur est réalisable, on calcule $\xi_{\max}$ que
        l'on trouverait si le calcaire était limitant, c'est-à-dire en résolvant
        $\num{10} - \xi_{\max} = 0$~: on trouve naturellement \fbox{$\xi_{\max}
        = \SI{10}{mmol}$}.\bigbreak

        On sait que la valeur de $\xi_f$ est la plus petite valeur absolue entre
        $\xi_{\eq}$ et $\xi_{\max}$. Or, ici on trouve $\xi_{\eq} >
        \xi_{\max}$, ce qui veut dire que \textbf{l'équilibre ne peut être
        atteint} et qu'on aura ainsi
        \[\boxed{\xi_f = \xi_{\max} = \SI{10}{mmol}}\]
        c'est-à-dire que \textbf{la réaction est totale}. On peut donc remplir
        la dernière ligne du tableau d'avancement.
\end{enumerate}
\begin{center}
    \renewcommand{\arraystretch}{1.3}
    \centering
    \begin{tabularx}{\linewidth}{|p{3.5cm}||
        Y @{$=$} Y @{$+$} Y || M{3cm} |}\hline
        Équation            &
        $\ce{CaCO3\sol{}} $ &
        $\ce{CaO\sol{}}$    &
        $\ce{CO2\gaz{}}$    &
        $n_{\rm tot, gaz}$ (\si{mmol})
    \end{tabularx}
    \par\vspace{-\lineskip}%
    \begin{tabularx}{\linewidth}{|p{3.5cm}||
        *3{Y|} |M{3cm}|}\hline
        État initial (\si{mmol}) &
        $\num{10} $&
        $0 $&
        $0 $&
        $0 $\\
        \hline
        En cours (\si{mmol}) &
        $\num{10} - \xi $&
        $\xi $&
        $\xi $&
        $\xi $\\
        \hline
        État final (\si{mmol}) &
        $0$&
        $\num{10} $&
        $\num{10} $&
        $\num{10} $\\
        \hline
    \end{tabularx}
\end{center}

\begin{enumerate}[resume]
    \item En ne partant que de calcaire, dès que $\xi_f$ atteint $\xi_{\eq}$
        la réaction s'arrêtera puisqu'on aura atteint l'équilibre. Mettre plus
        de calcaire ne formera pas plus de chaux, l'excédent de réactif initial
        ne réagira simplement pas. Ainsi, \textbf{la quantité de matière de
        calcaire maximale qui puisse être transformée est de \SI{39}{mmol}}.
\end{enumerate}

\section{Équilibre en solution aqueuse}
\begin{enumerate}
    \item Pour déterminer le sens d'évolution du système, on calcule $Q_{r,i}$
        et on le compare à $K$~:
        \[\boxed{
                Q_{r,i} = \frac{[\ce{CH3CHH\moin{}\aqu{}}]_0[\ce{HF}]_O}
                    {[\ce{CH3COOH}]_0[\ce{F\moin{}}]_0} = 0 < K}
        \]
        La réaction évoluera donc \textbf{dans le sens direct}.\bigbreak

        Pour trouver l'avancement à l'équilibre, on dresse le tableau
        d'avancement, que l'on peut directement faire en concentrations puisque
        le volume ne varie pas (ce qui est toujours le cas cette année)~:
\end{enumerate}

\begin{center}
    \renewcommand{\arraystretch}{1.3}
    \centering
    \begin{tabularx}{\linewidth}{|p{3cm}||
        Y @{$+$} Y @{$=$} Y @{$+$} Y |}\hline
        Équation     &
        $\ce{CH3COOH\aqu{}} $ &
        $\ce{F\moin{}\aqu{}}$ &
        $\ce{CH3COO\moin{}\aqu{}}$ &
        $\ce{HF\aqu{}}$
    \end{tabularx}
    \par\vspace{-\lineskip}%
    \begin{tabularx}{\linewidth}{|p{3cm}|| *4{Y|}}\hline
        État initial &
        $c $&
        $c $&
        $0 $&
        $0 $\\
        \hline
        État final &
        $c -x_{\eq}$&
        $c -x_{\eq}$&
        $x_{\eq} $&
        $x_{\eq} $\\
        \hline
    \end{tabularx}
\end{center}

\begin{enumerate}[resume]
    \item[] D'après la loi d'action des masses, on a
        \begin{gather*}
            K = \frac{x_{\eq}{}^2}{(c-x_{\eq})^2}
            \Longleftrightarrow
            \sqrt{K} = \frac{x_{\eq}}{c-x_{\eq}}
            \Longleftrightarrow
            x_{\eq} = \sqrt{K}(c-x_{\eq})\\
            \Longleftrightarrow
            \boxed{x_{\eq} = \frac{\sqrt{K}}{1+ \sqrt{K}}c}
            \qavec
            \left\{
                \begin{array}{rcl}
                    K & = & \num{e-1.60}\\
                    c & = & \SI{0.1}{mol.L^{-1}}
                \end{array}
            \right.\\
            \mathrm{A.N.~:}\quad
            \boxed{x_{\eq} = \SI{1.4e-2}{mol.L^{-1}}}
        \end{gather*}
        En encadrant le résultat, on vérifie la cohérence physico-chimique de la
        réponse~: ici c'est bien cohérent de trouver $x_{\eq} > 0$ puisqu'on
        avait déterminer que la réaction se faisait dans le sens direct.

    \item De la même manière, pour déterminer le sens d'évolution du système, on
        calcule $Q_{r,i}$ et on le compare à $K$~:
        \[\boxed{
                Q_{r,i} = \frac{[\ce{CH3CHH\moin{}\aqu{}}]_0[\ce{HF}]_O}
                    {[\ce{CH3COOH}]_0[\ce{F\moin{}}]_0}
                        = \frac{c^2}{c^2} = 1 > K}
        \]
        La réaction évoluera donc \textbf{dans le sens indirect}.\bigbreak

        On effectue un bilan de matière grâce à un tableau d'avancement~:
\end{enumerate}

\begin{center}
    \renewcommand{\arraystretch}{1.3}
    \centering
    \begin{tabularx}{\linewidth}{|p{3cm}||
        Y @{$+$} Y @{$=$} Y @{$+$} Y |}\hline
        Équation     &
        $\ce{CH3COOH\aqu{}} $ &
        $\ce{F\moin{}\aqu{}}$ &
        $\ce{CH3COO\moin{}\aqu{}}$ &
        $\ce{HF\aqu{}}$
    \end{tabularx}
    \par\vspace{-\lineskip}%
    \begin{tabularx}{\linewidth}{|p{3cm}|| *4{Y|}}\hline
        État initial &
        $c $&
        $c $&
        $c $&
        $c $\\
        \hline
        État final &
        $c -x_{\eq}$&
        $c -x_{\eq}$&
        $c +x_{\eq} $&
        $c +x_{\eq} $\\
        \hline
    \end{tabularx}
\end{center}

\begin{enumerate}[resume]
    \item[] D'après la loi d'action des masses, on a
        \begin{gather*}
            K = \frac{(c+x_{\eq})^2}{(c-x_{\eq})^2}
            \Longleftrightarrow
            \sqrt{K} = \frac{c+x_{\eq}}{c-x_{\eq}}
            \Longleftrightarrow
            c+x_{\eq} = \sqrt{K}(c-x_{\eq})\\
            \Longleftrightarrow
            \boxed{x_{\eq} = \frac{\sqrt{K}-1}{\sqrt{K}+1}c}
            \qavec
            \left\{
                \begin{array}{rcl}
                    K & = & \num{e-1.60}\\
                    c & = & \SI{0.1}{mol.L^{-1}}
                \end{array}
            \right.\\
            \mathrm{A.N.~:}\quad
            \boxed{x_{\eq} = -\SI{5.3e-2}{mol.L^{-1}}}
        \end{gather*}
        De même que précédemment, on vérifie qu'il est logique de trouver
        $x_{\eq} < 0$~: la réaction se fait bien dans le sens indirect.
\end{enumerate}

\section{Équilibre en phase gazeuse}
On peut dresser le tableau d'avancement initial dans cette situation~:
\begin{center}
    \renewcommand{\arraystretch}{1.3}
    \centering
    \begin{tabularx}{.7\linewidth}{|p{3cm}||
        Y @{$=$} Y || M{2cm} |}\hline
        Équation             &
        $\ce{2FeCl3\gaz{}} $ &
        $\ce{Fe2Cl6\gaz{}}$  &
        $n_{\tot}$
    \end{tabularx}
    \par\vspace{-\lineskip}%
    \begin{tabularx}{.7\linewidth}{|p{3cm}||
        *2{Y|} | M{2cm} |}\hline
        État initial &
        $n_0 $       &
        $n_0 $       &
        $2n_0 $\\
        \hline
    \end{tabularx}
\end{center}
\begin{enumerate}
    \item Par la loi d'action des masses et les activités de constituants
        gazeux~:
        \[\boxed{K\degree = \frac{p_{\ce{Fe2Cl6}}p\degree}{p_{\ce{FeCl3}}{}^2}}\]
    \item Pour passer des pressions partielles aux quantités de matière, on
        utilise la loi de \textsc{Dalton}~:
\end{enumerate}
\begin{NCrapp}{Rappel~: loi de \textsc{Dalton}}
    Soit un mélange de gaz parfaits de pression $P$. Les pressions
    partielles $P_i$ de chaque constituant $\mathrm{X}_i$ s'exprime
    \[\boxed{P_i = x_iP}\]
    avec $x_i$ la fraction molaire du constituant~:
    \[\boxed{x_i = \frac{n_i}{n_{\tot}}}\]
\end{NCrapp}
\begin{enumerate}[resume]
    \item[] On écrit donc
        \[
            p_{\ce{Fe2Cl6}} = \frac{n_{\ce{Fe2Cl6}}}{n_{\tot}}\times p_{\rm
                tot}
            \qquad
            p_{\ce{FeCl3}} = \frac{n_{\ce{FeCl3}}}{n_{\tot}}\times p_{\rm
                tot}
        \]

        Pour simplifier l'écriture, on peut séparer les termes de pression
        totale des termes de matière en comptant combien vont arriver «~en
        haut~» et combien «~en bas~»~: 1 en haut contre 2 en bas, on se
        retrouvera avec $p_{\tot}$ au dénominateur, ce qui est logique par
        homogénéité vis-à-vis de $p\degree$ qui reste au numérateur. Comme
        $n_{\tot}$ apparaît le même nombre de fois que $p_{\tot}$ mais
        avec une puissance -1, on sait aussi qu'il doit se retrouver au
        numérateur, là aussi logiquement pour avoir l'homogénéité vis-à-vis de
        la quantité de matière. Ainsi,

        \[\boxed{
                Q_r = \frac{n_{\ce{Fe2Cl6}}/\cancel{n_{\tot}}
                    \times\bcancel{p_{\tot}}}
                    {n_{\ce{FeCl3}}{}^2/n_{\tot}^{\cancel{2}}
                    \times p_{\tot}^{\bcancel{2}}} p\degree
                    = \frac{n_{\ce{Fe2Cl6}} n_{\tot}}
                    {n_{\ce{FeCl3}}{}^2} \frac{p\degree}{p_{\tot}}
                }
        \]
        Avec $p_{\tot} = 2p\degree$ et $n_{\ce{Fe2Cl6}} = n_0 =
        n_{\ce{FeCl3}}$, on a $n_{\tot} = 2n_0$ (cf.\ tableau d'avancement),
        d'où

        \[
            Q_{r,i} = \frac{n_0\times2n_0}{n_0{}^2} \frac{1}{2}
            \Leftrightarrow
            \boxed{Q_{r,i} = 1}
        \]
    \item Le système serait à l'équilibre si $Q_{r,i} = K\degree$~; or, ici
        $Q_{r,i} \neq K$, donc l'équilibre n'est pas atteint. De plus, $Q_{r,i}
        < K$ donc le système évoluera dans le sens direct.
    \item On dresse le tableau d'avancement pour effectuer un bilan de matière
        dans cette nouvelle situation~:
\end{enumerate}
\begin{center}
    \renewcommand{\arraystretch}{1.3}
    \centering
    \begin{tabularx}{.7\linewidth}{|p{3cm}||
        Y @{$=$} Y || M{2cm} |}\hline
        Équation             &
        $\ce{2FeCl3\gaz{}} $ &
        $\ce{Fe2Cl6\gaz{}} $ &
        $n_{\tot}$
    \end{tabularx}
    \par\vspace{-\lineskip}%
    \begin{tabularx}{.7\linewidth}{|p{3cm}||
        *2{Y|} | M{2cm} |}\hline
        État initial &
        $n $         &
        $0 $         &
        $n $\\
        \hline
        État final &
        $n -2\xi$    &
        $ \xi $      &
        $n -\xi$\\
        \hline
    \end{tabularx}
\end{center}
\begin{enumerate}[resume]
    \item[] On reprend l'expression du quotient réactionnel initial en
        remplaçant les quantités de matière par leur expression selon $\xi$ pour
        déterminer l'avancement à l'équilibre, décrit par $K\degree$~:
        \begin{gather*}
            K\degree = \frac{\xi(n-\xi)}{(n-2\xi)^2}
            \underbrace{\frac{p\degree}{p_{\tot}}}_{=\frac{1}{2}}
            \Leftrightarrow
            K\degree = \underbrace{\cancel{\frac{n^2}{n^2}}}_{=1}
                \frac{\xi/n(1-\xi/n)}
                {\left( 1 - 2\xi/n \right)^2} \frac{1}{2}\\
        \end{gather*}
        Pour simplifier les calculs, posons $z = \frac{\xi}{n}$. L'équation
        précédente devient~:
        \begin{gather*}
            K\degree = \frac{1}{2} \frac{z(1-z)}{\left( 1 - 2z \right)^2}
            \Leftrightarrow
            2K\degree(1-2z)^2 = z(1-z)\\
            \Leftrightarrow
            2K\degree(1-4z + 4z^2) = z - z^2
            \Leftrightarrow
            z^2(8K\degree +1) - z(8K\degree +1) + 2K\degree = 0
        \end{gather*}
        On trouve un polynôme du second degré. Pour simplifier l'application
        numérique, on convertit directement $K\degree$ dans sa valeur pour avoir
        \[\boxed{z^2(167.4) - z(167.4) + 41.6 = 0}\]
        Soit $\Delta$ son discriminant~: on a \fbox{$\Delta = \num{167.4}$} et
        les racines sont
            $\DS\left\{
                \begin{array}{rcl}
                    z_1 & = & \num{0.54} \\
                    z_2 & = & \num{0.46}
                \end{array}
            \right.$.\bigbreak

        Étant donné qu'on part de $\xi = 0$ et que $\xi$ augmente, la valeur que
        prendrait $z_{\eq}$ serait $z_{\eq} = \num{0.46}$. On doit cependant
        vérifier que cette valeur est bien possible, en déterminant $z_{\max}$~:
        pour cela, on résout $n-2\xi = 0$, ce qui donne $z_{\max} = \num{0.5}$.
        On a bien $z_{\eq} < z_{\max}$, donc \textbf{l'équilibre est atteint} et
        on a \fbox{$\xi/n = \num{0.46}$}.
\end{enumerate}

\section{Transformations de gaz}
\begin{enumerate}
    \item Par la loi d'action des masses, on a
        \begin{gather*}
            K\degree_1 =
                \frac{p_{\ce{SO2}}p_{\ce{H2O}}}{p_{\ce{O2}}^{3/2}p_{\ce{H2S}}}
                \underbrace{\frac{p\degree^{5/2}}{p\degree^2}}_{= p\degree^{1/2}}
            \Leftrightarrow
            K\degree_1 =
                \frac{n_{\ce{H2O}}n_{\ce{SO2}}n_{\tot,
                gaz}^{1/2}}{n_{\ce{H2S}}n_{\ce{O2}}{}^{3/2}}
                \left(\frac{p\degree}{p}\right)^{1/2}
        \end{gather*}

        Soit $n_0$ la quantité de matière de $\ce{H2S}$ introduite
        initialement~: pour que $\ce{O2}$ soit introduit dans les proportions
        stœchiométriques, on relie sa quantité initiale à celle de $\ce{H2S}$
        \textit{via} les coefficients stœchiométriques tel que $n_{\ce{O2}}^0 =
        \frac{3}{2}n_{\ce{H2S}}^0 = \frac{3}{2}n_0$. D'où le tableau
        d'avancement~:
\end{enumerate}
\begin{center}
    \renewcommand{\arraystretch}{1.3}
    \centering
    \begin{tabularx}{\linewidth}{|p{3cm}||
        Y @{$+$} Y @{$=$} Y @{$+$} Y || M{3cm}|}\hline
        Équation           &
        $\ce{H2S\gaz{}} $  &
        $\ce{3/2O2\gaz{}}$ &
        $\ce{H2O\gaz{}}$   &
        $\ce{SO2\gaz{}}$   &
        $n_{\tot, gaz}$
    \end{tabularx}
    \par\vspace{-\lineskip}%
    \begin{tabularx}{\linewidth}{|p{3cm}||
    *4{Y|} |M{3cm}|}\hline
        État initial      &
        $n_0 $            &
        $\frac{3}{2}n_0 $ &
        $0 $              &
        $0 $              &
        $\frac{5}{2}n_0 $\\
        \hline
        État final                              &
        $n_0 -\xi_{\eq}$                        &
        $\frac{3}{2}n_0 - \frac{3}{2}\xi_{\eq}$ &
        $\xi_{\eq} $                            &
        $\xi_{\eq} $                            &
        $\frac{5}{2}n_0 - \frac{1}{2}\xi_{\eq}$\\
        \hline
    \end{tabularx}
\end{center}

\begin{enumerate}[resume]
    \item[] On peut donc remplacer les quantités de matière de l'expression de
        $K\degree_1$ par les expressions avec $\xi_{\eq}$~:
        \[\boxed{
            K\degree_1 =
                \frac{\xi_{\eq}^2( \frac{5}{2}n_0- \frac{1}{2}\xi_{\eq})^{1/2}}
                {(n_0-\xi_{\eq})( \frac{3}{2}n_0 - \frac{3}{2}\xi_{\eq})^{3/2}}
                \left(\frac{p\degree}{p}\right)^{1/2}}
            \]
    \item Par la loi d'action des masses, on a
        \begin{gather*}
            K\degree_2 =
                \frac{p_{\ce{H2O}}{}^2p\degree}{p_{\ce{SO2}}p_{\ce{H2S}{}^2}}
            \Leftrightarrow
            K\degree_1 =
                \frac{n_{\ce{H2O}}{}^2n_{\tot,
                gaz}}{n_{\ce{H2S}}{}^2n_{\ce{SO2}}}
                \frac{p\degree}{p}
        \end{gather*}

        Soit $n_1$ la quantité de matière de $\ce{H2S}$ introduite initialement,
        et $n_2$ la quantité de matière initiale en $\ce{SO2}$~:
       \end{enumerate}
\begin{center}
    \renewcommand{\arraystretch}{1.3}
    \centering
    \begin{tabularx}{\linewidth}{|p{3cm}||
        Y @{$+$} Y @{$=$} Y @{$+$} Y || M{3cm}|}\hline
        Équation           &
        $\ce{2H2S\gaz{}} $  &
        $\ce{SO2\gaz{}}$ &
        $\ce{2H2O\gaz{}}$   &
        $\ce{3S\liq{}}$   &
        $n_{\tot, gaz}$
    \end{tabularx}
    \par\vspace{-\lineskip}%
    \begin{tabularx}{\linewidth}{|p{3cm}||
    *4{Y|} |M{3cm}|}\hline
        État initial &
        $n_1 $       &
        $n_2 $       &
        $0 $         &
        $0 $         &
        $n_1+n_2$\\
        \hline
        État final        &
        $n_1 -2\xi_{\eq}$ &
        $n_2 - \xi_{\eq}$ &
        $2\xi_{\eq} $     &
        $3\xi_{\eq} $     &
        $n_1+n_2 - \xi_{\eq}$\\
        \hline
    \end{tabularx}
\end{center}

\begin{enumerate}[resume]
    \item[] On peut donc remplacer les quantités de matière de l'expression de
        $K\degree_1$ par les expressions avec $\xi_{\eq}$~:
        \[\boxed{
            K\degree_1 =
                \frac{4\xi_{\eq}^2(n_1+n_2-\xi_{\eq})}
                {(n_1-2\xi_{\eq})^2(n_2-\xi_{\eq})}
                \frac{p\degree}{p}}
            \]
    \item L'équation bilan équilibrée est~:
        \[\ce{CH4\gaz{} + 2O2\gaz{} \rightarrow CO2\gaz{} + 2H2O\liq{}}\]
        Soit $n_0$ la quantité initiale en dioxygène. Si la moitié seulement est
        consommée, alors que la réaction est totale, c'est que le méthane est
        limitant. On trouve la quantité de $\ce{CH4}$ à introduire initialement
        en dressant le tableau d'avancement pour que $n_{\ce{H2O},f} =
        \frac{1}{2}n_0$, c'est-à-dire $n_0 - 2\xi_{\max} = 0$~: on obtient
        \[\boxed{\xi_{\max} = \frac{1}{4}n_0}
        \qor
        n_{\ce{CH4}}^0 - \xi_{\max} = 0
        \qdonc
        \boxed{n_{\ce{CH4}}^0 = \frac{1}{4}n_0}
        \]
\end{enumerate}
\begin{center}
    \renewcommand{\arraystretch}{1.3}
    \centering
    \begin{tabularx}{\linewidth}{|p{3cm}||
        Y @{$+$} Y @{$=$} Y @{$+$} Y || M{3cm}|}\hline
        Équation          &
        $\ce{CH4\gaz{}} $ &
        $\ce{2O2\gaz{}}$  &
        $\ce{CO2\gaz{}}$  &
        $\ce{2H2O\liq{}}$ &
        $n_{\tot, gaz}$
    \end{tabularx}
    \par\vspace{-\lineskip}%
    \begin{tabularx}{\linewidth}{|p{3cm}||
    *4{Y|} |M{3cm}|}\hline
        État initial       &
        $ \frac{1}{4}n_0 $ &
        $n_0 $             &
        $0 $               &
        $0 $               &
        $ \frac{5}{4}n_0$\\
        \hline
        État initial            &
        $ \frac{1}{4}n_0 - \xi$ &
        $n_0 - 2\xi $           &
        $\xi $                  &
        $2\xi $                 &
        $ \frac{5}{4}n_0 - 2\xi$\\
        \hline
        État final         &
        $ 0$               &
        $ \frac{1}{2}n_0$  &
        $ \frac{1}{4}n_0 $ &
        $ \frac{1}{2}n_0 $ &
        $ \frac{3}{4}n_0$\\
        \hline
    \end{tabularx}
\end{center}

\section{Coefficient de dissociation}
\begin{enumerate}
    \item Pour connaître l'état de l'eau, on détermine la température en degrés
        pour interpréter par des connaissances élémentaires si elle est solide
        (glace), liquide, ou vapeur~: \SI{400}{K} = \SI{127}{\degreeCelsius}, et
        la pression est de \SI{1}{bar}, c'est-à-dire presque la pression
        habituelle. À cette température, l'eau est sous forme vapeur.
    \item La constante est extrêmement petite~: $K \ll \num{e-4}$, donc la
        réaction est \textbf{quasi-nulle} dans ce sens~: l'eau ne se dissocie
        pratiquement pas de cette manière et est par conséquent très stable.
    \item Si on introduit de l'eau pure, on n'a pas les autres composants au
        départ. Soit $n_0$ la quantité de matière d'eau pure introduite~: on
        dresse le tableau d'avancement
\end{enumerate}
\begin{center}
    \renewcommand{\arraystretch}{1.3}
    \centering
    \begin{tabularx}{\linewidth}{|p{3cm}||
        Y @{$=$} Y @{$+$} Y ||M{3cm}|}\hline
        Équation     &
        $\ce{2H2O\gaz{}} $ &
        $\ce{2H2\gaz{}}$ &
        $\ce{O2\gaz{}}$ &
        $n_{\tot, gaz}$
    \end{tabularx}
    \par\vspace{-\lineskip}%
    \begin{tabularx}{\linewidth}{|p{3cm}||
        *4{Y|} |M{3cm}|}\hline
        État initial &
        $n_0 $       &
        $0 $         &
        $0 $         &
        $n_0 $\\
        \hline
        État final  &
        $n_0 -2\xi$ &
        $2\xi   $   &
        $\xi   $    &
        $n_0 +\xi$\\
        \hline
    \end{tabularx}
\end{center}

\begin{enumerate}[resume]
    \item[] Le coefficient de dissociation correspond à la quantité d'eau
        transformée sur la quantité initiale, c'est-à-dire
        \[\alpha = \frac{2\xi}{n_0}\]
        On va donc exprimer la constante d'équilibre en fonction des quantités
        de matière pour introduire $\xi$ et faire apparaître $\alpha$, à l'aide
        de l'activité d'un gaz, de la loi de \textsc{Dalton} puis de la
        définition de la fraction molaire~:
        \begin{gather*}
            K = \frac{p_{\ce{H2}}^2p_{\ce{O2}}}{p_{\ce{H2O}^2}p\degree}
            \Longleftrightarrow
            K = \frac{n_{\ce{H2}}^2n_{\ce{O2}}}{n_{\ce{H2O}^2}n_{\tot, gaz}}
            \underbrace{\frac{p}{p\degree}}_{=1}\\
            \Longleftrightarrow
            K = \frac{4\xi^3}{(n_0-2\xi)^2(n_0+\xi)}
            \Longleftrightarrow
            K = \underbrace{\cancel{\frac{n_0{}^2}{n_0{}^3}}}_{=1}
                \frac{4 \left(\frac{xi}{n_0}\right)^3}
                {\left( 1- \frac{2\xi}{n_0} \right)^2
                \left( 1+ \frac{\xi}{n_0} \right)}\\
            \Longleftrightarrow
            K = \frac{4 \left(\frac{\alpha}{2}\right)^3}
                {\left( 1- \alpha \right)^2
                \left( 1+ \frac{\alpha}{2} \right)}
            \Longleftrightarrow
            \boxed{
            K = \frac{\alpha^3}
                {\left( 1- \alpha \right)^2
                \left( 2+ \alpha \right)}
        }
        \end{gather*}
        Une résolution numérique (\texttt{Python} ou calculatrice) donne
        \[\boxed{\alpha = \num{3.97e-20} \ll 1}\]
        Ceci est en accord avec le très faible avancement de la réaction.
    \item En prenant $\alpha = \num{0.30}$, cela veut dire que 30\% de l'eau se
        dissocie, l'eau ne serait plus stable dans ces conditions. La valeur de
        $K$ correspondant est $K = \num{2.4e-2}$, ce qui est peut favorisé dans
        le sens direct mais pas quasi-nulle.
\end{enumerate}

\section{Ions mercure}
\begin{enumerate}
    \item On détermine les concentrations en mercure (I) et (II)~:
        \begin{gather*}
            [\ce{HG\plus{2}\aqu{}}] = \frac{c_2V_2}{V_1 + V_2} = c_2' =
            \SI{0.4}{mmol.L^{-1}}
            \qqet
            [\ce{HG_2\plus{2}\aqu{}}] = \frac{c_1V_1}{V_1 + V_2} = c_1' =
            \SI{0.8}{mmol.L^{-1}}
        \end{gather*}
        On peut donc calculer le quotient de réaction initial, avec
        $a(\ce{Hg\liq{}}) )= 1$~:
        \[\boxed{Q_{r,i} = \frac{c_1'}{c_2'} = 2 < K}
        \quad\Longrightarrow\quad
        \text{évolution sens direct}\]
    \item 
        On dresse le tableau d'avancement en concentration~:
\end{enumerate}
\begin{center}
    \renewcommand{\arraystretch}{1.3}
    \centering
    \begin{tabularx}{\linewidth}{|p{4cm}||
        Y @{$+$} Y @{$=$} Y |}\hline
        Équation                 &
        $\ce{Hg\plus{2}\aqu{}} $ &
        $\ce{Hg\liq{}}$          &
        $\ce{Hg2\plus{2}\aqu{}}$ 
    \end{tabularx}
    \par\vspace{-\lineskip}%
    \begin{tabularx}{\linewidth}{|p{4cm}||
        *4{Y|}}\hline
        État initial &
        $c_2' $      &
        excès        &
        $c_1' $      \\
        \hline
        En cours   &
        $c_2' - x$ &
        excès      &
        $c_1' + x$ \\
        \hline
        État final (\si{mmol.L^{-1}}) &
        $\num{0.013}$                 &
        excès                         &
        $\num{1.187}$                 \\
        \hline
    \end{tabularx}
\end{center}

\begin{enumerate}[resume]
    \item[] Par la loi d'action des masses, on trouve en effet
        \begin{gather*}
            K\degree = \frac{c_1'+x_{\eq}}{c_2'-x_{\eq}}
            \Leftrightarrow
            \boxed{x_{\eq} = \frac{K\degree c_2' - c_1'}{K\degree + 1} =
            \SI{0.387}{mmol.L^{-1}}}
        \end{gather*}
        ce qui est bien inférieur à $x_{\max} = c_2' = \SI{0.4}{mmol.L^{-1}}$~:
        l'équilibre est atteint.
\end{enumerate}

\end{document}
