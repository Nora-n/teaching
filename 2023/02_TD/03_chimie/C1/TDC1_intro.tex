\documentclass[../../main/main.tex]{subfiles}
\graphicspath{{./figures/}}

\makeatletter
\renewcommand{\@chapapp}{Chimie -- chapitre}
\makeatother

\toggletrue{student}
\HideSolutionstrue
% \toggletrue{corrige}

\begin{document}
\setcounter{chapter}{1}

\chapter{\switch{Correction du TD}{TD~: Transformation et \'equilibre chimique}}

\resetQ
\section{Transformations}
Identifier la nature des transformations suivantes~:\\

\begin{minipage}{0.40\linewidth}
  \QR{$\ce{CH4 + 2O2 = CO2 + 2H2O}$}{Chimique}
  \QR{$\ce{C\sol{} + OS\gaz{} = CO2\gaz{}}$}{Chimique}
  \QR{$\ce{^{14}_{7}N + ^{1}_{0}n \ra ^{14}_{6}C + ^1_1p}$}{Nucléaire}
  \QR{$\ce{^14C + O2 \ra ^14CO2}$}{Chimique}
\end{minipage}
\hfill
\begin{minipage}{0.60\linewidth}
  \QR{$\ce{Fe\sol{} = Fe\liq{}}$}{Physique}
  \QR{$\ce{CH3COOH + CH3CH2OH = CH3COOCH2CH3 + H2O}$}{Chimique}
  \QR{$\ce{Zn + Cu^{+2} = Zn^{+2} + Cu}$}{Chimique}
  \QR{$\ce{CH3COOH + HO^{-} = H2O + CH3COO^{-}}$}{Chimique}
\end{minipage}

\resetQ
\section{Calculs de quantités de matière}

\begin{tcb}(data){Données}
\[
  M(\ce{Fe}) = \SI{55.8}{g.mol^{-1}}
  \qet
  M(\ce{Cu}) = \SI{63.5}{g.mol^{-1}}
\]

\end{tcb}


\QR{
  On verse dans un bécher une masse $m = \SI{350}{mg}$ de poudre de fer
  métallique. Quelle est la quantité de matière $n_{\ce{Fe}}$ correspondante~?
}{
  X
}

\QR{
  On dispose d'un flacon contenant $V_0 = \SI{800}{mL}$ de solution de sulfate
  de cuivre contenant les ions $\ce{Cu^{2+}}$ à la concentration $C =
  \SI{0.50}{mol.L^{-1}}$. Quelle est la quantité de matière correspondante~?
}{
  solu
}

\QR{
  On prélève $V = \SI{50}{mL}$ de cette solution. Quelle est la concentration du
  prélèvement~? Quelle est la quantité de matière $n_{\ce{Cu^{2+}}}$ prélevée~?
}{
  solu
}

Le prélèvement est versé dans le bécher~; une transformation chimique a lieu.

\QR{
  À l'issue de cette transformation, on obtient du cuivre métallique en quantité
  de matière $n_f = \SI{4.8}{mmol}$. Quelle est la masse correspondante~?
}{
  solu
}

\QR{
  On obtient également la même quantité de matière $n_f$ d'ions $\ce{Fe^{2+}}$.
  Quelle est la concentration correspondante~?
}{
  solu
}

\resetQ
\section{Dilution et mélange}

On dispose d'une solution de sulfate de cuivre contenant les ions $\ce{Cu^{2+}}$
et les ions sulfate $\ce{SO_4^{2-}}$ à la même concentration $C_0 =
\SI{1e-2}{mol.L^{-1}}$. On en prélève à la pipette jaugée un volume $V_0 =
\SI{10}{mL}$ que l'on verse dans une fiole jaugée de volume $V_1 = \SI{50}{mL}$.
On remplit la fiole d'eau distillée jusqu'au trait de jauge.

\QR{
  Quelle est la concentration $C_1$ en ions $\ce{Cu^{2+}}$ et en ions
  $\ce{SO_4^{2-}}$ dans la fiole~?
}{
  solu
}

On verse le contenu de cette fiole dans un bécher. On y ajoute un volume $V_2 =
\SI{20}{mL}$ d’une solution de sulfate de magnésium, contenant les ions
$\ce{Mg^{2+}}$ et les ions $\ce{SO_4^{2-}}$ à la même concentration $C_2 =
\SI{2e-2}{mol.L^{-1}}$.

\QR{
  Calculer les concentrations des trois ions après le mélange.
}{
  solu
}

\resetQ
\section{Concentration en soluté apporté}
\begin{tcb}(data){}
\[
  M(\ce{Mg}) = \SI{24.3}{g.mol^{-1}}
  \qet
  M(\ce{Cl}) = \SI{35.5}{g.mol^{-1}}
\]

\end{tcb}

\QR{
  Identifier les ions présents dans l'acide sulfurique $\ce{H_2SO_4}$. Écrire
  l'équation de dissolution.
}{
  solu
}
\QR{
  On ajoute une quantité de matière $n_{\rm app} = \SI{2e-2}{mol}$ en acide
  sulfurique dans de l'eau distillée. Déterminer les quantités de matière de
  chaque ion dans la solution formée.
}{
  solu
}
\QR{
  La solution des questions précédentes a un volume $V = \SI{200}{mL}$. Calculer
  la concentration en soluté approté, puis les concentrations des ions dans la
  solution après dissolution.
}{
  solu
}
\QR{
  On considère une solution de chlorure de chrome $\ce{CrCl_3}$ de concentration
  en soluté apporté $c = \SI{5e-3}{mol.L^{-1}}$. Déterminer les concentrations
  des ions dans la solution.
}{
  solu
}
\QR{
  On dissout $m = \SI{6.0}{g}$ de chlorure de magnésium $\ce{MgCl_2}$ dans
  \SI{200}{mL} d'eau distillée. Calculer la concentration en soluté approté,
  puis les concentrations des ions dans la solution
}{
  solu
}


\end{document}
