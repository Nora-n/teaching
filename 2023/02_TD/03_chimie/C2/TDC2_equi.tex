\documentclass[../../main/main.tex]{subfiles}
\graphicspath{{./figures/}}

\makeatletter
\renewcommand{\@chapapp}{Chimie -- chapitre}
\makeatother

% \toggletrue{student}
\HideSolutionstrue
\toggletrue{corrige}
% \renewcommand{\mycol}{black}
\renewcommand{\mycol}{gray}

\begin{document}
\setcounter{chapter}{1}

\chapter{\cswitch{Correction du TD}
  {TD~: Transformation et \'equilibre chimique}}

\section{Transformations totales}
\enonce{%
	Compléter les tableaux suivants. Les gaz seront supposés parfaits. Dans la
	ligne $t$, on demande d'exprimer la quantité de matière en fonction de
	l'avancement molaire $\xi(t)$ à un instant $t$ quelconque.
}
\QR{%
	Réaction de l'oxydation du monoxyde d'azote en phase gazeuse, à $T =
		\SI{25}{\degreeCelsius}$ dans un volume $V = \SI{10.0}{L}$~:
	\begin{center}
		\def\rhgt{0.50}
		\centering
		\begin{tabularx}{\linewidth}{|l|c||YdYdY||Y|Y|}
			\hline
			\multicolumn{2}{|c||}{
				$\xmathstrut{\rhgt}$
			\textbf{Équation}}      &
			$\ldots\ce{NO\gaz{}}$   & $+$   &
			$\ldots\ce{O_2\gaz{}}$  & $\ra$ &
			$\ldots\ce{NO_2\gaz{}}$ &
			$n_{\tot, gaz} $        &
			$P_{\tot} (\si{bar})$                             \\
			\hline
			$\xmathstrut{\rhgt}$
			Initial (\si{mol}) & $\xi = 0$                  &
			$\num{1.00}$       & \vline                     &
			$\num{2.00}$       & \vline                     &
			$\num{0.00}$       &
			                   &
			\\
			\hline
			$\xmathstrut{\rhgt}$
			Interm. (\si{mol}) & $\xi$                      &
			                   & \vline                     &
			                   & \vline                     &
			                   &
			                   &
			\\
			\hline
			$\xmathstrut{\rhgt}$
			Final (\si{mol})    & $\xi_f = \wht{\num{0.50}}$ &
			                    & \vline                     &
			                    & \vline                     &
			                    &
			                    &
			\\
			\hline
		\end{tabularx}
	\end{center}
}{%
  Pour la quantité totale de gaz, il suffit de sommer les quantité de matière de
  chacun des gaz~: ici, initialement on a $n_0(\ce{NO\gaz{}}) +
  n_0(\ce{O_2\gaz{}}) = \SI{3.00}{mol}$ de gaz. Ensuite, pour la pression totale
  on utilise l'équation d'état des gaz parfaits~:
	\begin{tcb}(rapp){Rappel~: gaz parfait}
    \vspace{-15pt}
		\begin{gather*}
			\boxed{pV = nRT}
			\qavec
			\left\{
			\begin{array}{ll}
				p & \text{en Pa }                 \\
				V & \text{en m}^3                 \\
				n & \text{en mol}                 \\
				T & \text{en \textbf{Kelvin} (K)}
			\end{array}
			\right.\\
			\text{et }
			\boxed{R = \SI{8.314}{J.mol^{-1}.K^{-1}}}\\
			\text{est la constante des gaz parfaits}
		\end{gather*}
	\end{tcb}
	Il faut donc convertir le volume en $\si{m^3}$. Pour cela, il suffit
	d'écrire
	\[\SI{10.0}{L} = \SI{10.0}{dm^3} = \num{10.0}(\SI{e-1}{m})^3 =
		\SI{10.0e-3}{m^3}\]
	Il est très courant d'oublier les puissances sur les conversions du
	genre~: n'oubliez pas les parenthèses. Il nous faut de plus convertir la
	température en Kelvins, attention à ne pas vous tromper de sens~: il
	faut ici \textbf{ajouter} \SI{273.15}{K} à la température en degrés
	Celsius, ce qui donne $T = \SI{298.15}{K}$. On peut donc faire
	l'application numérique pour $P_{\tot}$ initial.
	\bigbreak
	On remplit la deuxième ligne du tableau avec les coefficients
	stœchiométriques algébriques des constituants en facteur de chaque
	$\xi$, et on somme les quantités de matière de gaz pour $n_{\tot,
				gaz}$. En réalité, il est plus simple de partir de la valeur totale de
	la première ligne et de compter algébriquement le nombre de $\xi$~: on
	en perd 3 avec les réactifs pour en gagner 2 avec les produits, donc en
	tout la quantité de matière totale de gaz décroit de 1$\xi$. On ne peut
	pas calculer précisément la valeur de $P_{\tot}$ ici, il faudrait
	l'exprimer en fonction de $\xi$ (ça viendra dans d'autres
	exercices).
  \bigbreak
	Enfin, pour trouver le réactif limitant, on résout~:
	\begin{gather*}
		\left\{
		\begin{array}{rcl}
			n_0(\ce{NO\gaz{}})-2\xi_f & = & 0 \\
			n_0(\ce{O_2\gaz{}})-\xi_f & = & 0
		\end{array}
		\right.
		\Longleftrightarrow
		\left\{
		\begin{array}{rcl}
			\xi_f & = & \SI{0.50}{mol}              \\
			\xi_f & = & \SI{1.00}{mol} > \xi_{\max}
		\end{array}
		\right.
	\end{gather*}
	La seule valeur possible est la plus petite, $\xi_f = \SI{0.50}{mol}$~:
	si on prenait \SI{1.00}{mol} on trouverait une quantité négative de
	\ce{NO} à l'état final, ce qui, vous en conviendrez, est une absurdité.
	Même travail qu'initialement pour $n_{\tot, gaz}$ et $P_{\tot}$.
	D'où le tableau~:
	\begin{center}
		\def\rhgt{0.50}
		\centering
		\begin{tabularx}{\linewidth}{|l|c||YdYdY||Y|Y|}
			\hline
			\multicolumn{2}{|c||}{
				$\xmathstrut{\rhgt}$
			\textbf{Équation}}  &
			$2\ce{NO\gaz{}}$    & $+$                  &
			$\ce{O_2\gaz{}}$    & $\ra$                &
			$2\ce{NO_2\gaz{}}$  &
			$n_{\tot, gaz} $    &
			$P_{\tot} (\si{bar})$                        \\
			\hline
			$\xmathstrut{\rhgt}$
			Initial (\si{mol})  & $\xi = 0$            &
			$\num{1.00}$        & \vline               &
			$\num{2.00}$        & \vline               &
			$\num{0.00}$        &
			$\num{3.00}$        &
			$\num{7.40}$                                 \\
			\hline
			$\xmathstrut{\rhgt}$
			Interm. (\si{mol})  & $\xi$                &
			$\num{1.00} - 2\xi$ & \vline               &
			$\num{2.00} - \xi$  & \vline               &
			$2\xi$              &
			$\num{3.00} - \xi$  &
			--                                           \\
			\hline
			$\xmathstrut{\rhgt}$
			Final (\si{mol}) & $\xi_f = \num{0.50}$ &
			\num{0.00}       & \vline                     &
			\num{1.50}       & \vline                     &
			\num{1.00}       &
			\num{2.50}       &
			\num{6.20}                                   \\
			\hline
		\end{tabularx}
	\end{center}
}
\QR{%
  Réaction de combustion de l'éthanol dans l'air. Les réactifs sont
  introduits dans les proportions stœchiométriques. Le dioxygène provient
  de l'air, qui contient 20\% de $\ce{O2}$ et 80\% de $\ce{N2}$ en
  fraction molaire.
  \begin{center}
    \def\rhgt{0.50}
    \centering
    \begin{tabularx}{\linewidth}{|l|c||YdYdYdY||Y|Y|}
      \hline
      \multicolumn{2}{|c||}{
        $\xmathstrut{\rhgt}$
      \textbf{Équation} (\si{mol})} &
      $\ce{C_2H_2OH\liq{}}$         & $+$                        &
      $3\ce{O_2\gaz{}}$             & $\ra$                      &
      $2\ce{CO_2\gaz{}}$            & $+$                        &
      $3\ce{H_2O\gaz{}}$            &
      $n_{\ce{N_2}}$                &
      $n_{\tot, gaz}$                                              \\
      \hline
      $\xmathstrut{\rhgt}$
      Initial                       & $\xi = 0$                  &
      $\num{2.00}$                  & \vline                     &
                                    & \vline                     &
                                    & \vline                     &
                                    &
                                    &
      \\
      \hline
      $\xmathstrut{\rhgt}$
      Interm.                       & $\xi$                      &
                                    & \vline                     &
                                    & \vline                     &
                                    & \vline                     &
                                    &
                                    &
      \\
      \hline
      $\xmathstrut{\rhgt}$
      Final                         & $\xi_f = \wht{\num{2.00}}$ &
                                    & \vline                     &
                                    & \vline                     &
                                    & \vline                     &
                                    &
                                    &
      \\
      \hline
    \end{tabularx}
  \end{center}
}{%
  Pour une réaction $a{\ce{A}} + b{\ce{B}} = c{\ce{C}} + d{\ce{D}}$, le fait
  que les réactifs soient introduits dans les proportions stœchiométriques
  se traduit par
  \[ \frac{n_{\ce{A}}^0}{a} = \frac{n_{\ce{B}}^0}{b}
    \Longleftrightarrow
    n_{\ce{B}}^0 = \frac{b}{a}n_{\ce{A}}^0
  \]
  Ici, on a donc $n_{\ce{O2}}^0 = 3n_{\ce{C2H2OH}}^0$, c'est-à-dire
  $n_{\ce{O2}}^0 = \SI{6.00}{mol}$. On peut donc remplir cette
  case.
  \bigbreak
  On suppose qu'on commence sans $\ce{CO2}$ ou $\ce{H2O}$ initialement,
  puisque rien n'est indiqué~; en revanche, on sait qu'il y a déjà du
  diazote dans le milieu puis que le dioxygène vient de l'air, comme c'est
  indiqué. Comme il y a 80\% de \ce{N2} pour 20\% de \ce{O2}, cela veut
  dire qu'il y a 4 fois plus de diazote que de dioxygène, donc
  \SI{24.00}{mol}. Ici, la colonne $n_{\tot, gaz}$ n'a pas grande
  utilité puisqu'il n'y a qu'un gaz, mais c'est une bonne pratique à ne
  pas oublier.
  \bigbreak
  Le reste du remplissage est le même que pour la question 1. On trouve
  évidemment $\xi_f = \SI{2.00}{mol}$ avec les deux réactifs limitant,
  c'est le principe des proportions stœchiométriques.
  \begin{center}
    \def\rhgt{0.50}
    \centering
    \begin{tabularx}{\linewidth}{|l|c||YdYdYdY||Y|Y|}
      \hline
      \multicolumn{2}{|c||}{
        $\xmathstrut{\rhgt}$
      \textbf{Équation} (\si{mol})} &
      $\ce{C_2H_2OH\liq{}}$         & $+$                  &
      $3\ce{O_2\gaz{}}$             & $\ra$                &
      $2\ce{CO_2\gaz{}}$            & $+$                  &
      $3\ce{H_2O\gaz{}}$            &
      $n_{\ce{N_2}}$                &
      $n_{\tot, gaz}$                                        \\
      \hline
      $\xmathstrut{\rhgt}$
      Initial                       & $\xi = 0$            &
      $\num{2.00}$                  & \vline               &
      $\num{6.00}$                  & \vline               &
      $\num{0.00}$                  & \vline               &
      $\num{0.00}$                  &
      $\num{24.00}$                 &
      $\num{30.00}$                                          \\
      \hline
      $\xmathstrut{\rhgt}$
      Interm.                       & $\xi$                &
      $\num{2.00} - \xi$            & \vline               &
      $\num{6.00} - 3\xi$           & \vline               &
      $2\xi$                        & \vline               &
      $3\xi$                        &
      $\num{24.00}$                 &
      $\num{30.00} + 2\xi$                                   \\
      \hline
      $\xmathstrut{\rhgt}$
      Final                         & $\xi_f = \num{2.00}$ &
      $\num{0.00}$                  & \vline               &
      $\num{0.00}$                  & \vline               &
      $\num{4.00}$                  & \vline               &
      $\num{6.00}$                  &
      $\num{24.00}$                 &
      $\num{34.00}$                                          \\
      \hline
    \end{tabularx}
  \end{center}
}

\section{Équilibre… ou pas~!}
\resetQ
\enonce{%
La dissociation du peroxyde de baryum sert à l'obtention de dioxygène avant la
mise au point de la liquéfaction de l'air, selon l'équation

\centersright{%
	$\ce{2BaO2\sol{} <=> 2BaO\sol{} + O2\gaz{}}$%
}{%
	$K\degree(\SI{795}{\degreeCelsius}) = \num{0.50}$%
}

Le volume de l'enceinte, initialement vide de tout gaz, vaut $V = \SI{10}{L}$.
On rappelle que $R = \SI{8.314}{J.K^{-1}.mol^{-1}}$.
}

\begin{blocQR}
	\item
	\QR{%
		Exprimer la constante d'équilibre $K\degree$ en fonction de la pression
		partielle à l'équilibre $p_{\ce{O2}, \eq}$.
	}{%
		Par définition, $K\degree = Q_{r, \eq}$. On exprime donc le
		quotient de réaction avec les activités à l'équilibre~:
		\[
      \boxed{K\degree = \frac{p_{\ce{O2,\eq}}}{p\degree}}
    \]
	}

	\QR{%
		En déduire la valeur numérique de $p_{\ce{O2}, \eq}$.
	}{%
		On a la valeur de $K\degree$ et la valeur de $p\degree$~: de
		l'équation précédente on isole $p_{\ce{O2}, \eq}$~:
		\begin{gather*}
			\boxed{p_{\ce{O2},\eq} = K\degree p\degree}
			\qavec
			\left\{
			\begin{array}{rcl}
				K\degree & = & \num{0.50}     \\
				p\degree & = & \SI{1.00}{bar}
			\end{array}
			\right.\\
			\AN
			\xul{p_{\ce{O2},\eq} = \SI{0.50}{bar} = \SI{5.0e4}{Pa}}
		\end{gather*}
	}
	\QR{%
		Calculer le nombre de moles de dioxygène qui permet d'atteindre
		cette pression dans l'enceinte.
	}{%
		Avec la loi des gaz parfaits, on a
		\begin{gather*}
			p_{\ce{O2},\eq}V = n_{\ce{O2}, \eq}RT
			\Lra
			\boxed{n_{\ce{O2}, \eq} = \frac{p_{\ce{O2},\eq}V}{RT}}
			\qavec
			\left\{
			\begin{array}{rcl}
				V & = & \SI{10e-3}{m^3} \\
				T & = & \SI{1068.15}{K}
			\end{array}
			\right.\\
			\AN
			\xul{n_{\ce{O2}, \eq} = \SI{0.056}{mol}}
		\end{gather*}
	}
\end{blocQR}
\begin{blocQR}
	\item Cas 1~:
	\enonce{
		\begin{center}
			\def\rhgt{0.50}
			\centering
			\begin{tabularx}{\linewidth}{|l|c||YdYdY||Y|}
				\hline
				\multicolumn{2}{|c||}{
					$\xmathstrut{\rhgt}$
				\textbf{Équation}}  &
				$2\ce{BaO_2\sol{}}$ & $\rightleftharpoons$ &
				$2\ce{BaO\sol{}}$   & $+$                  &
				$\ce{O_2\gaz{}}$    &
				$n_{\tot, gaz}$                              \\
				\hline
				$\xmathstrut{\rhgt}$
				Initial (\si{mol})  & $\xi = 0$            &
				$\num{0.20}$        & \vline               &
				$\num{0.00}$        & \vline               &
				$\num{0.00}$        &
				$\num{0.00}$                                 \\
				\hline
				$\xmathstrut{\rhgt}$
				Interm. (\si{mol})  & $\xi$                &
				                    & \vline               &
				                    & \vline               &
				                    &
				\\
				\hline
				$\xmathstrut{\rhgt}$
				Final  (\si{mol})   & $\xi = \xi_f$        &
				                    & \vline               &
				                    & \vline               &
				                    &
				\\
				\hline
			\end{tabularx}
		\end{center}
	}
	\ifcorrige{
		\begin{center}
			\def\rhgt{0.50}
			\centering
			\begin{tabularx}{\linewidth}{|l|c||YdYdY||Y|}
				\hline
				\multicolumn{2}{|c||}{
					$\xmathstrut{\rhgt}$
				\textbf{Équation}}  &
				$2\ce{BaO_2\sol{}}$ & $\rightleftharpoons$ &
				$2\ce{BaO\sol{}}$   & $+$                  &
				$\ce{O_2\gaz{}}$    &
				$n_{\tot, gaz}$                              \\
				\hline
				$\xmathstrut{\rhgt}$
				Initial (\si{mol})  & $\xi = 0$            &
				$\num{0.20}$        & \vline               &
				$\num{0.00}$        & \vline               &
				$\num{0.00}$        &
				$\num{0.00}$                                 \\
				\hline
				$\xmathstrut{\rhgt}$
				Interm. (\si{mol})  & $\xi$                &
				$\num{0.20} - 2\xi$ & \vline               &
				$2\xi$              & \vline               &
				$\xi$               &
				$\xi$                                        \\
				\hline
				$\xmathstrut{\rhgt}$
				Final  (\si{mol})   & $\xi = \xi_f$        &
				$\num{0.088}$       & \vline               &
				$\num{0.112}$       & \vline               &
				$\num{0.056}$       &
				$\num{0.056}$                                \\
				\hline
			\end{tabularx}
		\end{center}
	}
	\QR{%
		Calculer le quotient de réaction initial $Q_{r,0}$ et en déduire le sens
		d'évolution du système.
	}{%
		On change juste $p_{\ce{O2},\eq}$ de la première question en
		$p_{\ce{O2}, 0}$~; sachant qu'on commence sans gaz dans l'enceinte,
		cette pression est nulle~:
		\[\boxed{Q_{r,0} = \frac{p_{\ce{O2}, 0}}{p\degree} = 0}\]
		On a donc $Q_{r,0} < K$, et l'évolution se fait en sens direct.
	}
	\QR{%
		Remplir le tableau d'avancement et remplir la ligne intermédiaire dans le
		tableau en fonction de $\xi$.
	}{%
		Voir tableau.
	}
	\QR{%
		Déterminer $\xi_f$ en précisant si l'équilibre est atteint ou pas. On
		rappelle que l'équilibre correspond à la coexistence de toutes les
		espèces.
	}{%
		~
		\vspace{-15pt}
		\begin{tcb}(tool){État d'équilibre}
			Pour trouver l'état final dans cette situation, \textbf{on
				détermine $\xi_{\eq}$ s'il y avait équilibre, et on regarde
				si c'est compatible avec $\xi_{\max}$ si la réaction était
				totale}.
		\end{tcb}
		S'il y a équilibre, ça veut dire que $n_{\ce{O2}, \eq} =
			\SI{0.056}{mol}$ comme déterminé au début. Or, le tableau nous
		indique que $n_{\ce{O2}, f} = \xi_f$, donc si c'est un équilibre
		\xul{$\xi_{\eq} = \SI{0.056}{mol}$}.
    \bigbreak
		L'avancement est maximal si \ce{BaO2} est limitant~: on trouve donc
		$\xi_{\max}$ en résolvant $\num{0.20} - 2\xi_{\max} = 0$,
		c'est-à-dire \xul{$\xi_{\max} = \SI{0.1}{mol}$}.
    \bigbreak
		La valeur est finale $\xi_f$ est la plus petite valeur (en valeur
		absolue) de $\xi_{\eq}$ et $\xi_{\max}$~; or ici $\xi_{\eq} <
			\xi_{\max}$~: il y a donc bien équilibre, et on a
		\[\xul{\xi_f = \xi_{\eq} = \SI{0.056}{mol}}\]
	}
	\QR{%
		Remplir la dernière ligne du tableau d'avancement.
	}{%
		Voir tableau.
	}
\end{blocQR}
\QR{%
	Mêmes questions dans le cas 2~:
	\begin{center}
		\def\rhgt{0.50}
		\centering
		\begin{tabularx}{\linewidth}{|l|c||YdYdY||Y|}
			\hline
			\multicolumn{2}{|c||}{
				$\xmathstrut{\rhgt}$
			\textbf{Équation}}  &
			$2\ce{BaO_2\sol{}}$ & $\rightleftharpoons$ &
			$2\ce{BaO\sol{}}$   & $+$                  &
			$\ce{O_2\gaz{}}$    &
			$n_{\tot, gaz}$                              \\
			\hline
			$\xmathstrut{\rhgt}$
			Initial (\si{mol}) & $\xi = 0$ &
			$\num{0.10}$       & \vline    &
			$\num{0.00}$       & \vline    &
			$\num{0.00}$       &
			$\num{0.00}$                               \\
			\hline
			$\xmathstrut{\rhgt}$
			Interm. (\si{mol})  & $\xi$                &
			                    & \vline               &
			                    & \vline               &
			                    &
			\\
			\hline
			$\xmathstrut{\rhgt}$
			Final  (\si{mol})   & $\xi = \xi_f$        &
			                    & \vline               &
			                    & \vline               &
			                    &
			\\
			\hline
		\end{tabularx}
	\end{center}
}{%
	Cas 2~:
	\begin{center}
		\def\rhgt{0.50}
		\centering
		\begin{tabularx}{\linewidth}{|l|c||YdYdY||Y|}
			\hline
			\multicolumn{2}{|c||}{
				$\xmathstrut{\rhgt}$
			\textbf{Équation}}       &
			$2\ce{BaO_2\sol{}}$      & $\rightleftharpoons$ &
			$2\ce{BaO\sol{}}$        & $+$                  &
			$\ce{O_2\gaz{}}$         &
			$n_{\tot, gaz}$                                   \\
			\hline
			$\xmathstrut{\rhgt}$
			Initial (\si{mol})       & $\xi = 0$            &
			$\num{0.10}$             & \vline               &
			$\num{0.00}$             & \vline               &
			$\num{0.00}$             &
			$\num{0.00}$                                      \\
			\hline
			$\xmathstrut{\rhgt}$
			Interm. (\si{mol})       & $\xi$                &
			$\num{0.10} - 2\xi$      & \vline               &
			$2\xi$                   & \vline               &
			$\xi$                    &
			$\xi$                                             \\
			\hline
			$\xmathstrut{\rhgt}$
			Final  (\si{mol})        & $\xi = \xi_f$        &
			$\ifcorrige{\num{0.00}}$ & \vline               &
			$\ifcorrige{\num{0.10}}$ & \vline               &
			$\ifcorrige{\num{0.05}}$ &
			$\ifcorrige{\num{0.05}}$                          \\
			\hline
		\end{tabularx}
	\end{center}
	\begin{enumerate}[leftmargin=20pt, label=\alph* --]
		\item On a toujours aucun gaz au départ, donc ici aussi
		      \[\boxed{Q_{r,0} = \frac{p_{\ce{O2}, 0}}{p\degree} = 0}\]
		      et la réaction est en sens direct.
		\item Voir tableau.
		\item Même procédé~: \textbf{on détermine $\xi_{\eq}$ s'il y avait
			      équilibre, et on regarde si c'est compatible avec $\xi_{\max}$
			      si la réaction était totale}.
            \bigbreak
		      S'il y a équilibre, ça veut dire que $n_{\ce{O2}, \eq} =
			      \SI{0.056}{mol}$ comme déterminé au début. Or, le tableau nous
		      indique que $n_{\ce{O2}, f} = \xi_f$, donc si c'est un équilibre
		      \xul{$\xi_{\eq} = \SI{0.056}{mol}$}.
          \bigbreak
		      L'avancement est maximal si \ce{BaO2} est limitant~: on trouve donc
		      $\xi_{\max}$ en résolvant $\num{0.10} - 2\xi_{\max} = 0$,
		      c'est-à-dire \xul{$\xi_{\max} = \SI{0.050}{mol}$}.
          \bigbreak
		      La valeur est finale $\xi_f$ est la plus petite valeur (en valeur
		      absolue) de $\xi_{\eq}$ et $\xi_{\max}$~; or ici $\xi_{\eq} >
			      \xi_{\max}$~: il n'y a donc \textbf{pas équilibre}, et on a
		      \[\xul{\xi_f = \xi_{\max} = \SI{0.050}{mol}}\]
		\item Voir tableau.
	\end{enumerate}
}
\QR{%
	Mêmes questions dans le cas 3~:
	\begin{center}
		\def\rhgt{0.50}
		\centering
		\begin{tabularx}{\linewidth}{|l|c||YdYdY||Y|}
			\hline
			\multicolumn{2}{|c||}{
				$\xmathstrut{\rhgt}$
			\textbf{Équation}}  &
			$2\ce{BaO_2\sol{}}$ & $\rightleftharpoons$ &
			$2\ce{BaO\sol{}}$   & $+$                  &
			$\ce{O_2\gaz{}}$    &
			$n_{\tot, gaz}$                              \\
			\hline
			$\xmathstrut{\rhgt}$
			Initial (\si{mol})  & $\xi = 0$            &
			$\num{0.10}$         & \vline               &
			$\num{0.050}$        & \vline               &
			$\num{0.10}$         &
			$\num{0.10}$                                  \\
			\hline
			$\xmathstrut{\rhgt}$
			Interm. (\si{mol})  & $\xi$                &
			                    & \vline               &
			                    & \vline               &
			                    &
			\\
			\hline
			$\xmathstrut{\rhgt}$
			Final  (\si{mol})   & $\xi = \xi_f$        &
			                    & \vline               &
			                    & \vline               &
			                    &
			\\
			\hline
		\end{tabularx}
	\end{center}
}{%
	Cas 3~:
	\begin{center}
		\def\rhgt{0.50}
		\centering
		\begin{tabularx}{\linewidth}{|l|c||YdYdY||Y|}
			\hline
			\multicolumn{2}{|c||}{
				$\xmathstrut{\rhgt}$
			\textbf{Équation}}   &
			$2\ce{BaO_2\sol{}}$  & $\rightleftharpoons$ &
			$2\ce{BaO\sol{}}$    & $+$                  &
			$\ce{O_2\gaz{}}$     &
			$n_{\tot, gaz}$                               \\
			\hline
			$\xmathstrut{\rhgt}$
			Initial (\si{mol})   & $\xi = 0$            &
			$\num{0.10}$         & \vline               &
			$\num{0.050}$        & \vline               &
			$\num{0.10}$         &
			$\num{0.10}$                                  \\
			\hline
			$\xmathstrut{\rhgt}$
			Interm. (\si{mol})   & $\xi$                &
			$\num{0.10} - 2\xi$  & \vline               &
			$\num{0.050} + 2\xi$ & \vline               &
			$\num{0.10} + \xi$   &
			$\num{0.10} + \xi$                            \\
			\hline
			$\xmathstrut{\rhgt}$
			Final  (\si{mol})    & $\xi = \xi_f$        &
			$\num{0.15}$         & \vline               &
			$\num{0.00}$         & \vline               &
			$\num{0.075}$        &
			$\num{0.075}$                                 \\
			\hline
		\end{tabularx}
	\end{center}
	\begin{enumerate}[leftmargin=20pt, label=\alph* --]
		\item On a cette fois du gaz au départ, donc ici
		      \begin{gather*}
			      Q_{r,0}
			      = \frac{p_{\ce{O2}, 0}}{p\degree}
            \Lra
            \boxed{
              Q_{r,0}
              = \frac{n_{\ce{O2}, 0}RT}{Vp\degree}
            }
			      \qavec
			      \left\{
			      \begin{array}{rcl}
				      n_{\ce{O2}, 0} & = & \SI{0.10}{mol}  \\
				      T              & = & \SI{1069.15}{K} \\
				      V              & = & \SI{10e-3}{m^3}
			      \end{array}
			      \right.\\
			      \AN
			      \xul{Q_{r,0} = \num{0.89}}
		      \end{gather*}
		      Cette fois, $Q_{r,0} > K$ donc la réaction se fait en sens
		      indirect.
		\item Voir tableau.
		      \begin{tcb}(impo){Important}

			      Le procédé de remplissage du tableau \textbf{ne doit pas changer}
			      même si la réaction se fait dans le sens indirect~: les coefficients
			      stœchiométriques de la réaction n'ont pas changé, donc les facteurs
			      devant des $\xi(t)$ non plus.
			      \smallbreak
			      Certes, on aura $\xi < 0$ mais il est plus naturel et moins
			      perturbant de garder la forme de base du remplissage du tableau
			      plutôt que de s'embêter à repenser l'écriture du tableau.

		      \end{tcb}
		\item Même procédé~: \textbf{on détermine $\xi_{\eq}$ s'il y
			      avait équilibre, et on regarde si c'est compatible avec
			      $\xi_{\max}$ si la réaction était totale}.
            \bigbreak
		      S'il y a équilibre, ça veut dire que $n_{\ce{O2}, \eq} =
			      \SI{0.056}{mol}$ comme déterminé au début. Or, le tableau nous
		      indique que $n_{\ce{O2}, f} = \num{0.10} + \xi_f$, donc si c'est
		      un équilibre \xul{$\xi_{\eq} = -\SI{0.044}{mol}$}.
          \bigbreak
		      L'avancement est maximal si \ce{BaO} ou \ce{O2} sont limitant~:
		      on résout donc
		      \begin{gather*}
			      \left\{
			      \begin{array}{rcl}
				      n_{\ce{BaO}}^{0}+2\xi_{\max} & = & 0 \\
				      n_{\ce{O_2}}^{0}+\xi_{\max}  & = & 0
			      \end{array}
			      \right.
			      \Longleftrightarrow
			      \left\{
			      \begin{array}{rcl}
				      \xi_{\max} & = & -\SI{0.025}{mol} \\
				      \xi_{\max} & = & -\SI{0.050}{mol}
			      \end{array}
			      \right.
		      \end{gather*}
		      Le seul $\xi_{\max}$ possible est le plus petit \textbf{en
			      valeur absolue}, c'est-à-dire \xul{$\xi_{\max} = -\SI{0.025}{mol}$}.
              \bigbreak
          La valeur est finale $\xi_f$ est la plus petite valeur \textbf{en
          valeur absolue} de $\xi_{\eq}$ et $\xi_{\max}$~; or ici
          $\abs{\xi_{\eq}} > \abs{\xi_{\max}}$~: il n'y a donc \textbf{pas
          équilibre}, et on a
		      \[\xul{\xi_f = \xi_{\max} = -\SI{0.025}{mol}}\]
		\item Voir tableau.
	\end{enumerate}
}

\section{Combinaisons de réactions et constantes d'équilibre}
\resetQ
\QR{%
	On considère les réactions numérotées $(1)$ et $(2)$ ci-dessous~:
	\[\ce{4Cu\sol{} + O2\gaz{} = 2Cu2O\sol{}}\quad K_1\degree
		\qqet
		\ce{2Cu2O\sol{} + O2\gaz{} = 4CuO\sol{}}\quad K_2\degree\]
	Exprimer les constantes d'équilibre des trois réactions ci-dessous en fonction
	de $K_1\degree$ et $K_2\degree$~:
	\[
		\ce{2Cu\sol{} + O2\gaz{} = 2CuO\sol{}}
    \quad ; \quad 
		\ce{8Cu\sol{} + 2O2\gaz{} = 3Cu2O\sol{}}
    \quad ; \quad 
		\ce{2CuO\sol{} = 4Cu\sol{}+ O2\gaz{}}
	\]
}{%
	Dans cet exercice, on introduit le lien entre relation sur les équations-bilan
	et les constantes d'équilibre associées. En effet, on a vu dans le cours que
	\[a{\ce{A}} + b{\ce{B}} = c{\ce{C}} + d{\ce{D}}\]
	a pour constante d'équilibre
	\[K_1\degree = \prod_i a({\ce{X}}_i)^{\nu_{i, 1}}\]
	Si on inverse la réaction pour avoir
	\[c{\ce{C}} + d{\ce{D}} = a{\ce{A}} + b{\ce{B}}\]
	alors on prend l'opposé de chaque coefficient stœchiométrique~: $\nu_{i, 2} =
		- \nu_{i,1}$, ce qui fait que cette réaction a pour constante d'équilibre
	\[
		K_2\degree
		= \prod_i a({\ce{X}}_i)^{\nu_{i,2}}
		= \prod_i a({\ce{X}}_i)^{-\nu_{i,1}}
		= \left(\prod_i a({\ce{X}}_i)^{\nu_{i,1}}\right)^{-1}
		= \left( K_1\degree \right)^{-1}
	\]

	Le même raisonnement tient pour montrer que
	\[2a{\ce{A}} + 2b{\ce{B}} = 2c{\ce{C}} + 2d{\ce{D}}\]
	a pour constante d'équilibre
	\[K_3\degree = K_1\degree^2\]

	On étend le raisonnement pour montrer que si on ajoute deux réactions (1) et
	(2) pour avoir une équation (3), alors on aura $K_3\degree = K_1\degree\times
		K_2\degree$, et que si on a $(3) = \alpha(1) + \beta(2)$, alors
	$K_3\degree = K_1\degree^{\alpha}\times K_2\degree^{\beta}$.
	\bigbreak
	Ainsi, dans cet exercice il suffit de trouver les relations entre les
	équations (3), (4), (5) et les équations (1) et (2) de constantes respectives
	$K_1\degree$ et $K_2\degree$. On trouve alors~:
	\begin{enumerate}[leftmargin=20pt, label=\alph* --]
		\item
		      \[
			      (3)
			      = \frac{(1)+(2)}{2}
			      \Lra
			      K_3\degree
			      = \left(K_1\degree\times K_2\degree\right)^{1/2}
			      = \sqrt{K_1\degree K_2\degree}
		      \]
		\item
		      \[
			      (4) = 2(1)
			      \Lra
			      K_4\degree
			      = \left( K_1\degree \right)^2
		      \]
		\item
		      \[
			      (5)
			      = -(1)
			      \Lra
			      K_5\degree
			      = \left( K_1\degree \right)^{-1}
		      \]
	\end{enumerate}
	Tout ceci se vérifie bien sûr en écrivant les constantes de chacune des
	réactions~:
	\[
		K_1\degree = \frac{p\degree}{p_{\ce{O2}}}
    \quad ; \quad 
		K_2\degree = \frac{p\degree}{p_{\ce{O2}}}
    \quad ; \quad 
		K_3\degree = \frac{p\degree}{p_{\ce{O2}}}
    \quad ; \quad 
		K_4\degree = \frac{p\degree^2}{p_{\ce{O2}}{}^2}
    \quad ; \quad 
		K_5\degree = \frac{p_{\ce{O2}}}{p\degree}
	\]
}

\section{Équilibre avec des solides}
\resetQ
\enonce{%
	La chaux vive, solide blanc de formule $\ce{CaO\sol{}}$, est un des produits
	de chimie industrielle les plus communs. Utilisée depuis l'Antiquité,
	notamment dans le domaine de la construction, elle est aujourd'hui utilisée
	comme intermédiaire en métallurgie. Elle est obtenue industriellement par
	dissociation thermique du calcaire dans un four à $T = \SI{1100}{K}$. On
	modélise cette transformation par la réaction d'équation~:

	\centersright{%
		$\ce{CaCO3\sol{} = CaO\sol{} + CO2 \gaz{}}$
	}{%
		$K\degree(\SI{1100}{K}) = \num{0.358}$
	}
}

\QR{%
	Dans un récipient de volume $V = \SI{10}{L}$ préalablement vide, on
	introduit \SI{10}{mmol} de calcaire à température constante $T =
		\SI{1100}{K}$. Déterminer le sens d'évolution du système chimique.
}{%
	Comme on ne part que de calcaire, la réaction \textbf{ne peut avoir
		lieu que dans le sens direct}. On vérifie cette intuition en calculant
	$Q_{r,0}$ pour le comparer à $K$, sachant qu'on part d'un récipient vide
	de gaz au début~:
	\[\boxed{Q_{r,0} = \frac{p_{\ce{CO2}, 0}}{p\degree} = 0 < K\degree}\]
	La réaction se fait bien dans le sens direct.
}

\QR{%
	Supposons que l'état final est un état d'équilibre. Déterminer la
	quantité de matière de calcaire qui devrait avoir réagi. Conclure sur
	l'hypothèse faite.
}{%
	Si l'état final est un état d'équilibre, alors avec l'équation
	précédente on aura
	\begin{gather*}
		p_{\ce{CO2}, \eq}
		= K\degree p\degree\\
		\Longleftrightarrow
		n_{\ce{O2}, \eq}
		= \frac{p_{\ce{O2},\eq}V}{RT}
		= \frac{Kp\degree V}{RT}
	\end{gather*}
	Or, un tableau d'avancement donne que $n_{\ce{O2}, \eq} = \xi_{\eq}$~; on
	trouve donc
	\begin{gather*}
		\boxed{\xi_{\eq} = \frac{K\degree p\degree V}{RT}}
		\qavec
		\left\{
		\begin{array}{rcl}
			K        & = & \num{0.358}              \\
			V        & = & \SI{10e-3}{m^3}          \\
			T        & = & \SI{1100}{K}             \\
			p\degree & = & \SI{1}{bar}              \\
			R        & = & \SI{8.314}{J.mol.K^{-1}}
		\end{array}
		\right.\\
		\mathrm{A.N.~:}\quad
		\xul{\xi_{\eq} = \SI{39}{mmol}}
	\end{gather*}
	Pour savoir si cette valeur est réalisable, on calcule $\xi_{\max}$ que
	l'on trouverait si le calcaire était limitant, c'est-à-dire en résolvant
	$\num{10} - \xi_{\max} = 0$~: on trouve naturellement \xul{$\xi_{\max}
			= \SI{10}{mmol}$}.
	\bigbreak
	On sait que la valeur de $\xi_f$ est la plus petite valeur absolue entre
	$\xi_{\eq}$ et $\xi_{\max}$. Or, ici on trouve $\xi_{\eq} >
		\xi_{\max}$, ce qui veut dire que \textbf{l'équilibre ne peut être
		atteint} et qu'on aura ainsi
	\[\xul{\xi_f = \xi_{\max} = \SI{10}{mmol}}\]
	c'est-à-dire que \textbf{la réaction est totale}. On peut donc remplir
	la dernière ligne du tableau d'avancement.
	\begin{center}
		\def\rhgt{0.35}
		\centering
		\begin{tabularx}{\linewidth}{|l|c||YdYdY||Y|}
			\hline
			\multicolumn{2}{|c||}{
				$\xmathstrut{\rhgt}$
			\textbf{Équation}}  &
			$\ce{CaCO_3\sol{}}$ & $=$                  &
			$\ce{CaO\sol{}}$    & $+$                  &
			$\ce{CO_2\gaz{}}$   &
			$n_{\tot, gaz}$                              \\
			\hline
			$\xmathstrut{\rhgt}$
			Initial (\si{mmol}) & $\xi = 0$            &
			$10$                & \vline               &
			$0$                 & \vline               &
			$0$                 &
			$0$                                          \\
			\hline
			$\xmathstrut{\rhgt}$
			Interm. (\si{mmol}) & $\xi$                &
			$10 - \xi$          & \vline               &
			$\xi$               & \vline               &
			$\xi$               &
			$\xi$                                        \\
			\hline
			$\xmathstrut{\rhgt}$
			Final  (\si{mmol})  & $\xi_f = \xi_{\max}$ &
			$0$                 & \vline               &
			$10$                & \vline               &
			$10$                &
			$10$                                         \\
			\hline
		\end{tabularx}
	\end{center}
}
\QR{%
	Si on part de \SI{50}{mmol} de calcaire, quelle est la quantité de chaux
	obtenue~? Comment faire pour augmenter la quantité de chaux produite~?
}{%
	En ne partant que de calcaire, dès que $\xi_f$ atteint $\xi_{\eq}$ la réaction
	s'arrêtera puisqu'on aura atteint l'équilibre. Mettre plus de calcaire ne
	formera pas plus de chaux, l'excédent de réactif initial ne réagira simplement
	pas. Ainsi, \textbf{la quantité de matière de calcaire maximale qui puisse
		être transformée est de $\xi_{\max} = \SI{39}{mmol}$}.
	\bigbreak
	Pour déplacer l'équilibre dans le sens direct, il faut diminuer la quantité de
	$\ce{CO_2\gaz{}}$~: on diminue alors $Q_r$ qui peut repasser en-dessous de
	$K\degree$. Il suffit pour ça de \textbf{travailler en volume ouvert} ou
	\textbf{d'aspirer le $\ce{CO_2\gaz{}}$}.
}

\section{Équilibre en solution aqueuse}
\resetQ
\enonce{%
	Considérons un système de volume \SI{20}{mL} évoluant selon la réaction
	d'équation bilan~:

	\centersright{%
		$\ce{CH3COOH\aqu{} + F^{-}\aqu{} <=> CH3COO^{-}\aqu{} + HF\aqu{}}$
	}{%
		$K\degree(\SI{25}{\degreeCelsius}) = \num{e-1.60}$
	}

	Déterminer le sens d'évolution du système et l'avancement à l'équilibre en
	partant des deux situations initiales suivantes~:
}

\QR{%
$[\ce{CH3COOH}]_0 = [\ce{F^{-}}]_0 = c = \SI{0.1}{mol.L^{-1}}$ et
$[\ce{CH3COO^{-}}]_0 = [\ce{HF}]_0 = 0$
}{%
Pour déterminer le sens d'évolution du système, on calcule $Q_{r,0}$
et on le compare à $K\degree$~:
\[\boxed{
	Q_{r,0} = \frac{[\ce{CH3COO^{-}}]_0[\ce{HF}]_0}
	{[\ce{CH3COOH}]_0[\ce{F^{-}}]_0} = 0 < K\degree}
\]
La réaction évoluera donc \textbf{dans le sens direct}.
\bigbreak
Pour trouver l'avancement à l'équilibre, on dresse le tableau
d'avancement, que l'on peut directement faire en concentrations puisque
le volume ne varie pas (ce qui est toujours le cas cette année)~:
\begin{center}
	\def\rhgt{0.35}
	\centering
	\begin{tabularx}{\linewidth}{|l|c||YdYdYdY|}
		\hline
		\multicolumn{2}{|c||}{
			$\xmathstrut{\rhgt}$
		\textbf{Équation}}       &
		$\ce{CH_3COOH\aqu{}}$    & $+$             &
		$\ce{F^{-}\aqu{}}$       & $=$             &
		$\ce{CH_3COO^{-}\aqu{}}$ & $+$             &
		$\ce{HF\aqu{}}$                              \\
		\hline
		$\xmathstrut{\rhgt}$
		Initial                  & $x = 0$         &
		$c$                      & \vline          &
		$c$                      & \vline          &
		$0$                      & \vline          &
		$0$                                          \\
		\hline
		$\xmathstrut{\rhgt}$
		Final                    & $x_f = x_{\eq}$ &
		$c - x_{\eq}$          & \vline          &
		$c - x_{\eq}$          & \vline          &
		$x_{\eq}$              & \vline          &
		$x_{\eq}$                                  \\
		\hline
	\end{tabularx}
\end{center}
D'après la loi d'action des masses, on a
\begin{gather*}
	K\degree = \frac{x_{\eq}{}^2}{(c-x_{\eq})^2}
	\Longleftrightarrow
	\sqrt{K\degree} = \frac{x_{\eq}}{c-x_{\eq}}
	\Longleftrightarrow
	x_{\eq} = \sqrt{K\degree}(c-x_{\eq})\\
	\Longleftrightarrow
	\boxed{x_{\eq} = \frac{\sqrt{K\degree}}{1+ \sqrt{K\degree}}c}
	\qavec
	\left\{
	\begin{array}{rcl}
		K\degree & = & \num{e-1.60}         \\
		c        & = & \SI{0.1}{mol.L^{-1}}
	\end{array}
	\right.\\
	\mathrm{A.N.~:}\quad
	\xul{x_{\eq} = \SI{1.4e-2}{mol.L^{-1}}}
\end{gather*}
En encadrant le résultat, on vérifie la cohérence physico-chimique de la
réponse~: ici c'est bien cohérent de trouver $x_{\eq} > 0$ puisqu'on
avait déterminé que la réaction se faisait dans le sens direct.
}
\QR{%
$[\ce{CH3COOH}]_0 = [\ce{F^{-}}]_0 = [\ce{CH3COO^{-}}]_0 =
	[\ce{HF}]_0 = c = \SI{0.1}{mol.L^{-1}}$
}{%
De la même manière, pour déterminer le sens d'évolution du système, on
calcule $Q_{r,0}$ et on le compare à $K$~:
\[\boxed{
	Q_{r,0} = \frac{[\ce{CH3COO^{-}}]_0[\ce{HF}]_0}
	{[\ce{CH3COOH}]_0[\ce{F^{-}}]_0}
	= \frac{c^2}{c^2} = 1 > K\degree}
\]
La réaction évoluera donc \textbf{dans le sens indirect}.
\bigbreak
On effectue un bilan de matière grâce à un tableau d'avancement~:
\begin{center}
	\def\rhgt{0.35}
	\centering
	\begin{tabularx}{\linewidth}{|l|c||YdYdYdY|}
		\hline
		\multicolumn{2}{|c||}{
			$\xmathstrut{\rhgt}$
		\textbf{Équation}}       &
		$\ce{CH_3COOH\aqu{}}$    & $+$             &
		$\ce{F^{-}\aqu{}}$       & $=$             &
		$\ce{CH_3COO^{-}\aqu{}}$ & $+$             &
		$\ce{HF\aqu{}}$                              \\
		\hline
		$\xmathstrut{\rhgt}$
		Initial                  & $x = 0$         &
		$c$                      & \vline          &
		$c$                      & \vline          &
		$c$                      & \vline          &
		$c$                                          \\
		\hline
		$\xmathstrut{\rhgt}$
		Final                  & $x_f = x_{\eq}$ &
		$c - x_{\eq}$          & \vline          &
		$c - x_{\eq}$          & \vline          &
		$c + x_{\eq}$          & \vline          &
		$c + x_{\eq}$                              \\
		\hline
	\end{tabularx}
\end{center}
D'après la loi d'action des masses, on a
\begin{gather*}
	K\degree = \frac{(c+x_{\eq})^2}{(c-x_{\eq})^2}
	\Longleftrightarrow
	\sqrt{K\degree} = \frac{c+x_{\eq}}{c-x_{\eq}}
	\Longleftrightarrow
	c+x_{\eq} = \sqrt{K\degree}(c-x_{\eq})\\
	\Longleftrightarrow
	\boxed{x_{\eq} = \frac{\sqrt{K\degree}-1}{\sqrt{K\degree}+1}c}
	\qavec
	\left\{
	\begin{array}{rcl}
		K\degree & = & \num{e-1.60}         \\
		c        & = & \SI{0.1}{mol.L^{-1}}
	\end{array}
	\right.\\
	\mathrm{A.N.~:}\quad
	\xul{x_{\eq} = -\SI{5.3e-2}{mol.L^{-1}}}
\end{gather*}
De même que précédemment, on vérifie qu'il est logique de trouver
$x_{\eq} < 0$~: la réaction se fait bien dans le sens indirect.
}


\section{Équilibre en phase gazeuse}
\resetQ
\enonce{%
	On étudie en phase gazeuse l'équilibre de dimérisation de \ce{FeCl3}, de
	constante d'équilibre $K\degree(T)$ à une température $T$ donnée et
	d'équation-bilan
	\[\ce{2FeCl3\gaz{} = Fe2Cl6\gaz{}}\]

	La réaction se déroule sous une pression totale constante $p_{\tot} =
		2p\degree = \SI{2}{bars}$. À la température $T_1 = \SI{750}{K}$, la constante
	d'équilibre vaut $K\degree(T_1) = \num{20.8}$. Le système est maintenu à la
	température $T_1 = \SI{750}{K}$. Initialement le système contient $n_0$ moles
	de \ce{FeCl3} et de \ce{Fe2Cl6}. Soit $n_{\tot}$ la quantité totale de matière
	d'espèces dans le système.
}

\QR{%
	Exprimer la constante d'équilibre en fonction des pressions partielles
	des constituants à l'équilibre et de $p\degree$.
}{%
	On peut dresser le tableau d'avancement initial dans cette situation~:
	\begin{center}
		\def\rhgt{0.35}
		\centering
		\begin{tabularx}{.7\linewidth}{|l|c||YdY||Y|}
			\hline
			\multicolumn{2}{|c||}{
				$\xmathstrut{\rhgt}$
			\textbf{Équation}}    &
			$2\ce{FeCl_3\gaz{}}$  & $=$       &
			$\ce{Fe_2Cl_6\gaz{}}$ &
			$n_{\tot, gaz}$                     \\
			\hline
			$\xmathstrut{\rhgt}$
			Initial               & $\xi = 0$ &
			$n_0$                 & \vline    &
			$n_0$                 &
			$2n_0$                              \\
			\hline
		\end{tabularx}
	\end{center}
	Par la loi d'action des masses et les activités de constituants
	gazeux~:
	\[
		\boxed{
			K\degree = \frac{p_{\ce{Fe2Cl6}}p\degree}{p_{\ce{FeCl3}}{}^2}
		}
	\]
}
\QR{%
Exprimer le quotient de réaction $Q_r$ en fonction de la quantité de
matière de chacun des constituants, de la pression totale $p_{\tot}$
et de $p\degree$. Calculer la valeur initial $Q_{r,0}$ du quotient de
réaction.
}{%
Pour passer des pressions partielles aux quantités de matière, on
utilise la loi de \textsc{Dalton}~:
\begin{tcb}(rapp){Rappel~: loi de \textsc{Dalton}}
	Soit un mélange de gaz parfaits de pression $P$. Les pressions
	partielles $P_i$ de chaque constituant $\mathrm{X}_i$ s'exprime
	\[\boxed{P_i = x_iP}\]
	avec $x_i$ la fraction molaire du constituant~:
	\[\boxed{x_i = \frac{n_i}{n_{\tot}}}\]
\end{tcb}
On écrit donc
\[
	p_{\ce{Fe2Cl6}} = \frac{n_{\ce{Fe2Cl6}}}{n_{\tot}}\times p_{\tot}
	\qquad
	p_{\ce{FeCl3}} = \frac{n_{\ce{FeCl3}}}{n_{\tot}}\times p_{\tot}
\]

Pour simplifier l'écriture, on peut séparer les termes de pression
totale des termes de matière en comptant combien vont arriver «~en
haut~» et combien «~en bas~»~: 1 en haut contre 2 en bas, on se
retrouvera avec $p_{\tot}$ au dénominateur, ce qui est logique par
homogénéité vis-à-vis de $p\degree$ qui reste au numérateur. Comme
$n_{\tot}$ apparaît le même nombre de fois que $p_{\tot}$ mais
avec une puissance -1, on sait aussi qu'il doit se retrouver au
numérateur, là aussi logiquement pour avoir l'homogénéité vis-à-vis de
la quantité de matière. Ainsi,

\[\boxed{
	Q_r = \frac{n_{\ce{Fe2Cl6}}/\cancel{n_{\tot}}
	\times\bcancel{p_{\tot}}}
	{n_{\ce{FeCl3}}{}^2/n_{\tot}^{\cancel{2}}
	\times p_{\tot}^{\bcancel{2}}} p\degree
	= \frac{n_{\ce{Fe2Cl6}} n_{\tot}}
	{n_{\ce{FeCl3}}{}^2} \frac{p\degree}{p_{\tot}}
	}
\]
Avec $p_{\tot} = 2p\degree$ et $n_{\ce{Fe2Cl6}} = n_0 =
	n_{\ce{FeCl3}}$, on a $n_{\tot} = 2n_0$ (cf.\ tableau d'avancement),
d'où

\[
	Q_{r,0} = \frac{n_0\times2n_0}{n_0{}^2} \frac{1}{2}
	\Leftrightarrow
	\xul{Q_{r,0} = 1}
\]
}
\QR{%
	Le système est-il initialement à l'équilibre thermodynamique~?
	Justifier la réponse. Si le système n'est pas à l'équilibre, dans quel
	sens se produira l'évolution~?
}{%
	Le système serait à l'équilibre si $Q_{r,0} = K\degree$~; or, ici
	$Q_{r,0} \neq K\degree$, donc l'équilibre n'est pas atteint. De plus, $Q_{r,0}
		< K\degree$ donc le système évoluera dans le sens direct.
}
\enonce{%
	On considère désormais une enceinte indéformable, de température
	constante $T_1 = \SI{750}{K}$, initialement vide. On y introduit une
	quantité $n$ de \ce{FeCl3} gazeux et on laisse le système évoluer de
	telle sorte que la pression soit maintenu constante et égale à $p =
		2p\degree = \SI{2}{bars}$. On désigne par $\xi$ l'avancement de la
	réaction.
}
\QR{%
	Calculer à l'équilibre la valeur du rapport $z = \xi/n$.
}{%
	On dresse le tableau d'avancement pour effectuer un bilan de matière
	dans cette nouvelle situation~:
	\begin{center}
		\def\rhgt{0.35}
		\centering
		\begin{tabularx}{.7\linewidth}{|l|c||YdY||Y|}
			\hline
			\multicolumn{2}{|c||}{
				$\xmathstrut{\rhgt}$
			\textbf{Équation}}    &
			$2\ce{FeCl_3\gaz{}}$  & $=$           &
			$\ce{Fe_2Cl_6\gaz{}}$ &
			$n_{\tot, gaz}$                         \\
			\hline
			$\xmathstrut{\rhgt}$
			Initial               & $\xi = 0$     &
			$n$                   & \vline        &
			$0$                   &
			$n$                                     \\
			\hline
			$\xmathstrut{\rhgt}$
			Final                 & $\xi = \xi_f$ &
			$n-2\xi$              & \vline        &
			$\xi$                 &
			$n-\xi$                                 \\
			\hline
		\end{tabularx}
	\end{center}
	On reprend l'expression du quotient réactionnel initial en
	remplaçant les quantités de matière par leur expression selon $\xi$ pour
	déterminer l'avancement à l'équilibre, décrit par $K\degree$~:
	\begin{gather*}
		K\degree = \frac{\xi(n-\xi)}{(n-2\xi)^2}
		\underbrace{\frac{p\degree}{p_{\tot}}}_{=\frac{1}{2}}
		\Leftrightarrow
		K\degree = \underbrace{\cancel{\frac{n^2}{n^2}}}_{=1}
		\frac{\xi/n(1-\xi/n)}
		{\left( 1 - 2\xi/n \right)^2} \frac{1}{2}
	\end{gather*}
	Pour simplifier les calculs, posons $z = \frac{\xi}{n}$. L'équation
	précédente devient~:
	\begin{align*}
		K\degree                                                & = \frac{1}{2} \frac{z(1-z)}{\left( 1 - 2z \right)^2}
		\\\Lra
		2K\degree(1-2z)^2                                       & = z(1-z)
		\\\Lra
		2K\degree(1-4z + 4z^2)                                  & = z - z^2
		\\\Lra
		\Aboxed{z^2(8K\degree +1) - z(8K\degree +1) + 2K\degree & = 0}
	\end{align*}
	On trouve un polynôme du second degré. Soit $\Delta$ son discriminant~:
	\begin{align*}
		\Delta         & = \left( 8K\degree+1 \right)^{2} -
		4\left(8K\degree+1\right)\times 2K\degree
		\\\Lra
		\Delta         & = \left( 8K\degree+1 \right)\left(
		\cancel{8K\degree}+1-\cancel{8K\degree} \right)
		\\\Lra
		\Aboxed{\Delta & = 8K\degree+1}
		\qav
		\left\{
		\begin{array}{rcl}
			K\degree & = & \num{20.8}
		\end{array}
		\right.                                             \\
		\makebox[0pt][l]{$\phantom{\AN}\xul{\phantom{\Delta = \num{167.4}}}$}
		\AN
		\Delta         & = \num{167.4}
	\end{align*}
	Les racines sont
	$\DS\left\{
		\begin{array}{rcl}
			z_1 & = & \num{0.54} \\
			z_2 & = & \num{0.46}
		\end{array}
		\right.$.
	\bigbreak
	Étant donné qu'on part de $\xi = 0$ et que $\xi$ augmente, la valeur que
	prendrait $z_{\eq}$ serait $z_{\eq} = \num{0.46}$. On doit cependant
	vérifier que cette valeur est bien possible, en déterminant $z_{\max}$~:
	pour cela, on résout $n-2\xi = 0$, ce qui donne $z_{\max} = \num{0.5}$.
	On a bien $z_{\eq} < z_{\max}$, donc \textbf{l'équilibre est atteint} et
	on a \xul{$\xi/n = \num{0.46}$}.
}

\section{Transformations de gaz}
\resetQ
\QR{%
On considère l'équilibre suivant~:

\centersright{%
	$\DS\ce{H2S\gaz{} + \frac{3}{2}O2\gaz{} = H2O\gaz{} + SO2\gaz{}}$
}{%
	$K_1\degree$
}

Donner l'expression de la constante d'équilibre $K_1\degree$. En
supposant les réactifs introduits dans les proportions stœchiométriques,
faire un bilan de matière à l'équilibre. Exprimer $K_1\degree$ en
fonction de $\xi_{\eq}$.
}{%
Par la loi d'action des masses, on a
\begin{gather*}
	K_1\degree =
	\frac{p_{\ce{SO2}}p_{\ce{H2O}}}{p_{\ce{O2}}^{3/2}p_{\ce{H2S}}}
	\underbrace{\frac{p\degree^{5/2}}{p\degree^2}}_{= p\degree^{1/2}}
	\Leftrightarrow
	K_1\degree =
	\frac{n_{\ce{H2O}}n_{\ce{SO2}}n_{\tot,
	gaz}^{1/2}}{n_{\ce{H2S}}n_{\ce{O2}}{}^{3/2}}
	\left(\frac{p\degree}{p}\right)^{1/2}
\end{gather*}
Soit $n_0$ la quantité de matière de $\ce{H2S}$ introduite
initialement~: pour que $\ce{O2}$ soit introduit dans les proportions
stœchiométriques, on relie sa quantité initiale à celle de $\ce{H2S}$
\textit{via} les coefficients stœchiométriques tel que $n_{\ce{O2}}^0 =
	\frac{3}{2}n_{\ce{H2S}}^0 = \frac{3}{2}n_0$. D'où le tableau
d'avancement~:
\begin{center}
	\def\rhgt{0.35}
	\centering
	\begin{tabularx}{\linewidth}{|l|c||YdYdYdY||Y|}
		\hline
		\multicolumn{2}{|c||}{
			$\xmathstrut{\rhgt}$
		\textbf{Équation}}                      &
		$\ce{H_2S\gaz{}}$                       & $+$                 &
		$\frac{3}{2}\ce{O_2\gaz{}}$             & $=$               &
		$\ce{H_2O\gaz{}}$                       & $+$                 &
		$\ce{SO_2\gaz{}}$                       &
		$n_{\tot, gaz}$                                                 \\
		\hline
		$\xmathstrut{\rhgt}$
		Initial                                 & $\xi = 0$           &
		$n_0$                                   & \vline              &
		$\frac{3}{2}n_0$                        & \vline              &
		$0$                                     & \vline              &
		$0$                                     &
		$\frac{5}{2}n_0$                                                \\
		\hline
		$\xmathstrut{\rhgt}$
		Final                                   & $\xi_f = \xi_{\eq}$ &
		$n_0 - \xi_{\eq}$                       & \vline              &
		$\frac{3}{2}n_0 - \frac{3}{2}\xi_{\eq}$ & \vline              &
		$\xi_{\eq}$                             & \vline              &
		$\xi_{\eq}$                             &
		$\frac{5}{2}n_0 - \frac{1}{2}\xi_{\eq}$                         \\
		\hline
	\end{tabularx}
\end{center}
On peut donc remplacer les quantités de matière de l'expression de
$K_1\degree$ par les expressions avec $\xi_{\eq}$~:
\[\boxed{
		K_1\degree =
		\frac{\xi_{\eq}^2( \frac{5}{2}n_0- \frac{1}{2}\xi_{\eq})^{1/2}}
		{(n_0-\xi_{\eq})( \frac{3}{2}n_0 - \frac{3}{2}\xi_{\eq})^{3/2}}
		\left(\frac{p\degree}{p}\right)^{1/2}}
\]
}

\QR{%
	On considère l'équilibre suivant~:

	\centersright{%
		$\DS\ce{2H2S\gaz{} + SO2\gaz{} = 2H2O\gaz{} + 3S\liq{}}$
	}{%
		$K_2\degree$}

	Donner l'expression de la constante d'équilibre $K_2\degree$. On
	introduit les réactifs avec des quantités quelconques. Faire un bilan de
	matière à l'équilibre. Exprimer $K_2\degree$ en fonction de $\xi_{\eq}$.
}{%
	Par la loi d'action des masses, on a
	\begin{gather*}
		K_2\degree =
		\frac{p_{\ce{H2O}}{}^2p\degree}
    {p_{\ce{SO2}}p_{\ce{H2S}}{}^2}
		\Lra
		K_1\degree =
		\frac{n_{\ce{H2O}}{}^2n_{\tot, gaz}}
    {n_{\ce{SO2}}n_{\ce{H2S}}{}^2}
		\frac{p\degree}{p}
	\end{gather*}
	Soit $n_1$ la quantité de matière de $\ce{H2S}$ introduite initialement,
	et $n_2$ la quantité de matière initiale en $\ce{SO2}$~:
	\begin{center}
		\def\rhgt{0.35}
		\centering
		\begin{tabularx}{\linewidth}{|l|c||YdYdYdY||Y|}
			\hline
			\multicolumn{2}{|c||}{
				$\xmathstrut{\rhgt}$
			\textbf{Équation}} &
			$2\ce{H_2S\gaz{}}$ & $+$           &
			$\ce{SO_2\gaz{}}$  & $=$         &
			$2\ce{H_2O\gaz{}}$ & $+$           &
			$3\ce{S\liq{}}$    &
			$n_{\tot, gaz}$                      \\
			\hline
			$\xmathstrut{\rhgt}$
			Initial            & $\xi = 0$     &
			$n_1$              & \vline        &
			$n_2$              & \vline        &
			$0$                & \vline        &
			$0$                &
			$n_1+n_2$                            \\
			\hline
			$\xmathstrut{\rhgt}$
			Final              & $\xi_f = \xi_{\eq}$ &
			$n_1 - 2\xi_{\eq}$     & \vline        &
			$n_2 - \xi_{\eq}$      & \vline        &
			$0 + 2\xi_{\eq}$       & \vline        &
			$0 + 3\xi_{\eq}$       &
			$n_1+n_2-\xi_{\eq}$                  \\
			\hline
		\end{tabularx}
	\end{center}
	On peut donc remplacer les quantités de matière de l'expression de
	$K_1\degree$ par les expressions avec $\xi_{\eq}$~:
	\[\boxed{
			K_1\degree =
			\frac{4\xi_{\eq}^2(n_1+n_2-\xi_{\eq})}
			{(n_1-2\xi_{\eq})^2(n_2-\xi_{\eq})}
			\frac{p\degree}{p}}
	\]
}

\QR{%
	On fait brûler du méthane dans de l'oxygène~:
	\[
		\ce{\ldots CH4\gaz{} + \ldots O2\gaz{}
			-> \ldots CO2\gaz{} + \ldots H2O\liq{}}
	\]
	Équilibrer l'équation de la réaction. Elle peut être considérée
	comme totale. On introduit les réactifs de façon à consommer la moitié
	du dioxygène. Décrire l'état final du système.
}{%
	L'équation bilan équilibrée est~:
	\[\ce{CH4\gaz{} + 2O2\gaz{} \rightarrow CO2\gaz{} + 2H2O\liq{}}\]
	Soit $n_0$ la quantité initiale en dioxygène. Si la moitié seulement est
	consommée, alors que la réaction est totale, c'est que le méthane est
	limitant. On trouve la quantité de $\ce{CH4}$ à introduire initialement
	en dressant le tableau d'avancement pour que $n_{\ce{H2O},f} =
		\frac{1}{2}n_0$, c'est-à-dire $n_0 - 2\xi_{\max} = 0$~: on obtient
	\[\boxed{\xi_{\max} = \frac{1}{4}n_0}
		\qor
		n_{\ce{CH4}}^0 - \xi_{\max} = 0
		\qdonc
		\boxed{n_{\ce{CH4}}^0 = \frac{1}{4}n_0}
	\]
	\begin{center}
		\def\rhgt{0.35}
		\centering
		\begin{tabularx}{\linewidth}{|l|c||YdYdYdY||Y|}
			\hline
			\multicolumn{2}{|c||}{
				$\xmathstrut{\rhgt}$
			\textbf{Équation}}     &
			$\ce{CH_4\gaz{}}$      & $+$                  &
			$2\ce{O_2\gaz{}}$      & $\ra$                &
			$\ce{CO_2\gaz{}}$      & $+$                  &
			$2\ce{H_2O\liq{}}$     &
			$n_{\tot, gaz}$                                 \\
			\hline
			$\xmathstrut{\rhgt}$
			Initial                & $\xi = 0$            &
			$\frac{1}{4}n_0$       & \vline               &
			$n_0$                  & \vline               &
			$0$                    & \vline               &
			$0$                    &
			$\frac{5}{4}n_0$                                \\
			\hline
			$\xmathstrut{\rhgt}$
			Interm.                & $\xi$                &
			$\frac{1}{4}n_0 - \xi$ & \vline               &
			$n_0 - 2\xi$           & \vline               &
			$\xi$                  & \vline               &
			$2\xi$                 &
			$\frac{5}{4}n_0 - 2\xi$                         \\
			\hline
			$\xmathstrut{\rhgt}$
			Final                  & $\xi_f = \xi_{\max}$ &
			$0$                    & \vline               &
			$\frac{1}{2}n_0$       & \vline               &
			$\frac{1}{4}n_0$       & \vline               &
			$\frac{1}{2}n_0$       &
			$\frac{3}{4}n_0$                                \\
			\hline
		\end{tabularx}
	\end{center}
}

\section{Coefficient de dissociation}
\resetQ
\enonce{%
  On considère l'équilibre de l'eau en phase gazeuse~:
  \[
    \ce{2H2O = 2H2\gaz{} + O2\gaz{}}
  \]
}

\QR{%
  On se place à \SI{400}{K} sous une pression constante $P = \SI{1.00}{bar}$.
  Sous quelle forme se trouve l'eau~?
}{%
  Pour connaître l'état de l'eau, on détermine la température en degrés
  pour interpréter par des connaissances élémentaires si elle est solide
  (glace), liquide, ou vapeur~: \SI{400}{K} = \SI{127}{\degreeCelsius}, et
  la pression est de \SI{1}{bar}, c'est-à-dire presque la pression
  habituelle. À cette température, l'eau est sous forme vapeur.
}
\QR{%
  La valeur de la constante vaut $K(\SI{400}{K}) = \num{3.12e-59}$. Conclure
  sur la stabilité de l'eau dans ces conditions.
}{%
  La constante est extrêmement petite~: $K \ll \num{e-4}$, donc la
  réaction est \textbf{quasi-nulle} dans ce sens~: l'eau ne se dissocie
  pratiquement pas de cette manière et est par conséquent très stable.
}
\QR{%
  On supposant que l'on introduit de l'eau pure, calculer le coefficient
  de dissociation de l'eau.\\
  \textit{Rappel}~: le coefficient de dissociation $\alpha$ est
  égal au rapport de la quantité ayant été dissociée sur la quantité
  initiale.
}{%
  Si on introduit de l'eau pure, on n'a pas les autres composants au
  départ. Soit $n_0$ la quantité de matière d'eau pure introduite~: on
  dresse le tableau d'avancement~:
  \begin{center}
      \def\rhgt{0.35}
      \centering
      \begin{tabularx}{\linewidth}{|l|c||YdYdY||Y|}
      \hline
      \multicolumn{2}{|c||}{
        $\xmathstrut{\rhgt}$
      \textbf{Équation}} &
      $2\ce{H_2O\gaz{}}$        & $=$                 &
      $2\ce{H_2\gaz{}}$        & $+$                 &
      $\ce{O_2\gaz{}}$ &
      $n_{\tot, gaz}$                              \\
      \hline
      $\xmathstrut{\rhgt}$
      Initial            & $\xi = 0$           &
      $n_0$               & \vline              &
      $0$               & \vline              &
      $0$  &
      $n_0$                                     \\
      \hline
      $\xmathstrut{\rhgt}$
      Final              & $\xi_f = \xi_{\eq}$         &
      $n_0 - 2\xi_{\eq}$     & \vline              &
      $2\xi_{\eq}$     & \vline &
      $\xi_{\eq}$&
      $n_0 + \xi_{\eq}$                           \\
      \hline
      \end{tabularx}
  \end{center}
  Le coefficient de dissociation correspond à la quantité d'eau
  transformée sur la quantité initiale, c'est-à-dire
  \[\alpha = \frac{2\xi_{\eq}}{n_0}\]
  On va donc exprimer la constante d'équilibre en fonction des quantités
  de matière pour introduire $\xi_{\eq}$ et faire apparaître $\alpha$, à l'aide
  de l'activité d'un gaz, de la loi de \textsc{Dalton} puis de la
  définition de la fraction molaire~:
  \begin{gather*}
      K\degree = \frac{p_{\ce{H2}}{}^2p_{\ce{O2}}}
      {p_{\ce{H2O}}{}^2p\degree}
      \Lra
      K\degree = \frac{n_{\ce{H2}}{}^2n_{\ce{O2}}}
        {n_{\ce{H2O}}{}^2n_{\tot, gaz}}
      \overbracket[1pt]{\frac{p}{p\degree}}^{=1}
      \\\Lra
      K\degree = \frac{4\xi_{\eq}^3}{(n_0-2\xi_{\eq})^2(n_0+\xi_{\eq})}
      \Lra
      K\degree = \underbrace{\cancel{\frac{n_0{}^2}{n_0{}^3}}}_{=1}
          \frac{4 \left(\frac{\xi_{\eq}}{n_0}\right)^3}
          {\left( 1- \frac{2\xi_{\eq}}{n_0} \right)^2
          \left( 1+ \frac{\xi_{\eq}}{n_0} \right)}
      \\\Lra
      K\degree = \frac{4 \left(\frac{\alpha}{2}\right)^3}
          {\left( 1- \alpha \right)^2
          \left( 1+ \frac{\alpha}{2} \right)}
      \Lra
      \boxed{
      K\degree = \frac{\alpha^3}
          {\left( 1- \alpha \right)^2
          \left( 2+ \alpha \right)}
  }
  \end{gather*}
  Une résolution numérique (\texttt{Python} ou calculatrice) donne
  \[\boxed{\alpha = \num{3.97e-20} \ll 1}\]
  Ceci est en accord avec le très faible avancement de la réaction.
}
\QR{%
    À \SI{3000}{K}, toujours sous une pression de \SI{1}{bar}, le coefficient de
    dissociation vaut $\alpha = \num{0.30}$. Calculer $K\degree(\SI{3000}{K})$.
    Conclure sur la stabilité de l'eau dans ces conditions.
}{%
  En prenant $\alpha = \num{0.30}$, cela veut dire que 30\% de l'eau se dissocie,
  l'eau ne serait plus stable dans ces conditions. La valeur de $K\degree$
  correspondant est $K\degree = \num{2.4e-2}$, ce qui est peu favorisé dans le
  sens direct mais pas quasi-nulle.
}

\section{Ions mercure}
\resetQ
\enonce{%
  Les ions mercure (II) $\ce{Hg^{+2}}$ peuvent réagir avec le métal liquide
(insoluble dans l'eau) mercure \ce{Hg} pour donner les ions mercure (I)
$\ce{Hg2^{+2}}$ selon l'équilibre chimique ci-dessous~:

\centersright{%
$\ce{Hg^{+2}\aqu{} + Hg\liq{} = Hg2^{+2}\aqu{}}$
}{%
$K\degree(\SI{25}{\degreeCelsius}) = 91$}
}

\QR{%
    Dans quel sens évolue un système obtenu en mélangeant du mercure
        liquide en large excès avec $V_1 = \SI{40.0}{mL}$ d'une solution de
        chlorure de mercure (I) à $c_1 = \SI{1.0e-3}{mol.L^{-1}}$ et $V_2 =
          \SI{10.0}{mL}$ d'une solution de chlorure de mercure (II) à $c_2 =
          \SI{2.0e-3}{mol.L^{-1}}$~?
}{%
On détermine les concentrations en mercure (I) et (II)~:
        \begin{gather*}
            [\ce{Hg^{+2}\aqu{}}] = \frac{c_2V_2}{V_1 + V_2} = c_2' =
            \SI{0.4}{mmol.L^{-1}}
            \qqet
            [\ce{Hg_2^{+2}\aqu{}}] = \frac{c_1V_1}{V_1 + V_2} = c_1' =
            \SI{0.8}{mmol.L^{-1}}
        \end{gather*}
        On peut donc calculer le quotient de réaction initial, avec
        $a(\ce{Hg\liq{}}) )= 1$~:
        \[\boxed{Q_{r,0} = \frac{c_1'}{c_2'} = 2 < K}
        \quad\Longrightarrow\quad
        \text{évolution sens direct}\]
}
\QR{%
    Déterminer la composition finale de la solution.
}{%
        On dresse le tableau d'avancement en concentration~:
        \begin{center}
           \def\rhgt{0.35}
            \centering
           \begin{tabularx}{\linewidth}{|l|c||YdYdY|}
            \hline
            \multicolumn{2}{|c||}{
              $\xmathstrut{\rhgt}$
            \textbf{Équation}}   &
            $\ce{Hg^{2+}\aqu{}}$ & $+$ &
            $\ce{Hg\liq{}}$      & $=$ &
            $\ce{Hg_2^{2+}\aqu}$\\
            \hline
            $\xmathstrut{\rhgt}$
            Initial & $x = 0$ &
            $c_2'$  & \vline  &
            excès   & \vline  &
            $c_1'$              \\
            \hline
            $\xmathstrut{\rhgt}$
            Interm.      & $x$    &
            $c_2' - \xi$ & \vline &
            excès        & \vline &
            $c_1' + \xi$   \\
            \hline
            $\xmathstrut{\rhgt}$
            Final          & $x_f = x_{\eq}$ &
            $c_2' - \xi_f$ & \vline          &
            excès          & \vline          &
            $c_1' + \xi_f$  \\
            \hline
           \end{tabularx}
        \end{center}
        Par la loi d'action des masses, on trouve en effet
        \begin{gather*}
            K\degree = \frac{c_1'+x_{\eq}}{c_2'-x_{\eq}}
            \Leftrightarrow
            \boxed{x_{\eq} = \frac{K\degree c_2' - c_1'}{K\degree + 1} =
            \SI{0.387}{mmol.L^{-1}}}
        \end{gather*}
        ce qui est bien inférieur à $x_{\max} = c_2' = \SI{0.4}{mmol.L^{-1}}$~:
        l'équilibre est atteint.
}

\end{document}
