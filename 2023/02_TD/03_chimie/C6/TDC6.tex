\documentclass[a4paper, 10pt, final, garamond]{book}
\usepackage{cours-preambule}

\makeatletter
\renewcommand{\@chapapp}{Chimie -- chapitre}
\makeatother

\hfuzz=5.002pt

% \toggletrue{student}
\toggletrue{corrige}
% \renewcommand{\mycol}{black}
\renewcommand{\mycol}{gray}

\begin{document}
\setcounter{chapter}{5}

% \settype{enon}
% \settype{solu_prof}
% \settype{solu_stud}

\chapter{\cswitch{Correction du TD}{TD~: Acide-base et précipitation}}

\resetQ
\section{Équations bilan d'oxydorédution}
\enonce{%
  On s'intéresse aux couples $\ce{MnO_4^-}/\ce{Mn^2+}$,
$\ce{HClO_{\rm(aq)}}/\ce{Cl_2}_{\rm(aq)}$ et $\ce{Cl_2}_{\rm(g)}/\ce{Cl^-}$. On
rappelle que $\ce{MnO_4^-}$ est l'ion permanganate et \ce{HClO} l'acide
hypochloreux.
}%

\QR{%
  Écrire et équilibrer les demi-équations de chacun des couples en milieu acide.
}{%
  solu
}%
\QR{%
  Lorsque la réaction est possible, écrire l'équation-bilan de la réaction
  entre~:
  \noindent
  \begin{minipage}[c]{.45\linewidth}
    \begin{enumerate}[label=\alph*)]
      \item L'acide hypochloreux et l'ion manganèse~;
      \item l'ion manganèse et l'ion chlorure~;
      \item l'ion manganèse et le dichlore~;
    \end{enumerate}
  \end{minipage}
  \hfill
  \begin{minipage}[c]{.45\linewidth}
    \begin{enumerate}[label=\alph*)]
      \item le permanganate et le dichlore~;
      \item le permanganate et l'ion chlorure~;
      \item le dichlore sur lui-même.
    \end{enumerate}
  \end{minipage}
}{%
  solu
}%

\resetQ
\section{Nombres d'oxydation du chrome}
\enonce{%
  Le chrome \ce{Cr} a pour numéro atomique $Z = 24$, et il est moins
  électronégatif que l'oxygène.
}%
\QR{%
  Donner le \no du chrome au sein des espèces $\ce{Cr_{\rm(s)}}$, \ce{Cr^2+} et
  \ce{Cr^3+}.
}{%
  solu
}%
\QR{%
  Sans représenter les schémas de \textsc{Lewis}, déterminer le \no du chrome
  dans les espèces \ce{CrO4^2-} et \ce{Cr2O7^2-}. On précise qu'il n'y a pas de
  liaison \ce{Cr-Cr} dans le dichromate.
}{%
  solu
}%
\QR{%
  Justifier que \ce{Cr2O7^2-} et \ce{Cr^3+} forment un couple rédox. Identifier
  l'oxydant et le réducteur sans utiliser la demi-équation. Écrire
  \textbf{ensuite} la demi-équation associée, en milieu acide et en milieu
  basique.
}{%
  solu
}%
\QR{%
  Justifier que \ce{CrO4^2-} et \ce{Cr2O7^2-} ne forment pas un couple rédox.
  Montrer qu'il s'agit cependant d'un couple acide-base par écriture d'une
  demi-équation.
}{%
  solu
}%

\end{document}
