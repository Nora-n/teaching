\documentclass[a4paper, 10pt, final, garamond]{book}
\usepackage{cours-preambule}

\makeatletter
\renewcommand{\@chapapp}{Chimie -- chapitre}
\makeatother

\hfuzz=5.002pt

% \toggletrue{student}
\toggletrue{corrige}
% \renewcommand{\mycol}{black}
\renewcommand{\mycol}{gray}

\begin{document}
\setcounter{chapter}{5}

% \settype{enon}
% \settype{solu_prof}
% \settype{solu_stud}

\chapter{\cswitch{Correction du TD}{TD~: oxydorédution}}

\resetQ
\section{Équations bilan d'oxydorédution}
\enonce{%
On s'intéresse aux couples
$\ce{{MnO_4^{-}}_{\rm(aq)}}/\ce{{Mn}^2+_{\rm(aq)}}$,
$\ce{HClO_{\rm(aq)}}/\ce{{Cl_2}_{\rm(aq)}}$ et
$\ce{{Cl_2}_{\rm(g)}}/\ce{Cl^-_{\rm(aq)}}$. On rappelle que $\ce{MnO_4^-}$ est
l'ion permanganate et \ce{HClO} l'acide hypochloreux.
}%

\QR{%
	Écrire et équilibrer les demi-équations de chacun des couples en milieu acide.
}{%
	\vspace{-15pt}
	\begin{alignat*}{4}
		\beforetext{On obtient}
		\ce{{MnO_4^{-}}_{\rm(aq)} & + 8 {H}^+_{\rm(aq)}      &  & +5e^- &  & = & {Mn}^2+_{\rm(aq)} & + 4 H_2O_{\rm(l)}}
		\\
		\ce{2{HClO}_{\rm(aq)}     & + 2 {H}^+_{\rm(aq)}      &  & +2e^- &  & = & {Cl_2}_{\rm(g)}   & + 2 H_2O_{\rm(l)}}
		\\
		\ce{                      & \wht{+2} {Cl_2}_{\rm(g)} &  & +2e^- &  & = & 2{Cl}^-_{\rm(g)}  & }
	\end{alignat*}
}%
\QR{%
	Lorsque la réaction est possible, écrire l'équation-bilan de la réaction
	entre~:
	\smallbreak
	\noindent
	\begin{minipage}[c]{.45\linewidth}
		\begin{enumerate}[label=\alph*)]
			\item L'acide hypochloreux et l'ion manganèse~;
			\item l'ion manganèse et l'ion chlorure~;
			\item l'ion manganèse et le dichlore~;
		\end{enumerate}
	\end{minipage}
	\hfill
	\begin{minipage}[c]{.45\linewidth}
		\begin{enumerate}[label=\alph*), start=4]
			\item le permanganate et le dichlore~;
			\item le permanganate et l'ion chlorure~;
			\item le dichlore sur lui-même.
		\end{enumerate}
	\end{minipage}
}{%
	\begin{enumerate}[label=\alph*)]
		\item Pas de difficulté~:
		      \[
			      \ce{10 HClO_{\rm(aq)} + 2 {Mn}^2+_{\rm(aq)} = 5 {Cl_2}_{\rm(g)} + 2
			      H_2O_{\rm(l)} + 2 {MnO_4^{-}}_{\rm(aq)} + 6 {H}^+_{\rm(aq)}}
		      \]
		\item Pas de réaction entre $\ce{Mn^2+}$ et $\ce{Cl^-}$ puisque ce sont deux
		      réducteurs.
		\item $\ce{Mn^2+}$ est un réducteur, donc $\ce{Cl_2}$ intervient en tant
		      qu'oxydant~:
		      \[
			      \ce{
			      2 {Mn}^2+_{\rm(aq)} + 8 H_2O_{\rm(l)} + 5{Cl_2}_{\rm(g)} =
			      2 {MnO_4^{-}}_{\rm(aq)} + 16 {H}^+_{\rm(aq)} + 10 {Cl}^-_{\rm(aq)}
			      }
		      \]
		\item Comme $\ce{MnO_4^{-}}$ est un oxydant, $\ce{Cl_2}$ intervient e tant
		      que réducteur~:
		      \[
			      \ce{2 {MnO_4^{-}}_{\rm(aq)} + 6 {H}^+_{\rm(aq)} + 5 {Cl_2}_{\rm(g)} =
			      2 {Mn}^2+_{\rm(aq)} + 10 HClO_{\rm(aq)}}
		      \]
		\item Pas de difficulté~:
		      \[
			      \ce{2 {MnO_4^{-}} + 16 {H}^+_{\rm(aq)} + 10 {Cl}^- _{\rm(aq)} =
			      2 {Mn}^2+_{\rm(aq)} + 8 H_2O_{\rm(l)} + 5 {Cl_2}_{\rm(g)}}
		      \]
		\item Le dichlore intervient en tant qu'oxydant et réducteur, c'est une
		      dismutation~:
		      \[
			      \ce{{Cl_2}_{\rm(g)} + H_2O_{\rm(l)} = {Cl}^-_{\rm(aq)} + HClO_{\rm(aq)}+
			      {H}^+_{\rm(aq)}}
		      \]
	\end{enumerate}
}%

\resetQ
\section{Nombres d'oxydation du chrome}
\enonce{%
	Le chrome \ce{Cr} a pour numéro atomique $Z = 24$, et il est moins
	électronégatif que l'oxygène.
}%
\QR{%
Donner le \no du chrome au sein des espèces $\ce{Cr_{\rm(s)}}$,
$\ce{{Cr}^2+_{\rm(aq)}}$ et $\ce{{Cr}^3+_{\rm(aq)}}$.
}{%
On fait attention à bien parler du nombre d'oxydation du chrome \textbf{dans
	l'édifice $\ce{Cr^2+}$}, et dans ce cas le \no est égal à la charge~:
\[
	\boxed{
		\no{Cr \in Cr} = 0
		\qquad
		\no{Cr \in Cr^2+} = +\myRoman{2}
		\qquad
		\no{Cr \in Cr^3+} = +\myRoman{3}
	}
\]
}%
\QR{%
	Sans représenter les schémas de \textsc{Lewis}, déterminer le \no du chrome
	dans les espèces \ce{CrO4^2-} et \ce{Cr2O7^2-}. On précise qu'il n'y a pas de
	liaison \ce{Cr-Cr} dans le dichromate.
}{%
	On suppose que $\no{O} = - \myRoman{2}$, puisque l'oxygène est l'élément le
	plus électronégatif et qu'il ne lui manque que 2 électrons pour remplir sa
	couche de valence. Avec la somme des \no qui doit être égale à la charge
	totale de l'édifice, on obtient
	\begin{itemize}
		\item $q (\ce{CrO_4^{2-}}) = -2 = \no{Cr} + 4 \no{O} \Lra \boxed{\no{Cr \in
			      CrO_4^{2+}} = + \myRoman{6}}$~;
		\item $q (\ce{Cr_2O_7^{2-}}) = -2 = 2 \no{Cr} + 7 \no{O} \Lra \boxed{\no{Cr
			      \in Cr_2O_7^{2-}} = + \myRoman{6}}$.
	\end{itemize}
}%
\QR{%
	Justifier que \ce{Cr2O7^2-} et \ce{Cr^3+} forment un couple rédox. Identifier
	l'oxydant et le réducteur sans utiliser la demi-équation. Écrire
	\textbf{ensuite} la demi-équation associée, en milieu acide et en milieu
	basique.
}{%
	Un couple rédox échange des électrons, donc les deux espèces \textbf{ont
		forcément des \no différents}~: c'est bien le cas du chrome dans le chrome III
	et du chrome dans les ions dichromates. On identifie l'oxydant comme étant
	celui de \no le plus élevé, ici \textbf{\ce{Cr2O7^{2-}} est l'oxydant}, et le
	réducteur comme celui de \no le plus bas, ici \textbf{\ce{Cr^3+} est le
		réducteur}. On obtient~:
	\begin{align*}
		\ce{
		2 {Cr}^3+_{\rm(aq)} + 7 H_2O_{\rm(l)}
		 & =
		{Cr_2O_7}^{2-}_{\rm(aq)} + 14 {H}^+_{\rm(aq)} + 6 e^-
		}
		\tag*{milieu acide}
		\\
		\beforetext{On ajoute \ce{14 HO-}}
		\ce{
		2 {Cr}^3+_{\rm(aq)} + 14 {HO}^-_{\rm(aq)}
		 & =
		{Cr_2O_7}^{2-}_{\rm(aq)} + 7 H_2O_{\rm(l)} + 6 e^-
		}
		\tag*{milieu basique}
	\end{align*}
}%
\QR{%
Justifier que \ce{CrO4^2-} et \ce{Cr2O7^2-} ne forment pas un couple rédox.
Montrer qu'il s'agit cependant d'un couple acide-base par écriture d'une
demi-équation.
}{%
Le chrome a le \textbf{même \no dans les deux cas}, donc ce n'est pas un
couple rédox. Par contre ils peuvent échanger des protons. Pour s'en assurer,
on équilibre la réaction comme d'habitude mais sans rajouter d'électrons~:
\[
	\boxed{\ce{
	{Cr_2O_7^{2-}}_{\rm(aq)} + H_2O_{\rm(l)} =
	2 {CrO_4^{2-}}_{\rm(aq)} + 2 {H}^+_{\rm(aq)}
	}}
\]
On a bien «~acide + eau = base + proton~», et pas d'électrons~: c'est un couple
acide-base~!
}%

% TODO: oubli du n.o. du soufre piqué à langevin

\resetQ
\section{Dismutation du dioxyde d'azote}
\enonce{%
En présence d'eau, le dioxyde d'azote $\ce{{NO_2}_{\rm(g)}}$ peut se dismuter
en ions nitrates $\ce{{NO_3}^-_{\rm(aq)}}$ et nitrites
$\ce{{NO_2}^-_{\rm(aq)}}$. Cette réaction produit des protons
$\ce{{H}^+_{\rm(aq)}}$, à l'origine des pluies acides.
}%
\QR{%
Écrire les demi-équations de transfert électronique et la relation de
\textsc{Nernst} pour les deux couples
$\ce{{NO_3^{-}}_{\rm(aq)}}/\ce{{NO_2}}_{\rm(g)}$ ($E_1^\circ = \SI{0.83}{V}$)
et $\ce{{NO_2}_{\rm(g)}/{NO_2^{-}}_{\rm(aq)}}$ ($E_2^\circ = \SI{0.85}{V}$).
}{%
On a~:
\begin{itemize}
	\item \leftcenters{%
	      $\ce{(NO_3^{-}_{\rm(aq)}/{NO_2}_{\rm(g)})}$~:
	      }{%
	      $\ce{{NO_2}_{\rm(g)} + H_2O_{\rm(l)} = {NO_3^{-}}_{\rm(aq)} +
		      2{H}^+_{\rm(aq)} + e^-}$
	      }%
	      \vspace{-15pt}
	      \[
		      \Ra
		      E_1 = E_1^\circ + \num{0.06} \log
		      \frac{[\ce{NO_3^{-}}][\ce{H^+}]^2p^\circ}{p_{\ce{NO_2}}{c^\circ}^3}
	      \]
	\item \leftcenters{%
	      $\ce{{NO_2}_{\rm(g)}/{NO_2^{-}}_{\rm(aq)}}$~:
	      }{%
	      $\ce{{NO_2^{-}}_{\rm(aq)} = {NO_2}_{\rm(g)} + e^-}$
	      }%
	      \vspace{-15pt}
	      \[
		      \Ra
		      E_2 = E_2^\circ + \num{0.06}\log \frac{p_{\ce{NO_2}}c^\circ}{p^\circ
		      [\ce{NO_2^{-}}]}
	      \]
\end{itemize}
}%
\QR{%
	Justifier à l'aide de diagrammes de prédominance que \ce{NO2} se dismute. On
	choisira $p_{\ce{NO_2}} = \SI{1}{bar}$ et une concentration frontière
	(convention de tracé) de $\SI{1}{mol.L^{-1}}$ à pH nul.
}{%
	\noindent
	\begin{minipage}[c]{.48\linewidth}
		On calcule les potentiels frontière connaissant la convention de tracé~:
		\begin{gather*}
			E\ind{1,front} = E_1^\circ + \num{0.06}\log c_t = E_1^\circ = \SI{0.83}{V}
			\\\beforetext{et}
			E\ind{2,front} = E_2^\circ + \num{0.06}\log c_t = E_2^\circ = \SI{0.85}{V}
		\end{gather*}
	\end{minipage}
	\hfill
	\begin{minipage}[c]{.48\linewidth}
		D'où le diagramme~:
		\begin{center}
			\includegraphics[width=.9\linewidth]{predom_no2}
		\end{center}
	\end{minipage}
	\smallbreak
	Ainsi, \textbf{les domaines de prédominance de \ce{NO2} sont disjoints}~: il
	va spontanément réagir avec lui-même pour donner des formes qui peuvent
	coexister au même potentiel, c'est-à-dire qu'il se dismute.
	\smallbreak
	On observe cependant que les potentiels frontières sont très proches, donc la
	réaction associée sera très limitée (grossièrement, il faut
	$\abs{\Delta{E\ind{lim}}} \approx \SI{0.20}{V}$ pour avoir totalité~; cela
	dépend du nombre d'électrons échangés mais sûrement avec une différence de
	$\SI{0.02}{V}$ on n'y est pas~!)
}%
\QR{%
Écrire l'équation bilan de l'équation de dismutation.
}{%
\centers{$\ce{2 {NO_2}_{\rm(g)} + H2O_{\rm(l)} = {NO_3^{-}}_{\rm(aq)} +
	{NO_2^{-}}_{\rm(aq)} + 2 {H}^+_{\rm(aq)}}$}
}%
\QR{%
	Exprimer sa constante d'équilibre $K^\circ$ en fonction des potentiels
	standard et calculer sa valeur numérique.
}{%
	On pourrait refaire le calcul du cours, mais ça n'est pas demandé~: on donne
	simplement
	\[
		K^\circ =
		10^{\DS +\frac{\abs{\Delta{E^\circ}}}{\num{0.06}}} =
		10^{\DS \frac{E_2^\circ - E_1^\circ}{\num{0.06}}}
		Lra
		K^\circ = \num{2.15}
	\]
	en prenant la valeur absolue puisqu'on a déterminé que la réaction était
	favorisée à la question précédente (domaines disjoints $\Lra K^\circ > 1 \Lra$
	réaction favorisée). Elle est donc en effet favorisée, mais très peu, on a
	$K^\circ$ proche de l'unité, c'est cohérent.
}%

\resetQ
\section{Éthylotest}
\enonce{%
	\noindent
	\begin{minipage}[c]{.20\linewidth}
		\begin{center}
			\includegraphics[width=\linewidth]{ethylotest}
		\end{center}
	\end{minipage}
	\hfill
	\begin{minipage}[c]{.75\linewidth}
		Peu après avoir été consommé, l'alcool (éthanol de formule \ce{CH3CH2OH})
		passe dans le sang au niveau de l'intestin grêle. Ensuite, des échanges
		gazeux s'effectuent dans les alvéoles pulmonaires~: le sang se charge en
		dioxygène et se libère du dioxyde de carbone, ainsi que d'une partie de
		l'alcool. Ces vapeurs sont expirées dans l'air avec une concentration en
		alcool \num{2100} fois inférieure à celle du sang. Le seuil limite autorisé
		pour la conduite est de \SI{0.50}{g} d'éthanol par litre de sang.
		\smallbreak
		Les alcootests jetables sont constitués d'un sachet gonflable de capacité
		\SI{1}{L} et d'un tube en verre contenant des cristaux orangés de dichromate
		de potassium \ce{K2Cr2O7} en milieu acide. Ceux-ci se colorent en vert au
		contact de l'alcool.
	\end{minipage}
	\begin{tcb}(data)<lftt>{Données}
		\begin{itemize}
			\item Potentiels standard~: $E^\circ(\ce{Cr_2O_7^2-/Cr^3+}) = E_1^\circ =
				      \SI{1.33}{V}$~; $E^\circ(\ce{CH_3COOH/CH_3CH_2OH}) = E_2^\circ =
				      \SI{0.19}{V}$~;
			\item %
			      \leftcenters{%
				      Masses molaires atomiques~:
			      }{%
				      \begin{tabular}{cccccc}
					      \toprule
					      Élément               & H & C  & O  & K  & Cr
					      \\
					      $M (\si{g.mol^{-1}})$ & 1 & 12 & 16 & 39 & 52
					      \\
					      \bottomrule
				      \end{tabular}
			      }%
		\end{itemize}
	\end{tcb}
}%
\QR{%
	Écrire l'équation de la transformation responsable du changement de couleur.
	Identifier l'espèce oxydée et l'espèce réduite.
}{%
	Les espèces en présence sont l'éthanol et le dichromate. On écrit les deux
	demi-équations qu'on combine ensuite en éliminant les électrons~:
	\begin{align*}
		\ce{
		2{Cr^3+}_{\rm(aq)} + 7 H_2O_{\rm(l)}
		 & =
		{Cr_2O_7^{2-}}_{\rm(aq)} + 14 {H}^+_{\rm(aq)} + 6 e^-
		}
		\tag{1}
		\\
		\ce{
		CH_3CH_2OH_{\rm(aq)} + H_2O_{\rm(l)}
		 & =
		CH_3COOH_{\rm(aq)} + 4 {H}^+_{\rm(aq)} + 4 e^-
		}
		\tag{2}
		\\
		\ce{
		3 CH_3CH_2OH_{\rm(aq)} + 2 {Cr_2O_7^{2-}}_{\rm(aq)} + 16 {H}^+_{\rm(aq)}
		 & =
		3 CH_3COOH_{\rm(aq)} + 4 {Cr}^3+_{\rm(aq)} + 11 H_2O_{\rm(l)}
		}
		\tag*{$(3) = 3(2) - 2(1)$}
	\end{align*}
	D'après la donnée des couples, l'éthanol est le réducteur, il se fait donc
	\textbf{oxyder}, alors que le dichromate est l'oxydant, donc il est réduit.
}%
\QR{%
Calculer la constante d'équilibre de la réaction. Commenter.
}{%
On a 12 électrons échangés, d'où la constante
\[
K^\circ = 10^{\DS \frac{12}{\num{0.06}}\abs{\Delta{E^\circ}}} = 10^{\num{228}}
		\]
		On prend la valeur absolue puisque la réaction est favorisée dans le sens
		direct, sinon l'éthylotest ne marcherait pas. Une autre manière de s'en
		convaincre est de faire un diagramme de prédominance/une échelle en potentiels
		limites. Ici, pour simplifier on peut supposer que les potentiels limites sont
		les potentiels standard, et on obtient le diagramme de prédominance suivant~:
		\begin{center}
			\includegraphics[scale=1]{predom_ch3cooh}
		\end{center}
		On voit bien que la réaction prépondérante est favorisée puisque les domaines
		de prédominance des réactifs sont disjoints~! Par ailleurs, comme la
		différence de $E^\circ$ est grande (i.e., $> \SI{0.20}{V}$), elle sera en
		effet totale.
	}%
\QR{%
	Déterminer la quantité de matière d'alcool expirée par litre d'air, dans
	l'hypothèse d'une alcoolémie atteignant le seuil de \SI{0.50}{g.L^{-1}}
	d'alcool dans un litre de sang.
}{%
	À la limite tolérée dans le sang, on a la concentration massique
	\begin{gather*}
		c_{m,\rm sang} = \SI{0.50}{g.L^{-1}}
		\qor
		c_{m,\rm air} = \frac{c_{m,\rm sang}}{\num{2100}}
		\qet
		c\ind{air} = \frac{c_{m, \rm air}}{M\ind{éthanol}}
		\Lra
		c\ind{air} = \frac{c_{m,\rm sang}}{\num{2100}M\ind{éthanol}}
		\\\AN
		\xul{c\ind{air} = \SI{5.2e-6}{mol.L^{-1}}}
	\end{gather*}
}%
\QR{%
	En déduire la masse de dichromate de potassium devant être placée avant le
	trait de jauge afin que celui-ci indique le seuil limite.
}{%
	Il faut comprendre comment le système fonctionne, et pour ça rien de mieux
	qu'un schéma. On a un ballon de volume $V = \SI{1}{L}$ rempli d'air \textit{a
		priori} à la concentration $c\ind{air}$ en éthanol, soit une quantité $n_a =
		c\ind{air}V = \SI{5.2e-6}{mol}$.
	\smallbreak
	Au-dessus de ce ballon se situe l'éthylotest, dans lequel se trouvent des
	cristaux de dichromate de potassium qui se colorient en vert (présence d'ions
	$\ce{Cr^3+}$) au contact de l'alcool de l'air. Ainsi, on suppose que le trait
	de jauge mentionné dans l'exercice doit faire référence à une limite placée
	sur l'appareil pour que l'utilisateur-ice puisse repérer le seuil toléré pour
	la conduite.
	\smallbreak
	Si on veut que cette limite corresponde à la quantité $n_a$ d'alcool dans
	l'air, c'est donc que la quantité de matière de dichromate qui réagit jusqu'à
	ce trait de jauge est introduit en \textbf{proportions stœchiométriques} avec
	l'éthanol de l'air. C'est en fait une sorte de dosage par titrage, et on
	cherche la relation à l'équivalence~!
	\smallbreak
	En dressant un rapide tableau d'avancement, on trouve la relation
	\begin{gather*}
		n\ind{dichromate} = n_i - 2 \xi_f = 0
		\qet
		n\ind{éthanol} = n_a - 3 \xi_f = 0
		\qso
		\xi_f = \frac{n_i}{2} = \frac{n_a}{3}
		\Lra
		n_i = \frac{2}{3}n_a
		\\
		\beforetext{Pour la masse}
		\boxed{m_i = \frac{2}{3}n_a\,M_{\ce{K_2CR_2O_7}}}
		\Ra
		\xul{m_i = \SI{1.0}{mg}}
	\end{gather*}
}%

\resetQ
\section{Pile argent-zinc}
\enonce{%
On s'intéresse à une pile schématisée par
$\ce{Ag_{\rm(s)}|{Ag}^+_{\rm(aq)}||{Zn}^2+_{\rm(aq)}|Zn_{\rm(s)}}$, avec
$[\ce{Ag+}]_i = c = \SI{0.18}{mol.L^{-1}}$ et $[\ce{Zn^2+}]_i = c' =
	\SI{0.30}{mol.L^{-1}}$. Le compartiment de gauche a un volume $V =
	\SI{100}{mL}$, celui de droite un volume $V' = \SI{250}{mL}$.
\begin{tcb}(data)<lfnt>{Données}
	$E^\circ(\ce{Zn^2+/Zn}) = \SI{-0.76}{V}$ et $E^\circ(\ce{Ag^+/Ag}) =
		\SI{0.80}{V}$.
\end{tcb}
}%
\QR{%
	Déterminer la f.é.m.\ de la pile. Identifier alors l'anode et la cathode par
	un raisonnement sur le trajet des électrons.
}{%
	La f.é.m.\ de la pile est la différence entre les \textbf{potentiels rédox}
	des deux électrodes \textbf{dans la situation initiale} avant qu'elle n'ait
	commencé à débité, donnés par la relation de \textsc{Nernst}. On écrit donc
	les demi-équations et les potentiels associés avant de faire la différence~:
	\begin{gather*}
		\ce{Ag_{\rm(s)} = {Ag}^+_{\rm(aq)} + e^-}
		\quad \Ra \quad
		E_{\ce{Ag^{+}/Ag}} = E^\circ_{\ce{Ag/Ag^+}} + \num{0.06} \log c = \SI{0.76}{V}
		\\
		\ce{Zn_{\rm(s)} = {Zn}^2+_{\rm(aq)} + 2e^-}
		\quad \Ra \quad
		E_{\ce{Zn^{2+}/Zn}} = E^\circ_{\ce{Zn^{2+}/Zn}} + \frac{\num{0.06}}{2} \log
		c' = \SI{-0.78}{V}
		\\\Lra
		\boxed{e = \abs{\Delta{E}} = \SI{1.54}{V}}
	\end{gather*}
	puisque la f.é.m.\ est forcément positive.
	\bigbreak
	Dans un circuit, les électrons migrent du \textbf{potentiel le plus bas} vers
	le \textbf{potentiel le plus haut} ($\Ff\ind{lorentz} = q\Ef$ et $\Ef$
	analogue à $\gf$ donc va des potentiels les plus hauts aux potentiels les plus
	bas, mais $q\ind{électron} = -e$ donc $\Ff$ dans le sens opposé), donc ici du
	\textbf{zinc vers l'argent}. Or, par définition, c'est \textbf{l'oxydation qui
		créé les électrons}, et \textbf{la réduction qui les consomme}~; forcément, à
	l'électrode de zinc il y a formation d'électrons et à l'électrode d'argent il
	y a réception~; ce sont donc respectivement \textbf{l'anode} et \textbf{la
		cathode}.
}%
\QR{%
	Écrire les réactions électrochimiques aux électrodes, puis la réaction de
	fonctionnement qui se produit lorsque la pile débite.
}{%
	\vspace{-15pt}
	\begin{align*}
		\ce{
		{Ag}^+_{\rm(aq)} + e^- = Ag_{\rm(s)}
		}
		\tag*{cathode = réduction}
		\\
		\ce{
		Zn_{\rm(s)} = {Zn}^2+_{\rm(aq)} + 2e^-
		}
		\tag*{anode = oxydation}
		\\\Ra
		\beforetext{on compense les électrons}
		\ce{
		2 {Ag}^+_{\rm(aq)} + Zn_{\rm(s)} & =
		2 Ag_{\rm(s)} + {Zn^2+}_{\rm(aq)}
		}
	\end{align*}
}%
\QR{%
	Schématiser le déplacement des porteurs de charge dans chaque partie de la
	pile lorsqu'elle débite du courant.
}{%
	\begin{center}
		\includegraphics[width=.5\linewidth, valign=t]{pile_ag-zn}
	\end{center}
}%
\QR{%
	Déterminer la composition de la pile lorsqu'elle est usée. Quelle quantité
	d'électricité, en coulombs d'abord puis en \si{A.h} ensuite, a-t-elle débité~?
	Ça fait combien de smartphones~?
}{%
	On cherche l'état final. Il nous faut donc la valeur de $K^\circ$. Or, on a
	\[
		K^\circ =
		10^{\DS \frac{2}{\num{0.06}} \abs{\Delta{E^\circ}}} =
		10^{\DS \frac{2}{\num{0.05}} (E^\circ_{\ce{Ag^{+}/Ag}} -
				E^\circ_{\ce{Zn^{2+}/Zn}})}
		\Lra
		\boxed{K^\circ = \num{e53}}
	\]
	On pourra donc \textbf{supposer la réaction totale} (ce qu'on vérifiera après
	coup). On dresse donc le tableau d'avancement, et on fait attention à bien
	faire un \textbf{bilan en quantité de matière} puisque les volumes sont
	différents~:
	\begin{center}
		\def\rhgt{0.35}
		\centering
		\begin{tabularx}{\linewidth}{|l|c||YdYdYdY|}
			\hline
			\multicolumn{2}{|c||}{
				$\xmathstrut{\rhgt}$
			\textbf{Équation}}       &
			$2\ce{{Ag}^+_{\rm(aq)}}$ & $+$       &
			$\ce{Zn_{\rm(s)}}$       & $\ra$     &
			$2\ce{Ag_{\rm(s)}}$      & $+$       &
			$\ce{Zn^{2+}}$                         \\
			\hline
			$\xmathstrut{\rhgt}$
			Initial                  & $\xi = 0$ &
			$cV$                     & \vline    &
			excès                    & \vline    &
			excès                    & \vline    &
			$c'V'$                                 \\
			\hline
			$\xmathstrut{\rhgt}$
			Final                    & $\xi_f$   &
			$cV - 2\xi_{f} = \ep V$  & \vline    &
			excès                    & \vline    &
			excès                    & \vline    &
			$c'V' + \xi_{f}$                       \\
			\hline
		\end{tabularx}
	\end{center}
	En supposant la réaction totale, on a donc $\xi_f = \xi\ind{max} = \frac{cV}{2}$, soit
	\[
		[\ce{Zn^2+}]_f = \frac{c'V' + \frac{cV}{2}}{V'}
		\Lra
		\boxed{
			[\ce{Zn^2+}]_f = c' + \frac{cV}{2V'}
		}
		\Ra
		\xul{
			[\ce{Zn^2+}]_f = \SI{0.34}{mol.L^{-1}}
		}
	\]
	On \textbf{vérifie l'hypothèse de totalité} en appelant $\ep$ la concentration
	de $\ce{Ag^+}$ restante, et on la trouve grâce à la constante de réaction~:
	\[
		K^\circ = \frac{[\ce{Zn^2+}]_fc^\circ}{\ep^2}
		\Lra
		\boxed{\ep = \sqrt{\frac{[\ce{Zn^2+}]_f}{K^\circ}}}
		\Ra
		[\ce{Ag^+}]_f = \SI{1.8e-27}{mol.L^{-1}}
	\]
	Ce qui est bien négligeable~: \textbf{l'hypothèse est validée \iconchek}.
	\bigbreak
	On trouve la quantité d'électricité en multipliant la quantité de matière
	d'électrons échangés ($n_{\ce{e^-},\rm tot}\xi_f$) par la charge électrique
	d'une mole d'électrons ($\Fc = \Nc_A e$)~:
	\[
		Q = 2\xi_f \cdot \Fc
		\Lra
		\boxed{Q = cV \Fc}
		\Ra
		\xul{Q = \SI{1.7e4}{C} = \SI{4.7}{A.h}}
	\]
	Un smartphone gourmand tourne autour des $\SI{4000}{mA.h}$, soit
	$\SI{4}{A.h}$~: ça fait donc suffisamment de charge pour \num{1.2}
	smartphone~!
}%

\resetQ
\section{Stabilisation du cuivre I par précipitation}
\enonce{%
	L'objectif de cet exercice est d'étudier la stabilisation du cuivre de
	$\no{\ce{Cu}} = \myRoman{+1}$ par précipitation, qui illustre plus
	généralement l'influence de la précipitation sur l'oxydoréduction.
	\begin{tcb}(data)<lfnt>{Données}
		$E^\circ(\ce{Cu^+/Cu}) = E_1^\circ = \SI{0.52}{V}$~;
		$E^\circ(\ce{Cu^2+/Cu^+}) = E_2^\circ = \SI{0.16}{V}$
	\end{tcb}
}%
\QR{%
	Montrer à partir de diagrammes de stabilité que l'ion \ce{Cu^+} est instable.
	Pour simplifier, on prendra \SI{1}{mol.L^{-1}} comme concentration frontière.
	Qu'observe-t-on~?
}{%
	solu
}%
\enonce{%
	Les ions cuivre I forment avec les ions iodure \ce{I^-} le précipité
	$\ce{CuI_{\rm(s)}}$, de produit de solubilité $K_s = \num{e-11}$.
}%
\QR{%
	Écrire l'équation de dissolution du précipité, puis les demi-équations rédox
	pour les couples \ce{CuI/Cu} et \ce{Cu^2+/CuI}.
}{%
	solu
}%
\QR{%
	En déduire la relation de \textsc{Nernst} pour les couples \ce{CuI/Cu} et
	\ce{Cu^2+/CuI} en notant leurs potentiels standard $E_3^\circ$ et $E_4^\circ$,
	respectivement. Exprimer alors $E_3^\circ$ en fonction de $\pk[s]$ et
	$E_1^\circ$ d'une part, puis $E_4^\circ$ en fonction de $\pk[s]$ et
	$E_2^\circ$ d'autre part. Calculer les valeurs numériques.
}{%
	solu
}%
\QR{%
	Expliquer alors en quoi les ions cuivre I sont stabilisés en présence d'ions
	iodure.
}{%
	solu
}%

\resetQ
\section{Dosage colorimétrique en retour}
\enonce{%
	On s'intéresse à un dosage colorimétrique d'une solution de dichromate de
	potassium par les ions fer II en présence d'acide sulfurique, garantissant un
	pH très acide. On donne les potentiels standard
	\[
		E^\circ (\ce{Cr_2O_7^2-/Cr^3+}) = E_1^\circ = \SI{1.33}{V}
		\qet
		E^\circ(\ce{Fe^3+/Fe^2+}) = E_2^\circ = \SI{0.77}{V}
	\]
	En milieu acide, l'ion dichromate est orange et l'ion chrome III est vert,
	alors que l'ion \ce{Fe^2+} est vert pâle et l'ion \ce{Fe^3+} est jaune-orangé.
}%
\QR{%
	Écrire l'équation bilan du titrage rédox direct.
}{%
	solu
}%
\QR{%
	Calculer sa constante d'équilibre. Cette réaction est-elle adaptée à un
	titrage~? Pourquoi est-elle malgré tout peu adaptée à un titrage
	colorimétrique~?
}{%
	solu
}%
\QR{%
	Justifier qu'il serait possible de suivre la réaction par potentiométrie.
	Détreminer le sens du saut de potentiel qui serait observé~: est-il descendant
	ou montant~?
}{%
	solu
}%
\enonce{%
	Pour contourner la difficulté sans montage de potentiométrie, on effectue un
	dosage en retour. Dans un bécher, on verse $V_1 = \SI{4.0}{mL}$ de la solution
	de dichromate de potassium dont on cherche la concentration $c_1$. On y ajoute
	$V_2 = \SI{10.0}{mL}$ d'une solution de sulfate de fer II en milieu
	sulfurique, de concentration $c_2 = \SI{0.10}{mol.L^{-1}}$ et $V_2 -
		\SI{90.0}{mL}$ d'eau. On verse ensuite par une burette une solution de
	permanganate de potassium de concentration $c_3 = \SI{1.0e-2}{mol.L^{-1}}$.
	Une coloration violette, caractéristique du permanganate en solution, apparaît
	lorsque que $V_{3,\eqi} = \SI{12}{mL}$ ont été versés.
}%
\QR{%
	Comment peut-on s'assurer qualitativement que les ions fer II ont bien été
	apportés en excès par rapport au dichromate~?
}{%
	solu
}%
\QR{%
	Écrire l'équation bilan du titrage en retour.
}{%
	solu
}%
\QR{%
	Déterminer la concentration $c_1$ de la solution de dichromate de potassium.
}{%
	solu
}%

\resetQ
\section{Pile à combustible à oxyde solide \hfill \small écrit PT 2015}
\enonce{%
Le principe de la pile à combustible consiste à utiliser du dihydrogène pour
stocker et transporter de l'énergie. Une pile à combustible est un assemblage
de cellules élémentaires, en nombre suffisant pour assurer la production
électrochimique d'électricité dans les conditions de tension et d'intensité
voulues. De façon générale, le fonctionnement électrochimique d'une cellule
élémentaire de pile à combustible peut être représenté selon le schéma de la
Figure~\ref{fig:pile_comb}
\begin{center}
	\includegraphics[width=.8\linewidth]{pile_comb}
	\captionof{figure}{Schéma de principe d'une pile à combustible.}
	\label{fig:pile_comb}
\end{center}
Chaque cellule élémentaire est constituée de deux compartiments disjoints,
alimentés chacun en gaz dihydrogène et dioxygène. Les électrodes sont séparées
par un électrolyte solide qui laisse passer les anions oxygène. Les couples
d'oxydorédution mis en jeu dans la réaction sont
$\ce{{H}^+_{\rm(aq)}/{H_2}_{\rm(g)}}$ et $\ce{{O_2}_{\rm(g)}/H_2O_{\rm(l)}}$.
}%
\QR{%
	Indiquer la position des atomes constitutifs dans les réactifs et du produit.
	En déduire les schémas de \textsc{Lewis} des trois molécules.
}{%
	solu
}%
\QR{%
	À partir des informations du schéma, attribuer et justifier le choix de la
	cathode et de l'anonde aux électrodes 1 et 2, ainsi que le sens de circulation
	des électrons.
}{%
	solu
}%
\QR{%
	Écrire les demi-équations électroniques pour chaque couple mis en jeu, quand
	la pile débite.
}{%
	solu
}%
\QR{%
	Le réactif qui est oxydé est appelé le combustible de la pile. Parmi les
	espèces chimiques présentes dans les couples, laquelle constitue le
	combustible~?
}{%
	solu
}%
\QR{%
	En déduire l'équation de la réaction modélisant la transformation ayant lieu
	dans la cellule de réaction.
}{%
	solu
}%
\enonce{%
	Dans un véhicule motorisé fonctionnant grâce à une pile à combustible, on
	estime à \SI{1.5}{kg} la masse de dihydrogène nécessaire pour parcourir
	\SI{250}{km}.
}%

\QR{%
	Calculer la quantité de matière de dihydrogène correspondant à cette masse,
	puis le volume occupé par cette quantité de gaz à $\SI{20}{\degreeCelsius}$
	sous pression atmosphérique.
}{%
	solu
}%
\QR{%
	Quel est l'avantage pour l'environnement de l'utilisation d'une pile à
	combustible au dihydrogène par rapport à un carburant classique~? Quel en est
	l'inconvénient majeur~?
}{%
	solu
}%

\resetQ
\section{Accumulateur lithium métal \hfill \small oral banque PT}
\enonce{%
	On étudie ici l'accumulateur lithium-oxyde de manganèse, qui représente
	environ 80\% du marché des batteries au lithium. La première électrode est en
	dioxyde de manganèse \ce{MnO2}, la deuxième en lithium \ce{Li}. Ces deux
	électrodes baignent dans un électrolyte organique contenant des ions \ce{Li+}.
	\begin{tcb}(data)<lfnt>{Données}
		\begin{itemize}
			\item Numéro atomique du lithium~: $Z = 3$.
			\item Masse molaire du lithium~: $M_{\ce{Li}} = \SI{5.9}{g.mol^{-1}}$.
			\item Potentiels standard~: $E_1^\circ(\ce{Li^+/Li_{\rm(s)}}) =
				      -\SI{3.03}{V}$ et
			      $E_2^\circ(\ce{{MnO_2}_{\rm(s)}/{LiMnO_2}_{\rm(s)}})$ = \SI{0.65}{V}.
		\end{itemize}
	\end{tcb}
}%
\QR{%
	Indiquer la position di lithium dans le tableau périodique. Pourquoi choisir
	un électrolyte organique plutôt que de l'eau~?
}{%
	solu
}%
\QR{%
	Écrire les réactions aux électrodes lorsque l'accumulateur fonctionne en
	générateur, ainsi que la réaction globale de fonctionnement.
}{%
	solu
}%
\QR{%
	La pile contient-elle un pont salin ou équivalent~? Pourquoi~?
}{%
	solu
}%
\QR{%
	Déterminer la force électromotrice de la pile.
}{%
	solu
}%
\QR{%
	Déterminer la capacité $C$ de la pile en \si{A.h} pour une masse initiale de
	\SI{2}{g} de lithium.
}{%
	solu
}%

\resetQ
\section{Dosage d'une solution d'hypochlorite de sodium \hfill \small écrit PT
  2016}

\enonce{%
Après avoir introduit un volume $V_0 = \SI{2.00}{mL}$ d'une solution
commerciale d'hypochlorite de sodium $(\ce{Na^+}~;~\ce{ClO^-})$ dans une fiole
jaugée de volume $V_f = \SI{100}{mL}$, on complète avec de l'eau distillée
jusqu'au trait de jauge. À un volume $V = \SI{10.0}{mL}$ de cette solutio
fille, on ajoute environ \SI{10}{mL} d'une solution d'iodure de potassium
$(\ce{K^+}~;~\ce{I^-})$ à 15\% en masse et \SI{5.0}{mL} d'acide éthanoïque
$\ce{CH_3CO_2H_{\rm(aq)}}$ à \SI{3.0}{mol.L^{-1}}. L'échantillon obtenu est
titré par une solution de thiosulfate de sodium
$(\ce{2Na^+}~;~\ce{{S_2O_3}^{2-}})$ de concentration $c =
	\SI{2.0e-2}{mol.L^{-1}}$. Le volume équivalent est égal à $V' =
	\SI{16.0}{mL}$.
\begin{tcb}(data)<lftt>{Données à \SI{298}{K}}
	\[
		E^\circ(\ce{ClO^-/Cl^-}) = \SI{0.89}{V}
		\qquad
		E^\circ(\ce{I_2/I^-}) = \SI{0.54}{V}
		\qquad
		E^\circ(\ce{{S_4O_6}^2-/{S_2O_3}^2-}) = \SI{0.08}{V}
	\]
\end{tcb}
}%
\QR{%
	Proposer une équation pour la réaction entre les ions hypochlorite
	$\ce{ClO^-}$ et les ions iodure \ce{I^-}. Prévoir qualitativement le caractère
	favorisé ou défavorisé de la réaction.
}{%
	solu
}%
\QR{%
Proposer une équation pour la réaction de titrage du diiode \ce{I2} par les
ions thiosulfate $\ce{S_2O_3^{2-}}$. Prévoir qualitativement le caractère
favorisé ou défavorisé de la réaction.
}{%
solu
}%
\QR{%
	Sachant que les ions iodure et l'acide éthanoïque sont introduits en excès,
	déterminer la concentration en ions hypochlorite dans la solution commerciale.
}{%
	solu
}%

\end{document}
