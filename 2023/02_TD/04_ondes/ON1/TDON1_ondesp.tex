\documentclass[a4paper, 12pt, final, garamond]{book}
\usepackage{cours-preambule}
% \usepackage{bpep_full}
% \usepackage{chngcntr}
%
% \counterwithin*{equation}{section}
% \counterwithin*{equation}{subsection}

\renewcommand{\f}[2]{{
			\mathchoice
			{\dfrac{#1}{#2}}
			{\dfrac{#1}{#2}}
			{\frac{#1}{#2}}
			{\frac{#1}{#2}}
		}}

\newcommand{\e}[1]{{}_{\text{#1}}}

\raggedbottom

\makeatletter
\renewcommand{\@chapapp}{Ondes -- chapitre}
\makeatother

\toggletrue{student}
\HideSolutionstrue
\toggletrue{corrige}

\begin{document}
\setcounter{chapter}{-1}

\chapter{\sujetUniquement{TD~: Ondes progressives}\siCorrige{Correction du TD}}

\resetQ
\section{Quelques ondes}
\partie{Onde sur une corde}

\enonce{
	On excite l’extrémité d’une corde à une fréquence de $\SI{50}{\hertz}$. Les vibrations se propagent le long de la corde avec une célérité de $\SI{10}{\metre \per \second}$.
}

\QR{
	Quelle est la longueur d’onde ?
}{
	\[
		\lambda = \frac{c}{f} = \frac{10}{50} = \boxed{\SI{0,2}{\metre}}
	\]
}

\partie{Ondes infrasonores des éléphants}

\enonce{
	Les éléphants émettent des infrasons dont la fréquence est inférieure à $\SI{20}{\hertz}$. Cela leur permet de communiquer sur de longues distances et de se rassembler. Un éléphant est sur le bord d’une étendue d’eau et désire indiquer à d’autres éléphants sa présence. Pour cela, il émet un infrason. Un autre éléphant, situé à une distance $L = \SI{24,0}{\kilo\metre}$, reçoit l’onde au bout d’une durée $\Delta t=\SI{70.6}{\second}$.

	\textbf{Donnée numérique.} $24/7,06 \approx 3, 3994334$.
}

\QR{
Quelle est la valeur de la célérité $c$ de l’infrason dans l’air ?
}{
\[
	c=\frac{L}{\Delta t}=\frac{24.10^{3}}{70,6} = 3,4.10^{2} = \boxed{\SI{340}{\metre/\second}}.
\]
}

\partie{Ondes à la surface de l’eau}

\enonce{Au laboratoire, on dispose d’une cuve à onde contenant de l’eau immobile à la surface de laquelle flotte un petit morceau de polystyrène. On laisse tomber une goutte d’eau au-dessus de la cuve, à l’écart du morceau de polystyrène. Une onde se propage à la surface de l’eau. Quelles sont les affirmations exactes ?
}

\QR{
	Ceci correspond :
	\begin{tasks}(3)
		\task à une onde mécanique,
		\task à une onde longitudinale,
		\task à une onde transversale.
	\end{tasks}
}{
	\begin{tasks}(3)
		\task oui
		\task non
		\task oui
	\end{tasks}
}

\QR{
	L'onde atteint le morceau de polystyrène.
	\begin{enumerate}
		\item Celui-ci se déplace parallèlement à la direction de propagation de l’onde,
		\item Celui-ci se déplace perpendiculairement à la direction de propagation de l’onde,
		\item Celui-ci monte et descend verticalement,
		\item Celui-ci reste immobile.
	\end{enumerate}
}{
	\begin{tasks}(4)
		\task non
		\task oui
		\task oui
		\task non
	\end{tasks}
}

\resetQ
\subimport{/home/nora/Documents/Enseignement/Prepa/bpep/exercices/TD/app_cours_propagation_ondes/}{sujet.tex}

\resetQ
\subimport{/home/nora/Documents/Enseignement/Prepa/bpep/exercices/TD/cuve_a_ondes/}{sujet.tex}


\resetQ
\subimport{/home/nora/Documents/Enseignement/Prepa/bpep/exercices/TD/dispersion_SUP/}{sujet.tex}


\resetQ
\subimport{/home/nora/Documents/Enseignement/Prepa/bpep/exercices/TD/distance_foudre/}{sujet.tex}


\resetQ
\subimport{/home/nora/Documents/Enseignement/Prepa/bpep/exercices/TD/sonar_sous_marin/}{sujet.tex}


\resetQ
\subimport{/home/nora/Documents/Enseignement/Prepa/bpep/exercices/TD/telemetre/}{sujet.tex}

\end{document}
