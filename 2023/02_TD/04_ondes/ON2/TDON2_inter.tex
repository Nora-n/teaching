\documentclass[a4paper, 12pt, final, garamond]{book}
\usepackage{cours-preambule}

\raggedbottom

\makeatletter
\renewcommand{\@chapapp}{Ondes -- chapitre}
\makeatother

\toggletrue{student}
% \HideSolutionstrue
% \toggletrue{corrige}

\begin{document}
\setcounter{chapter}{1}

\chapter{\cswitch{Correction du TD}{TD~: Interf\'erences \`a deux ondes}}

\resetQ
\section{Interférences de 2 ondes sonores frontales}

\enonce{%
	\begin{minipage}{0.65\linewidth}
		Dans le montage ci-contre, les deux haut-parleurs, notés HP1 et HP2 et
		séparés de la distance $2D$, sont alimentés en parallèle par une même
		tension électrique~: les deux sources sonores émettent donc des vibrations
		$p_1$ et $p_2$ de même pulsation $\w$, même phase à l'origine $\f_0$ et même
		amplitude $P_0$. Les deux ondes arrivent au point M d'abscisse $x$ avec des
		phases différentes et donc interfèrent. On considère que les ondes sonores
		se propagent sans déformation ni atténuation à la célérité $c$ constante.
	\end{minipage}
	\hfill
	\begin{minipage}{0.35\linewidth}
		\begin{center}
			\includegraphics[width=\linewidth]{ondes_front-plain}
		\end{center}
	\end{minipage}
}

\QR{%
	Exprimer le déphasage $\D\f$ au point M entre les ondes issues de HP1
	et HP2.
}{%
	À partir de HP1, les ondes parcourent la distance $D+x$ pour arriver
	au micro. À partir de HP2, elles parcourent la distance $D-x$. Ainsi,
	\begin{align*}
		\D\f_{1/2}(\Mr)
		 & = -k\D L_{1/2}(\Mr) + \underbrace{\D\f_0(\Mr)}_
		{\mathclap{= 0 \text{ d'après l'énoncé}}}
		\\
		 & = - k \left( {\rm \abs{HP_1M} - \abs{HP_2M}} \right)
		\\
		 & = - k \left( \cancel{D}+x - (\cancel{D}-x) \right)
		\\
		\Leftrightarrow
		\Aboxed{
			\D\f_{1/2}(\Mr)
		 & = - 2kx}
	\end{align*}


}

\QR{%
	En déduire l'amplitude de l'onde sonore résultante au point M.
}{%
	Les ondes $p_1$ et $p_2$ étant de même amplitude $P_0$, on a que
	l'onde somme $p(t) = p_1(t) + p_2(t)$ est d'amplitude $P$ telle que
	\[P = 2P_0\cos(\frac{\D\f(\Mr)}{2})
		\Leftrightarrow
		\boxed{P = 2P_0\cos(-kx)}\]
}

\QR{%
	Déterminer les positions $x_n$ pour lesquelles il y a interférences
	constructives au point M.
}{%
	On a interférences constructives si l'amplitude est maximale, ici pour
	$\cos(-kx_n) = \pm 1 \Leftrightarrow -kx_n = n\pi$. Or,
	\begin{gather*}
		-kx_n = n\pi
		\Leftrightarrow
		- \frac{2\cancel{\pi}}{\lambda}x_n = n\cancel{\pi}
		\Leftrightarrow
		\boxed{
			x_n = n \frac{\lambda}{2}}
	\end{gather*}
}

\QR{%
	Exprimer la distance $d$ entre deux maximums successifs d'intensité
	sonore.
}{%
	Les maximums se trouvent aux positions $x_n$. La distance entre deux
	maximums est donc
	\begin{gather*}
		\boxed{d = x_{n+1} - x_n = \frac{\lambda}{2}}
	\end{gather*}
}

\QR{%
	Expérimentalement on trouve $d = \SI{21.2}{cm}$ pour une fréquence
	sonore $f = \SI{800}{Hz}$. En déduire la valeur de la célérité du son
	dans l'air pour cette expérience.
}{%
	Étant donné que $\lambda = cT = c/f$, on trouve
	\begin{gather*}
		\frac{\lambda}{2} = d
		\Leftrightarrow
		\frac{c}{2f} = d
		\Leftrightarrow
		\boxed{c = 2df}
		\qavec
		\left\{
		\begin{array}{rcl}
			d & = & \SI{21.2e-2}{m} \\
			f & = & \SI{800}{Hz}
		\end{array}
		\right.\\
		\mathrm{A.N.~:}\quad
		\boxed{c = \SI{339}{m.s^{-1}}}
	\end{gather*}
	C'est la valeur usuelle de célérité du son dans l'air à
	\SI{20}{\degreeCelsius}.
}


\resetQ
\section{Interférences sur la cuve à ondes}

\enonce{%
	La figure ci-dessous représente une cuve à ondes éclairée en éclairage
	stroboscopique. Deux pointes distantes de a frappent la surface de l'eau de
	manière synchrone (même fréquence et phase à l'origine), générant deux ondes qui
	interfèrent. La figure est claire là où la surface de l'eau est convexe et
	foncée là où elle est concave. L'amplitude d'oscillation est plus faible là où
	la figure est moins contrastée.

	\begin{center}
		\includegraphics[width=.8\linewidth]{cuve_ondes-plain}
	\end{center}
}

\QR{%
	On suppose pour simplifier que des ondes sinusoïdales partent des deux
	points $S_1$ et $S_2$ où les pointes frappent la surface. En notant
	$\lambda$ la longueur d'onde, donner la condition pour que
	l'interférence en un point M situé aux distances $d_1$ et $d_2$
	respectivement de $S_1$ et $S_2$, soit destructrice. Cette condition
	fait intervenir un entier $m$.

}{%
	Par définition,
	\begin{gather*}
		\D\f_{1/2}(\Mr) = -k\D L_{1/2}(\Mr) = -k(d_1 - d_2) =
		\frac{2\pi}{\lambda}(d_2-d_1)
	\end{gather*}
	Et pour avoir des interférences destructives,
	\begin{gather*}
		\D\f_{1/2}(\Mr) = (2m+1)\pi
		\Leftrightarrow
		\frac{2\cancel{\pi}}{\lambda}(d_2-d_1) = (2m+1)\cancel{\pi}
		\Leftrightarrow
		\boxed{d_2-d_1 = \left(m+\frac{1}{2}\right)\lambda}
	\end{gather*}
}

\QR{%
	Pour chaque entier $m$ le lieu des points vérifiant cette condition
	est une courbe que l'on appelle dans la suite ligne de vibration
	minimale. Les lignes de vibration minimale sont représentées sur la
	figure de droite~: ce sont des hyperboles. Les parties $x < -a/2$ et $x
		> a/2$ de l'axe (O$x$) sont des lignes de vibration minimale. En déduire
	un renseignement sur $a/\lambda$.

}{%
	Avec ${\rm S_1S_2} = a$, on observe que tout l'axe $x > a/2$
	correspond à une ligne de vibration minimale, c'est-à-dire un endroit de
	l'espace où les interactions sont destructives, i.e. $d_2-d_1 =
		(m+1/2)\lambda$. Or, pour $x > a/2$, on a
	\begin{gather*}
		d_2 - d_1 = {\rm S_2M} - {\rm S_1M}
		= \cancel{\rm S_2M} - {\rm S_1S_2} + \cancel{\rm S_2M}
		\Leftrightarrow
		\boxed{d_2 - d_1 = -a}
	\end{gather*}
	On en déduit donc
	\begin{gather*}
		\boxed{\abs{\frac{a}{\lambda}} = m+\frac{1}{2}}
	\end{gather*}
	c'est-à-dire que $a/\lambda$ est un demi-entier (1/2, 3/2, 5/2…). Le
	résultat est le même en raisonnant sur $x < -a/2$.
}

\QR{%
	Sur le segment S$_1$S$_2$, quel est l'intervalle de variation de $d_2
		- d_1$~? Déduire de la figure la valeur de $a/\lambda$.
}{%
	Entre $\rm S_1$ et $\rm S_2$, on prend 3 cas extrêmes pour déterminer
	l'amplitude de $d_2 - d_1$~:
	\begin{itemize}
		\item En $\rm S_1$, $d_2 = -a$ et $d_1 = 0$, donc
		      \[d_2 - d_1 = -a\]
		\item En O, $d_2 = -a/2$ et $d_1 = a/2$, donc
		      \[d_2 - d_1 = 0\]
		\item En $\rm S_2$, $d_2 = 0$ et $d_1 = a$, donc
		      \[d_2 - d_1 = -a\]
	\end{itemize}
	Ainsi,
	\begin{gather*}
		\boxed{-a \leqslant d_2 - d_1 \leqslant a}
	\end{gather*}
	Or, entre $\rm S_1S_2$ on observe plusieurs vibrations minimales,
	donnant chacune $d_2 - d_1 = (m+\frac{1}{2})\lambda$. On en compte 8
	entre $\rm S_1S_2$, correspondant chacune à un ordre d'interférence $m$.
	À partir de O et vers les $x$ croissants, on a la première vibration
	minimale pour $m=0$, la deuxième pour $m=1$, la troisième pour $m=2$ et
	la dernière pour $m=3$~; on a de même par symétrie vers les $x$
	décroissants. Ainsi, \textbf{l'ordre d'interférence obtenu le plus grand
		est $m=3$}, et \textbf{on n'a pas l'ordre d'interférence $m=4$} sinon on
	aurait une parabole en plus de chaque côté. Ainsi,
	\begin{gather*}
		\left(3+\frac{1}{2}\right)\lambda
		< a \leqslant
		\left(4+\frac{1}{2}\right)\lambda
	\end{gather*}
	puisqu'on observe qu'il reste une distance sur $\rm S_1S_2$ après
	l'ordre 3 avant d'atteindre S$_2$ et que si $a$ dépasse $(4+1/2)\lambda$
	on verrait la parabole correspondant à l'ordre 4. Comme on a déterminé à
	la question précédente que $\frac{a}{\lambda} = m + \frac{1}{2}$, avec
	cette étude on a $3 < m \leqslant 4$ avec $m \in \Nb$, autrement dit
	\fbox{$m = 4$}, soit
	\[\boxed{\frac{a}{\lambda} = \frac{9}{2}}\]
}

\QR{%
	Expliquer pourquoi l'image est bien contrastée au voisinage de l'axe
	(O$y$).
}{%
	Le contraste correspond à une grande différence entre les valeurs
	maximales et minimales. Or, sur (O$y$) on a $d_2 = d_1$ donc $d_2-d_1 =
		0$, c'est-à-dire que les ondes sont en phase et les interférences
	constructives, donc l'amplitude est maximale et le contraste est élevé.
}


\resetQ
\section{Trombone de \textsc{Kœnig}}

\enonce{%
	\begin{minipage}{0.70\linewidth}
		Le trombone de \textsc{Kœnig} est un dispositif de laboratoire permettant de
		faire interférer deux ondes sonores ayant suivi des chemins différents. Le
		haut-parleur, alimenté par un générateur de basses fréquences, émet un son
		de fréquence $f = \SI{1500\pm1}{Hz}$. On mesure le signal à la sortie avec
		un microphone branché sur un oscilloscope.
	\end{minipage}
	\hfill
	\begin{minipage}{0.30\linewidth}
		\begin{center}
			\includegraphics[width=\linewidth]{keonig-plain_white}
		\end{center}
	\end{minipage}
}

\QR{%
	Exprimer en fonction de la distance $d$ de coulissage de $T_2$ par
	rapport à $T_1$ le déphasage au niveau de la sortie entre l'onde sonore
	passée par $T_2$ et celle passée par $T_1$.
}{%
	\begin{gather*}
		\D\f_{2/1}(\Mr) = -k\D L_{2/1}(\Mr) = -k(\rm OT_2 - OT_1)
	\end{gather*}
	Or, si on déplace $T_2$ par rapport à $T_1$ de $d$, l'onde passant dans
	$T_2$ doit parcourir $2d$ de plus, une fois pour chaque partie
	rectiligne~; ainsi
	\[\boxed{\D\f_{2/1}(\Mr) = -2kd}\]
}

\QR{%
	En déplaçant la partie mobile $T_2$, on fait varier l'amplitude du
	signal observé. On observe que lorsqu'on déplace $T_2$ de $d =
		\SI{11.5\pm0.2}{cm}$, on passe d'un minimum d'amplitude à un autre. En
	déduire la valeur de la célérité du son dans l'air à
	\SI{20}{\degreeCelsius}, température à laquelle l'expérience est faite.
}{%
	Cette observation traduit qu'un décalage de \SI{11.5}{cm} fait passer
	d'une interférence destructive à celle qui la suit, donc augmente le
	déphasage de $2\pi$ ou la différence de marche de $\lambda$. On a donc
	\begin{gather*}
		\abs{\bcancel{2}kd} = \bcancel{2}\pi
		\Leftrightarrow
		\frac{2\cancel{\pi}}{\lambda}d = \cancel{\pi}
		\Leftrightarrow
		\boxed{2df = c}
		\qavec
		\left\{
		\begin{array}{rcl}
			d & = & \SI{11.5e-2}{m} \\
			f & = & \SI{1500}{Hz}
		\end{array}
		\right.\\
		\mathrm{A.N.~:}\quad
		\boxed{c = \SI{345}{m.s^{-1}}}
	\end{gather*}
}

\resetQ
\section{Interférences et écoute musicale}
\enonce{%
	\begin{minipage}{0.60\linewidth}
		La qualité de l'écoute musicale que l'on obtient avec une chaîne hi-fi
		dépend de la manière dont les enceintes sont disposées par rapport à
		l'auditaire. On dit qu'il faut absolument éviter la configuration représentée
		sur la figure~: présence d'un mur à une «~petite~» distance $D$ derrière
		l'auditaire.
	\end{minipage}
	\hfill
	\begin{minipage}{0.40\linewidth}
		\begin{center}
			\includegraphics[width=\linewidth]{ecoute_musicale-plain_white}
		\end{center}
	\end{minipage}
	\bigbreak
	Comme représenté sur la figure, l'onde issue de l'enceinte se réfléchit
	sur le mur. On note $c = \SI{342}{m.s^{-1}}$ la célérité du son dans l'air.
}

\QR{%
	Exprimer le décalage temporel $\tau$ qui existe entre les deux ondes
	arrivant dans l'oreille de l'auditaire~: l'onde arrivant directement et
	l'onde réfléchie.
}{%
	Chaque onde parcourt la distance enceinte -- auditaire directement,
	mais l'onde réfléchie parcourt en plus $2D$ entre l'auditaire et le mur.
	Ainsi, la célérité étant notée $c$, on a
	\[\tau = \frac{2D}{c}\]
}

\QR{%
	En déduire le déphasage $\D\f$ de ces deux ondes supposées
	sinusoïdales de fréquence $f$. La réflexion sur le mur ne s'accompagne
	d'aucun déphasage pour la vibration acoustique.
}{%
	La source étant similaire pour les deux ondes, la phase à l'origine
	des temps est la même~; de plus il est indiqué que la réflexion
	sur le mur n'implique pas de déphasage supplémentaire, donc le déphasage
	n'est dû qu'à la propagation. Ainsi, l'onde réfléchie a un déphasage
	\[\D\f_{r/i}(\Mr) = \w\tau = \frac{4\pi fD}{c}\]
}

\QR{%
	Expliquer pourquoi il y a risque d'atténuation de l'amplitude de
	l'onde pour certaines fréquences. Exprimer ces fréquences en fonction
	d'un entier $n$. Quelle condition devrait vérifier $D$ pour qu'aucune de
	ces fréquences ne soit dans le domaine audible. Est-elle réalisable~?
}{%
	Il peut y avoir une atténuation de l'amplitude si les deux ondes sont
	en opposition de phase, et donc que les interférences sont destructives,
	c'est-à-dire
	\begin{gather*}
		\D\f_{r/i}(\Mr) = (2n+1)\pi
		\Leftrightarrow
		\frac{4\cancel{\pi}fD}{c} = (2n+1)\cancel{\pi}
		\Leftrightarrow
		\boxed{f = (2n+1) \frac{c}{4D}}
	\end{gather*}
	avec $n\in\Nb$. Étant donné que le domaine audible s'étant de
	$\SIrange{20}{20e3}{Hz}$, il faudrait que la plus petite fréquence
	d'atténuation, celle avec $n=0$, soit au-delà de \SI{20}{kHz}~;
	autrement dit on cherche
	\begin{gather*}
		f_{\max} < \frac{c}{4D}
		\Leftrightarrow
		\boxed{D < \frac{c}{4f_{\max}}}
		\qavec
		\left\{
		\begin{array}{rcl}
			c        & = & \SI{342}{m.s^{-1}} \\
			f_{\max} & = & \SI{20}{kHz}
		\end{array}
		\right.\\
		\mathrm{A.N.~:}\quad
		\boxed{D < \SI{4.3}{mm}}
	\end{gather*}
	On est donc sûrx de ne pas avoir d'atténuation dans l'audible si on
	colle notre oreille au mur… ce qui est réalisable, mais correspond
	presque à ne pas avoir d'interférences du tout.
}

\QR{%
	Expliquer qualitativement pourquoi on évite l'effet nuisible en
	éloignant l'auditaire du mur.
}{%
	Quand $D$ augmente, l'onde réfléchie par le mur finit par avoir une
	amplitude faible devant l'onde directe étant donné qu'une onde sphérique
	voit son amplitude diminuer avec le rayon~: les interférences deviennent
	de plus en plus négligeables.
}

\resetQ
\section{Mesure de l'épaisseur d'une lame de verre}

\enonce{%
	On considère un dispositif de trous d'\textsc{Young} composé de deux trous $T_1$
	et $T_2$ séparés d'une distance $a = \SI{100}{\micro m}$. Ce dispositif est
	éclairé par une source ponctuelle S monochromatique de longueur d'onde dans
	l'air $\lambda = \SI{532}{nm}$ située sur l'axe optique. La figure
	d'interférences est observée sur un écran situé à une distance $D =
		\SI{1.00}{m}$ du plan des trous. Une lame de verre à faces parallèles
	d'épaisseur $e$ inconnue et d'indice $n_v = \num{1.57}$ est positionnée en
	sortie du trou $T_1$. L'indice optique de l'air est supposé égal à 1.

	\begin{center}
		\includegraphics[width=0.8\linewidth]{lame_verre-plain}
	\end{center}
}

\QR{%
	Montrer que la différence de marche $\delta(\Mr)$ en un point M de l'écran
	s'écrit
	\[\delta(\Mr) = \frac{ax}{D} + (n_v - 1)e\]
}{%
	En notant $({\rm SM})$ le chemin optique de S à M, la différence de
	marche en M est donnée par
	\begin{gather*}
		\de_{1/2}(\Mr)
		= \rm (ST_1M) - (ST_2M)
		= \rm \cancel{\rm (ST_1)} + (T_1M) - \cancel{\rm (ST_2)} - (T_2M)
	\end{gather*}
	La source étant sur l'axe optique et l'indice étant le même sur cette
	portion, on a \fbox{$\rm (ST_1) = (ST_2)$}. On se retrouve donc à calculer le
	chemin optique à partir des trous. Or, le chemin de T$_2$ à $M$ se fait
	dans l'air, donc \fbox{(T$_2$M) = T$_2$M}. En notant F$_1$ et F$_2$ les
	points d'entrée et de sortie du rayon lumineux dans la lame de verre
	tels que ${\rm F_1F_2}=e$, on a
	\begin{align*}
		({\rm T_1M})
		 & = \rm (T_1F_1) + (F_1F_2) + \rm (F_2M)                 \\
		 & = {\rm T_1F_1} + n_ve + {F_2M}                         \\
		 & = {\rm T_1F_1} + n_ve + {\rm F_1F_2-F_1F_2} + \rm F_2M \\
		 & = {\rm T_1F_1 + F_1F_2 + F_2M} + (n_v-1)e              \\
		 & = {\rm T_1M} + (n_v-1)e
	\end{align*}
	Avec $\rm T_1M = T_1F_1+F_1F_2+F_2M$. Autrement dit,
	\begin{gather*}
		\de_{1/2}(\Mr) = {\rm T_1M - T_2M} + (n_v-1)e
	\end{gather*}
	et avec le résultat usuel de différence de marche des trous
	d'\textsc{Young}, c'est-à-dire $\D L_{1/2}(\Mr) = ax/D$ (attention à la
	notation de la distance entre les fentes~!), on trouve bien
	\[\boxed{\de_{1/2}(\Mr) = \frac{ax}{D} + (n_v-1)e}\]
	Autrement dit, la différence de chemin optique est celle sans la
	lame à laquelle s'ajoute le retard pris par l'onde issue de T$_1$ qui
	va moins vite/parcourt une plus grande distance (à la célérité $c$) à
	cause du verre. On retrouve bien que si $n_v = 1$, la différence de
	chemin optique est celle attendue sans lame de verre.
}

\QR{%
	Déterminer la position $x_c$ sur l'écran de la frange centrale
	correspondant à $\delta(\Mr) = 0$. De quelle distance s'est déplacée
	cette frange par rapport au cas où la lame est absente~?
}{%
	\begin{gather*}
		\de_{1/2}(\Mr) = 0
		\Leftrightarrow
		\frac{ax_c}{D} - (n_v-1)e = 0
		\Leftrightarrow
		\boxed{x_c = \frac{(n_v-1)eD}{a}}
	\end{gather*}
	En l'absence de la lame de verre, la frange centrale serait sur l'axe
	optique, en $x = 0$~: dans cette situation, elle s'est donc décalée de
	$x_c$.
}

\QR{%
	Exprimer l'épaisseur $e$ de la lame en fonction de $x_c$ , $a$, $n_v$
	et $D$.
}{%
	On isole~:
	\[\boxed{e = \frac{ax_c}{D(n_v-1)}}
		\qavec
		\left\{
		\begin{array}{rcl}
			a   & = & \SI{100}{\micro m}    \\
			D   & = & \SI{1.00e9}{\micro m} \\
			n_v & = & \num{1.57}            \\
			x_c & = & \SI{28.5e7}{\micro m}
		\end{array}
		\right.\]
}

\QR{%
	Calculer $e$ pour $x_c = \SI{28.5}{cm}$.
}{%
	Application numérique~:
	\[\boxed{e = \SI{50.0}{\micro m}}\]
}

\QR{%
	Expliquer pourquoi en réalité la position de la frange centrale ne
	peut être connue que modulo l'interfrange $i$. Qu'est-ce que cela
	implique sur $e$~? L'expérience vous paraît-elle réalisable~?
}{%
	La frange centrale, en première approximation, n'est pas distinguable
	des autres franges brillantes correspondant également à des
	interférences constructives~: on a donc sa position modulo
	l'interfrange, soit
	\[x_c \equiv x_c \quad \left[ \frac{\lambda D}{a} \right]\]
	et ainsi
	\[e \equiv e \quad \left[ \frac{\lambda}{n_v-1} \right]\]
	Autrement dit, la mesure de $e$ n'est possible que modulo
	$\lambda/(n_v-1) = \SI{0.9}{\micro m}$~: la mesure de la lame de verre
	ne serait donc pas réalisable avec cette expérience, puisqu'elle est
	plus grande que \SI{0.9}{\micro m}.
}

\resetQ
\section{Contrôle actif du bruit en conduite}

\enonce{%
	\begin{minipage}{0.70\linewidth}
		On s'intéresse à un système conçu pour l'élimination d'un bruit in-
		désirable transporté par une conduite. Le bruit est détecté par un premier
		micro dont le signal est reçu par un contrôleur électronique. Le contrôleur,
		qui est le centre du système, envoie sur un haut-parleur la tension adéquate
		pour générer une onde de signal exactement opposé à celui du bruit de
		manière à ce que l'onde résultante au point A (voir figure ci-contre) et
		au-delà de A soit nulle.
	\end{minipage}
	\hfill
	\begin{minipage}{0.30\linewidth}
		\begin{center}
			\includegraphics[width=\linewidth]{conduite-plain_white}
		\end{center}
	\end{minipage}
}

\QR{%
	Exprimer, en fonction de $L$, $l$ et de la célérité $c$ du son, le
	temps disponible pour le calcul du signal envoyé sur le haut-parleur.
}{%
	Entre l'instant où le signal est détecté par le micro 1 et l'instant
	où ce signal passe en A, il s'écoule un temps égal à $L/c$. Pendant ce
	temps, il faut que le contrôleur calcule et produise le signal qu'il
	envoie dans le haut-parleur, et que ce signal se propage jusqu'à A, ce
	qui prend le temps $\ell/c$. Ainsi, le temps disponible pour le calcul
	est
	\[\boxed{\frac{L-\ell}{c}}\]
}

\QR{%
	On suppose le bruit sinusoïdal de pulsation $\omega$. On appelle
	$\f_1$ la phase initiale du signal détecté par le micro 1 et $\f_{\rm
			HP}$ la phase initiale du signal émis par le haut-parleur. Exprimer, en
	fonction de $\omega$, $c$, $L$ et $l$, la valeur que doit avoir $\D\f =
		\f_{\rm HP} - \f_1$
}{%
	La phase du signal de bruit arrivant en A est
	\[
		\f_{\rm bruit} = \f_1-kL
	\]
	La phase du signal de correction arrivant en A est
	\[
		\f_{\rm corr} = \f_{\rm HP} -k\ell
	\]
	Pour avoir interférences destructives, il faut que $\f_{\rm corr} =
		\f_{\rm bruit} + \pi$, c'est-à-dire
	\[
		\boxed{\D\f_{c/b}({\rm A})
			= \f_{\rm HP} - \f_1
			= \frac{\w}{c}(\ell-L)+\pi}
	\]
}

\QR{%
	L'onde émise par le haut-parleur se propage dans la conduite dans les
	deux sens à partir de A. Expliquer l'utilité du micro 2.
}{%
	Le micro 1 capte un signal qui est la superposition du bruit et du
	signal émis par le haut-parleur se propageant à partir de A vers
	l'amont. Le micro 2 donne un contrôle du résultat et permet la
	détermination du meilleur signal de correction.
}

\resetQ
\section{Mesure de la vitesse du son avec des trous d'\textsc{Young}}

\enonce{%
	On considère un dispositif composé de deux trous d'\textsc{Young} percés dans un
	écran opaque et séparés d'une distance $a = \SI{10.0}{cm}$. Une onde ultrasonore
	de fréquence $f = \SI{40}{kHz}$ est envoyée en direction des trous. L'amplitude
	de l'onde en sortie des trous est mesurée en utilisant un récepteur qui peut
	être translaté suivant un axe (O$x$) parallèle à la direction des trous et situé
	à une distance $D = \SI{50.0}{cm}$ du plan des trous. Le dispositif expérimental
	est représenté sur la figure 1. Par la suite, les valeurs de $D$ et $a$ sont
	supposées connues avec une précision de \SI{1}{mm} et l'incertitude-type sur la
	valeur de $f$ est supposée négligeable.

	\begin{center}
		\includegraphics[width=\linewidth]{young_son-plain}
	\end{center}
}

\QR{%
	En supposant que la condition $D \gg a,\,\,x$ est vérifiée, donner
	l'expression de l'interfrange $i$ correspondant à la distance sur l'axe
	(O$x$) entre deux interférences constructives.
}{%
	L'interfrange dans une expérience de trous d'\textsc{Young} dont les
	fentes sont séparées de $a$ est
	\[\boxed{i = \frac{\lambda D}{a}}\]
}

\enonce{%
	Le résultat de la mesure de l'amplitude du signal électrique délivré par le
	récepteur en différentes positions sur l'axe (O$x$) est représenté sur la figure
	2.
}

\QR{%
	À partir de la figure 2, estimer la valeur de l'interfrange ainsi que
	son incertitude-type.
}{%
	On mesure avec une règle graduée au millimètre pour mesurer
	(conversion d'échelle comprise) $4i = \SI{17.1}{cm}$. La précision
	est ici limitée par l'écart entre deux positions de mesure du
	détecteur. Avec l'échelle de la figure et le facteur $1/\sqrt{3}$, on
	trouve l'incertitude-type de mesure $u_{4i} = \SI{0.8}{cm}$. Ainsi,
	\[\boxed{i = \SI{4.3\pm0.2}{cm}}\]
}

\QR{%
	En déduire une estimation de la célérité $c$ du son dans l'air ainsi
	que de son incertitude-type. On néglige toute incertitude sur la
	fréquence $f$.
}{%
	En utilisant l'expression de l'interfrange et de $\lambda = c/f$, on a
	\[
		c = \lambda f = \frac{fa}{D}
		\Leftrightarrow
		c = \SI{3.4e2}{m.s^{-1}}
	\]
	On détermine son incertitude avec la formule de propagation~:
	\begin{gather*}
		\frac{u(\lambda)}{\lambda}
		= \sqrt{\left(\frac{u(i)}{i}\right)^2 +
			\left(\frac{u(a)}{a}\right)^2 +
			\left(\frac{u(D)}{D}\right)^2}
		\qavec
		\left\{
		\begin{array}{rcl}
			\lambda & = & \SI{8.4}{mm}                               \\
			i       & = & \SI{4.3}{cm}                               \\
			u(i)    & = & \SI{0.2}{cm}                               \\
			a       & = & \SI{10.0}{cm}                              \\
			u(a)    & = & \frac{\SI{1}{mm}}{\sqrt{3}} = \SI{0.6}{mm} \\
			D       & = & \SI{50.0}{cm}                              \\
			u(D)    & = & \frac{\SI{1}{mm}}{\sqrt{3}} = \SI{0.6}{mm}
		\end{array}
		\right.\\
		\mathrm{A.N.~:}\quad
		\boxed{c = \SI{3.4\pm0.1e2}{m.s^{-1}}}
	\end{gather*}
}

\enonce{%
	Un phénomène de diffraction est observé lorsqu'une onde traverse un trou de
	rayon $r \approx \lambda$. Le faisceau en sortie du trou présente alors un
	demi-angle d'ouverture $\theta$ tel que $\sin(\theta) \approx \lambda/2r$.
}

\QR{%
	À partir de la figure 2, estimer l'ordre de grandeur du rayon des
	trous utilisés dans l'expérience.
}{%
	La diminution de l'amplitude des interférences lorsque $x$ augmente
	est due au phénomène de diffraction par un trou d'\textsc{Young}. Sur la
	figure 2, on peut voir que l'amplitude des interférences s'annule pour
	$x_a \approx \SI{15}{cm}$. Or, d'après la figure 1, $\tan(\theta) =
		x_a/D$~; ainsi, en combinant avec $\sin(\theta) \approx \lambda/2r$ et
	avec l'approximation des petits angles ($\tan(\theta) \approx \theta$ et
	$\sin(\theta) \approx \theta$), on a
	\[
		\frac{x_a}{D} \approx \frac{\lambda}{2r}
		\Leftrightarrow
		\boxed{r \approx \frac{\lambda D}{2x_a} \approx \SI{1.4}{cm}}
	\]
}

\resetQ
\section{Interférences ultrasonores sur un cercle}

\enonce{%
	\begin{minipage}{0.80\linewidth}
		Deux émetteurs E$_1$ et E$_2$ émettent des ondes ultrasonores de même
		fréquence $f = \SI{40}{kHz}$ (ce qui correspond à une longueur d'onde
		$\lambda = \SI{8.5}{nm}$) et en phase. On note O le milieu du segment [E$_1$
				E$_2$] de longueur $a = \SI{4}{cm}$, et (O$x$) l'axe situé sur la médiatrice
		de ce
		segment. On déplace le microphone sur un grand cercle de rayon $R =
			\SI{0.5}{m}$
		et on relève l'évolution de l'amplitude mesurée en fonction de l'angle
		$\theta$ que fait la direction $\vec{\rm OM}$ avec l'axe (O$x$).
	\end{minipage}
	\hfill
	\begin{minipage}{0.20\linewidth}
		\begin{center}
			\includegraphics[width=\linewidth]{cercle-plain_white}
		\end{center}
	\end{minipage}
}

\begin{blocQR}
	\item
	\QR{%
		Faire une figure faisant apparaître les points O, E$_1$, E$_2$
		et M, pour un petit angle $\theta$ non nul.
	}{%
		On a
		\begin{center}
			\includegraphics[width=0.2\linewidth]{cercle_corr-white}
		\end{center}
	}

	\QR{%
		Tracer l'arc de cercle de centre M passant par E$_2$. On note
		H son intersection avec la droite (E$_1$M). Que représente
		E$_1$H~?
	}{%
		E$_1$H est la différence ${\rm E_1M - E_2M} = r_1-r_2 = \D
			L_{1/2}(\Mr)$ avec les notations du cours~; autrement dit, c'est
		\fbox{la différence de marche} entre les deux ondes.
	}

	\QR{%
		Puisque $R \gg a$, on peut assimiler H et le projeté
		orthogonal de E$_2$ sur (E$_1$M). En déduire une expression du
		déphasage entre les ondes reçues en M en fonction de $\theta$,
		$a$ et $\lambda$.
	}{%
		En raisonnant dans le triangle $\rm E_1E_2H$, considéré rectangle, on
		a ${\rm E_1H} = a \sin\theta$. D'où le déphasage~:
		\[\boxed{\D\f_{2/1}(\Mr) = \frac{2\pi a\sin\theta}{\lambda}}\]
	}

	\QR{%
		Quelles sont, dans l'intervalle \SIrange{-30}{30}{\degree},
		les valeurs de $\theta$ où on observe un maximum d'amplitude~?
	}{%
		L'amplitude est maximale pour des interférences constructives, soit
		pour $\D\f_{2/1}(\Mr) = 2p\pi$ avec $p\in\Zb$~; sur $\theta$ ça donne
		donc
		\[
			\boxed{\sin\theta = p \frac{\lambda}{a}}
			\Leftrightarrow
			\theta = \asin(p \frac{\lambda}{a})
		\]
		On regarde donc quels sont les ordres d'interférences $p$ tels que
		$\theta \in \SIrange{-30}{30}{\degree}$~:
		\begin{itemize}
			\item $p=0 \Rightarrow \theta = \ang{0;;}$, soit un maximum
			      pour tout l'axe $x$~: c'était attendu étant donné les symétries
			      du problème~;
			\item $p = \pm 1 \Rightarrow \theta = \pm\ang{12;;}$, donnant deux
			      points symétriques par rapport à (O$x$)~;
			\item $p = \pm 2 \Rightarrow \theta = \pm\ang{25;;}$, pratiquement
			      le double des valeurs précédentes.
		\end{itemize}
		$p > 2$ donne des valeurs en-dehors de l'intervalle.
	}

	\item
	\QR{%
		Sur l'intervalle précédent, quelles sont les positions où un
		minimum d'amplitude est attendu~?
	}{%
		On a interférences destructives si $\D\f_{2/1}(\Mr) =
			(2p+1)\pi$, soit
		\[
			\boxed{\sin\theta = \left(p+\frac{1}{2}\right) \frac{\lambda}{a}}
			\Leftrightarrow
			\theta = \asin(\left(p+\frac{1}{2}\right) \frac{\lambda}{a})
		\]
		\begin{itemize}
			\item $p=0 \Rightarrow \theta = \pm\ang{6;;}$~;
			\item $p=1 \Rightarrow \theta = \pm\ang{19;;}$.
		\end{itemize}
	}

	\QR{%
		Si les ondes émises ont même amplitude, quelle est la valeur
		minimale d'amplitude pour le signal somme~?
	}{%
		Pour des ondes reçues avec la même amplitude, l'opposition de
		phase conduit à une annulation totale de l'amplitude somme.
	}

\end{blocQR}

\end{document}
