\documentclass[a4paper, 12pt, final, garamond]{book}
\usepackage{cours-preambule}

\raggedbottom

\makeatletter
\renewcommand{\@chapapp}{M\'ecanique -- chapitre}
\makeatother

\begin{document}
\setcounter{chapter}{3}

\chapter{Correction TD entra\^inement}
\section{Chute sur corde en escalade}
\begin{enumerate}
    \item Pendant la chute libre, la grimpeuse ne subit que l'action du poids,
        qui est conservatif. On peut donc utiliser le TEM, avec~:
        \begin{itemize}[label=$\diamond$]
            \litem{Au début de la chute libre~:} $z = h$, $v=0 \Ra \Ec_{p,p} = mgh$ et
                $\Ec_c = 0$
            \litem{À la fin de la chute libre~:} $z = 0$, $v=v \Ra \Ec_{p,p} = 0$ et
                $\Ec_c = mv^2/2$.
        \end{itemize}
        D'où
        \begin{gather*}
            \frac{1}{2}mv^2 = mgh
            \Lra
            \boxed{v = \sqrt{2gh}}
            \qavec
            \left\{
                \begin{array}{rcl}
                    g & = & \SI{10}{m.s^{-2}}\\
                    h & = & \SI{5}{m}
                \end{array}
            \right.\\
            \AN
            \boxed{v = \SI{10}{m.s^{-1}}}
        \end{gather*}
    \item On peut utiliser le TEM entre le point tout en haut et le point le
        plus bas, ou entre le point O et le point le plus bas. Faisons le
        premier cas~:
        \begin{itemize}[label=$\diamond$]
            \litem{Au début de la chute libre~:} $z = h$, $v=0 \Ra \Ec_{p,p} = mgh$ et
                $\Ec_c = 0$
            \litem{À la fin de la chute amortie~:} $z = -\D l$, $v=0 \Ra \Ec_{p,p}
                = -mg\D l$, \fbox{$\Ec_{p,el} = k\D l^2/2$} et $\Ec_c = 0$.
        \end{itemize}
        Ainsi,
        \begin{gather*}
            mgh = \frac{1}{2}k\D l^2 + mg(-\D l)
            \Lra
            mg(h+\underbracket[1pt]{\cancel{\D l}}_{\ll h}) = \frac{1}{2}k\D l^2
            \Lra
            \boxed{\D l = \sqrt{\frac{2mgh}{k}}}
            \qed
        \end{gather*}
        La solution trouvée est plausible~: homogène, augmente avec $m$, $h$ et
        $g$ mais diminue avec $k$.
    \item En norme, une force de rappel s'exprime $F = k(\ell -\ell_0)$, soit
        ici
        \begin{gather*}
            F_{\max} = k\D l = \sqrt{2mgh\,k} = \sqrt{2mgh\frac{\a}{L_0}}
            \\\Lra
            \boxed{F_{\max} = \sqrt{2mg\a f}}
            \qed
        \end{gather*}
    \item On fait l'application numérique~:
        \begin{gather*}
            \qavec
            \left\{
                \begin{array}{rcl}
                    m & = & \SI{50}{kg}\\
                    g & = & \SI{10}{m.s^{-2}}\\
                    \a & = & \SI{5.0e4}{N}\\
                    f & = & \num{1}
                \end{array}
            \right.\\
            \AN
            \boxed{F_{\max} = \SI{10}{kN}}
        \end{gather*}
        Il n'y a donc pas de risque aggravé pour la grimpeuse avec cette chute.
    \item Dans le premier cas, $f_1 = 2$~; dans le second, $f_2 = \num{0.5}$. Or,
        $F_{\max}$ évolue en $\sqrt{f}$, donc plus $f$ augmente plus la force
        subie augmente~: le premier cas est donc 2 fois plus dangereux que le
        premier~!
\end{enumerate}

\section{Pendule électrique}
\begin{enumerate}
    \item Pour exprimer la distance AM, on la décompose par des vecteurs connus
        et on pourra prendre la norme du vecteur $\vv{\rm AM}$ avec
        $\sqrt{x_{\rm AM}{}^2 + y_{\rm AM}{}^2}$, ou $\sqrt{\vv{\rm
        AM}\cdot\vv{\rm AM}}$. Notamment, $\vv{\rm AM} = \vv{\rm AO} + \OM$.
        \begin{minipage}[t]{0.33\linewidth}
            ~%\vspace{-24pt}
            \begin{center}
                \includegraphics[width=\linewidth]{pendule_elec_am}
                \captionsetup{justification=centering}
                \captionof{figure}{Détermination de AM}
                \label{fig:pendule_am}
            \end{center}
        \end{minipage}
        \hfill
        \begin{minipage}[t]{0.65\linewidth}
            Il faut donc décomposer $\vv{\rm AO}$ et $\OM$ sur la même base,
            comme on le fait pour le poids sur un plan incliné. En effet,
            \begin{align*}
                \vv{\rm AO} &= 2R\uz\\
                \OM &= R\ur
            \end{align*}
            mais on ne peut pas sommer les deux dans des bases différentes.
            Décomposons $\ur$ sur $(\ux, \uz)$~: on trouve
            \[\ur = \sin\tt\ux -\cos\tt\uz\]
            Ainsi,\vspace{-24pt}
            \begin{align*}
                \vv{\rm AM} &= \vv{\rm AO} + \OM
                \\\Lra
                \vv{\rm AM} &= \mqty(R\sin\tt\\2R-R\cos\tt)
                \\\Ra
                \norm{\vv{\rm AM}} &= \sqrt{R^2\sin^2\tt + (2R-R\cos\tt)^2}
                \\\Lra
                {\rm AM} &= \sqrt{R^2\sin^2\tt + 4R^2 -2R^2\cos\tt +R^2\cos^2\tt}
                \\\Lra
                {\rm AM} &= \sqrt{5R^2 -2R^2\cos\tt}
                \qavec
                \cos^2\tt + \sin^2\tt = 1
                \\\Lra
                \Aboxed{{\rm AM} &= R\sqrt{5-2\cos\tt}}
                \qed
            \end{align*}
        \end{minipage} \bigbreak
    \item Une force est conservative si son travail élémentaire s'exprime sous
        la forme $-\dd\Ec_p$. Calculons son travail élémentaire~:
        \begin{align*}
            \de W(\Ff_e) &= \Ff_e\cdot\dd{\vv{\rm AM}}
            \\\Lra
            \de W(\Ff_e) &= \frac{k}{\rm AM^3}{\vv{\rm AM}}\cdot\dd{\vv{\rm AM}}
            \\\Lra
            \de W(\Ff_e) &= \frac{k}{\rm AM^3} \underbracket[1pt]{\norm{\vv{\rm
                AM}}}_{=\rm AM} \norm{\dd{\vv{\rm AM}}}
                \underbracket[1pt]{\cos(\vv{\rm AM}, \dd{\vv{\rm AM}})}_{=1}
            \\\Lra
            \de W(\Ff_e) &= \frac{k}{\rm AM^2} \cancel{\underbracket[1pt]{\frac{\rm
                AM}{\rm AM}}_{=1}} \dd{\rm AM}
            \\\Lra
            \de W(\Ff_e) &= -k\dd(\frac{1}{\rm AM})
            \\\Lra
            \Aboxed{\de W(\Ff_e) &= -\dd{\Ec_{p,e}}}
            \\\qavec
            \Aboxed{\Ec_{p,e} &= \frac{k}{\rm AM} = \frac{k}{R\sqrt{5-4\cos\tt}}}
            \qed
        \end{align*}
    \item La boule M a également une énergie potentielle de pesanteur. En
        prenant O comme origine de l'altitude, l'altitude de la boule M $z(\tt)$
        s'exprime
        \[z(\tt) = -R\cos\tt\]
        Ainsi,
        \begin{align*}
            \Ec_p(\tt) &= \Ec_{p,p}(\tt) + \Ec_{p,e}(\tt)
            \\\Lra
            \Aboxed{\Ec_p(\tt) &= \frac{k}{R\sqrt{5-4\cos\tt}} - mgR\cos\tt}
            \qed
        \end{align*}
    \item On observe en tout 5 positions d'équilibres~: deux stables dans les
        puits de potentiel vers $\pm\SI{1}{rad}$, et trois instables (maxima
        locaux d'énergie potentielle) en $-\pi$, $0$ et $\pi$.
    \item Le mouvement du pendule ne se fait que dans les zones du graphique où
        $\Ec_p < \Ec_m$. On distingue donc 4 cas~:
        \[
            \begin{array}{lrcll}
                \textbf{Cas 1} &\quad \SI{0}{J} &< \Ec_m &< \SI{3.5e-2}{J} & \Ra
                \text{pas de mouvement}
                \\
                \textbf{Cas 2} &\quad \SI{3.5e-2}{J} &< \Ec_m &< \SI{4.4e-2}{J} & \Ra
                \text{oscillations $\approx$ position stable}
                \\
                \textbf{Cas 3} &\quad \SI{4.4e-2}{J} &< \Ec_m &< \SI{5.4e-2}{J} & \Ra
                \text{mouvement périodique entre $\Ec_{p,\max}$}
                \\
                \textbf{Cas 4} &\quad \SI{5.4e-2}{J} &< \Ec_m &< +\infty & \Ra
                \text{mouvement révolutif~: tours à l'infini}
            \end{array}
        \]
        \begin{center}
            \includegraphics[width=.7\linewidth]{pendule_elec_ep_corr}
            \captionof{figure}{Mouvement selon $\Ec_m$}
            \label{fig:pend_elec_em}
        \end{center}
\end{enumerate}

\section{Recul d'un canon}
\begin{enumerate}
    \item Au repos, la tension du ressort est nulle, donc $\ell = L_0$.
    \item 
        \begin{itemize}[label=$\diamond$, leftmargin=10pt]
            \litem{Système~:} \{canon\}, repéré par G de masse $M$
            \litem{Référentiel~:} $\Rc\ind{sol}$, supposé galiléen
            \litem{Repère~:} mouvement horizontal donc cartésien, $(\Or,\ux,\uz)$
                avec $\uz$ vertical ascendant
            \litem{Repérage~:}
                \begin{align*}
                    \OG &= x\ux\\
                    \vf &= \xp\ux\\
                    \af &= \xpp\ux
                \end{align*}
            \litem{BDF~:}
                \[
                    \begin{array}{ll}
                        \textbf{Poids} & \Pf = -mg\uz\\
                        \textbf{Réaction} & \Nf = N\uz\\
                        \textbf{Ressort} & \Ff = -k_1(x-L_0)\ux
                    \end{array}
                \]
                Le poids et la tension du ressort sont conservatives, et la
                réaction du sol ne travaille pas~: on a donc un système
                conservatif, et on applique simplement le TEM~:
            \litem{Au moment du tir~:} $v=v_c$, $x=L_0 \Ra \Ec_{c,0} =
                Mv_c{}^2/2$ et $\Ec_{p,el} = k_1(L_0-L_0)^2/2 = 0$
            \litem{Après le recul~:} $v=0$, $x=L_0-d \Ra \Ec_{c,f} = 0$ et
                $\Ec_{p, el} = k_1d^2/2$
            \litem{TEM~:}
                \begin{align*}
                    \frac{1}{2}k_1d^2 &=
                    \frac{1}{2}M\underbracket[1pt]{v_c}_{\mathclap{=mv_0/M}}{}^2
                    \\\Lra
                    d^2 &= \frac{m^2}{k_1M}v_0{}^2
                    \\\Lra\,
                    \Aboxed{d &= \frac{m}{\sqrt{k_1M}}v_0}
                    \qed
                    \\\Lra\,
                    \Aboxed{k_1 &= \frac{m^2v_0{}^2}{d^2M}}
                    \qed
                    \qavec
                    \left\{
                        \begin{array}{rcl}
                            m & = & \SI{2.0}{kg}\\
                            M & = & \SI{800}{kg}\\
                            v_0 & = & \SI{600}{m.s^{-1}}\\
                            d & = & \SI{1.0}{m}
                        \end{array}
                    \right.\\
                    \AN
                    \Aboxed{k_1 &= \SI{1800}{N.m^{-1}}}
                \end{align*}
        \end{itemize}
    \item Avec le \textbf{PFD} et en projetant sur $\ux$ (on a $N = mg$ sur 
        $\uz$)~:
        \begin{align*}
            M\xpp &= -k_1(x-L_0)
            \\\Lra
            \xpp + \w_0{}^2x &= \w_0{}^2L_0
            \qavec
            \w_0 = \sqrt{\frac{k_1}{M}}
            \\\Ra
            x(t) &= A\cos(\w_0t+\f) + L_0
            \shortintertext{Or,}
            x(t=0) = L_0 &\Ra A\cos\f = 0
            \shortintertext{On choisit $\f = -\pi/2$, et ainsi}
            x(t) &= A\sin(\w_0t) + L_0
            \\\Ra
            \xp(t) &= A\w_0\cos(\w_0t)
            \shortintertext{Or,}
            \xp(t=0) &= -\frac{m}{M}v_0
            \\\Ra
            A &= -\frac{m}{M} \frac{v_0}{\w_0}
            \\\Ra
            \Aboxed{x(t) = -\frac{mv_0}{\sqrt{k_1M}}\sin(\w_0t)+L_0}
            \qed
        \end{align*}
        On obtient alors $d$ comme étant l'amplitude du sinus, c'est-à-dire le
        résultat précédent.
    \item On vient donc de démontrer qu'avec un seul ressort, le canon va
        osciller et donc après le recul, il va repartir vers l'avant.
        L'amplitude va diminuer petit à petit à cause des frottements
        inéluctables, mais le temps avant immobilisation sera important~: on a
        donc intérêt à ajouter une force de frottements visqueux.
    \item Le système n'est plus conservatif, et la variation d'énergie mécanique
        est maintenant égale à l'énergie absorbée par le dispositif de freinage,
        c'est-à-dire
        \[\D\Ec_m = \Ec_{m,f} - \Ec{m,i} = -\Ec_a\]
        puisque l'énergie cinétique doit décroître et que $\Ec_a$ est positive.
        Or, initialement et finalement,
        \[
            \Ec_{m,i} = \Ec_c = \frac{1}{2}Mv_c{}^2
            \qet
            \Ec_{m,f} = \Ec_p = \frac{1}{2}k_2d^2
        \]
        Soit
        \begin{align*}
            \frac{1}{2}k_2d^2 - \frac{1}{2}Mv_c{}^2 &= -\Ec_a
            \\\Lra
            k_2 &= \frac{1}{d^2}\left(Mv_c{}^2 -2\Ec_a\right)
            \\\Lra
            \Aboxed{k_2 &= \frac{1}{d^2}\left(\frac{m^2}{M}v_0{}^2
            -2\Ec_a\right)}
            \qed
            \qavec
            \left\{
                \begin{array}{rcl}
                    m     & = & \SI{2.0}{kg}\\
                    M     & = & \SI{800}{kg}\\
                    v_0   & = & \SI{600}{m.s^{-1}}\\
                    \Ec_a & = & \SI{778}{J}
                \end{array}
            \right.\\
            \AN
            \Aboxed{k_2 &= \SI{244}{N.m^{-1}}}\\
            \text{De plus, }
            \Aboxed{\w_0 &= \sqrt{\frac{k_2}{M}}}
            \qavec
            \left\{
                \begin{array}{rcl}
                    k_2 & = & \SI{244}{N.m^{-1}}\\
                    M   & = & \SI{800}{kg}
                \end{array}
            \right.\\
            \AN
            \Aboxed{\w_0 &= \SI{0.55}{rad.s^{-1}}}
        \end{align*}
    \item On reprend la question 3) mais avec la force de frottements, pour
        obtenir l'équation d'un oscillateur amorti~:
        \[\xpp + \frac{\lb}{M}\xp + \w_0{}^2x = \w_0{}^2L_0\]
        Le discriminant de l'équation caractéristique associée est
        \[\D = \left( \frac{\lb}{M} \right)^2 - 4\w_0{}^2\]
        et on a un régime critique quand ce discriminant est nul~; soit
        \begin{align*}
            \Aboxed{\lb &= 2M\w_0}
            \qed
            \qavec
            \left\{
                \begin{array}{rcl}
                    M    & = & \SI{800}{kg}\\
                    \w_0 & = & \SI{0.55}{rad.s^{-1}}
                \end{array}
            \right.\\
            \AN
            \Aboxed{\lb &= \SI{884}{kg.s^{-1}}}
        \end{align*}
    \item Avec le régime critique, on a
        \begin{align*}
            x(t) &= (At+B)\exp(-\frac{\lb t}{2M}) + L_0
            \shortintertext{Or,}
            x(0) = 0 &\Ra \boxed{B=0}
            \\\Ra
            \xp(t) &= A\exp(-\frac{\lb t}{2M})\left(1-\frac{\lb}{2M}t\right)
            \shortintertext{Or,}
            \xp(0) = v_c &\Ra \boxed{A = v_c}
            \\\Ra\quad
            \Aboxed{\xp(t) &= -\frac{m}{M}\exp(-\frac{\lb t}{2M})
            \left(1-\frac{\lb}{2M}t\right)}
            \\\text{et}\quad
            \Aboxed{x(t) &= -\frac{m}{M}v_0t\exp(-\frac{\lb t}{2M}) + L_0}
            \qed
        \end{align*}
        Le recul est maximal quand la vitesse s'annule, soit
        \begin{gather*}
            t_m = \frac{2M}{\lb} = \SI{1.8}{s}
        \end{gather*}
        On calcule $x(t_m)$, sachant qu'on a par définition $x(t_m) = L_0 - d$~:
        \begin{align*}
            x(t_m) &= -\frac{m}{M}v_0\frac{2M}{\lb}\exr^{-1} + L_0
            \\\Lra
            L_0 - d &= L_0 - \frac{2mv_0}{\lb \exr}
            \\\Lra
            \Aboxed{d &= \frac{2mv_0}{\lb\exr}}
            \qed
        \end{align*}
        et l'application numérique donne
        \[\boxed{d = \SI{1.0}{m}}\]
        On retrouve bien la distance de recul précédente, mais cette fois il n'y
        a pas d'oscillation~! Cahier des charges rempli.
\end{enumerate}

\section{Positions d'équilibre d'un anneau sur un cercle}
\begin{enumerate}
    \item
        \begin{minipage}[t]{0.25\linewidth}
            ~
            \begin{center}
                \includegraphics[width=\linewidth]{anneau_cercle_ressort-proj}
                \captionsetup{justification=centering}
                \captionof{figure}{Détermination de $\ell$}
                \label{fig:anncercproj}
            \end{center}
        \end{minipage}
        \hfill
        \begin{minipage}[t]{0.70\linewidth}
            On peut réutiliser la relation de \textsc{Chasles} pour écrire
            $\vv{\rm AM} = \vv{\rm AO} + \OM$ et déterminer la distance en prenant
            la norme, mais ici une simple utilisation du théorème de
            \textsc{Pythagore} suffit. On projette M sur l'axe $x$ pour avoir
            \begin{align*}
                \ell^2 &= (R+R\cos\tt)^2 + (R\sin\tt)^2
                \\\Lra
                \ell^2 &= R^2 + 2R^2\cos\tt + R^2(\cos^2\tt + \sin^2\tt)
                \\\Lra
                \ell^2 &= 2R^2(1+\cos\tt)
                \\\Lra
                \Aboxed{\ell &= R\sqrt{2(1+\cos\tt)}}
                \qed
            \end{align*}
        \end{minipage} \bigbreak
    \item L'énergie potentielle totale $\Ec_p$ est constituée de l'énergie
        potentielle de pesanteur de l'anneau et de l'énergie potentielle
        élastique du ressort. Pour $\Ec_{p,p}$ avec origine en O, on a une
        altitude $R\sin\tt$~; pour $\Ec_{p,el}$ on a la différence de longueur à
        a vide $\ell -\ell_0$ avec $\ell_0 = 0$, d'où
        \begin{align*}
            \Ec_p &= \Ec_{p,p} + \Ec_{p,el}
            \\\Lra
            \Ec_p &= mgR\sin\tt + \frac{k}{2}\ell^2
            \\\Lra
            \Aboxed{\Ec_p &= mgR\sin\tt + kR^2(1+\cos\tt)}
            \qed
        \end{align*}
    \item On trouve les positions d'équilibre de l'anneau en trouvant les angles
        $\tt_{\eq}$ tels que la dérivée de $\Ec_p$ s'annule, soit
        \begin{align*}
            \eval{\dv{\Ec_p}{\tt}}_{\tt_{\eq}} &= -kR^2\sin\tt_{\eq} +
            mgR\cos\tt_{\eq} = 0
            \\\Lra
            \sin\tt_{\eq} &= \frac{mg\cancel{R}}{kR^{\cancel{2}}}\cos\tt_{\eq}
            \\\Lra
            \tan\tt_{\eq} &= \frac{mg}{kR}
            \\\Lra
            \boxed{\tt_{\eq,1} = \arctan(\frac{mg}{kR})}
            \quad&\text{et}\quad
            \boxed{\tt_{\eq,2} = \pi + \arctan(\frac{mg}{kR})}
            \qed
        \end{align*}
        avec $\tt_{\eq,1}$ compris entre 0 et \ang{90}, et $\tt_{\eq,2}$ compris
        entre 180 et \ang{270}.
    \item On étudie la stabilité des positions en évaluant la dérivée seconde de
        $\Ec_p$ en ce point et en vérifiant son signe. On obtient
        \begin{align*}
            \eval{\dv[2]{\Ec_p}{\tt}}_{\tt_{\eq}} &= -kR^2\cos\tt_{\eq}
            -mgR\sin\tt_{\eq}
            \\\Lra
            \eval{\dv[2]{\Ec_p}{\tt}}_{\tt_{\eq}} &= -\left(kR^2 +
                \frac{m^2g^2}{k}\right)\cos\tt_{\eq}
        \end{align*}
        en utilisant les résultats précédents sur la dérivée première de
        $\Ec_p$. L'intérieur de la parenthèse étant positif, le signe de cette
        dérivée seconde est opposé à celui du cosinus de la position
        d'équilibre. Or,
        $\cos\tt_{\eq,1} > 0$ et $\cos\tt_{\eq,2} < 0$, donc
        \begin{gather*}
            \boxed{\eval{\dv[2]{\Ec_p}{\tt}}_{\tt_{\eq,1}} < 0}
            \qet
            \boxed{\eval{\dv[2]{\Ec_p}{\tt}}_{\tt_{\eq,2}} > 0}
            \qed
        \end{gather*}
        La première position est donc instable, et la seconde stable.
\end{enumerate}
\begin{figure}[htbp]
    \centering
    \includegraphics[width=.5\linewidth]{anneau_cercle_ressort-corr_eq}
    \captionsetup{justification=centering}
    \caption{Positions d'équilibre du système}
    \label{fig:anncerccorr}
\end{figure}

\section{Oscillateur de \textsc{Landau}}
\begin{enumerate}
    \item Comme l'anneau est contraint de se déplacer sur une ligne horizontale,
        son énergie potentielle de pesanteur est constante. Ainsi, la seule
        contribution à l'énergie potentielle est d'origine élastique,
        \[\Ec_p(x) = \frac{1}{2}k({\rm AM} -\ell_0)^2\]
        La longueur AM s'exprime à partir du théorème de \textsc{Pythagore},
        \[
            {\rm AM}^2 = a^2 + x^2
            \quad\text{d'où}\quad
            \boxed{\Ec_p(x) = \frac{1}{2}k\left(\sqrt{a^2+x^2} - \ell_0\right)^2}
        \]
    \item Qualitativement, il est assez simple de comprendre pourquoi certaines
        courbes font apparaître deux minima et d'autre un seul. Si $a < \ell_0$,
        alors deux positions de M, symétriques par rapport à O sont telles que
        ${\rm AM} = \ell_0$. Dans ce cas, l'énergie potentielle élastique est
        nulle. Au contraire, si $a > \ell_0$, le ressort est toujours étiré et
        l'énergie potentielle élastique jamais nulle.
        \begin{bexem}{}
            \bfseries Ce raisonnement se retrouve tout à fait sur l'expression
            mathématique de $\Ec_p$~!
        \end{bexem}
        Ainsi on peut identifier la courbe en \textbf{pointillés violets au cas
        $\mathbf{a_4 = 3\ell_0}$}. La courbe en \textbf{points verts} ne fait
        apparaître qu'un seul minimum, mais son énergie potentielle est nulle~:
        elle correspond au cas $\mathbf{a_3 = \ell_0}$. Enfin, il reste à
        identifier les deux dernières courbes, ce qui peut se faire à partir de
        la valeur de l'énergie potentielle en $x = 0$. Elle est plus élevée sur
        la courbe bleue que sur la courbe rouge, signe que le ressort est
        davantage comprimé. On en déduit que la \textbf{courbe bleue} est celle
        du cas $\mathbf{a_1 = \ell_0/10}$ alors que la courbe \textbf{rouge}
        correspond à $\mathbf{a_2 = \ell_0/3}$. \bigbreak

    \item Quelles que soient les conditions initiales, le mouvement est borné
        car $\Ec_p$ diverge en $\pm\infty$, et il est donc périodique. Dans le
        cas $a \leq \ell_0$, si les conditions initiales sont telles que $\Ec_m
        < \Ec_p (x = 0)$, alors le mouvement est restreint à un côté $x < 0$ ou
        $x > 0$ car l'anneau n'a pas assez d'énergie pour franchir la barrière
        de potentiel en $x = 0$. Si les conditions initiales sont en revanche
        telles que $\Ec_m > \Ec_p (x = 0)$, le mouvement a lieu de part et
        d'autre de la barrière, et il est symétrique car le profil d'énergie
        potentielle l'est. C'est également le cas si $a > \ell_0$, et ce quelles
        que soient les conditions initiales. \bigbreak

    \item La condition initiale est très simple à déterminer~: c'est le seul
        point commun à toutes les trajectoires de phase. Compte tenu de la
        symétrie des portraits de phase et des profils d'énergie potentielle,
        seule la norme de la vitesse peut être déterminée. On trouve
        \[
            x_0 = \num{0.4}\ell_0
            \qet
            \xp_0 = \num{0.5}\ell_0 \sqrt{\frac{k}{m}}
        \]
        Seule la trajectoire de phase représentée en \textbf{bleu} n'est pas
        symétrique par rapport à $x = 0$. Elle correspond donc au cas où la
        barrière de potentiel centrale est la plus élevée, donc \textbf{le cas}
        $\mathbf{a_1 = \ell_0/10}$. La trajectoire de phase représentée en
        \textbf{rouge} montre une réduction de vitesse en $x = 0$~: elle
        correspond donc au cas où il y a une barrière de potentiel, mais moins
        élevée, c'est-à-dire le cas $\mathbf{a_2 = \ell_0/3}$. Enfin, la
        trajectoire de phase \textbf{verte} est plus aplatie que la trajectoire
        de phase violette. Cet aplatissement se retrouve dans les courbes
        d'énergie potentielle~: la courbe verte correspond au cas $\mathbf{a_3 =
        \ell_0}$ et la courbe \textbf{violette} au cas $\mathbf{a_4 = 3\ell_0}$.
\end{enumerate}

\end{document}
