\documentclass[a4paper, 12pt, final, garamond]{book}
\usepackage{cours-preambule}

\raggedbottom

\makeatletter
\renewcommand{\@chapapp}{M\'ecanique -- chapitre}
\makeatother

% \toggletrue{student}
% \HideSolutionstrue
\toggletrue{corrige}
\renewcommand{\mycol}{black}
% \renewcommand{\mycol}{gray}

\begin{document}
\setcounter{chapter}{4}

\chapter{\cswitch{Correction du TD d'entra\^inement}{TD entra\^inement~: mouvement de
	  particules charg\'ees}}

\resetQ
\section{Séparation isotopique}

\enonce{%
	Le spectromètre de \textsc{Dempster} permet, entre autres, de séparer les
	différents isotopes chargés d'un élément dans un échantillon.
	\bigbreak
	\noindent
	\begin{minipage}{0.50\linewidth}
		Considérons un faisceau de particules chargées, constitué des ions de deux
		isotopes de mercure~: $\ce{^200_80Hg\plus{2}}$ et $\ce{^202_80Hg\plus{2}}$,
		notés respectivement (1) et (2). Ce faisceau sort de la chambre d'ionisation
		avec une vitesse négligeable, puis accéléré par une tension $U_{PP'}$
		appliquée entre les deux plaques $P$ et $P'$. Les ions traversent ensuite
		une zone de déviation où règne un champ magnétique transversal uniforme, tel
		que $\Bf = B\uz$. \bigbreak
	\end{minipage}
	\hfill
	\begin{minipage}{0.50\linewidth}
		\begin{center}
			\includegraphics[width=6cm]{dempster}
			\captionof{figure}{Schéma du dispositif}
			\label{fig:dempster}
		\end{center}
	\end{minipage}

	On donne~: $m\ind{nucléon} = \SI{1.67e-27}{kg}$, $m\ind{électron}$ négligeable
	devant $m\ind{nucléon}$, $\abs{U_{PP'}} = \SI{10}{kV}$, $B = \SI{0.10}{T}$ et $e
		= \SI{1.6e-19}{C}$.
}

\QR{%
	Quel doit être le signe de $U_{PP'}$ pour que les ions soient
	effectivement accélérés entre $P$ et $P'$~?
}{%
	Les ions étant positifs, ils subissent la force $\Ff_e = q\Ef$ dans le
	même sens que $\Ef$. Il faut donc que $\Ef$ soit selon $\uy$. Or, $\Ef =
		-\gd V$ indique que $\Ef$ va des hauts potentiels aux bas potentiels
	($\gd$ indique le sens des grandes variations, $-\gd$ indique
	l'inverse)~: on veut donc que $V_P$ soit plus grand que $V_{P'}$, soit
	\[U_{PP'} = V_P - V_{P'} > 0\]
}

\QR{%
Exprimer les vitesses $v_1$ et $v_2$ des isotopes suite à
l'accélération.
}{%
L'ion, assimilable à un point matériel M$_i$, de masse $m_i$, est
soumis dans le référentiel du laboratoire supposé galiléen à la force
électrique qui est conservative. Donc le système est conservatif~:
$\Ec_m(P)=\Ec_m(P')$. L'énergie potentielle électrique s'écrit
$\Ec_p=qV$ (on prend la constante nulle), on part à vitesse nulle et on
accélère jusqu'à $P'$, d'où

\[
	qV(P)=\frac{1}{2}m_iv_i{}^2+qV(P')
	\quad\Ra\quad
	\boxed{v_i=\sqrt{\frac{2qU_{PP'}}{m_i}}}
\]
}

\QR{%
Déterminer les trajectoires des ions dans la zone de déviation.
Exprimer les rayons $R_1$ et $R_2$ des trajectoires.
}{%
Cf.\ cours~: $R_i=\frac{mv_i}{\abs{q}B}$, donc
$R_i=\frac{\sqrt{2Um_i}}{B\sqrt{q}}$
\begin{itemize}
	\bitem{BDF~:}
	\[
		\begin{array}{ll}
			\textbf{Poids}            & \text{négligeable \textbf{devant }}\Ff \\
			\textbf{Force magnétique} & \Ff = q\vf\wedge\Bf =
			\mqty(q\xp                                                         \\q\yp\\q\zp)\wedge\mqty(0\\0\\B)\\
			\Lra                      & \Ff = q\yp B\ux -q\xp B\uy
		\end{array}
	\]
	\bitem{PFD~:}
	\[m_i\af = \Ff\]
	\bitem{Équations scalaires~:}
	\begin{empheq}[left=\empheqlbrace]{align*}
		m_i\xpp &= q\yp(t)B\\
		m_i\ypp &= -q\xp(t)B\\
		m_i\zpp &= 0
	\end{empheq}
\end{itemize}
D'où~:
\begin{gather*}
	\left\{
	\begin{aligned}
		\xpp(t) & = \frac{qB}{m_i}\yp(t)  \\
		\ypp(t) & = -\frac{qB}{m_i}\xp(t)
	\end{aligned}
	\right.
	\Lra
	\left\{
	\begin{aligned}
		\xp(t) & = \frac{qB}{m_i}y(t) + K   \\
		\yp(t) & = -\frac{qB}{m_i}x(t) + K'
	\end{aligned}
	\right.
\end{gather*}
ainsi avec les conditions initiales~:
\begin{gather*}
	\xp(0) = 0
	\qor
	\xp(0) = \frac{qB}{m_i}\underbracket[1pt]{y(0)}_{=0} + K = K
	\qdonc
	K = 0\\
	\yp(0) = v_i
	\qor
	\yp(0) = -\frac{qB}{m_i}\underbracket[1pt]{x(0)}_{=0} + K' = K'
	\qdonc
	K' = v_i
\end{gather*}
Soit
\begin{empheq}[left=\empheqlbrace]{align*}
	\xp(t) &= \frac{qB}{m_i}y(t)\\
	\yp(t) &= -\frac{qB}{m_i}x(t) + v_i
\end{empheq}
Étant donné que $v_x(t)^2 + v_y(t)^2 = \cte = v_0{}^2$, on a
\[\left(\frac{qB}{m_i}y(t)\right)^2 + \left(-\frac{qB}{m_i}x(t) + v_i\right)^2 =
	v_i{}^2\]
soit, en notant $\w_c = \abs{q}B/m_i$~:
\[\left(y\right)^2 + \left(x + \frac{m_iv_i}{qB}\right)^2 =
	\left( \frac{v_i}{\w_c} \right)^2\]
On reconnaît l'équation d'un cercle en coordonnées cartésiennes. Dans
notre cas, les centres sont $\DS\left(-\frac{m_iv_i}{qB},0\right)$ et les
rayons sont $\DS R_i = \frac{v_i}{\w_c} = \frac{m_iv_i}{\abs{q}B}$, soit
\[\boxed{R_i = \frac{\sqrt{2Um_i}}{B\sqrt{q}}}\]
}

\QR{%
	On recueille les particules sur une plaque photographique sous $P'$
	après leur demi-tour. Exprimer puis calculer la distance $d$ entre les
	deux traces observées.
}{%
	Graphiquement, on a $d=2(R_2-R_1)$. Donc
	\[
		\boxed{d=\cfrac{2\sqrt{2}}{B}
			\sqrt{\cfrac{U}{q}}\left(\sqrt{m_2}-\sqrt{m_1}\right)}
	\]
}

\resetQ
\section{Cyclotron \hfill {\small Inspiré CCP PC 2014, oral banque PT}}
\enonce{%
	Un cyclotron est formé de deux enceintes demi-cylindriques $D_1$ et $d_2$,
	appelées \textit{dees} en anglais, séparées d'une zone étroite d'épaisseur $a$.
	Les \textit{dees} sont situés dans l'entrefer d'un électroaimant qui fournit un
	champ magnétique uniforme $\Bf = B\ez$, de norme $B = \SI{1.5}{T}$. Une tension
	sinusoïdale d'amplitude $U_m = \SI{200}{kV}$ est appliquée entre les deux
	extrémités de la bande intermédiaire, si bien qu'il y règne en champ électrique
	orienté selon $\ex$.
	\bigbreak
	On injecte des protons de masse $m = \SI{1.7e-27}{kg}$ au sein de la zone
	intermédiaire avec une vitesse initiale négligeable.

	\noindent
	\begin{minipage}{0.45\linewidth}
		\begin{center}
			\includegraphics[width=5cm]{cyclotron_exo}
			\captionof{figure}{Schéma de principe.}
			\label{fig:cyclo_exo}
		\end{center}
	\end{minipage}
	\hfill
	\begin{minipage}{0.45\linewidth}
		\begin{center}
			\includegraphics[width=5cm]{cyclotron_rutgers}
			\captionof{figure}{Photon du cyclotron de l'université de
				\textsc{Rutgers}, mesurant $\approx \SI{30}{cm}$ en diamètre.}
			\label{fig:rutgers}
		\end{center}
	\end{minipage}
}

\QR{%
	Montrer qu'à l'intérieur d'un \textit{dee}, la norme de la vitesse des
	protons est constante.
}{%
	\begin{itemize}[label=$\diamond$]
		\bitem{Système}~: {proton}, assimilé à un point matériel de masse $m$ et de
		charge $q$.
		\bitem{Référentiel}~: lié au cyclotron, donc référentiel du laboratoire
		supposé galiléen.
		\bitem{BDF}~:
		\[
			\begin{array}{ll}
				\textbf{Poids}                     & \text{négligeable \textbf{devant }}\Ff \\
				\textbf{Force de \textsc{Lorentz}} & \Ff = e(\Ef + \vf\wedge\Bf)
			\end{array}
		\]
	\end{itemize}
	À l'intérieur des \textit{dees}, seule la force magnétique $\Ff_m =
		e\vf\wedge\Bf$ existe. Ainsi, d'après le TPC,
	\[
		\dv{\Ec_c}{t} =
		e\underbracket[1pt]
		{\underbracket[1pt]
			{\vf\wedge\Bf}_{\perp\vf}\cdot\vf}_{=0} = 0
		\qsoit
		mv\dv{v}{t} = 0
		\qMath{d'où}
		\boxed{\dv{v}{t} = 0}
	\]
}

\QR{%
	En déduire le rayon de courbure $R$ de la trajectoire des protons
	ayant une vitesse $v$ ainsi que le temps que passe un proton dans un
	\textit{dee}.
}{%
	La trajectoire d'un proton dans un champ magnétique est un arc de
	cercle, parcouru à vitesse constante. Utilisons un repérage
	\textbf{polaire}, centré sur le centre de l'arc de cercle. D'après la
	deuxième loi de \textsc{Newton},
	\[m\af = e\vf\wedge\Bf\]
	soit en utilisant les résultats connus sur la cinématique du mouvement
	circulaire,
	\[m\left(-\frac{v^2}{R}\ur\right) =
		evB\underbracket[1pt]{(-\ut\wedge\uz)}_{=\ur} = -evB\er\]
	avec $\vf = -v\ut$~: la trajectoire est parcourue en sens horaire pour
	un proton (à connaître ou à retrouver par la cohérence des signes).
	Finalement,
	\[
		\frac{mv^2}{R} = evB
		\qMath{d'où}
		\boxed{R = \frac{mv}{eB}}
	\]
	La trajectoire dans un des \textit{dee} est ainsi un demi-cercle, de
	longueur $\pi R$ et, \textbf{puisque la vitesse est constante},
	parcourue en un temps
	\[
		\boxed{\D t_d = \frac{\pi R}{v} = \frac{\pi m}{eB} = \SI{22}{ns}}
	\]
	On remarque ainsi que $\D t_d$ ne dépend pas de la vitesse du proton,
	mais seulement du champ appliqué dans le \textit{dee} (en plus des
	variables intrinsèques au proton, $e$ et $m$).
}

\QR{%
	Quelle doit être la fréquence $f$ de la tension pour que le proton
	soit accéléré de façon optimale à chaque passage entre les \textit{dee}~?
	Pour simplifier on pourra supposer $a \ll R$. Justifier le choix d'une
	tension harmonique au lieu, par exemple, d'une tension créneau.
}{%
	Pour que le proton soit accéléré de façon optimale à chaque passage
	entre les \textit{dees}, il faut que la force électrique qu'il subit
	soit alternativement orientée selon $+\ux$ lorsqu'il passe de $D_2$ à
	$D_1$, et selon $-\ux$ en passant de $D_1$ à $D_2$. En négligeant le
	temps de passage dans l'espace entre les \textit{dees} ($a \ll \pi R$),
	il faut donc qu'une demi-période de la tension appliquée soit égale à
	$\D t_d$, soit pour une période entière~:
	\[
		T = 2\D t_d = \frac{2\pi m}{eB}
		\qet
		\boxed{f = \frac{eB}{2\pi m} = \SI{23}{MHz}}
	\]
	Utiliser une tension harmonique plutôt qu'une tension créneau a
	l'intérêt de regrouper tous les protons pour que leur passage dans les
	\textit{dees} soit en phase avec la tension. Regrouper les protons
	permet aux impulsions du faisceau d'être plus puissantes. De plus, en
	pratique, un tension créneau requiert beaucoup d'harmoniques qu'il peut
	ne pas être simple d'imposer à de telles fréquences.


}

\QR{%
Exprimer en fonction de $n$ la vitesse $v_n$ puis le rayon $R_n$ de la
trajectoire d'un proton après $n$ passages dans la zone d'accélération.
Le demi-cercle $n=1$ est celui qui suit la première phase
d'accélération.
}{%
Jusqu'à présent, nous avons relié le rayon à la vitesse du proton. Il
faut donc maintenant relier la vitesse du proton au nombre de passage
dans les \textit{dees}, ou plutôt au nombre de passage dans la zone
accélératrice. Comme on ne s’intéresse qu'à la norme, le théorème de
l'énergie cinétique est le plus adapté. Appliquons ce théorème sur une
trajectoire entre la sortie d’un \textit{dee} et l'entrée de l'autre, en
supposant que le passage du proton se fait au moment où la tension
atteint son maximum (justifié par la question précédente), et en
supposant aussi que la durée de passage dans la zone accélératrice est
négligeable devant la période de la tension, ce qui permet de supposer
que la tension est presque constante égale à $U_m$. Sous ces hypothèses,
on trouve~:
\[
	\frac{1}{2}mv_{n+1}{}^2 - \frac{1}{2}mv_n{}^2 = W(\Ff_e) =
	e\frac{U_m}{a}a
\]
En raisonnant par récurrence, on obtient
\[
	\frac{1}{2}m v_n{}^2 - \frac{1}{2}mv_0{}^2
	\approx \frac{1}{2}mv_n{}^2 = neU_m
	\qsoit
	v_n = \sqrt{\frac{2neU_m}{m}}
\]
et en utilisant le résultat d'une question précédente,
\[
	R_n = \frac{m}{eB} \sqrt{\frac{2neU_m}{m}}
	\qsoit
	\boxed{R_n = \sqrt{\frac{2nmU_m}{B^2e}}}
\]
}

\QR{%
	Calculer numériquement le rayon de la trajectoire après un tour (donc un
	passage dans chaque \textit{dee}), puis après dix tours.
}{%
	Remarquons bien que $n$ compte le nombre de passages dans la zone
	accélératrice, faire un tour complet revient donc à passer de $n$ à
	$n+2$. Après un seul tour, $n=2$, et
	\[
		v_2 = \sqrt{\frac{4eU_m}{m}}
		\qet
		R_2 = 2 \sqrt{\frac{mU_m}{eB^2}} = \SI{6.1}{cm}
	\]
	Après dix tours, $n = 20$ et
	\[
		\boxed{R_{20} = \sqrt{10}R_2 = \SI{19}{cm}}
	\]
}

\enonce{%
	Le rayon de la dernière trajectoire décrite par les protons accélérés avant de
	bombarder une cible est $R_N = \SI{35}{cm}$.
}

\QR{%
Déterminer l'énergie cinétique du proton avant le choc contre la cible
proche du cyclotron, puis le nombre de tours parcourus par le proton.
}{%
Avec $R_N = \SI{35}{cm}$, la vitesse finale vaut
\[
	v_{\rm fin} = \frac{eBR_N}{m}
	\qMath{d'où}
	\boxed{\Ec_{c,\rm fin} = \frac{e^2B^2R_N{}^2}{2m}
	= \SI{2.1e-12}{J} = \SI{14}{MeV}}
\]
puis
\[
	\Ec_{c,\rm fin} = NeU_m
	\qMath{d'où}
	\boxed{N = \frac{\Ec_{c,\rm fin}}{eU_m} = 33}
\]
ce qui correspond à 16 tours et demi au sein du cyclotron.
}

\resetQ
\section{Chambre à bulles}

\enonce{%
	La chambre à bulles est un dispositif mis au point en 1952 par Donald Arthur
	\textsc{Glaser} (prix \textsc{Nobel} 1960), et destiné à visualiser des
	trajectoires de particules subatomiques. Il s'agit d'une enceinte remplie d'un
	liquide (généralement du dihydrogène) à une température légèrement supérieure à
	celle de vaporisation~: le passage d'une particule chargée déclenche la
	vaporisation et les petites bulles formées ainsi matérialisent la trajectoire de
	la particule.
	\bigbreak
	L'ensemble est plongé dans un champ magnétique uniforme et
	stationnaire, qui courbe les trajectoires et permet ainsi d'identifier les
	particules (à partir de leur masse et de leur charge).
	\bigbreak
	On étudie ici une particule P de masse $m$, de charge $q$ (positive ou
	négative), introduite à $t = 0$ dans la chambre à bulles où règne le champ $\Bf
		= B \ez$ (avec $B > 0$). Sa position initiale est l'origine O du repère, et sa
	vitesse initiale est $\vfo = v_0\ey$ (avec $v_0 > 0$). Le poids de la particule
	est négligé dans tout le problème. Le référentiel du laboratoire est supposé
	galiléen.
	\bigbreak
	\textit{Dans un premier temps, on suppose que les frottements du liquide sur la
		particule P sont négligeables.}
}

\QR{%
	Établir les équations différentielles du mouvement de P. On posera $\w
		= qB/m$.
}{%
	Dans le référentiel terrestre supposé galiléen, la particule P n'est
	soumise qu'à la force magnétique $\Ff = q\vf\wedge\Bf$, en négligeant le
	poids devant cette force. Avec le principe fondamental de la dynamique,
	on trouve
	\begin{gather*}
		\left\{
		\begin{aligned}
			m\xpp & = q\yp(t)B  \\
			m\ypp & = -q\xp(t)B \\
			m\zpp & = 0
		\end{aligned}
		\right.
		\Lra
		\left\{
		\begin{aligned}
			\xpp(t) & = \frac{qB}{m}\yp(t)  \\
			\ypp(t) & = -\frac{qB}{m}\xp(t) \\
			\zpp(t) & = 0
		\end{aligned}
		\right.
	\end{gather*}
	soit
	\begin{empheq}[box=\fbox, left=\empheqlbrace]{align}
		\label{eq:chb1}
		\xpp(t) &= +\w\yp(t)\\
		\label{eq:chb2}
		\ypp(t) &= -\w\xp(t)\\
		\label{eq:chb3}
		\zpp(t) &= 0
	\end{empheq}
}

\QR{%
	En déduire les équations horaires de P et indiquer précisément la
	nature de sa trajectoire. Représenter sur un même schéma les
	trajectoires d'un proton (charge $q = +e$ et masse $m_p$) et d'un
	électron ($q = -e$ et $m_e \ll m_p$).
}{%
	L'équation~\ref{eq:chb3} donne successivement $\zp = \cte = 0$ puis
	\fbox{$z = \cte = 0$}~: le mouvement a donc lieu dans le plan $(\Or
		xy)$. \smallbreak
	Pour les équations horaires, on intègre une fois les deux premières
	équations~\ref{eq:chb1} et~\ref{eq:chb2}~:
	\begin{gather*}
		\left\{
		\begin{aligned}
			\xpp(t) & = \w \yp(t)  \\
			\ypp(t) & = -\w \xp(t)
		\end{aligned}
		\right.
		\Lra
		\left\{
		\begin{aligned}
			\xp(t) & = \w y(t) + K   \\
			\yp(t) & = -\w x(t) + K'
		\end{aligned}
		\right.
	\end{gather*}
	ainsi avec les conditions initiales~:
	\begin{gather*}
		\xp(0) = 0
		\qor
		\xp(0) = \w\underbracket[1pt]{y(0)}_{=0} + K = K
		\qdonc
		K = 0\\
		\yp(0) = v_0
		\qor
		\yp(0) = -\w\underbracket[1pt]{x(0)}_{=0} + K' = K'
		\qdonc
		K' = v_0
	\end{gather*}
	Soit
	\begin{empheq}[left=\empheqlbrace]{align*}
		\xp(t) &= \w y(t)\\
		\yp(t) &= -\w x(t) + v_0
	\end{empheq}
	et on injecte l'expression de $\yp$ dans~\ref{eq:chb1} et inversement~:
	\begin{gather*}
		\left\{
		\begin{aligned}
			\xpp(t) + \w^2 x(t) & = \w v_0 \\
			\ypp(t) + \w^2 y(t) & = 0
		\end{aligned}
		\right.
	\end{gather*}
	en résolvant, on trouve finalement
	\[
		\boxed{x(t) = \frac{v_0}{\w}\left(1-\cos(\wt)\right)}
		\qet
		\boxed{y(t) = \frac{v_0}{\w}\sin(\wt)}
	\]
	donnant l'équation cartésienne
	\[\left(x - \frac{v_0}{\w}\right)^2 + y^2 =
		\left(\frac{v_0}{\w}\right)^2\]
	correspondant à l'équation d'un cercle de centre
	$\Omega\left(\frac{v_0}{\w},0,0\right)$ et de rayon $R =
		\frac{v_0}{\abs{\w}} = \frac{mv_0}{\abs{q}B}$.
		{
			\floatsetup[figure]{capposition=beside, capbesideposition={center,right}}
			\begin{figure}[h]
				\centering
				\includegraphics[width=\linewidth]{chb_traj}
				\caption{Trajectoires pour un proton et un électron.
					\smallbreak
					Pour un proton, $\frac{v_0}{\w} = \frac{mv_0}{qB} >0$, donc la
					trajectoire est à droite, et le mouvement se fait dans le sens
					horaire.
					\smallbreak
					À l'inverse, pour l'électron la trajectoire est à gauche et se fait
					dans le sens direct, mais avec un rayon beaucoup plus petit puisque
					proportionnel à $m$.}
				\label{fig:chb_traj}
			\end{figure}
		}
}

\enonce{%
	\textit{Les frottements du liquide sont maintenant modélisés par la force $\Ff =
			-\lb\vf_{P}$ avec $\lb$ une constante positive. On pose $\a = \lb/m$.}
}

\QR{%
	Établir les nouvelles équations différentielles du mouvement (avec les
	paramètres $\w$ et $\a$). Montrer que le mouvement reste plan.

}{%
	On réemploie le PFD~:
	\begin{gather*}
		m\af = q\vf\wedge\Bf -\lb\vf
		\Lra
		\left\{
		\begin{aligned}
			m\xpp & = +qB\yp -\lb\xp  \\
			m\ypp & = -qB\xp -\lb \yp \\
			m\zpp & = -\lb\zp
		\end{aligned}
		\right.
	\end{gather*}
	soit
	\begin{empheq}[box=\fbox, left=\empheqlbrace]{align}
		\label{eq:chb4}
		\xpp &= +\w\yp -\a\xp\\
		\label{eq:chb5}
		\ypp &= -\w\xp -\a\yp\\
		\label{eq:chb6}
		\zpp &= -\a\zp
	\end{empheq}
	Le mouvement reste plan, puisque la solution de l'équation~\ref{eq:chb6}
	est $\zp(t) = D\exp(-\a t)$, mais que $\zp(0) = 0 \Ra D = 0$, soit
	\fbox{$z = \cte =0$}.
}

\QR{%
	Déterminer complètement les équations horaires de P. On pourra poser
	la variable complexe $\uu = x + \jj y$, et déterminer tout d'abord
	$\xul{\dot{u}}$.

}{%
	En posant, comme suggéré, $\uu = x + \jj y$, on combine
	$\eqref{eq:chb4}+\jj\eqref{eq:chb5}$ pour avoir
	\[\ddot{\uu} + (\a + \jw)\dot{\uu} = 0\]
	qui est une \textbf{équation différentielle d'ordre 2 sans ordre 0},
	donc d'ordre 1 en $\dot{\uu}$~: on trouve donc les solutions avec une
	simple exponentielle~:
	\begin{gather*}
		\dot{\uu}(t) = A\exp(-(\a + \jw)t)
		\\
		\qavec
		\dot{\uu}(0) = 0 + \jj v_0 = A
		\qsoit
		\dot{\uu}(t) = \jj v_0\exp(-(\a+\jw)t)\\
		\text{ainsi}\quad
		\uu(t) = \frac{\jj v_0}{-(\a+\jw)}\exp(-(\a+\jw)t) + B
		\qor
		\uu(0) = 0 \Lra B = \frac{\jj v_0}{\a+\jw}\\
		\text{finalement}\quad
		\boxed{\uu(t) = \frac{\jj
				v_0}{\a+\jw}\left(1-\exp(-(\a+\jw)t)\right)}
	\end{gather*}
	En mettant la fraction avec un dénominateur réel et en séparant les
	exponentielles~:
	\[
		\boxed{\uu(t) = \frac{\jj v_0\a + v_0\w}{\a^2+\w^2}
			\left(1-\exp(-\at)\exp(-\jwt)\right)}
	\]
	puis en prenant la partie
	réelle pour obtenir $x(t)$ et la partie imaginaire pour obtenir $y(t)$
	(\textbf{attention à bien distribuer la fraction}),
	\begin{empheq}[box=\fbox, left=\empheqlbrace]{align*}
		x(t) &= \frac{v_0\w}{\a^+\w^2}\left(1-\exp(-\at)\cos(\wt)\right)
		- \frac{v_0\a}{\a^2+\w^2}\exp(-\a t)\sin(\wt)\\
		y(t) &= \frac{v_0\w}{\a^+\w^2}\exp(-\a t)\sin(\wt)
		+ \frac{v_0\a}{\a^2+\w^2}\left(1-\exp(-\at)\cos(\wt)\right)\\
	\end{empheq}
}

\QR{%
Déterminer les coordonnées du point asymptotique ${\rm P}_{+\infty} =
	{\rm P}(t \ra \infty)$. Représenter sur un schéma la trajectoire d'un
proton.
}{%
Pour $t\ra\infty$, le point d'asymptote est
\[
	\boxed{
		{\rm P}_{\infty} = \mqty(\DS\frac{v_0\w}{\a^2+\w^2}\\[1em]
		\DS\frac{v_0\a}{\a^2+\w^2}\\[1em]
		0)
	}
\]
\begin{minipage}{0.65\linewidth}
	La particule tourne toujours à cause des facteurs sinusoïdaux, mais le
	rayon de courbure diminue exponentiellement~: la trajectoire est une
	spirale, tournant toujours vers la droite pour un proton, et s'enroulant
	autour du point P$_{\infty}$.
\end{minipage}
\hfill
\begin{minipage}{0.30\linewidth}
	%~\vspace{-12pt}
	\begin{center}
		\includegraphics[width=\linewidth]{chb_spi}
	\end{center}
\end{minipage}
}

\end{document}
