\documentclass[a4paper, 12pt, final, garamond]{book}
\usepackage{cours-preambule}

\raggedbottom

\makeatletter
\renewcommand{\@chapapp}{Architecture de la mati\`ere -- chapitres}
\renewcommand\thechapter{1 et 2}
\makeatother

\begin{document}
% \setcounter{chapter}{0}

\chapter{Correction du TD}

\section{Structures de \textsc{Lewis}}
\begin{enumerate}
    \item On a~:
        \begin{itemize}[label=$\diamond$, leftmargin=10pt]
            \litem{Dichlorométhane \ce{CH2Cl2}}~:
                \begin{itemize}[label=$\triangleright$, leftmargin=20pt]
                    \item Décompte des électrons~:
                        \begin{itemize}[label=$\ra$, leftmargin=20pt]
                            \item $[\ce{H}]~: \rm \color{red}1s^1$
                                donc 1 électron de valence
                            \item $[\ce{C}]~: \rm 1s^2\color{red}2s^22p^2$
                                donc 4 électrons de valence
                            \item $[\ce{Cl}]~: \rm 1s^22s^22p^6\color{red}3s^23p^5$
                                donc 7 électrons de valence.
                            \item Total~: $2*1 + 4 + 2*7 = 20$ électrons, 10
                                doublets.
                        \end{itemize}
                    \item Méthode simple~:
                \end{itemize}
        \end{itemize}
\end{enumerate}
\[
    \cfig{\clew
        (!{&0}\hlew{4})
        (!{&2}\hlew{6})
        (!{&4}\cllew{0})
        (!{&6}\cllew{2})
    }
    \qqdonc
    \cfig{C(-[0]H)(-[2]H)(-[4]\lewis{246,Cl})(-[6]\lewis{046,Cl})}
\]
\bigbreak
\begin{enumerate}
    \item[]
        \begin{itemize}[label=$\diamond$, leftmargin=10pt]
            \litem{Dioxygène \ce{O2}}~:
                \begin{itemize}[label=$\triangleright$, leftmargin=20pt]
                    \item Décompte des électrons~:
                        \begin{itemize}[label=$\ra$, leftmargin=20pt]
                            \item $[\ce{O}]~: \rm 1s^2\color{red}2s^22p^4$
                                donc 6 électrons de valence
                            \item Total~: $2*6 = 12$ électrons, 6
                                doublets.
                        \end{itemize}
                    \item Méthode simple~:
                \end{itemize}
        \end{itemize}
\end{enumerate}
\[
    \cfig{%
        \odlew{0}!{&0}\odlew{4}
    }
    \qqdonc
    \cfig{\lewis{35,O}=\lewis{17,O}}
\]
\begin{enumerate}
    \item[]
        \begin{itemize}[label=$\diamond$, leftmargin=10pt]
            \litem{Éthène \ce{C2H4}}~:
                \begin{itemize}[label=$\triangleright$, leftmargin=20pt]
                    \item Décompte des électrons~:
                        \begin{itemize}[label=$\ra$, leftmargin=20pt]
                            \item $[\ce{H}]~: \rm \color{red}1s^1$
                                donc 1 électron de valence
                            \item $[\ce{C}]~: \rm 1s^2\color{red}2s^22p^2$
                                donc 4 électrons de valence
                            \item Total~: $4*1 + 2*4 = 12$ électrons, 6
                                doublets.
                        \end{itemize}
                    \item Méthode simple~:
                \end{itemize}
        \end{itemize}
\end{enumerate}
\[
    \cfig{
        \lewis{0:2.6.,C}
        (
            !{&0}\lewis{2.4:6.,C}
            (!{&2}\hlew{6})
            (!{&6}\hlew{2})
        )
        (!{&2}\hlew{6})
        (!{&6}\hlew{2})
    }
    \qqdonc
    \cfig{
        C(-[2]H)(-[6]H)=C(-[2]H)(-[6]H)
    }
\]
\begin{enumerate}
    \item[]
        \begin{itemize}[label=$\diamond$, leftmargin=10pt]
            \litem{Ion oxonium \ce{H3O+}}~:
                \begin{itemize}[label=$\triangleright$, leftmargin=20pt]
                    \item Décompte des électrons~:
                        \begin{itemize}[label=$\ra$, leftmargin=20pt]
                            \item $[\ce{H}]~: \rm \color{red}1s^1$
                                donc 1 électron de valence
                            \item $[\ce{O}]~: \rm 1s^2\color{red}2s^22p^4$
                                donc 6 électrons de valence
                            \item Une charge $\oplus$, donc 1 électron en moins
                            \item Total~: $3*1 + 6 - 1 = 8$ électrons, 4
                                doublets.
                        \end{itemize}
                    \item Méthode simple~: \ce{H} est moins électronégatif que
                        \ce{O}, mais si la charge $\oplus$ était portée par un
                        \ce{H} il ne pourrait pas se lier~: c'est forcément
                        \ce{O} qui la porte.
                \end{itemize}
        \end{itemize}
\end{enumerate}
\smallbreak
\[
    \cfig{
        \chemabove{\lewis{0.24.6.,O}}{\vspace{10pt}\hspace{20pt}\oplus}
        (!{&0}\hlew{4})
        (!{&4}\hlew{0})
        (!{&6}\hlew{2})
    }
    \qqdonc
    \cfig{
        \chemabove{\lewis{2,O}}{\vspace{10pt}\hspace{20pt}\oplus}
        (-[0]H)
        (-[4]H)
        (-[6]H)
    }
\]
\begin{enumerate}
    \item[]
        \begin{itemize}[label=$\diamond$, leftmargin=10pt]
            \litem{Ion hydroxyde \ce{HO-}}~:
                \begin{itemize}[label=$\triangleright$, leftmargin=20pt]
                    \item Décompte des électrons~:
                        \begin{itemize}[label=$\ra$, leftmargin=20pt]
                            \item $[\ce{H}]~: \rm \color{red}1s^1$
                                donc 1 électron de valence
                            \item $[\ce{O}]~: \rm 1s^2\color{red}2s^22p^4$
                                donc 6 électrons de valence
                            \item Une charge $\ominus$, donc 1 électron en plus
                            \item Total~: $1 + 6 + 1 = 8$ électrons, 4
                                doublets.
                        \end{itemize}
                    \item Méthode simple~:
                \end{itemize}
        \end{itemize}
\end{enumerate}
\[
    \cfig{\hlew{0}!{&0}\chemabove{\lewis{024.6,O}}{\vspace{10pt}\hspace{20pt}\ominus}}
    \qdonc
    \cfig{H-\chemabove{\lewis{026,O}}{\vspace{10pt}\hspace{20pt}\ominus}}
\]
\begin{enumerate}
    \item[]
        \begin{itemize}[label=$\diamond$, leftmargin=10pt]
            \litem{Méthanal \ce{H2CO}}~:
                \begin{itemize}[label=$\triangleright$, leftmargin=20pt]
                    \item Décompte des électrons~:
                        \begin{itemize}[label=$\ra$, leftmargin=20pt]
                            \item $[\ce{H}]~: \rm \color{red}1s^1$
                                donc 1 électron de valence
                            \item $[\ce{C}]~: \rm 1s^2\color{red}2s^22p^2$
                                donc 4 électrons de valence
                            \item $[\ce{O}]~: \rm 1s^2\color{red}2s^22p^4$
                                donc 6 électrons de valence
                            \item Total~: $2*1 + 4 + 6 = 12$ électrons, 6
                                doublets.
                        \end{itemize}
                    \item Méthode simple~: on a forcément une liaison \ce{C-O},
                        mais il n'est pas évident de savoir où les \ce{H} vont
                        se lier. On commence donc par les placer entre \ce{C} et
                        \ce{O}.
                \end{itemize}
        \end{itemize}
\end{enumerate}
\smallbreak
\[
    \cfig{\vphantom{X}
        (-[0,0.5,,,draw=none]\odlew{4})
        (!{&2}\hlew{6})
        (-[4,0.5,,,draw=none]\lewis{0:2.6.,C})
        (!{&6}\hlew{2})
    }
    \qqdonc
    \cfig{
        C
        (=\lewis{17,O})
        (-[2]H)
        (-[6]H)
    }
\]
\begin{enumerate}
    \item[]
        \begin{itemize}[label=$\diamond$, leftmargin=10pt]
            \litem{Dioxyde de silicium \ce{SiO2}}~:
                \begin{itemize}[label=$\triangleright$, leftmargin=20pt]
                    \item Décompte des électrons~:
                        \begin{itemize}[label=$\ra$, leftmargin=20pt]
                            \item $[\ce{O}]~: \rm 1s^2\color{red}2s^22p^4$
                                donc 6 électrons de valence
                            \item $[\ce{Si}]~: \rm 1s^22s^22p^6\color{red}3s^23p^2$
                                donc 4 électrons de valence
                            \item Total~: $2*6 + 4 = 16$ électrons, 8
                                doublets.
                        \end{itemize}
                    \item Méthode simple~:
                \end{itemize}
        \end{itemize}
\end{enumerate}
\smallbreak
\[
    \cfig{
        \lewis{0:4:,Si}
        (!{&0}\odlew{4})
        (!{&4}\odlew{0})
    }
    \qqdonc
    \cfig{
        Si
        (=[0]\lewis{17,O})
        (=[4]\lewis{35,O})
    }
\]
\begin{enumerate}
    \item[]
        \begin{itemize}[label=$\diamond$, leftmargin=10pt]
            \litem{Méthylamine \ce{CH3NH2}}~:
                \begin{itemize}[label=$\triangleright$, leftmargin=20pt]
                    \item Décompte des électrons~:
                        \begin{itemize}[label=$\ra$, leftmargin=20pt]
                            \item $[\ce{H}]~: \rm \color{red}1s^1$
                                donc 1 électron de valence
                            \item $[\ce{C}]~: \rm 1s^2\color{red}2s^22p^2$
                                donc 4 électrons de valence
                            \item $[\ce{N}]~: \rm 1s^2\color{red}2s^22p^3$ donc
                                5 électrons de valence
                            \item Total~: $5*1 + 4 + 5 = 14$ électrons, 7
                                doublets.
                        \end{itemize}
                    \item Méthode simple~: \ce{H} ne peut former qu'une seule
                        liaison, on a forcément une liaison \ce{C-N}.
                \end{itemize}
        \end{itemize}
\end{enumerate}
\smallbreak
\[
    \cfig{
        \clew
        (!{&0}\hlew{4})
        (!{&2}\hlew{6})
        (!{&6}\hlew{2})
        (!{&4}\nlew{4}
            (!{&2}\hlew{6})
            (!{&6}\hlew{2})
        )
    }
    \qqdonc
    \cfig{
        C
        (-[0]H)
        (-[2]H)
        (-[6]H)
        (-[4]\lewis{4,N}
            (-[2]H)
            (-[6]H)
        )
    }
\]
\begin{enumerate}[start=2]
    \item ~
        \begin{itemize}[label=$\diamond$, leftmargin=10pt]
            \litem{Ozone \ce{O3}}~:
                \begin{itemize}[label=$\triangleright$, leftmargin=20pt]
                    \item Décompte des électrons~:
                        \begin{itemize}[label=$\ra$, leftmargin=20pt]
                            \item $[\ce{O}]~: \rm 1s^2\color{red}2s^22p^4$
                                donc 6 électrons de valence
                            \item Total~: $3*6 = 18$ électrons, 9
                                doublets.
                        \end{itemize}
                    \item Méthode simple~: à tenter, mais pas simple d'obtenir
                        un résultat convaincant.
                    \item Méthode générale~:
                        \begin{itemize}[label=$\ra$, leftmargin=20pt]
                            \litem{Squelette}~: immédiat car la molécule est
                                linéaire.
                            \litem{Recherche de liaisons multiples}~: le squelette
                                implique au moins 2 liaisons, soit 7 doublets
                                restants à placer. \textbf{Si tous les doublets
                                restant étaient non liants}, pour respecter
                                l'octet il en faudrait 3
                                sur les atomes du bout et 2 sur l'atome du
                                milieu, soit $2*3+2 = 8$~: c'est un de plus que
                                disponible. Il y a donc \textbf{une liaison
                                double}.
                            \item On pose donc les doublets.
                            \litem{Recherche des charges formelles}~:
                            \begin{itemize}[label=$\bullet$, leftmargin=20pt]
                                \item Atome de gauche~: il a 6 électrons qui
                                    l'entourent, contre 6 dans son état isolé~:
                                    pas de charge.
                                \item Atome central~: il a 5 électrons qui
                                    l'entourent, donc une charge $\oplus$.
                                \item Atome de droite~: il a 7 électrons qui
                                    l'entourent, donc une charge $\ominus$.
                            \end{itemize}
                        \end{itemize}
                    \item Conclusion~:
                \end{itemize}
        \end{itemize}
\end{enumerate}
\[
    \cfig{\lewis{35,O}=
        \chemabove{\lewis{2,O}}{\vspace{10pt}\hspace{20pt}\oplus}
        -\chemabove{\lewis{026,O}}{\vspace{-12pt}\hspace{20pt}\ominus}
    }
\]
\begin{enumerate}[start=3]
    \item 
        \begin{enumerate}[leftmargin=20pt]
            \item
            \begin{itemize}[label=$\diamond$, leftmargin=10pt]
                \litem{Acide sulfurique \ce{H2SO4}}~:
                    \begin{itemize}[label=$\triangleright$, leftmargin=20pt]
                        \item Décompte des électrons~:
                            \begin{itemize}[label=$\ra$, leftmargin=20pt]
                                \item $[\ce{H}]~: \rm \color{red}1s^1$
                                    donc 1 électron de valence
                                \item $[\ce{O}]~: \rm 1s^2\color{red}2s^22p^4$
                                    donc 6 électrons de valence
                                \item $[\ce{S}]~: \rm
                                    1s^22s^22p^6\color{red}3s^23p^4$
                                    donc 6 électrons de valence
                                \item Total~: $2*1 + 6 + 4*6 = 32$ électrons, 16
                                    doublets.
                            \end{itemize}
                        \item Méthode simple~: on nous donne l'information que
                            les 4 atomes d'oxygène sont reliés au silicium. On a
                            donc le squelette, et on complète les doublets en
                            trouvant les deux liaisons doubles qui évitent les
                            deux charges $\ominus$ sur les oxygènes. Comme le
                            silicium appartient à la 3ème période, il est
                            possible qu'il soit entouré de plus de 4 doublets
                            (hypervalent).
                    \end{itemize}
            \end{itemize}
        \end{enumerate}
    \smallbreak
    \[
        \cfig{\lewis{0.2:4.6:,S}
            (!{&0}\lewis{0.24.6,O}
            !{&0}\hlew{4})
            (!{&4}\lewis{0.24.6,O}
            !{&4}\hlew{0})
            (!{&2}\odlew{6})
            (!{&6}\odlew{2})
        }
        \qqdonc
        \cfig{
            S
            (-[0]\lewis{26,O}-[0]H)
            (-[4]\lewis{26,O}-[4]H)
            (=[2]\lewis{13,O})
            (=[6]\lewis{57,O})
        }
    \]
    \item 
        \begin{enumerate}[leftmargin=20pt]
            \item
            \begin{itemize}[label=$\diamond$, leftmargin=10pt]
                \litem{Ions \ce{HSO4-} et \ce{SO4^{2-}}}~:
                    \begin{itemize}[label=$\triangleright$, leftmargin=20pt]
                        \item Décompte des électrons~:
                            Déjà effectué
                        \item Méthode simple~: on enlève un atome d'hydrogène à
                            chaque fois, laissant donc le doublet de la liaison
                            \ce{O-H} sur l'atome d'oxygène correspondant, lui
                            laissant une charge $\ominus$.
                    \end{itemize}
            \end{itemize}
        \end{enumerate}
    \smallbreak
    \[
        \cfig{
            S
            (-[0]\chemabove{\lewis{026,O}}{\hspace{20pt}\ominus})
            (-[4]\lewis{26,O}-[4]H)
            (=[2]\lewis{13,O})
            (=[6]\lewis{57,O})
        }
        \qqet
        \cfig{
            S
            (-[0]\chemabove{\lewis{026,O}}{\hspace{20pt}\ominus})
            (-[4]\chemabove{\lewis{246,O}}{\hspace{-20pt}\ominus})
            (=[2]\lewis{13,O})
            (=[6]\lewis{57,O})
        }
    \]
    \item 
        \begin{itemize}[label=$\diamond$, leftmargin=10pt]
            \litem{Ion hydrogénocarbonate \ce{HCO3-}}~:
                \begin{itemize}[label=$\triangleright$, leftmargin=20pt]
                    \item Décompte des électrons~:
                        \begin{itemize}[label=$\ra$, leftmargin=20pt]
                            \item $[\ce{H}]~: \rm \color{red}1s^1$
                                donc 1 électron de valence
                            \item $[\ce{C}]~: \rm 1s^2\color{red}2s^22p^2$
                                donc 4 électrons de valence
                            \item $[\ce{O}]~: \rm 1s^2\color{red}2s^22p^4$
                                donc 6 électrons de valence
                            \item Une charge $\ominus$, donc un électron en plus
                            \item Total~: $1 + 4 + 3*6 +1 = 24$ électrons, 12
                                doublets.
                        \end{itemize}
                    \item Méthode simple~: fonctionne… si on part bien~! \ce{O}
                        est le plus électronégatif des atomes de l'ion, on en
                        déduit que c'est un des oxygènes qui la porte.
                    \item Méthode générale~:
                        \begin{itemize}[label=$\ra$, leftmargin=20pt]
                            \litem{Squelette}~: on pourrait partir avec une
                            solution incluant une liaison \ce{O-O} avec le
                            \ce{H} sur \ce{C}, mais comme on sait qu'on va
                            former \ce{CO3^{2-}} ensuite, il faut pouvoir
                            enlever l'atome d'hydrogène sans se retrouver avec
                            $\chemabove{\lewis{2,C}}{\hspace{20pt}\ominus}$ qui
                            est très peu commun. On mettra comme d'habitude
                            l'atome qui fait le plus de liaisons au centre, avec
                            l'hydrogène lié à un oxygène.
                            \litem{Recherche de liaisons multiples}~: le squelette
                                implique au moins 4 liaisons, soit 8 doublets
                                restants à placer. \textbf{Si tous les doublets
                                restant étaient non liants}, pour respecter
                                l'octet il en faudrait 9 en tout~: c'est un de plus que
                                disponible. Il y a donc \textbf{une liaison
                                double}, une \ce{C=O}.
                            \item On pose donc les doublets.
                            \litem{Recherche des charges formelles}~:
                            \begin{itemize}[label=$\bullet$, leftmargin=20pt]
                                \item Oxygène en haut~: il a 7 électrons qui
                                    l'entourent, contre 6 dans son état isolé~:
                                    charge $\ominus$.
                                \item Carbone~: il a 4 électrons qui
                                    l'entourent, donc pas de charge.
                                \item Autres oxygène~: ils ont 6 électrons qui
                                    les entourent, donc pas de charge.
                            \end{itemize}
                        \end{itemize}
                    \item Conclusion~:
                \end{itemize}
        \end{itemize}
    \[
        \cfig{
            \clew
            (!{&0}\odlew{4})
            (!{&6}\odlew{2})
            (!{&4}\hlew{0})
            (!{&2}\chemabove{\lewis{0246.,O}}{\hspace{20pt}\ominus})
            }
        \qqdonc
        \cfig{
            C
            (=[0]\lewis{17,O})
            (-[2]\chemabove{\lewis{024,O}}{\hspace{20pt}\ominus})
            (-[6]\lewis{46,O}-[0]H)
        }
    \]
        \begin{itemize}[label=$\diamond$, leftmargin=10pt]
            \litem{Ion carbonate \ce{CO3^{2-}}}~:
                \begin{itemize}[label=$\triangleright$, leftmargin=20pt]
                    \item Décompte des électrons~: déjà fait
                    \item Méthode simple~: on enlève un hydrogène en rabattant
                        le doublet liant en non-liant.
                \end{itemize}
        \end{itemize}
    \smallbreak
    \[
        \cfig{
            C
            (=[0]\lewis{17,O})
            (-[2]\chemabove{\lewis{024,O}}{\hspace{20pt}\ominus})
            (-[6]\chemabove{\lewis{046,O}}{\hspace{20pt}\ominus})
        }
    \]
    \item 
        \begin{itemize}[label=$\diamond$, leftmargin=10pt]
            \litem{Benzène \ce{C6H6}}~:
                \begin{itemize}[label=$\triangleright$, leftmargin=20pt]
                    \item Décompte des électrons~:
                        \begin{itemize}[label=$\ra$, leftmargin=20pt]
                            \item $[\ce{H}]~: \rm \color{red}1s^1$
                                donc 1 électron de valence
                            \item $[\ce{C}]~: \rm 1s^2\color{red}2s^22p^2$
                                donc 4 électrons de valence
                            \item Total~: $6*1 + 6*4 = 30$ électrons, 15
                                doublets.
                        \end{itemize}
                    \item Méthode simple~: comme on sait que la molécule est
                        cyclique, il est plus simple de représenter le cycle dès
                        le départ avec des liaisons simples. Ensuite, chaque
                        \ce{H} est lié à un \ce{C}. On a donc consommé $6+6=12$
                        doublets pour ce squelette, il en reste 3 qui ne peuvent
                        être des doublets non liants (sinon charge formelle)~:
                        il a donc 3 liaisons doubles.
                \end{itemize}
        \end{itemize}
    \[
        \cfig{
            \lewis{3.5.,C}
            (!{&4}\lewis{0.,H})
            *6
            (-\lewis{4.6.,C}
                (!{&5}\lewis{1.,H})
            -\lewis{0.6.,C}
                (!{&7}\lewis{3.,H})
            -\lewis{7.1.,C}
                (!{&0}\lewis{4.,H})
            -\lewis{0.2.,C}
                (!{&1}\lewis{5.,H})
            -\lewis{2.4.,C}
                (!{&3}\lewis{7.,H})
            -
            )}
        \qqdonc
        \cfig{
            C
            (-[4]H)
            *6
            (=C
                (-[:-120]H)
            -C
                (-[:-60]H)
            =C
                (-[0]H)
            -C
                (-[:60]H)
            =C
                (-[:120]H)
            -)
        }
    \]

\end{enumerate}

\section{Le phosphore}
\begin{enumerate}
    \item On écrit sa configuration~: $[\ce{P}]~: \elconf{P}$, soit 5 électrons
        de valence (et on voit avec \textsc{Pauli} et \textsc{Hund} qu'il a 3
        électrons parallèles
        dans les OA de la sous-couche 3p, et 2 remplissant l'OA 3s,
        d'où un doublet liant et 3 électrons simples).
    \item Le chlore a pour configuration $[\ce{Cl}]~:\elconf{Cl}$, soit 7
        électrons de valence (et on voit avec \textsc{Pauli} et \textsc{Hund}
        qu'il a 3 doublets
        non-liants et un seul électron célibataire). On met donc le phosphore au
        centre de l'édifice pour obtenir~:
        \[
            \cfig{
                \lewis{0.2.46.,P}
                (!{&0}\lewis{6024.,Cl})
                (!{&2}\lewis{6.024,Cl})
                (!{&6}\lewis{602.4,Cl})
            }
            \qqdonc
            \cfig{
                \lewis{4,P}
                (-[0]\lewis{026,Cl})
                (-[2]\lewis{024,Cl})
                (-[6]\lewis{046,Cl})
            }
        \]
    \item Le phosphore appartient à la troisième période, et peut donc être
        hypervalent (grâce à la sous-couche $d$ non-remplie). On obtient alors
        \[
            \cfig{
                P
                (-[0]\lewis{026,Cl})
                (-[2]\lewis{024,Cl})
                (-[3]\lewis{246,Cl})
                (-[5]\lewis{246,Cl})
                (-[6]\lewis{046,Cl})
            }
        \]
\end{enumerate}

\section{Caractéristiques de quelques solvants}
\begin{enumerate}
    \item Tous sont polaires, sauf l'hexane.
    \item L'eau et le méthanol sont protiques, pas les autres.
    \item L'hexane n'est pas dispersif, le méthanol, le DMF et l'acétonitrile
        sont dispersifs, l'eau est fortement dispersive.
    \item La présence d'un moment dipolaire sur chacune de ces molécules permet
        des interactions attractives de \textsc{Van der Waals}. L'hexane ne
        présente pas ces propriétés et ne se mélange donc pas bien aux autres.
\end{enumerate}

\section{Températures de changements d'état}
\begin{enumerate}
    \item Toutes ces molécules sont apolaires, donc seule l'interaction de
        \textsc{London} permet leur cohésion interne. Pour les 4 premières, on
        n'augmente que peu la taille de l'atome et donc que peu sa
        polarisabilité, mais ce qui augmente néanmoins les interactions
        attractives et donc la température d'ébullition (exception avec le fluor
        très électronégatif qui réduit sa polarisabilité). Pour les deux
        dernières, on reste sur la même famille en augmentant de période, donc
        la taille de l'atome augmente subitement et sa polarisabilité avec, d'où
        les températures observées.
    \item On nous indique que leurs tailles sont comparables, on suppose donc
        leur polarisabilité similaire. Ce qui les différencie ici est leur
        moment dipolaire~: il est plus élevé dans le cas de \ce{H2S}, ce qui
        fait que les interactions de \textsc{Debye} et \textsc{Keesom} sont plus
        intenses, et la température d'ébullition donc plus élevée.
    \item Fusion~: passage du solide au liquide. C'est l'hélium qui a la plus
        faible, étant donné qu'il correspond au plus petit corps pur stable et
        inerte, n'ayant aucune interaction de \textsc{VdW} notable. L'élément
        qui a la plus haute sera l'acide éthanoïque, étant donné sa capacité à
        créer des liaisons hydrogènes en sont sein (\ce{H} relié à \ce{O}). On
        peut terminer en supposant que l'argon aura une température supérieure à
        celle de l'hélium, mais sûrement proche de celle du méthane puisque
        bien qu'apolaires les deux commencent à être relativement polarisables.
    \item L'interaction de \textsc{London} est du même ordre de grandeur (même
        taille $\approx$ même polarisabilité), mais les interactions de
        \textsc{Debye} et \textsc{Keesom} sont accrues pour le composé (Z) qui
        présente un moment dipolaire.
\end{enumerate}

\section{Moment dipolaire et charges partielles}
\begin{enumerate}
    \item $\chi_{\ce{H}} < \chi_{\ce{F}}$, donc \ce{H} porte la charge $+\de$
        et \ce{F} la charge $-\de$, avec $p = \de\ell_{\ce{H-F}}$~; ainsi
        \[
            \de = \frac{p}{\ell_{\ce{H-F}}} = \SI{6.6e-20}{C} \approx
            \num{0.4}\times e
        \]
    \item $\pf$ est le long de la liaison covalente et dirigé de \ce{F} vers
        \ce{Li}. Son moment dipolaire $p$ est~:
        \[p = \de\ell_{\ce{Li-F}} = \SI{6.57}{D}\]
\end{enumerate}

\section{Monoxyde de carbone}
\begin{enumerate}
    \item $[\ce{O}]~: \elconf{O}$ donc 6 électrons de valence. \smallbreak
        $[\ce{C}]~: \elconf{C}$ donc 4 électrons de valence.
    \item Le carbone est tétravalent puisqu'il appartient à la deuxième période,
        et donc respecte forcément la règle de l'octet~: comme il compte quatre
        électrons de valence, il peut former 4 liaisons pour s'entourer d'un
        octet. \bigbreak
        L'étude précise de sa configuration électronique donne cependant un état
        de base avec un doublet non-liant et 2 électrons célibataires
        (\textsc{Hund})~; il se trouve qu'on le trouve le plus souvent sous une
        forme excitée \writeelconf{2,1+3} où tous les spins sont parallèles et
        tous les électrons sont célibataires, d'où les 4 liaisons (hors
        programme).
    \item Nombre total d'électrons de valence~: $4+6=10$ soit 5 doublets. On a
        forcément une liaison simple au moins, et il resterait 4 doublets à
        placer. S'ils étaient tous non liants, il en faudrait 3 sur \ce{C} et 3
        sur \ce{O} pour leur faire respecter l'octet~: c'est 2 de plus que
        disponible. Il y a donc une \textbf{liaison triple}, et finalement~:
        \[
            \cfig{
                \chemabove{\lewis{4,C}}{\hspace{-20pt}\ominus}
                =
                \chemabove{\lewis{0,O}}{\hspace{20pt}\oplus}
            }
        \]
    \item La formule proposée n'est pas en accord avec les électronégativités
        puisque $\chi_{\ce{O}} > \chi_{\ce{C}}$, mais c'est \ce{C} qui porte la
        charge moins (excès d'électrons). Il n'est pas possible de construire un
        schéma de \textsc{Lewis} respectant à la fois la règle de l'octet et
        l'électronégativité, c'est ici la règle de l'octet qui s'impose.
\end{enumerate}

\section{Températures d'ébullition}
\begin{enumerate}
    \item 
        \begin{enumerate}
            \item Par symétrie, la molécule de méthane ne possède pas de moment
                dipolaire permanent (les moments des quatre liaisons \ce{C-H} se
                compensent).
            \item Tous les éléments d'une même colonne ont le même nombre
                d'électrons de valence. Par conséquent, leurs composés
                hydrogénés ont tous la même structure, et en particulier leur
                géométrie est la même que celle de la molécule de méthane en ne
                changeant que l'atome central. De même, tous les composés
                hydrogénés de la colonne du carbone n'ont pas de moment
                dipolaire permanent.
        \end{enumerate}
    \item Les éléments de la famille des halogènes sont bien plus
        électronégatifs que l'hydrogène et les molécules ne sont pas
        symétriques. Tous les composés de type \ce{H-X} où \ce{X} est un
        halogène sont donc polaires. Ainsi les forces de \textsc{Van der Waals}
        entre les composés hydrogénés de la colonne 17 sont plus importantes
        qu'entre les composés hydrogénés de la colonne 14. Ce qui explique les
        différences de température d'ébullition.
    \item La masse molaire de \ce{HI} est plus élevée que celle de \ce{HCl}, ce
        qui indique que la molécule est davantage polarisable. Les interactions
        de \textsc{Van der Waals} entre molécules sont donc plus fortes dans le
        cas de l'iode que dans le cas du chlore, ce qui explique la croissance
        observée.
    \item L'atome de fluor appartient à la deuxième période et il est fortement
        électronégatif. Des liaisons hydrogène peuvent donc se former entre
        molécules de \ce{HF}, ce qui n'est pas possible dans les autres espèces
        chimiques. Comme ces liaisons sont beaucoup plus fortes que les autres
        interactions faibles, elles expliquent la forte anomalie de température
        d'ébullition observée pour \ce{HF}.
\end{enumerate}

\end{document}
