\documentclass[a4paper, 10pt, final, garamond]{book}
\usepackage{cours-preambule}
\usepackage[french]{babel}

\raggedbottom

\makeatletter
\renewcommand{\@chapapp}{Architecture de la matière -- chapitre}
\makeatother

\begin{document}
\setcounter{chapter}{2}

\chapter{TD~: Solides cristallins}

\section{Structure cristalline du niobium}
Le niobium (Nb), de numéro atomique $Z = 41$ et de masse molaire $M =
\SI{92.0}{g.mol^{-1}}$, cristallise à température ambiante dans la structure
cubique centrée, de paramètre de maille $a = \SI{330}{pm}$.
\begin{enumerate}
  \item Déterminer la population $N$ de la maille.
  \item Calculer la masse volumique $\rho$ du niobium.
  \item Déterminer le rayon métallique $r$ du niobium, en précisant où à lieu le
    contact.
  \item Définir et calculer la compacité $C$ de la structure cubique centrée.
\end{enumerate}

\section{Galène}
L'élaboration du plomb par voie sèche repose sur l'extraction et l'exploitation
d'un minerai appelé galène~: le sulfure plomb \ce{PbS}. Ce minerai cristallise
selon une structure type \ce{NaCl}, avec \ce{S^{2-}} sur les nœuds d'un réseau
CFC et \ce{Pb^{2+}} sur les sites octaédriques.
\begin{rdefi}{Données}
  $M_{\ce{Pb}} = \SI{207.2}{g.mol^{-1}}$~; $M_{\ce{S}} =
  \SI{32.1}{g.mol^{-1}}$~; densité $d = \num{7.62}$.
\end{rdefi}
\begin{enumerate}
  \item Représenter la maille élémentaire de la galène.
  \item Déterminer la coordinence de chacun des ions dans cette structure.
  \item Déterminer le paramètre de maille $a$ de la structure.
\end{enumerate}

\section{Trioxyde de tungstène}
Le trioxyde de tungstène \ce{WO3} solide est, en première approche, un solide
ionique. Il présente une structure cubique telle que les ions tungstène
\ce{W^{6+}} occupent les sommets de la maille et les ions oxyde \ce{O^{2-}} le
milieu des arêtes. On note $a$ le paramètre de maille.

\begin{enumerate}
  \item Dessiner une maille et vérifier la stœchiométrie du cristal.
  \item On admet une tangence cation-anion. Calculer la compacité du cristal
    \ce{WO3}.
  \item Le centre du cube et les centres des faces de la maille dessinée
    précédemment sont vides. Calculer le rayon maximal d'un hétéroélément qui
    pourrait s'insérer dans ces sites sans déformation de la structure.
  \item On observe expérimentalement que les cations \ce{M^{+}}, avec \ce{M} qui
    peut être \ce{H}, \ce{Li}, \ce{Na} ou \ce{K}, peuvent s'insérer dans le
    cristal et occupent tous le même type de site. En déduire de quel site il
    s'agit.
\end{enumerate}
\begin{table}[h!]
  \begin{center}
    \caption{Données des rayons ioniques.}
    \label{tab:wo3}
    \begin{tabular}[c]{lcccccc}
      \toprule
      Espèce & \ce{H+} & \ce{Li+} & \ce{Na+} & \ce{K+} & \ce{O^{2-}} & \ce{W^{6+}}
      \\\midrule
      Rayon ionique (\si{pm}) & \num{e-5} & \num{78.0} & \num{98.0} & \num{133}
                              & \num{132} & \num{62.0}
      \\\bottomrule
    \end{tabular}
  \end{center}
\end{table}

\section{Alliages du cuivre}
le cuivre peut être utilisé pur, notamment pour des applications exploitant sa
haute conductivité électrique, ou bien en alliage tels que le laiton (alliage
cuivre-zinc) et le bronze (cuivre-étain).
\begin{rdefi}{Données}
  \begin{itemize}[label=$\diamond$]
    \item $\rho_{\ce{Cu}} = \SI{8.96e3}{kg.m ^{-3}}$~;
    \item $M_{\ce{Cu}} = \SI{63.5}{g.mol^{-1}}$~; $M_{\ce{Ag}} =
      \SI{108}{g.mol^{-1}}$~; $M_{\ce{Zn}} = \SI{65.4}{g.mol^{-1}}$~;
    \item $r_{\ce{Cu}} = \SI{128}{pm}$~; $r_{\ce{Ag}} = \SI{144}{pm}$~;
      $r_{\ce{Zn}} = \SI{134}{pm}$.
  \end{itemize}
\end{rdefi}
\begin{enumerate}
  \item Le cuivre pur cristallins dans un réseau CFC. Représenter la maille et
    déterminer sa population. Déterminer le paramètre de maille $a$.
\end{enumerate}
Lorsqu'un atome a un rayon voisin de celui du cuivre, il peut former des
alliages dits de substitution, où l'hétéroatome remplace un pour plusieurs
atomes de cuivres par maille.
\begin{enumerate}[resume]
  \item L'alliage \ce{Cu}-\ce{Ag} est utilisé pour augmenter la résistance à la
    température du matériau. Dans cette structure, les atomes d'argent
    remplacent les atomes de cuivre aux sommets de la maille.
    \begin{enumerate}
      \item Faire un schéma de la maille. Quelle est la stœchiométrie de
        l'alliage~?
      \item Déterminer le nouveau paramètre de maille $a'$ ainsi que la masse
        volumique $\rho'$ de l'alliage. Commenter.
    \end{enumerate}
  \item Le laiton, alliage \ce{Cu}-\ce{Zn}, est l'alliage le plus fabriqué. Il
    permet d'augmenter la résistance mécanique et la dureté du cuivre, mais
    diminue la densité et la conductivité thermique. La structure du laiton peut
    être décrite par un réseau cubique hôte d'atomes de cuivres avec un atome de
    zinc au centre du cube.
    \begin{enumerate}
      \item Faire un schéma de la maille. Quelle est la stœchiométrie de
        l'alliage~?
      \item Déterminer le nouveau paramètre de maille $a''$ ainsi que la masse
        volumique $\rho''$ de l'alliage.
    \end{enumerate}
  \item Les différences structurales induites par la substitution sont
    responsables d'une modification des propriétés de conduction électrique et
    de résistance mécanique. Proposer une explication.
\end{enumerate}

\section{Structure d'un alliage du titane}
L'alliage le plus utilisé dans l'industrie aéronautique a pour formule brute
\ce{Al_xNi_yTi_z}. Le titane y est présent sous forme $\beta$~: son système
cristallographique est le CFC. Les atomes d'aluminium occupent la totalité des
sites octaédriques, et ceux de nickel occupent tous les sites tétraédriques. Le
paramètre de la maille ainsi formée vaut $a = \SI{589}{pm}$.
\begin{rdefi}{Données}
  \begin{center}
    \begin{tabular}{ccc}
      \toprule
      Atome & Rayon atomique (\si{pm}) & Masse molaire (\si{g.mol^{-1}})
      \\\midrule
      \ce{Ti} & \num{147} & \num{47.90}
      \\
      \ce{Al} & \num{143} & \num{26.98}
      \\
      \ce{Ni} & \num{124} & \num{58.70}
      \\\bottomrule
    \end{tabular}
  \end{center}
\end{rdefi}
\begin{enumerate}
  \item Représenter la maille cubique en perspective.
  \item Déterminer la formule de l'alliage.
  \item Calculer l'habitabilité des sites T et O.
  \item Calculer la compacité et la masse volumique de cet alliage.
  \item Comparer les valeurs trouvées précédemment aux caractéristiques moyennes
    d'un acier courant~: $\rho_a = \SI{7800}{kg.m ^{-3}}$, de compacité $C_{\rm
    acier} = \num{0.70}$. À qualités mécaniques équivalentes, expliquer en quoi
    l'alliage de titane présente de l'intérêt.
\end{enumerate}

\section{Carboglace}
À \SI{195}{K}, le dioxyde de carbone se solidifie dans une structure cristalline
appelée \textit{carboglace}.
\begin{rdefi}{Données}
  $M_{\ce{C}} = \SI{12.0}{g.mol^{-1}}$ et $r_{\ce{C}} = \SI{77.0}{pm}$~;
  $M_{\ce{O}} = \SI{16.0}{g.mol^{-1}}$ et $r_{\ce{O}} = \SI{73.0}{pm}$.
\end{rdefi}
\begin{enumerate}
  \item Rappeler la géométrie de la molécule de dioxyde de carbone \ce{CO2}.
  \item Les atomes de carbone occupent un réseau CFC, de paramètre de maille $a
    = \SI{558}{pm}$. Les molécules s'orientent ensuite selon les diagonales des
    face du cube.
    \begin{enumerate}
      \item Représenter cette maille. Déterminer la population d'une maille.
      \item Déterminer la distance $d$ entre deux atomes de carbone voisins.
        Commenter cette valeur, par comparaison avec la longueur de la liaison
        $\ce{C=O}$ dans \ce{CO2}~: $d_{\ce{C=O}} = \SI{120}{pm}$.
    \end{enumerate}
  \item Déterminer la compacité de cette structure. On considérera que le modèle
    des sphères dures s'applique aux atomes et non à la molécule.
  \item Déterminer la densité de la carboglace.
\end{enumerate}

\end{document}
