\documentclass[../TDAM3.tex]{subfiles}%

\begin{document}
\section{Structure cristalline du niobium}
\enonce{%
	Le niobium (Nb), de numéro atomique $Z = 41$ et de masse molaire $M =
		\SI{92.0}{g.mol^{-1}}$, cristallise à température ambiante dans la structure
	cubique centrée, de paramètre de maille $a = \SI{330}{pm}$.
}%
\QR{%
	Déterminer la population $N$ de la maille.
}{%
	Un atome sur un des sommets est partagé entre huit mailles et compte
	pour 1/8, l'atome central n'appartient qu'à une seule maille donc
	\[
		\boxed{N = 8\times1/8+1=2}
	\]
}%

\QR{%
	Calculer la masse volumique $\rho$ du niobium.
}{%
	La masse d'un atome de niobium est égale à $m_{\ce{Nb}} = M/\Nc_A$,
	masse d'une maille vaut donc $2M/\Nc_A$~; ainsi~:
	\[
		\boxed{\rho = \frac{2M}{\Nc_A a^3} = \SI{8.51e3}{kg.m ^{-3}}}
	\]
}%

\QR{%
	Déterminer le rayon métallique $r$ du niobium, en précisant où à lieu le
	contact.
}{%
	La distance entre atomes situés sur deux sommets vaut $a$, celle entre
	atomes situés sur un sommet et au centre de la maille vaut $a \sqrt{3}/2 <
		a$~: le contact a donc lieu \textbf{le long de la grande diagonale} du cube.
	Ainsi, en comptant successivement les atomes,
	\[
		a \sqrt{3} = r+2r+r
		\qqdonc
		\boxed{r = a \frac{\sqrt{3}}{4} = \SI{143}{pm}}
	\]
}%

\QR{%
	Définir et calculer la compacité $C$ de la structure cubique centrée.
}{%
	La compacité est la proportion du volume de la maille réellement occupé
	par la matière. Pour la structure CC,
	\[
		C = \frac{2\times \frac{4}{3}\pi r^3}{a^3} = \frac{8\pi}{3}\times \left(
		\frac{\sqrt{3}}{4} \right)^{3}
		\qqdonc
		\boxed{C = \frac{\pi \sqrt{3}}{8} = \num{0.68}}
	\]
}%

\end{document}
