\documentclass[./TDT3.tex]{subfiles}%

\begin{document}

\section[s]"1"{Échauffement d'une bille en mouvement dans l'air}
\enonce{%
	Une bille métallique, de capacité thermique massique $c$ (supposée constante)
	est lancée vers le haut avec une vitesse $\vfo$ dans le champ de pesanteur
	$\gf$ uniforme. Elle atteint une altitude $h$ puis redescend.
}%
\begin{tcn}(data)<lfnt>{Données}
	$g = \SI{9.81}{m.s^{-2}}~;~c = \SI{0.4}{kJ.kg^{-1}}~;~v_0 =
		\SI{10}{m.s^{-1}}~;~h = \SI{5}{m}$.
\end{tcn}
\QR{%
	Déterminer l'altitude maximale $h_0$ que peut atteindre la bille si on néglie
	les forces de frottement fluide entre l'air et la bille. Exprimer $h_0$ en
	fonction de $v_0$ et $g$.
}{%
	solu
}%
\enonce{%
	On constate que l'altitude $h$ est inférieure à $h_0$, à cause des forces de
	frottement. Calculer la variation de température $\Delta{T}$ de cette bille
	entre l'instant où elle est lancée et l'instant où elle atteint son point le
	plus haut, en supposant que~:
	\begin{itemize}
		\item On néglige toute variation de volume de la bille~;
		\item l'air ambiant reste macroscopiquement au repos~;
		\item le travail des forces de frottement se dissipe pour moitié dans l'air
		      ambiant et pour moitié dans la bille.
	\end{itemize}
}%

\QR{%
	Exprimer $\Delta{T}$ en fonction de $h_0$, $h$, $g$ et $c$.
}{%
	solu
}%
\QR{%
	Calculer $h_0$ puis $\Delta{T}$.
}{%
	solu
}%

\end{document}
