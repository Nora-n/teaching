\documentclass[./TDT3.tex]{subfiles}%

\begin{document}

\section[s]"1"{Détente de \textsc{Joule Gay-Lussac}}
\enonce{%
	Le dispositif étudié dans cet exercice a été mis eu point au
	\textsc{xix}\ieme{} siècle par \textbf{Joule} et \textbf{Gay-Lussac} en vue
	d’étudier le comportement des gaz. Deux compartiments indéformables aux parois
	calorifugées communiquent par un robinet initialement fermé. Le compartiment
	(1), de volume $V_1$, est initialement rempli de gaz en équilibre à la
	température $T_i$. Le vide est fait dans le compartiment (2). Une fois le
	robinet ouvert, un nouvel équilibre s’établit, caractérisé par une température
	$T_f$ du gaz.
}%
\QR{%
	% Lalande + schéma Langevin
	Faire un schéma des états initial et final. En considérant comme système fermé
	le contenu des deux compartiments, caractériser la transformation subie par ce
	système.
}{%
	solu
}%
\QR{%
	% Langevin
	Montrer que cette détente est isoénergétique, c'est-à-dire que l'énergie
	interne du gaz ne varie pas aucours de la transformation. Cette propriété
	dépend-elle du gaz~?
}{%
	solu
}%
\QR{%
	% Langevin
	Déterminer la température $T_f$ dans le cas où le gaz est parfait.
}{%
	solu
}%
\enonce{%
En réalité, on observe une légère diminution de la température du gaz dans la
quasi-totalité des cas. L'expérience est ici réalisée avec du dioxygène, qui
peut être efficacement modélisé par un gaz de \textsc{van der Waals}.
L'équation d'état d'une tel gaz s'écrit
\[
	\left( P + \frac{an^2}{V^2} \right) (V-nb) = nRT
	\qqMath{tel que}
	U = nC_{V,m}T - \frac{na^2}{V}
\]
avec $a$ et $b$ deux constantes positives caractéristiques du gaz. Les travaux
de \textsc{van der Waals} sur le comportement microscopique des gaz ont été de
première importance, et il en a été récompensé par le prix \textsc{Nobel}
1910. Pour le dioxygène, $C_{V,m} = \SI{21}{J.K^{-1}.mol^{-1}}$ et $a =
	\SI{1.32}{USI}$.
}%
\QR{%
	% Lalande et Langevin
	Interpréter physiquement l'origine du terme de cohésion $a$ et du volume exclu
	$b$, et donner leur dimension. Nommer et interpréter la constante $C_{V,m}$.
}{%
	solu
}%
\QR{%
	% Lalande
	Déterminer l'expression de la température finale $T_f$ du gaz.
}{%
	solu
}%
\QR{%
	Effectuer l'application numérique de $\Delta{T}$ pour $n = \SI{0.80}{mol}$ et
	$V_1 = V_2 = \SI{5.0}{L}$.
}{%
	solu
}%

\end{document}
