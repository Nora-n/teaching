\documentclass[./TDT3.tex]{subfiles}%

\begin{document}

\section[s]"2"{Transformation polytropique}
\enonce{%
	Une transformation polytropique est une transformation d’un gaz pour laquelle
	il existe un coefficient $k > 0$ tel que $PV^k = \cte$ tout au long de la
	transformation. De telles transformations sont intermédiaires entre des
	adiabatiques et des isothermes, et se rencontrent en thermodynamique
	industrielles, par exemple lorsque le système réfrigérant ne permet pas
	d’éliminer tout le transfert thermique produit par une réaction chimique. On
	raisonnera à partir d’une transformation quasi-statique d’un gaz parfait.
	\begin{tcn}(data)<lfnt>{Données}
		\vspace{-15pt}
		\leftcenters{%
			Pour un gaz parfait,
		}{%
			$\DS C_V = \frac{nR}{\gamma-1}$ et $C_P = \frac{\gamma nR}{\gamma-1}$.
		}%
		\vspace{-15pt}
	\end{tcn}
}%
\QR{%
	À quelles transformations connues correspondent les cas $k = 0$, $k=1$ et
	$k=+\infty$~?
}{%
	solu
}%
\QR{%
	Calculer le travail des forces de pression pour un gaz subissant uns
	transformation polytropique entre deux états $(P_0,V_0,T_0)$ et
	$(P_1,V_1,T_1)$ en fonction d'abord des pressions et des volumes, puis dans un
	second temps des températures seulement.
}{%
	solu
}%
\QR{%
	Montrer que le transfert thermique au cours de la transformation précédente
	s'écrit
	\[
		Q = nR \left( \frac{1}{\gamma-1}-\frac{1}{\gamma-1} \right) (T_1-T_0)
	\]
}{%
	solu
}%
\QR{%
	Analyser les cas $k = 0$, $k = 1$ et $k = +\infty$, et vérifier la cohérence
	avec l'analyse initiale.
}{%
	solu
}%
\QR{%
	À quel type de transformation correspond le cas $k = \gamma$~?
}{%
	solu
}%

\end{document}
