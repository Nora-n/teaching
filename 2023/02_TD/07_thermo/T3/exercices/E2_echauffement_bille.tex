\documentclass[../TDT3.tex]{subfiles}%

\begin{document}

\section[s]"1"{Échauffement d'une bille en mouvement dans l'air}
\enonce{%
	Une bille métallique, de capacité thermique massique $c$ (supposée constante)
	est lancée vers le haut avec une vitesse $\vfo$ dans le champ de pesanteur
	$\gf$ uniforme. Elle atteint une altitude $h$ puis redescend.
}%
\begin{tcn}(data)<lfnt>{Données}
	$g = \SI{9.81}{m.s^{-2}}~;~c = \SI{0.4}{kJ.kg^{-1}}~;~v_0 =
		\SI{10}{m.s^{-1}}~;~h = \SI{5}{m}$.
\end{tcn}
\QR{%
Déterminer l'altitude maximale $h_0$ que peut atteindre la bille si on néglie
les forces de frottement fluide entre l'air et la bille. Exprimer $h_0$ en
fonction de $v_0$ et $g$.
}{%
Système = \{bille\} dans $\Rc\ind{terre}$ supposé galiléen. Seule force subie
est le poids. Avec $\zp(0) = v_0$ et $z(0) = 0$~:
\[
	\zpp = -g \Ra \zp = -gt + v_0 \Ra z = -\frac{1}{2}gt^2+ v_0t
\]
Altitude maximale quand $\DS \zp(t\ind{max}) = 0 \Lra t\ind{max} = \frac{v_0}{g}$,
soit
\[
	\boxed{h_0 = z(t\ind{max}) = \frac{v_0{}^2}{2g}}
\]
}%
\enonce{%
	On constate que l'altitude $h$ est inférieure à $h_0$, à cause des forces de
	frottement. Calculer la variation de température $\Delta{T}$ de cette bille
	entre l'instant où elle est lancée et l'instant où elle atteint son point le
	plus haut, en supposant que~:
	\begin{itemize}
		\item On néglige toute variation de volume de la bille~;
		\item l'air ambiant reste macroscopiquement au repos~;
		\item le travail des forces de frottement se dissipe pour moitié dans l'air
		      ambiant et pour moitié dans la bille.
	\end{itemize}
}%

\QR{%
	Exprimer $\Delta{T}$ en fonction de $h_0$, $h$, $g$ et $c$.
}{%
	L'énergie mécanique de la bille, sans frottement, est constante, et en
	utilisant l'énergie au maximum d'altitude s'écrit $E_{m,0} = mgh_0$. Avec
	frottement, son énergie n'est \textbf{plus constante}, mais à son maximum
	d'altitude on a $E_m = mgh$.
	\smallbreak
	La différence entre ces énergie est égale à l'énergie perdue par frottement
	avec l'air, soit le travail reçu de l'air~:
	\[
		W_f = E_{m,0} - E_m = mg(h_0 - h)
	\]
	Or, la moitié de cette énergie est évacuée dans l'air, tandis que l'autre
	moitié est emmagasinée sous forme d'énergie interne de la bille. Ainsi,
	\begin{gather*}
		\Delta{U} = \frac{1}{2} mg (h_0-h)
		\Lra
		mc\Delta{T} = \frac{1}{2}mg (h_0-h)
		\\\Lra
		\boxed{\Delta{T} = \frac{g}{2c}(h_0-h)}
	\end{gather*}
	étant donné que la bille est supposée incompressible et indilatable, et
	qu'elle suit donc la première loi de \textsc{Joule}.
}%
\QR{%
	Calculer $h_0$ puis $\Delta{T}$.
}{%
	On trouve $h_0 \approx \SI{5.097}{m}$, soit $\xul{\Delta{T} \approx
			\SI{1.2}{mK}}$.
}%

\end{document}
