\documentclass[../TDT3.tex]{subfiles}%

\begin{document}

\section[s]"1"{Calorimétrie du fer}
\enonce{%
	La calorimétrie consiste en la mesure d’échanges thermiques. On utilise pour
	cela un calorimètre, dont les parois son conçues pour minimiser les échanges
	thermiques entre l’intérieur et l’extérieur du calorimètre. Ces échanges
	seront considérés comme nuls. Les transformations se font à la pression
	atmosphérique constante, et sont donc supposées monobares.
}%
\begin{blocQR}
	\item
	\enonce{%
	\textbf{Mesure de la capacité thermique $C$ du calorimètre}
	\smallbreak
	Le calorimètre contient initialement une masse $m = \SI{100}{g}$ d'eau,
	l'ensemble étant à la température ambiante $\th_0 =
		\SI{20.0}{\degreeCelsius}$. On ajoute alors la même masse d'eau à $\th_1 =
		\SI{80.0}{\degreeCelsius}$. On remue pour homogénéiser le système et on
	mesure la température $\th_f = \SI{43.6}{\degreeCelsius}$. La capacité
	thermique massique de l'eau est $c = \SI{4.18}{kJ.K^{-1}.kg^{-1}}$.
	}%
	\QR{%
		Montrer que l'enthalpie du système \{parois + eau\} reste constante au
		cours de la transformation.
	}{%
		solu
	}%
	\QR{%
		En déduire la capacité thermique massique $C$ des parois internes du
		calorimètre.
	}{%
		solu
	}%
	\QR{%
		Quelle est la masse en eau $m_0$ du calorimètre~?
	}{%
		solu
	}%
	\QR{%
		Pourquoi ne faut-il pas trop attendre pour mesurer la température finale~?
		Quelle serait sa valeur si on attendait un temps très long~?
	}{%
		solu
	}%
	\item
	\enonce{%
		\textbf{Mesure de la capacité thermique massique du fer}
		\smallbreak
		On considère l'état initial où une masse $m = \SI{100}{g}$ d'eau et une
		masse $m_f = \SI{140}{g}$ de fer sont dans le calorimètre. Une résistance
		électrique de masse négligeable est aussi immergée dans le liquide.
		Lensemble est initialement à la température $\th_0 =
			\SI{20.0}{\degreeCelsius}$. Pendant une durée $\tau = \SI{30}{s}$, un
		générateur électrique fourni à la résistance une puissance $\Pc =
			\SI{350}{W}$. On homogénéise la solution et on mesure la température $\th_f'
			= \SI{34.8}{\degreeCelsius}$.
	}%
	\resetQ
	\QR{%
		Exprimer la variation d'enthalpie du système \{parois + eau + fer +
		résistance\} au cours de la transformation précédente.
	}{%
		solu
	}%
	\QR{%
		En déduire l'expression puis la valeur de la capacité thermique massique du
		fer $c_{\ce{Fe}}$.
	}{%
		solu
	}%
\end{blocQR}

\end{document}
