\documentclass[../TDT3.tex]{subfiles}%

\begin{document}

\section[s]"2"{Transformation polytropique}
\enonce{%
	Une transformation polytropique est une transformation d’un gaz pour laquelle
	il existe un coefficient $k > 0$ tel que $PV^k = \cte$ tout au long de la
	transformation. De telles transformations sont intermédiaires entre des
	adiabatiques et des isothermes, et se rencontrent en thermodynamique
	industrielles, par exemple lorsque le système réfrigérant ne permet pas
	d’éliminer tout le transfert thermique produit par une réaction chimique. On
	raisonnera à partir d’une transformation quasi-statique d’un gaz parfait.
	\begin{tcn}(data)<lfnt>{Données}
		\vspace{-15pt}
		\leftcenters{%
			Pour un gaz parfait,
		}{%
			$\DS C_V = \frac{nR}{\gamma-1}$ et $\DS C_P = \frac{\gamma nR}{\gamma-1}$.
		}%
		\vspace{-15pt}
	\end{tcn}
}%
\QR{%
	À quelles transformations connues correspondent les cas $k = 0$, $k=1$ et
	$k=+\infty$~?
}{%
	\leavevmode\vspace*{-15pt}\relax
	\begin{itemize}
		\item $k=0 \Ra P = \cte$ donc \textbf{isobare}~;
		\item $k=1 \Ra PV = \cte = nRT$ donc \textbf{isotherme}~;
		\item $K = +\infty \Ra V = \cte$ donc \textbf{isochore} (sinon toute infime
		      variation de volume entraîne des effets infinis).
	\end{itemize}
}%
\QR{%
Calculer le travail des forces de pression pour un gaz subissant uns
transformation polytropique entre deux états $(P_0,V_0,T_0)$ et
$(P_1,V_1,T_1)$ en fonction d'abord des pressions et des volumes, puis dans un
second temps des températures seulement.
}{%
Par définition, on a tout au long de la transformation $PV^k = P_0V_0{}^k$,
soit $\boxed{P = P_0 (V_0/V)^k}$. Si de plus la transformation est
quasi-statique, c'est-à-dire mécaniquement réversible donc que l'équilibre
mécanique est atteint tout au long de la transformation, on aura $P =
	P\ind{ext}$. Le travail reçu entre les états est donc~:
\begin{DispWithArrows*}
	W &=
	-\int_{V_0}^{V_1} P \dd{V}
	\\\Lra
	W &=
	-\int_{V_0}^{V_1} P_0V_0{}^k \frac{\dd{V}}{V^k}
	\\\Lra
	W &=
	-P_0V_0{}^k \frac{V_1^{1-k} - V_0^{1-k}}{1-k}
	\Arrow{$P_0V_0{}^k = P_1V_1{}^k$\\$P_1V_1{}^kV^{1-k} = P_1V_1$}
	\\\Lra
	W &=
	-\frac{1}{1-k} \left( P_1V_1 - P_0V_0 \right)
	\Arrow{$PV = nRT$}
	\\\Lra
	\Aboxed{W &= \frac{nR}{k-1} (T_1-T_0)}
	\qed
\end{DispWithArrows*}
}%
\QR{%
	Montrer que le transfert thermique au cours de la transformation précédente
	s'écrit
	\[
		Q = nR \left( \frac{1}{\gamma-1}-\frac{1}{k-1} \right) (T_1-T_0)
	\]
}{%
	\leavevmode\vspace*{-15pt}\relax
	\begin{gather*}
		\Delta{U} = W + Q
		\Lra
		Q = \Delta{U} - W
		\Lra
		Q = C_V\Delta{T} - \frac{nR}{k-1}\Delta{T}
		\\\Lra
		\boxed{Q = nR \left( \frac{1}{\gamma-1}-\frac{1}{k-1} \right) (T_1-T_0)}
		\qed
	\end{gather*}
}%
\QR{%
	Analyser les cas $k = 0$, $k = 1$ et $k = +\infty$, et vérifier la cohérence
	avec l'analyse initiale.
}{%
	\leavevmode\vspace*{-15pt}\relax
	\begin{itemize}
		\item $k=0$ donne par le calcul~:
		      \begin{gather*}
			      \boxed{Q = \frac{\gamma nR}{\gamma-1} = C_P\Delta{T}}
			      \intertext{Or, $k=0 \Ra$ isobare, donc}
			      \Delta{H} = \underbracket[1pt]{W_u}_{0} + Q
			      \\\Lra
			      C_P\Delta{T} = Q
		      \end{gather*}
		      On retrouve bien la même chose.
		\item $k=1$ donne une forme indéterminée mathématiquement, ce qui implique
		      qu'on ait $\Delta{T} = 0$ pour la réalité physique~: c'est cohérent
		      avec le résultat précédent de transformation isotherme.
		\item $k\to \infty$ donne
		      \begin{gather*}
			      Q = C_V\Delta{T}
			      \Lra
			      \Delta{U} = C_V\Delta{T}
		      \end{gather*}
		      ce qui est effectivement la relation connue pour un gaz parfait.
	\end{itemize}
}%
\QR{%
	À quel type de transformation correspond le cas $k = \gamma$~?
}{%
	$k = \gamma$ donne $\boxed{Q = 0}$~: c'est une transformation
	\textbf{adiabatique}.
}%

\end{document}
