\documentclass[../TDT3.tex]{subfiles}%

\begin{document}
\section[s]"3"{Chauffage d'une chambre}

\enonce{%
	On étudie le chauffage d'une chambre au dernier étage de l'internat en hiver.
	On installe un radiateur électrique d'appoint fournissant une puissance de
	chauffe $\Pc_c$. Le volume de la chambre est $V = \SI{36}{m^3}$, et est rempli
	d'air de capacité thermique molaire $C_{V,m} = \frac{5}{2}R$. On la suppose
	vide de meubles.
	\smallbreak
	Les échanges thermiques se font \textit{via} par deux surfaces~: le mur et les
	vitres en contact avec l'extérieur et le toit, de surfaces égales $S =
		\SI{12}{m^2}$. Les autres surfaces sont supposées à l'équilibre thermique du
	fait des chambres voisines et en-dessous. On note $T\ind{int}(t)$ la
	température intérieure, et $T\ind{ext} = \SI{10}{\degreeCelsius}$ la
	température extérieure, supposée constante.
	\smallbreak
	Les fuites thermiques à la date $t$ à travers le mur sont données par la
	puissance $\Pc\ind{mur} = g\ind{mur}S (T\ind{int}(t) - T\ind{ext})$, et celles
	à travers le toit par $\Pc\ind{toit} = g\ind{toit}S (T\ind{int}(t) -
		T\ind{ext})$.
	\smallbreak
	On souhaite maintenir la température à une température de confort $T_c =
		\SI{19}{\degreeCelsius}$. La pression de l'air intérieur est $P_0 =
		\SI{1.0}{bar}$ à cette température.
	\begin{tcn}(data)<lfnt>{Données}
		$g\ind{mur} = \SI{2.90}{W.m^{-2}.K^{-1}}$ et $g\ind{toit} =
			\SI{0.50}{W.m^{-2}.K^{-1}}$, $R = \SI{8.314}{J.K^{-1}.mol^{-1}}$.
	\end{tcn}
}%
\QR{%
	Faire un schéma représentant la pièce, le radiateur et l'extérieur, en faisant
	apparaître les transferts thermiques entrant en rouge et les transferts
	thermiques sortant en bleu.
}{%
	solu
}%
\QR{%
	Calculer le nombre de moles d'air présentes dans la véranda dans les
	conditions $(T_c,P_0)$. En déduire la capacité thermique $C_V$ de l'air
	contenu dans la vérande. Faire l'application numérique.
}{%
	solu
}%
\QR{%
	Quelle est la puissance $\Pc_c$ fournie par le radiateur pour maintenir une
	telle température de confort dans les conditions mentionnées ci-dessus~?
}{%
	solu
}%
\enonce{%
	On doit partir pour une khôlle et dîner, et on se demande s'il vaut mieux
	couper le chauffage ou le maintenir. On suppose alors qu'on arrête le
	chauffage à $t=0$, et qu'on revient \SI{3}{h} plus tard au temps $t_1$.
}%
\QR{%
	En supposant qu'il n'y a pas de circulation d'air, appliquer le premier
	principe sous forme différentielle à l'air de la chambre et déterminer
	l'équation différentielle vérifier par $T\ind{int}(t)$ pour $t \in [0,t_1]$.
	On introduira un temps caractéristique $\tau$ que l'on calculera.
}{%
	solu
}%
\QR{%
	Tracer cette évolution au cours du temps, et déterminer la température
	$T\ind{int,f}$ lors du retour dans la chambre.
}{%
	solu
}%
\QR{%
	Comme il fait très froid, on pousse la puissance de chauffe à son maximum,
	$\Pc_{c,\rm max} = \SI{2.0}{kW}$. Écrire la nouvelle équation différentielle
	satisfaite par $T\ind{int}(t)$, la résoudre et calculer la durée nécessaire
	pour retrouver la température de confort $T_c$. On appelle cet instant $t_2$.
}{%
	solu
}%
\QR{%
	Déterminer alors la différence d'énergie entre les deux situations~:
	\begin{itemize}
		\item On garde le chauffage à la puissance $\Pc_c$ de $t = 0$ à $t_2$~;
		\item On a éteint le chauffage de $t=0$ à $t_1$, mais on le rallume de $t_1$
		      à $t_2$ avec $\Pc_{c,\rm max}$.
	\end{itemize}
	On suppose que l'énergie électrique est parfaitement convertie en chaleur.
	Sachant que pour l'électricité on a $\SI{1}{kWh} \approx
		\SI{0.27}{\text{\euro}}$ avec l'augmentation de février 2024, déterminer
	l'écart financier entre ces deux méthodes. Commenter.
}{%
	solu
}%

\end{document}
