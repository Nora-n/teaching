\documentclass[a4paper, 10pt, final, garamond]{book}
\usepackage{cours-preambule}

\makeatletter
\renewcommand{\@chapapp}{Thermodynamique -- chapitre}
\makeatother

\hfuzz=5.003pt

% \toggletrue{student}
% \toggletrue{corrige}
% \renewcommand{\mycol}{black}
% \renewcommand{\mycol}{gray}

\begin{document}
\setcounter{chapter}{3}

\settype{enon}
\settype{solu_prof}
\settype{solu_stud}

\chapter{\cswitch{Correction du TD}{TD~: Second principe de la thermodynamique}}

\enonce{%
	\begin{tcn}(defi){Données}
		Pour un système fermé, de température $T$, de pression $P$ et de volume $V$
		subissant une transformation entre deux états d'équilibre $(i)$ et $(f)$, la
		variation d'entropie est~:
		\begin{itemize}
			\item pour un gaz parfait,
			      \begin{gather*}
				      \boxed{\Delta S = C_V \ln \frac{P_f}{P_i} + C_P \ln \frac{V_f}{V_i}}
				      \qou
				      \boxed{\Delta S = C_V \ln \frac{T_f}{T_i} + nR \ln \frac{V_f}{V_i}}
				      \qou
				      \boxed{\Delta S = C_P \ln \frac{T_f}{T_i} - nR \ln \frac{P_f}{P_i}}
			      \end{gather*}
			\item pour une phase condensée,
			      \begin{gather*}
				      \boxed{\Delta S = C \ln \frac{T_f}{T_i}}
			      \end{gather*}
		\end{itemize}
	\end{tcn}
}%

\resetQ
\subfile{exercices/E1_melange.tex}

\resetQ
\subfile{exercices/E2_enceinte.tex}

\resetQ
\subfile{exercices/E3_joule.tex}

\resetQ
\subfile{exercices/E4_cycle.tex}

\resetQ
\subfile{exercices/E5_nthermo.tex}

\resetQ
\subfile{exercices/E6_nmasse.tex}
\end{document}
