\documentclass[../TDT4.tex]{subfiles}%

\begin{document}
\section{Méthode des mélanges dans un calorimètre}
\enonce{%
	Un calorimètre de capacité thermique $C = \SI{150}{J.K^{-1}}$ contient
	initialement une masse $m_1 = \SI{200}{g}$ d'eau à $\theta_1 =
		\SI{20}{\degreeCelsius}$, en équilibre thermique avec le calorimètre. On plonge
	dans l'eau un bloc de fer de masse $m_2 = \SI{100}{g}$ initialement à la
	température $\theta_2 = \SI{80.0}{\degreeCelsius}$.
	\begin{tcn}(defi)<lftt>{Données}
		$c_{\ce{Fe}} = \SI{452}{J.K^{-1}.kg^{-1}}$ et $c_{\rm eau} =
			\SI{4185}{J.K^{-1}.kg^{-1}}$.
	\end{tcn}
}%

\QR{%
	Calculer la température d'équilibre $T_f$.
}{%
	solu
}%
\QR{%
	Calculer la variation d'entropie de l'eau, du fer et du calorimètre.
}{%
	solu
}%
\QR{%
	En déduire l'entropie créée au cours de la transformation. Celle-ci
	est-elle réversible~?
}{%
	solu
}%
\end{document}
