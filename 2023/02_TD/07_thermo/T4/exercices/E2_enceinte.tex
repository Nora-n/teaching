\documentclass[../TDT4.tex]{subfiles}%

\begin{document}
\section[s]"1"{Équilibre d'une enceinte à deux compartiments}
\enonce{%
	Une enceinte indéformable aux parois calorifugées est séparée en deux
	compartiments par une cloison étanche, diatherme et mobile sans frottement.
	Les deux compartiments contiennent un même gaz parfait. Dans l'état initial,
	la cloison est maintenue au milieu de l'enceinte. Le gaz du compartiment 1 est
	dans l'état $(T_0,P_0,V_0)$ et le gaz du compartiment 2 dans l'état
	$(T_0,2P_0,V_0)$. On laisse alors la cloison bouger librement jusqu'à ce que
	le système atteigne un état d'équilibre.
}%
\QR{%
	Exprimer les quantités de matière $n_1,n_2$ dans chaque compartiment en
	fonction de $n_0 = P_0V_0/RT_0$.
}{%
	solu
}%
\QR{%
	Exprimer la température, le volume et la pression du gaz de chaque
	compartiment dans l'état final, en fonction de $n_0,T_0$ et $V_0$.
}{%
	solu
}%
\QR{%
	Exprimer l'entropie créée en fonction de $n_0$.
}{%
	solu
}%
\end{document}
