\documentclass[../TDT4.tex]{subfiles}%

\begin{document}
\section[s]"2"{Effet \textsc{Joule}}
\enonce{%
	Considérons une masse $m = \SI{100}{g}$ d'eau, dans laquelle plonge un
	conducteur de résistance $R = \SI{20}{\ohm}$. L'ensemble forme un système
	$\Sigma$, de température initiale $T_0 = \SI{20}{\degreeCelsius}$. On impose
	au travers de la résistance un courant $I = \SI{1}{A}$ pendant une durée $\tau
		= \SI{10}{s}$. L'énergie électrique dissipée dans la résistance peut être
	traitée du point de vue de la thermodynamique comme un transfert thermique
	$Q\ind{élec}$ reçu par $\Sigma$.
	\begin{tcn}(defi)<lftt>{Données}
		\begin{itemize}
			\item Capacité thermique de la résistance~: $C_R = \SI{8}{J.K^{-1}}$
			\item Capacité thermique massique de l'eau~: $c\ind{eau} =
				      \SI{4.18}{J.g^{-1}.K^{-1}}$.
		\end{itemize}
	\end{tcn}
}%

\QR{%
	La température de l'ensemble est maintenue constante. Quelle est la
	variation d'entropie du système~? Quelle est l'entropie créée~?
}{%
	solu
}%
\QR{%
	Commenter le signe de l'entropie créée. Que peut-on en déduire à propos
	du signe d'une résistance~?
}{%
	solu
}%
\QR{%
	Le même courant passe dans le même conducteur pendant la même durée,
	mais cette fois $\Sigma$ est isolé thermiquement. Calculer sa variation
	d'entropie et l'entropie créée.
}{%
	solu
}%
\end{document}
