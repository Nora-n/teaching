\documentclass[../TDT4.tex]{subfiles}%

\begin{document}
\section[s]"2"{Possibilité d'un cycle}
\enonce{%
	On raisonne sur une quantité de matière $n = \SI{1}{mol}$ de gaz parfait qui
	subit la succession de transformations (idéalisées) suivantes~:
	\begin{itemize}
		\item[b]{AB}: détente isotherme de $P\ind{A} = \SI{2}{bar}$ et $T\ind{A} =
			\SI{300}{K}$, jusqu'à $P\ind{B} = \SI{1}{bar}$ en restant en contact avec
		un thermostat de température $T_0 = T\ind{A}$~;
		\item[b]{BC}: évolution isobare jusqu'à $V\ind{C} = \SI{20.5}{L}$, toujours
		en restant en contact avec le thermostat à $T_0$~;
		\item[b]{CA}: compression adiabatique réversible jusqu'à revenir à l'état A.
	\end{itemize}
	On suppose le gaz diatomique.
}%

\QR{%
	Quel est le coefficient adiabatique~? Représenter ce cycle en diagramme de
	\textsc{Watt} $(P,V)$.
}{%
	solu
}%
\QR{%
	À partir du diagramme, déterminer le signe du travail total des forces de
	pression au cours du cycle. En déduire s'il s'agit d'un cycle moteur ou d'un
	cycle récepteur.
}{%
	solu
}%
\QR{%
	Déterminer l'entropie créée entre A et B. Commenter.
}{%
	solu
}%
\QR{%
	Calculer la température en C, le travail $W\ind{BC}$ et le transfert thermique
	$Q\ind{BC}$ reçus par le gaz au cours de la transformation BC. En déduire
	l'entropie échangée avec le thermostat ainsi que l'entropie créée. Conclure~:
	le cycle proposé est-il réalisable~? Le cycle inverse l'est-il~?
}{%
	solu
}%

\end{document}
