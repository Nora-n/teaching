\documentclass[../TDT4.tex]{subfiles}%

\begin{document}
\section{Corps en contact avec $n$ thermostats quasi-statiques}
\enonce{%
	Un métal de capacité thermique $C_p$ passe de la température initiale $T_0$ à
	la température finale $T_f = T_N$ par contacts successifs avec une suite $N$
	thermostats de températures $T_i$ étagées entre $T_0$ et $T_f$. On prendra le
	rapport $T_{i+1}/T_i = \alpha$ constant.
}%
\QR{%
	Exprimer pour chaque étape la variation d'entropie du corps $\Delta S$ en
	fonction de $m, c$ et $\alpha$.
}{%
	solu
}%
\QR{%
	Calculer le transfert thermique reçu par le métal sur une étape en fonction de
	$T_{i+1}$ et $T_i$, puis l'entropie échangée $S\ind{ech}$ en fonction de $m,
		c$ et $\alpha$.
}{%
	solu
}%
\QR{%
	Calculer la variation d'entropie du corps $\Delta S$, l'entropie échangée
	$S\ind{ech}$ ainsi que l'entropie créée $S_c$ sur l'ensemble en fonction de
	$C_p, \alpha$ et $N$.
}{%
	solu
}%
\QR{%
	Étudier $S\ind{cr}$ pour $N \ra \infty$. On exprimera $\alpha$ en fonction de
	$T_f, T_i$ et $N$, et on utilisera le développement limité $\exp(x) =
		1+x+x^2/2$ pour $x$ petit devant 1. Conclure.
}{%
	solu
}%

\end{document}
