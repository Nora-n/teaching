\documentclass[../TDT4.tex]{subfiles}%

\begin{document}
\section[s]"2"{Masse posée sur un piston}
\enonce{%
	Considérons une enceinte hermétique, diatherme, fermée par un piston de masse
	négligeable pouvant coulisser sans frottements. Cette enceinte contient un gaz
	supposé parfait. Elle est placée dans l'air, à température $T_0$ et pression
	$P_0$.
}%
\QR{%
	On place une masse $m$ sur le piston. Déterminer les caractéristiques du gaz
	une fois les équilibres thermique et mécanique atteints.
}{%
	\leavevmode\vspace*{-20pt}\relax
	\begin{align*}
		\beforetext{Équilibre thermique $\Ra$}
		\Aboxed{T_F & = T_I = T_0}
		\\\beforetext{Équilibre mécanique $\Ra$}
		P_I S = P_0 S
		            & \qet
		P_FS = P_0S + mg
		\\\Lra
		\Aboxed{P_F & = P_0 + \frac{mg}{S}}
		\\\beforetext{Gaz parfait $\Ra$}
		\Aboxed{V_F & = \frac{nRT_0}{P_0 + \frac{mg}{S}}}
	\end{align*}
}%
\QR{%
	Déterminer le transfert thermique échangé $Q$ et l'entropie créée.
}{%
	\leavevmode\vspace*{-20pt}\relax
	\begin{itemize}
		\item[b]{Transfert thermique}: on étudie le système \{gaz\} (le piston étant
		négligé). Il est soumis à la pression extérieure $P_0$ et à la force
		exercée par la masse $m$, donnant un surplus de pression $mg/S$. Ainsi, le
		système est soumis à une pression apparent
		\[
			P\ind{app} = P_0 + \frac{mg}{S}
		\]
		qui est constante. On a donc une transformation monobare, soit
		\begin{align*}
			W         & = -P\ind{app} (V_F-V_I)
			\\\beforetext{1\iere{} loi de \textsc{Joule}, $T_I = T_F$~:}
			\Delta{U} & = W + Q
			\\\Ra
			Q         & = P\ind{app}(V_F-V_I)
			\\\Lra
			Q         & = P\ind{app} nRT_0 \left( \frac{1}{P\ind{app}} - \frac{1}{P_0}
			\right)
			\\\Lra
			Q         & =
			\cancel{P\ind{app}} nRT_0
			\frac{\overbracket{P_0 - P\ind{app}}^{=-mg/S}}{P_0 \cancel{P\ind{app}}}
			\\\Lra
			\Aboxed{Q & = - \frac{nRT_0}{P_0}\frac{mg}{S}}
		\end{align*}
		\item[b]{Entropie créée}:
		\leavevmode\vspace*{-25pt}\relax
		\begin{align*}
			\Delta{S} & =
			C_P \underbracket[1pt]{\ln \frac{T_F}{T_I}}_{=0} -
			nR \ln \frac{P_F}{P_I} = -nR \ln (1+\frac{mg}{P_0S})
			\\\Lra
			S\ind{cr} & = \Delta{S} - \frac{Q}{T_0}
			\\\Lra
			\Aboxed{
			S\ind{cr} & = nR \left(
				\frac{mg}{P_0S} - \ln (1 + \frac{mg}{P_0S})
				\right)
			}
		\end{align*}
	\end{itemize}
}%
\QR{%
	On réalise la même expérience, mais en $N$ étapes successives, par exemple en
	ajoutant du sable «~grain à grain~». Déterminer l'entropie créée dans la
	limite $N \to \infty$.
}{%
	Dans le cas où la transformation est réalisée en $N \gg 1$ étapes, une masse
	$m/N$ est ajoutée à chaque étape. Au cours d'une étape, on a donc création de
	\begin{align*}
		S_{\mathrm{cr},i} & =
		nR \left(
		\frac{mg/N}{P_0S} - \ln (1 + \frac{mg/N}{P_0S})
		\right)
		\\\beforetext{Développement limité}
		S_{\mathrm{cr},i} & =
		nR \left(
		\frac{mg/N}{P_0S} - \frac{mg/N}{P_0S} +
		\frac{1}{2} \left( \frac{mg/N}{P_0S} \right)^2\right)
		\\\Lra
		S_{\mathrm{cr},i} & = \frac{nR}{2} \left( \frac{mg}{NP_0S} \right)^2
		\\\beforetext{Par somme~:}
		S\ind{cr}         & = NS_{\mathrm{cr},i}
		\\\Lra
		S\ind{cr}         & = \frac{nR}{2N} \left( \frac{mg}{P_0S} \right)^2
		\\\Lra
		S\ind{cr}         & \opto{}{N\to\infty} 0
	\end{align*}
	Ainsi, lorsqu'elle est réalisée \textbf{suffisamment lentement}, la
	transformation tend vers une \textbf{transformation réversible}.
	\begin{tcn}(impo){Attention~!}
		Comme le terme d'ordre le plus bas du développement limité s'annule, il est
		nécessaire de poursuivre à l'ordre suivant pour obtenir un résultat correct.
		\smallbreak
		On peut également reprendre la méthode de la question précédente, mais la
		transformation n'est plus monobare, ce qui change donc le calcul du
		transfert thermique.
	\end{tcn}
}%

\end{document}
