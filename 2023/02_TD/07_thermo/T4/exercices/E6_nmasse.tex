\documentclass[../TDT4.tex]{subfiles}%

\begin{document}
\section[s]"2"{Masse posée sur un piston}
\enonce{%
	Considérons une enceinte hermétique, diatherme, fermée par un piston de masse
	négligeable pouvant coulisser sans frottements. Cette enceinte contient un gaz
	supposé parfait. Elle est placée dans l'air, à température $T_0$ et pression
	$P_0$.
}%
\QR{%
	On place une masse $m$ sur le piston. Déterminer les caractéristiques du gaz
	une fois les équilibres thermique et mécanique atteints.
}{%
	\begin{align*}
		\beforetext{Équilibre thermique $\Ra$}
		\Aboxed{T_F & = T_I = T_0}
		\\\beforetext{Équilibre mécanique $\Ra$}
		P_I S = P_0 S
		            & \qet
		P_FS = P_0S + mg
		\\\Lra
		\Aboxed{P_F & = P_0 + \frac{mg}{S}}
		\\\beforetext{Gaz parfait $\Ra$}
		\Aboxed{V_F & = \frac{nRT_0}{P_0 + \frac{mg}{S}}}
	\end{align*}
}%
\QR{%
	Déterminer le transfert thermique échangé $Q$ et l'entropie créée.
}{%
	solu
}%
\QR{%
	On réalise la même expérience, mais en $N$ étapes successives, par exemple en
	ajoutant du sable «~grain à grain~». Déterminer l'entropie créée dans la
	limite $N \to \infty$.
}{%
	solu
}%

\end{document}
