\documentclass[a4paper, 10pt, final, garamond]{book}
\usepackage{cours-preambule}
\usepackage[french]{babel}

\raggedbottom

\makeatletter
\renewcommand{\@chapapp}{Thermodynamique -- chapitre}
\makeatother

\begin{document}
\setcounter{chapter}{2}

\chapter{TD~: Second principe et machines thermiques}

\begin{tdefi}{Données}
  Pour un système fermé, de température $T$, de pression $P$ et de volume $V$
  subissant une transformation entre deux états d'équilibre $(i)$ et $(f)$, la
  variation d'entropie est~:
  \begin{itemize}
    \item pour un gaz parfait,
      \begin{gather*}
        \boxed{\Delta S = C_V \ln \frac{P_f}{P_i} + C_P \ln \frac{V_f}{V_i}}
        \qou
        \boxed{\Delta S = C_V \ln \frac{T_f}{T_i} + nR \ln \frac{V_f}{V_i}}
        \qou
        \boxed{\Delta S = C_P \ln \frac{T_f}{T_i} - nR \ln \frac{P_f}{P_i}}
      \end{gather*}
    \item pour une phase condensée,
      \begin{gather*}
        \boxed{\Delta S = C \ln \frac{T_f}{T_i}}
      \end{gather*}
  \end{itemize}
\end{tdefi}

\section{Méthode des mélanges dans un calorimètre}
Un calorimètre de capacité thermique $C = \SI{150}{J.K^{-1}}$ contient
initialement une masse $m_1 = \SI{200}{g}$ d'eau à $\theta_1 =
\SI{20}{\degreeCelsius}$, en équilibre thermique avec le calorimètre. On plonge dans 
l'eau un bloc de fer de masse $m_2 = \SI{100}{g}$ initialement à la température
$\theta_2 = \SI{80.0}{\degreeCelsius}$.
\begin{enumerate}
  \item Calculer la température d'équilibre $T_f$.
  \item Calculer la variation d'entropie de l'eau, du fer et du calorimètre.
  \item En déduire l'entropie créée au cours de la transformation. Celle-ci
    est-elle réversible~?
\end{enumerate}
\begin{rdefi}{Données}
  $c_{\ce{Fe}} = \SI{452}{J.K^{-1}.kg^{-1}}$ et $c_{\rm eau} =
  \SI{4185}{J.K^{-1}.kg^{-1}}$.
\end{rdefi}

\section{Équilibre d'une enceinte à deux compartiments}
Une enceinte indéformable aux parois calorifugées est séparée en deux
compartiments par une cloison étanche, diatherme et mobile sans frottement. Les
deux compartiments contiennent un même gaz parfait. Dans l'état initial, la
cloison est maintenue au milieu de l'enceinte. Le gaz du compartiment 1 est dans
l'état $(T_0,P_0,V_0)$ et le gaz du compartiment 2 dans l'état $(T_0,2P_0,V_0)$.
On laisse alors la cloison bouger librement jusqu'à ce que le système atteigne un
état d'équilibre.
\begin{enumerate}
  \item Exprimer les quantités de matière $n_1,n_2$ dans chaque compartiment en
    fonction de $n_0 = P_0V_0/RT_0$.
  \item Exprimer la température, le volume et la pression du gaz de chaque
    compartiment dans l'état final, en fonction de $n_0,T_0$ et $V_0$.
  \item Exprimer l'entropie créée en fonction de $n_0$.
\end{enumerate}

\section{Corps en contact avec $n$ thermostats quasi-statiques}
Un métal de capacité thermique $C_p$ passe de la température initiale $T_0$ à la
température finale $T_f = T_n$ par contacts successifs avec une suite $n$
thermostats de températures $T_i$ étagées entre $T_0$ et $T_f$. On prendra le
rapport $T_{i+1}/T_i = \alpha$ constant.
\begin{enumerate}
  \item Exprimer pour chaque étape la variation d'entropie du corps $\Delta S$
    en fonction de $m, c$ et $\alpha$.
  \item Calculer le transfert thermique reçu par le métal sur une étape en
    fonction de $T_{i+1}$ et $T_i$, puis l'entropie échangée $S_e$ en fonction
    de $m, c$ et $\alpha$.
  \item Calculer la variation d'entropie du corps $\Delta S$, l'entropie
    échangée $S_e$ ainsi que l'entropie créée $S_c$ sur l'ensemble en fonction
    de $C_p, \alpha$ et $n$.
  \item Étudier $S_c$ pour $n \ra \infty$. On exprimera $\alpha$ en fonction de
    $T_f, T_i$ et $n$, et on utilisera le développement limité $\exp(x) =
    1+x+x^2/2$ pour $x$ petit devant 1.
\end{enumerate}

\section{Questions de cours~: efficacités de \textsc{Carnot}}
\begin{enumerate}
  \item Énoncer (sans démonstration) le premier principe pour une machine
    ditherme.
  \item Énoncer (sans démonstration) l'inégalité de \textsc{Clausius} pour une
    machine ditherme.
  \item Indiquer sans démonstration le sens des transferts thermiques et du
    travail pour les machines dithermes suivantes~:
    \begin{itemize}
      \item moteur~;
      \item réfrigérateur~;
      \item pompe à chaleur.
    \end{itemize}
  \item Pour un moteur, établir l'expression du rendement de \textsc{Carnot}.
  \item De même pour une pompe à chaleur.
\end{enumerate}

\section{Moteur à explosion -- cycle de Beau de \textsc{Rochas}}
Dans un moteur à explosion, $n$ moles de gaz parfait subit le cycle de Beau de
\textsc{Rochas}, composé de deux adiabatiques et de deux isochores~:
\begin{itemize}
  \item compression adiabatique de l'état $(P_1,V_1,T_1)$ à l'état
    $(P_2,V_2,T_2)$~;
  \item échauffement isochore de l'état $(P_2,V_2,T_2)$ à l'état
    $(P_3,V_3,T_3)$~;
  \item détente adiabatique de l'état $(P_3,V_3,T_3)$ à l'état $(P_4,V_4,T_4)$~;
  \item refroidissement isochore qui ramène le fluide à l'état initial.
\end{itemize}
Les transformations sont supposées quasi-statiques.
\begin{enumerate}
  \item Représenter le cycle dans un diagramme de \textsc{Clapeyron} $(P,V)$.
  \item Exprimer les travaux et transferts thermiques au cours des différentes
    étapes en fonction $n,R,\gamma$ et des températures. En déduire le
    rendement théorique $r$ de ce cycle en fonction des températures
    $T_1,T_2,T_3$ et $T_4$.
  \item En déduire l'expression de $r$ en fonction du rapport volumétrique $x =
    \frac{V_1}{V_2}$ et du coefficient adiabatique $\gamma = \frac{C_P}{C_V}$ du
    fluide.
  \item Le piston du cylindre où évolue l'air ($\gamma = \num{1.4}$) a une
    course $\ell = \SI{10}{cm}$, une section $S = \SI{50}{cm^2}$ et emprisonne
    un volume d'air de \SI{100}{cm^3} en fin de compression. Calculer~:
    \begin{enumerate}
      \item le rendement théorique du cycle~;
      \item le travail fourni au cours d'un cycle, si l'air est admis à une
        pression de \SI{1}{bar} et à \SI{300}{K} et si la température maximale
        est de \SI{900}{K}.
    \end{enumerate}
\end{enumerate}

\section{Étude d'un moteur de \textsc{Stirling}}
Un cycle de \textsc{Stirling} est formé de deux isothermes ($T_F < T_C$) et de
deux isochores alternées. Le cycle est décrit de façon quasistatique dans le
sens moteur par $n$ moles de gaz parfait, caractérisé par un coefficient
adiabatique $\gamma$ supposé constant, et commence par une compression isotherme.
\begin{enumerate}
  \item Représenter ce cycle dans un diagramme de \textsc{Clapeyron}.
  \item En fonction des températures $T_F$ et $T_C$, du taux de compression $a =
    \frac{V_1}{V_2}$ et de $n$, $R$ et $\gamma$, établir les expressions~:
    \begin{enumerate}
      \item de la quantité de chaleur reçue par le système au cours d'un cycle
        (notée $Q_C$ et égale à $Q_{23}+Q_{34}$)~;
      \item de la quantité de chaleur cédée par le système au cours d'un cycle
        (notée $Q_F$ et égale à $Q_{41}+Q_{12}$)~;
      \item du rendement thermodynamique de ce cycle.
    \end{enumerate}
  \item On admet que la chaleur fournie au fluide lors du chauffage isochore est
    récupérée par un régénérateur lors du refroidissement isochore. Que devient
    le rendement~? Comparer ce rendement à celui de \textsc{Carnot}. Que peut-on
    en déduire sur l'entropie créée au cours du cycle~?
\end{enumerate}

\end{document}
