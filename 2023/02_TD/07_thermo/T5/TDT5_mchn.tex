\documentclass[a4paper, 10pt, final, garamond]{book}
\usepackage{cours-preambule}
\usepackage[french]{babel}

\raggedbottom

\makeatletter
\renewcommand{\@chapapp}{Thermodynamique -- chapitre}
\makeatother

\begin{document}
\setcounter{chapter}{2}

\chapter{TD~: Second principe et machines thermiques}

\section{Questions de cours~: efficacités de \textsc{Carnot}}
\begin{enumerate}
  \item Énoncer (sans démonstration) le premier principe pour une machine
    ditherme.
  \item Énoncer (sans démonstration) l'inégalité de \textsc{Clausius} pour une
    machine ditherme.
  \item Indiquer sans démonstration le sens des transferts thermiques et du
    travail pour les machines dithermes suivantes~:
    \begin{itemize}
      \item moteur~;
      \item réfrigérateur~;
      \item pompe à chaleur.
    \end{itemize}
  \item Pour un moteur, établir l'expression du rendement de \textsc{Carnot}.
  \item De même pour une pompe à chaleur.
\end{enumerate}

\section{Moteur à explosion -- cycle de Beau de \textsc{Rochas}}
Dans un moteur à explosion, $n$ moles de gaz parfait subit le cycle de Beau de
\textsc{Rochas}, composé de deux adiabatiques et de deux isochores~:
\begin{itemize}
  \item compression adiabatique de l'état $(P_1,V_1,T_1)$ à l'état
    $(P_2,V_2,T_2)$~;
  \item échauffement isochore de l'état $(P_2,V_2,T_2)$ à l'état
    $(P_3,V_3,T_3)$~;
  \item détente adiabatique de l'état $(P_3,V_3,T_3)$ à l'état $(P_4,V_4,T_4)$~;
  \item refroidissement isochore qui ramène le fluide à l'état initial.
\end{itemize}
Les transformations sont supposées quasi-statiques.
\begin{enumerate}
  \item Représenter le cycle dans un diagramme de \textsc{Clapeyron} $(P,V)$.
  \item Exprimer les travaux et transferts thermiques au cours des différentes
    étapes en fonction $n,R,\gamma$ et des températures. En déduire le
    rendement théorique $r$ de ce cycle en fonction des températures
    $T_1,T_2,T_3$ et $T_4$.
  \item En déduire l'expression de $r$ en fonction du rapport volumétrique $x =
    \frac{V_1}{V_2}$ et du coefficient adiabatique $\gamma = \frac{C_P}{C_V}$ du
    fluide.
  \item Le piston du cylindre où évolue l'air ($\gamma = \num{1.4}$) a une
    course $\ell = \SI{10}{cm}$, une section $S = \SI{50}{cm^2}$ et emprisonne
    un volume d'air de \SI{100}{cm^3} en fin de compression. Calculer~:
    \begin{enumerate}
      \item le rendement théorique du cycle~;
      \item le travail fourni au cours d'un cycle, si l'air est admis à une
        pression de \SI{1}{bar} et à \SI{300}{K} et si la température maximale
        est de \SI{900}{K}.
    \end{enumerate}
\end{enumerate}

\section{Étude d'un moteur de \textsc{Stirling}}
Un cycle de \textsc{Stirling} est formé de deux isothermes ($T_F < T_C$) et de
deux isochores alternées. Le cycle est décrit de façon quasistatique dans le
sens moteur par $n$ moles de gaz parfait, caractérisé par un coefficient
adiabatique $\gamma$ supposé constant, et commence par une compression isotherme.
\begin{enumerate}
  \item Représenter ce cycle dans un diagramme de \textsc{Clapeyron}.
  \item En fonction des températures $T_F$ et $T_C$, du taux de compression $a =
    \frac{V_1}{V_2}$ et de $n$, $R$ et $\gamma$, établir les expressions~:
    \begin{enumerate}
      \item de la quantité de chaleur reçue par le système au cours d'un cycle
        (notée $Q_C$ et égale à $Q_{23}+Q_{34}$)~;
      \item de la quantité de chaleur cédée par le système au cours d'un cycle
        (notée $Q_F$ et égale à $Q_{41}+Q_{12}$)~;
      \item du rendement thermodynamique de ce cycle.
    \end{enumerate}
  \item On admet que la chaleur fournie au fluide lors du chauffage isochore est
    récupérée par un régénérateur lors du refroidissement isochore. Que devient
    le rendement~? Comparer ce rendement à celui de \textsc{Carnot}. Que peut-on
    en déduire sur l'entropie créée au cours du cycle~?
\end{enumerate}

\end{document}
