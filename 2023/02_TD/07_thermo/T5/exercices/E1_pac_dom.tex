\documentclass[../TDT5.tex]{subfiles}%

\begin{document}
\section{Pompe à chaleur domestique}
\enonce{%
	On veut maintenir la température d'une maison à $T_1 =
		\SI{20}{\degreeCelsius}$ alors que la température extérieure est égale à $T_2
		= \SI{5}{\degreeCelsius}$, en utilisant une pompe à chaleur. L'isolation
	thermique de la maison est telle qu'il faut lui fournir un transfert thermique
	égal à \SI{200}{kJ} par heure pour cet effet.
}%
\QR{%
	Rappeler le schéma de principe d'une pompe à chaleur ditherme et le sens réel
	des échanges d'énergie du fluide caloporteur.
}{%
	solu
}%
\QR{%
	Quel doit être le cycle thermodynamique suivi par le fluide pour que
	l'efficacité de la pompe à chaleur soit maximale~?
}{%
	solu
}%
\QR{%
	Définir et calculer l'efficacité théorique maximale de la pompe dans ces
	conditions. Montrer qu'elle ne dépend que de la différence de température
	entre l'intérieur et l'extérieur. Quel est le sens physique de l'efficacité~?
}{%
	solu
}%
\QR{%
	En déduire la puissance électrique minimale consommée par la pompe à chaleur.
}{%
	solu
}%
\QR{%
	En supposant la température intérieure imposée, pour quelle température
	extérieure l'efficacité est-elle maximale~?
}{%
	solu
}%

\end{document}
