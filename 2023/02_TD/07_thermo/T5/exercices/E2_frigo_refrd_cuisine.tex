\documentclass[../TDT5.tex]{subfiles}%

\begin{document}
\section{Rafraîchir sa cuisine en ouvrant son frigo}
\enonce{%
	Un refrigérateur est une marchine thermique à écoulement, dans laquelle un
	fluide subit une série de transformations thermodynamiques cyclique. À chaque
	cycle, le fluide extrait de l'intérieur du frigo un transfert thermique
	$\abs{Q\ind{int}}$, cède un transfert thermique $\abs{Q\ind{ext}}$ à la pièce
	dans laquelle se trouve le frigo et reçoit un travail $\abs{W}$ fourni par un
	moteur électrique.
	\smallbreak
	On fait l'hypothèse que l'intérieur du réfrigérateur et l'air ambiant
	constituent deux thermostats aux températures respectives $T\ind{int} =
		\SI{268}{K}$ et $T\ind{ext} = \SI{293}{K}$, et qu'en dehors des échanges avec
	ces thermostats les transformations sont adiabatiques.
}%
\QR{%
	Quel est le signe des énergies échangées~?
}{%
	solu
}%
\QR{%
	Lorsqu'il fait très chaud en été, est-ce une bonne idée d'ouvrir la porte de
	son frigo pour refroidir sa cuisine~? Pourquoi~?
}{%
	solu
}%
\QR{%
	Quelle est la différence avec un climatiseur~?
}{%
	solu
}%


\end{document}
