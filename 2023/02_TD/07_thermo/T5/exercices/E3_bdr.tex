\documentclass[../TDT5.tex]{subfiles}%

\begin{document}
\section{Moteur à explosion -- cycle de Beau de \textsc{Rochas}}
\enonce{%
	Dans un moteur à explosion, $n$ moles de gaz parfait subit le cycle de
	\textsc{Beau de Rochas}, composé de deux adiabatiques et de deux isochores~:
	\begin{itemize}
		\item compression adiabatique de l'état $(P_1,V_1,T_1)$ à l'état
		      $(P_2,V_2,T_2)$~;
		\item échauffement isochore de l'état $(P_2,V_2,T_2)$ à l'état
		      $(P_3,V_3,T_3)$~;
		\item détente adiabatique de l'état $(P_3,V_3,T_3)$ à l'état $(P_4,V_4,T_4)$~;
		\item refroidissement isochore qui ramène le fluide à l'état initial.
	\end{itemize}
	Les transformations sont supposées quasi-statiques.
}%
\QR{%
	Représenter le cycle dans un diagramme de \textsc{Watt} $(P,V)$.
}{%
	solu
}%

\QR{%
	Exprimer les travaux et transferts thermiques au cours des différentes
	étapes en fonction $n,R,\gamma$ et des températures. En déduire le
	rendement théorique $r$ de ce cycle en fonction des températures
	$T_1,T_2,T_3$ et $T_4$.
}{%
	solu
}%

\QR{%
	En déduire l'expression de $r$ en fonction du rapport volumétrique $x =
		\frac{V_1}{V_2}$ et du coefficient adiabatique $\gamma =
		\frac{C_P}{C_V}$ du fluide.
}{%
	solu
}%

\begin{blocQR}
	\item \enonce{%
		Le piston du cylindre où évolue l'air ($\gamma = \num{1.4}$) a une
		course $\ell = \SI{10}{cm}$, une section $S = \SI{50}{cm^2}$ et emprisonne
		un volume d'air de \SI{100}{cm^3} en fin de compression. Calculer~:
	}%
	\QR{%
		le rendement théorique du cycle~;
	}{%
		solu
	}%

	\QR{%
		le travail fourni au cours d'un cycle, si l'air est admis à une
		pression de \SI{1}{bar} et à \SI{300}{K} et si la température
		maximale est de \SI{900}{K}.
	}{%
		solu
	}%
\end{blocQR}
\end{document}
