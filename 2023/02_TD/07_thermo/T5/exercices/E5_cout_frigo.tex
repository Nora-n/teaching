\documentclass[../TDT5.tex]{subfiles}%

\begin{document}
\section{Coût énergétique d'un goûter}
\enonce{%
	Pour préparer le goûter de fin d'année avec vos professeures, vous achetez six
	bouteilles de \SI{1}{L} de différents jus que vous rangez dans votre
	réfrigérateur. Une heure plus tard, elles sont à la température du frigo.
	\begin{tcn}(defi)<lftt>{Données}
		\begin{itemize}
			\item L'efficacité thermodynamique du refrigérateur vaut $70\%$ de
			      l'efficacité de \textsc{Carnot}~;
			\item L'isolation imparfaite du refrigérateur se traduit par des fuites
			      thermiques de puissance $\SI{10}{W}$~;
			\item Tarifs électricité~: $\SI{1}{kWh}$ coûte $\SI{0.25}{\text{\euro}}$.
		\end{itemize}
	\end{tcn}
}%
\QR{%
	Combien vous coûte ce refroidissement~?
}{%
	solu
}%

\end{document}
