\documentclass[../main/main.tex]{subfiles}

\makeatletter
\renewcommand{\@chapapp}{Travaux pratiques -- TP}
\makeatother

% \toggletrue{student}
% \HideSolutionstrue

\begin{document}
\setcounter{chapter}{-1}

\chapter*{Utiliser la calculatrice pour analyser des donn\'ees}

\section{Objectifs}

\begin{itemize}
	\item Utiliser les fonctions usuelles de sa calculatrice~;
	\item Comprendre l'intérêt et savoir mettre en place une régression linéaire
	      pour vérifier un modèle.
\end{itemize}

\section{Fonctions de base}

\subsection{Calcul num\'erique}

\begin{enumerate}

	\item Calculer les fractions suivantes à l'aide de la calculatrice (attention
	      à vos parenthèses)~:
	      \begin{align*}
		      \frac{7}{3\times 5}+4
		       & \qet
		      \frac{2}{9\times8 + 5} \\
		      \psw{ \frac{67}{15} = 4.47 }
		       & \psw{ \qet }
		      \psw{ \frac{2}{77} = \num{2.6e-2} }
	      \end{align*}
	\item Le rayon de Bohr $a_0$ (caractéristique de la taille d'un atome) est
	      donné par la formule~:
	      \begin{gather*}
		      a_0 = \frac{4\pi \ep_0 \hbar^2}{m_ee^2}
		      \qav
		      \left\{
		      \begin{array}{rcl}
			      e                   & = & \SI{1.60e19}{C}
			      \\
			      \frac{1}{4\pi\ep_0} & = & \SI{9.00e9}{USI} \psw{~
			      = \SI{9.00e9}{N.m^{2}.C^{-2}}
			      }
			      \\
			      \hbar               & = & \SI{5e-34}{J.s}
			      \\
			      m_e                 & = & \SI{9.11e-31}{kg}
		      \end{array}
		      \right.\\
		      \mathrm{A.N.~:}\enskip
		      \xul{a_0 = \psw{\SI{5.25e-11}{m}}
		      }
	      \end{gather*}
	      Déterminer l'unité de $1/(4\pi\ep_0)$ en newtons, mètres et
	      coulombs, ainsi que la valeur de $a_0$.
	      \psw{
		      \begin{gather*}
			      \left[\frac{1}{4\pi\ep_0}\right] =
			      \left[\frac{\hbar^{2}}{a_0m_{e}e^{2}}\right] =
			      \frac{\si{J^{2}.s^{2}}}{\si{m.kg.C^{2}}} =
			      \frac{\si{N.m^{2}.s^{2}}}{\si{m.kg.C^{2}}} =
			      \frac{\si{N^{2}.m.s^{2}.m.s^{-2}}}{\si{C^{2}.N}}
			      \\\Lra
			      \boxed{\left[\frac{1}{4\pi\ep_0}\right] = \si{N.m^{2}.C^{-2}}}
		      \end{gather*}
	      }
	      \vspace{-20pt}
\end{enumerate}

\subsection{Utilisation des angles}

Lorsque l'on travaille avec les fonctions trigonométriques (cos, sin, tan...),
il faut être très vigilant-e à l'unité des angles utilisée par votre
calculatrice.

\xul{Remarque}~: Dans la suite, on distinguera les degrés par \si{\degree}
et les radians par \si{rad}, mais ça ne sera pas à écrire dans la calculatrice~!

\begin{tcbraster}[raster columns=3, raster equal height=rows]
	\begin{tcb}(expe)<itc>{Casio}
		Dans le mode \texttt{RUN}, on choisit entre degrés et radians en allant dans
		le mode \texttt{SET UP}.
	\end{tcb}
	\begin{tcb}*(expe)<itc>{TI}
		Dans le menu \texttt{MODE}, on choisit entre degrés et radians en allant sur
		la troisième ligne.
	\end{tcb}
	\begin{tcb}*(expe)<itc>{Numworks}
		Dans le menu \texttt{Paramètres}, on choisit entre degrés et radians en
		choisissant l'unité de l'angle.
	\end{tcb}
\end{tcbraster}

\begin{enumerate}
	\item Mettez-vous en radians. Vérifier alors que
	      \[
		      \cos(\SI{\pi}{rad}) = -1
		      \qet
		      \cos(\ang{180}) \approx -\num{0.5985}
	      \]
	\item Mettez-vous en degrés. Vérifier alors que
	      \[
		      \cos(\SI{\pi}{rad}) \approx \num{0.9985}
		      \qet
		      \cos(\ang{180}) =-1
	      \]
	\item Faire les applications numériques suivantes~:
	      \begin{align*}
		      \tan(\SI{2}{rad})
		       & \qet
		      \cos^2\left(\frac{\pi}{3}\,\si{rad}\right) -\sin(\ang{10;;})
		      \\
		      \psw{
			      \tan(\SI{2}{rad}) \approx \num{-2.19}
		      }
		       & \psw{\qet}
		      \psw{
			      \cos^2\left(\frac{\pi}{3}\,\si{rad}\right) -\sin(\ang{10;;}) \approx
			      \num{7.63e-2}
		      }
	      \end{align*}
	\item Calculer $n$ tel que~:
	      \vspace{-20pt}
	      \begin{gather*}
		      n =
		      \frac{\sin\left(\frac{\DS D_m+A}{2}\right)}{\sin\left(\frac{A}{2}\right)}
		      \qav
		      \left\{
		      \begin{array}{rcl}
			      D_m & = & \ang{5.85}
			      \\
			      A   & = & \pi/3 \si{rad}
		      \end{array}
		      \right.\\
		      \mathrm{A.N.~:}\enskip
		      \xul{
			      n = \psw{ \num{1.09} }
		      }
	      \end{gather*}
\end{enumerate}

\subsection{R\'esolution des équations d'ordre 2}

\begin{tcb}(expe)<itc>{Casio}
	Dans le mode \texttt{EQUA}, on sélectionne le type de l'équation à résoudre.
	Il y a~:
	\begin{itemize}
		\item \texttt{SIML} (bouton \texttt{F1}) pour résoudre un système
		      d'équations à plusieurs inconnues~;
		\item \texttt{POLY} (bouton \texttt{F2}) pour résoudre une équation
		      polynômiale~;
		\item \texttt{SOLV} (bouton \texttt{F3}) pour résoudre une équation plus
		      complexe.
	\end{itemize}
	Choisir ici le mode \texttt{POLY}. On peut ensuite choisir le degré de
	l'équation (2 ou 3). Dans notre cas, choisir 2. On est alors invité-e à
	rentrer les coefficients $a$, $b$ et $c$ de l'équation $ax^2+bx+c=0$. Une fois
	cela fait, presser \texttt{SOLV} (bouton \texttt{F1}). On obtient alors les
	deux solutions de l'équation.
\end{tcb}

\begin{tcb}(expe)<itc>{TI}
	Aller dans \texttt{Apps}. Choisir \texttt{PlySmlt2} (bouton 4) puis
	\texttt{Poly Root Finder} (bouton 1). Choisir ensuite l'ordre 2 et presser
	\texttt{ENTER} et \texttt{NEXT}. L'équation s'affiche alors sous la forme
	$a2*x^2+a1*x+a0=0$. Renseigner alors les coefficients $a2$, $a1$ et $a0$.
	Presser \texttt{SOLVE} pour obtenir les deux solutions de l'équation.
\end{tcb}

\begin{tcb}(expe)<itc>{Numworks}
	Aller dans le menu \texttt{Équations}. Ajouter une équation, et choisir le
	modèle d'équation voulu. Compléter les coefficients. Choisir \texttt{résoudre
		l'équation}.
\end{tcb}

Donner les solutions des équations suivantes~:
\begin{gather*}
	\begin{array}{ccccc}
		2x^2+3 = 0
		\qquad & ; & \qquad
		3x^2 = 2x+1
		\qquad & ; & \qquad
		x^2+x+1 = 0
		\\
		\psw{
			x = \pm \num{1.22}\Ir
		}
		\qquad & ; & \qquad
		\psw{
			x_1 = \num{1} \qet x_2 = \num{0.33}
		}
		\qquad & ; & \qquad
		\psw{
			x_\pm = \num{-0.5} \pm \num{0.87}\Ir
		}
	\end{array}
\end{gather*}

Pour résoudre une équation aux racines complexes~:

\begin{tcbraster}[raster columns=3, raster equal height=rows]
	\begin{tcb}(expe)<itc>{Casio}
		Dans le menu polynômial \texttt{POLY}, degré $? \rightarrow
			2$ puis $\texttt{shift} \rightarrow \texttt{setup} \rightarrow
			\texttt{complex mode a+ib}$.
	\end{tcb}
	\begin{tcb}*(expe)<itc>{TI}
		Dans le menu où l'on choisit l'ordre, sélectionner $\texttt{a+ib}$.
	\end{tcb}
	\begin{tcb}*(expe)<itc>{Numworks}
		Dans les paramètres, mettre la forme complexe en algébrique.
	\end{tcb}
\end{tcbraster}

\subsection{Stockage et utilisation de valeurs}
Lorsqu'une expression mathématique est lourde à taper, il peut devenir
indispensable d'utiliser les caractères alphabétiques pour stocker des valeurs.
Ainsi, la lecture des valeurs est claire, comme pour les A.N.\ écrites, la
modification d'une d'entre elle également, et il en est de même pour
l'expression en elle-même.

\begin{tcb}[sidebyside](expe)<itc>{Casio}
	Depuis le menu \texttt{RUN}, entrer une valeur suivie d'une flèche $\ra$ et
	d'une lettre, accessible par \texttt{ALPHA} puis une touche.
	\tcblower
	\texttt{\num{5.85}x$\pi$/180$\ra$D}
	\smallbreak
	\texttt{$\pi/3\ra$A}
	\smallbreak
	\texttt{sin((D+A)/2)/sin(A/2)$\ra$n}
\end{tcb}

\section{Régression linéaire}

\underline{Note}~: Pour cette partie, s'aider de la fiche pratique «~régression
linéaire~».

\subsection{Exemple 1~: la loi d'\textsc{Ohm}}

La tension est l'intensité est mesurée au travers d'une résistance de $R=
	\SI{1}{k\Omega}$ d'après le constructeur.

\begin{center}
	\begin{tabular}{ l  c  c  c  c  c  c }
		\toprule
		$I$ (en A)  & \num{0,010} & \num{0,020} & \num{0,030} & \num{0,040} &
		\num{0,050} & \num{0,060}                                             \\

		$U$ (en V)  & \num{10,1}  & \num{20,0}  & \num{29,8}  & \num{40,2}  &
		\num{50,0}  & \num{60,1}                                              \\

		\bottomrule
	\end{tabular}
\end{center}

\begin{enumerate}
	\item Tracer le graphe de la tension en fonction de l'intensité sur votre
	      calculatrice.
	\item Réaliser la régression linéaire. Relever les valeurs des coefficients de
	      régression $a$ et $b$ ainsi que le coefficient de corrélation linéaire
	      $r$ et le coefficient de détermination $r^2$.
	      \psw{
		      \begin{align*}
			      a = \SI{1001}{\Omega} & \qet b = \SI{-6.66e-3}{V}
			      \\
			      r = \num{0.99997}     & \qet r^{2} = \num{0.99994}
		      \end{align*}
	      }
	\item Les données suivent-elles bien la loi d'\textsc{Ohm}~? Vérifier la
	      valeur de la résistance.
	      \bigbreak
	      \psw{
		      Les données relevées suivent bien la loi d'\textsc{Ohm}, étant donné
		      l'aspect de la droite de régression.
	      }
	\item Conclure quant à la valeur constructeur. Quel est l'écart relatif entre
	      la valeur constructeur et la valeur expérimentale de $R$~? Commenter.
	      \bigbreak
	      \psw{
		      La valeur constructeur est précise. On trouve un écart relatif de~:
		      \[
			      \ep_r =
			      \frac{
				      \abs{R_{\rm const} - R_{\rm reg}}
			      }{
				      \abs{R_{\rm const}}
			      } = \xul{\num{0.1}\%}
		      \]
		      Ce qui est tout à fait satisfaisant.
	      }
\end{enumerate}

\subsection{Exemple 2~: Cinétique chimique}

\subsubsection{Position du problème}

Les relations que l'on étudie en science ne sont pas toujours linéaires.
Pourtant, il est tout de même possible d'exploiter la méthode de
régression linéaire pour s'assurer de la validité du modèle et
déterminer des coefficients numériques inconnus. À titre d'exemple, on va
étudier l'évolution de la concentration $c(t)$ d'une espèce chimique en
solution lors d'une réaction chimique. La forme de cette évolution peut
être de deux types~:

\begin{tcolorbox}[blankest]
	\begin{isd}
		\tcbsubtitle{\fatbox{Ordre 1}}
		\[
			c(t) = c_0 \exr^{-kt}
		\]
		\tcblower
		\tcbsubtitle{\fatbox{Ordre 2}}
		\[
			\frac{1}{c(t)} = \frac{1}{c_0} + kt
		\]
	\end{isd}
\end{tcolorbox}

% \begin{enumerate}
% 	\item Avec une cinétique d'ordre 1, la concentration évolue selon
%
% 	      \[c(t) = c_0 e^{-kt}\]
%
% 	\item Avec une cinétique d'ordre 2, la concentration évolue selon
%
% 	      \[\frac{1}{c(t)} = \frac{1}{c_0} + kt\]
% \end{enumerate}

Ces deux modèles sont non linéaires. Afin de vérifier si les données
expérimentales valident (ou non) l'un de ces modèles à l'aide d'une régression
linéaire, il convient de «~linéariser~» (rendre linéaire) les données.

\begin{enumerate}
	\item Dans le cas d'une cinétique d'ordre 1, on remarque que
	      \[
		      \ln(c(t)) = \ln(c_0)-kt
	      \]
	      Ainsi, si ce modèle est vérifié alors le tracé de $\ln(c(t))$ en
	      fonction de $t$ doit aboutir à une droite de coefficient directeur
	      $a=-k$ et d'ordonnée à l'origine $b = \ln(c_0)$.
	\item Dans le cas d'une cinétique d'ordre 2, on remarque que, si ce modèle
	      est vérifié, alors le tracé de $1/c(t)$ en fonction de $t$ doit aboutir
	      à une droite de coefficient directeur $a=k$ et d'ordonnée à l'origine $b
		      = 1/c_0$.
\end{enumerate}

\begin{tcb}(rema){Remarque}
	Il est possible d'effectuer directement des opérations sur les listes avec les
	calculatrices (voir la fiche pratique «~régression linéaire~») pour avoir de
	l'aide.
\end{tcb}

\subsubsection{Application}
La concentration $c(t)$ d'une espèce chimique est mesurée dans la solution au
cours du temps. On obtient les données suivantes~:

\begin{center}
	\begin{tabular}{ l  c  c  c  c  c  c }
		\toprule
		$t$ (en s)                        & 20  & 40  & 60  & 80  & 100 & 120 \\
		$c$ (en $\si{\micro mol.L^{-1}}$) & 278 & 192 & 147 & 119 & 100 & 86  \\
		\bottomrule
	\end{tabular}
\end{center}

\begin{enumerate}
	\item Réaliser les régressions linéaires et donner l'équation de la
	      droite (coefficients $a$ et $b$) ainsi que la valeur des coefficients de
	      corrélation dans les cas suivants~:
	      \begin{enumerate}
		      \item $c$ en fonction de $t$ ;
		      \item $\ln(c)$ en fonction de $t$ ;
		      \item $1/c$ en fonction de $t$.
	      \end{enumerate}
	      Avec quel modèle les résultats expérimentaux s'accordent-ils le mieux ?
	      Conclure.
\end{enumerate}

\vfill

Résultats attendus~:

\begin{tcb}*[aide](expe)<itc>{Aide}
	\begin{enumerate}
		\item $c = f(t)$~: $c = -\num{1,8}.10^{-6} t + \num{0,0003}$; $r^2 =
			      \num{0,89}$ et $r = - \num{0,94}$.
		\item $\ln(c) = f(t)$~: $\ln(c) = -\num{0,0115} t - \num{8,0597}$; $r^2 =
			      \num{0,97}$ et $r = - \num{0,98}$.
		\item $1/c = f(t)$~: $1/c = \num{80,185} t + \num{1993,6}$; $r^2 =
			      \num{0,999991}$ et $r = - \num{0,999995}$.
	\end{enumerate}
\end{tcb}

\end{document}
