\documentclass[../main/main.tex]{subfiles}
\graphicspath{{./figures/}}

\makeatletter
\renewcommand{\@chapapp}{Travaux pratiques -- TP}
\makeatother

\toggletrue{student}
\HideSolutionstrue

\begin{document}
\setcounter{chapter}{1}

\chapter{Formation et observation d'images \`a distance finie~: mesures de
  distances}

\ifstudent{
	\begin{prgm}
		\begin{tcb}*(ror)"how"{Savoir-faire}
			\begin{itemize}[label=$\diamond$, leftmargin=10pt]
				\item Mesure de longueurs sur un banc d'optique.
				\item Mettre en œuvre une mesure de longueur par déplacement du viseur
				      entre deux positions.
				\item Utiliser un viseur à frontale fixe, une lunette auto-collimatrice.
				\item Utiliser des vis micrométriques et un réticule pour
				      tirer parti de la précision affichée de l'appareil utilisé.
			\end{itemize}
		\end{tcb}
	\end{prgm}
}

\section{Objectifs}

\begin{itemize}[label=$\diamond$, leftmargin=10pt]
	\item Réaliser des alignements sur un banc d'optique.
	\item Utiliser un viseur à frontale fixe pour :
	      \begin{itemize}[label=$\triangleright$]
		      \item Réaliser des pointés transversaux permettant de mesurer des
		            dimensions d'objet (orthogonaux à l'axe optique).
		      \item Réaliser des pointés longitudinaux permettant de mesurer des
		            distances entre objets suivant l'axe optique.
		      \item Vérifier la formule de \textsc{Descartes} des lentilles.
		      \item Vérifier la méthode de \textsc{Bessel}.
		      \item Utiliser et apprendre à lire une vis micrométrique.
	      \end{itemize}
\end{itemize}

\section{S'approprier}

\subsection{Principe de fonctionnement d'un viseur à frontale fixe}

\begin{wrapfigure}[5]{r}{0.4\textwidth}
	\vspace*{-40pt}
	\begin{center}
		\includegraphics[width=0.4\textwidth]{dispositif}
	\end{center}
\end{wrapfigure}

Un viseur à frontale fixe est constitué :

\begin{itemize}
	\item d'un objectif
	\item d'un réticule
	\item d'un oculaire
	\item d'une vis micrométrique permettant de translater le réticule selon un
	      axe orthogonal à l'axe optique.
\end{itemize}

L'objectif forme l'image A'B' de AB dans le plan focal objet de l'oculaire,
tel que l'image finale est à l'infini, permettant pour l'œil une observation
sans accommodation. Le réticule est dans ce même plan et est donc visible
simultanément à l'objet AB. Globalement, le système peut être représenté par
trois lentilles minces convergentes (dont le cristallin de l'œil) selon

\[
	\rm AB
	\opto{\Lc_1}{\text{objectif}}
	A_1B_1
	\opto{\Lc_2}{\text{oculaire}}
	\underbrace{\rm A'B'}_{\infty}
	\opto{\Lc_3}{\text{cristallin}}
	\text{image sur la rétine}
\]

\begin{center}
	\includegraphics[width=.7\linewidth]{microscope}
\end{center}

\begin{tcb}(impl){À comprendre}
	Un réticule est placé dans le plan focal objet de l'oculaire. Ainsi, il
	convient de translater le viseur sur le banc optique par rapport à l'objet
	pour voir simultanément l'objet et le réticule net. Nous noterons $D$ la
	distance séparant le centre optique de l'objectif de l'objet visé AB.
	Cette distance est fixe, ce qui explique la dénomination de viseur à
	frontale fixe.
\end{tcb}

\subsection{Principe de lecture d'une vis micrométrique}

La vis micrométrique permet de déplacer un double fil vertical dans le plan du
réticule. La vis a un pas (translation réalisée pour un tour complet) de
\SI{0,5}{mm} qui correspond à 50 graduations du tambour. On peut ainsi en
déplaçant le fil vertical du réticule, mesurer la dimension de l'image A'B'
réalisée de l'objet AB au \SI{1/100}{mm}.

\begin{figure}[h]
	\centering
	\includegraphics[width=0.7\linewidth]{vis_micro}
	\captionsetup{justification=centering}
	\caption{Principe de lecture sur une vis micrométrique. Un tour complet
		correspond à $\SI{0,5}{mm}$.}
	\label{fig:vis_micro}
\end{figure}

La lecture de gauche donne : $33+\num{0,5}+\num{0,245} = \SI{33,745}{mm}$.\\
La lecture de droite donne : $33+\num{0,250} = \SI{33,250}{mm}$.

\begin{enumerate}[label=\clenumi]
	\item Quelle est l'incertitude-type sur une telle mesure ? (s'aider de la
	      fiche pratique sur la théorie de la mesure).
\end{enumerate}

\subsection{Principe de réalisation d'un pointé longitudinal avec un viseur}

On cherche à déterminer la distance algébrique longitudinale $d = \obar{\rm
		A_1A_2}$ séparant deux objets sur le banc optique. L'objet $\rm A_1B_1$ est
constitué par exemple par la lettre $F$ et l'objet $\rm A_2B_2$ par un
quadrillage dessiné sur feuille transparente. On séparera les deux objets d'une
quinzaine de centimètres. On éclaire l'ensemble avec la lanterne (en $\SI{6}{V}$
alternatif) devant laquelle on interpose un écran dépoli pour limiter le flux
lumineux.
\bigbreak
La première étape consiste à régler le viseur. Pour cela, on translate
l'oculaire en agissant sur l'œilleton de l'oculaire afin de mettre au point le
réticule (c'est-à-dire voir les fils croisés nets). Ce réglage dépend de chacun,
aussi il faut donc en théorie le réaliser à chaque fois que l'observataire
change. En pratique, si votre œil est sans défaut, ou si ces défauts ont été
corrigés, il n'est pas nécessaire de modifier les réglages pour les membres d'un
binôme.
\bigbreak
La mesure se déroule en deux étapes~:
\begin{tcb}(expe)<itc>{Pointé longitudinal}
	\begin{enumerate}
		\item Viser l'objet $\rm A_1B_1$ (viser signifie voir l'image nette de
		      l'objet à travers le viseur). L'abscisse du viseur est alors~:
		      \[
			      x_{v}({\rm A_1B_1}) = x_{\rm A_1}+D
		      \]
		\item Viser l'objet $\rm A_2B_2$. On a alors
		      \[
			      x_{v}({\rm A_2B_2}) = x_{\rm A_2}+D
		      \]
	\end{enumerate}

	On en déduit alors la distance $d = \obar{\rm A_1A_2}$ comme étant :
	\[
		x_{v}({\rm A_2B_2}) - x_{v}({\rm A_1B_1}) =
		(x_{\rm A_2}+D) - (x_{\rm A_1}+D) =
		x_{\rm A_2} - x_{\rm A_1} = d
	\]
\end{tcb}

\begin{figure}[htbp]
	\centering
	\includegraphics[width=.5\linewidth]{principe1}
\end{figure}

\begin{tcb}[width=\linewidth](rema){Remarque} Évidemment, dans la situation
	présente, la méthode semble un peu artificielle, puisqu'il est possible de
	lire directement sur le banc optique la position des objets. Néanmoins, dans
	une situation plus réaliste, une telle méthode peut s'avérer très utile
	puisqu'il devient alors possible de déterminer de loin la distance séparant
	deux objets.
\end{tcb}

\subsection{Mesure du grandissement transversal du viseur}

\begin{wrapfigure}{r}{0.62\textwidth}
	\begin{center}
		\includegraphics[width=0.6\textwidth]{PointeTransversal}
	\end{center}
\end{wrapfigure}

Le but de cette première manipulation est de déterminer le grandissement
transversal de l'objectif du viseur : $\gamma = \ABp/\AB$. Pour
cela, on détermine grâce à la vis micrométrique du viseur, la dimension de
l'image A'B' d'un objet AB de dimension connue. L'objet AB de dimension
connue est un micromètre, c'est-à-dire un axe gradué où 100 traits
représentent $\SI{1}{cm}$. Le micromètre doit être horizontal. Réaliser le
montage ci-contre. Avant toute mesure, il faut régler l'alignement des
instruments d'optique.
\bigbreak
Approcher le viseur près du micromètre, vérifier les alignements des
instruments, puis s'en écarter tout doucement jusqu'à voir net le micromètre
dans le viseur en superposition avec le réticule.

\begin{tcb}(expe)<itc>{Pointé transversal}
	\begin{enumerate}
		\item Grâce à la vis micrométrique, déplacer le fil vertical mobile du
		      réticule sur une graduation centrale du micromètre et noter
		      l'indication de la vis.
		\item Faire précisément 2 tours avec la vis micrométrique~: cela représente
		      un déplacement de $\SI{1}{mm}$ côté image.
		\item Relever, dans le viseur, la nouvelle graduation du micromètre pointée
		      par le fil vertical mobile.
		\item Déduire de ces relevés la valeur absolue du grandissement
		      $\abs{\gamma}$ de l'objectif du viseur.
		\item Sachant que l'objectif du viseur est constitué d'une lentille
		      convergente, en déduire le grandissement algébrique $\gamma$ de
		      l'objectif du viseur.
	\end{enumerate}
\end{tcb}

\subsection{Focométrie~: rappel TP précédent (\textsc{Bessel} et
	\textsc{Silbermann})}

\begin{tcb}(rapp){Rappel}
	La focométrie consiste à déterminer expérimentalement la distance focale
	d'une lentille $(\Lc)$ inconnue.
\end{tcb}

\noindent
\begin{minipage}[t]{.6\linewidth}
	À l'aide d'une lentille mince convergente $(\Lc)$ de distance focale
	image $f'$ inconnue, on veut former l'image d'un objet réel sur un écran
	situé à une distance $D$ de l'objet. En déplaçant la lentille, on trouve
	deux positions O$_1$ et O$_2$ qui donnent une image nette sur l'écran (cf.\
	figure ci-contre) à condition que $D \geq 4f'$. On a vu qu'alors la focale
	image $f'$ pouvait s'écrire~:
	\[
		f' = \dfrac{D^2-d^2}{4D}
	\]
\end{minipage}
\hfill
\begin{minipage}[t]{.38\linewidth}
	~
	\begin{center}
		\includegraphics[width=\linewidth]{lent_conv-condition_bessel}
	\end{center}
\end{minipage}

La méthode de \textsc{Silbermann} est le cas particulier où $x_1 = x_2$.
\begin{enumerate}[label=\clenumi, start=2]
	\item Rappeler le principe de cette méthode.
\end{enumerate}

\section{Réaliser et valider}

Notebook \texttt{Capytale}
disponible\ftn{\url{https://capytale2.ac-paris.fr/web/c/8e2d-1856963}}.

\subsection{Pointé longitudinal}

\begin{enumerate}[label=\sqenumi, start=3]
	\item Suivre le protocole permettant de réaliser un pointé longitudinal et
	      \textbf{valider} la méthode. Comparer pour ce faire la distance obtenue
	      par pointé avec le viseur à la distance réelle entre les deux objets
	      (lue directement sur le banc optique). Déterminer les incertitudes
	      relatives, et en déduire l'écart normalisé.
\end{enumerate}

\subsection{Pointé transversal~: largeur d'un fil de cuivre}

\begin{enumerate}[label=\sqenumi, start=4]
	\item Prendre comme objet un fil de cuivre, le viseur, positionner
	      correctement le fil vertical du réticule et, connaissant le grandissement de
	      l'objectif du viseur, en déduire l'épaisseur de celui-ci.
\end{enumerate}

%\subsection{\'Evaluation de la distance frontale du viseur}
%
%%\'Evaluer la distance focale de la lentille constituant le viseur, en mesurant la distance frontale $D$ (entre ce qu'on vise et le viseur) avec un réglet. Exprimer f' en fonction de ? et D. Faire l'application numérique.

\section{Valider la relation de conjugaison de \textsc{Descartes}}

\subsection{Montage}

\begin{wrapfigure}[4]{r}{0.62\textwidth}
	\vspace*{-40pt}
	\begin{center}
		\includegraphics[width=0.6\textwidth]{pointeLongitudinal}
	\end{center}
\end{wrapfigure}

Prendre comme objet AB la plaque constituée des lettres $F$ sur papier
translucide et réaliser le montage ci-contre avec une lentille convergente de la
focale de votre choix.

\subsection{Mesures}

\begin{tcb}[breakable](expe)<itc>{Relation de conjugaison}
	\begin{enumerate}
		\item Chercher la position approximative de l'image A'B' de AB ainsi obtenue
		      grâce à un écran.
		\item Viser ensuite cette image A'B' avec le viseur. Noter l'abscisse $x_1 =
			      x_{\rm A'} + D$ du viseur sur le banc.
		\item Mettre un petit morceau de papier sur la lentille $\Lc$, viser celle-ci
		      avec le viseur en réalisant la mise au point sur les déchirures du
		      papier.
		\item Noter l'abscisse $x_2 = x_0 + D$ du viseur sur le banc.
		\item Viser enfin l'objet AB avec le viseur (après avoir ôté la lentille).
		\item Noter l'abscisse $x_3 = x_{A} + D$ du viseur sur le banc. Déduire des
		      mesures $\OA$ et $\OAp$ en fonction de $x_1$, $x_2$ et $x_3$.
	\end{enumerate}
\end{tcb}

\begin{enumerate}[label=\sqenumi, start=5]
	\item Faire les applications numériques et à l'aide de la relation de
	      conjugaison avec origine au centre des lentilles minces, calculer
	      $f_{\rm desc}'$.
	\item Estimez les incertitudes, puis calculer
	      l'écart normalisé entre $f_{\rm desc}'$ et $f_{\rm th}'$ (distance
	      focale indiquée sur la lentille).
\end{enumerate}

\subsection{Observation d'une image virtuelle}

\begin{enumerate}[label=\clenumi, start=7]
	\item Proposer et réaliser un protocole permettant d'observer à l'aide du
	      viseur une image virtuelle avec la lentille $V = \SI{-10}{\delta}$.
	      Quelles sont les contraintes à respecter ?
\end{enumerate}
Faire constater votre montage (réussi) par la professeure.

\section{Focométrie par la méthode de \textsc{Bessel} avec un viseur}

Le viseur joue alors le rôle de l'écran dans la méthode de \textsc{Bessel} telle que vue dans la partie théorique.

\begin{tcb}(expe)<itc>{\textsc{Bessel} au viseur}
	\begin{enumerate}
		\item Fixer la position de l'objet AB.
		\item Pointer l'objet AB avec le viseur et noter $x_0$ l'abscisse du
		      viseur.
		\item Interposer la lentille $\SI{8}{\delta}$ entre l'objet et le viseur.
		\item Déplacer le viseur d'une distance $D > 4f'$ simple et facile à
		      repérer. L'abscisse du viseur est alors $x_0 + D$.
		\item Déterminer les positions de la lentille O$_1$ et O$_2$ qui réalisent
		      la conjugaison objet image en déplaçant la lentille entre l'objet et le
		      viseur, afin d'obtenir une image nette à travers le viseur. Relever les
		      deux positions correspondantes de la lentille notées $x_1$ et $x_2$.
	\end{enumerate}
\end{tcb}

\begin{enumerate}[label=\sqenumi, start=8]
	\item Calculer $d = x_2-x_1$, puis en déduire $f'_{\rm bess}$.
	\item Appliquer la méthode de \textsc{Silbermann} pour obtenir $f'_{\rm silb}$
	      par une autre méthode.
	\item Avec l'écart normalisé, comparer les valeurs de $f'$ obtenues
	      expérimentalement par chacune des méthodes avec la valeur donnée sur la
	      monture de la lentille, en estimant les différentes incertitudes.
\end{enumerate}

\end{document}
