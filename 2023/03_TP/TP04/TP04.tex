\documentclass[a4paper, 12pt, final, garamond]{book}
\usepackage{cours-preambule}

\raggedbottom

\makeatletter
\renewcommand{\@chapapp}{Travaux pratiques -- TP}
\makeatother

\let\SavedIndent\indent
\protected\def\indent{%
  \begingroup
    \parindent=\the\parindent
    \SavedIndent
  \endgroup
}
\setlength{\parindent}{0pt}

\begin{document}
\setcounter{chapter}{3}

\chapter{Goniom\`etre \`a r\'eseau~: spectrom\'etrie}

\section{Objectifs}

\begin{itemize}
    \item Connaître le protocole des réglages optiques du goniomètre.
    \item Réaliser les réglages géométriques d'un goniomètre à partir d'un
        protocole fourni.
    \item Vérifier la formule des réseaux en incidence normale.
    \item Déterminer le pas d'un réseau.
    \item Utiliser un réseau pour déterminer les longueurs d'onde émises par une
        lampe spectrale après étalonnage du réseau.
\end{itemize}


\section{S'approprier}
\subsection{Les réseaux de diffraction} 

\subsubsection{Qu'est-ce qu'un réseau~?}

Un réseau de diffraction par transmission est constitué d'une plaque de verre
sur laquelle ont été gravées des stries parallèles, laissant apparaître entre
elles des bandes très fines, transparentes, parallèles et équidistantes,
équivalentes à des fentes. La distance entre deux telles fentes successives est
notée $a$ et est appelée le pas du réseau. 

Son ordre de grandeur est le micromètre~: un bon réseau comporte plusieurs
dizaines de milliers de «~fentes~» (le réseau que vous allez utilisez comporte
600 traits par millimètre). Comme dans le cas des fentes d'Young, un réseau
fonctionne par interférences. Mais comme on additionne non pas deux ondes mais
plusieurs milliers, les interférences destructives sont «~plus destructives~».
En d'autres termes, de la lumière ne pourra apparaître que lorsque les
interférences seront exactement constructives. 


\subsubsection{Formule des réseaux par transmission}

\begin{wrapfigure}{r}{0.42\textwidth} 
\vspace{-40pt}

  \begin{center}
      \includegraphics[width=0.4\textwidth]{reseau}
  \end{center}

  \vspace{-40pt}
\end{wrapfigure} 


Si un tel réseau est éclairé en lumière incidente
parallèle et monochromatique de longueur d'onde $\lambda$ sous incidence
$i$, on n'observe de la lumière que dans les directions $i_{k}'$ vérifiant~:
 
 \encadre{$\sin(i_{k}')-\sin(i) = k\frac{\lambda}{a} \qqavec k\in \Zb$}
 
Avec $k$ l'ordre d'interférence (constructive). Ainsi, on ne voit de la lumière
que dans quelques directions, dont la direction incidente (pas de déviation,
$i=0$). L'examen de la formule des réseaux montre que $k$ est «~petit~»~: il ne
dépasse pas 2 ou 3 en général.

\medskip

Si la lumière incidente est polychromatique, pour un ordre $k$ donné, l'angle
$i_{k}'$ dépend de $\lambda$~: les différentes radiations sont donc
angulairement séparées (sauf pour $k = 0$ où toutes les couleurs se
superposent), et on obtient donc le spectre de la lumière incidente. On note
aussi que pour un ordre $k$ donné, à l'inverse du prisme, le violet est moins
dévié que le rouge. Ce constat n'est pas particulièrement surprenant puisque ce
n'est pas du tout les mêmes phénomènes physiques qui sont mis en jeu. En effet,
le prisme dévie la lumière de manière différenciée selon la fréquence car
l'indice optique du matériau qui le compose dépend de ladite fréquence. Dans le
cas du réseau, le phénomène est uniquement dû aux interférences entre un grand
nombre de sources. 

\medskip

Dans le TP, \textbf{le réseau sera utilisé en incidence normale}, c'est-à-dire
pour $i = 0$. Par ailleurs, le spectre sera établi autour du premier ordre
d'interférence $k = \pm 1$. 

\subsection{Présentation du goniomètre}

Un goniomètre est un appareil qui sert à mesurer des angles avec une précision
d'une minute d'angle (donc inutile de donner des résultats de précision
supérieure, mais il serait malvenu d'utiliser un appareil qui coûte le prix
d'une Twingo pour donner des résultats avec une précision de $\ang{1;;}$). C'est
donc un appareil adapté pour évaluer les déviations de rayons lumineux par un
réseau (ou un prisme).

\medskip

Pour rappel~: $\ang{;60;}$ d'angle correspond à $\ang{1;;}$. (Attention, on a
donc~: $\ang{1.2;;}= \ang{1;12;}$).

\bigskip

Un exemple de goniomètre est proposé ci-dessous. Il comporte~: 

\begin{itemize}
    \item Un collimateur réglable créant un objet lumineux à l'infini à l'aide
        d'une fente éclairée avec une lampe spectrale et d'un objectif de
        distance focale $\SI{160}{mm}$. En tirant sur l'objectif, on peut placer
        la fente au foyer objet de ce dernier afin d'avoir une image à l'infini.
    \item Une lunette autocollimatrice montée sur un support mobile en rotation
        autour d'un axe central. Cette lunette de visée est constituée d'un
        objectif de distance focale $\SI{130}{mm}$ et d'un oculaire
        autocollimateur et permet de repérer un rayon émergent du réseau.
        L'horizontalité de l'axe de la lunette est réglable.
    \item Un plateau central, lui aussi mobile en rotation autour de l'axe
        central, monté sur un socle métallique fixe, le tout pouvant être rendu
        horizontal à l'aide de trois vis de réglage.
\end{itemize}

\medskip

Puisqu'on cherche à effectuer des mesures précises, il est nécessaire de régler
\textbf{parfaitement} l'appareil grâce à la démarche donnée ci-après. Cette
démarche est longue et demande beaucoup de précisions. Soyez concentrés car si
vous faites mal une étape et que vous vous en rendez compte à l'étape 3, il
faudra tout recommencer depuis le début…

\begin{center}
    \includegraphics[width=0.8\textwidth]{goniometre}
\end{center}
  
\underline{À toutes fins utiles}~: Les épreuves de travaux pratiques aux
concours adorent l'optique. Le matériel est facile à installer et l'évaluation
est relativement simple~: le dispositif est bien réglé par le candidat… ou pas~!
Je vous invite donc à savoir faire ce réglage pour vos concours.

\section{Réaliser~: réglage du goniomètre}

\subsection{Horizontalité grossière du plateau}

Dans un premier temps, régler grossièrement (à l'œil)  l'horizontalité du
plateau. En particulier, si l'une des vis de réglage semble particulièrement
vissée ou dévissée, la placer dans une position intermédiaire. 

\subsection{Réglage de la lunette autocollimatrice}

Cette lunette doit être réglée de façon à donner d'un objet à l'infini une image
à l'infini pour éviter toute fatigue de l'œil. Le système doit donc être afocal.

\subsubsection{Régler l'oculaire à votre vue}

Allumer la lampe latérale de la lunette qui éclaire le réticule.
Régler l'oculaire à votre vue~: mettre au point le réticule en agissant sur
l'œilleton de l'oculaire. Ce réglage est personnel et nécessaire avant toute
manipulation~; La lunette est réglée quand on voit les deux fils croisés nets.


\subsubsection{Régler la lunette sur l'infini}

Positionner à la sortie de la lunette le petit miroir plan circulaire et tourner
la molette intermédiaire afin de régler la position de l'objectif  pour voir à
la fois le réticule et son image nets. Cette méthode est appelée
auto-collimation. 


\subsubsection{Régler l'horizontalité de la lunette (étape délicate~! )}

\begin{wrapfigure}{r}{0.42\textwidth} 
\vspace{-30pt}

  \begin{center}
      \includegraphics[width=0.4\textwidth]{reglage1}
  \end{center}

  \vspace{-10pt}
  \vspace{1pt}
\end{wrapfigure} 

\begin{wrapfigure}{r}{0.52\textwidth} 
\vspace{-160pt}

  \begin{center}
      \includegraphics[width=0.5\textwidth]{reglage2}
  \end{center}

  \vspace{-20pt}
  \vspace{1pt}
\end{wrapfigure} 

Le but de ce réglage est de rendre l'axe de la lunette orthogonal à l'axe de la
platine sans que celui-ci soit nécessairement vertical.


\fbox{1} Poser le miroir sans tain sur le plateau de façon à ce qu'elle soit
approximativement parallèle à l'axe $V_2-V_3$. Viser à la lunette pour observer
l'image du réticule. En général, le réticule $(R)$ et son image $(R')$ ne sont
pas  confondus. (cf ci-contre).

\medskip

\fbox{2} En agissant sur la vis $V'$ de réglage de l'horizontalité de l'axe de
la lunette (sous la lunette) et sur la vis $V_1$ du plateau, réduire l'écart
initial entre $(R)$ et $(R')$ de sa \textbf{moitié}~: un quart en agissant sur
$V'$ et un quart en agissant sur $V_1$. \textbf{Ne pas chercher à faire
coïncider les deux réticules}.  

\medskip

\fbox{3} Tourner le plateau de $\ang{180;;}$. En général, l'image du fil
horizontal $(R')$ ne coïncide pas encore avec $(R)$. 

\medskip

\fbox{4} Agir sur les deux mêmes vis de façon à diviser à nouveau l'écart entre
les fils horizontaux par 2. 

\medskip

\fbox{5} Recommencer ainsi jusqu'à ce que les fils horizontaux de $(R)$ et
$(R')$ coïncident de chaque côté. Le réglage de la lunette est alors terminé.
Cinq à dix itérations sont souvent nécessaires. Notez que l'on cherche ici la
coïncidence selon l'horizontale, pas selon la verticale. 

\medskip

\textbf{Ne plus toucher par la suite aux vis de réglages de la lunette~!}

\medskip

Retirer le miroir sans tain du plateau.
Basculer la lame semi-réfléchissante de la lunette autocollimatrice pour
éteindre sa lampe.

\medskip

Allumer la lampe à vapeur de mercure (attention~: ne pas l'éteindre indûment,
car, pour pouvoir la rallumer, il faudrait attendre qu'elle soit refroidie ce
qui peut demander plusieurs minutes).

\subsection{Réglage du tirage du collimateur}

Le collimateur doit donner de la fente une image à l'infini.
Diriger la lunette vers le collimateur $K$. Ouvrir la fente de $\SI{0,5}{mm}$
environ et l'éclairer par la source qui sera utilisée dans la manipulation
suivante (ici, par la lampe à vapeur de mercure pour commencer).

\medskip

Observer alors l'image de la fente donnée par le collimateur à travers la
lunette. Si le faisceau issu de $K$ est un faisceau de rayons parallèles,
l'image donnée par $K$ est à l'infini (ce qui implique que la fente source soit
dans le plan focal objet de $K$) et, dans la lunette, on observe une image nette
de la fente dans le plan du réticule. Si ce n'est pas le cas, la fente est mal
placée par rapport au collimateur, il faut déplacer la fente par tirage du
collimateur jusqu'à ce que l'image dans la lunette soit nette (en particulier
les bords de la fente). Attention, vous ne devez plus toucher aux réglages de la
lunette~! Refermer ensuite légèrement la fente, l'œil restant derrière
l'oculaire de la lunette, de manière à observer un trait lumineux de faible
largeur. 

\medskip

\textbf{Ne plus toucher par la suite à la bague de réglage du tirage du
collimateur.}

\medskip

Viser la position angulaire de la fente en superposant l'axe vertical de votre
réticule sur la fente. Lire l'angle associé que l'on notera $\alpha_0$. C'est
votre angle de référence pour toute la suite. 

\subsection{Réglage de l'horizontalité du plateau (suite et fin)}

Poser le réseau au centre de la plate-forme, \textbf{perpendiculairement à la
position du miroir sans tain que vous venez de retirer}. Basculer de nouveau la
lame semi-réfléchissante et allumer la lampe de la lunette autocollimatrice.

\medskip

Observer l'image du réticule par réflexion sur le réseau (qui est de mauvaise
qualité vu la nature de la surface du réseau). Faire tourner la plate-forme de
façon à amener en coïncidence le fil vertical du réticule et son image. 

\medskip

Agir sur les vis $V_2$ et $V_3$ (attention il ne faut plus toucher à $V_1$ déjà
réglée lors des précédentes étapes) d'inclinaison de la plate-forme pour amener
le fil horizontal du réticule en coïncidence avec son image. Faire la rotation
de $\ang{180;;}$ du plateau et recommencer l'opération. Quand les fils
horizontaux de $(R)$ et $(R')$ coïncident de chaque côté, le plateau est
horizontal.

\medskip

Basculer la lame semi-réfléchissante de la lunette, éteindre sa lampe, ouvrir
légèrement la fente du collimateur, les mesures peuvent commencer.


\section{Réaliser~: Utiliser le goniomètre comme un spectromètre}

\subsection{Relevé des valeurs de déviation}

La première étape consiste à étalonner le goniomètre en déterminant la position
angulaire des raies spectrales de la lampe à vapeur de mercure. Tout d'abord,
nous allons positionner le réseau afin de se placer en indidence normale sur le
réseau. Pour ce faire, 

\begin{enumerate}
    \item Positionner la lunette d'observation précisément à l'angle $\alpha_0$. 

    \item Allumer de nouveau la lampe auxiliaire de la lunette, le réseau étant
        toujours positionné sur le plateau. La lunette étant toujours à la
        position $\alpha_0$, faire tourner le plateau afin que l'image du
        réticule se superpose parfaitement à lui-même dans la lunette. Eteindre
        la lampe auxiliaire. Le réseau est alors orthogonal au faisceau
        incident. Verrouiller le plateau, il ne devra absolument plus être
        touché.

    \item Observer le spectre d'ordre 1 ($k = 1$) en tournant la lunette.
        Relever les angles $\alpha$ correspondants aux raies visibles de
        différentes couleurs. Vous ouvrirez la fente afin de voir les raies les
        moins lumineuses puis la refermerez pour augmenter votre précision de
        pointé de chacune des raies.
\end{enumerate}

Le tableau suivant précise la longueur d'onde et l'intensité (en unités
arbitraires) des différentes raies visibles de la lampe à vapeur de mercure. Il
est possible que vous ne parveniez pas à toutes les observer. 

\begin{center}
    \includegraphics[width=\textwidth]{table1}
\end{center}

À partir de vos mesures, recopiez sur votre copie puis complétez le tableau suivant.   
  
\begin{center}
    \includegraphics[width=0.9\textwidth]{table2}
\end{center}

\subsection{Tracé de la courbe d'étalonnage}

À l'aide de Regressi, répresenter $\sin(i_1)$ en fonction de $\lambda$ puis
montrer que la courbe est modélisable par une droite linéaire dont vous
déterminerez le coefficient directeur et le coefficient de corrélation. 

\section{Valider~: Résolution du doublet jaune du sodium}

Changer de lampe et prendre dorénavant la lampe à vapeur de sodium (attention la
lampe doit chauffer au moins 5 minutes avant d'être utilisée). Pour chacune des
deux raies du doublet jaune du sodium, déterminer l'angle de déviation
correspondant à l'ordre 1. En déduire alors les longueurs d'onde correspondantes
à partir de la régression de la courbe d'étalonnage précédemment établie (vous
donnerez vos résultats avec 4 chiffres significatifs). 

\medskip

Les valeurs tabulées pour le doublet du sodium sont $\lambda_1 =
\SI{589,00}{nm}$ et $\lambda_2 = \SI{589,59}{nm}$. Calculer les écarts relatifs
avec les valeur trouvées expérimentalement. 

\medskip

Que pensez-vous de la précision de cet appareil~? Le prix d'achat est-il
justifié~? 

\end{document}
