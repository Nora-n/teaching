\documentclass[a4paper, 11pt, final, garamond]{book}
\usepackage{cours-preambule}

\raggedbottom

\makeatletter
\renewcommand{\@chapapp}{Travaux pratiques -- TP}
\makeatother

\let\SavedIndent\indent
\protected\def\indent{%
  \begingroup
    \parindent=\the\parindent
    \SavedIndent
  \endgroup
}
\setlength{\parindent}{0pt}

\begin{document}
\setcounter{chapter}{11}

\chapter{\'Etude des oscillations forc\'ees d'un oscillateur \'electrique amorti}

\begin{bror}{\includehand{-90}{0.8cm}}
    Ce TP est court et non guidé. L'objectif est que vous fassiez preuve
    d'initiatives personnelles. Vous allez devoir construire en autonomie le
    protocole puis le réaliser en vue de mesurer la grandeur demandée.
\end{bror}

\section{Objectifs}

\begin{itemize}
    \item Mise en place expérimentale d'un circuit RLC série. 
    \item Utiliser correctement des dispositifs de mesure de tension. 
    \item Vérifier les caractéristiques de la résonance en intensité d'un
        circuit RLC série. 
\end{itemize}

\section{Analyser}

\begin{enumerate}[label=\sqenumi]
    \item Proposer un protocole expérimental permettant de mettre en évidence la
        résonance en intensité (en terme d'amplitude et de déphasage). Pour ce
        faire, on réalisera des mesures à plusieurs fréquences d'excitation. 
    \item On choisira la valeur de $C$, $R$ et $L$ afin que la fréquence de
        résonance soit de l'ordre de $\SI{1}{kHz}$.
    \item On précisera le montage utilisé, les branchements des appareils de
        mesures et les réglages du générateur. Un schéma électrique est attendu. 
\end{enumerate}

Le matériel suivant est disponible~: 
\medskip

\begin{minipage}{0.50\linewidth}
    \begin{itemize}
        \item Une résistance variable~;
        \item Une capacité variable~;
        \item Une bobine d'inductance $L = \SI{0,1}{H}$~;
        \item Un générateur basse fréquence (GBF)~; 
    \end{itemize}
\end{minipage}
\begin{minipage}{0.50\linewidth}
    \begin{itemize}
        \item Un multimètre~;
        \item Un oscilloscope~;
        \item Des fils~;
        \item Un générateur de tension continue. 
    \end{itemize}
\end{minipage}

\section{Réaliser}

\begin{enumerate}
    \item Réaliser le montage pour l'intensité.
\end{enumerate}
\begin{enumerate}[label=\sqenumi, start=4]
    \item Tracer l'allure des courbes de résonance en amplitude et en déphasage
\end{enumerate}

\section{Valider et conclure}

\begin{enumerate}[label=\sqenumi, start=5]
    \item Les courbes obtenues ont-elles l'allure attendue~?
    \item La largeur de la bande passante est-elle celle attendue par la théorie~?
\end{enumerate}

\end{document}
