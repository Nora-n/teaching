\documentclass[../main/main.tex]{subfiles}
\graphicspath{{./figures/}}
\usepackage{pdfpages}

\makeatletter
\renewcommand{\@chapapp}{Travaux pratiques -- TP}
\makeatother

% \toggletrue{student}
\HideSolutionstrue
\toggletrue{corrige}
% \renewcommand{\mycol}{black}
\renewcommand{\mycol}{gray}

\begin{document}
\setcounter{chapter}{15}

\chapter{\cswitch{Correction du TP}
  {Ondes ultrasonores~: mesure de caract\'eristiques}}

\enonce{
	\begin{prgm}
		\begin{tcb}*(ror)"how"{Savoir-faire}
			\begin{itemize}
				\item Mesurer la vitesse de phase, la longueur d’onde et le déphasage dû à la propagation d’un phénomène ondulatoire.
				\item Étudier le filtrage linéaire d’un signal non sinusoïdal à partir
				      d’une analyse spectrale.
				\item Détecter le caractère non linéaire d’un système par l’apparition
				      de nouvelles fréquences.
			\end{itemize}
		\end{tcb}
	\end{prgm}
	\vspace{-10pt}
	\section{Objectifs}
	\begin{itemize}
		\item Se familiariser avec les logiciels \texttt{Oscillo 5} et
		      \texttt{LatisPro}.
		\item Utiliser l'interface Sysam.
		\item Mesurer la fréquence et la longueur d'onde pour une onde ultrasonore
		\item Mesurer la vitesse de propagation des ultrasons dans l'air à la
		      température de la salle.
		\item Reconnaître une avance ou un retard entre deux courbes visualisées sur
		      un oscilloscope.
		\item Repérer le passage par un déphasage $0$ ou $\pi$ en mode \texttt{XY}.
		\item Évaluer une incertitude de type A.
		\item Simuler un processus aléatoire de variation des valeurs
		      expérimentales de l'une des grandeurs – simulation \textsc{Monte-Carlo}
		      – pour évaluer l'incertitude sur les paramètres du modèle.
	\end{itemize}
}

\section{S'approprier~: Outils théoriques et matériel disponible}

\enonce{%
	Un émetteur d'ultrasons émet des ondes ultrasonores (fréquence supérieure à
	$\SI{20}{kHz}$ donc non audibles) en continu ou en salves (appelées trains
	d'ondes). Les ondes ultrasonores sont des ondes mécaniques, longitudinales
	(lorsqu'elles se propagent dans les fluides), de compression-dilatation à trois
	dimensions.
}

\setlist[blocQR,1]{leftmargin=10pt, label=\clenumi}
\QR{%
	Rappeler la différence entre ondes longitudinales et ondes
	transversales. Donner un exemple pour chacune.
}{%
	Pour les ondes transversales, la direction de la perturbation est
	perpendiculaire à la direction de propagation. Pour les ondes longitudinales,
	les directions son parallèles.
	\smallbreak
	L'onde de compression dans un ressort est longitudinale, une onde sur une
	corde est transversale.
}

\enonce{%
	Selon les expériences, on disposera en plus d'un ou de deux récepteurs adaptés.
	On observe les ondes qui sont émises par l'émetteur et celles qui sont
	éventuellement reçues par le ou les récepteur(s) sur l'écran du logiciel
	\texttt{Oscillo 5}.
}

\section{Mesure de la période $T$ des ondes}

\enonce{%
	\subsection{Réaliser}

	\begin{tcb}(expe)<itc>{Connexion à l'interface Sysam}
		\begin{itemize}
			\item Relier l'émetteur d'ultrasons à la sortie analogique SA1 de
			      l'interface Sysam et les masses des appareils entre elles. Cette sorte
			      remplace un GBF et alimente le générateur d'ultrasons.
			\item Relier l'émetteur d'ultrasons à la voie EA0 du canal 0 de
			      l'interface pour visualiser le signal de l'émetteur sur \texttt{Oscillo
				      5} et les masses des appareils entre elles.
			\item Allumer l'ordinateur sur votre session.
		\end{itemize}
	\end{tcb}

	\begin{tcb}(expe)<itc>{Démarrer Oscillo 5 et régler le GBF}
		\begin{itemize}
			\item Ouvrir Oscillo5 (Programmes Physique-chimie $\rightarrow$ Eurosmart
			      $\rightarrow$ Oscillo5)~;
			\item Cliquer sur \texttt{Voir GBF 1} dans le panneau de contrôle (permet
			      d'accéder à un menu de réglage du GBF).
			\item Cliquer sur \texttt{marche} du bouton marche/arrêt.
			\item Sélectionner \texttt{sinusoïde}.
			\item Régler la fréquence à \SI{40}{kHz} avec les curseurs et l'amplitude à
			      \SI{10}{V}.
			\item Les réglages sont terminés. Vous pouvez cacher le panneau de contrôle
			      du GBF.
		\end{itemize}
	\end{tcb}

	\begin{tcb}(expe)<itc>{Visualiser la voie EA0 et vérifier la fréquence}
		\begin{itemize}
			\item Activer la voie.
			\item Choisir la base de temps (menu balayage) et l'amplitude du signal de
			      façon à pouvoir mesurer la période du signal.
			\item En utilisant les curseurs, dans le menu mesures en bas à droite,
			      déplacer les curseurs pour mesurer de nouveau la période.
			\item En déduire la fréquence et la comparer à celle du constructeur~:
			      \SI{40}{kHz}.
		\end{itemize}
	\end{tcb}
}

\setcounter{subsection}{1}
\subsection{Valider}
Activité \texttt{Capytale} disponible\ftn{\url{https://capytale2.ac-paris.fr/web/c/2a24-2849120}}
\setlist[blocQR,1]{leftmargin=10pt, label=\sqenumi}
\QR<[start=1]>{%
	Mettre en commun vos résultats de mesures de $T$ entre les différents
	groupes. Vous ferez un tableau \texttt{numpy} sur \texttt{Capytale}.
	\smallbreak
	En déduire le résultat du mesurage de $T$ en calculant la moyenne et
	l'incertitude de type A.
}{%
	Cf.\ \texttt{Capytale}~: \url{https://capytale2.ac-paris.fr/web/c/dc4d-2946813}
	% \begin{python}
	% import numpy as np
	%
	% T_val = np.array([25.3, 24.9, 25.1, 25.2, 23.8, 25.2, 25, 23, 24.9, 24.6, 25.1])
	% nb_val = len(T_val)
	% moy = np.mean(T_val)
	%
	% # Calculs à la main avec numpy
	% T_moins_moy = T_val - moy
	% sig = np.sqrt(1/(nb_val-1)*np.sum(T_moins_moy**2))
	% print(f'sigma/sqrt(n) =  {sig/np.sqrt(nb_val)}')
	%
	% # Calculs automatiques avec numpy
	% sig = np.std(T_val, ddof=1)
	% print(f'std/sqrt(n) = {sig/np.sqrt(nb_val)}')
	% \end{python}
}

% \begin{tcb}(rapp){Rappel~: incertitude de type A}
%     Considérons la grandeur physique $x$. L'évaluation de l'incertitude de type
%     A permet de donner un résultat de mesurage $M(x)$ par une \textbf{analyse
%     statistique} réalisées sur plusieurs mesures $m_i(x)$ réalisées dans des
%     \textbf{conditions de répétabilité} (les grandeurs d'influence n'ont pas
%     changé). L'avantage est qu'il n'est \textbf{pas nécessaire de connaître
%     les caractéristiques techniques des dispositifs de mesure employés}. Pour un
%     ensemble $m_i(x)$ de $n$ mesures de $x$, 
%     \[
%         \boxed{M(x) = \obar{m}(x) \pm \frac{1}{\sqrt{n}} \sigma(x)}
%     \]
%     % Avec\vspace*{-24pt}
%     \[
%         \obar{m}(x) = \frac{1}{n} \, \sum_{i=1}^{n} m_i(x)
%         \qqet
%         \sigma(x) = \sqrt{\frac{1}{n-1} \, \sum_{i=1}^{n} (m_i(x)-\obar{m}(x))^2}
%     \]
% \end{tcb}

\section{Mesure de la vitesse de propagation $c$}

\subsection{Analyser~: proposer un protocole}

\enonce{%
	Vous disposez d'un émetteur, d'un récepteur, d'une règle graduée au millimètre.
	L'émetteur peut envoyer un signal continu ou des salves.
}

\setlist[blocQR,1]{leftmargin=10pt, label=\clenumi}
\QR<[start=2]>{%
	Proposer un protocole expérimental vous permettant de mesurer la
	vitesse de propagation des ultrasons dans l'air. Pour plus de précision,
	vous prendrez plusieurs mesures vous permettant de réaliser une
	régression linéaire.
}{%
	On propose~:
	\begin{itemize}
		\item Placer les deux émetteurs face à face, à une distance $d_1$ grande et
		      connue mesurée à la règle graduée.
		\item Envoyer des salves suffisamment espacées pour que les signaux émis et
		      reçus ne se superposent pas.
		\item Sur \texttt{Oscillo5}, mesurer l'écart temporel $\Delta{t_1}$ entre
		      l'émission de la salve et le début de sa réception.
		\item Calculer $c_1 = d_1/\Delta{t_1}$.
		\item Diminuer la distance et réitérer l'opération pour obtenir les
		      distances $d_i$, les écarts temporels $\Delta{t_i}$ et les célérités
		      $c_i$.
		\item Tracer $d = f(\Delta{t})$ et en effectuer la régression linéaire~: le
		      coefficient directeur sera la célérité.
	\end{itemize}
}

\enonce{%
	\subsection{Réaliser}

	Réaliser le protocole précédemment proposé.
	\begin{tcb}(expe)<itc>{Indications}
		\begin{itemize}
			\item Reprendre le montage précédent~;
			\item Relier en plus le récepteur sur la voie EA1 de l'interface (et laisser
			      l'émetteur sur EA0)~;
			\item Accéder au menu GBF sur \texttt{Oscillo 5}~;
			\item Sélectionner \texttt{salve}. Laisser $\SI{10}{ms}$ de durée et
			      $\SI{10}{V}$ d'amplitude~;
			\item Les nouveaux réglages sont terminés. Cacher le panneau GBF~;
			\item Visualiser les deux voies EA0 et EA1. Synchro auto et préacquisition
			      $10 \%$~;
			\item Cliquer sur \texttt{monocoup} pour réaliser l'acquisition~;
			\item La fonction curseur permet de réaliser la mesure souhaitée.
		\end{itemize}
	\end{tcb}
}

\setcounter{subsection}{2}
\subsection{Valider}

\QR{%
	Pré-remplir (en vous aidant des fiches pratiques «~Régression linéaire~»
	et «~Survivre en \texttt{Python}~») sur \texttt{Capytale}
	un script \texttt{Python} permettant de réaliser la régression linéaire
	dont $c$ est le coefficient directeur.
}{%
	Cf.\ correction sur \textsc{Capytale}.
}
\QR{%
	Faites de même pour le traitement de l'incertitude sur $c$ par méthode
	\textsc{Monte-Carlo}.
}{%
	Idem.
}

\setlist[blocQR,1]{leftmargin=10pt, label=\sqenumi}
\QR<[start=2]>{%
	Déterminer l'incertitude de type B sur vos mesures de temps et de distance.
	Évaluer alors l'incertitude-type sur la mesure de $c$ par méthode
	\textsc{Monte-Carlo}.
}{%
	Idem.
}
\QR{%
	Comparer la valeur de célérité des ondes ultrasonores dans l'air avec
	la formule empirique donnant $c$ dans l'air en fonction de la
	température
	\footnote{\url{https://hypertextbook.com/facts/2000/CheukWong.shtml}}~:
	\[
		c = \num{331,5} + \num{0,60}\,\theta
	\]
	avec $c$ la vitesse en $\si{m.s^{-1}}$ et $\theta$ la température de
	l'air en $\degreeCelsius$, à l'aide d'un écart \textbf{normalisé}.
}{%
	Idem.
}

\section{Détermination de la longueur d'onde $\lambda$}

\enonce{%
	Ne faites cette partie que si vous avez pu terminer \textbf{proprement} les deux
	mesures précédentes.
}

\subsection{Réaliser}

\enonce{%
	\begin{tcb}(expe)<itc>{Mesure de longueur d'onde}
		\begin{itemize}
			\item L'émetteur d'ultrasons émet maintenant en direction des deux
			      récepteurs adaptés situés côte à côte de chaque côté de la règle.
			\item On observe sur l'écran de l'oscilloscope deux signaux correspondants
			      aux ondes reçues par chacun des deux récepteurs. Positionner de nouveau
			      le générateur en position continue sinusoïde comme au III.
			\item Relier les deux récepteurs aux voies EA1 et EA2 de l'interface Sysam.
			\item Superposer la ligne de «~zéro~» de chacune des deux voies.
			\item Régler correctement les sensibilités horizontales et verticales
			      permettant d'observer les sinusoïdes des voies EA1 et EA2.
		\end{itemize}
	\end{tcb}
}

\QR{%
	Mesurer la plus petite distance $d$ entre récepteurs pour
	laquelle les deux courbes se retrouvent en phase. La mesure de $d$
	est-elle précise~? Quel est le lien entre $d$ et $\lambda$~?
	Vérifier que les deux courbes se retrouvent en phase pour une distance
	entre les récepteurs égale à un multiple entier de d.
}{%
	\begin{gather*}
		\beforetext{On trouve}
		\[
			\boxed{d \approx \SI{10}{mm}}
		\]
	\end{gather*}
	Cette mesure n'est cependant pas précise, on a un intervalle de $\pm
		\SI{5}{mm}$ pour retrouver les signaux en phase. Or, on a $\lambda = d$. Pour
	plus de précisions, on pourra mesurer plus de situations en phase et diviser
	la distance totale par le nombre d'occurrence de retour en phase.

}
\QR{%
	Mesurer la distance entre les deux récepteurs correspondant
	à $10$ longueurs d'onde et en déduire une valeur de $\lambda$.
}{%
	\begin{gather*}
		d_{10} = \SI{8.7\pm 0.5e-2}{m}
		\Lra
		\boxed{\lambda = \frac{d_{10}}{10} = \SI{8.7\pm0.05e-3}{m}}
	\end{gather*}
}

\QR{%
	\textbf{BONUS} \enspace
	Dans le menu horizontal d'\texttt{Oscillo 5}, sélectionner
	\texttt{X-Y}. Observer l'allure des courbes quand elles sont en phase.
	Que remarquez-vous~? Cette méthode est-elle plus précise que la
	précédente~? Quelle est l'allure en mode \texttt{X-Y} lorsque les
	courbes sont en phase, en opposition de phase ou en quadrature de
	phase~? Comment différencier des signaux en phase et des signaux en
	opposition de phase~?
}{%
	On remarque que les courbes se superposent en une ligne droite. Elle est plus
	précise, on trouve plus facilement la position telle que la ligne soit droite.
	En dehors de la phase, les signaux forment une ellipse penchée~; elle est la
	plus grande quand les signaux sont en opposition de phase. Avec de bons
	réglages, on peut atteindre un cercle.
}

\subsection{Valider}

\QR{%
	Vos trois mesures indépendantes de $\lambda$, $c$ et $T$
	respectent-elles la relation~:
	\[
		c = \frac{\lambda}{T}
	\]
}{%
	On trouve des mesures compatibles (cf.\ corrigé en ligne).
}

\end{document}
