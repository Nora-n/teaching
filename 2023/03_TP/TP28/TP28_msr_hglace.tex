\documentclass[../main/main.tex]{subfiles}
\graphicspath{{./figures/}}

\makeatletter
\renewcommand{\@chapapp}{Travaux pratiques -- TP}
\makeatother

% \toggletrue{student}
% \toggletrue{corrige}
% \renewcommand{\mycol}{black}
% \renewcommand{\mycol}{gray}

\hfuzz=5.003pt

\begin{document}
\setcounter{chapter}{27}

\settype{enon}
\settype{solu_prof}
\settype{solu_stud}

\chapter{\cswitch{%
	  Correction du TP
  }{%
	  Mesures d'une enthalpie de changement d'état
  }%
 }

\enonce{%
	% \begin{tcn}*(exem)<ctc>"how"'t'{Capacités exigibles}
	% 	\begin{itemize}
	% 		\item Mettre en œuvre un protocole expérimental de mesure d'une grandeur
	% 		      thermodynamique énergétique
	% 	\end{itemize}
	% \end{tcn}
	% \vspace{-10pt}

	\section{Objectifs}

	\begin{itemize}
		\item Déterminer l'enthalpie de fusion de l'eau.
	\end{itemize}

	\section{S'approprier}
	On appelle enthalpie massique de fusion de l'eau, notée $\ell_F$, l'enthalpie
	massique de changement d'état lorsqu'une unité de masse d'eau passe de l'état
	solide à l'état liquide, à pression et température constantes.
	\begin{tcbraster}[raster equal height=rows, raster columns=2]
		\begin{tcn}(data){Données}
			Pour une pression de \SI{1}{bar}~:
			\begin{itemize}
				\item $c\ind{eau,L} = \SI{4.18}{kJ.K^{-1}.kg^{-1}}$~;
				\item $c\ind{eau,S} = \SI{2.10}{kJ.K^{-1}.kg^{-1}}$~;
				\item $c\ind{alu,S} = \SI{0.897}{kJ.K^{-1}.kg^{-1}}$~;
				\item $C\ind{calo} = \SI{61.8}{cal.K^{-1}}$ (sans le vase en aluminium)~;
				\item $\rho\ind{eau,L} = \SI{1.00}{kg.L^{-1}}$.
			\end{itemize}
		\end{tcn}
		\begin{tcn}(mate)"tool"'r'{Matériel}
			\begin{itemize}
				\item Calorimètre avec vase en aluminium et agitateur~;
				\item Thermomètre à alcool et sonde de température interfaçable avec
				      l'ordinateur~;
				\item Balance de précision~;
				\item Éprouvette graduée~;
				\item Glaçons sortis du congélateur et stockés dans une glacière~;
				\item Eau liquide du robinet.
			\end{itemize}
		\end{tcn}
	\end{tcbraster}
}%
\setcounter{section}{1}
\section{Analyser}

\setlist[blocQR,1]{label=\sqenumi}
\QR{%
	Proposer un protocole expérimental permettant la mesure de l'enthalpie
	massique de fusion de l'eau, notée $\ell_F$. En particulier, vous prendrez
	soin de réfléchir aux points suivants~:
	\begin{itemize}
		\item Comment préparer simplement de la glace à une température connue~?
		\item Quelle quantité de glace est-il raisonnable de prendre~? Comment
		      mesurer sa masse~?
		\item Quelle quantité d'eau liquide est-il raisonnable de mettre dans le
		      calorimètre~? À quelle température~?
		\item Est-il nécessaire, comme la semaine dernière, de procéder à une
		      correction calorimétrique des pertes du calorimètre~?
	\end{itemize}
}{%
	solu
}%
\section{Réaliser}
\QR{%
	Réaliser le protocole précédent. Comparer votre résultat de mesure de $\ell_F$
	à la valeur attendue de \SI{334}{kJ.kg^{-1}}.
}{%
	solu
}%
\QR{%
	(bonus) Adapter le protocole précédent pour un glaçon prélevé directement dans
	le congélateur du laboratoire.
}{%
	solu
}%

\end{document}
