\documentclass[a4paper, 12pt, final, garamond]{book}
\usepackage{cours-preambule}

\raggedbottom

\makeatletter
\renewcommand{\@chapapp}{Programme de kh\^olle -- semaine}
\makeatother

\begin{document}
\setcounter{chapter}{0}

\chapter{Du 18 au 21 septembre}

\section{Cours et exercices}

\section*{Optique chapitre 2 -- Base de l'optique géométrique}
\begin{enumerate}[label=\Roman*]
	\item \textbf{Propriétés générales}~: optique non géométrique~: diffraction,
	      approximation de l'optique géométrique~: notion de rayon lumineux,
	      propriétés d'un rayon lumineux, limites.
	\item \textbf{Lois de \textsc{Snell-Descartes}}~: changement de milieu, lois
	      de \textsc{Snell-Descartes} pour la réflexion et la réfraction, phénomène
	      de réflexion totale.
	\item \textbf{Généralités sur les systèmes optiques}~: système, rayons,
	      faisceaux~; objets et images réelles ou virtuelles, conjugaison et
	      schématisation $\rm A \opto{\rm S}{}A'$, objet étendu et grandissement
	      transversal, foyers principaux et secondaire d'un S.O.\ et propriétés
	      associées.
	\item \textbf{Approximation de Gauss}~: définition stigmatisme, aplanétisme,
	      rigoureux ou approché, rayons paraxiaux, conditions et approximation de
	      Gauss.
\end{enumerate}

\section*{Optique chapitre 3 -- Miroir plan et lentilles minces}
\begin{enumerate}[label=\Roman*]
	\item \textbf{Miroir plan}~: définition, stigmatisme et aplanétisme
	      rigoureux, construction pour objet réel et virtuel, relation de
	      conjugaison (démonstration), grandissement transversal (démonstration).
	\item \textbf{Lentilles minces}~: définition lentille, minces, convergentes
	      et divergentes, stigmatisme et aplanétisme, centre optique et propriété,
	      distance focale image, vergence, construction rayons parallèles à l'axe
	      optique pour divergente et convergente, règles primaires des
	      constructions géométriques, cas simples pour lentille convergente et
	      divergente, cas divers, \textbf{relation de conjugaison} et
	      grandissement transversal, \textbf{condition de netteté}, exercices
	      d'application.
\end{enumerate}

\section{Questions de cours possibles}
\begin{enumerate}
	\item Énoncer les lois de Snell-Descartes pour la réflexion et la réfraction
	      \textit{avec un schéma}, énoncer les conditions de réflexion totale
	      \textit{avec un schéma}, donner et démontrer la valeur de l'angle limite
	      $i_{\rm lim}$ en fonction de $n_2$ et $n_1$~;
	\item Définir la notion de stigmatisme et d'aplanétisme, les conditions de
	      Gauss et leur conséquence. Schéma demandé pour le stigmatisme, mais non
	      demandé pour l'aplanétisme.
	\item Construire l'image d'un objet (point ou étendu, réel ou virtuel) par
	      un miroir plan, donner et démontrer la relation de conjugaison d'un
	      miroir plan~;
	\item Définir le grandissement transversal, donner et démontrer
	      (schématiquement au moins) sa valeur pour un miroir plan, donner ses
	      expressions pour une lentille.
	\item Plusieurs tracés peuvent être demandés parmi~:
	      \begin{enumerate}
		      \item Construire l'image d'un objet étendu réel ou virtuel par une
		            lentille quelconque en présentant les règles primaires et en
		            précisant la nature de l'objet et de l'image~;
		      \item Construire le rayon émergent d'un rayon quelconque en
		            présentant les règles de construction secondaires et nommant
		            tous les points d'intérêt.
	      \end{enumerate}
	\item Savoir utiliser les relations de conjugaison pour trouver la position
	      et la taille de l'image d'un objet par une lentille mince (accompagné
	      d'un schéma)~;
	\item Savoir refaire la démonstration de la condition de netteté pour
	      l'image réelle d'un objet réel d'une lentille convergente ($D \geq
		      4f'$)~; les conditions du système seront redonnées~;
	\item Savoir refaire l'exercice «~champ de vision à travers un miroir
	      plan~»~:
	      \begin{tcb}(appl){Champ de vision à travers un miroir plan}
		      Une personne dont les yeux se situent à $h = \SI{1.70}{m}$ du sol
		      observe une mare gelée (équivalente à un miroir plan) de largeur $l =
			      \SI{5.00}{m}$ et située à $d = \SI{2.00}{m}$ d'elle.
		      \begin{enumerate}
			      \item Peut-elle voir sa propre image~? Quelle est la nature de
			            l'image~?
			      \item Quelle est la hauteur maximale $H$ d'un arbre situé de l'autre
			            côté de la mare (en bordure de mare) qu'elle peut voir par
			            réflexion dans la mare~? On notera $D = l+d$.
		      \end{enumerate}
	      \end{tcb}
\end{enumerate}

\section{Consignes}
\begin{enumerate}
	\item Une question de cours non connue entraîne une note maximale à 10/20 si
	      exercice parfait~;
	\item Les schémas des questions de cours sont obligatoires~: s'ils manquent,
	      la question ne saurait être notée au-dessus de 5~;
	\item \textbf{Pas d'association de lentilles en cours ou exercice cette
		      semaine}. Association lentille-miroir éventuellement.
	\item Chacune des règles suivantes doit être respectée~:
	      \begin{enumerate}
		      \item Les schémas optiques doivent comporter le sens de comptage
		            algébrique des distances et des angles~;
		      \item Les rayons lumineux doivent avoir un sens de propagation~;
		      \item Les angles doivent être orientés.
	      \end{enumerate}
\end{enumerate}

\end{document}
