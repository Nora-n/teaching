\documentclass[a4paper, 12pt, final, garamond]{book}
\usepackage{cours-preambule}

\raggedbottom

\makeatletter
\renewcommand{\@chapapp}{Programme de kh\^olle -- semaine}
\makeatother

\begin{document}
\setcounter{chapter}{1}

\chapter{Du 25 au 28 septembre}

\section{Cours et exercices}

\section*{Optique chapitre 3 -- Miroir plan et lentilles minces}
\begin{enumerate}[label=\Roman*]
	\item \textbf{Miroir plan}~: définition, stigmatisme et aplanétisme
	      rigoureux, construction pour objet réel et virtuel, relation de
	      conjugaison (démonstration), grandissement transversal (démonstration).
	\item \textbf{Lentilles minces}~: définition lentille, minces, convergentes
	      et divergentes, stigmatisme et aplanétisme, centre optique et propriété,
	      distance focale image, vergence, construction rayons parallèles à l'axe
	      optique pour divergente et convergente, règles primaires et secondaires
	      des constructions géométriques, tous les cas pour lentilles convergentes
	      et divergentes, relations de conjugaison + démonstration, grandissement
	      transversal.
	\item \textbf{Quelques applications}~: condition de netteté (méthode de
	      Bessel, $D \geq 4f'$), champ de vision à travers un miroir plan et
	      hauteur d'un arbre.
\end{enumerate}

\section*{Optique chapitre 4 -- Dispositifs optiques}
\begin{enumerate}[label=\Roman*]
	\item \textbf{L'œil}~: présentation et modélisation, accommodation et
	      focales minimales et maximales, réglage d'un instrument optique,
	      résolution angulaire et vocabulaire sur les défauts.
	\item \textbf{La loupe}~: présentation de l'effet loupe, définition
	      grossissement général et propriété $G = d_m/f'$ pour la loupe avec
	      démonstration.
	\item \textbf{Appareil photo}~: description, modélisation simple, champ et
	      influence de la focale et de la taille du capteur, distance de mise au
	      point, profondeur de champ et influence de la distance de mise au point,
	      de la focale et de l'ouverture.
	\item \textbf{Systèmes optiques à plusieurs lentilles}~: association
	      quelconque, notion de microscope, définition lunettes astronomiques
	      Kepler et Galilée, définition système afocal, calcul d'encombrement,
	      grossissement $G=-f'_1/f'_2$ et démonstration.
\end{enumerate}

\section{Questions de cours possibles}
\begin{enumerate}
	\item [] \textbf{Chapitre 3}
	\item Énoncer les lois de Snell-Descartes pour la réflexion et la réfraction
	      \textit{avec un schéma}, énoncer les conditions de réflexion totale
	      \textit{avec un schéma}, donner et démontrer la valeur de l'angle limite
	      $i_{\rm lim}$ en fonction de $n_2$ et $n_1$~;
	\item Construire l'image d'un objet (point ou étendu, réel ou virtuel) par
	      un miroir plan, donner et démontrer la relation de conjugaison d'un
	      miroir plan, donner et démontrer la valeur de son grandissement~;
	\item Plusieurs tracés \textbf{doivent} être demandés parmi~:
	      \begin{enumerate}
		      \item Construire l'image d'un objet étendu réel ou virtuel par une
		            lentille quelconque en présentant les 3 règles primaires et en
		            précisant la nature de l'objet et de l'image~;
		      \item Construire le rayon émergent d'un rayon quelconque en
		            présentant les règles de construction secondaires et nommant
		            tous les points d'intérêt.
	      \end{enumerate}
	\item Donner les relations de conjugaison de \textsc{Descartes} et de
	      \textsc{Newton} pour les lentilles minces, démontrer celles de
	      \textsc{Newton} et les expressions du grandissement (\textit{avec des
		      schémas})~;
	\item Savoir refaire la démonstration de la condition de netteté pour
	      l'image réelle d'un objet réel d'une lentille convergente ($D \geq 4f'$)
	      et donner les expressions des deux positions possibles de la lentille~;
	\item Savoir refaire l'exercice «~champ de vision à travers un miroir plan~»~:
	      \begin{tcb}(appl){Champ de vision à travers un miroir plan}
		      Une personne dont les yeux se situent à $h = \SI{1.70}{m}$ du sol
		      observe une mare gelée (équivalente à un miroir plan) de largeur $l =
			      \SI{5.00}{m}$ et située à $d = \SI{2.00}{m}$ d'elle.
		      \begin{enumerate}
			      \item Peut-elle voir sa propre image~? Quelle est la nature de
			            l'image~?
			      \item Quelle est la hauteur maximale $H$ d'un arbre situé de l'autre
			            côté de la mare (en bordure de mare) qu'elle peut voir par
			            réflexion dans la mare~? On notera $D = l+d$.
		      \end{enumerate}
	      \end{tcb}
	\item[] \textbf{Chapitre 4}
	\item Décrire les caractéristiques d'un œil et donner son modèle en optique
	      géométrique. Définir la plage d'accommodation, le pouvoir de résolution
	      et donner des ordres de grandeur. Décrire les principaux défauts.
	      Refaire l'exercice~:
	      \begin{tcb}(exem){Exercice~:}
		      Quelles sont les valeurs maximale et minimale de la focale du cristallin pour
		      un œil emmétrope~? On rappelle que la distance cristallin-rétine est $d
			      \approx \SI{22.3}{mm}$.
	      \end{tcb}
	\item Décrire l'effet loupe, montrer qu'on ne peut pas modifier la taille
	      \textbf{perçue} d'une image vue au travers d'une loupe, définir le
	      grossissement et démontrer sa formule pour une loupe~;
	\item Décrire un modèle simple de l'appareil photographique. Définir le
	      champ, la mise au point et la profondeur de champ d'un appareil photo~:
	      3 schémas de mise en situation sont attendus. Connaître, si demandé, la
	      manière dont un paramètre de l'appareil (focale, position capteur,
	      taille du capteur et diaphragme) modifie une caractéristique
	      photographique (profondeur de champ, champ, mise au point)~;
	\item Tracer l'image d'une association quelconque de 2 lentilles donnée par
	      l'examinataire. Qu'est-ce qu'un microscope~? Le représenter par un schéma
	      optique ($\rm A \opto{\Lc}{\rm O} A'$).
	\item Décrire les deux lunettes astronomiques vues en cours. Schéma et
	      schématisation optique ($\rm A \opto{\Lc}{\rm O} A'$) nécessaires.
	      Exprimer leur encombrement en fonction de $V_1$ et $V_2$ les vergences
	      des lentilles. Établir la formule du grossissement.
	\item Démontrer le théorème des vergences pour les lentilles accolées, et
	      démontrer la relation du grandissement d'une association de lentilles en
	      fonction du grandissement de chacune des lentilles~;
\end{enumerate}

\end{document}

