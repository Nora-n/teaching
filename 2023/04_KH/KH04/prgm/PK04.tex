\documentclass[a4paper, 12pt, final, garamond]{book}
\usepackage{cours-preambule}

\raggedbottom

\makeatletter
\renewcommand{\@chapapp}{Programme de kh\^olle -- semaine}
\makeatother

\begin{document}
\setcounter{chapter}{3}

\chapter{Du 03 au 07 octobre}

\section{Cours et exercices}

\section*{Optique chapitre 3 -- Miroir plan et lentilles minces}
\begin{enumerate}[label=\Roman*]
    \item \textbf{Miroir plan}~: définition, stigmatisme et aplanétisme
        rigoureux, construction pour objet réel et virtuel, relation de
        conjugaison (démonstration), grandissement transversal (démonstration).
    \item \textbf{Lentilles minces}~: définition lentille, minces, convergentes
        et divergentes, stigmatisme et aplanétisme, centre optique et propriété,
        distance focale image, vergence, construction rayons parallèles à l'axe
        optique pour divergente et convergente, règles primaires et secondaires
        des constructions géométriques, tous les cas pour lentilles convergentes
        et divergentes, relations de conjugaison + démonstration, grandissement
        transversal.
    \item \textbf{Quelques applications}~: condition de netteté (méthode de
        Bessel, $D \geq 4f'$), champ de vision à travers un miroir plan et
        hauteur d'un arbre.
\end{enumerate}

\section*{Optique chapitre 4 -- Dispositifs optiques}
\begin{enumerate}[label=\Roman*]
    \item \textbf{L'œil}~: présentation et modélisation, accommodation et
        focales minimales et maximales, réglage d'un instrument optique,
        résolution angulaire et vocabulaire sur les défauts.
    \item \textbf{La loupe}~: présentation de l'effet loupe, définition
        grossissement général et propriété $G = d_m/f'$ pour la loupe avec
        démonstration.
    \item \textbf{Appareil photo}~: description, modélisation simple, champ et
        influence de la focale et de la taille du capteur, distance de mise au
        point, profondeur de champ et influence de la distance de mise au point,
        de la focale et de l'ouverture.
    \item \textbf{Systèmes optiques à plusieurs lentilles}~: association
        quelconque, convergente+convergente en cours, notion de microscope,
        définition lunettes astronomiques Kepler et Galilée, définition système
        afocal, calcul d'encombrement, grossissement $G=-f'_1/f'_2$ et
        démonstration.
\end{enumerate}

\section{Cours uniquement}

\section*{Électrocinétique chapitre 1 -- Circuits électriques dans l'ARQS}
\begin{enumerate}[label=\Roman*]
    \item \textbf{Courant électrique et intensité}~: charge électrique, courant
        électrique, sens conventionnel.
    \item \textbf{Tension et potentiel}~: définition, additivité, masse,
        analogie électro-hydraulique.
    \item \textbf{Vocabulaire des circuits électriques}~: circuit, schéma,
        dipôle, nœud, branche, maille~; conventions générateur et récepteur,
        dipôles en série ou dérivation, mesures de tensions et d'intensités.
    \item \textbf{Lois fondamentales des circuits électriques dans l'ARQS}~:
        approximation, application, loi des branches et nœuds, loi des mailles,
        puissance électrocinétique, fonctionnement générateur et récepteur, et
        conservation de l'énergie.
\end{enumerate}

\section*{Électrocinétique chapitre 2 -- Résistances et sources}
\begin{enumerate}[label=\Roman*]
    \item \textbf{Généralité sur les dipôles}~: caractéristique courant-tension,
        vocabulaire associé.
    \item \textbf{Résistance}~: définition et schéma, association en série
        \textbf{et démonstration}, association en parallèle \textbf{et
        démonstration}, pont diviseur de tension \textbf{et démonstration}, pont
        diviseur de courant \textbf{et démonstration}.
    \item \textbf{Sources}~: sources idéale et réelle de tension, sources idéale
        et réelle de courant, résistances de sortie.
\end{enumerate}
\section*{Électrocinétique chapitre 3 -- Condensateurs et bobines}
\begin{enumerate}[label=\Roman*]
    \item \textbf{Condensateur idéal}~: présentation et lien $q=Cu$,
        caractéristique, continuité et régime permanent, énergie stockée
        \textbf{et démonstration}.
    \item \textbf{Bobine idéale}~: présentation, caractéristique, continuité et
        régime permanent, énergie stockée \textbf{et démonstration}.
\end{enumerate}

\section{Questions de cours possibles}
\begin{enumerate}
    \item Savoir refaire la démonstration de la condition de netteté pour
        l'image réelle d'un objet réel d'une lentille convergente ($D \geq 4f'$)
        et donner les expressions des deux positions possibles de la lentille~;
    \item Démontrer le théorème des vergences pour les lentilles accolées, et
        démontrer la relation du grandissement d'une association de lentilles en
        fonction du grandissement de chacune des lentilles~;
    \item Présenter le défaut d'un œil hypermétrope \textbf{avec un schéma},
        comment corriger ce défaut et les points caractéristique du verre
        correcteur et de l'œil qui doivent être confondus pour corriger la
        vision de loin. Une schématisation optique (du type $AB \opto{\Lc}{O}
        A'B'$) et un schéma sont nécessaires~;
    \item Savoir comment se modélise un microscope et construire l'image d'un
        objet avant le foyer objet de la première lentille. Les positions des
        points d'intérêt nécessaires au tracé seront données par l'examinataire.
        Définir alors le grossissement \textbf{sans} donner ou démontrer son
        expression, en donner un ordre de grandeur et commenter son signe~;
    \item Savoir comment se modélise une lunette de \textbf{Kepler} et
        construire le chemin de deux rayons parallèles quelconques. Les
        positions des points d'intérêt nécessaires au tracé seront données par
        l'examinataire. Définir alors le grossissement, \textbf{donner et
        démontrer} son expression, en donner un ordre de grandeur et commenter
        son signe~;
    \item Énoncer et expliquer les conditions de l'ARQS, donner des exemples
        d'application et non-application~;
    \item Démontrer puis utiliser la loi des mailles pour trouver l'intensité
        dans un circuit simple~;
    \item Démontrer les relations des associations séries et parallèles
        \textbf{et} déterminer la résistance équivalente d'une portion de
        circuit donné par l'examinataire~;
    \item Démontrer les relations des ponts diviseurs de tension et de courant~;
    \item Présenter et démontrer les caractéristiques d'un condensateur et d'une
        bobine~: relation courant-tension (sans démonstration pour la bobine),
        continuité, régime permanent, énergie stockée.
\end{enumerate}

% \section{Consignes}
% \begin{enumerate}
% 
%     \item \textbf{Les relations de conjugaison \underline{sont} à connaître}.
%     \item Une question de cours non connue entraîne un 0 à cette partie (note
%         maximale 10/20 si exercice parfait)~;
%     \item \textbf{Les schémas des questions de cours sont obligatoires~: s'ils
%         manquent, la question ne saurait être notée au-dessus de 5}~;
%     \item Chacune des règles suivantes qui ne serait respectée enlèvera
%         \textbf{un  point}~:
%         \begin{enumerate}
%             \item Les schémas optiques doivent comporter le sens de comptage
%                 algébrique des distances et des angles~;
%             \item Les rayons lumineux doivent avoir un sens de propagation~;
%             \item Les angles doivent être orientés.
%             \item Tous les dipôles doivent être fléchés en courant et tension
%                 sur les schémas.
%         \end{enumerate}
% \end{enumerate}

\end{document}
