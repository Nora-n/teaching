\documentclass[a4paper, 12pt, final, garamond]{book}
\usepackage{cours-preambule}

\raggedbottom

\makeatletter
\renewcommand{\@chapapp}{Programme de kh\^olle -- semaine}
\makeatother

\begin{document}
\setcounter{chapter}{4}

\chapter{Du 16 au 19 octobre}

\section{Cours et exercices}

\section*{Électrocinétique chapitre 2 -- Résistances et sources}
\begin{enumerate}[label=\Roman*]
	\bitem{Généralité sur les dipôles}~: caractéristique courant-tension,
	vocabulaire associé.
	\bitem{Résistance}~: définition et schéma, association en série
	\textbf{et démonstration}, association en parallèle \textbf{et
		démonstration}, pont diviseur de tension \textbf{et démonstration}, pont
	diviseur de courant \textbf{et démonstration}.
	\bitem{Sources}~: sources idéale et réelle de tension, sources idéale
	et réelle de courant, résistances de sortie.
\end{enumerate}

\section*{Électrocinétique ch.\ 3 -- Capacités et inductances~: circuits du
  1\ier{} ordre}
% \import{../../../01_C/02_elec/E3/}{E3_capaind.mtc1}
\begin{enumerate}[label=\Roman*]
	\bitem{Condensateur et circuit RC}~:
	\begin{enumerate}[label=\Alph*]
		\bitem{Présentation condensateur}~: relation fondamentale, RCT, continuité et
		RP, associations série et parallèle, condensateur réel et énergie stockée.
		\bitem{Circuit RC série~: charge}~: échelon montant, présentation, équa.
		diff., dimension de $RC$, méthode de résolution et solution, réprésentation
		graphique, constante de temps et temps de réponse à 99\%, intensité, bilan de
		puissance et d'énergie.
		\bitem{Circuit RC série~: décharge}~: idem sans bilan.
	\end{enumerate}
	\bitem{Bobine et circuit RL}~:
	\begin{enumerate}[label=\Alph*]
		\bitem{Présentation bobine}~: RCT, continuité et RP, associations, bobine
		réelle et énergie stockée.
		\bitem{Circuit RL série~: échelon montant}~: idem RC charge.
		\bitem{Circuit RL série~: décharge}~: idem RC décharge.
	\end{enumerate}
\end{enumerate}

\section{Cours uniquement}
\section*{Électrocinétique chapitre 4 -- Oscillateurs harmonique et amorti}

\begin{enumerate}[label=\Roman*]
	\bitem{Oscillateurs harmoniques}~:
	\begin{enumerate}[label=\Alph*]
		\bitem{Introduction harmonique}~: signal sinusoïdal, équation différentielle
		générale et solution, changement de variable, exemple expérimental LC.
		\bitem{Oscillateur harmonique LC libre}~: présentation, équation
		différentielle, unité de $\w_0$, solutions $u_C(t)$ et $i(t)$, graphique,
		bilan énergétique et graphique.
	\end{enumerate}
\end{enumerate}

\section{Questions de cours possibles}
\begin{enumerate}
	\item[] \textbf{Chapitre 2}
		% \item Démontrer puis utiliser la loi des mailles pour trouver l'intensité
		%       dans un circuit simple (deux mailles possible)~;
		% \item Démontrer les relations des associations séries et parallèles des
		%       résistances \textbf{et} déterminer la résistance équivalente d'une
		%       portion de circuit donné par l'examinataire~;
	\item Démontrer les relations des ponts diviseurs de tension et de courant
	      et en utiliser sur un schéma donné par l'examinataire~;
	\item Présenter les sources réelles de tension et de courant. Comment
	      s'appellent ces modèles~? À l'aide de relations de ponts diviseurs,
	      démontrer dans quelles conditions on peut les considérer comme idéales.
	\item[] \textbf{Chapitre 3}
	\item Présenter et démontrer les caractéristiques d'un condensateur et d'une
	      bobine~: relation courant-tension (sans démonstration pour la bobine),
	      continuité, régime permanent, énergie stockée.
	\item Démontrer les relations des associations séries et parallèles
	      d'un condensateur \textbf{et} d'une bobine.
	\item Présenter le schéma et la condition initiale, donner et démontrer
        l'équation différentielle, \textbf{justifier l'unité de $\tau$}, établir
        la solution et la tracer pour un des quatre circuits suivants~:
	      \begin{tasks}[label=\protect\fbox{\Alph*}, label-width=4ex](4)
		      \task RC en charge
		      \task RC en décharge
		      \task RL montant
		      \task RL régime libre
	      \end{tasks}
	\item Faire un bilan de puissance, éventuellement un bilan d'énergie,
        démontrer comment trouver graphiquement la constante de temps et établir
        le temps de réponse à 99\% pour un des circuits suivants~:
	      \begin{tasks}[label=\protect\fbox{\Alph*}, label-width=4ex](2)
		      \task Circuit RC en charge
		      \task Circuit RL échelon montant
	      \end{tasks}
	\item[] \textbf{Chapitre 4}
	\item Donner la forme générale d'un signal sinusoïdal en détaillant les
	      paramètres, expliquer ce qu'est la pulsation et exprimer la période en
	      fonction de la pulsation.
	\item Donner l'équation différentielle générale d'un oscillateur harmonique
	      et les deux formes de solutions associées. Expliquer le principe du
	      changement de variable avec cette équation comme exemple, et résoudre
	      l'équation du RL montant avec cette méthode.
	\item Pour le circuit LC en régime libre, présenter le schéma et les
        conditions initales, établir l'équation différentielle,
        \textbf{justifier l'unité de $\w_0$}, établir les solutions de $u_C(t)$
        et $i(t)$ et les tracer en fonction du temps \textbf{puis} dans l'espace
        des phases (axe $x = u_C(t)$, axe $y = i(t)$) sans tenir compte des
        constantes mutiplicatives.
\end{enumerate}

\end{document}
