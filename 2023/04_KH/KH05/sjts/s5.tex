\documentclass[a4paper, 11pt]{book}
\usepackage{/home/nicolas/Documents/Enseignement/Prepa/bpep/fichiers_utiles/preambule}

\newcommand{\dsNB}{5}
\makeatletter
\renewcommand{\@chapapp}{Kh\^olles MPSI3 -- semaine \dsNB}
\makeatother

\toggletrue{corrige}  % décommenter pour passer en mode corrigé

\begin{document}

\resetQ
\newpage

\chapter{Sujet 1\siCorrige{\!\!-- corrig\'e}}
\section{Question de cours}

\QR{
    \begin{minipage}{0.6\linewidth} Démontrer les relations des ponts diviseurs
        de tension et de courant. Exprimer $U$ en fonction de $E$ sur le circuit
        suivant.
    \end{minipage}
    \begin{minipage}{0.4\linewidth}
        \begin{center}
            \includegraphics[width=\linewidth]{divtens_exo}
        \end{center}
    \end{minipage}
}{
    \begin{minipage}{0.45\linewidth}
        \begin{center}
            \includegraphics[width=\linewidth]{divtens-rangle_3r}
        \end{center}
    \end{minipage}
    \hfill
    \begin{minipage}{0.45\linewidth}
        Dans ce circuit, $I = \frac{E}{R_{\rm eq}}$ et $U_{R_x} = R_xI$, d'où
        \begin{empheq}[box=\fbox]{equation*}
            U_{R_x} = \frac{R_x}{R_{\rm eq}}E
        \end{empheq}
    \end{minipage}
    \begin{minipage}{0.49\linewidth}
        \includegraphics[width=\linewidth]{divtens_exo-corr}
    \end{minipage}
    \hfill
    \begin{minipage}{0.49\linewidth}
        \centering Avec les schémas précédents,
        \begin{align*}
            U_{AB}                 & = \frac{
            \frac{3}{\cancel{4}}\bcancel{R}}{\frac{11}{\cancel{4}}\bcancel{R}}E\\
            \Leftrightarrow U_{AB} & = \frac{3}{11}E\\
            U                      & = -\frac{\cancel{R}}{3\cancel{R}}U_{AB}\\
            \Leftrightarrow U      & = - \frac{1}{11}E
        \end{align*}
    \end{minipage}}

\resetQ

\subimport{/home/nicolas/Documents/Enseignement/Prepa/bpep/exercices/TD/loi_noeuds_potentiel/}{sujet.tex}

\resetQ
\newpage

\chapter{Sujet 2\siCorrige{\!\!-- corrigé}}
\section{Question de cours}

Présenter le circuit RC en charge sous un échelon de tension $E$ (schéma et
condition initiale), donner et démontrer l'équation différentielle sur $u_C$,
donner la solution et la tracer. Indiquer sans le démontrer comment trouver la
constante de temps et le régime permanent.

\section{Pont de Wheatstone}

\QR{\begin{minipage}{0.65\linewidth}

        En électronique, on réalise régulièrement des ponts de mesure pour
        mesurer indirectement une résistance. On dispose d'un circuit comprenant
        un générateur de tension qui alimente un pont de Wheatstone composé des
        résistances $R_1$ et $R_2$. La résistance $R_i$ est inconnue, et la
        résistance $R$ est variable (il s'agit d'un potentiomètre). On fait
        évoluer $R$ jusqu'à ce que le voltmètre indique une tension nulle. Le
        pont est alors équilibré. \bigbreak

        À l'aide des lois de Kirchhoff, déterminer l'expression de la valeur de
        $R_i$ en fonction des valeurs des autres résistances lorsque le pont est
        équilibré.
    \end{minipage}
    \begin{minipage}{0.35\linewidth}
        \begin{center}
            \includegraphics[width=\linewidth]{wheatstone-plain}
        \end{center}
\end{minipage}}{
\begin{tcbraster}[raster columns=6, raster equal height=rows]
    \begin{NCdefi}[raster multicolumn=2]{Schéma}
        \begin{center}
            \includegraphics{wheatstone}
        \end{center}
    \end{NCdefi}
    \begin{tcolorbox}[blankest, raster multicolumn=1, space to=\myspace]
        \begin{tcbraster}[raster columns=1]
            \begin{NCprop}[add to natural height=\myspace]{Résultat attendu}

                    \fontsize{10pt}{12pt}\selectfont On cherche $R_i$, ou
                    $U_{DC}$ quand «~le pont est équilibré~».
                \end{NCprop}
                \begin{NCrapp}{Outil}

                    \fontsize{10pt}{12pt}\selectfont D'après l'énoncé, le pont
                    est équilibré quand $V = 0$, soit quand $V_B = V_D$.

                \end{NCrapp}
            \end{tcbraster}
        \end{tcolorbox}
        \begin{NCexem}[raster multicolumn=3]{Application} Si le pont est
            équilibré, alors $U_{AB} = U_{AD}$ et $U_{BC} = U_{DC}$. Or, avec le
            pont diviseur de tension, on a à la fois
            \begin{align*}
                U_{BC} & = E \frac{R_1}{R_1+R}\\
                U_{DC} & = E \frac{R_i}{R_i+R_2}
            \end{align*}
            Donc
            \begin{align*}
                U_{BC} &= U_{DC}\\
                \Leftrightarrow \cancel{E} \frac{R_1}{R_1+R}
                       & = \cancel{E} \frac{R_i}{R_i+R_2}\\
                \Leftrightarrow R_1(\cancel{R_i}+R_2) & = R_i(\cancel{R_1}+R)\\
                \Leftrightarrow \Aboxed{R_i & = \frac{R_1R_2}{R}}
            \end{align*}
        \end{NCexem}
    \end{tcbraster}
}

\resetQ

\subimport{/home/nicolas/Documents/Enseignement/Prepa/bpep/exercices/Colle/loi_kirchhoff_4/}{sujet.tex}

\resetQ
\newpage

\chapter{Sujet 3\siCorrige{\!\!-- corrigé}}
\section{Question de cours}

Présenter et démontrer les caractéristiques d'un condensateur et d'une bobine~:
relation courant-tension (sans démonstration pour la bobine), continuité, régime
permanent, énergie stockée.

\subimport{/home/nicolas/Documents/Enseignement/Prepa/bpep/exercices/Colle/batterie_tampon/}{sujet.tex}

\resetQ
\newpage

\chapter{Sujet 4\siCorrige{\!\!-- corrigé}}

\section{Pont diviseur de courant}

\QR{ Exprimer l'intensité $I$ en fonction de $I_0$.
    \begin{center}
        \includegraphics[scale=1]{divcour_last-plain}
    \end{center}
}{
    \begin{NCrapp}[sidebyside]{Outil}
        \begin{center}
            \includegraphics[width=\linewidth]{divcour_3-eq}
        \end{center}
        \tcblower
        \begin{center}
            Dans le circuit ci-contre, \[ \boxed{I_x = \frac{R_{\rm
            eq}}{R_x}I}\]
        \end{center}
    \end{NCrapp}
    \begin{NCdefi}[]{Schéma}
        \begin{center}
            \includegraphics[width=\linewidth]{divcour_last-a}
        \end{center}
    \end{NCdefi}
    \begin{NCexem}[]{Application} En premier lieu, \[ I =
        \frac{\frac{2r}{3}}{r}I_d = \frac{2}{3}I_d\] Ensuite, \[ I_d = \frac{
        \frac{2r}{5}}{ \frac{2r}{3}}I_0 = \frac{3}{5}I_0\] Ainsi, \[ \boxed{I =
\frac{2}{5}I_0}\]
\end{NCexem}
}

\resetQ

\section{Association de générateurs~: application}

\QR{Deux générateurs de tension ($E_1$, $r_1$) et ($E_2$, $r_2$) sont placés en
    parallèle l'un de l'autre. Ils alimentent une résistance $R_4$, également
    placée en parallèle sur les générateurs.
    \begin{enumerate}
        \item Dessiner le schéma normalisé de ce montage et flécher les courants
            et les tensions.
        \item Exprimer l'intensité du courant qui circule dans $R_4$.
        \item Exprimer la tension aux bornes de $R_4$.
\end{enumerate}}{
    \begin{tcbraster}[raster columns=2, raster equal height=rows]
        \begin{NCdefi}{Schéma}
            \begin{center}
                \includegraphics{assogen_parr-ldm}
            \end{center}
        \end{NCdefi}
        \begin{tcolorbox}[blankest, space to=\myspace]
            \begin{tcbraster}[raster columns=1]
                \begin{NCprop}[add to natural height=\myspace]{Résultat attendu}
                    On cherche $I_4$ puis $U_4 = R_4I_4$.
                \end{NCprop}
                \begin{NCrapp}{Outils}
                    \begin{itemize}
                        \item LdM 1~: $I_4R_4 + I_1r_1 = E_1 \quad \color{ForestGreen}(1)$~;
                        \item LdM 2~: $I_4R_4 + I_2r_2 = E_2 \quad \color{ForestGreen}(2)$~;
                        \item LdN 1~: $I_1 + I_2 = I_4 \quad \color{ForestGreen}(3)$.
                    \end{itemize}
                \end{NCrapp} 
            \end{tcbraster}
        \end{tcolorbox}
    \end{tcbraster}
    \begin{tcbraster}[raster columns=7, raster equal height=rows]
        \begin{NCror}[raster multicolumn=3]{Approche méthodique}
            Notre but est de trouver une équation contenant $I_4$ et des valeurs
            connues, c'est-à-dire tout sauf $I_1, I_2$.
            \bigbreak
            L'équation \textcolor{ForestGreen}{(1)} peut nous aider~; on peut la
            transformer en remplaçant $I_1$ par $I_4-I_2$ grâce à
            \textcolor{ForestGreen}{(3)} pour avoir une équation
            \textcolor{ForestGreen}{(4)} avec $I_4$ et $I_2$.
            \bigbreak
            Mais comme \textcolor{ForestGreen}{(2)} nous permet d'isoler $I_2$ et de
            l'exprimer en fonction de $I_4$, en injectant cette expression dans
            \textcolor{ForestGreen}{(4)} on obtient une équation entre $I_4$ et les
            éléments du circuit. Question résolue !
        \end{NCror}
        \begin{NCexem}[raster multicolumn=4]{Application}
            Avec \textcolor{ForestGreen}{(3)} dans \textcolor{ForestGreen}{(1)}~:
            \[I_4R_4 + (I_4-I_2)r_1 = E_1 \quad \color{ForestGreen}(4)\]
            En réexprimant \textcolor{ForestGreen}{(2)}~:
            \[I_2 = (E_2 - I_4R_4)/r_2\]
            En injectant \textcolor{ForestGreen}{(2)} dans
            \textcolor{ForestGreen}{(4)}~:
            \begin{align*}
                I_4(R_4+r_1) - (E_2-I_4R_4) \frac{r_1}{\color{brandeisblue}r_2}
                    &= E_1\\
                    \Leftrightarrow I_4(\textcolor{orange}{R_4} +
                                        \textcolor{red}{r_1})
                                        {\color{brandeisblue}r_2}
                                        -
                                        (E_2-I_4\textcolor{orange}{R_4})
                                        \textcolor{red}{r_1}
                    &= E_1{\color{brandeisblue}r_2}\\
                \Leftrightarrow I_4(\textcolor{red}{r_1}
                                    \textcolor{brandeisblue}{r_2} +
                                    \textcolor{red}{r_1}\textcolor{orange}{R_4} +
                                    \textcolor{brandeisblue}{r_2}\textcolor{orange}{R_4})
                    &= E_1r_2 +E_2r_1
            \end{align*}
            Soit
            \[\boxed{I_4 = \frac{E_1r_2 + E_2r_1}{r_1r_2+r_1R_4+r_2R_4}} \quad
            \text{et} \quad \boxed{U_{R_4} = R_4\times I_4}\]
        \end{NCexem}
    \end{tcbraster}
}

\resetQ

\end{document}
