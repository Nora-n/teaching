\documentclass[a4paper, 12pt, final, garamond]{book}
\usepackage{cours-preambule}

\raggedbottom

\makeatletter
\renewcommand{\@chapapp}{Programme de kh\^olle -- semaine}
\makeatother

\begin{document}
\setcounter{chapter}{5}

\chapter{Du 06 au 09 novembre}

\section{Cours et exercices}

\section*{Électrocinétique ch.\ 3 -- Capacités et inductances~: circuits du
  1\ier{} ordre}
\begin{enumerate}[label=\Roman*]
	\bitem{Condensateur et circuit RC}~:
	\begin{enumerate}[label=\Alph*]
		\bitem{Présentation condensateur}~: relation fondamentale, RCT, continuité et
		RP, associations série et parallèle, condensateur réel et énergie stockée.
		\bitem{Circuit RC série~: charge}~: échelon montant, présentation, équa.
		diff., dimension de $RC$, méthode de résolution et solution, réprésentation
		graphique, constante de temps et temps de réponse à 99\%, intensité, bilan de
		puissance et d'énergie.
		\bitem{Circuit RC série~: décharge}~: idem sans bilan.
	\end{enumerate}
	\bitem{Bobine et circuit RL}~:
	\begin{enumerate}[label=\Alph*]
		\bitem{Présentation bobine}~: RCT, continuité et RP, associations, bobine
		réelle et énergie stockée.
		\bitem{Circuit RL série~: échelon montant}~: idem RC charge.
		\bitem{Circuit RL série~: décharge}~: idem RC décharge.
	\end{enumerate}
\end{enumerate}

\section*{Électrocinétique ch.\ 4 -- Oscillateurs harmonique et amorti}
\begin{enumerate}[label=\Roman*]
	\bitem{Oscillateurs harmoniques}~:
	\begin{enumerate}[label=\Alph*]
		\bitem{Introduction harmonique}~: signal sinusoïdal, équation différentielle
		générale et solution, changement de variable, exemple expérimental LC.
		\bitem{Oscillateur harmonique LC libre}~: présentation, équation
		différentielle, unité de $\w_0$, solutions $u_C(t)$ et $i(t)$, graphique,
		bilan énergétique et graphique.
    \bitem{Ressort horizontal libre}~: force de rappel (de \textsc{Hooke}),
    schéma et situation initiale, équation différentielle et solution, analogie
    LC-ressort, bilan de puissance et conservation de l'énergie avec définition
    énergie potentielle élastique, tracé dans l'espace des phases.
    \bitem{Complément~: circuit LC montant}~: présentation, équation solutions
    et tracé, bilan d'énergie et tracé.
	\end{enumerate}
\end{enumerate}

\section{Cours uniquement}

\section*{Électrocinétique ch.\ 4 -- Oscillateurs harmonique et amorti}
\begin{enumerate}[label=\Roman*, start=2]
  \bitem{Oscillateurs amortis}~:
	\begin{enumerate}[label=\Alph*]
		\bitem{Introduction amorti}~: exemple expérimental RLC, vocabulaire, équation
    différentielle générale, dimension de $Q$, équation caractéristique et
    régimes de solutions, présentation des solutions générales.
		\bitem{Oscillateur amorti RLC libre}~: présentation, bilan de puissance,
    équation différentielle, et solutions dans tous les régimes avec tracé et
    transitoire à 95\%~; extrapolation à $Q \ra \infty$ et $Q \ra 0$~; tracés
    dans l'espace des phases.
    \bitem{Ressort horizontal amorti libre}~: 
    schéma et situation initiale, équation différentielle, analogie RLC-ressort
    amorti, bilan de puissance.
	\end{enumerate}
\end{enumerate}

\section{Questions de cours possibles}
\begin{enumerate}
	\item[] \textbf{Chapitre 3}
	\item Présenter le schéma et la condition initiale, donner et démontrer
        l'équation différentielle, \textbf{justifier l'unité de $\tau$}, établir
        la solution et la tracer pour un des quatre circuits suivants~:
	      \begin{tasks}[label=\protect\fbox{\Alph*}, label-width=4ex](4)
		      \task RC en charge
		      \task RC en décharge
		      \task RL montant
		      \task RL régime libre
	      \end{tasks}
	\item Faire un bilan de puissance, éventuellement un bilan d'énergie,
        démontrer comment trouver graphiquement la constante de temps et établir
        le temps de réponse à 99\% pour un des circuits suivants~:
	      \begin{tasks}[label=\protect\fbox{\Alph*}, label-width=4ex](4)
		      \task RC en charge
		      \task RC en décharge
		      \task RL montant
		      \task RL régime libre
	      \end{tasks}
	\item[] \textbf{Chapitre 4~: harmonique}
	\item Présenter le schéma et les conditions initiales, établir l'équation
        différentielle, \textbf{justifier l'unité de $\w_0$}, établir les
        solutions de $u_C(t)$ et $i(t)$ (ou $x(t)$ (ou $\ell(t)$) et $v(t)$)
        et les tracer en fonction du temps \textbf{puis} dans l'espace des
        phases sans tenir compte des constantes multiplicatives pour un des
        systèmes suivants~:
	      \begin{tasks}[label=\protect\fbox{\Alph*}, label-width=4ex](3)
		      \task LC libre
		      \task LC montant
          \task Ressort libre sans frottements
	      \end{tasks}

  \item Faire un \textbf{bilan de puissance} pour le circuit \textbf{LC libre}
      \textit{et} le \textbf{ressort sans frottements}, démontrer la
      conservation de l'énergie totale, tracer la forme des graphiques.

  \item Faire l'analogie complète entre les deux systèmes harmoniques LC libre
      et ressort sans frottement~: présentation, conditions initiales,
      équations différentielles \textbf{sans démonstration}, correspondance
      entre les grandeurs, tracé de la solution \textbf{dans l'espace des
      phases} sans résolution et commenter sur la conservation de l'énergie
      visible dans le graphique.

  \item[] \textbf{Chapitre 4~: amorti}
  \item Présenter (schéma et conditions initiales),
        donner et \textbf{démontrer} l'équation différentielle sous
        forme canonique \textbf{qu'on ne cherchera pas à résoudre}, vérifier son
        homogénéité, présenter les graphiques des solutions selon les valeurs de
        $Q$ dans l'espace temporel \textbf{et} dans l'espace des phases en donnant
        un approximation de la durée du régime transitoire à 95\% pour un des
        systèmes suivants~:
	      \begin{tasks}[label=\protect\fbox{\Alph*}, label-width=4ex](2)
		      \task RLC libre
		      \task Ressort horizontal amorti
	      \end{tasks}

  \item Faire l'analogie complète entre les deux systèmes amortis RLC libre et
        ressort avec frottement fluide~: présentation, conditions initiales,
        équations différentielles \textbf{sans démonstration}, correspondance
        entre les grandeurs, tracer de solutions \textbf{dans l'espace des
        phases} selon différentes valeurs de $Q$ sans résolution et commenter
        sur l'évolution de l'énergie visible dans le graphique.

  \item Résoudre l'équation différentielle d'un oscillateur amorti de conditions
        initiales données par l'interrogataire pour l'un des trois régimes
        possibles.

  \item Faire le \textbf{bilan de puissance} de l'oscillateur amorti électrique
        \textbf{RLC} libre \textit{et} du \textbf{ressort horizontal avec frottement
        fluide}, identifier les termes du bilan et expliciter la signification
        physique de chacun des termes.
\end{enumerate}

\end{document}

