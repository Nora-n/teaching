\documentclass[a4paper, 12pt, final, garamond]{book}
\usepackage{cours-preambule}

\raggedbottom

\makeatletter
\renewcommand{\@chapapp}{Programme de kh\^olle -- semaine}
\makeatother

\begin{document}
\setcounter{chapter}{6}

\chapter{Du 07 au 11 novembre}

\section{Cours et exercices}

\section*{Électrocinétique chapitre 3 -- Condensateurs et bobines}
\begin{enumerate}[label=\Roman*]
    \item \textbf{Condensateur idéal}~: présentation et lien $q=Cu$,
        caractéristique, continuité et régime permanent, énergie stockée
        \textbf{et démonstration}.
    \item \textbf{Bobine idéale}~: présentation, caractéristique, continuité et
        régime permanent, énergie stockée \textbf{et démonstration}.
    \item \textbf{Circuit RC série~: charge}~: présentation, équation
        différentielle, résolution avec méthode, représentation graphique et
        constante de temps + régimes transitoire, permanent, évolution de
        l'intensité, bilans de puissance et d'énergie.
    \item \textbf{Circuit RC série~: décharge}~: présentation, équation
        différentielle, résolution avec méthode, représentation graphique et
        constante de temps + régimes transitoire, permanent, évolution de
        l'intensité.
    \item \textbf{Circuit RL série~: échelon montant}~: présentation, équation
        différentielle, résolution avec méthode, représentation graphique et
        constante de temps + régimes transitoire, permanent, évolution de
        la tension, bilan de puissance.
    \item \textbf{Circuit RL série~: échelon descendant}~: présentation, équation
        différentielle, résolution avec méthode, représentation graphique et
        constante de temps + régimes transitoire, permanent, évolution de
        la tension.
\end{enumerate}

\section*{Électrocinétique chapitre 4 -- Oscillateurs harmonique et amorti}

\begin{enumerate}[label=\Roman*]
    \item \textbf{Introduction harmonique}~: description générale d'un signal
        sinusoïdal, équation différentielle d'un oscillateur harmonique et
        solution générale, changement de variable général $\rightarrow$
        homogène, exemple courbe expérimentale oscillateur LC.
    \item \textbf{Oscillateur harmonique électrique LC libre}~: présentation,
        équation différentielle, unité de $\w_0$, résolution avec 2 méthodes
        pour les constantes d'intégration, tracé de $u_C(t)$ et $i(t)$, aspect
        énergétique démonstration conservation et représentation graphique.
    \item \textbf{Oscillateur harmonique mécanique ressort libre}~: définition
        force de rappel, présentation, équation différentielle pour $\ell$ et
        $x$ et démonstration, analogie LC-ressort libre, aspect énergétique~:
        définition énergie potentielle élastique et mécanique, démonstration
        conservation, graphique et \textbf{visualisation dans l'espace des
        phases}.
    \item \textbf{Oscillateur harmonique électrique LC montant}~: présentation,
        équation différentielle, résolution avec changement de variable, tracé
        de $u_C(t)$ et $i(t)$ (et $u_L(t)$), représentation graphique uniquement
        de $\Ec_{\rm tot}$.
\end{enumerate}

\section{Cours uniquement}
\bigbreak
\begin{enumerate}[label=\Roman*, start=5]
    \item \textbf{Introduction amorti}~: exemple RLC différents régimes ($Q =$
        13, 3, 0.5 et 0.2) et analyse, équation différentielle générale et
        analyse $Q$, définition équation caractéristique, discriminant et
        différents régimes, solutions générales.
    \item \textbf{Oscillateur amorti électrique RLC libre}~: présentation,
        \textbf{bilan énergétique} et analyse, équation différentielle et
        conditions initiales, solution, démonstrations, régimes transitoires à
        95\% et visualisation dans l'espace des phases \textbf{pour tous les
        régimes}, limite $Q \rightarrow \infty$.
    \item \textbf{Oscillateur amorti mécanique ressort frottements fluides}~:
        présentation et force de frottement fluide, équation différentielle et
        solution pour $\ell$ et $x$, analogie RLC-ressort amorti, sur toutes les
        grandeurs ($x$, $v$, $m$, $k$, $\alpha$, $\w_0$, $Q$), et résumé complet
        oscillateurs amortis.
\end{enumerate}

\section{Questions de cours possibles}
\begin{enumerate}
    % \item Présenter le circuit RC en charge sous un échelon de tension $E$
    %     (schéma et condition initiale), donner et \textbf{démontrer} l'équation
    %     différentielle sur $u_C$, donner et \textbf{démontrer} la solution et la
    %     tracer. Indiquer \textbf{et démontrer} comment trouver la constante de temps
    %     et le régime permanent.
    % \item Présenter le circuit RC en décharge depuis une tension $E$ aux bornes
    %     du condensateur (schéma et condition initiale), donner et démontrer
    %     l'équation différentielle sur $u_C$, \textbf{démontrer} la solution et
    %     la tracer. Indiquer \textbf{et démontrer} comment trouver la constante de
    %     temps et le régime permanent.
    % \item Présenter le circuit RL soumis à un échelon de tension $E$ (schéma et
    %     condition initiale), donner et \textbf{démontrer} l'équation différentielle sur
    %     $i$, donner \textbf{et démontrer} la solution et la tracer. Indiquer
    %     \textbf{et démontrer} comment trouver la constante de temps et le régime
    %     permanent.
    \item Présenter le circuit RL soumis à un échelon de tension descendant
        (schéma et condition initiale), donner et \textbf{démontrer} l'équation
        différentielle sur $i$, donner \textbf{et démontrer} la solution et la
        tracer. Indiquer \textbf{et démontrer} comment trouver la constante de temps
        et le régime permanent.

    % \item Connaître les caractéristiques d'un signal sinusoïdal (amplitude,
    %     phase initiale, pulsation et lien avec la période et réflexion en
    %     vitesse angulaire) et \textbf{les présenter sur un graphique}, savoir
    %     retrouver l'amplitude, la phase initiale et la période/pulsation sur un
    %     graphique ou l'équation.

    \item Présenter le circuit LC libre (schéma et conditions initiales), donner
        et \textbf{démontrer} l'équation différentielle sur $u_C$, vérifier son
        homogénéité, donner et \textbf{démontrer} la solution et la tracer en
        espace temporel \textbf{et} dans l'espace des phases ($u_C$, $i$).
    % \item Présenter le circuit LC montant (schéma et conditions initiales), donner
    %     et \textbf{démontrer} l'équation différentielle sur $u_C$, vérifier son
    %     homogénéité, donner et \textbf{démontrer} la solution et la tracer en
    %     espace temporel \textbf{et} dans l'espace des phases ($u_C$, $i$).

    \item Présenter le ressort horizontal sans frottements (schéma, conditions
        initiales et bilan des forces), donner et \textbf{démontrer} l'équation
        différentielle sur $\ell$ ou $x$, vérifier son homogénéité, donner et
        \textbf{démontrer} la solution et la tracer en espace temporel
        \textbf{et} dans l'espace des phases ($x$ ou $\ell$, $v$).

    \item Faire un bilan d'énergie pour le circuit LC libre, démontrer la
        conservation de l'énergie totale, tracer la forme du graphique, et faire
        un bilan d'énergie pour le ressort horizontal sans frottement, démontrer
        la conservation de l'énergie mécanique, tracer le graphique
        correspondant.

    \item Faire l'analogie complète entre les deux systèmes harmoniques LC libre
        et ressort sans frottement~: présentation, conditions initiales,
        équations différentielles \textbf{sans démonstration}, correspondance
        entre les grandeurs, tracer de la solution \textbf{dans l'espace des
        phases} sans résolution et commenter sur la conservation de l'énergie
        visible dans le graphique.

    \item Présenter le circuit RLC libre (schéma et conditions initiales),
        donner et \textbf{démontrer} l'équation différentielle sur $u_C$ sous
        forme canonique \textbf{qu'on ne cherchera pas à résoudre}, vérifier son
        homogénéité, présenter les graphiques des solutions selon les valeurs de
        $Q$ dans l'espace temporel \textbf{et} dans l'espace des phases ($u_C$,
        $i$) en donnant un approximation de la durée du régime transitoire à
        95\%.
    \item Présenter le ressort horizontal avec frottement fluide (schéma et
        conditions initiales), donner et \textbf{démontrer} l'équation
        différentielle sur $x$ ou $\ell$ sous forme canonique \textbf{qu'on ne
        cherchera pas à résoudre}, vérifier son homogénéité, présenter les
        graphiques des solutions selon les valeurs de $Q$ dans l'espace temporel
        \textbf{et} dans l'espace des phases ($u_C$, $i$) en donnant une
        approximation de la durée du régime transitoire à 95\%.

    \item Faire l'analogie complète entre les deux systèmes amortis RLC libre et
        ressort avec frottement fluide~: présentation, conditions initiales,
        équations différentielles \textbf{sans démonstration}, correspondance
        entre les grandeurs, tracer de solutions \textbf{dans l'espace des
        phases} selon différentes valeurs de $Q$ sans résolution et commenter
        sur l'évolution de l'énergie visible dans le graphique.

    \item Faire l'étude énergétique de l'oscillateur amorti électrique RLC libre
        et du ressort horizontal avec frottement fluide, identifier les termes
        du bilan et expliciter la signification physique de chacun des termes.
\end{enumerate}

% \section{Consignes}
% \begin{enumerate}
% 
%     \item \textbf{Les relations de conjugaison \underline{sont} à connaître}.
%     \item Une question de cours non connue entraîne un 0 à cette partie (note
%         maximale 10/20 si exercice parfait)~;
%     \item \textbf{Les schémas des questions de cours sont obligatoires~: s'ils
%         manquent, la question ne saurait être notée au-dessus de 5}~;
%     \item Chacune des règles suivantes qui ne serait respectée enlèvera
%         \textbf{un  point}~:
%         \begin{enumerate}
%             \item Les schémas optiques doivent comporter le sens de comptage
%                 algébrique des distances et des angles~;
%             \item Les rayons lumineux doivent avoir un sens de propagation~;
%             \item Les angles doivent être orientés.
%             \item Tous les dipôles doivent être fléchés en courant et tension
%                 sur les schémas.
%         \end{enumerate}
% \end{enumerate}

\end{document}

