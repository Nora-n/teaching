\documentclass[a4paper, 12pt, final, garamond]{book}
\usepackage{cours-preambule}

\raggedbottom

\makeatletter
\renewcommand{\@chapapp}{Programme de kh\^olle -- semaine}
\makeatother

\begin{document}
\setcounter{chapter}{8}

\chapter{Du 27 au 30 novembre}

\section{Cours et exercices}
\subsection{Chimie chapitre 2 -- Transformation et équilibre chimique}
\begin{enumerate}[label=\Roman*]
	\bitem{Avancement d'une réaction}~: présentation, avancements molaire
	et volumique, tableau d'avancement, coefficients stœchiométriques
	algébriques.
	\bitem{États final et d'équilibre d'un système chimique}~: réactions
	totales et limitées et exercice d'application, quantifications de
	l'avancement~: taux de conversion, coefficient de dissociation,
	rendement~; quotient de réaction et exercice d'application, constante
	d'équilibre et exercice d'application, réactions quasi-nulles et
	quasi-totales.
	\bitem{Évolution d'un système chimique}~: quotient réactionnel et
	évolution et exercice d'application, cas des ruptures d'équilibre,
	résumé pratique de résolution.
\end{enumerate}

\subsection{Chimie chapitre 3 -- Cinétique chimique}
\begin{enumerate}[label=\Roman*]
	\bitem{Introduction}~: réactions lentes et rapides, méthodes de
	suivi, exemple de suivi cinétique, facteurs cinétiques.
	\bitem{Vitesse(s) de réaction}~: hypothèses de travail, vitesse de
	réaction, vitesses de formation/disparition.
	\bitem{Concentration et ordre de réaction}~: ordre d'une réaction,
	ordre initial et courant, cas particulier des réactions simples loi de
	\textsc{Van't Hoff}, cas particulier dégénérescence de l'ordre et
	proportions stœchiométriques.
	\bitem{Méthodes de résolution}~: temps de demi-réaction, ordres 0, 1
	et 2 par rapport à un réactif~: hypothèse de départ, unité de $k$,
	équation différentielle, résolution et $t_{1/2}$~; résumé méthodes en
	pratique et résumé.
	\bitem{Température et loi d'\textsc{Arrhénius}}~: phénoménologie, expression
	de $k(T)$, exemple d'utilisation pour deux températures et pour une
	sucession de températures
	\bitem{Méthodes de suivi cinétique expérimental}~: dosage par
	titrage et trempe chimique, dosage par étalonnage~: loi de
	\textsc{Beer-Lambert} et loi de \textsc{Kohlrausch}.
\end{enumerate}

\section{Questions de cours possibles}
\subsection{Chapitre~2}
\begin{enumerate}
	% \item[] \textbf{Chapitre 2}
	\item Réaction et avancement~: \textbf{définir le taux de conversion,
		      le coefficient de dissociation et le rendement} et refaire l'exemple
	      du cours sur la combustion totale du méthane $\ce{CH4\gaz{} +
			      2O2\gaz{} \rightarrow CO2\gaz{} + 2H2O\gaz{}}$ avec $n_{\ce{CH4}}^0 =
		      \SI{2}{mol}$ et $n_{\ce{O2}}^0 = \SI{3}{mol}$.
	\item Donner les différentes expressions de l'activité d'un constituant
	      selon sa nature, exprimer le quotient de réaction d'une équation-bilan
	      générale $0=\sum_i \nu_i\ce{X}_i$ ou $\alpha_1\ce{R}_1 + \alpha_2{\rm
			      R}_2 + … = \beta_1\ce{P}_1 + \beta_2\ce{P}_2 + …$ et la constante
	      d'équilibre associée, et exprimer $Q_r$ pour les réactions~:
	      \begin{enumerate}
		      \item $\ce{2I^-\aqu{} + S2O8^{2-}\aqu{} = I2\aqu{} +2SO4^{2-}\aqu}$
		      \item $\ce{Ag+\aqu{} + Cl^-\aqu{} = AgCl\sol}$
		      \item $\ce{2FeCl3\gaz{} = Fe2Cl6\gaz{}}$
	      \end{enumerate}
	\item Soit la réaction de l'acide éthanoïque avec l'eau~:
	      \[\ce{CH3COOH\aqu{} + H2O\liq{} = CH3COO^{-}\aqu{} + H3O^{+}\aqu{}}\]
	      de constante $K = \num{1.78e-5}$. On introduit $c =
		      \SI{1.0e-1}{mol.L^{-1}}$ d'acide éthanoïque et on note $V$ le volume de
	      solution. \textbf{Déterminer la composition à l'état final}.

	\item Indiquer comment prévoir le sens d'évolution d'un système.
	      Soit la synthèse de l'ammoniac~:
	      \[
		      \ce{N2\gaz{} + 3H2\gaz{} = 2NH3\gaz{}}
		      \tag*{$K = \num{0.5}$}
	      \]
	      On introduit \SI{3}{mol} de diazote, \SI{5}{mol} de dihydrogène et
	      \SI{2}{mol} d'ammoniac sous une pression de \SI{200}{bars}.
	      \textbf{Déterminer les pressions partielles des gaz} et \textbf{indiquer
		      dans quel sens se produit la réaction}.
	\item Considérons la dissolution du chlorure de sodium, de masse molaire
	      $M(\ce{NaCl}) = \SI{58.44}{g.mol^{-1}}$~:
	      \[
		      \ce{NaCl\sol{} = Na^{+}\aqu{} + Cl^{-}\aqu{}}
		      \tag*{$K=33$}
	      \]
	      On introduit \SI{2.0}{g} de sel dans \SI{100}{mL} d'eau.
	      \textbf{Déterminer l'état d'équilibre}.

	\item Déterminer, à l'aide d'un tableau d'avancement, la \textbf{composition à
		      l'état final} de la réaction totale de la combustion de \SI{2.00}{mol}
	      d'éthanol dans l'air. On précise que les réactifs sont introduits dans
	      les proportions stœchiométriques, et que le dioxygène provient de
	      l'air (20\% \ce{O2} et 80\% \ce{N2} en mole). Quelle est la
	      \textbf{pression finale} pour $V = \SI{1.00}{m^3}$ et $T =
		      \SI{293}{K}$, $R = \SI{8.314}{J.K^{-1}.mol^{-1}}$~?
	      \[
		      \ce{C_2H_5OH\liq{} + 3O_2\gaz{} -> 2CO_2\gaz{} + 3H_2O\gaz{}}
	      \]
\end{enumerate}
\subsection{Chapitre~3}

\begin{enumerate}[resume]
	% \item[] \textbf{Chapitre 3}
	\item Définir la vitesse d'une réaction, de formation d'un produit, de
	      disparition d'un réactif et le lien entre vitesse de réaction et
	      variation de la concentration d'un constituant en fonction de son nombre
	      stœchiométrique algébrique, puis exprimer $v$ en fonction des
	      concentrations pour la réaction
	      \[
		      \ce{6H^{+}\aqu{} + 5Br^{-}\aqu{} + BrO3^{-}\aqu{}
			      =
			      3Br2\aqu{} + 2H2O\liq{}}
	      \]
	\item Donner la loi de vitesse d'une réaction $a\ce{A} + b\ce{B} = c\ce{C}
		      + d\ce{D}$ admettant un ordre, la loi de vitesse de la même réaction
	      si elle est simple, montrer l'intérêt de la dégénérescence de l'ordre
	      et des proportions stœchiométriques.
	\item À partir d'une loi de vitesse d'ordre \textbf{choisi par
		      l'interrogataire} par rapport à un unique réactif $[\ce{A}]$, donner
	      l'unité de $k$, démontrez l'équation différentielle vérifiée par
	      $[\ce{A}]$ et la solution associée, indiquer quelle régression linéaire
	      pourrait permettre de vérifier cette loi et donner le temps de
	      demi-réaction.
	\item Énoncer la loi d'\textsc{Arrhénius}, indiquer une manière d'utiliser
	      deux constantes de vitesse à deux températures différentes pour
	      déterminer l'énergie d'activation, et une autre manière d'utiliser
	      plusieurs constantes de vitesse à différentes températures pour
	      déterminer l'énergie d'activation.
\end{enumerate}
\end{document}
