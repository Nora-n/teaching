\documentclass[a4paper, 12pt, final, garamond]{book}
\usepackage{cours-preambule}

\raggedbottom

\makeatletter
\renewcommand{\@chapapp}{Programme de kh\^olle -- semaine}
\makeatother

\begin{document}
\setcounter{chapter}{9}

\chapter{Du 28 novembre au 02 d\'ecembre}

\section{Cours et exercices}
\section*{Chimie chapitre 2 -- Transformation et équilibre chimique}
\begin{enumerate}[label=\Roman*]
    \item \textbf{Avancement d'une réaction}~: présentation, avancements molaire
        et volumique, tableau d'avancement, coefficients stœchiométriques
        algébriques.
    \item \textbf{État d'équilibre et final d'un système chimique}~: réactions
        totales et limitées et exercice d'application, quantifications de
        l'avancement~: taux de conversion, coefficient de dissociation,
        rendement~; quotient de réaction et exercice d'application, constante
        d'équilibre et exercice d'application, réactions quasi-nulles et
        quasi-totales.
    \item \textbf{Évolution d'un système chimique}~: quotient réactionnel et
        évolution et exercice d'application, cas des ruptures d'équilibre,
        résumé pratique de résolution.
\end{enumerate}

\section*{Chimie chapitre 3 -- Cinétique chimique}
\begin{enumerate}[label=\Roman*]
    \item \textbf{Introduction}~: réactions lentes et rapides, méthodes de
        suivi, exemple de suivi cinétique, facteurs cinétiques.
    \item \textbf{Vitesse(s) de réaction}~: hypothèses de travail, vitesse de
        réaction, vitesses de formation/disparition.
    \item \textbf{Concentration et ordre de réaction}~: ordre d'une réaction,
        ordre initial et courant, cas particulier des réactions simples loi de
        \textsc{Van't Hoff}, cas particulier dégénérescence de l'ordre et
        proportions stœchiométriques.
    \item \textbf{Méthodes de résolution}~: temps de demi-réaction, ordres 0, 1
        et 2 par rapport à un réactif~: hypothèse de départ, unité de $k$,
        équation différentielle, résolution et $t_{1/2}$~; résumé méthodes en
        pratique et résumé.
    \item \textbf{Température et loi d'\textsc{Arrhénius}}~: expression de
        $k(T)$, exemple d'utilisation pour $k(T_1)$ et $k(T_2)$.
    \item \textbf{Méthodes de suivi cinétique expérimental}~: dosage par
        titrage et trempe chimique, dosage par étalonnage~: loi de
        \textsc{Beer-Lambert} et loi de \textsc{Kohlrausch}.
\end{enumerate}

\section{Questions de cours possibles}
\begin{enumerate}
    \item Donner les différentes expressions de l'activité d'un constituant
        selon sa nature, exprimer le quotient de réaction d'une équation-bilan
        générale $0=\sum_i \nu_i{\rm X}_i$ ou $\alpha_1{\rm R}_1 + \alpha_2{\rm
        R}_2 + … = \beta_1{\rm P}_1 + \beta_2{\rm P}_2 + …$ et la constante
        d'équilibre associée, et exprimer $Q_r$ pour les réactions~:
        \begin{enumerate}
            \item $\ce{2I^-\aqu{} + S2O8^{2-}\aqu{} = I2\aqu{} +2SO4^{2-}\aqu}$
            \item $\ce{Ag+\aqu{} + Cl^-\aqu{} = AgCl\sol}$
            \item $\ce{2FeCl3\gaz{} = Fe2Cl6\gaz{}}$
        \end{enumerate}
    \item Refaire l'exercice d'application de la réaction de l'acide éthanoïque
        avec l'eau~:
\end{enumerate}
\begin{NCexem}[width=\linewidth, breakable]{Exercice}
    Soit la réaction de l'acide éthanoïque avec l'eau~:
    \[\ce{CH3COOH\aqu{} + H2O\liq{} = CH3COO\moin{}\aqu{} + H3O\plus{}\aqu{}}\]
    de constante $K = \num{1.78e-5}$. On introduit $c = \SI{1.0e-1}{mol.L^{-1}}$
    d'acide éthanoïque et on note $V$ le volume de solution. \textbf{Déterminer
    la composition à l'état final}.
\end{NCexem}

\begin{enumerate}[resume]
    \item Indiquer comment prévoir le sens d'évolution d'un système, et refaire
        l'exercice~:
\end{enumerate}
\begin{NCexem}[width=\linewidth, breakable]{Exercice}
    Soit la synthèse de l'ammoniac~:

    \centersright{$\ce{N2\gaz{} + 3H2\gaz{} = 2NH3\gaz{}}$}{$K = \num{0.5}$}

    On introduit \SI{3}{mol} de diazote, \SI{5}{mol} de dihydrogène et
    \SI{2}{mol} d'ammoniac sous une pression de \SI{200}{bars}.
    \textbf{Déterminer les pressions partielles des gaz} et \textbf{indiquer
    dans quel sens se produit la réaction}.
\end{NCexem}
\begin{enumerate}[resume]
    \item Rupture d'équilibre~: refaire l'exercice d'application~:
\end{enumerate}
\begin{NCexem}[width=\linewidth]{Exercice}

    Considérons la dissolution du chlorure de sodium, de masse molaire
    $M(\ce{NaCl}) = \SI{58.44}{g.mol^{-1}}$~:

    \centersright{$\ce{NaCl\sol{} = Na\plus{}\aqu{} + Cl\moin{}\aqu{}}$}{$K=33$}

    On introduit \SI{2.0}{g} de sel dans \SI{100}{mL} d'eau. \textbf{Déterminer
    l'état d'équilibre}.
\end{NCexem}
\begin{enumerate}[resume]
    \item Définir la vitesse d'une réaction, de formation d'un produit, de
        disparition d'un réactif et le lien entre vitesse de réaction et
        variation de la concentration d'un constituant en fonction de son nombre
        stœchiométrique algébrique, puis exprimer $v$ en fonction des
        concentrations pour la réaction
        \[
            \ce{6H\plus{}\aqu{} + 5Br\moin{}\aqu{} + BrO3\moin{}\aqu{}
            =
            3Br2\aqu{} + 2H2O\liq{}}
        \]
    \item Donner la loi de vitesse d'une réaction $a\rm{A} + b\rm{B} = c\rm{C}
        + d\rm{D}$ admettant un ordre, la loi de vitesse de la même réaction si
        elle est simple, montrer l'intérêt de la dégénérescence de l'ordre et
        des proportions stœchiométriques.
    \item À partir d'une loi de vitesse d'ordre 0 par rapport à un unique
        réactif $[{\rm A}]$, donner l'unité de $k$, démontrez l'équation
        différentielle vérifiée par $[{\rm A}]$ et la solution associée,
        indiquer quelle régression linéaire pourrait permettre de vérifier cette
        loi et donner le temps de demi-réaction.
    \item À partir d'une loi de vitesse d'ordre 1 par rapport à un unique
        réactif $[{\rm A}]$, donner l'unité de $k$, démontrez l'équation
        différentielle vérifiée par $[{\rm A}]$ et la solution associée,
        indiquer quelle régression linéaire pourrait permettre de vérifier cette
        loi et donner le temps de demi-réaction.
    \item À partir d'une loi de vitesse d'ordre 2 par rapport à un unique
        réactif $[{\rm A}]$, donner l'unité de $k$, démontrez l'équation
        différentielle vérifiée par $[{\rm A}]$ et la solution associée,
        indiquer quelle régression linéaire pourrait permettre de vérifier cette
    \item Énoncer la loi d'\textsc{Arrhénius}, indiquer une manière d'utiliser
        deux constantes de vitesse à deux températures différentes pour
        déterminer l'énergie d'activation, et une autre manière d'utiliser
        plusieurs constantes de vitesse à différentes températures pour
        déterminer l'énergie d'activation.
\end{enumerate}
\end{document}
