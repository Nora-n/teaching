\documentclass[a4paper, 12pt, final, garamond]{book}
\usepackage{cours-preambule}
\graphicspath{{../figures/}}

\raggedbottom

\makeatletter
\renewcommand{\@chapapp}{Programme de kh\^olle -- semaine}
\makeatother

\begin{document}
\setcounter{chapter}{13}

\chapter{Du 15 au 18 janvier}

\section{Exercices uniquement}
\ssubsection{E7}{Filtrage linéaire}
\ssubsection{ON1}{Ondes progressives}
\section{Cours et exercices}
\ssubsection{ON2}{Interférences à deux ondes}
\begin{enumerate}[label=\Roman*]
	\bitem{Rappel déphasages}~: définition, valeurs particulières,
	lecture graphique.
	\bitem{Superposition d'ondes sinusoïdales de mêmes fréquences}~:
	introduction, signaux de même amplitude, signaux d'amplitudes
	différentes, bilan.
	\bitem{Approximation par une onde plane}~: sources ponctuelles,
	différence de marche, exercice d'application.
	\bitem{Interférences lumineuses}~: cohérence, intensité, formule de
	\textsc{Fresnel}, chemin optique.
	\bitem{Expérience des trous d'\textsc{Young}}~: introduction,
	présentation, détermination de l'interfrange.
\end{enumerate}

\section{Cours uniquement}
\ssubsection{M1}{Cinématique du point}
\begin{enumerate}[label=\Roman*]
	\bitem{Système et point matériel}~: définition système, point
	matériel.
	\bitem{Description et paramétrage du mouvement}~: notion de
	référentiel, relativité du mouvement, exemples de référentiels, vecteur
	base de projection et repère, projection de vecteurs.
	\bitem{Position, vitesse et accélération}~: position et déplacement
	élémentaire, équations horaires et trajectoires~; vitesse et vitesse
	instantanée, notation pointée~; accélération et accélération
	instantanée.
	\bitem{Exemples de mouvements}~: rectiligne uniforme, rectiligne
	uniformément accéléré, courbe uniformément accéléré.
\end{enumerate}

\ssubsection{M2}{Dynamique du point}
\begin{enumerate}[label=\Roman*]
	\bitem{Introduction}~: inertie et quantité de mouvement, forces
	fondamentales.
	\bitem{Trois lois de \textsc{Newton}}~: principe d'inertie, principe
	fondamental de la mécanique, loi des actions réciproques.
	\bitem{Ensembles de points}~: centre d'inertie, quantité de mouvement
	d'un ensemble de points, théorème de la résultante cinétique, méthode
	générale de résolution.
	\bitem{Forces usuelles}~: poids, chute libre avec angle initial~; poussé
	d'\textsc{Archimède}~; frottements fluides, chute libre avec frottements
	linéaires et quadratique, résolution par adimensionnement~; frottements
	solides~; force de rappel d'un ressort et longueur d'équilibre vertical.
\end{enumerate}

\section{Questions de cours possibles}
\ssubsection{ON2}{Interférences à deux ondes}
\begin{enumerate}
	% \litem{\strr}%
	% Déterminer l'expression du signal somme de deux ondes sinusoïdales de
	% même fréquence \textbf{et même amplitude} en introduisant $\Delta
	% 	\f_{1/2}(\Mr)$ et $\f_0(\Mr)$. On rappelle la formule de trigonométrie
	% \[
	% 	\cos p + \cos q =
	% 	2\cos \left( \frac{p+q}{2} \right)\cos \left( \frac{p-q}{2} \right)
	% \]
	% Décrire le signal obtenu. Détailler les
	% cas extrêmes et les valeurs de déphasage correspondantes (on utilisera
	% l'ordre d'interférence et non la congruence). Qu'est-ce qui change si
	% les signaux n'ont pas la même amplitude~? Définir les termes
	% d'interférences constructives et destructives.
	%
	% \litem{\strrr}%
	% Déterminer l'expression du signal somme de deux ondes sinusoïdales de
	% même fréquence \textbf{et d'amplitudes différentes} en introduisant $\Delta
	% 	\f_{1/2}(\Mr)$ et $\f_0(\Mr)$. On rappelle les formules de trigonométrie
	% \smallbreak
	% \begin{isd}
	% 	\vspace*{-15pt}
	% 	\begin{align*}
	% 		\cos(a+b) & = \cos a\cos b - \sin a\sin b \\
	% 		\cos(a-b) & = \cos a\cos b + \sin a\sin b
	% 	\end{align*}
	% 	\tcblower
	% 	\vspace*{-15pt}
	% 	\begin{gather*}
	% 		\cos\theta = \frac{\exr^{\jj\theta}+\exr^{-\jj\theta}}{2}\\
	% 		\abs{\zu}^2 = \zu\times\zu^*
	% 		\qet
	% 		\tan\arg*{\zu} = \frac{\Im(\zu)}{\Re(\zu)}
	% 	\end{gather*}
	% \end{isd}

	\litem{\strr}%
	Démontrer le lien entre déphasage et différence de marche en prenant
	l'exemple de deux sources dont les ondes se superposent en un point $M$ dans
	l'approximation des ondes planes. Expliquer avec vos mots ce que représente
	la différence de marche.
	\smallbreak
	Démontrer les valeurs de différence de marche
	correspondant aux situations de signaux en phase et en opposition de phase
	pour $\Delta\f_0 = 0$.
	\smallbreak
	Définir et expliquer ce qu'est le chemin optique d'un rayon lumineux, et
	donner le lien entre entre déphasage et chemin optique.

	\litem{\str}%
	Soient 2 émetteurs sonores envoyant une onde progressive
	sinusoïdale de même fréquence, amplitude et phase à l'origine. Le premier
	est fixé à l'origine du repère, l'émetteur 2 est mobile et à une distance
	$d$ du premier, et un microphone est placé à une distance fixe $x_0$ de
	l'émetteur 1 et est aligné avec les deux émetteurs. On néglige l'influence
	de l'émetteur 2 sur l'émetteur 1 et toute atténuation.
	\begin{enumerate}[label=\sqenumi]
		\item Faire un schéma.
		\item Lorsque $d=0$, qu'enregistre-t-on au niveau du microphone~?
		\item On part de $d=0$ et on augmente $d$ jusqu'à ce que le signal
		      enregistré soit nul. Ceci se produit pour $d = \SI{6.0}{cm}$.
		      Expliquer cette extinction.
		\item En déduire la longueur d'onde du son émis.
		\item Pour $d = \SI{12.0}{cm}$, quelle sera l'amplitude du signal
		      enregistré~?
	\end{enumerate}
	\litem{\strr}%
	Expliquer ce qu'est la cohérence et pourquoi on ne fait des
	interférences qu'avec une unique source pour des signaux lumineux. Définir
	ce qu'est l'intensité d'un signal. Démontrer la formule de \textsc{Fresnel}
	pour deux signaux sinusoïdaux de même fréquence et d'amplitudes différentes.
	On supposera connue l'amplitude de la somme~:
	\[
		A = \sqrt{A_1{}^{2} + A_2{}^{2} + 2A_1A_2 \cos(\Delta{\f}_{1/2}(\Mr))}
	\]
	La simplifier pour des signaux de même amplitude.

	\litem{\strr}%
	Trous d'\textsc{Young}~: présenter l'expérience et montrer que la
	différence de chemin $\delta_{2/1}(\Mr)$ s'écrit \hfill $\delta =
		2ax/D$ avec $2a$ la distance entre les fentes. Donner les conditions sur
	$x$ pour avoir interférences constructives ou destructives.
	\smallbreak
	On donne le développement limité suivant~:
	\[\sqrt{1+\ep} = 1 + \ep/2 + o(\ep)\]
\end{enumerate}

\ssubsection{M1}{Cinématique du point}
\begin{enumerate}[resume]
	\litem{\str}%
	Déterminer les équations horaires du mouvement rectiligne uniformément
	accéléré. Un attention particulière sera portée à l'établissement du
	système d'étude.
\end{enumerate}
\ssubsection{M2}{Dynamique du point}
\begin{enumerate}[resume]
	\litem{\strr}%
	Énoncer les trois lois de \textsc{Newton}, définir le centre d'inertie
	d'un ensemble de points, le vecteur quantité de mouvement d'un
	ensemble de points et son lien avec le centre d'inertie, énoncer et
	démontrer le théorème de la résultante cinétique.

	\litem{\strrr}%
	Déterminer les \textbf{équations horaires} ainsi que la
	\textbf{trajectoire} du lancer d'une masse avec une vitesse initiale
	$\vf_0$ faisant un angle $\alpha$ avec l'horizontale. Une attention
	particulière sera portée à l'établissement du système d'étude.
	Déterminer alors la portée, la flèche du tir ainsi que le temps de vol,
	au choix (potentiellement multiple) de l'interrogataire.

	\litem{\str}%
	Déterminer la proportion immergée d'un glaçon. On donne $\rho_{\rm
			eau} = \SI{1.00e3}{km.m^{-3}}$ et $\rho_{\rm glace} =
		\SI{9.17e2}{kg.m^{-3}}$.

	\litem{\strr}%
	Déterminer la vitesse limite et le temps caractéristique du
	mouvement pour une chute libre sans vitesse initiale avec frottements
	\textbf{linéaires}. Les approches d'adimensionnement d'équation différentielle,
	de solution particulière ou de résolution totale directe sont possibles.

	\litem{\strr}%
	Déterminer la vitesse limite et le temps caractéristique du
	mouvement pour une chute libre sans vitesse initiale avec frottements
	\textbf{quadratiques} par une approche d'adimensionnement.
\end{enumerate}

\end{document}
