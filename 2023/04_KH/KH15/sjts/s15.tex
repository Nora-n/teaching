\documentclass[a4paper, 11pt]{book}
\usepackage{/home/nicolas/Documents/Enseignement/Prepa/bpep/fichiers_utiles/preambule}

\newcommand{\dsNB}{15}
\makeatletter
\renewcommand{\@chapapp}{Kh\^olles MPSI3 -- semaine \dsNB}
\makeatother

% \toggletrue{corrige}  % décommenter pour passer en mode corrigé

 % IMPORTS automatiques
\newcommand{\f}[2]{{
		\mathchoice
		{\dfrac{#1}{#2}}
		{\dfrac{#1}{#2}}
		{\frac{#1}{#2}}
		{\frac{#1}{#2}}
}}

\newcommand{\e}[1]{{}_{\text{#1}}}

 % fin des IMPORTS automatiques

\begin{document}

\chapter{Sujet 1\siCorrige{\!\!-- corrig\'e}}
\section{Question de cours}

Déterminer les équations horaires du mouvement courbe uniformément accéléré avec
$\vv{v}_0$ faisant un angle $\alpha$ avec l’horizontale. Une attention particulière
sera portée à l’établissement du système d’étude.

\resetQ
\subimport{/home/nicolas/Documents/Enseignement/Prepa/bpep/exercices/TD/interference_corde_de_melde/}{sujet.tex}

\chapter{Sujet 2\siCorrige{\!\!-- corrig\'e}}
\section{Question de cours}
Trous d'\textsc{Young}~: présenter l'expérience et montrer que la différence de
chemin $\delta_{2/1}({\rm M})$ s'écrit $\delta = 2ax/D$ avec $2a$ la distance
entre les fentes. Donner les conditions sur $x$ pour avoir interférences
constructives et destructives.

\resetQ
\subimport{/home/nicolas/Documents/Enseignement/Prepa/bpep/exercices/TD/ecoute_musicale/}{sujet.tex}


\chapter{Sujet 3\siCorrige{\!\!-- corrig\'e}}
\section{Question de cours}

Déterminer la vitesse limite et le temps caractéristique du mouvement pour une
chute libre sans vitesse initiale avec frottements linéaires. Une approche
d’adimensionnement d’équation différentielle, de solution particulière ou de
résolution totale directe est possible.

\resetQ
\subimport{/home/nicolas/Documents/Enseignement/Prepa/bpep/exercices/Colle/miroir_de_lloyd/}{sujet.tex}

\label{LastPage}
\end{document}
