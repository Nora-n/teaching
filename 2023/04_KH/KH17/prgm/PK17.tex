\documentclass[a4paper, 12pt, final, garamond]{book}
\usepackage{cours-preambule}

\raggedbottom

\makeatletter
\renewcommand{\@chapapp}{Programme de kh\^olle -- semaine}
\makeatother

\begin{document}
\setcounter{chapter}{16}

\chapter{Du 05 au 08 f\'evrier}

\section{Exercices uniquement}
\ssubsection{M2}{Dynamique du point}
% \begin{enumerate}[label=\Roman*]
% 	\bitem{Introduction}~: inertie et quantité de mouvement, forces
% 	fondamentales.
% 	\bitem{Trois lois de \textsc{Newton}}~: principe d'inertie, principe
% 	fondamental de la mécanique, loi des actions réciproques.
% 	\bitem{Ensembles de points}~: centre d'inertie, quantité de mouvement
% 	d'un ensemble de points, théorème de la résultante cinétique, méthode
% 	générale de résolution.
% 	\bitem{Forces usuelles}~: poids, chute libre avec angle initial~; poussé
% 	d'\textsc{Archimède}~; frottements fluides, chute libre avec frottements
% 	linéaires et quadratique, résolution par adimensionnement~; frottements
% 	solides~; force de rappel d'un ressort et longueur d'équilibre vertical.
% \end{enumerate}

\ssubsection{M3}{Mouvements courbes}
% \begin{enumerate}[label=\Roman*]
% 	\bitem{Mouvement courbe dans le plan}~: position, vitesse,
% 	déplacement élémentaire, accélération en coordonnées polaires.
% 	\bitem{Exemples de mouvements plans}~: mouvement circulaire,
% 	circulaire uniforme, repère de \textsc{Frenet}.
% 	\bitem{Application~: pendule simple}~: tension d'un fil, pendule simple.
% 	\bitem{Mouvement courbe dans l'espace}~: coordonnées cylindriques,
% 	coordonnées sphériques.
% \end{enumerate}

\section{Cours et exercices}
\ssubsection{M4}{Approche énergétique}
\begin{enumerate}[label=\Roman*]
	\bitem{Notions énergétiques}~: énergie, conservation, puissance.
	\bitem{Énergie cinétique et travail d'une force constante}~: définitions,
	exemples, travail du poids.
	\bitem{Puissance d'une force et travail élémentaire}~: définitions, TPC,
	TEC, et applications, comment choisir~?
	\bitem{Énergie potentielle}~: forces conservatives ou
	non, travail d'une force conservative, gradient d'une fonction scalaire,
	opérateur différentiel, lien à l'énergie potentielle
	\bitem{Énergie mécanique}~: définition, TEM et TPM et applications.
	\bitem{Énergie potentielle et équilibres}~: notion d'équilibre, lien
	avec $\Ec_p$, équilibres stables et instables, lien avec
	$\dv[2]{\Ec_p}{x}$ , étude générale autour d'un point d'équilibre
	stable~: oscillateur harmonique.
	\bitem{Énergie potentielle et trajectoire}~: détermination
	qualitative d'une trajectoire, état lié et diffusion~; cas du pendule
	simple, étude mouvement selon $\Ec_p$ et $\Ec_m$.
\end{enumerate}

\section{Cours uniquement}
\ssubsection{M5}{Mouvement de particules chargées}
\begin{enumerate}[label=\Roman*]
	\bitem{Champs électrique et magnétique}~: définitions, exemples condensateur
	et bobine.
	\bitem{Force de \textsc{Lorentz}}~: définition, comparaison au poids,
	remarque produit vectoriel, puissance de la force de \textsc{Lorentz},
	potentiel électrostatique.
	\bitem{Mouvement dans un champ électrique}~: situation générale, accélération
	pour $\vfo\parr\Ef$, déviation pour $\vfo\perp\Ef$, angle de déviation,
	applications (accélérateur linéaire, oscilloscope analogique).
	\bitem{Mouvement dans un champ magnétique}~: mise en équation, cas
	$\vfo\parr\Bf$, cas $\vfo\perp\Bf$~: trajectoire et équations horaires
	cyclotron~; cas général (mouvement hélicoïdal), applications
	(spectromètre de masse, cyclotron, effet \textsc{Hall})
\end{enumerate}

\newpage
\section{Questions de cours possibles}
\ssubsection{M4}{Approche énergétique}
\begin{enumerate}
	\item Énoncer et démontrer les théorèmes de la puissance cinétique et de
	      l'énergie cinétique. Appliquer le TEC pour trouver la vitesse d'une
	      skieuse en bas d'une piste d'un dénivelé de hauteur $h$. On néglige les
	      frottements.
	\item Retrouver les énergies potentielles de forces classiques (poids,
	      rappel élastique). Comment trouver une force à partir à partir de son
	      énergie potentielle~?
	\item Retrouver l'équation différentielle sur $\tt$ du pendule simple non
	      amorti à l'aide du TPC.
	\item Énoncer et démontrer les théorèmes de la puissance mécanique et de
	      l'énergie mécanique.
	\item Savoir discuter le mouvement d'une particule en comparant son profil
	      d'énergie potentielle et son énergie mécanique~; état lié ou de
	      diffusion. Expliquer l'obtention des positions d'équilibre et leur
	      stabilité sur un graphique $\Ec_p(x)$. Démontrer l'équilibre et sa
	      stabilité en terme de conditions sur la dérivée première et seconde de
	      l'énergie potentielle.
	\item Savoir réaliser l'approximation harmonique d'une cuvette de potentiel
	      par développement limité. En déduire que tout système décrit par une
	      énergie potentielle présentant un minimum local est assimilable à un
	      oscillateur harmonique.
\end{enumerate}
\ssubsection{M5}{Mouvement de particules chargées}
\begin{enumerate}[resume]
	\item Définir la force de \textsc{Lorentz}~; comparer les ordres de
	      grandeurs des forces électriques et magnétiques au poids~; déterminer la
	      puissance de la force de \textsc{Lorentz} et discuter des conséquences.
	      Démontrer qu'elle est conservative et déterminer l'expression de
	      l'énergie potentielle associée.
	\item Action de $\Ef$ uniforme entre deux grilles chargées sur une particule
	      chargée avec $\vfo\parr\Ef$~: présenter la situation, faire un bilan
	      énergétique pour calculer la vitesse de sortie en fonction de la
	      différence de potentiel $U$.
	\item Action de $\Ef$ uniforme entre deux grilles chargées sur une particule
	      chargée avec $\vfo\perp\Ef$~: présenter la situation, déterminer le
	      temps de vol et l'angle de déviation en fonction de $U$.
	\item Action de $\Bf$ uniforme sur une particule chargée avec
	      $\vfo\perp\Bf$~: présenter la situation, et prouver que le mouvement est
	      uniforme, plan et circulaire. On déterminera l'équation de la
	      trajectoire en introduisant le rayon et la pulsation cyclotron, ainsi
	      que les équations scalaires.
\end{enumerate}

\end{document}
