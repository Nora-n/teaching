\documentclass[a4paper, 12pt, final, garamond]{book}
\usepackage{cours-preambule}

\raggedbottom

\makeatletter
\renewcommand{\@chapapp}{Programme de kh\^olle -- semaine}
\makeatother

\begin{document}
\setcounter{chapter}{18}

\chapter{Du 19 au 23 f\'evrier}

\section{Exercices uniquement}
\ssubsection{M5}{Mouvement de particules chargées}

\section{Cours et exercices}
\ssubsection{AM1}{Structure des entités chimiques}
\begin{enumerate}[label=\Roman*]
	\bitem{Niveaux d'énergie d'un électron dans un atome}~: nombres quantiques et
	orbitales atomiques (HP), niveaux d'énergie (HP), électrons de cœur et de
	valence.
	\bitem{Tableau périodique}~: construction et blocs, analyse par période,
	analyse par famille.
	\bitem{Structure électronique des molécules}~: représentation de
	\textsc{Lewis} des atomes, liaison covalente, notation de \textsc{Lewis}
	des molécules, écarts à la règle de l'octet.
	\bitem{Géométrie et polarité des entités chimiques}~: modèle VSEPR, polarité
	des liaisons et des molécules, polarisabilité.
\end{enumerate}

\ssubsection{AM2}{Propriétés physico-chimiques macroscopiques}
\begin{enumerate}[label=\Roman*]
	\bitem{Interactions de \textsc{Van der Waals}}~: \textsc{Keesom}
	permanent/permanent, \textsc{Debye} permanent/induit, \textsc{London}
	induit/induit, bilan et remarque répulsion.
	\bitem{Températures de changement d'état}~: influence du moment dipolaire,
	influence de la polarisabilité.
	\bitem{Liaison hydrogène}~: introduction expérimentale, définition et
	exemples.
	\bitem{Solubilité, miscibilité}~: classement des solvants, solubilité, mise
	en solution d'espèces ioniques, miscibilité.
\end{enumerate}

\section{Cours uniquement}
\ssubsection{M6}{Moment cinétique d'un point matériel}
\begin{enumerate}[label=\Roman*]
	\bitem{Moment d'une force}~: par rapport à un point, définition et exemples~;
	par rapport à un axe orienté~: définition et exemples~; bras de levier
	d'une force~: propriété, méthode et application~; exemples de calcul de
	moments.
	\bitem{Moment cinétique}~: par rapport à un point, définition et exemples~;
	par rapport à un axe orienté, définition et exemples.
	\bitem{Théorème du moment cinétique}~: par rapport à un point fixe, énoncé et
	démonstration~; par rapport à un axe orienté fixe~: énoncé et
	démonstration.
	\bitem{Exemple du pendule simple}~: équation du mouvement par TMC, avec et
	sans bras de levier.
\end{enumerate}

\newpage
\section{Questions de cours possibles}
\ssubsection{AM1}{Structure des entités chimiques}
\begin{enumerate}
	\item Savoir comment construire (pas connaître par cœur) les 4 premières
	      lignes du tableau périodique. Définir et placer les blocs $s$, $p$ et
	      $d$. Préciser les colonnes des familles des gaz rares, des halogènes,
	      des métaux alcalins et alcalino-terreux. Placer les métaux et
	      non-métaux. Placer un élément ($Z \leq 36$) sur le tableau à partir de
	      son numéro atomique \textbf{\xul{et/ou}} déterminer son numéro atomique
	      à partir de sa position~; dans tous les cas donner son nombre
	      d'électrons de valence et son schéma de \textsc{Lewis} (bloc $s$ ou
	      $p$).
	\item Établir (pas «~juste~» donner) les représentations de \textsc{Lewis}
	      de molécules simples (\ce{CO2}, \ce{CH4}, \ce{H2O}, \ce{NH3}…) et
	      indiquer leurs représentations spatiales liées à la méthode VSEPR en
	      donnant un ordre de grandeur des angles.
	\item Établir les représentations de \textsc{Lewis} et les charges formelles
	      de $\ce{HO^-, CN^-, NO3^-}$.
	\item Définir l'électronégativité d'un élément et donner (en le justifiant)
	      son évolution par colonne, par famille et globalement dans le tableau.
	      Définir le moment dipolaire d'une liaison, d'une molécule et la
	      polarisabilité, et déterminer le moment dipolaire de \ce{H2O}
	      connaissant $p_{\ce{HO}} = \SI{1.51}{D}$ et $\widehat{({\rm HOH})} =
		      \ang{104.45}$.
\end{enumerate}
\ssubsection{AM2}{Propriétés physico-chimiques macroscopiques}
\begin{enumerate}[resume]
	\item Définir ce qu'est la liaison hydrogène, donner un ordre de grandeur de
	      l'énergie d'une LH, les représenter sur les molécules d'eau et indiquer,
	      avec 2 valeurs numériques, l'impact de la LH sur la température
	      d'ébullition de l'eau. Indiquer (sans valeur numérique nécessaire) et
	      justifier l'évolution des températures d'ébullition des composés
	      hydrogénés de la 14\ieme\ colonne (\ce{CH4}, \ce{SiH4}, \ce{GeH4} et
	      \ce{SnH4}).
	\item Définir ce qu'est un solvant polaire, protique, et dispersant.
	      Déterminer, à partir de la représentation d'une molécule de solvant et
	      de sa valeur de permittivité relative, s'il est polaire, protique et
	      dispersant ou non. Indiquer comment choisir un solvant connaissant le
	      soluté à dissoudre.
\end{enumerate}
\ssubsection{M6}{Moment cinétique d'un point matériel}
\begin{enumerate}[resume]
	\item Définir le moment cinétique d'un point matériel par rapport à un point
	      et à un axe, et le moment d'une force par rapport à un point et à un
	      axe. Expliquer ce qu'est le bras de levier \textbf{avec un schéma}, et
	      énoncer le lien entre moment d'une force et bras de levier.
	      Démonstration \textbf{pour $\Ff \perp$ à l'axe} (dans le plan de
	      rotation).
	\item Énoncer et démontrer le théorème du moment cinétique par rapport à un
	      point et à un axe~; application au pendule simple pour retrouver
	      l'équation du mouvement \textbf{avec ou sans bras de levier} (au choix
	      de l'interrogataire).
\end{enumerate}

\end{document}
