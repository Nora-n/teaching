\documentclass[a4paper, 11pt]{book}
\usepackage{/home/nicolas/Documents/Enseignement/Prepa/bpep/fichiers_utiles/preambule}
\newcommand{\qed}{\tag*{$\blacksquare$}}

\newcommand{\dsNB}{19}
\makeatletter
\renewcommand{\@chapapp}{Kh\^olles MPSI2 -- semaine \dsNB}
\makeatother

% \toggletrue{corrige}  % décommenter pour passer en mode corrigé

% IMPORTS automatiques
\newcommand{\f}[2]{{
		\mathchoice
		{\dfrac{#1}{#2}}
		{\dfrac{#1}{#2}}
		{\frac{#1}{#2}}
		{\frac{#1}{#2}}
}}

\newcommand{\e}[1]{{}_{\text{#1}}}
\renewcommand{\a}[0]{\alpha}
\newcommand{\w}[0]{\omega}

\usepackage{physics}

 % fin des IMPORTS automatiques

\begin{document}

\chapter{Sujet 1\siCorrige{\!\!-- corrig\'e}}
\section{Question de cours}

Savoir construire les 4 premières lignes du tableau en y plaçant les blocs s, p
et d. Ajouter les 4 familles. Préciser la position métaux/non-métaux. Placer le
germanium ($Z=32$), établir sa configuration électronique et son schéma de
\textsc{Lewis}. Le manganèse \ce{Mn} est période 4 colonne 7~: donner son
numéro atomique. Définir l'électronégativité d'un élément et indiquer et
expliquer son évolution dans le tableau.

\resetQ
\subimport{/home/nicolas/Documents/Enseignement/Prepa/bpep/exercices/TD/oscilloscope_analogique/}{sujet.tex}

\chapter{Sujet 2\siCorrige{\!\!-- corrig\'e}}
\section{Question de cours}

Pour les molécules de \ce{HCl}, \ce{CO2}, \ce{CH4}, \ce{H2O}, \ce{NH3}~:
dessiner leur schéma de \textsc{Lewis}. Donner leur forme géométrique liée à la
méthode VSEPR et l'ordre de grandeurs des angles, savoir les dessiner
correctement. Dire si elles sont polaires et dessiner leur moment dipolaire s'il
existe. On donne~: \ce{_1H}, \ce{_6C}, \ce{_7N}, \ce{_8O}, \ce{_17Cl}.

\resetQ
\subimport{/home/nicolas/Documents/Enseignement/Prepa/bpep/exercices/Colle/spectrometre_de_Dempster/}{sujet.tex}

\chapter{Sujet 3\siCorrige{\!\!-- corrig\'e}}
\section{Question de cours}

Action d'un champ magnétique uniforme sur une particule chargée dans le vas où
$\vfo\perp\Bf$. Déterminer le rayon de la trajectoire en utilisant la base de
\textsc{Frenet}.

\resetQ
\subimport{/home/nicolas/Documents/Enseignement/Prepa/bpep/exercices/TD/moment_dipolaire_eau/}{sujet.tex}

\chapter{Sujet 4\siCorrige{\!\!-- corrig\'e}}
\resetQ
\subimport{/home/nicolas/Documents/Enseignement/Prepa/bpep/exercices/TD/molecules_polaires/}{sujet.tex}
\section{Exercice}

\resetQ
\begin{enumerate}
    \item Donner le schéma de \textsc{Lewis} des espèces suivantes~:
        \[
            \ce{CH2Cl2}
            \qquad
            \ce{O2}
            \qquad
            \ce{C2H4}
            \qquad
            \ce{H3O+}
            \qquad
            \ce{HO-}
            \qquad
            \ce{H2CO}
            \qquad
            \ce{SiO2}
            \qquad
            \ce{CH3NH2}
        \]
    \item L'ozone \ce{O3} est une molécule non cyclique. Proposer une structure.
    \item \textit{Formule de \textsc{Lewis} de l'acide sulfurique}
    \begin{enumerate}[]
        \item Donner le schéma de \textsc{Lewis} de l'acide sulfurique
            \ce{H2SO4}. Dans cette molécule, les quatre atomes d'oxygène sont
            reliés à l'atome de soufre.
        \item En déduire celles des ions \ce{HSO4-} et \ce{SO4^{2-}}.
    \end{enumerate}
    \item Donner le schéma de \textsc{Lewis} des ions hydrogénocarbonate
        \ce{HCO3-} et carbonate \ce{CO3^{2-}}.
    \item Donner le schéma de \textsc{Lewis} du benzène \ce{C6H6}, qui est une
        molécule cyclique.
\end{enumerate}

\label{LastPage}
\end{document}
