\documentclass[a4paper, 12pt, final, garamond]{book}
\usepackage{cours-preambule}

\raggedbottom

\makeatletter
\renewcommand{\@chapapp}{Programme de kh\^olle -- semaine}
\makeatother

\begin{document}
\setcounter{chapter}{21}

\chapter{Du 25 au 28 mars}

\section{Cours et exercices}
\ssubsection{C4}{Réactions acido-basiques}
\begin{enumerate}[label=\Roman*]
	\bitem{Acides et bases}~: couples, pH.
	\bitem{Réactions acido-basiques}~: constantes
	d'acidité, autoprotolyse de l'eau, réactions entre couples et calculs de
	constantes.
	\bitem{Distribution des espèces d'un couple}~: lien pH et concentration
	(relation de \textsc{Henderson}), diagramme de prédominance (force des
	acides et échelle des $\pk[A]$), diagramme
	de distribution.
	\bitem{Méthode de détermination d'un pH}.
\end{enumerate}

\ssubsection{C5}{Réactions de précipitation}
\begin{enumerate}[label=\Roman*]
	\bitem{Équilibre d'un solide en solution}~: dissolution et précipitation,
	équilibre, condition d'existence d'un précipité.
	\bitem{Facteurs influençant la solubilité}~: température, ions communs,
	influence du pH.
\end{enumerate}

\section{Cours uniquement}
\ssubsection{C6}{Réactions d'oxydoréduction}
\begin{enumerate}[label=\Roman*]
	\bitem{Oxydants et réducteurs}~: couples rédox, nombre d'oxydation.
	\bitem{Distribution des espèces d'un couple}~: potentiel de \textsc{Nernst},
	diagramme de prédominance.
	\bitem{Réactions entre couples}~: réactions d'oxydoréduction, sens spontané de
	réaction, cas particuliers.
	%, calcul de constantes d'équilibres.
	% \bitem{Piles électrochimiques}~: présentation, force électromotrice, charge
	% totale d'une pile.
\end{enumerate}

\newpage

\section{Questions de cours possibles}
\begin{enumerate}[label=\sqenumi]
	\ssubsection{C4}{Réactions acido-basiques}
	\item Définir le pH, la constante d'acidité d'un couple acide/base,
	      l'autoprotolyse de l'eau et le produit ionique de l'eau. Écrire la
	      réaction associée à la constante d'acidité du couple \ce{H3O+/H2O},
	      exprimer la constante d'acidité en fonction de \ce{[H3O+]} et en déduire
	      pK$_a$(\ce{H3O+/H2O}) = 0. Faire de même avec la réaction associée à la
	      constante d'acidité du couple \ce{H2O/HO-}, et en déduire
	      pK$_a$(\ce{H2O/HO-}) = pK$_e$.
	\item Connaître nom, formule et équation entre acide et base des couples
	      contenant~: acide sulfurique, acide nitrique, acide chlorhydrique, acide
	      phosphorique, acide éthanoïque, acide carbonique, ion ammonium, ion
	      hydroxyde. À partir du lien entre pH et pK$_a$ d'un couple acide-base,
	      justifier et tracer un diagramme de prédominance.
	      % \item Tracer qualitativement le diagramme de distribution de l'acide
	      %     carbonique \ce{H2CO3}. Identifier les espèces sur le schéma, indiquer
	      %     comment lire le pK$_a$ des couples, et le lien entre les concentrations
	      %     des espèces des couples quand pH = pK$_a$.
	\item On mélange $V_0 = \SI{50}{mL}$ d'une solution d'acide éthanoïque à $c_0
		      = \SI{0.10}{mol.L^{-1}}$, et le même volume d'une solution de nitrite de
	      sodium $\left( \ce{Na+};\ce{NO_2-} \right)$ à la même concentration. On
	      donne
	      \[
		      \pk[A,1] = \pk(\ce{CH_3COOH}/\ce{CH_3COO-}) = \num{4.74}
		      \qet
		      \pk[A,2] = \pk(\ce{HNO_2}/\ce{NO_2-}) = \num{3.2}
	      \]
	      \textbf{Déterminer les concentrations des espèces à l'équilibre et le
		      pH}

	      \ssubsection{C5}{Réactions de précipitation}
	\item Définir le produit de solubilité avec un exemple. Déterminer la
	      condition d'existence d'un précipité lors d'une précipitation avec cet
	      exemple. Application~: on ajoute $n = \SI{e-5}{mol}$ d'ions \ce{Cl-}
	      dans $V_0 = \SI{10}{mL}$ de nitrate d'argent $\left(\ce{Ag+},
		      \ce{NO_3^-}
		      \right)$ à $c_0 = \SI{e-3}{mol.L^{-1}}$. On donne $\pk[s](\ce{AgCl}) =
		      \num{9.8}$. \textbf{Obtient-on un précipité de chlorure d'argent
		      \ce{AgCl}~?}

	\item Définir la solubilité. Calculer la solubilité de \ce{NaCl} et de
	      \ce{PbI2}, sachant que $\pk[s](\ce{NaCl}) = 36$ et $\pk[s](\ce{PbI2}) =
		      8$.

	\item Présenter ce qu'est un diagramme d'existence de manière générale.
	      Tracer le diagramme d'existence de $\ce{AgCl}_{\rm(s)}$ en fonction de
	      $\prm \ce{Cl}$ pour une solution de \ce{Ag^+} à $c_0 =
		      \SI{0.10}{mol.L^{-1}}$.

	\item Donnez les paramètres influençant la solubilité. Donner un exemple
	      d'application pour chacun d'eux. En particulier, connaissant
	      p$K_s(\ce{AgCl}) = 9.8$, déterminer la solubilité de $\ce{AgCl\sol{}}$
	      dans une solution aqueuse contenant déjà $c = \SI{0.1}{mol.L^{-1}}$ de
	      \ce{Cl-}.

	      \ssubsection{C6}{Réactions d'oxydoréduction}
	\item Donner les couples et les demi-équations rédox des couples ions
	      tétrathionates/ion thiosulfate, ion permanganate/ion manganèse II, ion
	      dichromate/ion chrome III.
	      Définir puis
	      calculer le nombre d'oxydation des éléments dans ces équilibres.
	\item Pour une demi-réaction rédox générale, donner la formule de
	      \textsc{Nernst}, puis la forme commune à \SI{25}{\degreeCelsius}.
	      Application pour le couple $(\ce{MnO4^{-}\aqu{}/Mn^{2+}\aqu{}})$.
	\item Équilibrer la réaction entre les ions fer II et les ions permanganate.
	      Sachant que $E^\circ(\ce{Fe^3+/Fe^2+}) = \SI{0.77}{V}$ et
	      $E^\circ(\ce{Ce^4+/Ce^3+}) = \SI{1.74}{V}$, déterminer de deux manières
	      différentes si une réaction spontané survient. Qu'est-ce qu'une
	      dismutation~? Une médiamutation~?
\end{enumerate}

\end{document}
