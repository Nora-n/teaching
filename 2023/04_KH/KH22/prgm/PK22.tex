\documentclass[a4paper, 12pt, final, garamond]{book}
\usepackage{cours-preambule}

\raggedbottom

\makeatletter
\renewcommand{\@chapapp}{Programme de kh\^olle -- semaine}
\makeatother

\begin{document}
\setcounter{chapter}{21}

\chapter{Du 20 au 27 mars}

\section{Cours et exercices}
\section*{Mécanique ch. 7 -- Mouvement à force centrale conservative}
\begin{enumerate}[label=\Roman*]
    \litem{Forces centrales conservatives}~: définition force centrale,
        définition force centrale conservative et exemples.
    \litem{Quantités conservées}~: moment cinétique, loi des aires, énergie
        mécanique et énergie potentielle effective.
    \litem{Champs de force newtoniens}~: définition, cas attractif, cas répulsif.
    \litem{Mécanique céleste}~: lois des \textsc{Kepler}, mouvement circulaire.
    \litem{Satellite en orbite terrestre}~: vitesses cosmiques, satellite
        géostationnaire.
\end{enumerate}

\section*{Mécanique ch. 8 -- Mécanique du solide}
\begin{enumerate}[label=\Roman*]
    \litem{Système de points matériels}~: Systèmes discret et continu, centre
    d'inertie, mouvements d'un solide indéformable~: translation, rotation.
    \litem{Rappel~: TRC}~: quantité de mouvement d'un ensemble de points, forces
    intérieures et extérieures, théorème de la résultante cinétique.
    \litem{Énergétique des systèmes de points}~: énergie cinétique, puissances
    intérieures et extérieures, théorèmes énergétiques.
    \litem{Moments pour un système de points}~: moment cinétique et moment
    d'inertie, moments intérieurs et extérieurs, théorème du moment cinétique,
    énergétique d'un solide en rotation.
    \litem{Cas particuliers et application}~: notion de couple, liaison pivot,
    pendule pesant.
\end{enumerate}

\section{Cours uniquement}
\section*{Chimie chapitre 4 -- Réactions acido-basiques}
\begin{enumerate}[label=\Roman*]
    \item{Acides et bases}~: définitions, pH.
    \item{Rappel état d'équilibre}~: introduction, transformations totales et
        limitées, quotient de réaction et constante d'équilibre, évolution d'un
        système chimique.
    \item{Réactions acido-basiques}~: autoprotolyse de l'eau, constantes
        d'acidité, calcul de constantes de réactions.
    \item{Distribution des espèces d'un couple}~: lien pH et concentration
        (relation de \textsc{Henderson}), diagramme de prédominance, diagramme
        de distribution.
    \item{Prédiction des réactions et des équilibres}~: sens d'échange des
        protons et diagramme de pKa, pH et composition à l'équilibre.
    \item{Titrages acido-basiques}~: définition et exemple, méthodes de suivi.
\end{enumerate}

\section{Questions de cours possibles}
\begin{enumerate}[label=\sqenumi]
    \item Présenter ce qu'est une force centrale, démontrer que le moment
        cinétique se conserve, en déduire l'expression de la constante des
        aires, prouver que le mouvement est donc plan, et démontrer la loi des
        aires.
    \item En utilisant la constante des aires, déterminer l'expression de
        l'énergie potentielle effective pour un mouvement à force centrale
        conservative. Donner $\Ec_p$ pour un champ de force newtonien, 
        représenter $\Ec_{p,\rm eff}$ et discuter de la nature du mouvement en
        fonction de l'énergie mécanique totale (cas attractif \textbf{et}
        répulsif).
    \item Énoncer les trois lois de \textsc{Kepler}, démontrer la troisième loi
        de \textsc{Kepler} pour le cas spécifique de l'orbite circulaire~:
        vitesse, période, et énergie mécanique.
    \item Définir et démontrer les expressions des vitesses cosmiques en
        justifiant les valeurs d'énergie mécanique à atteindre à l'aide du
        schéma de l'énergie potentielle effective.
    \item Définir le moment d'inertie d'un solide, donner et démontrer la
        relation entre moment cinétique scalaire et moment d'inertie d'un
        solide. Retrouver le TMC pour un solide en rotation, en supposant acquis
        que la somme des moments intérieurs est nulle. Définir un couple, une
        liaison pivot et une liaison pivot parfaite.
    \item Établir l'équation différentielle du mouvement pour le pendule
        \textbf{pesant} grâce au TMC scalaire. Une approche par bras de levier
        uniquement peut être demandée.
    \item Donner l'expression de l'énergie cinétique d'un solide en translation
        en fonction de sa masse et de la vitesse de son centre d'inertie,
        \textbf{et} dans le cas particulier d'un solide en rotation autour d'un
        axe fixe. Exprimer les théorèmes énergétiques pour les solides. Donner
        l'expression de la puissance des forces extérieures pour un solide en
        rotation en fonction du moment des forces extérieures. Démonstration
        pour une force $\Ff$ dans le sens de $\ut$.
    \item Définir le pH, la constante d'acidité d'un couple acide/base,
        l'autoprotolyse de l'eau et le produit ionique de l'eau. Écrire la
        réaction associée à la constante d'acidité du couple \ce{H3O+/H2O},
        exprimer la constante d'acidité en fonction de \ce{[H3O+]} et en déduire
        pK$_a$(\ce{H3O+/H2O}) = 0. Faire de même avec la réaction associée à la
        constante d'acidité du couple \ce{H2O/HO-}, et en déduire
        pK$_a$(\ce{H2O/HO-}) = pK$_e$.
    \item Connaître nom, formule et équation entre acide et base des couples
        contenant~: acide sulfurique, acide nitrique, acide chlorhydrique, acide
        phosphorique, acide éthanoïque, acide carbonique, ion ammonium, ion
        hydroxyde. À partir du lien entre pH et pK$_a$ d'un couple acide-base,
        justifier et tracer un diagramme de prédominance.

    \item Tracer qualitativement le diagramme de distribution de l'acide
        carbonique \ce{H2CO3}. Identifier les espèces sur le schéma, indiquer
        comment lire le pK$_a$ des couples, et le lien entre les concentrations
        des espèces des couples quand pH = pK$_a$.
\end{enumerate}
\vspace{-5pt}

\begin{framed}
    \centering\bfseries\large
    Les fiches doivent être \ul{succinctes} et ne pas faire 3 copies doubles.
    Synthétisez l'information. Il est interdit de copier-coller le cours.
    \bigbreak \Huge
    Les fiches de plus de 2 copies doubles impliqueront un malus de 1 point sur
    la question de cours.
\end{framed}

\end{document}
