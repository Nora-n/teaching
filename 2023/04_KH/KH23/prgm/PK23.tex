\documentclass[a4paper, 11pt, final, garamond]{book}
\usepackage{cours-preambule}

\raggedbottom

\makeatletter
\renewcommand{\@chapapp}{Programme de kh\^olle -- semaine}
\makeatother

\begin{document}
\setcounter{chapter}{22}

\chapter{Du 02 au 04 avril}

\section{Exercices uniquement}
\ssubsection{C4}{Réactions acido-basiques}

\section{Cours et exercices}
\ssubsection{C5}{Réactions de précipitation}
\begin{enumerate}[label=\Roman*]
	\bitem{Équilibre d'un solide en solution}~: dissolution et précipitation,
	équilibre, condition d'existence d'un précipité.
	\bitem{Facteurs influençant la solubilité}~: température, ions communs,
	influence du pH.
\end{enumerate}

\ssubsection{C6}{Réactions d'oxydoréduction}
\begin{enumerate}[label=\Roman*]
	\bitem{Oxydants et réducteurs}~: couples rédox, nombre d'oxydation.
	\bitem{Distribution des espèces d'un couple}~: potentiel de \textsc{Nernst},
	diagramme de prédominance.
	\bitem{Réactions entre couples}~: réactions d'oxydoréduction, sens spontané de
	réaction, cas particuliers, calcul de constantes d'équilibres.
	\bitem{Piles électrochimiques}~: présentation, force électromotrice, charge
	totale d'une pile.
\end{enumerate}

\section{Cours uniquement}
\ssubsection{C7}{Diagrammes $E-\pH$}
\begin{enumerate}[label=\Roman*]
	\bitem{Présentation}~: nécessité, analyse des frontières, diagramme de l'eau.
	\bitem{Construction et lecture}~: remplissage des espèces, position des
	frontières~: applications sur le diagramme du fer.
	\bitem{Application}~: sens spontané de réaction, stabilité d'une espèce dans
	l'eau (cas du fer), cas particuliers des dismutations (cas de l'iode).
\end{enumerate}

\newpage
\section{Questions de cours possibles}
\begin{enumerate}[label=\sqenumi]
	\ssubsection{C5}{Réactions de précipitation}
	\litem{23pt}{\str}%
	Définir le produit de solubilité avec un exemple. Déterminer la
	condition d'existence d'un précipité lors d'une précipitation avec cet
	exemple. Application~: on ajoute $n = \SI{e-5}{mol}$ d'ions \ce{Cl-}
	dans $V_0 = \SI{10}{mL}$ de nitrate d'argent $\left(\ce{Ag+},
		\ce{NO_3^-}
		\right)$ à $c_0 = \SI{e-3}{mol.L^{-1}}$. On donne $\pk[s](\ce{AgCl}) =
		\num{9.8}$. \textbf{Obtient-on un précipité de chlorure d'argent
		\ce{AgCl}~?}

	\litem{23pt}{\str}%
	Définir la solubilité. Calculer la solubilité de \ce{NaCl} et de
	\ce{PbI2}, sachant que $\pk[s](\ce{NaCl}) = 36$ et $\pk[s](\ce{PbI2}) =
		8$.

	\litem{23pt}{\str}%
	Présenter ce qu'est un diagramme d'existence de manière générale.
	Tracer le diagramme d'existence de $\ce{AgCl}_{\rm(s)}$ en fonction de
	$\prm \ce{Cl}$ pour une solution de \ce{Ag^+} à $c_0 =
		\SI{0.10}{mol.L^{-1}}$.

	\litem{23pt}{\strr}%
	Donnez les paramètres influençant la solubilité. Donner un exemple
	d'application pour chacun d'eux. En particulier, connaissant
	p$K_s(\ce{AgCl}) = 9.8$, déterminer la solubilité de $\ce{AgCl\sol{}}$
	dans une solution aqueuse contenant déjà $c = \SI{0.1}{mol.L^{-1}}$ de
	\ce{Cl-}.

	\ssubsection{C6}{Réactions d'oxydoréduction}
	\litem{23pt}{\str}%
	Équilibrer la réaction entre les ions fer II et les ions permanganate.
	Sachant que $E^\circ(\ce{Fe^3+/Fe^2+}) = \SI{0.77}{V}$ et
	$E^\circ(\ce{Ce^4+/Ce^3+}) = \SI{1.74}{V}$, déterminer de deux manières
	différentes si une réaction spontané survient. Qu'est-ce qu'une
	dismutation~? Une médiamutation~?

	\litem{23pt}{\strr}%
	Établir l'expression d'une constante d'équilibre redox en fonction des
	potentiels standards des couples fournis par l'interrogataire (exemple du
	cours~: réaction de \ce{H2O2} et \ce{MnO4^-} des couples \ce{O2/H2O2} et
	\ce{MnO4^{-}/Mn^2+}). Conclure sur la nature de la réaction. Peut-on
	déterminer le sens de réaction à partir de ce calcul~? Pourquoi~?

	\litem{23pt}{\strr}%
	Présenter ce qu'est une pile avec l'exemple de la pile
	\textsc{Daniell}~: schéma, vocabulaire, explication. Déterminer à l'aide
	de la formule de \textsc{Nerst}, l'anode et la cathode. Justifier ces
	rôles par un raisonnement sur le mouvement des électrons. Établir
	l'expression de la capacité d'une pile en fonction du nombre d'électrons
	échangés, de l'avancement à l'équilibre et du nombre de \textsc{Faraday}
	à partir de l'exemple de la pile \textsc{Daniell}.

	\ssubsection{C7}{Diagrammes $E-\pH$}
	\litem{23pt}{\str}%
	Présenter les similitudes entre le cours acide/base et le cours rédox.
	Doivent apparaître (suffisant mais pas que~!) la formule
	d'\textsc{Henderson} pour les réactions acide-bases (lien pH et $\pk$) et
	la formule de \textsc{Nerst} pour les réactions redox. Indiquer par
	exemple à l'aide de diagrammes comment déterminer le sens qualitatif d'une
	réaction, et appliquer ces visions aux diagrammes $E-\pH$ (position des
	acides et des bases, position des oxydants et réducteurs).

	\litem{23pt}{\strr}%
	Établir et tracer le diagramme potentiel-pH de l'eau. Une attention
	particulière sera portée à l'établissement du lien entre $E$ et pH et à
	l'utilisation des conventions de tracé. On prendra $p_t = \SI{1}{bar}$.

	\litem{23pt}{\strr}%
	À partir du schéma du diagramme potentiel-pH du fer, attribuer les
	différentes espèces possibles (données) aux domaines. Expliquer comment
	évolue le nombre d'oxydation dans un diagramme $E-\pH$ et tracer le
	diagramme de situation.

	\litem{23pt}{\strrr}%
	À partir du diagramme $E-\pH$ du fer dont les espèces sont placées,
	déterminer la position des frontières verticales et horizontales et les
	pentes des frontières inclinées. On donne
	\begin{itemize}
		\item $E_1^\circ(\ce{{Fe}^2+_{\rm(aq)}/Fe}) = \SI{-.44}{V}$~;
		      $E_2^\circ(\ce{{Fe}^3+_{\rm(aq)}/{Fe}^2+_{\rm(aq)}}) = \SI{0.77}{V}$~;
		\item $\pk[s,2] = \pk[s](\ce{Fe(OH)_2}) = 15$ et $\pk[s,3] =
			      \pk[s](\ce{Fe(OH)_3}) = 38$~;
		\item Convention de tracé $c_t = \SI{0.01}{mol.L^{-1}}$.
	\end{itemize}
\end{enumerate}

\end{document}
