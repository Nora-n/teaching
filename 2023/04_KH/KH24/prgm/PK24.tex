\documentclass[a4paper, 12pt, final, garamond]{book}
\usepackage{cours-preambule}

\raggedbottom

\makeatletter
\renewcommand{\@chapapp}{Programme de kh\^olle -- semaine}
\makeatother

\begin{document}
\setcounter{chapter}{23}

\chapter{Du 03 au 07 avril}

\section{Cours et exercices}
\section*{Chimie chapitre 4 -- Réactions acido-basiques}
\begin{enumerate}[label=\Roman*]
    \litem{Acides et bases}~: définitions, pH.
    \litem{Rappel état d'équilibre}~: introduction, transformations totales et
        limitées, quotient de réaction et constante d'équilibre, évolution d'un
        système chimique.
    \litem{Réactions acido-basiques}~: autoprotolyse de l'eau, constantes
        d'acidité, calcul de constantes de réactions.
    \litem{Distribution des espèces d'un couple}~: lien pH et concentration
        (relation de \textsc{Henderson}), diagramme de prédominance, diagramme
        de distribution.
    \litem{Prédiction des réactions et des équilibres}~: sens d'échange des
        protons et diagramme de pKa, pH et composition à l'équilibre.
    \litem{Titrages acido-basiques}~: définition et exemple, méthodes de suivi.
\end{enumerate}

\section*{Chimie chapitre 5 -- Réactions de précipitation}
\begin{enumerate}[label=\Roman*]
    \litem{Observations expérimentales}~: exemple et définition précipité.
    \litem{Produit de solubilité}~: définition et exemples.
    \litem{Condition d'existence}~: existence en fonction de $K_s$.
    \litem{Solubilité}~: définition, dans l'eau pure, paramètres d'influence~:
        température, ions communs, pH.
\end{enumerate}

\section*{Chimie chapitre 6 -- Réactions d'oxydoréduction}
\begin{enumerate}[label=\Roman*]
    \litem{Oxydants et réducteurs}~: introduction, définition, réactions
        d'oxydoréduction, équilibrage des demi-équations et couples à connaître,
        équilibrage des réactions rédox~; nombre d'oxydation, introduction,
        règles de calcul, interprétation, lien avec la position dans la
        classification périodique.
    \litem{Piles}~: introduction, vocabulaire, potentiel d'électrode, application
        calcul f.é.m., capacité d'une pile.
    \litem{Réactions d'oxydoréduction}~: diagramme de prédominance, sens de
        réaction et diagramme en potentiel standard, calcul des constantes
        d'équilibre, et application, dismutation et médiamutation
\end{enumerate}

\section{Cours uniquement}
\section*{Chimie chapitre 7 -- Diagrammes potentiel-pH}
\begin{enumerate}[label=\Roman*]
    \litem{Influence pH et oxydoréduction}~: potentiel standard apparent,
    convention de tracé, existence d'un précipité.
    \litem{Diagramme $E$-pH de l'eau}~: tracé.
    \litem{Diagramme $E$-pH du fer}~: introduction, attribution, frontières
    horizontales, verticales et droites frontières.
\end{enumerate}

\section{Questions de cours possibles}
\begin{center}
    \begin{framed}
      \huge
        Vous ferez bien attention à \textbf{prendre vos feuilles de khôlle}
        ainsi que de \textbf{remplir les éventuels trous} dans celles-ci.
    \end{framed}
\end{center}

\begin{enumerate}[label=\sqenumi]
    \item Définir le produit de solubilité avec un exemple. Déterminer la
        condition d'existence d'un précipité. \textbf{Établir et tracer le
        diagramme d'exitence de \ce{AgCl} en fonction de p\ce{Cl} pour
      $[\ce{Ag+}]_0$ donné par l'examinataire}.

    \item Donnez les paramètres influençant la solubilité. Donner un exemple
      d'application pour chacun d'eux. En particulier, connaissant
      p$K_s(\ce{AgCl}) = 9.8$, déterminer la solubilité de $\ce{AgCl\sol{}}$
      dans une solution aqueuse contenant déjà $c = \SI{0.1}{mol.L^{-1}}$ de
      \ce{Cl-}. On supposera $s \ll c$.

    \item On introduit des ions \ce{Ag+} dans une solution contenant des ions
      chlorure et chromate, de concentrations respectives $c_0 =
      \SI{7.5e-2}{mol.L^{-1}}$ et $c_1 = \SI{0.20}{mol.L^{-1}}$. On donne
      p$K_s(\ce{AgCl}) = \num{9.8}$ et p$K_s(\ce{Ag2CrO4}) = \num{11.9}$.
      Prévoir quel précipité se forme en premier à l'aide d'un diagramme
      d'existence en fonction de p\ce{Ag}.
    \item Donner les couples et les demi-équations redox des couples contenant~:
        ions thiosulfate, ion permanganate, ion hypochlorite. Donner le nombre
        d'oxydation des éléments. Équilibrer la réaction entre
        $\ce{Fe^{2+}\aqu{}}$ et $\ce{MnO4^-\aqu{}}$.
    \item Présenter les similitudes entre la formule d'\textsc{Henderson} pour
      les réactions acide-bases (lien pH et p$K_a$) et
        la formule de \textsc{Nerst} pour les réactions redox, notamment à
        l'aide de diagrammes en p$K_a$ et en $E$\textdegree pour déterminer le
        sens qualitatif de réaction, et de diagrammes de prédominance/existence.
    \item Présenter ce qu'est une pile avec l'exemple de la pile
      \ce{Zn^{2+}/Ag}~: schéma, vocabulaire, explication. Déterminer, à l'aide
      de la formule de \textsc{Nerst}, l'anode et la cathode.
    \item Établir l'expression de la capacité d'une pile en fonction du nombre
        d'électrons échangés, de l'avancement à l'équilibre et du nombre de
        \textsc{Faraday} à partir de l'exemple de la pile \textsc{Daniell}.
    \item Calculer une constante d'équilibre redox en fonction des potentiels
      standards des couples fournis par l'interrogataire. Conclure sur la nature
      de la réaction.
    \item Établir et tracer le diagramme potentiel-pH de l'eau. Une attention
      particulière sera portée à l'établissement du lien entre $E$ et pH et à
      l'utilisation des conventions de tracé. On prendra $p_t = \SI{1}{bar}$.
    \item À partir du schéma du diagramme potentiel-pH du fer, attribuer les
      différentes espèces possibles (données) aux domaines. Expliquer comment
      évolue le nombre d'oxydation dans un diagramme $E$-pH.
\end{enumerate}
\vspace{-5pt}

\begin{framed}
    \centering\bfseries\large
    Les fiches doivent être \ul{succinctes} et ne pas faire 3 copies doubles.
    Synthétisez l'information. Il est interdit de copier-coller le cours.
    \bigbreak \Huge
    Les fiches de plus de 2 copies doubles impliqueront un malus de 1 point sur
    la question de cours.
\end{framed}

\end{document}
