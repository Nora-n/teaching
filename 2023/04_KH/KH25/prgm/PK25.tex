\documentclass[a4paper, 11pt, final, garamond]{book}
\usepackage{cours-preambule}

\raggedbottom

\makeatletter
\renewcommand{\@chapapp}{Programme de kh\^olle -- semaine}
\makeatother

\begin{document}
\setcounter{chapter}{24}

\chapter{Du 15 au 18 avril}

\section{Exercices uniquement}
\subsection(C6){Réactions d'oxydoréduction}
\subsection(C7){Diagrammes $E-\pH$}

\section{Cours et exercices}
\subsection(T1){Description d'un système à l'équilibre}
\begin{enumerate}[label=\Roman*]
	\item[b]{Introduction}~: ordres de grandeur, échelles de description.
	\item[b]{Système}~: définition, grandeurs d'état (intensive, extensives,
	massique et molaire) et fonction d'état, grandeurs usuelles~: température,
	pression, énergie interne, capacité thermique.
	\item[b]{Équilibre thermodynamique}~: définition, exemple, conditions
	d'équilibres thermique et mécanique.
	\item[b]{Description d'un gaz}~: modélisation, loi du gaz parfait et pertinence
	expérimentale (diagrammes d'\textsc{Amagat} et de \textsc{Clapeyron}),
	énergétique (température cinétique, énergie interne
	et capacité thermique), vitesse quadratique moyenne.
	\item[b]{Phases condensées}~: modélisation, équation d'état, énergétique
	(énergie interne, capacité thermique).
\end{enumerate}

\begin{framed}
	\begin{center}
		\large
		Les exercices doivent rester simples avec des conditions d'équilibre mais
		pas de premier principe.
	\end{center}
\end{framed}

\section{Cours uniquement}
\subsection(T2){Échanges d'énergie des transformations thermodynamiques}
\begin{enumerate}[label=\Roman*]
	\item[b]{Introduction}~: nécessité, transformations, types, influence du choix
	du système.
	\item[b]{Travail des forces de pression}~: expression générale, transformations
	isochore et isobare, transformation mécaniquement réversible et cycles.
	\item[b]{Transfert thermique}~: définition, types de transferts, cas
	particuliers (thermostat)
	%, adiabatique), bien comprendre les transferts thermiques (adiabatique vs.\
	%isotherme), Loi de Laplace.
\end{enumerate}

\newpage

\section{Questions de cours possibles}
\begin{enumerate}[label=\sqenumi]
	\subsection(T1){Description d'un système à l'équilibre}
	\item[s]"1"
	Présenter le vocabulaire de la thermodynamique~: libre parcours moyen,
	échelles de descriptions~; système isolé, fermé, ouvert~: grandeur et
	fonction d'état. Définir l'énergie interne et la capacité thermique
	volumique (et massique et molaire, avec les unités) d'un système, donner un
	exemple.

	\item[s]"1"
	Donner la modélisation du gaz parfait, en le distinguant
	explicitement d'un gaz réel grâce à un schéma. Indiquez la loi du gaz
	parfait avec les unités. À quoi ressemblent les isothermes d'\textsc{Amagat}
	pour le diazote~? pour un gaz réel en diagramme de \textsc{Clapeyron}~?
	Commentez l'écart au modèle.

	\item[s]"2"
	Donner la définition de la température cinétique en
	fonction du degré de liberté $D$. Déterminer alors l'énergie interne d'un
	gaz parfait mono- puis diatomique en fonction de $R$ qu'on reliera à deux
	autres constantes. En déduire les capacités thermiques $C_{V,\rm
				mono}\sup{G.P.}$ et $C_{V,\rm dia}\sup{G.P.}$

	\item[s]"1"
	Représenter la distribution des vitesses des molécules d'un gaz et ses
	propriétés, définir la vitesse quadratique moyenne et la relier à la
	température cinétique.

	\item[s]"1"
	Présenter ce qu'est une phase condensée et incompressible, donner
	l'équation d'état. Démontrer de quelle variable dépend l'énergie interne
	molaire d'une phase condensée, en déduire le lien entre la capacité
	thermique molaire et l'énergie interne d'une phase condensée.

	\subsection(T2){Échanges d'énergie des transformations thermodynamiques}
	\item[s]"1"
	Présenter les transformations de la thermodynamique~: système isolé,
	fermé, ouvert~; transformations, transformations isochore (avec exemple),
	monotherme, isotherme, monobare, isobare (avec exemple), adiabatique,
	mécaniquement réversible.

	\item[s]"2"
	Soit une enceinte indéformable, séparée en deux compartiments par une
	cloison étanche et mobile. Le premier a pour état initial $(T_i,P_i,V_i,n)$,
	le second $(T_i,2P_i,V_i,2n)$. Une cale bloque initialement la cloison
	mobile. On enlève la cale et on place l'enceinte dans un environnement à
	température $T_0$.
	\begin{enumerate}[label=\alph*)]
		\item Faire un schéma.
		\item Quelles sont les variables d'état des gaz dans l'état
		      d'équilibre final~?
		\item Qualifier la transformation selon le système étudié.
	\end{enumerate}

	\item[s]"2"
	Établir l'expression générale du travail des forces de pression.
	Préciser la nature du système (moteur, récepteur) selon le signe de $W$.
	Présenter le lien avec l'aire sous la courbe d'un diagramme de
	\textsc{Watt} $(P,V)$, et mettre en évidence la dépendance de $W$ au
	chemin suivi.

	\item[s]"2"
	Démontrer la valeur ou l'expression de $W$ pour une
	transformation isochore, pour une transformation monobare, et pour une
	transformation quasi-statique isotherme d'un gaz parfait, en fonctin des
	volumes d'abord puis des pressions ensuite. Vérifiez son signe selon le
	l'évolution du volume.

	\item[s]"2"
	Cycle de \textsc{Lenoir}~: pour une mole de gaz parfait à $P_\Ar =
		\SI{2e5}{Pa}$ et $V_\Ar = \SI{14}{L}$, on effectue les transformations
	suivantes de manière quasi-statique~:
	\begin{enumerate}[label=\alph*)]
		\item chauffage isochore jusqu'à $P_{\rm B} = \SI{4e5}{Pa}$~;
		\item détente isotherme jusqu'à $V_{\rm C} = \SI{28}{L}$~;
		\item refroidissement isobare jusqu'au retour à l'état initial.
	\end{enumerate}
	Représenter ce cycle sur un diagramme de \textsc{Watt} et en déduire le
	signe du travail total. Calculer $P$, $V$ et $T$ à chaque étape puis
	calculer les travaux associés aux transformations AB, BC et CA et sur le
	cycle. Conclure sur la nature du système.
\end{enumerate}

\end{document}
