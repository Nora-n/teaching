\documentclass[a4paper, 11pt, final, garamond]{book}
\usepackage{cours-preambule}

\raggedbottom

\makeatletter
\renewcommand{\@chapapp}{Programme de kh\^olle -- semaine}
\makeatother

\begin{document}
\setcounter{chapter}{25}

\chapter{Du 13 au 16 mai}

\section{Cours et exercices}
\subsection(T1){Description d'un système à l'équilibre}
\begin{enumerate}[label=\Roman*]
	\item[b]{Introduction}~: ordres de grandeur, échelles de description.
	\item[b]{Système}~: définition, grandeurs d'état (intensive, extensives,
	massique et molaire) et fonction d'état, grandeurs usuelles~: température,
	pression, énergie interne, capacité thermique.
	\item[b]{Équilibre thermodynamique}~: définition, exemple, conditions
	d'équilibres thermique et mécanique.
	\item[b]{Description d'un gaz}~: modélisation, loi du gaz parfait et
	pertinence expérimentale (diagrammes d'\textsc{Amagat} et de
	\textsc{Clapeyron}), énergétique (température cinétique, énergie interne et
	capacité thermique), vitesse quadratique moyenne.
	\item[b]{Phases condensées}~: modélisation, équation d'état, énergétique
	(énergie interne, capacité thermique).
\end{enumerate}

\subsection(T2){Échanges d'énergie des transformations thermodynamiques}
\begin{enumerate}[label=\Roman*]
	\item[b]{Introduction}~: nécessité, transformations, types, influence du choix
	du système.
	\item[b]{Travail des forces de pression}~: expression générale,
	transformations isochore et isobare, transformation mécaniquement réversible
	et cycles.
	\item[b]{Transfert thermique}~: définition, types de transferts, cas
	particuliers (thermostat, adiabatique), bien comprendre les transferts
	thermiques (adiabatique vs.\ isotherme), Loi de \textsc{Laplace}.
\end{enumerate}

\subsection(T3){Premier principe de la thermodynamique}
\begin{enumerate}[label=\Roman*]
	\item[b]{Énoncé du premier principe}~: énoncé général, première loi de
	\textsc{Joule}, cas particuliers, premier principe entre deux états voisins.
	\item[b]{Transformation monobare et enthalpie}~: enthalpie et premier
	principe, seconde loi de \textsc{Joule}, capacités thermiques et relations
	associées, calorimétrie.
\end{enumerate}

\newpage

\section{Questions de cours possibles}

\begin{enumerate}[label=\sqenumi]
	\subsection(T1){Description d'un système à l'équilibre}
	\item[s]"2"
	Donner la définition de la température cinétique en
	fonction du degré de liberté $D$. Déterminer alors l'énergie interne d'un
	gaz parfait mono- puis diatomique en fonction de $R$ qu'on reliera à deux
	autres constantes. En déduire les capacités thermiques $C_{V,\rm
				mono}\sup{G.P.}$ et $C_{V,\rm dia}\sup{G.P.}$

	\subsection(T2){Échanges d'énergie des transformations thermodynamiques}
	\item[s]"1"
	Présenter les transformations de la thermodynamique~: système isolé,
	fermé, ouvert~; transformations, transformations isochore (avec exemple),
	monotherme, isotherme, monobare, isobare (avec exemple), adiabatique,
	mécaniquement réversible.

	\item[s]"2"
	Établir l'expression générale du travail des forces de pression.
	Préciser la nature du système (moteur, récepteur) selon le signe de $W$.
	Présenter le lien avec l'aire sous la courbe d'un diagramme de
	\textsc{Watt} $(P,V)$, et démontrer comment le sens de parcours sur un
	cycle se relie au signe du travail.

	\item[s]"2"
	Démontrer la valeur ou l'expression de $W$ pour une transformation isochore,
	pour une transformation monobare, et pour une transformation quasi-statique
	isotherme d'un gaz parfait, en fonction des volumes d'abord puis des
	pressions ensuite. Vérifiez son signe selon l'évolution du volume.

	\item[s]"2"
	Cycle de \textsc{Lenoir}~: pour une mole de gaz parfait à $P_\Ar =
		\SI{2e5}{Pa}$ et $V_\Ar = \SI{14}{L}$, on effectue les transformations
	suivantes de manière quasi-statique~:
	\begin{enumerate}[label=\alph*)]
		\item chauffage isochore jusqu'à $P_{\rm B} = \SI{4e5}{Pa}$~;
		\item détente isotherme jusqu'à $V_{\rm C} = \SI{28}{L}$~;
		\item refroidissement isobare jusqu'au retour à l'état initial.
	\end{enumerate}
	Représenter ce cycle sur un diagramme de \textsc{Watt} et en déduire le
	signe du travail total. Calculer $P$, $V$ et $T$ à chaque étape puis
	calculer les travaux associés aux transformations AB, BC et CA et sur le
	cycle. Conclure sur la nature du système. Déterminer ensuite les variations
	d'énergie interne puis les transferts thermiques de chaque transformation.

	\subsection(T3){Premier principe de la thermodynamique}
	\item[s]"2" Énoncer le premier principe de la thermodynamique, en version
	intégrale et différentielle, en détaillant les termes. Préciser lesquels sont
	des fonctions d'état, lesquels ne le sont pas. Étudier les cas particuliers
	des transformations adiabatique, isochore et cyclique. \textbf{Expliquer la
		différence entre adiabatique et isotherme}.

	\item[s]"2" Définir l'enthalpie d'un corps et ses propriétés. Démontrer ensuite
	l'expression du premier principe enthalpique en indiquant ses conditions
	d'application.

	\item[s]"2" Présenter les deux lois de \textsc{Joule}. Rappeler les expressions
	de $U$ et $C_V$ pour un gaz parfait mono- et diatomique. Démontrer la seconde
	loi de \textsc{Joule} pour les phases condensées avec un calcul et pour les
	gaz parfait en donnant l'expression de l'enthalpie molaire en fonction du
	nombre de degré de liberté du gaz.

	\item[s]"1" Définir les capacités thermiques à volume et pression constantes
	dans le cas général. Les relier aux variations $\Delta{U}$ et $\Delta{H}$ pour
	un gaz parfait, et présentez la différence entre ces variations sur un
	diagramme de \textsc{Watt} présentant deux isothermes. Définir le coefficient
	adiabatique $\gamma$, démontrer la relation de \textsc{Mayer} et établir les
	expressions de $C_V$ et de $C_P$.

	\item[s]"2" Calorimétrie~: dans un calorimètre parfaitement isolé de masse en eau
	$m_0 = \SI{24}{g}$, on place $m_1 = \SI{150}{g}$ d'eau à $T_1 =
		\SI{298}{K}$. On ajoute $m_2 = \SI{100}{g}$ de cuivre à $T_2 = \SI{353}{K}$.
	Sachant que $c_{\rm Cu} = \SI{385}{J.K^{-1}.kg^{-1}}$ et $c_{\rm eau} =
		\SI{4185}{J.K^{-1}.kg^{-1}}$, déterminer $T_f$.

\end{enumerate}

\end{document}
