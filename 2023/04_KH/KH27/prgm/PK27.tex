\documentclass[a4paper, 12pt, final, garamond]{book}
\usepackage{cours-preambule}
\usepackage[french]{babel}

\raggedbottom

\makeatletter
\renewcommand{\@chapapp}{Programme de kh\^olle -- semaine}
\makeatother

\begin{document}
\setcounter{chapter}{26}

\chapter{Du 22 au 26 mai}

\section{Cours et exercices}

\section*{Thermodynamique chapitre 2 -- Premier principe}
\begin{enumerate}[label=\Roman*]
  \litem{Vocabulaire}~: système, transformations et exemples.
  \litem{Énergie interne}~: définition, énergie interne gaz parfait, capacité
    thermique à volume constant, capacité thermique d'une phase condensée.
  \litem{Travail des forces de pression}~: expression générale, cas particulier
    isochore et monobare, transformation quasi-statique (définition, exemples de
    diagrammes, corrélation avec l'aire sous la courbe~; application cycle de
    \textsc{Lenoir}), travail électrique.
  \litem{Transferts thermiques}~: définition, différents types de transferts
    thermiques, cas particuliers (adiabatique, thermostat), loi de
    \textsc{Laplace}.
  \litem{Premier principe de la thermodynamique}~: énoncé, application $Q$ cycle
    de \textsc{Lenoir}.
  \litem{Transformation monobare et enthalpie}~: démonstration $H = U+PV$,
    introduction $C_{P}$~: gaz parfait ($\gamma$, \textsc{Mayer}, $C_{V}$ et
    $C_P$ fonction de $\gamma$, lois de \textsc{Joule}) et phases condensées~;
    application calorimètre~; retour sur principe du thermostat.
\end{enumerate}

\section*{Thermo. chapitre 3 -- Second principe, machines thermiques}
\begin{enumerate}[label=\Roman*]
  \litem{L'entropie}~: irréversibilité, second principe et cas particuliers,
    comment faire une transformation réversible, expressions de l'entropie~:
    phases condensées, gaz parfait, démonstration lois des \textsc{Laplace}.
  \litem{Machines thermiques}~: introduction, principe général, machine
    monotherme, machines dithermes~: diagramme de \textsc{Raveau}, moteur
    ditherme et rendement de \textsc{Carnot}, cycle de \textsc{Carnot} complet,
    réfrigérateur et pompe à chaleur~; exemple moteur à explosion cycle de Beau
    de \textsc{Rochas}~: fonctionnement réel et modélisation.
  \litem{Notations infinitésimales et interprétation microscopique}~: premier
    principe sous forme $\dd{U} = \de{W}+\de{Q}$, application équation
    différentielle temps de chauffe~; second principe sous la forme $\dd{S} =
    \de{S_{\rm ech}+\de{S_{\rm cr}}}$ et première identité thermodynamique
    $\dd{U} = T\dd{S} - p\dd{V}$, application détermination sens transfert
    thermique entre deux solides~; interprétation microscopique de l'entropie~:
    formule de \textsc{Boltzmann}, exemple détente de \textsc{Joule-Gay-Lussac}
    et correspondance calcul de l'entropie avec les deux méthodes.
\end{enumerate}

\section{Cours uniquement}

\section*{Thermodynamique chapitre 4 -- Changements d'états}
\begin{enumerate}[label=\Roman*]
  \litem{Introduction}~: vocabulaire, observation changement température fixée
    pour pression fixée, hypothèses de travail.
  \litem{Diagramme $(P,T)$}~: cas fréquent, cas rare, vocabulaire courbes
    d'équilibre diphasé, point triple, point critique, pression de vapeur
    saturante.
  \litem{Diagramme de \textsc{Clapeyron}}~: expérience compression d'un gaz et
    changement d'état liquide, forme d'une courbe en $(P,V)$, et bilan
    isothermes d'\textsc{Andrews}~: courbes de rosée et d'ébullition.
\end{enumerate}

\section{Questions de cours possibles}

\begin{enumerate}[label=\sqenumi]
    \item[] \textbf{Chapitre 2}

    \item Établir l'expression générale du travail des forces de pression.
      Préciser la nature du système (moteur, récepteur) selon le signe de $W$.
      Présenter le lien avec l'aire sous la courbe d'un diagramme de
      \textsc{Clapeyron}, et mettre en évidence la dépendance de $W$ au chemin
      suivi. Démontrer la valeur ou l'expression de $W$ pour une transformation
      isochore, pour une transformation monobare, et pour une transformation
      quasi-statique isotherme d'un gaz parfait.

    \item Énoncer le premier principe de la thermodynamique, en détaillant les
      termes. Préciser lesquels sont des fonctions d'état, lesquels ne le sont
      pas. Étudier les cas particuliers des transformations adiabatique,
      isochore et cyclique. \textbf{Expliquer la différence entre adiabatique et
      isotherme}. Démontrer ensuite l'expression de l'enthalpie dans les
      conditions requises et réécrire le premier principe dans ces conditions.

    \item Définir les capacités thermiques à volume et pression constantes dans
      le cas général. Pour un gaz parfait, définir le coefficient adiabatique
      $\gamma$ et établir les expressions de $C_V$ et de $C_P$ en passant par la
      relation de \textsc{Mayer}. Introduire la capacité thermique d'une phase
      condensée et justifier, avec une application numérique, qu'on n'utilise
      qu'une capacité thermique pour les phases condensées.

    \item Calorimétrie~: dans un calorimètre parfaitement isolé de capacité
      thermique $C = \SI{100}{J.K^{-1}}$, on place $m_1 = \SI{150}{g}$ d'eau à
      $T_1 = \SI{298}{K}$. On ajoute $m_2 = \SI{100}{g}$ de cuivre à $T_2 =
      \SI{353}{K}$. Sachant que $c_{\rm Cu} = \SI{385}{J.K^{-1}.kg^{-1}}$ et
      $c_{\rm eau} = \SI{4185}{J.K^{-1}.kg^{-1}}$, déterminer $T_f$. Définir
      alors ce qu'est un thermostat et comment justifier leur existence.
      Calculer la variation d'entropie des différents éléments (on donne $\Delta
      S = C \ln \frac{T_f}{T_{i}}$ pour une phase condensée), et justifier que
      la transformation soit irréversible.

    \item[] \textbf{Chapitre 3}
    \item Présenter ce qu'on appelle une transformation irréversible. Énoncer
      donc le second principe de la thermodynamique et son lien avec
      l'irréversibilité. Que peut-on dire dans le cas particulier d'un cycle~?
      pour un système isolé~? Comment obtenir une transformation réversible~:
      exemple et vocabulaire associé (deux conditions).

    \item Énoncer les 3 lois de \textsc{Laplace} en précisant leurs conditions
      d'application. Comment qualifier ces transformations en terme d'entropie~?
      À partir d'une expression de l'entropie pour un GP (rappelée par
      l'interrogataire), démontrer l'une d'entre elle. Retrouver les deux autres
      à partir de celle-ci. Application~: on prend \SI{20}{L} de gaz à $T =
      \SI{293}{K}$ et à \SI{1}{bar}. Sous les conditions d'application
      précédentes, on le comprime jusqu'à un volume de \SI{10}{L}. Calculer la
      pression et la température, connaissant $\gamma = \num{1.4}$.

    \item Présenter le principe général d'une machine \textbf{ditherme}.
      Démontrer les deux relations utiles pour les machines à partir du premier
      et du second principe (inégalité de \textsc{Clausius}). Pourquoi ne
      peut-on pas réaliser de moteur monotherme~? Construire le diagramme de
      \textsc{Raveau} pour les machines dithermes, en précisant les domaines des
      moteurs et des réfrigérateurs.

    \item Présenter le moteur ditherme, le réfrigérateur ou la pompe à chaleur
      (au choix de l'interrogataire), en différenciant les sens conventionnel
      et réel des échanges. Définir son efficacité thermodynamique, et établir
      l'efficacité de \textsc{Carnot}.

    \item Cycle de \textsc{Carnot}~: définir les transformations, traduire le
      vocabulaire associé, le dessiner dans un diagramme $(P,V)$, trouver le
      travail total (on admet l'expression du travail pour une isotherme
      quasi-statique~: $W_{\rm isoT} = nRT_{\rm iso} \ln \left(
      V_{i}/V_{f} \right)$), la chaleur échangée et l'expression finale
      du rendement.

    \item[] \textbf{Chapitre 4}
    \item Présenter le diagramme $(P,T)$ des transitions de phase. Présenter
      l'expérience du cours permettant de tracer une transition gaz/liquide dans
      un diagramme $(P,V)$, et la tracer dans ce diagramme en faisant
      correspondre les différentes étapes sur la courbe. Présenter alors les
      isothermes d'\textsc{Andrews}.

\end{enumerate}
\vspace{-5pt}

\end{document}
