\documentclass[a4paper, 11pt]{book}
\usepackage{/home/nora/Documents/Enseignement/Prepa/bpep/fichiers_utiles/preambule}
\usepackage[french]{babel}

\newcommand{\qed}{\tag*{$\blacksquare$}}

\newcommand{\dsNB}{27}
\makeatletter
\renewcommand{\@chapapp}{Kh\^olles MPSI3 -- semaine \dsNB}
\makeatother

% \toggletrue{corrige}  % décommenter pour passer en mode corrigé

% IMPORTS automatiques
\tikzstyle simple4=[postaction={decorate,decoration={markings, mark=at position .5 with {\arrow[scale=1.2,>=latex]{>}}}}]
\tikzstyle simple7=[postaction={decorate,decoration={markings, mark=at position .8 with {\arrow[scale=1,>=latex]{>}}}}]
\newcommand{\croch}[1]{\ensuremath{\left[#1\right]}}
\newcommand{\f}[2]{{
		\mathchoice
		{\dfrac{#1}{#2}}
		{\dfrac{#1}{#2}}
		{\frac{#1}{#2}}
		{\frac{#1}{#2}}
}}
\newcommand{\e}[1]{
  _{\text{#1}}
}
\newcommand{\evt}{\phantom{\frac{\frac{\frac{1}{1}}{1}}{\frac{\frac{1}{1}}{1}}}}

% \newcommand{\e}[1]{{}_{\text{#1}}}
\renewcommand{\a}[0]{\alpha}
\newcommand{\w}[0]{\omega}

\usepackage{physics}

 % fin des IMPORTS automatiques

\begin{document}

\chapter{Sujet 1\siCorrige{\!\!-- corrigé}}

\resetQ
\subimport{/home/nora/Documents/Enseignement/Prepa/bpep/exercices/TD/moteur_reel/}{sujet.tex}

\chapter{Sujet 2\siCorrige{\!\!-- corrigé}}

\resetQ
\subimport{/home/nora/Documents/Enseignement/Prepa/bpep/exercices/TD/cycle_joule/}{sujet.tex}


\chapter{Sujet 3\siCorrige{\!\!-- corrigé}}

\resetQ
\subimport{/home/nora/Documents/Enseignement/Prepa/bpep/exercices/TD/performance_congelateur/}{sujet.tex}

\resetQ

\chapter{Sujet 4\siCorrige{\!\!-- corrigé}}

\resetQ
\subimport{/home/nora/Documents/Enseignement/Prepa/bpep/exercices/TD/pompe_a_chaleur_2/}{sujet.tex}

\resetQ

\chapter{Sujet 5\siCorrige{\!\!-- corrigé}}

\resetQ
\subimport{/home/nora/Documents/Enseignement/Prepa/bpep/exercices/Colle/pseudo_source/}{sujet.tex}

\end{document}
