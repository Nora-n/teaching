\documentclass[a4paper, 11pt]{book}
\usepackage{/home/nicolas/Documents/Enseignement/Prepa/bpep/fichiers_utiles/preambule}

\newcommand{\dsNB}{6}
\makeatletter
\renewcommand{\@chapapp}{Kh\^olles MPSI3 -- semaine \dsNB}
\makeatother

\toggletrue{corrige}  % décommenter pour passer en mode corrigé

\begin{document}

\resetQ
\newpage

\chapter{Sujet 1\siCorrige{\!\!-- corrig\'e}}
\section{Question de cours}

Présenter le circuit RC en décharge depuis une tension $E$ aux bornes du
condensateur (schéma et condition initiale), donner et démontrer l'équation
différentielle sur $u_C$, \textbf{démontrer} la solution et la tracer. Indiquer
sans le démontrer comment trouver la constante de temps et le régime permanent.

\resetQ

\subimport{/home/nicolas/Documents/Enseignement/Prepa/bpep/exercices/Colle/etincelle_rupture/}{sujet.tex}

\resetQ
\newpage

\chapter{Sujet 2\siCorrige{\!\!-- corrigé}}
\section{Question de cours}

Présenter le circuit RC en charge sous un échelon de tension $E$ (schéma et
condition initiale), donner et démontrer l'équation différentielle sur $u_C$,
donner la solution et la tracer. Indiquer sans le démontrer comment trouver la
constante de temps et le régime permanent.

\subimport{/home/nicolas/Documents/Enseignement/Prepa/bpep/exercices/Colle/intensite_debitee/}{sujet.tex}

\resetQ
\newpage

\chapter{Sujet 3\siCorrige{\!\!-- corrigé}}
\section{Question de cours}

Présenter le circuit LC soumis à un échelon de tension descendant (schéma et
condition initiale), donner et démontrer l'équation différentielle sur $u_C$,
donner \textbf{sans démontrer} la solution et la tracer.

\subimport{/home/nicolas/Documents/Enseignement/Prepa/bpep/exercices/Colle/loi_kirchhoff_3/}{sujet.tex}

\resetQ
\newpage

\chapter{Sujet 4\siCorrige{\!\!-- corrigé}}

\subimport{/home/nicolas/Documents/Enseignement/Prepa/bpep/exercices/TD/modelisation_pH_metre/}{sujet.tex}

\resetQ
\subimport{/home/nicolas/Documents/Enseignement/Prepa/bpep/exercices/Colle/batterie_tampon/}{sujet.tex}

\end{document}
