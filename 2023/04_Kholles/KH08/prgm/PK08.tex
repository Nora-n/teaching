\documentclass[a4paper, 12pt, final, garamond]{book}
\usepackage{cours-preambule}

\raggedbottom

\makeatletter
\renewcommand{\@chapapp}{Programme de kh\^olle -- semaine}
\makeatother

\begin{document}
\setcounter{chapter}{7}

\chapter{Du 14 au 18 novembre}

\section{Cours et exercices}
\section*{Électrocinétique chapitre 4 -- Oscillateurs harmonique et amorti}

\begin{enumerate}[label=\Roman*]
    \item \textbf{Introduction harmonique}~: description générale d'un signal
        sinusoïdal, équation différentielle d'un oscillateur harmonique et
        solution générale, changement de variable général $\rightarrow$
        homogène, exemple courbe expérimentale oscillateur LC.
    \item \textbf{Oscillateur harmonique électrique LC libre}~: présentation,
        équation différentielle, unité de $\w_0$, résolution avec 2 méthodes
        pour les constantes d'intégration, tracé de $u_C(t)$ et $i(t)$, aspect
        énergétique démonstration conservation et représentation graphique.
    \item \textbf{Oscillateur harmonique mécanique ressort libre}~: définition
        force de rappel, présentation, équation différentielle pour $\ell$ et
        $x$ et démonstration, analogie LC-ressort libre, aspect énergétique~:
        définition énergie potentielle élastique et mécanique, démonstration
        conservation, graphique et \textbf{visualisation dans l'espace des
        phases}.
    \item \textbf{Oscillateur harmonique électrique LC montant}~: présentation,
        équation différentielle, résolution avec changement de variable, tracé
        de $u_C(t)$ et $i(t)$ (et $u_L(t)$), représentation graphique uniquement
        de $\Ec_{\rm tot}$.
    \item \textbf{Introduction amorti}~: exemple RLC différents régimes ($Q =$
        13, 3, 0.5 et 0.2) et analyse, équation différentielle générale et
        analyse $Q$, définition équation caractéristique, discriminant et
        différents régimes, solutions générales.
    \item \textbf{Oscillateur amorti électrique RLC libre}~: présentation,
        \textbf{bilan énergétique} et analyse, équation différentielle et
        conditions initiales, solution, démonstrations, régimes transitoires à
        95\% et visualisation dans l'espace des phases \textbf{pour tous les
        régimes}, limite $Q \rightarrow \infty$.
    \item \textbf{Oscillateur amorti mécanique ressort frottements fluides}~:
        présentation et force de frottement fluide, équation différentielle et
        solution pour $\ell$ et $x$, analogie RLC-ressort amorti, sur toutes les
        grandeurs ($x$, $v$, $m$, $k$, $\alpha$, $\w_0$, $Q$), et résumé complet
        oscillateurs amortis.
\end{enumerate}


\section{Cours uniquement}
\section*{Chimie chapitre 1 -- Introduction}
\begin{enumerate}[label=\Roman*]
    \item \textbf{Vocabulaire général}~: atomes et molécules, classification par
        composition, états de la matière et systèmes physico-chimiques,
        transformations de la matière.
    \item \textbf{Quantification des systèmes}~: mole, masse molaire, fractions
        molaire et massique, masse volumique, concentrations molaire et
        massique, dilution~; pression d'un gaz, modèle du gaz parfait, volume
        molaire, pression partielle et loi de \textsc{Dalton}, intensivité et
        extensivité, activité d'un élément chimique.
\end{enumerate}

\section*{Chimie chapitre 2 -- Transformation et équilibre chimique}
\begin{enumerate}[label=\Roman*]
    \item \textbf{Avancement d'une réaction}~: présentation, avancements molaire
        et volumique, tableau d'avancement, coefficients stœchiométriques
        algébriques.
    \item \textbf{États final et d'équilibre d'un système chimique}~: réactions
        totales et limitées, exercice d'application.
\end{enumerate}


\section{Questions de cours possibles}
\begin{enumerate}
    \item Présenter le circuit RLC libre (schéma et conditions initiales),
        donner et \textbf{démontrer} l'équation différentielle sur $u_C$ sous
        forme canonique \textbf{qu'on ne cherchera pas à résoudre}, vérifier son
        homogénéité, présenter les graphiques des solutions selon les valeurs de
        $Q$ dans l'espace temporel \textbf{et} dans l'espace des phases ($u_C$,
        $i$) en donnant un approximation de la durée du régime transitoire à
        95\%.
    \item Présenter le ressort horizontal avec frottement fluide (schéma et
        conditions initiales), donner et \textbf{démontrer} l'équation
        différentielle sur $x$ ou $\ell$ sous forme canonique \textbf{qu'on ne
        cherchera pas à résoudre}, vérifier son homogénéité, présenter les
        graphiques des solutions selon les valeurs de $Q$ dans l'espace temporel
        \textbf{et} dans l'espace des phases ($u_C$, $i$) en donnant une
        approximation de la durée du régime transitoire à 95\%.

    \item Faire l'analogie complète entre les deux systèmes amortis RLC libre et
        ressort avec frottement fluide~: présentation, conditions initiales,
        équations différentielles \textbf{sans démonstration}, correspondance
        entre les grandeurs, tracer de solutions \textbf{dans l'espace des
        phases} selon différentes valeurs de $Q$ sans résolution et commenter
        sur l'évolution de l'énergie visible dans le graphique.

    \item Faire l'étude énergétique de l'oscillateur amorti électrique RLC libre
        et du ressort horizontal avec frottement fluide, identifier les termes
        du bilan et expliciter la signification physique de chacun des termes.

    \item Définir ce qu'est un corps pur, une phase, les 3 états de la matière,
        les différentes transformations de la matière avec des exemples.

    \item Définir la quantité de matière, la masse molaire et son lien avec la
        quantité de matière, les fractions molaire et massique, les
        concentrations molaire et massique et le lien entre les deux.

    \item Refaire l'exercice sur les fractions molaire et massique de dioxygène
        et diazote, l'exemple sur la concentration molaire de $\ce{Na+}$~:
\end{enumerate}
\begin{tcbraster}[raster columns=2, raster equal height=rows]
    \begin{NCexem}[width=\linewidth]{Exercice}
        L'air est constitué, en quantité de matière, à 80\% de diazote N$_2$ et
        à 20\% de dioxygène O$_2$. On a $M({\rm N_2}) = \SI{28.0}{g.mol^{-1}}$
        et $M({\rm O_2}) = \SI{32.0}{g.mol^{-1}}$. En déduire les fractions
        molaires puis les fractions massiques.
    \end{NCexem}
    \begin{NCexem}[]{Exercice}
        On dissout une masse $m = \SI{2.00}{g}$ de sel NaCl$\sol$ dans $V =
        \SI{100}{mL}$ d'eau. \textbf{Déterminer la concentration en
        Na$\plus{}$ dans la solution}. ($M(\ce{NaCl}) = \SI{58.44}{g.mol^{-1}}$)
    \end{NCexem}
\end{tcbraster}
\begin{enumerate}[resume]
    \item Savoir ajuster les deux équations suivantes (et toute autre équation
        proposée)~:
        \begin{align*}
            \ce{\ldots C6H12O6\liq{} + \ldots O2\gaz{}}
            &=
            \ce{\ldots CO2\gaz{} + \ldots H2O\gaz{}}\\
            \ce{\ldots I^-\aqu{} + \ldots Cr2O7\moin{2}\aqu{} + \ldots H\plus{}\aqu}
            &=
            \ce{\ldots Cr\plus{3}\aqu{} + \ldots I2\gaz{} + \ldots H2O\liq{}}
        \end{align*}

    \item Réaction totale et avancement maximal~: refaire l'exemple du cours sur
        la combustion du méthane $\ce{CH4\gaz{} + 2O2\gaz{} \rightarrow
        CO2\gaz{} + 2H2O\gaz{}}$ avec $n_{\ce{CH4}}^0 = \SI{2}{mol}$ et
        $n_{\ce{O2}}^0 = \SI{3}{mol}$.
\end{enumerate}

\section{Consignes}
\begin{framed}
    \begin{center}
        \Huge Les élèves doivent présenter des fiches des
        questions de cours à l'examinataire.
    \end{center}
\end{framed}

Une fiche doit~:
\begin{enumerate}
    \item Être \textbf{manuscrite} et bien sûr \textbf{personnelle}~;

    \item Comporter le nom du chapitre en haut de la première page puis le
        numéro de chaque page (les questions de cours relatives à
        \textbf{différents chapitres} doivent être sur \textbf{différentes
        fiches})~;

    \item L'intitulé \textbf{clairement identifiable} (couleur, encadré,
        centré, souligné, surligné…) de la question, éventuellement raccourci~:
        «~RLC libre différents $Q$ graphiques et $t_{95}$~» suffit pour la
        première question de cours~;

    \item Les schémas, hypothèses (conditions initales), résultats et
        démonstrations \textbf{expliquées}, le tout \textbf{correctement séparé}
        par des étapes \textbf{clairement identifiables} («~\underline{Loi des
        mailles}~: ») et avec des résultats \textbf{clairement mis en valeurs}.
\end{enumerate}

Le format (A4, A5, cartonné ou non) n'a aucune importance.

\begin{framed}
    \begin{center}
        \Huge La non-présentation des fiches entraînera une note de cours
        \textbf{maximale} à 5, et ce peu importe la qualité de la présentation
        orale, peu importe un éventuel oubli, sans concession aucune. \textbf{Le
        cours manuscrit n'est en aucun cas une fiche}, même bien écrit.
    \end{center}
\end{framed}

\end{document}
