\documentclass[a4paper, 11pt]{book}
\usepackage{/home/nicolas/Documents/Enseignement/Prepa/bpep/fichiers_utiles/preambule}

\newcommand{\dsNB}{9}
\makeatletter
\renewcommand{\@chapapp}{Kh\^olles MPSI3 -- semaine \dsNB}
\makeatother

\toggletrue{corrige}  % décommenter pour passer en mode corrigé

\begin{document}

\chapter{Sujet 1\siCorrige{\!\!-- corrig\'e}}
\section{Question de cours}

Donner les différentes expressions de l'activité d'un constituant selon sa
nature, exprimer le quotient de réaction d'une équation-bilan générale $0=\sum_i
\nu_i{\rm X}_i$ ou $\alpha_1{\rm R}_1 + \alpha_2{\rm R}_2 + … = \beta_1{\rm P}_1
+ \beta_2{\rm P}_2 + …$ et la constante d'équilibre associée, et exprimer $Q_r$
pour les réactions~:
\begin{enumerate}
    \item $\ce{2I^-\aqu{} + S2O8^{2-}\aqu{} = I2\aqu{} +2SO4^{2-}\aqu}$
    \item $\ce{Ag+\aqu{} + Cl^-\aqu{} = AgCl\sol}$
    \item $\ce{2FeCl3\gaz{} = Fe2Cl6\gaz{}}$
\end{enumerate}

\resetQ
\subimport{/home/nicolas/Documents/Enseignement/Prepa/bpep/exercices/Colle/synthese_methanol/}{sujet.tex}

\newpage

\chapter{Sujet 2\siCorrige{\!\!-- corrigé}}
\section{Question de cours}
\begin{NCexem}[width=\linewidth, breakable]{Exercice}
    Soit la réaction de l'acide éthanoïque avec l'eau~:
    \[\ce{CH3COOH\aqu{} + H2O\liq{} = CH3COO\moin{}\aqu{} + H3O\plus{}\aqu{}}\]
    de constante $K = \num{1.78e-5}$. On introduit $c = \SI{1.0e-1}{mol.L^{-1}}$
    d'acide éthanoïque et on note $V$ le volume de solution. \textbf{Déterminer
    la composition à l'état final}.
\end{NCexem}

\resetQ
\subimport{/home/nicolas/Documents/Enseignement/Prepa/bpep/exercices/Colle/rupture_equilibre_AgCl/}{sujet.tex}

\newpage

\chapter{Sujet 3\siCorrige{\!\!-- corrigé}}
\section{Question de cours}

Rupture d'équilibre~: refaire l'exercice d'application~:
\begin{NCexem}[width=\linewidth]{Exercice}

    Considérons la dissolution du chlorure de sodium, de masse molaire
    $M(\ce{NaCl}) = \SI{58.44}{g.mol^{-1}}$~:

    \centersright{$\ce{NaCl\sol{} = Na\plus{}\aqu{} + Cl\moin{}\aqu{}}$}{$K=33$}

    On introduit \SI{2.0}{g} de sel dans \SI{100}{mL} d'eau. \textbf{Déterminer
    l'état d'équilibre}.
\end{NCexem}

\resetQ
\subimport{/home/nicolas/Documents/Enseignement/Prepa/bpep/exercices/Colle/synthese_ammoniac/}{sujet.tex}

\newpage
\chapter{Sujet 4\siCorrige{\!\!-- corrigé}}

\resetQ
\subimport{/home/nicolas/Documents/Enseignement/Prepa/bpep/exercices/Colle/equilibre_chimique_Cu_Fe/}{sujet.tex}
\resetQ
\subimport{/home/nicolas/Documents/Enseignement/Prepa/bpep/exercices/Colle/equilibre_chimique_N2O4/}{sujet.tex}
\resetQ
\subimport{/home/nicolas/Documents/Enseignement/Prepa/bpep/exercices/Colle/Fluoration_dioxyde_uranium/}{sujet.tex}
\resetQ
\subimport{/home/nicolas/Documents/Enseignement/Prepa/bpep/exercices/Colle/constante_equilibre_gaz/}{sujet.tex}
\resetQ
\subimport{/home/nicolas/Documents/Enseignement/Prepa/bpep/exercices/Colle/constante_equilibre_HCl/}{sujet.tex}

\end{document}
