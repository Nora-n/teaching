\documentclass[a4paper, 11pt]{book}
\usepackage{/home/nicolas/Documents/Enseignement/Prepa/bpep/fichiers_utiles/preambule}
\usepackage{booktabs}

\newcommand{\dsNB}{11}
\makeatletter
\renewcommand{\@chapapp}{Kh\^olles MPSI3 -- semaine \dsNB}
\makeatother

% \toggletrue{corrige}  % décommenter pour passer en mode corrigé

\begin{document}

\chapter{Sujet 1\siCorrige{\!\!-- corrig\'e}}
\section{Question de cours}

Méthode des complexes en RSF~: donner la forme de réponse d'un système en RSF,
les relations entre les grandeurs réelle et complexe associée, l'intérêt pour la
dérivation et le lien entre une équation différentielle réelle et l'équation
algébrique complexe associée.

\resetQ
\subimport{/home/nicolas/Documents/Enseignement/Prepa/bpep/exercices/TD/impedance/}{sujet.tex}

\chapter{Sujet 2\siCorrige{\!\!-- corrig\'e}}
\section{Question de cours}

Circuit RC série~: présenter le système réel, le système en RSF complexe,
déterminer l'amplitude complexe sur la tension du condensateur ainsi que son
amplitude réelle et sa phase.

\resetQ
\subimport{/home/nicolas/Documents/Enseignement/Prepa/bpep/exercices/Colle/circuit_RSF_2/}{sujet.tex}

\chapter{Sujet 3\siCorrige{\!\!-- corrig\'e}}
\section{Question de cours}

Donner \textbf{et démontrer} les relations des ponts diviseur de tension et
diviseur de courant.

\resetQ
\subimport{/home/nicolas/Documents/Enseignement/Prepa/bpep/exercices/TD/impedance_equivalente/}{sujet.tex}

\chapter{Sujet 4\siCorrige{\!\!-- corrig\'e}}

\resetQ
\subimport{/home/nicolas/Documents/Enseignement/Prepa/bpep/exercices/TD/impedance_2/}{sujet.tex}

\resetQ
\subimport{/home/nicolas/Documents/Enseignement/Prepa/bpep/exercices/TD/impedance_reellee/}{sujet.tex}

\end{document}
