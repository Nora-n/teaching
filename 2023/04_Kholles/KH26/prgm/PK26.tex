\documentclass[a4paper, 12pt, final, garamond]{book}
\usepackage{cours-preambule}

\raggedbottom

\makeatletter
\renewcommand{\@chapapp}{Programme de kh\^olle -- semaine}
\makeatother

\begin{document}
\setcounter{chapter}{25}

\chapter{Du 09 au 13 mai}

\section{Cours et exercices}

\section*{Thermo. chapitre 1 -- Description d'un système à l'équilibre}
\begin{enumerate}[label=\Roman*]
  \litem{Introduction}~: ordres de grandeur, échelles de description.
  \litem{Grandeur d'état}~: définition, température, pression, autres exemples,
    variables extensives et intensives, grandeurs massiques et volume molaire.
  \litem{Description d'un gaz}~: comportement microscopique, loi du gaz parfait,
    diagramme de \textsc{Clapeyron}.
  \litem{Cas des phases condensées}~: définition, équation d'état, ordres de
    grandeur.
\end{enumerate}

\section*{Thermodynamique chapitre 2 -- Premier principe}
\begin{enumerate}[label=\Roman*]
  \litem{Vocabulaire}~: système, transformations et exemples.
  \litem{Énergie interne}~: définition, énergie interne gaz parfait, capacité
    thermique à volume constant, capacité thermique d'une phase condensée.
  \litem{Travail des forces de pression}~: expression générale, cas particulier
    isochore et monobare, transformation quasi-statique (définition, exemples de
    diagrammes, corrélation avec l'aire sous la courbe~; application cycle de
    \textsc{Lenoir}), travail électrique.
  \litem{Transferts thermiques}~: définition, différents types de transferts
    thermiques, cas particuliers (adiabatique, thermostat), loi de
    \textsc{Laplace}.
  \litem{Premier principe de la thermodynamique}~: énoncé, application $Q$ cycle
    de \textsc{Lenoir}.
  \litem{Transformation monobare et enthalpie}~: démonstration $H = U+PV$,
    introduction $C_{P}$~: gaz parfait ($\gamma$, \textsc{Mayer}, $C_{V}$ et
    $C_P$ fonction de $\gamma$, lois de \textsc{Joule}) et phases condensées~;
    application calorimètre~; retour sur principe du thermostat.
\end{enumerate}

\section{Cours uniquement}
\section*{Thermo. chapitre 3 -- Second principe, machines thermiques}
\begin{enumerate}[label=\Roman*]
  \litem{L'entropie}~: irréversibilité, second principe et cas particuliers,
    comment faire une transformation réversible, expressions de l'entropie~:
    phases condensées, gaz parfait, démonstration lois des \textsc{Lapace}.
  \litem{Machines thermiques}~: introduction, principe général, machine
    monotherme.
\end{enumerate}

\section{Questions de cours possibles}

\begin{enumerate}[label=\sqenumi]
    \item[] \textbf{Chapitre 1}
    \item Représenter la distribution des vitesses des molécules d'un gaz et ses
      propriétés, définir la vitesse quadratique moyenne et la température
      cinétique.

    \item Refaire l'exercice~:
      On considère une bouteille de volume constant $V = \SI{10}{L}$ contenant
      de l'hélium, modélisé comme un gaz parfait monoatomique, à la pression $p
      = \SI{2.1}{bar}$ et à la températire $T = \SI{300}{K}$. On donne $M_{\rm
      He} = \SI{4.0}{g.mol^{-1}}$, $k_B = \SI{1.38e-23}{J.K^{-1}}$ et
      $\mathcal{N}_A = \SI{6.02e23}{mol^{-1}}$.
      \begin{enumerate}
        \item Calculer la masse $m$ d'hélium contenue dans la bouteille, et la
          densité particulaire $n^*$, c'est-à-dire le nombre d'atomes par unité
          de volume.
        \item Calculer la vitesse quadratique moyenne $u$ des atomes.
        \item À la suite de l'ouverture de la outeille, la pression passe à $p'
          = \SI{1.4}{bar}$ et la température à $T' = \SI{290}{K}$. Calculer la
          masse $\D m$ de gaz qui s'est échappé de la bouteille.
        \item À quelle température $T''$ faudrait-il porter le gaz pour
          atteindre à nouveau la pression $p$~?
      \end{enumerate}

    \item[] \textbf{Chapitre 2}
    \item Présenter le vocabulaire de la thermodynamique~: système isolé, fermé,
      ouvert~; transformations, transformations isochore, monotherme, isotherme,
      monobare, isobare, \textbf{adiabatique} et \textbf{quasi-statique}.
      Définir l'énergie interne d'un système et l'échelle à laquelle elle se
      définit, puis déterminer l'énergie interne d'un gaz parfait monoatomique
      et diatomique, en justifiant les facteurs numériques.

    \item Définir les capacités thermiques à volume et pression constantes dans
      le cas général. En citant les lois de \textsc{Joule}, donner leurs
      expressions pour un gaz parfait monoatomique. Introduire la capacité
      thermique d'une phase condensée et justifier, avec une application
      numérique, qu'on n'utilise qu'une capacité thermique pour les phases
      condensées.

    \item Établir l'expression générale du travail des forces de pression.
      Préciser la nature du système (moteur, récepteur) selon le signe de $W$.
      Présenter le lien avec l'aire sous la courbe d'un diagramme de
      \textsc{Clapeyron}, et mettre en évidence la dépendance de $W$ au chemin
      suivi. Donner la valeur ou l'expression de $W$ pour une transformation
      isochore, pour une transformation monobare, et pour une transformation
      quasi-statique isotherme d'un gaz parfait.

    \item Cycle de \textsc{Lenoir}~: pour une mole de gaz parfait à $P_\Ar =
      \SI{2e5}{Pa}$ et $V_\Ar = \SI{14}{L}$, on effectue les transformations
      suivantes de manière quasi-statique~:
      \begin{enumerate}[label=\alph*)]
        \item chauffage isochore jusqu'à $P_{\rm B} = \SI{4e5}{Pa}$~;
        \item détente isotherme jusqu'à $V_{\rm C} = \SI{28}{L}$~;
        \item refroidissement isobare jusqu'au retour à l'état initial.
      \end{enumerate}
      Calculer $P$, $V$ et $T$ à chaque étape, puis représenter ce cycle sur un
      diagramme de \textsc{Clapeyron}, et calculer les travaux associés aux
      transformations AB, BC et CA et sur le cycle. Conclure sur la nature du
      système. Comment la repérer graphiquement~?

    \item Énoncer les 3 lois de \textsc{Laplace} en précisant leurs conditions
      d'application. À partir d'une expression de l'entropie pour un GP
      (rappelée par l'interrogataire), démontrer l'une d'entre elle. Retrouver
      les deux autres à partir de celle-ci. Application~: on prend \SI{20}{L} de
      gaz à $T = \SI{293}{K}$ et à \SI{1}{bar}. Sous les conditions
      d'application précédentes, on le comprime jusqu'à un volume de \SI{10}{L}.
      Calculer la pression et la température, connaissant $\gamma = \num{1.4}$.

    \item Énoncer le premier principe de la thermodynamique, en détaillant les
      termes. Préciser lesquels sont des fonctions d'état, lesquels ne le sont
      pas. Démontrer ensuite l'expression de l'enthalpie dans les conditions
      requises et réécrire le premier principe dans ces conditions.

    \item Calorimétrie~: dans un calorimètre parfaitement isolé de capacité
      thermique $C = \SI{100}{J.K^{-1}}$, on place $m_1 = \SI{150}{g}$ d'eau à
      $T_1 = \SI{298}{K}$. On ajoute $m_2 = \SI{100}{g}$ de cuivre à $T_2 =
      \SI{353}{K}$. Sachant que $c_{\rm Cu} = \SI{385}{J.K^{-1}.kg^{-1}}$ et
      $c_{\rm eau} = \SI{4185}{J.K^{-1}.kg^{-1}}$, déterminer $T_f$. Définir
      alors ce qu'est un thermostat et comment justifier leur existence.

    \item[] \textbf{Chapitre 3}
    \item Présenter ce qu'on appelle une transformation irréversible. Énoncer
      alors le second principe de la thermodynamique et son lien avec
      l'irréversibilité. Que peut-on dire dans le cas particulier d'un cycle~?
      pour un système isolé~? Comment obtenir une transformation réversible~:
      exemple et vocabulaire associé (deux conditions).

\end{enumerate}
\vspace{-5pt}



\end{document}
