\documentclass[a4paper, 10pt, final, garamond]{book}
\usepackage{cours-preambule}
\graphicspath{{./figures/}}

\makeatletter
\renewcommand{\@chapapp}{Contr\^ole de connaissances}
\makeatother

% \toggletrue{student}
% \toggletrue{corrige}
% \renewcommand{\mycol}{black}
% \renewcommand{\mycol}{gray}

\begin{document}
\setcounter{chapter}{25}

\settype{enon}
\settype{solu}

\chapter{Premier principe de la thermodynamique\ifstudent{~(13')}}

\begin{enumerate}[label=\sqenumi]
	\item[n]{6}
	Donner des conditions pour réaliser une transformation isotherme. Donner
	des conditions pour réaliser une transformation adiabatique. Expliquer
	succinctement la différence.
	\smallbreak
	\vspace{-15pt}
	\psw{%
		\begin{itemize}
			\item On réalise une transformation isotherme avec des parois
			      \textbf{diathermanes} \pt{1}, en \textbf{évolution lente} \pt{1} au
			      contact d'un \textbf{thermostat}. \pt{1}
			\item On réalise une transformation adiabatique avec des parois
			      \textbf{calorifugées} \pt{1} et une \textbf{transformation rapide}.
			      \pt{1}
		\end{itemize}
		Les deux modèles sont diamétralement opposés~: isotherme n'est possible
		que grâce aux échanges de chaleur, adiabatique c'est leur absence. Le plus
		souvent, $Q = 0 \Ra \Delta{T} \neq 0$. \pt{1}
	}%
	\item[n]{6}
	Énoncer les conditions du premier principe enthalpique, puis le démontrer.
	\smallbreak
	\psw{%
		Pour un transformation \textbf{monobare} avec  \textbf{équilibre mécanique
			dans l'état initial et final} ($P_i = P_f = P\ind{ext}$), on a~:
		\psw{%
			\begin{DispWithArrows*}[groups]
				\Delta{U} &\stm[-1]{=} W_p + W_u + Q
				\Arrow{$P\ind{ext} = P_0$ \pt{1}}
				\Arrow[jump=3,tikz-code=
						{\draw
							(#1) --++
							(3cm,0) |-
							node[pos=.25, sloped] {#3}
							(#2) ;
						}]{On combine}
				\\\text{Or,~~}
				W_p &\stm[-1]{=} -P_0(V_f - V_i)
				\Arrow{$P_i = P_f = P_0$ \pt{1}}
				\\\Ra
				W_p & = -P_f V_f + P_iV_i
				\Arrow{$W_p = -\Delta{PV}$}
				\\\Lra
				\Delta{U} &\stm[-1]{=} - \Delta{(PV)} + W_u + Q
				\\\Lra
				\Aboxed{\Delta{H} &= W_u + Q} \pt{1}
				\qed
			\end{DispWithArrows*}
		}%
	}%
	\item[n]{8}
	Dans un calorimètre parfaitement isolé de masse en eau $m_0 = \SI{24}{g}$, on
	place $m_1 = \SI{150}{g}$ d'eau à $T_1 = \SI{298}{K}$. On ajoute $m_2 =
		\SI{100}{g}$ de cuivre à $T_2 = \SI{353}{K}$, avec $c_{\ce{Cu}} =
		\SI{385}{J.K^{-1}.kg^{-1}}$. On cherche la température d'équilibre $T_f$.
	\begin{enumerate}[label=\sqenumi]
		\item Exprimer $\Delta{H}\ind{eau}$ en fonction de $m_1$, $c\ind{eau}$,
		      $T_1$ et $T_f$.
		\item Exprimer $\Delta{H}_{\ce{Cu}}$ en fonction de $m_2$, $c_{\ce{Cu}}$,
		      $T_2$ et $T_f$.
		\item Exprimer $\Delta{H}\ind{calo}$ en fonction de $m_0$, $c\ind{eau}$,
		      $T_1$ et $T_f$.
		\item Justifier que $\Delta{H}\ind{tot} = 0$.
		\item En déduire $T_f$.
	\end{enumerate}
	\smallbreak
	\begin{enumerate}[label=\sqenumi]
		\item[m]
			\psw{%
				\begin{gather*}
					\boxed{\Delta{H}\ind{eau} = m_1c\ind{eau} (T_f - T_1)}
					\hspace{10pt}
					\pt{1}
				\end{gather*}
			}%
		\item[m]
			\psw{%
				\begin{gather*}
					\boxed{\Delta{H}_{\ce{Cu}} = m_2c_{\ce{Cu}} (T_f - T_2)}
					\hspace{10pt}
					\pt{1}
				\end{gather*}
			}%
		\item[m]
			\psw{%
				\begin{gather*}
					\boxed{\Delta{H}\ind{calo} = m_0c\ind{eau} (T_f - T_1)}
					\hspace{10pt}
					\pt{1}
				\end{gather*}
			}%
		\item \psw{%
			      Calorimètre isolé donc $Q = 0$ \pt{1}, et pas d'autres travaux donc
			      $W_u = 0$~:
			      \[
				      \Delta{H} = W_u + Q \Lra \boxed{\Delta{H}\ind{tot} = 0}
				      \hspace{10pt}
				      \pt{1}
			      \]
		      }%
		\item [m][20]
		      \psw{%
			      \begin{align*}
				      (m_1 + m_0)c\ind{eau} (T_f - T_1) + m_2c_{\ce{Cu}}(T_f - T_2)
				          & = 0
				      \\\Lra
				      T_f \left( (m_1+m_0)c\ind{eau} + m_2c_{\ce{Cu}} \right)
				          & \stm[-1]{=}
				      T_1 \left( m_1+m_0 \right)c\ind{eau} + T_2m_2c_{\ce{Cu}}
				      \\\Lra
				      \Aboxed{
				      T_f & \stm[-1]{=}
				      \frac{\left( m_1+m_0 \right)c\ind{eau}T_1 + m_2c_{\ce{Cu}}T_2}
				      {(m_1+m_0)c\ind{eau} + m_2c_{\ce{Cu}}}
				      }
				      \\
				      \makebox[0pt][l]{$\phantom{\AN}\xul{\phantom{T_f = \SI{301}{K}}}$}
				      \AN
				      T_f & \stm{=} \SI{301}{K}
			      \end{align*}
		      }%
	\end{enumerate}
\end{enumerate}
\vspace{-15pt}
\ifstudent{
	\begin{tikzpicture}[remember picture, overlay]
		\node[anchor=north west, align=left]
		at ([shift={(1.4cm,0)}]current page.north west)
		{\\[5pt]\Large\bfseries Nom~:\\[10pt]\Large\bfseries Prénom~:};
		\node[anchor=north east, align=right]
		at ([shift={(-1.5cm,-17pt)}]current page.north east)
		{\Large\bfseries Note~:\hspace{1cm}/20};
	\end{tikzpicture}
}
\end{document}
