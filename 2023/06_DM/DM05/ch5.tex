\documentclass[a4paper, 11pt, oneside]{book}
\usepackage{/home/nicolas/Documents/Enseignement/Prepa/bpep/fichiers_utiles/preambule}

\newcommand{\dsNB}{4}
\makeatletter
\renewcommand{\@chapapp}{Devoir maison \dsNB\ -- à rendre pour le 27 février}
\makeatother

% \toggletrue{corrige}  % décommenter pour passer en mode corrigé
\newcommand{\f}[2]{{
		\mathchoice
		{\dfrac{#1}{#2}}
		{\dfrac{#1}{#2}}
		{\frac{#1}{#2}}
		{\frac{#1}{#2}}
}}
\newcommand{\e}[1]{{}_{\text{#1}}}
\begin{document}

\chapter{Pendule asymétrique\siCorrige{\!\!-- corrigé}}

\switch{
    \begin{center}
        Ce sujet comporte~\pageref{LastPage} pages et doit être traité en
        intégralité. Comme pour tous DMs, vous pouvez vous entraider pour les
        questions les plus difficiles. Cependant, \textbf{la rédaction doit
        rester personnelle}.
    \end{center}
}{ }

% la ligne suivante (contenant uniquement 4x* ) sera remplacée par les imports de vos exercices dans l'IHM
\resetQ
\subimport{/home/nicolas/Documents/Enseignement/Prepa/bpep/exercices/TD/pendule_asymetrique/}{sujet.tex}

\label{LastPage}

\end{document}
