\documentclass{standalone}
\usepackage{mintikz}

\begin{document}
\begin{tikzpicture}[scale=0.5]
	\def\xtr{19}
	\def\ytr{6}
	\node[] (Cp) at (-5.5,5.5) {$\oplus$};
	\draw[<->] (Cp.north west) |- (Cp.south east);
	% Sens de comptage
	\node[] (Cp) at (-3.5,5.5) {$\oplus$};
	\circledarrow{black}{(Cp)}{.5}
	% grid
	\draw[lightgray!10, ultra thin, help lines]
	(-\xtr,-\ytr) grid (8,\ytr);
	\draw[lightgray!10, ultra thin, help lines, dotted, step=.5cm]
	(-\xtr,-\ytr) grid (8,\ytr);
	% Axe optique
	\draw[thin, ->](-\xtr,0)--(8,0)node[below]{A.O.};
	% Lentille 1
	\coordinate (O1) at (-\xtr+3,0);
	\def \fa{16}
	\def \lsiz{5}
	\coordinate (F1) at ($(O1)+(-\fa,0)$);
	\coordinate (F1p) at ($(O1)+(\fa,0)$);

	\foreach \x/\z in {O1/\fa}{
			\node[] at (\x) {$\times$};
			\node[below right] at (\x) {$O_1$};
			\node[] at ([shift={(\z,0)}]\x) {$\times$};
			\node[below left] at ([shift={(\z,0)}]\x) {$F'_1$};
			%\node[] at ([shift={(-\z,0)}]\x) {$\times$};
			%\node[below] at ([shift={(-\z,0)}]\x) {$F_1$};
		}

	\draw[shift={(O1)}, ultra thick,
		<->, >=latex, name path=L1]
	(0,-\lsiz) coordinate (L1bot) --
	(0,\lsiz) coordinate (L1top);
	% node[above]{$\Lc_{1}$};
	% Monture
	\draw[line width=3pt] (L1top) --++ (0,\lsiz/3)
	node[right]{Monture de l'objectif};
	\draw[line width=3pt] (L1bot) --++ (0,-\lsiz/3)
	node[right]{Monture de l'objectif};
	% Lentille 2
	\coordinate (O2) at (2,0);
	\def \fb{2}
	\def \lsiz{5}
	\coordinate (F2) at ($(O2)+(-\fb,0)$);
	\coordinate (F2p) at ($(O2)+(\fb,0)$);

	\foreach \x/\z in {O2/\fb}{
			\node[] at (\x) {$\times$};
			\node[below right] at (\x) {$O_2$};
			\node[] at ([shift={(\z,0)}]\x) {$\times$};
			\node[below right] at ([shift={(\z,0)}]\x) {$F'_2$};
			\node[] at ([shift={(-\z,0)}]\x) {$\times$};
			\node[above right] at ([shift={(-\z,0)}]\x) {$F_2$};
		}

	\draw[shift={(O2)}, ultra thick,
		<->, >=latex, name path=L2] (0,-\lsiz)--(0,\lsiz)
	node[above]{$\Lc_{2}$};
	% Pinceau entrée
	\coordinate (It) at ([shift={(0,\lsiz)}]O1);
	\node at (It) {$\times$};
	\node[right] at (It) {M};
	\coordinate (Ib) at ([shift={(0,-\lsiz)}]O1);
	\node at (Ib) {$\times$};
	\node[right] at (Ib) {N};
	\coordinate (At) at ([shift={(180:5)}]It);
	\coordinate (Ab) at ([shift={(180:5)}]Ib);
	\draw[Purple!70, simple, thick] (At) -- (It);
	\draw[Purple!70, double, thick] (Ab) -- (Ib);
	% D1
	\draw[dashed, <->, >=latex, thick]
	($(Ab)+(1,0)$) --
	node[pos=.75, left]{$D_1$} ($(At)+(1,0)$);
	% % Rayon O1
	% \coordinate (A) at ([shift={(-6,0)}]O1);
	% \path[Brown!50] (A) -- (O1);
	% % Plan focal
	% \def\pfsiz{5}
	% \draw[shift={(F1p)}, dashed, thick,
	% 	gray!70, name path=PF1p] ($(F1p)+(0,-\pfsiz/2)$)
	% --++ (0,\pfsiz) node[above] {$\pi_1' = \pi_2$};
	% % Sortie rayon O1
	% \path[name path=B1path, draw] (O1) --++ ($4*(O1)-4*(A)$);
	% % Intersection plan focal
	% \path[name intersections={of=B1path and PF1p, by=B1F1p}];
	% \draw[] (B1F1p) node {+};
	% % Intersection lentille 2
	% \path[name intersections={of=B1path and L2, by=I2}];
	% % Draw
	% \path[Brown!50] (O1) -- (I2);
	% % Sortie pinceau
	% \path[name path=Atpath] (It) --++ ($1.3*(B1F1p)-1.3*(It)$);
	% \path[name path=Abpath] (Ib) --++ ($1.3*(B1F1p)-1.3*(Ib)$);
	% % Define intersections
	% \path[name intersections={of=Atpath and L2, by=It2}];
	% \path[name intersections={of=Abpath and L2, by=Ib2}];
	% % Draw
	% \draw[brandeisblue, simple, thick] (It) -- (It2);
	% \draw[brandeisblue, double, thick] (Ib) -- (Ib2);
	% % Name points
	% \node at (It2) {$\times$};
	% \node[below right] at (It2) {Q};
	% \node at (Ib2) {$\times$};
	% \node[above right] at (Ib2) {P};
	% % Image intermédiaire
	% \coordinate (A1) at (F1p);
	% \coordinate (B1) at (B1F1p);
	% \draw[brandeisblue, thick]
	% (A1) node [above left] {} --
	% (B1) node [below left] {};
	% % Rayons pour objet sur le foyer d'une lentille convergente
	% % Lengths
	% \len{(A1)}{(O2)}{\posi}
	% \len{(A1)}{(B1)}{\size}
	% \len{(O2)}{(F2)}{\foc}
	% % Rayon 1
	% \path[ForestGreen] (B1) -- (O2);
	% \def \mut{1}
	% \path[orange!70, name path=OBp]
	% (O2) --++ ($\mut*(O2)-\mut*(B1)$) coordinate (end);
	% % Rayon 2
	% \path[ForestGreen]
	% (B1) --++ (\posi,0) coordinate (I);
	% \def \mut{2}
	% \path[orange!70, name path=IBp]
	% (I) --++ ($\mut*(F2p)-\mut*(I)$);
	% % Sortie pinceau final
	% \def\mut{2.5}
	% \draw[Red!70, simple, thick]
	% (It2) --++
	% ($\mut*(O2)-\mut*(B1)$) coordinate (Itf);
	% \draw[Red!70, double, thick]
	% (Ib2) --++
	% ($\mut*(O2)-\mut*(B1)$) coordinate (Ibf);
	% % D
	% \draw[dashed, <->, >=latex, thick]
	% ($(Itf)-(1,0)$) --
	% node[pos=.75, right]{$D$} ($(Ibf)-(1,0)$);
	% % Actual pinceau
	% % \pattern[pattern=north east hatch,
	% % 	hatch distance=0.3cm, hatch thickness=0.5pt,
	% % 	pattern color=Purple!70]
	% % (At) -- (It) -- (Ib) -- (Ab) -- cycle;
	% % \pattern[pattern=north east hatch,
	% % 	hatch distance=0.31cm, hatch thickness=0.5pt,
	% % 	pattern color=brandeisblue]
	% % (It) -- (It2) -- (Ib2) -- (Ib) -- (It);
	% % \pattern[pattern=north east hatch,
	% % 	hatch distance=0.1cm, hatch thickness=0.5pt,
	% % 	pattern color=Red!70]
	% % (It2) -- (Itf) -- (Ibf) -- (Ib2) -- cycle;
	% % Angles
	% % Draw angles
	% % \pic[angle radius=0.5cm, angle eccentricity=1.5,
	% % 	draw, "$\theta'$" alias=thetap1,
	% % 	->, color=Pink]
	% % {angle=F2p--O2--end};
	% % \pic[angle radius=0.5cm, angle eccentricity=1.5,
	% % 	draw, "$\theta'$" alias=thetap2,
	% % 	->, color=Pink]
	% % {angle=F2--O2--B1};
	% % \pic[angle radius=1cm, angle eccentricity=1.2,
	% % 	draw, "$\theta$" alias=theta1,
	% % 	<-, color=Pink]
	% % {angle=B1--O1--F2};
	% % \coordinate (O1l) at ([shift={(-3,0)}]O1);
	% % \pic[angle radius=1cm, angle eccentricity=1.2,
	% % 	draw, "$\theta$" alias=theta2,
	% % 	<-, color=Pink]
	% % {angle=A--O1--O1l};
\end{tikzpicture}
\end{document}
