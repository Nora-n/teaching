\documentclass[a4paper, 10pt, garamond]{book}
\usepackage{cours-preambule}
\usepackage{tocloft}

\renewcommand{\mtcSfont}{\Large}
\setlength{\mtcindent}{5pt}
\mtcsetoffset{minitoc}{-5pt}
\addtolength{\cftsecnumwidth}{10pt}
\makeatletter
\makeatother
\dominitoc
\faketableofcontents

% \toggletrue{student}
\HideSolutionstrue
\toggletrue{corrige}
% \renewcommand{\mycol}{black}
\renewcommand{\mycol}{gray}

\makeatletter
\renewcommand{\@chapapp}{MPSI3 -- 24 novembre 2023 -- Devoir surveillé}
\makeatother

\graphicspath{{./figures/}{./figures/E1}{./figures/E2}{./figures/P1}{./figures/P2}}

\newcommand{\figsvg}[1]{
  \begin{center}
    \subimport{figures/}{#1}
  \end{center}
}
\newcommand{\figsvgCap}[2]{
  \begin{center}
    \subimport{figures/}{#1}
    \captionof{figure}{#2}
  \end{center}
}

\setlist[blocQR,1]{leftmargin=10pt, label=\sqenumi}
\counterwithin*{equation}{section}

\begin{document}
\setcounter{chapter}{2}
\chapter{\cswitch{Correction du DS}{Oscillateurs et transformation de la
	  matière}}
\label{ch:ds03}

\enonce{
	\begin{center}
		\Large\bfseries
		Tout moyen de communication est interdit
		\smallbreak
		Les téléphones portables doivent être éteints et rangés dans les sacs
		\smallbreak
		\xul{Les calculatrices sont \textit{autorisées}}
	\end{center}
	\begin{tcb}*[cnt, bld](ror)<itc>{Au programme}
		\large
		Régimes transitoires d'ordre 2 (mécanique et électricité), transformation et
		équilibre chimique.
	\end{tcb}
	\vfill
	\minitoc
	\vfill

	Les différentes questions peuvent être traitées dans l'ordre désiré.
	\textbf{Cependant}, vous indiquerez le numéro correct de chaque question. Vous
	prendrez soin d'indiquer sur votre copie si vous reprenez une question d'un
	exercice plus loin dans la copie, sous peine qu'elle ne soit ni vue ni corrigée.
	\bigbreak
	Vous porterez une attention particulière à la \textbf{qualité de rédaction}.
	Vous énoncerez clairement les hypothèses, les lois et théorèmes utilisés. Les
	relations mathématiques doivent être reliées par des connecteurs logiques.
	\bigbreak
	Vous prendre soin de la \textbf{présentation} de votre copie, notamment au
	niveau de l'écriture, de l'orthographe, des encadrements, de la marge et du
	cadre laissé pour la note et le commentaire. Vous \textbf{encadrerez les
		expressions littérales}, sans faire apparaître les calculs. Vous ferez
	apparaître cependant le détail des grandeurs avec leurs unités. Vous
	\textbf{soulignerez les applications numériques}.
	\bigbreak
	Ainsi, l'étudiant-e s'expose aux malus suivants concernant la forme et le fond~:
	\begin{tcb}*(prop)"bomb"{Malus}
		\begin{minipage}[t]{0.50\linewidth}
			\begin{itemize}
				\item A~: application numérique mal faite~;
				\item N~: numéro de copie manquant~;
				\item P~: prénom manquant~;
				\item E~: manque d'encadrement des réponses~;
				\item M~: marge non laissée ou trop grande~;
				\item V~: confusion ou oubli de vecteurs~;
			\end{itemize}
		\end{minipage}
		\begin{minipage}[t]{0.50\linewidth}
			\begin{itemize}
				\item Q~: question mal ou non indiquée~;
				\item C~: copie grand carreaux~;
				\item U~: mauvaise unité (flagrante)~;
				\item H~: homogénéité non respectée~;
				\item S~: chiffres significatifs non cohérents~;
				\item $\f$~: loi physique fondamentale brisée.
			\end{itemize}
		\end{minipage}
	\end{tcb}

	\begin{tcb}(impo){Exemple application numérique}
		\vspace*{-10pt}
		\begin{minipage}[c]{0.45\linewidth}
			\begin{gather*}
				\boxed{n = \frac{PV}{RT}}
				\qav
				\left\{
				\begin{array}{rcl}
					p & = & \SI{1.0e5}{Pa}                \\
					V & = & \SI{1.0e-3}{m^3}              \\
					R & = & \SI{8.314}{J.mol^{-1}.K^{-1}} \\
					T & = & \SI{300}{K}
				\end{array}
				\right.\\
				\mathrm{A.N.~:}\quad
				\xul{n = \SI{5.6e-4}{mol}}
			\end{gather*}
		\end{minipage}
		\hfill
		\cancel{\bcancel{
				\begin{minipage}[c]{0.45\linewidth}
					\begin{gather*}
						n = \frac{PV}{RT} = \frac{\num{e5}\cdot\num{1}}{8.32\cdot300}
						= 0.56
					\end{gather*}
				\end{minipage}
			}}
	\end{tcb}
	\newpage
}

\setcounter{section}{0}
\exercice[30]{Pentachlorure de phosphore}
\resetQ
\enonce{
Le pentachlorure de phosphore \ce{PCl5} est un composé très toxique, servant
de réactif en synthèse organique pour ajouter des atomes de chlore à une
chaîne carbonée. Mis en phase gazeuse, il se décompose spontanément en
trichlorure de phosphore et en dichlore, donnant naissance à un équilibre en
phase gazeuse.

Considérons un réacteur fermé de volume constant $V = \SI{2}{L}$ maintenu à
température constante $T = \SI{180}{\celsius}$. À cette température, la
constante thermodynamique de l’équilibre précédemment cité vaut $K^\circ =
	8$. On y met $n_0 = \SI{0,5}{\mol}$ de \ce{PCl5}.

On rappelle que $R = \SI{8.314}{J.K^{-1}.mol^{-1}}$.
}

\QR[1]{%
	Exprimer puis calculer la pression initiale dans le réacteur $p_0$ en fonction
	des données.
}{%
	On applique la loi des gaz parfaits~:
	\begin{gather*}
		\boxed{p_0=\frac{n_0RT}{V}}
		\qav
		\left\{
		\begin{array}{rcl}
			n_0 & = & \SI{0.5}{mol}
			\\
			R   & = & \SI{8.314}{J.K^{-1}.mol^{-1}}
			\\
			T   & = & \SI{453.15}{K}
			\\
			V   & = & \SI{2e-3}{m^{3}}
		\end{array}
		\right.\\
		\AN
		\xul{
			p_0 = \SI{9.41e5}{Pa}
		}
	\end{gather*}
}

\QR[1]{%
	Écrire l’équation de réaction modélisant le processus dans le réacteur, et
	dresser le tableau d'avancement correspondant.
}{%
	\begin{center}
		\def\rhgt{0.35}
		\centering
		\begin{tabularx}{.7\linewidth}{|l|c||YdYdY||Y|}
			\hline
			\multicolumn{2}{|c||}{%
				$\xmathstrut{\rhgt}$
			\textbf{Équation}}  &
			$\ce{PCl_5\gaz{}}$  & $=$                   &
			$\ce{PCl_3\gaz{}}$  & $+$                   &
			$\ce{Cl_2\gaz{}}$   &
			$n_{\tot, gaz}$                               \\
			\hline
			$\xmathstrut{\rhgt}$
			Initial             & $\xi = 0$             &
			$n_0$               & \vline                &
			$0$                 & \vline                &
			$0$                 &
			$n_0$                                         \\
			\hline
			$\xmathstrut{\rhgt}$
			Interm.             & $\xi$                 &
			$n_0 - \xi$         & \vline                &
			$\xi$               & \vline                &
			$\xi$               &
			$n_0 + \xi$                                   \\
			\hline
			$\xmathstrut{\rhgt}$
			Final               & $\xi_f = \xi\ind{eq}$ &
			$n_0 - \xi\ind{eq}$ & \vline                &
			$\xi\ind{eq}$       & \vline                &
			$\xi\ind{eq}$       &
			$n_0 + \xi\ind{eq}$                           \\
			\hline
		\end{tabularx}
	\end{center}
}

\QR[2]{%
	Exprimer les pressions partielles des gaz en fonction de $n_0$, $\xi$ et de la
	pression initiale $p_0$.
}{%
	On utilise à nouveau l'équation des gaz parfaits, avec $\frac{RT}{V} =
		\frac{p_0}{n_0}$~:
	\[
		\boxed{P_{\ce{PCl5}} = \frac{(n_0-\xi)RT}{V}=\frac{(n_0-\xi)p_0}{n_0}}
		\qqet
		\boxed{P_{\ce{PCl3}} = \frac{\xi p_0}{n_0}}
		\qqet
		\boxed{P_{\ce{Cl2}} = \frac{\xi p_0}{n_0}}
	\]
}

\QR[5]{%
	Exprimer le coefficient de dissociation à l’équilibre $\alpha = \xi\ind{eq}
		/n_0$ en fonction de $K^\circ $, $p^\circ $ et $p_0$. Faire l'application
	numérique. Que représente-t-il physiquement~?
}{%
	\noindent
	\begin{minipage}[t]{.5\linewidth}
		Loi d'action de masses~:
		\begin{DispWithArrows*}[fleqn, mathindent=10pt]
			K^\circ &=
			\eval{\frac{a(\ce{PCl3})a(\ce{Cl2})}{a(\ce{PCl5})}}\ind{eq}
			\Arrow{Activité d'un gaz}
			\\\Lra
			K^\circ &=
			\eval{
				\frac{\frac{P_{\ce{PCl3}}}{p^\circ}\frac{P_{\ce{Cl2}}}{p^\circ}}
				{\frac{P_{\ce{PCl5}}}{p^\circ}}}\ind{eq}
			\Arrow{Question 2}
			\\\Lra
			K^\circ &=
			\frac{\xi\ind{eq}^2}{n_0(n_0-\xi\ind{eq})}\frac{p_0}{p^\circ}
			\Arrow{On factorise}
			\\\Lra
			K^\circ &=
			\underbracket[1pt]{\cancel{\frac{n_0^{2}}{n_0^{2}}}}_{=1}
			\frac{\left( \frac{\xi\ind{eq}}{n_0} \right)^{2}}{1 \left(
				1-\frac{\xi\ind{eq}}{n_0} \right)}
			\frac{p_0}{p^\circ}
			\Arrow{$\alpha = \xi\ind{eq}/n_0$}
			\\\Lra
			K^\circ &=
			\frac{\alpha^2}{(1-\alpha)}\frac{p_0}{p^\circ }
		\end{DispWithArrows*}
	\end{minipage}
	\hfill
	\begin{minipage}[t]{.5\linewidth}
		Ainsi, en isolant~:
		\begin{DispWithArrows*}
			\alpha^2 &+ \alpha\left(\frac{K^\circ p^\circ }{p_0} \right) -
			\frac{K^\circ p^\circ }{p_0} = 0
			% \Arrow{$\Delta$ son discriminant}
			\\\Ra
			\Delta &=
			\left(\frac{K^\circ p^\circ }{p_0} \right)^2 +
			4\left(\frac{K^\circ p^\circ }{p_0} \right) > 0
			% \Arrow{Solution positive}
			\\\Lra
			\Aboxed{
				\alpha &= -\left(\frac{K^\circ p^\circ }{2p_0} \right) +
				\sqrt{\left(\frac{K^\circ p^\circ }{2p_0} \right)^2 +
					\left(\frac{K^\circ p^\circ }{p_0} \right)}}
			\\\Ra
			\makebox[0pt][l]{$\phantom{\AN}\xul{\phantom{\alpha = \num{0.59}}}$}
			\AN
			\alpha &= \num{0.59}
		\end{DispWithArrows*}
		$\alpha$ représente la proportion de réactif ayant effectivement réagit.
	\end{minipage}
}

\QR[2]{%
	Calculer la pression régnant dans le réacteur à l’équilibre.
}{%
	À l'aide du tableau d'avancement, on a $n\ind{tot, gaz}$, d'où

	\begin{gather*}
		P\ind{eq} = \frac{n_0+\xi\ind{eq}}{n_0}p_0
		\Lra
		\boxed{
			P\ind{eq} = (1+\alpha)p_0
		}
		\\
		\AN
		\xul{P\ind{eq} = \SI{15,0}{bar}}
	\end{gather*}
}

\iftoggle{corrige}{}{%
  \newpage
}

\resetQ
\exercice[30]{États finaux variés}
\enonce{
	On s'intéresse dans un premier temps à une solution aqueuse obtenue à
	\SI{298}{K} par un mélange d'acide éthanoïque \ce{CH3COOH} (concentration
	$c_1=\SI{0.10}{mol.L^{-1}}$ après mélange) et d'ions fluorure \ce{F-}
	(concentration $c_2=\SI{5.0e-2}{mol.L^{-1}}$ après mélange).
	La réaction \eqref{eq:R1} susceptible de se produire s'écrit~:
	\begin{alignat}{2}
		\ce{CH3COOH(aq) + F^-(aq) & = CH3COO^-(aq) & + HF(aq)}
		\tag{1}\label{eq:R1}
		\\
		\intertext{On connaît les constantes d'équilibre à \SI{298}{K} des réactions
			suivantes~:}
		\beforetext{$K^\circ_2 = \num{e-4.8}$}
		\ce{CH3COOH(aq) + H2O     & = CH3COO^-(aq) & + H3O^+(aq)}
		\tag{2}\label{eq:R2}
		\\
		\beforetext{$K^\circ_3 = \num{e-3.2}$}
		\ce{HF(aq) + H2O          & = F^-(aq)      & + H3O^+(aq)}
		\tag{3}\label{eq:R3}
	\end{alignat}
}

\QR[1]{%
	Calculer la constante d'équilibre $K^\circ_1$ relative à
	l'équilibre~\eqref{eq:R1}.
}{%
	On constate que l'équilibre%
	\sswitch{%
		$(1) = (2) - (3)$
	}{%
		$\eqref{eq:R1}=\eqref{eq:R2}-\eqref{eq:R3}$
	}%
	donc
	\[
		\boxed{K^\circ_1=\cfrac{K^\circ_2}{K^\circ_3}=\num{e-1.6}}
	\]
}

\QR[4]{%
	Déterminer l'état d'équilibre de la solution issue du mélange de l'acide
	éthanoïque et des ions fluorure~: exprimer l'équation dont l'avancement est
	solution, et l'expression littérale de la solution en fonction de $c_1$, $c_2$
	et $K_1^\circ$.
}{%
	On dresse le tableau d'avancement en concentration~:
	\begin{center}
		\def\rhgt{0.35}
		\centering
		\begin{tabularx}{\linewidth}{|l|c||YdYdYdY|}
			\hline
			\multicolumn{2}{|c||}{%
				$\xmathstrut{\rhgt}$
			\textbf{Équation}}        &
			$\ce{CH_3COOH(aq)}$       & $+$               &
			$\ce{F^{-}(aq)}$          & $\ra$             &
			$\ce{CH_3COO^{-}(aq)}$    & $+$               &
			$\ce{HF(aq)}$                                   \\
			\hline
			$\xmathstrut{\rhgt}$
			Initial                   & $x = 0$           &
			$c_1$                     & \vline            &
			$c_2$                     & \vline            &
			$0$                       & \vline            &
			$0$                                             \\
			\hline
			$\xmathstrut{\rhgt}$
			Interm.                   & $x$               &
			$c_1 - x$                 & \vline            &
			$c_2 - x$                 & \vline            &
			$x$                       & \vline            &
			$x$                                             \\
			\hline
			$\xmathstrut{\rhgt}$
			Final ($\si{mol.L^{-1}}$) & $x_f = x\ind{eq}$ &
			$\num{9.0e-2}$            & \vline            &
			$\num{4.0e-2}$            & \vline            &
			$\num{9.6e-3}$            & \vline            &
			$\num{9.6e-3}$                                  \\
			\hline
		\end{tabularx}
	\end{center}

	D'après la loi d'action de masse,
	\begin{DispWithArrows*}[groups]
		K^\circ_1 &= \cfrac{x\ind{eq}^2}{(c_1-x\ind{eq})\times (c_2-x\ind{eq})}
		\Arrow{On isole}
		\\\Lra
		(c_1-x\ind{eq})(c_2-x\ind{eq})K^\circ_1 &= x\ind{eq}^{2}
		\Arrow[new-group]{On rassemble}
		\\\Lra
		x\ind{eq}^{2} + (x\ind{eq}-c_1)(x\ind{eq}-c_2)K^\circ_1 &= 0
		\Arrow{On développe et factorise}
		\\\Lra
		\Aboxed{
			x\ind{eq}^{2}(1+K^\circ_1) -
			x\ind{eq}(c_1+c_2)K^\circ_1 + c_1c_2K^\circ_1 &= 0
		}
	\end{DispWithArrows*}
	Ainsi, avec $\Delta$ le discriminant de ce trinôme~:
	\begin{DispWithArrows*}
		\Delta &= \left( c_1+c_2 \right)^{2}\left( K_1^\circ \right)^{2}
		-4(1+K_1^\circ)c_1c_2K_1^\circ
		\Arrow{Solutions}
		\\\Ra
		\Aboxed{
			x_{\equ,\pm} &=
			\frac{
				(c_1+c_2)K_1^\circ \pm \sqrt{
					\left( c_1+c_2 \right)^{2}\left( K_1^\circ \right)^{2}
					-4(1+K_1^\circ)c_1c_2K_1^\circ}
			}{2(1+K_1^\circ)}
		}
		\Arrow{Calcul}
		\\
		\makebox[0pt][l]{%
			$\phantom{\AN}\xul{\phantom{x\ind{eq} = \SI{9.6e-3}{mol.L^{-1}}}}$%
		}
		\AN
		x\ind{eq} &= \SI{9.6e-3}{mol.L^{-1}}
		\qav
		\left\{
		\begin{array}{rcl}
			c_1       & = & \SI{0.1}{mol.L^{-1}}
			\\
			c_2       & = & \SI{5.0e-2}{mol.L^{-1}}
			\\
			K_1^\circ & = & \num{e-1.6}
		\end{array}
		\right.
	\end{DispWithArrows*}
	On en déduit les concentrations à l'équilibre indiquées dans le tableau.
}

\enonce{
	On étudie dans la suite de l'exercice quelques constituants du béton.
	L'hydroxyde de calcium \ce{Ca(OH)_2(s)} confère au béton ses propriétés
	basiques (au sens de acide ou base). Il se dissout en solution aqueuse selon
	la réaction~\eqref{eq:R4}~:
	\begin{gather}
		\beforetext{$K^\circ_4(\SI{298}{K}) = \num{e-5.2}$}
		\ce{Ca(OH)_2(s) = Ca^{2+}(aq) + 2HO^-(aq)}
		\tag{4}\label{eq:R4}
	\end{gather}
}

\QR[3]{%
	On introduit en solution aqueuse un net excès d'hydroxyde de calcium. La phase
	solide est alors présente en fin d'évolution. Calculer les concentrations de
	chacun des ions présents à l'équilibre.
}{%
	On dresse un tableau d'avancement en concentration~:
	\begin{center}
		\def\rhgt{0.35}
		\centering
		\begin{tabularx}{.7\linewidth}{|l|c||YdYdY|}
			\hline
			\multicolumn{2}{|c||}{%
				$\xmathstrut{\rhgt}$
			\textbf{Équation}}        &
			$\ce{Ca(OH)2(s)}$         & $=$               &
			$\ce{Ca^{2+}(aq)}$        & $+$               &
			$2\ce{HO^{-}(aq)}$                              \\
			\hline
			$\xmathstrut{\rhgt}$
			Initial                   & $x = 0$           &
			excès                     & \vline            &
			$0$                       & \vline            &
			$0$                                             \\
			\hline
			$\xmathstrut{\rhgt}$
			Interm.                   & $x$               &
			excès                     & \vline            &
			$x$                       & \vline            &
			$2x$                                            \\
			\hline
			$\xmathstrut{\rhgt}$
			Final ($\si{mol.L^{-1}}$) & $x_f = x\ind{eq}$ &
			excès                     & \vline            &
			$\num{1.2e-2}$            & \vline            &
			$\num{2.4e-2}$                                  \\
			\hline
		\end{tabularx}
	\end{center}
	\begin{gather*}
		\beforetext{Par la loi d'action de masse,}
		K^\circ_4=x\ind{eq}\times (2x\ind{eq})^2
		\qdonc
		\boxed{x\ind{eq}=\pa{\cfrac{K^\circ_4}{4}}^{1/3}}
		\\\AN
		\xul{x\ind{eq}=\SI{1.2e-2}{mol.L^{-1}}}
	\end{gather*}
	On en déduit les concentrations indiquées dans le tableau.
}

\QR[1]{%
	On donne la relation $[\ce{H_3O+}][\ce{HO^{-}}] = \num{e-14}$. Sachant que $\pH
		= -\log([\ce{H_3O^{+}}])$, déterminer le $\pH$ de la solution. Le milieu
	est-il acide, basique ou neutre~?
}{%
	\[
		\boxed{\pH =14+\log[\ce{HO-}]}
		\Ra
		\xul{\pH = \num{12,4}}
	\]
	ce qui corrspond bien à un milieu basique.
}

\enonce{
	Dans certains cas, la pollution urbaine liée à l'humidité entraine la
	dissolution du dioxyde de carbone atmosphérique dans l'eau à l'intérieur du
	béton (sous forme \ce{H2CO3}), provoquant la carbonatation du béton (formation
	de carbonate de calcium \ce{CaCO_3(s)}) par réaction de l'hydroxyde de calcium
	\ce{Ca(OH)_2(s)} avec la forme \ce{H2CO_3(aq)}.
}

\QR[2]{%
	Écrire la réaction $(5)$ mise en jeu dans la carbonatation du béton et calculer
	sa constante d'équilibre $K^\circ_5$ à \SI{298}{K}. On donne à \SI{298}{K} les
	constantes d'équilibre des réactions suivantes~:
	\begin{alignat}{2}
		\beforetext{$K^\circ_6=\num{e-8.4}$}
		\ce{CaCO_3(s)         & = Ca^{2+}(aq)   & + CO^{2-}_3(aq)}
		\tag{6}\label{eq:R6}
		\\
		\beforetext{$K^\circ_7=\num{e-6.4}$}
		\ce{H2CO_3(aq) + H2O  & = HCO^-_3(aq)   & + H3O^+(aq)}
		\tag{7}\label{eq:R7}
		\\
		\beforetext{$K^\circ_8=\num{e-10.3}$}
		\ce{HCO^-_3(aq) + H2O & = CO^{2-}_3(aq) & + H3O^+(aq)}
		\tag{8}\label{eq:R8}
		\\
		\beforetext{$K^\circ_9=\num{e-14}$}
		\ce{2H2O              & = HO^-(aq)      & + H3O^+(aq)}
		\tag{9}\label{eq:R9}
	\end{alignat}
}{%
	\ifstudent{
		\vspace*{-\dimexpr\baselineskip+\abovedisplayskip\relax}
	}
	\begin{gather}
		\beforetext{On écrit la réaction~\eqref{eq:R5}~:}
		% \beforetext{$K^\circ_5$}
		\ce{Ca(OH)_2(s) + H2CO_3(aq) = CaCO_3(s) + 2H2O}
		\tag{5} \label{eq:R5}
	\end{gather}
	On constate que la réaction
	\sswitch{%
		$(5)=(4)-(6)+(8)+(7)-2\times(9)$
	}{%
		$\eqref{eq:R5} = \eqref{eq:R4} - \eqref{eq:R6} + \eqref{eq:R8} +
			\eqref{eq:R7} -2\eqref{eq:R9}$
	}%
	donc
	\[
		\boxed{
			K^\circ_5 =
			\cfrac{K^\circ_4K^\circ_7K^\circ_8}
			{K^\circ_6(K^\circ_9)^2}
			= \num{e14.5}
		}
	\]
}

\enonce{
	On étudie désormais la réaction de décomposition du carconate de calcium
	\ce{CaCO_3(s)} en oxyde de calcium \ce{CaO(s)} et dioxyde de carbone
	\ce{CO_2(g)} de constante d'équilibre $K^\circ=0,20$ à \SI{1093}{K}.
	\[
		\ce{CaCO_3(s) = CaO(s) + CO2(g)}
	\]
	Soit un récipient indéformable de volume $V=\SI{10}{L}$, vidé au préalable de
	son air, et maintenu à la température constante de \SI{1093}{K}. On introduit
	progressivement une quantité de matière $n$ en carbonate de calcium solide et
	on mesure la pression $p$ à l'intérieur de l'enceinte.
}

\QR[2]{%
Lorsque l'équilibre est établi, calculer la quantité de matière en dioxyde de
carbone $n(\ce{CO2})\ind{eq}$ dans l'enceinte. On supposera les gaz comme
parfaits. On rappelle la constante des gaz parfaits
$R=\SI{8,31}{J.K^{-1}.mol^{-1}}$.
}{%
À l'équilibre, d'après la loi d'action de masse,
\begin{DispWithArrows*}[groups]
	K^\circ &= \cfrac{P(\ce{CO2})\ind{eq}}{P^\circ}
	\Arrow{$\DS P (\ce{CO_2})\ind{eq} = \frac{n(\ce{CO_2})\ind{eq}RT}{V}$}
	\\\Lra
	K^\circ &= \frac{n(\ce{CO_2})\ind{eq}RT}{VP^\circ}
	\Arrow{On isole}
	\\\Lra
	\Aboxed{n(\ce{CO2})\ind{eq} &= \cfrac{K^\circ P^\circ V}{RT}}
	\\
	\makebox[0pt][l]{%
		$\phantom{\AN}\xul{\phantom{n(\ce{CO_2})\ind{eq} = \SI{2.2e-2}{mol}}}$%
	}
	\AN n(\ce{CO_2})\ind{eq} &= \SI{2.2e-2}{mol}
	\qav
	\left\{
	\begin{array}{rcl}
		K^\circ & = & \num{0.20}
		\\
		P^\circ & = & \SI{1e5}{Pa}
		\\
		V       & = & \SI{10e-3}{m^{3}}
		\\
		R       & = & \SI{8.314}{J.K^{-1}.mol^{-1}}
		\\
		T       & = & \SI{1093}{K}
	\end{array}
	\right.
\end{DispWithArrows*}
}

\QR[2]{%
	On introduit une quantité de matière $n=\SI{1.0e-2}{mol}$ en carbonate de
	calcium. Décrire l'état final. On précisera notamment si l'état final est un
	état d'équilibre.
}{%
	Comme $n<n(\ce{CO2})\ind{eq}$, le quotient réactionnel évoluera de sa valeur
	initiale 0 jusqu'à sa valeur maximale $Q\ind{max}$, qui sera inférieure à
	$K^\circ$. Ainsi la réaction évoluera dans le sens direct jusqu'à
	\textbf{disparition complète} du carbonate de calcium~: c'est une
	\textbf{rupture d'équilibre}.
	\smallbreak
	Les quantités de matière sont~:
	\[
		\xul{
			n(\ce{CO2})_f=n(\ce{CaO})_f=n=\SI{1.0e-2}{mol}
		}
		\qet
		\xul{
			n(\ce{CaCO3})_f=0
		}
	\]
}

\QR[2]{%
	Reprendre la question précédente dans le cas où $n=\SI{5.0e-2}{mol}$.
}{%
	Comme $n>n(\ce{CO2})\ind{eq}$, le quotient réactionnel peut augmenter jusqu'à
	atteindre la constante d'équilibre. L'état final est donc \textbf{bien un état
		d'équilibre} avec
	\begin{gather*}
		\boxed{n(\ce{CO2})_f=n(\ce{CaO})_f=n(\ce{CO2})\ind{eq}=\SI{2.2e-2}{mol}}
		\\
		\boxed{n(\ce{CaCO3})_f=n-n(\ce{CO2})\ind{eq}=\SI{2.8e-2}{mol}}
	\end{gather*}
}

\QR[1]{%
	Montrer que la courbe $p=f(n)$, avec $p$ la pression à l'intérieur de
	l'enceinte, est constituée de deux segments de droites dont on donnera les
	équations pour $0\leq n\leq\SI{0.10}{mol}$.
}{%
	On utilise les résultats précédents en appliquant l'équation d'état des gaz
	parfaits~:
	\begin{itemize}
		\item $n<n(\ce{CO2})\ind{eq} \Ra p = \cfrac{nRT}{V} \propto n$~;
		\item $n\geq n(\ce{CO2})\ind{eq} \Ra p = \cfrac{n(\ce{CO2})\ind{eq}RT}{V}
			      =\cte$.
	\end{itemize}
}

\setcounter{section}{0}
\iftoggle{corrige}{}{%
  \newpage
}
\resetQ
\prblm[49]{Amortissement et facteur de qualité d'un circuit RLC}
% \enonce{
% 	\begin{figure}[htbp]
% 	  \centering
% 	  \includegraphics[width=.5\linewidth]{rlc_descendant}
% 	  \caption{Circuit.}
% 	  \label{fig:P1base}
% 	\end{figure}
% }
\enonce{%
  \noindent
  \begin{minipage}[c]{.45\linewidth}
      On considère le circuit RLC série représenté ci-contre.
    L'interrupteur $K$ est fermé à un instant $t=0$ choisi comme origine des
    temps. Le condensateur est initialement chargé~: $u(t=0)=u_0$.
  \end{minipage}
  \hfill
  \begin{minipage}[c]{.45\linewidth}
    ~
    % \vspace{-15pt}
    \begin{center}
      \includegraphics[width=\linewidth]{rlc_descendant}
      \captionof{figure}{Circuit.}
	  \label{fig:P1base}
    \end{center}
  \end{minipage}
}

\QR{%
    Établir l'équation différentielle vérifiée par $u(t)$ pour $t\geq 0$. La
    mettre sous la forme
    \[
      \dv[2]{u}{t} + \frac{\w_0}{Q}\dv{u}{t} + \w_0{}^{2}u = 0
    \]
    et donner les expressions de $\w_0$ et $Q$ en fonction de $R$, $L$ et $C$.
}{%
  \begin{isd}
      Avec la loi des mailles,
      \begin{DispWithArrows}[fleqn, mathindent=2pt]
        u_L + u_R + u_C &= 0
        \notag
        \Arrow{$u_L = L \dv{i}{t}$\\et $u_R = Ri$}
        \\\Lra
        L \dv{i}{t} + Ri + u_C &= 0
        \notag
        \Arrow{$i = C \dv{u_C}{t}$}
        \\\Lra
        LC \dv[2]{u_C}{t} + RC \dv{u_C}{t} + u_C                   &= 0
        \notag
        \Arrow{forme\\canonique}
        \\
        \Lra \dv[2]{u_C}{t} + \frac{R}{L} \dv{u_C}{t} + \frac{1}{LC}u_C &= 0
        \label{eq:rlsimple}
      \end{DispWithArrows}
    \tcblower
    On détermine l'expression de $Q$ par identification~:
      \begin{DispWithArrows*}[fleqn, mathindent=20pt]
        \frac{\w_0}{Q} &= \frac{R}{L}
        \Arrow{$\DS\w_0 = \frac{1}{\sqrt{LC}}$}
        \\\Lra
        \frac{1}{Q \sqrt{LC}} &= \frac{R}{L}
        \Arrow{On isole $Q$}
        \\\Lra
        Q &= \frac{L}{R \sqrt{LC}}
        \Arrow{$L = \sqrt{L}^{2}$}
        \\\Lra
        \Aboxed{Q &= \frac{1}{R}\sqrt{\frac{L}{C}}}
      \end{DispWithArrows*}
  \end{isd}
  }
\QR{%
  Montrer que le système répond différemment selon la valeur de $Q$. Nommer
  chaque régime possible, sans chercher à donner les formes de solutions
  correspondantes.
}{%
  \ifstudent{
		\vspace*{-\dimexpr\baselineskip+\abovedisplayskip\relax}
  }
  \begin{isd}
      Avec l'équation caractéristique~:
      \begin{gather*}
      r^2 + \frac{\w_0}{Q}r + \w_0{}^2 = 0
      \\\Ra 
      \Delta =
      \left( \frac{\w_0}{Q} \right)^2 - 4\w_0{}^2 =
      \frac{\w_0{}^{2}}{Q^{2}}\left( 1-4Q^{2} \right)
    \end{gather*}
    Selon la valeur du discriminant, on aura différentes valeurs de $r$,
    doubles réelles, simple réelle ou doubles complexes. On a en effet, avec $Q >
    0$,
    \tcblower
      \begin{gather*}
        \Delta > 0
        \Lra
        \cancel{\frac{w_0{}^{2}}{Q^{2}}}\left( 1-4Q^{2} \right) >0
        \Lra
        4Q^2 < 1
        \Lra
        Q < \frac{1}{2}
      \end{gather*}
    \begin{description}
      \item[$\mathbf{Q > 1/2}$]~: régime \textbf{pseudo-périodique},
        racines complexes et oscillations décroissantes~;
      \item[$\mathbf{Q = 1/2}$]~: régime \textbf{critique}, racine double
        réelle~;
      \item[$\mathbf{Q < 1/2}$]~: régime \textbf{apériodique}, racines
        réelles et décroissance exponentielle sans oscillation.
    \end{description}
  \end{isd}
}

\enonce{
	On suppose $Q >1/2$ dans la suite.
}
	\QR{%
    Définir la pseudo-pulsation $\W$ des oscillations libres en fonction de
    $\w_0$ et $Q$. Définir aussi le temps caractéristique $\tau$ d'amortissement
    exponentiel des oscillations libres en fonction de $\w_0$ et $Q$.
	}{%
    \begin{isd}
      \begin{DispWithArrows*}[fleqn, mathindent=10pt]
          r_\pm & = \frac{-\frac{\w_0}{Q} \pm \jj\sqrt{-\D}}{2}
          \Arrow{On injecte $\Delta$}
          \\\Lra
          r_\pm &= -\frac{\w_0}{2Q} \pm
          \frac{\jj}{2} \sqrt{\frac{\w_0{}^{2}}{Q^{2}}\left( 4Q^{2}-1 \right)}
          \Arrow{On extrait $\frac{\w_0}{Q}$}
          \\\Lra
          r_\pm &= - \frac{\w_0}{2Q} \pm \jj \frac{\w_0}{2Q} \sqrt{4Q^{2}-1}
          \Arrow{On définit $\W$}
          \\\Lra
          r_\pm &= - \frac{\w_0}{2Q} \pm \jj\W
        \end{DispWithArrows*}
      \tcblower
      d'où la définition de $\W$~:
      \[
        \boxed{\W = \frac{\w_0}{2Q}\sqrt{4Q^{2}-1}}
      \]
      Ensuite, avec la forme générale de la solution on a
        \begin{equation*}
          u(t) = \exp \left(-\frac{\w_0}{2Q}t\right)
          \left[ A\cos(\Wt) + B\sin(\Wt) \right]
        \end{equation*}
        On remarque donc qu'on peut assimiler le terme à l'intérieur de
        l'exponentielle comme l'inverse d'un temps, c'est-à-dire qu'on définit
        $\tau$ comme la partie réelle des racines~:
        \[
          \boxed{\tau = \frac{2Q}{\w_0}}
        \]
    \end{isd}
    % L'énoncé attend sans doute le temps de relaxation $\tau = \frac{Q}{\w_0}$
    % mais la forme canonique en $\dv[2]{u}{t} + \dfrac{1}{\tau} \dv{u}{t} +
    % \w_0^2 u = 0$ n'est pas à connaître. 
	}
	\QR{%
    Établir l'expression de $u(t)$ pour $t\geq 0$ en fonction de $u_0$, $\w_0$,
    $Q$ et $\Omega$, compte tenu des conditions initiales que vous expliciterez
    et justifierez.
	}{%
    On a donc
    \[
      u(t) = \exr^{-t/\tau} \left[ A\cos(\Wt) + B\sin(\Wt) \right]
    \]
	\begin{itemize}
		\item On trouve $A$ avec la première condition initiale (condensateur
      initialement chargé et tension condensateur continue)~:
			      \begin{gather*}
				      u(0) = u_0 = 1 \left[ A \cdot 1 + B \cdot 0 \right] = A
				      \quad \Ra \quad \boxed{A=u_0}
			      \end{gather*}
		\item On trouve $B$ avec la seconde CI (il n'y a pas de courant avant la
      fermeture de $K$ et courant continu dans la bobine)~:
			      \begin{gather*}
				      \dv{u}{t} =
				      -\frac{\w_0}{2Q}\exp \left( -\frac{\w_0}{2Q}t \right)\times
				      \left[ A\cos(\Wt) + B\sin(\Wt) \right] +
				      \exp \left( -\frac{\w_0}{2Q}t \right)
				      \left[ -A\W\sin(\Wt) + B\W\cos(\Wt) \right]
				      \\\Ra
				      \dv{u}{t}\/(0) = - \frac{\w_0}{2Q}A + \W B = 0
				      \\\Lra
				      \boxed{B = \frac{\w_0}{2Q\W}u_0}
              \\
              \beforetext{Ainsi,}
              \boxed{
                u(t) = u_0\exr^{-t/\tau}
                \left[ \cos(\Wt) + \frac{\w_0}{2Q\W}\sin(\Wt) \right]
              }
			      \end{gather*}
	\end{itemize}
	}

\enonce{
  On souhaite visualiser la tension $u(t)$ sur l'écran d'un oscilloscope dont
  l'entrée est modélisée par l'association en parallèle d'une résistance
  $R_0=\SI{1,0}{M\ohm}$ et d'une capacité $C_0=\SI{11}{p\farad}$.
}
	\QR{%
    Montrer que si l'on tient compte de l'oscilloscope, l'équation
    différentielle vérifiée par $u(t)$ devient:
    \[
            L(C+C_0)\dv[2]{u}{t }+
            \left(\frac{L}{R_0}+RC+RC_0\right)\dv{u}{t}+
            \left(1+\frac{R}{R_0}\right)u=0
    \]
	}{%
    \noindent
\begin{minipage}[t]{.30\linewidth}
    On commence par représenter le circuit en ajoutant en parallèle de $C$ la
    résistance $R_0$ et la capacité $C_0$.
  \begin{center}
      \includegraphics[width=\linewidth]{rlc_oscillo}
      \captionof{figure}{Circuit avec oscilloscope.}
      \label{fig:P1osci}
  \end{center}
\end{minipage}
\hfill
\begin{minipage}[t]{.65\linewidth}
    Loi des nœuds~:
    \begin{DispWithArrows*}
      i &= i_{R_0} + i_{C_0} + i_C
      \Arrow{$i_C = C \dv{u}{t}$\\$i_{C_0} = C_0 \dv{u}{t}$\\$i_{R_0} =
      \frac{u}{R_0}$}
      \\\Lra
      i &= \frac{u}{R_0} + (C+C_0)\dv{u}{t}
    \end{DispWithArrows*}
    Dans la loi des mailles,
    \begin{DispWithArrows*}[fleqn, mathindent=-15pt]
      u_L + u_R + u &= 0
      \Arrow{$u_L = L \dv{i}{t}$\\$u_R = Ri$}
      \\\Lra
      L \dv{i}{t} + Ri + u &= 0
      \Arrow{$i = \frac{u}{R_0}$\\$ + (C+C_0)\dv{u}{t}$}
      \\\Lra
      \frac{L}{R_0} \dv{u}{t} + L(C+C_0) \dv[2]{u}{t} +
      \frac{R}{R_0}u + R(C+C_0) \dv{u}{t} + u &=0
      \Arrow{On factorise}
      \\\Lra
      \Aboxed{
        L(C+C_0)\dv[2]{u}{t} + 
        \left(\frac{L}{R_0}+RC+RC_0\right)\dv{u}{t}+
        \left(1+\frac{R}{R_0}\right)u
        &= 0
      }
    \end{DispWithArrows*}
\end{minipage}
	}

	\QR{%
    Quelles relations qualitatives doivent vérifier $R$, $L$, $C$, $R_0$ et
    $C_0$ pour que la mise en place de l'oscilloscope ait une influence
    négligeable sur les oscillations étudiées~? Vérifier qu'avec les valeurs
    usuelles de $R$, $L$ et $C$ utilisées en travaux pratiques ces relations
    sont vérifiées.
	}{%
    Pour que l'oscilloscope ait le moins d'influence possible sur les
    oscillations, il faut que les coefficients de l'équation différentielle
    précédente diffèrent le moins possible de ceux de l'équation
    différentielle~\eqref{eq:rlsimple}~:
		\begin{itemize}
      \item $C \gg C_0$; les capacités utilisées en T.P.\ sont de l'ordre du
        $\si{nF}$ ou du $\si{\micro F}$. Comme $C_0=\SI{11}{pF}$, cette
        condition est bien vérifiée~;
      \item $R \ll R_0$; les résistances utilisées en T.P.\ sont de l'ordre du
        $\si{k\ohm}$. Comme  $R_0=\SI{1,0}{M\ohm}$ , cette condition
        est bien vérifiée~;
      \item $\frac{L}{R_0} \ll RC$ soit $R_0 \gg \frac{L}{RC} \approx
         \SI{e4}{\ohm}$~; cette condition est bien vérifiée.
		\end{itemize}
	}

	\QR{%
    On définit le décrément logarithmique comme étant la quantité
    $d_m=\ln\dfrac{u(t)}{u(t+mT)}$ où $T=2\pi/\w$ et $m$ est un entier
    strictement positif. Exprimer $d_m$ en fonction de $m$ et de $Q$.
	}{%
    En remarquant que $ \cos{(\W (t + T))} = \cos{(\W t + 2\pi)} = \cos{(\W
    t)}$, on montre facilement que $d_m =\ln { \left( \exp{\frac{\W_0 m T}{2Q}}
    \right) }$.
    On obtient donc~: $d_m = \dfrac{\w_0 m T}{2Q}$. En remplaçant $T$ par
    $\frac{2\pi}{\W}$ où $\W = \frac{\w_0}{Q} \sqrt{4Q^{2}-1}$, il vient~:
			\[
			  \boxed{d_m = \frac{2 \pi m}{\sqrt{4Q^2-1}}}
			\]
	}

	\QR{%
    On réalise un montage expérimental où le circuit RLC est excité par un
    générateur basses fréquences délivrant une tension créneau. Comment faut-il
    choisir le signal délivré par le générateur pour observer les oscillations
    libres du circuit~? Justifier à l'aide d'un schéma.
    \smallbreak
	}{%
		Pour observer les oscillations il faut que la demi-période du signal délivré par le G.B.F.\ soit égale à quelques $\tau$. \\
}

\QR{%
    La tension aux bornes du condensateur est enregistrée grâce à un logiciel
    d'acquisition. Le signal obtenu est représenté sur la
    figure~\ref{fig:P1graph}.
		\begin{figure}[htbp!]
		  \centering
		  \includegraphics[width=.8\linewidth]{carac-rlc-15}
		  \caption{Signal obtenu.}
		  \label{fig:P1graph}
		\end{figure}
		Estimer le facteur de qualité $Q$ du circuit.
}{%
    On lit graphiquement $u(0)=4,0$V et $u(2T)=1,4$V. On peut alors calculer
    $d_2=\ln\dfrac{u(0)}{u(2T)}$.

    Comme $d_2= \frac{4 \pi }{\sqrt{4Q^2-1}}$, on en déduit~: $\boxed{ Q =
  \sqrt{ \frac{1}{4} + 4 \left( \frac{\pi}{d_2} \right) ^2} }$. Application
numérique~: $ \underline{Q = 6,0}$ }


\enonce{
  On suppose $Q\gg 1$: la dissipation d'énergie par effet Joule est traitée
  comme une perturbation par rapport au cas du circuit non dissipatif ($R=0$).
  On prendra alors $\Omega \approx \w_0$. On rappelle par ailleurs le
  développement limité de l'exponentielle en 0~:
  \[
    \exr^{x} \Sim_{x\to0} 1 + x
  \]
}
	\QR{%
    Dans le cas où $R=0$, établir l'expression de la valeur moyenne temporelle
    $\moy{\Ec}$ de l'énergie électromagnétique stockée dans le circuit.
	}{%
    Dans le cas où $R = 0$, le circuit est non dissipatif donc l'énergie
    emmagasinée dans le condensateur et la bobine reste constante. On l'évalue
    facilement en $t=0$~:
    \[
      \Ec(t=0) = \frac{Cu_0^2}{2}
      \Ra 
      \boxed{\moy{\Ec} = \frac{Cu_0^{2}}{2}}
    \]
	}

	\QR{%
  Dans le cas où $R\neq 0$, montrer qu'au premier ordre en $1/Q$, l'énergie
  $\Ec_J$ dissipée par effet \textsc{Joule} dans le circuit RLC, pendant une
  pseudo-période, vérifie la relation~:
		\[
		  \Ec_J = \frac{2\pi}{Q}\moy{\Ec}
		\]
	}{%
  Il faut évaluer l'énergie emmagasinée par le condensateur et la bobine à
  l'instant $t$~:
  \[
    \Ec(t) = \frac{Cu^2(t)}{2} + \frac{Li^2(t)}{2}
  \]
	Pour $Q \gg 1$, on a $\W \approx \w_0$ , l'expression de $u(t)$ devient~:
	
  \begin{DispWithArrows*}[groups]
        u(t) &= u_0 \exr^{-\frac{t}{\tau}}
          \left( \cos{(\w t)} + \frac{\w_0}{2Q\w} \sin{(\w t)} \right)
          \Arrow{$\W \approx \w_0$}
          \\\Lra
      u (t) &\approx u_0 \exr^{-\frac{t}{\tau}}
      \left( \cos{(\w_0 t)} + \frac{1}{2Q} \sin{(\w_0 t)} \right)
      \Arrow{$Q \gg 1 \Lra \frac{1}{Q} \approx 0$}
      \\\Lra
      \Aboxed{
        u(t) &\approx u_0 \exr^{-\frac{t}{\tau}} \cos{(\w_0 t)}
      }
      \Arrow{$i(t) = C \dv{u}{t}$}
      \\\Ra 
      i(t) &= -Cu_0 \exr^{-\frac{t}{\tau}}
      \left( \w_0 \sin{(\w_0 t)} + \frac{1}{\tau} \cos{(\w_0 t)} \right)
      \Arrow{$\frac{1}{\tau} = \frac{\w_0}{2Q} \ll \w_0$}
      \\\Ra 
		  \Aboxed{i(t) &\approx -Cu_0 \w_0 \sin{(\w_0 t)} \exr^{-\frac{t}{\tau}}}
      \\\text{or}\qquad 
      \Ec(t) &= \frac{Cu^2(t)}{2} + \frac{Li^2(t)}{2}
      \Arrow{On injecte}
      \\\Lra
      \Aboxed{\Ec(t) &= \frac{1}{2} C u_0^2 \exr^{ -\frac{2t}{\tau}}}
  \end{DispWithArrows*}
	En une pseudo-période $T$, l'énergie décroît de la quantité~:
  \begin{DispWithArrows*}
    \Delta_T{\Ec} &=
      \Ec(t) - \Ec(t+T)
      \\\Lra
    \Delta_T{\Ec} &=
      \frac{1}{2}Cu_0^2\exr^{-\frac{2t}{\tau}}
      \left( 1 - \exr^{ -\frac{2T}{\tau} } \right)
      \Arrow{$\tau = \frac{2Q}{\w_0}$}
      \\\Lra
    \Delta_T{\Ec} &=
      \frac{1}{2}Cu_0^2\exr^{-\frac{\w_0t}{Q}}
      \left( 1 - \exr^{ -\frac{\w_0T}{Q} } \right)
      \Arrow{$\W \approx \w_0$ et $\W T = 2\pi$}
      \\\Lra
    \Delta_T{\Ec} &=
      \frac{1}{2}Cu_0^2\exr^{-\frac{\w_0t}{Q}}
      \left( 1 - \exr^{ -\frac{2\pi}{Q} } \right)
      \Arrow{$\exr^{-\frac{2\pi}{Q}} \Sim_{Q\to\infty} 1 - \frac{2\pi}{Q}$}
      \\\Lra
    \Delta_T{\Ec} &\Sim_{Q\to\infty} 
    \frac{1}{2}Cu_0^2 \underbracket[1pt]{\exr^{-\frac{\w_0t}{Q}}}_{\approx 1}
      \left( \cancel{1} - \left( \cancel{1} - \frac{2\pi}{Q} \right) \right)
      \Arrow{On simplifie}
      \\\Lra
    \Delta_T{\Ec} &\Sim_{Q\to\infty} \frac{2\pi}{Q} \moy{\Ec}
  \end{DispWithArrows*}
  Or, l'énergie dissipée par effet \textsc{Joule} en une pseudo-période
  correspond à l'énergie perdue par $L$ et $C$ pendant cette durée donc $\Ec_J =
  \Ec(t) - \Ec(t+T)$. Ainsi,
		\[
		  \boxed{\Ec_J \approx \frac{2\pi}{Q}\moy{\Ec}}
		\]
	}

\resetQ
% \iftoggle{corrige}{}{%
%   \clearpage
% }
% \prblm[30]{Modélisation des mouvements d'une plateforme offshore}
% \enonce{
% On s’intéresse à la résolution d’une équation du mouvement dans une approche classique de la mécanique afin d’étudier le mouvement simplifié d’une plateforme en mer. Le modèle envisagé est un système à un degré de liberté considéré comme un oscillateur harmonique~: une masse est reliée à un ressort, avec amortissement.
%
% On considère le mouvement d’une plateforme en mer soumise à un courant marin. Sa partie supérieure de masse $m = \SI{110}{tonnes}$ est considérée comme rigide et le mouvement principal de la plateforme a lieu suivant $x$ (cf figure 1(a)). Afin d’étudier le mouvement de cette plateforme, on la représente par une masse $m$, liée à un ressort de 
% constante de raideur $k$ et à un amortisseur de constante d’amortissement $\gamma$ comme schématisé sur la figure 1(b). 
% La masse se déplace selon une seule direction, parallèle à l’axe $Ox$ en fonction du temps $t$. Ainsi, les projections sur l’axe $Ox$ de la position, de la vitesse et de l’accélération de la masse en fonction du temps sont notées respectivement $x(t)$, $\xp(t)$ et $\xpp(t)$. Le vecteur unitaire de l’axe $Ox$ est noté $\vv{i}$.
%
% \begin{figure}[htbp!]
%   \centering
%   \includegraphics[width=.9\linewidth]{P2base}
% \end{figure}
%
% La masse se déplace sur la base horizontale sans frottements sur le support. La position d’équilibre de la masse sera choisie à $x = 0$.
%
% La force totale $\vv{F_{tot}}$ agissant sur la masse correspond à la réaction normale à la base horizontale $\vv{R_N}$, à la force de frottement $\vv{F_d}= - \gamma \vv{v}$  où $\gamma$ est la constante d’amortissement positive, permettant  de prendre en compte l’effet de l’eau environnante, à la force de rappel $\vv{F_k}$ du ressort et au poids $\vv{P}$ de la masse $m$.
% }
%
% \QR
% {Etablir l’équation différentielle du mouvement de la masse $m$ et la mettre sous la forme~:
%
% \centersright{$\xpp+2\xi\w_0 \xp+{\w_0}^2 x=0$}{(équation 1)}
%
% \noindent
% On exprimera $\w_0$ et $\xi$ en fonction de $k$, $m$ et $\gamma$. On rappelle que $\xi =Q/2$.}
% {
% \begin{itemize}
% \item Référentiel d'étude: Référentiel terrestre $\mathcal{R}(O, x, y)$ supposé galiléen.
% \item Base de projection~: Base cartésienne $(O, x, y)$ de vecteurs unitaires $\vv{i}$ et $\vv{j}$. L'origine est prise à la position d'équilibre comme indiqué dans l’énoncé. $\vv{j}$ est orienté vers le haut.
% \item Système: la plateforme $M$ de masse $m$. 
% \item Bilan des forces: 
% \begin{enumerate}
% \item Poids~: $\vv{P}=m\vv{g} = - mg\vv{j}$ .
% \item Réaction du support: $\vv{R_N}=R_N\vv{j}$~; $R$ est orthogonale au déplacement car mouvement sans frottements.
% \item Force de rappel du ressort~: $\vv{F}= - k \left(l-l_o\right)\vv{i} = - k x \vv{i}$ , car $\ell = \ell_0 + x$.
% \item Force de frottement $\vv{F_d}=-\gamma\vv{v}=-\gamma\xp\vv{i}$ .
% \end{enumerate}	
% \end{itemize}
%
% 2ème loi de Newton (principe fondamental de la dynamique):
%
% \centers{	$\sum{\vv{F}=m\vv{a}=m\dv{\vv{v}}{t}}$}
%
% \leftcenters{	d’où}{ $\vv{P}+\vv{R_N}+\vv{F}+\vv{F_d}=m\vv{a}$} 
%
% 	\noindent
% 	avec $\vv{a}=\xpp\vv{i}$ car le mouvement se fait sur $Ox$.
% 	Projetons sur les 2 axes.
%
% 	 \leftcentersright{Sur $\vv{i}$}{$m\xpp+kx+\gamma\xp=0$}	{(équation du mouvement)}
%
%       \leftcentersright{Sur $\vv{j}$~:} {$R_N- mg = 0$} {(pas de mouvement sur $Oy$)}
%
%       \noindent
% Reprenons la première équation en la mettant sous forme canonique, il vient~: 
%
% \centers{$\xpp+\frac{\gamma}{m}\xp+\frac{k}{m}x=0$}
%
% \leftcenters{De la forme}{  $\xpp+2\xi\w_0\xp+\w_0^2x=0$}
%
%
% \leftcenters{avec} {$\w_0=\sqrt{\frac{k}{m}} \quad \text{et} \quad 2\xi\w_0=\frac{\gamma}{m}$}
%
% \leftcenters{Soit} {$\xi=\frac{\gamma}{2m\w_0}=\frac{\gamma}{2m}\sqrt{\frac{m}{k}} = \frac{\gamma}{2}\sqrt{\frac{1}{mk}}$}
% }
%
% \QR
% {Dans le cas où  $\xi < 1$, justifier que $x\left(t\right)$ peut prendre la forme suivante~:
%
% \centersright{$x\left(t\right)=\exr^{-\xi\w_0 t}[A \cos(\W t)+B \sin(\W t)]$}{(équation 2)}
%
% où $\W$ est la pseudo-pulsation que l’on exprimera en fonction de $\w_0$ et $\xi$.
% De plus, en remarquant qu’à $t=0$, $x\left(0\right)=x_0$ et $\xp\left(0\right)=v_0$, déterminer les expressions des deux coefficients réels $A$ et $B$ en fonction de $x_0$, $v_0$ , $\xi$ , $\w_0$ et $\W$}
% {
% Solution de cette équation différentielle d’ordre 2~:
%
%  \centers{$\xpp+2\xi\w_0\xp+\w_0^2x=0$}
%
% \leftcenters{Equation caractéristique associée~:} {$r^2+2\xi\w_0 \, r+{\w_0}^2=0$}
%
%
% \leftcenters{Discriminant~: }{$\Delta=4\xi^2\, {\w_0}^2-4{\w_0}^2=4 {\w_0}^2(\xi^2-1)$}
%
% \noindent
% Ici, $\xi<1$, donc $\Delta <0$~; C’est un régime pseudo-périodique.	Les solutions de l’équation caractéristique sont alors~: 
%
% \centers{$r_{1,2}=\frac{-2\xi\w_0 \pm i\sqrt{-\Delta}}{2}= - \w_0\pa{\xi \pm i \sqrt{1-\xiç 2}}$}
%
% \noindent
% On pose   $\alpha=\frac{-b}{2a}  = -\xi\w_0$	et $\W=\frac{\sqrt{-\Delta}}{2a}= \w_0 \, \sqrt{1-\xi^2}$ la pseudo–pulsation. Alors
%
% \centers{ $x\left(t\right)=\exp{\left(\alpha t\right)}\left[A \cos\left(\W t\right)+B \sin\left(\W t\right)\right]$}
%
% \leftcenters{Soit }{$x\left(t\right)=e^{-\xi\w_0t}[A \cos\W t+B \sin(\W t)]$}
%
% De plus, les conditions initiales sont, à $t=0$, $x\left(0\right)=x_0$ , donc $A=x_0$. 
% Calculons de plus la dérivée 
%
% \centers{$\xp\left(t\right)=\left[-A\W\sin{\left(\W t\right)+}B\W \cos\left(\W t\right)\right] \exp{\left(\alpha t\right)}+\alpha\left[A\cos{\left(\W t\right)+}B \sin\left(\W t\right)\right]\exp(\alpha t)$}
%
%
% \leftcenters{Nouvelle condition initiale~:} {$\xp\left(0\right)=v_0$} 
%
% \leftcenters{Donc}{ $B\W+\alpha A=v_0$}
%
% \leftcenters{ Soit }{$B=\frac{v_0-\alpha A}{\W}=\frac{v_0+\xi\w_0x_0}{\W}=\frac{v_0+\xi\w_0x_0}{\w_0\sqrt{1-\xi^2}}$}
% }
%
% \QR
% {Montrer que l’on peut aussi obtenir une forme de la solution du type~:
%
% \centersright{$x\left(t\right)=X_m \exr^{-\xi\w_0t}\cos{(} \mathrm{\W t}+\varphi)$}{(équation 3)} 
%
% \noindent                                                                  
% On exprimera $X_m$ et $\varphi$ en fonction de $A$ et $B$. Quelques outils mathématiques sont donnés en fin de cet exercice. }
% {
% \leftcenters{On nous donne} {$x\left(t\right)=X_me^{-\xi\w_0t}\cos{(}\W t+\varphi)$}
%
% \leftcenters{Et} {$cos\left(a+b\right)=\cos{\left(a\right)}\cos{\left(b\right)}-\sin{\left(a\right)}\sin(b)$}
%
% \leftcenters{Soit}{ $x\left(t\right)=X_me^{-\xi\w_0t}\left[\cos{(\W t)}\cos{(\varphi)}-\sin{\left(\W t\right)\sin{(\varphi)}}\right]$}
%
%
% Par identification avec $x\left(t\right)=e^{-\xi\w_0t}[Acos\W t+Bsin(\W t)]$, il vient		 
%
% \centers{$X_m\cos{(\varphi)}=A \quad \text{et} \quad -X_m\sin\left(\varphi\right)=B$}
%
%
%
% \leftcenters{Ainsi }{$\tan{\left(\varphi\right)=-\frac{B}{A}} \quad \text{et} \quad A^2+B^2=X_{m}^2(\cos^2\left(\varphi\right)+\sin^2\left(\varphi\right))=X_{m}^2$}
%
% \leftcenters{D’où}{ $\varphi=-\arctan\left(\frac{B}{A}\right) \quad \text{et} \quad X_m=\sqrt{A^2+B^2}$}
% }
%
% \QR
% {Représenter qualitativement $x\left(t\right)$ en fonction de $t$ et indiquer sur le tracé $X_m e^{-\xi\w_0t}$ , $x_0$ et $T={2\pi}/{\W}$ la pseudo-période.}
% {
% Allure du graphe ci-contre~: 
%
% \begin{figure}[htbp!]
%   \centering
%   \includegraphics[width=.6\linewidth]{P1corr.jpg}
%   \caption{Allure du graphe. On note la dérivée non nulle en 0.}
%   \label{fig:P1corr}
% \end{figure}
%
% }
%
% \QR
% {Justifier qualitativement que l’énergie mécanique $E(t)$ est une fonction décroissante de $t$. À quoi cela est-il dû~?}
% {
% A cause des frottements, l’énergie mécanique $E(t)$ est une fonction décroissante de $t$.
% }
%
% \QR
% {On envisage deux temps successifs $t_1$ et $t_2$ pour lesquels les déplacements sont $x_1$ et $x_2$, tels que $t_2 > t_1$ 
% et $t_2-t_1=T$ , où $T$ est la période des oscillations amorties. 
% En utilisant l’équation (3) et en considérant que $\xi \ll 1$, montrer que~:            
%
% \centers{$\ln{\left(\frac{x_1}{x_2}\right)} \approx 2\pi\xi$}
% }
% {
% Cela fait penser au décrément logarithmique~:
%
% \centers{
%  $\delta=\ln{\frac{x\left(t\right)}{x\left(t+T\right)}=\ln{\left(\frac{x_1}{x_2}\right)}=\ln{\frac{{X_me}^{\alpha t}\cos{\left(\W t+\varphi\right)}}{{X_me}^{\alpha(t+T)}\cos{\left(\W(T+t)+\varphi\right)}}=\ln{\left(e^{-\alpha T}\right)}}}$}
%
%  \noindent
%   car cosinus est une fonction périodique de période $T$.
% Soit~: 
%
% \centers{$\delta=\ln{\left(\frac{x_1}{x_2}\right)}=-\alpha T=\xi\w_0 T=\xi\w_0\frac{2\pi}{\W}=\xi\w_0\frac{2\pi}{\w_0\sqrt{1-\xi^2}}= \xi\frac{2\pi}{\sqrt{1-\xi^2}}$}
%
%
% Or par hypothèse,  $\xi \ll 1$ , donc $1-\xi^2\approx 1$~; Alors
%
% \centers{ $\ln{\left(\frac{x_1}{x_2}\right)}\approx2\pi\xi$}
% }
%
% \QR
% {Toujours dans le cas où $\xi\ll 1$, le relevé du déplacement horizontal de la plateforme en fonction du temps 
% est représenté en figure 2 ci-dessous. En utilisant les deux points qui sont indiqués sur la figure 2, déterminer les valeurs numériques de $k$, $\xi$ et $\gamma$ (avec leurs unités). Comment ce tracé serait-il modifié si $\xi$ augmentait (un rapide graphique peut permettre d’être plus explicite)~? 
%
% \begin{figure}[htbp!]
%   \centering
%   \includegraphics[width=.6\linewidth]{P2graph}
%   \caption{ Relevé du déplacement horizontal $x$ (en m) de la plateforme de
%   masse $m = \SI{110}{tonnes}$ en fonction du temps $t$ (en s). Les deux temps
% $t_1$ et $t_2$ mentionnés en question Q6 sont indiqués.}
%   \label{fig:P2graph}
% \end{figure}
%
% }
% {
% On lit $x_1=\SI{0,014602}{m}$ et $t_1=\SI{4,004004}{s}$, puis $x_2=\SI{0,010661}{m}$ et $t_2=\SI{8,008008}{s}$.
% 	D’après l’énoncé, on a  $T=t_2-t_1$ et comme $\xi \ll 1$ alors
%
%  \centers{$\w_0\approx\W=\frac{2\pi}{T}=\frac{2\pi}{t_2-t_1}$}
%
%
%  \leftcenters{ car $\W  =\w_0\sqrt{1-\xi^2}$. De plus,}{ $\w_0=\sqrt{\frac{k}{m}}$}
%
% \leftcenters{ Donc}{ $k=m\w_0^2=m\frac{4\pi^2}{\left(t_2-t_1\right)^2} = 110.{10}^3\frac{4\pi^2}{\left(8,008008-4,00400\right)^2} = \SI{2,71e5} {N.m^{-1}}$}
%
%  \noindent
% 	D’autre part d’après Q6, 
%
% \centers{$\ln{\left(\frac{x_1}{x_2}\right)}=2\pi\xi$}
% \leftcenters{Soit }{$\xi=\frac{\ln{\left(\frac{x_1}{x_2}\right)}}{2\pi} = \frac{\ln{\left(\frac{0,014602}{0,010661}\right)}}{2\pi} = \num{5,01e-2}$}
%
% \noindent
% Remarque~: on trouve en effet comme attendu $\xi\ll1$. C'est cohérent. 
%
% \medskip
%
% \leftcenters{Enfin, d’après Q1,} {$\xi=\frac{\gamma}{2}\sqrt{\frac{1}{mk}}$}
%
% \leftcenters{Soit}{ $\gamma=2\xi\sqrt{mk} = 2\times5.01.{10}^{-2}\sqrt{110.{10}^3\times2,71.{10}^5} = \SI{1,73e4} {N.s.m^{-1}}$} .
%
% \noindent
% 	Si $\xi$ augmentait, l’amortissement augmenterait, la décroissance exponentielle serait plus rapide, on 
% verrait moins d’oscillations et la pseudo-pulsation $\W$  diminuerait et la pseudo-période $T={2\pi}/{\W}$ augmenterait.
% }
%
% \enonce
% {
% \medskip
%
% \noindent
% \underline{Outils mathématiques}~: 		
%
% \centers{$cos\left(a+b\right)=\cos{\left(a\right)}\cos{\left(b\right)}-\sin{\left(a\right)}\sin(b) \quad \text{et} \quad
% 					{cos}^2\left(\alpha\right)+{sin}^2\left(\alpha\right)=1$} 
% }

\subimport{/home/nora/Documents/Enseignement/Prepa/bpep/exercices/DS/assemblages_de_ressorts/}{sujet.tex}

\end{document}
