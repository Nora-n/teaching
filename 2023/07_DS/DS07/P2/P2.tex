\documentclass[../DS07.tex]{subfiles}
\graphicspath{{./figures/}}

% \subimport{/home/nora/Documents/Enseignement/Prepa/bpep/exercices/DS/satellites_de_telecommunication/}{sujet.tex}

\begin{document}
\prblm[64]{Satellites de télécommunication\ifcorrige{~\small\textit{(D'après
			Mines-Ponts MP 2007)}}}

\enonce{
	\noindent
	\begin{minipage}[c]{.65\linewidth}
		On se propose d'étudier quelques aspects du fonctionnement de satellites de
		télécommunication en orbite autour de la Terre. Sauf mention contraire, on
		considérera que la Terre est une sphère homogène de masse $M_T$, de rayon
		$R_T$ et de centre $O$, immobile dans l'espace, sans rotation propre.
		\smallbreak
		On donne les valeurs numériques suivantes~:
		\begin{center}
			\renewcommand{\arraystretch}{1.5}
			\begin{tabular}{c|c|c}
				$\Gc $                              & $R_T$            & $M_T$              \\
				\hline
				\hline
				$\SI{6,67e-11}{m^3 kg^{-1} s^{-2}}$ & $\SI{6 370}{km}$ & $\SI{5,97e24}{kg}$ \\
			\end{tabular}
		\end{center}
		\smallbreak
		Sur la Figure~\ref{fig:schema}, $N$ est le pôle Nord et $S$ le pôle Sud.
		Satellite $P$, point $Q$ et ligne des horizons $AB$. Le plan orbital
		représenté est dit polaire (la ligne des pôles est N'S').
	\end{minipage}
	\hfill
	\noindent
	\begin{minipage}[c]{.30\linewidth}
		\begin{center}
			\includegraphics[width=\linewidth]{schema}
			\captionof{figure}{Schema question~\ref{q:useschema}}
			\label{fig:schema}
		\end{center}
	\end{minipage}
}%

\subsection{Couverture d'un réseau de satellite}

\QR[9]{%
	\label{q:kepler}
	Un satellite de masse $M_S$ est en orbite circulaire de centre $O$, à une
	altitude $h=\SI{800}{km}$. Établir la relation entre la période de révolution
	$T$ et $h$. Exprimer de même la relation entre la vitesse $v = \norm{\vf}$ et
	$h$ puis effectuer les applications numériques pour $T$ et $v$.
}{%
	Dans le référentiel géocentrique considéré comme galiléen \pt{1} on ne prend
	en compte que la force de gravitation exercée par la Terre $\Ff \stm{=} -
		\cfrac{k}{r^2} \er$ avec $k = \Gc M_TM_S$. On a de plus $r=R_T+h$. On peut
	ainsi appliquer le PFD $m\af = \Ff$ au satellite dans le repère
	polaire $O, \er, \ez$~: \pt{1}
	\[
		\begin{array}{rlll}
			\DS
			-M_Sr \tp^2 & = - \frac{k}{r^2} & \pt{1} &
			\\
			M_Sr \tpp   & = 0               & \pt{1} &
		\end{array}
	\]
	De la deuxième équation, on obtient $\tp = cte \Rightarrow v = r
		\tp = \cte$. \pt{1} On peut ainsi ré-exprimer l'accélération radiale $a_r =
		- v^2/r$ d'où:
	\[
		M_S \frac{v^2}{r} = \frac{k}{r^2}
		\Lra
		\boxed{v \stm{=} \sqrt{\frac{GM_T}{R_T+h}}}
	\]
	De plus, on sait que $T = \frac{2\pi r}{v} \Rightarrow
		\boxed{\frac{T^2}{(R_T+h)^3} \stm{=} \frac{4 \pi^2}{GM_T} }$. On retrouve
	ainsi la troisième loi de Kepler. Les A.N.s donnent $\xul{T=\SI{6.07e2}{s}}$
	et $\xul{v = \SI{7.46}{km.s^{-1}}}$. \pt{1}
}%

\QR[2]{\label{q:viriel}
	Soient $\Ec_c$ et $\Ec_p$ l'énergie cinétique du satellite et son énergie
	potentielle dans le champ de gravitation de la Terre. Donner sans
	démonstration l'expression de $\Ec_p$ puis établir le «~théorème du
	viriel~»~: $2\Ec_c + \Ec_p = 0$.
}{%
	L'énergie potentielle a pour expression $\Ec_p(r) \stm{=} -\cfrac{k}{r}$.
	On a $2\Ec_c + \Ec_p = M_S v^2 - \frac{G M_T M_S}{r} =  M_S \pa{ \frac{GM_T}{r} -
			\frac{G M_T }{r}} \stm{=} 0$ d'où le résultat.
}%

\QR[5]{%
	\label{q:useschema}%
	À chaque position $P$ du satellite correspond un point $Q$ sur la Terre à la
	verticale de ce point. L'ensemble des points $Q$ définit la trace de la
	trajectoire.
	\smallbreak
	Pour une observatrice située en $Q$, la durée de visibilité $\tau$ d'un
	satellite est l'intervalle de temps entre son apparition sur l'horizon (point
	$A$ de la Figure~\ref{fig:schema}) et sa disparition sous l'horizon (point
	$B$). Exprimer $\tau$ en fonction de $\varphi$ et $T$ puis montrer que
	\[
		\tau =  2\arccos\frac{R_T}{R_T+h}\, f(h)
	\]
	et donner l'expression de $f(h)$.
	\smallbreak
	Réaliser l'application numérique toujours pour $h=\SI{800}{km}$.
}{%
	Il convient pour cela d'établir l'expression de l'angle $\varphi$ tel que
	$\cos(\varphi) = R_T/(R_T+h)$. \pt{1} La vitesse du satellite étant uniforme, on en
	déduit $\DS\tau \stm{=} \frac{2\varphi}{2\pi}T$ soit au final~:
	\[
		\boxed{
			\tau \stm{=} 2\arccos\frac{R_T}{R_T+h}
			\underbracket[1pt]{\sqrt{\frac{(R_T+h)^3}{GM_T}}}_{= f(h) \pt{1}}
		}
	\]
	L'application numérique donne $\xul{\tau=\SI{9.2e2}{s}}$. \pt{1}
}%

\QR[3]{%
	\label{q:train}
	Calculer $T/\tau$. Pour les besoins de la téléphonie mobile, on place sur des
	orbites polaires (c'est-à-dire contenues dans un plan méridien terrestre comme
	sur la figure~\ref{fig:schema}) un ensemble de satellites identiques, appelé
	«~train de satellites ~».
	\smallbreak
	Ces satellites sont disposés régulièrement sur leur orbite polaire commune, à
	l'altitude de \SI{800}{km}. Calculer le nombre minimal de satellites
	nécessaires pour former un «~train~» afin que tous les points au sol, dans le
	même plan méridien que l'orbite, voient au moins un satellite à tout instant.
}{%
	\label{q:train}
	On a simplement $\cfrac{T}{\tau} = \cfrac{2\pi}{2\varphi} \stm{=}
		\dfrac{\pi}{\arccos\frac{R_T}{R_T+h}} \stm{\approx} 6,6$. Le satellite est
	ainsi visible pendant $1/6,6$ ième de son trajet. Il faudra donc 7 satellites
	\pt{1} pour garantir la couverture permanente au sol (arrondi au supérieur).
}%

\QR[3]{%
	Combien d'orbites polaires de ce type faut-il pour couvrir la surface de la
	Terre, c'est à dire pour que chaque point de la surface terrestre voie au
	moins un satellite à tout instant~? Combien doit-on disposer de satellites en
	tout~?
}{%
	D'après la question précédente, il faudrait aussi 7 «~trains de satellites~»
	pour couvrir toutes les longitudes. \pt{1} Cependant, un train de satellite
	couvre «~deux côtés~» et donc $\lfloor 7/2 \rfloor = 4$ trains suffisent.
	\pt{1} On aboutit ainsi à un total de $7\times 4 = 28$ satellites. \pt{1}
}%

\QR[7]{\label{q:trot} La prise en compte de la rotation de la Terre modifie le
	résultat de la question précédente. Dans cette question, on s'interroge sur la
	pertinence d'utiliser plutôt un satellite géostationnaire. Calculer la période
	et l'altitude d'un satellite placé sur orbite géostationnaire. La notion de
	durée de visibilité garde-t-elle, dans ce cas, un sens~? Quels sont les
	avantages et les inconvénients d'un satellite géostationnaire comparé au train
	de la question \ref{q:train}~?
}{%
	Sur l'orbite géostationnaire, la période de révolution du satellite est celle
	de la terre $T_T \approx \SI{86e3}{s}$. \pt{1} On peut utiliser la 3ième loi
	de \textsc{Kepler}~:
	\[
		\frac{T_T^2}{(R_T+h_g)^3} =
		\frac{4 \pi^2}{GM_T} \qsoit
		\boxed{
			h_g \stm{=} \pa{\cfrac{GM_TT_T^2}{4\pi^2}}^{\mathrlap{1/3}}-R_T
			\stm{\approx}\SI{35700}{km}
		}
	\]
	La notion de «~visibilité~» est à prendre avec prudence~: pour un point du
	globe, le satellite est alors soit visible et la durée de visibilité est
	infinie, soit invisible. \pt{1} Il ne faut pas utiliser la formule de la question
	\ref{q:train} pour la durée de visibilité car on y faisait l'hypothèse d'une
	Terre immobile (le schéma permettant le calcul de $\varphi$ est incorrect dans
	ce cas~!~!).
	\smallbreak
	Pour une zone donnée de la Terre, il suffit de disposer d'un seul \pt{1}
	satellite au lieu d'une bonne quarantaine. Mais il est beaucoup plus éloigné,
	ce qui pose des problèmes de perte de transmission. \pt{1}
	\smallbreak
	Il faut aussi remarquer que les Pôles et les régions qui les entourent ne
	voient pas les satellites géostationnaires. \pt{1}
}%

\subsection{Influence des frottements aérodynamiques}

\QR[5]{\label{q:dhdt}
La Terre est entourée d'une atmosphère qui s'oppose au mouvement du satellite.
La force de frottement $\Ff_a$ créée par l'atmosphère est proportionnelle
au carré de la vitesse $v$ du satellite et elle s'exprime par $\Ff_a =  -
	\alpha M_S v \vf$, où $\alpha$ a une valeur positive, constante dans cette
question. Déterminer la dimension de $\alpha$ puis appliquer ensuite le
théorème de la puissance mécanique en supposant que le théorème du Viriel
établi à la question \ref{q:viriel} reste valable en présence de $\Ff_a$ .
En déduire finalement que~:
\begin{equation}
	\label{eq:h}
	\dv{h}{t} = -2 \alpha \sqrt{G M_T}\sqrt{R_T+h}
\end{equation}
}{%
On a $\dim F = {\rm M.L.T^{-2}} \Lra \dim (\alpha) . {\rm M.L^2.T^{-2}}$. On
en déduit par identification que $\boxed{\dim \alpha = {\rm L^{-1}}}$. \pt{1}
\smallbreak
Le TPM appliqué au satellite donne~:
\[
	\dv{\Ec_m}{t} \stm{=} \Pc(\Ff_a)
	\Lra
	\dv{\Ec_c +\Ec_p}{t} \stm{=} - \alpha M_S v^3 =  \frac{1}{2} \dv{\Ec_p}{t}
\]
De plus, $v^2 = 2\Ec_C/M_S \stm{=} -\Ec_P/M_S = GM_T/(R_T+h)$. On en déduit en
combinant ces résultats que~:
\[
	- \alpha M_S \frac{GM_T}{R_T+h}^{3/2} =
	\dot{h} \frac{GM_SM_T}{2(R_T+h)^2}
	\Lra
	\dv{h}{t} \stm{=}
	-2 \alpha \sqrt{G M_T}\sqrt{R_T+h}
\]
}%

\QR[8]{\label{q:chute}
	Un satellite placé sur une orbite d'altitude \SI{800}{km} subit une diminution
	d'altitude d'environ \SI{1}{m} par révolution~; sa vitesse est, en norme, très
	peu affectée au bout d'une révolution. En déduire une estimation au premier
	ordre de $\alpha$ (ne pas s'étonner de la petitesse extrême du résultat~!).
	Calculer, avec la même approximation, ce qu'il advient de l'altitude au bout
	de 10 ans de fonctionnement du satellite. Comparer à la solution exacte de
	l'équation (\ref{eq:h}) (obtenue par intégration de cette équation). Le fait
	d'avoir une augmentation de la vitesse en présence d'une force opposée au
	mouvement est-il paradoxal~?
}{%
	Entre le début et la fin de la révolution, $R_T+h$ n'a quasiment pas varié et
	on peut supposer ce terme constant (on note alors $h_0$ l'altitude initiale du
	satellite)~:
	\begin{gather*}
		\Delta h \stm{=}
		- 2\alpha \underbracket[1pt]{\Delta t}_{=T} \sqrt{G M_T} \sqrt{R_T+h_0}
		\Rightarrow
		\alpha = - \frac{\Delta h}{2T\sqrt{GM_T}\sqrt{R_T+h_0}}
		\\\Leftrightarrow
		\boxed{
			\alpha \stm{=}
			-\frac{\Delta h}{4\pi (R_T+h_0)^2} \approx
			\SI{1.53e-15}{m^{-1}}
		}
	\end{gather*}
	En dix années, on a effectué $n = \frac{\Delta T}{T} = 10 \frac{T_T}{T}\approx
		52000$ orbites donc au premier ordre (en supposant $\Delta h$ identique à
	chaque période), on a $\xul{\Delta h \approx -\SI{52}{km}}$. \pt{1}
	\smallbreak
	Une résolution exacte de l'équation (à l'aide de la méthode de séparation des
	variables \pt{1})~:
	\begin{align*}
		\frac{\dd{h}}{\sqrt{R_T+h}} & =
		-2\alpha\sqrt{GM_T} \dt
		\Lra
		2(\sqrt{R_T+h_1} - \sqrt{R_T+h_0}) \stm{=}
		-2\alpha\sqrt{GM_T} \Delta T
		\\\Rightarrow
		\Aboxed{
		\Delta h = h_1- h_0         & \stm{=}
			\pa{\sqrt{R_T+h_0} - \alpha \sqrt{GM_T} \Delta T}^2 - R_T - h_0 \approx
			- \SI{51,3}{km}
		}
		\pt{1}
	\end{align*}
	Ce résultat est très proche de celui obtenu à l'aide de l'approximation. Il
	peut sembler surprenant qu'une force qui s'oppose au mouvement se concrétise
	par une augmentation de vitesse~: le freinage d'une voiture (force
	aérodynamique par exemple) réduit sa vitesse. Mais c'est sans compter sur
	l'énergie potentielle~: à une orbite plus basse correspond une vitesse plus
	élevée. \pt{1}
}%

\QR[7]{%
	En réalité, les frottements dépendent de la densité de l'atmosphère et donc de
	l'altitude. Dans un certain domaine d'altitudes, $\alpha$ varie selon la loi
	$\alpha(h)= \cfrac{\gamma}{h^\beta}$, où $\gamma$ et $\beta$ sont positifs. Le
	même satellite que celui de la question \ref{q:chute} (perdant 1 mètre par
	révolution pour $h \approx \SI{800}{km}$) perd, à l'altitude de \SI{400}{km},
	2 mètres par révolution. Calculer $\gamma$ et $\beta$.
}{%
	L'énoncé nous donne que $\Delta{h\ind{haut}} = \frac{1}{2}\Delta{h\ind{bas}}$.
	\pt{1} De plus, on observe que $\dv{h}{t} \propto \alpha(h)$, \pt{1} car les
	autres facteurs varient peu (question~\ref{q:dhdt}). On en déduit ainsi que~:
	\[
		\frac{\dv{h}{t}\/(h=h\ind{haut})}{\dv{h}{t}\/(h=h\ind{bas})} \stm{\approx}
		\frac{\frac{\Delta{h\ind{haut}}}{T_T}}{\frac{\Delta{h\ind{bas}}}{T'_T}}
		\Lra
		\left( \frac{h\ind{bas}}{h\ind{haut}} \right)^\beta =
		\frac{1/T_T}{2/T_T'}
		\Lra
		\boxed{\beta = \dfrac{\log(T_T'/(2T_T))}{\log(h_{bas}/h_{haut})}} \pt{1}
	\]
	avec $T_T'$ la période de révolution à \SI{400}{km} d'altitude telle que,
	d'après~\ref{q:trot}, $T_T'/T=\frac{R_T+h\ind{bas}}{R_T+h\ind{haut}}^{3/2}
		\approx 0,917$. \pt{1} On en déduit au final $\xul{\beta \approx 1,13}$ \pt{1}
	puis $\gamma = h\ind{haut}^\beta \times \alpha(h\ind{haut}) \approx
		\SI{7,2e-9}{SI}$. \pt{1} En pratique, la valeur de $\gamma$ est très sensible
	aux différents arrondis réalisés pour obtenir $\beta$, et seul son ordre de
	grandeur à du sens.
}%

\subsection{Stabilisation de l'orientation d'un satellite par gradient de
	gravité}

\enonce{
	\noindent
	\begin{minipage}[c]{.53\linewidth}
		La méthode de stabilisation d'attitude par gradient de gravité a été mise en
		œuvre pour les satellites artificiels afin qu'ils présentent vers la Terre
		toujours le même côté. Elle ne requiert aucune ressource d'énergie
		embarquée. Le principe de cette méthode a été établi par Lagrange, au
		XVIIème, afin d'expliquer pourquoi la Lune présente toujours la même face
		vers la Terre.
		\smallbreak
		\textbf{Modèle~:} le satellite est constitué de deux points matériels $M_1$
		et $M_2$ de masses identiques $m  =\frac{1}{2} M_S$ reliés par une tige
		rigide de masse nulle et de longueur $2\ell$.
		\smallbreak
		Le centre de masse $S$ du satellite décrit autour de la Terre une orbite
		circulaire uniforme de rayon $r_0 = R_T + h$ avec $\ell \ll r_0$. Le
		référentiel géocentrique $(R)$ lié au repère $(Oxyz)$ est supposé galiléen.
		\smallbreak
		On appelle $\th$ l'angle de $\vvr{M_1M_2}$ avec l'axe $Ox'$ de $(R')$.
		On cherche à déterminer les éventuelles positions d'équilibre du satellite
		et leur stabilité. On suppose qu'il n'y a pas de frottements dans toute
		cette partie.
	\end{minipage}
	\hfill
	\begin{minipage}[c]{.43\linewidth}
		\vspace{0pt}
		\begin{center}
			\includegraphics[width=\linewidth]{schema2}
			\captionof{figure}{Le satellite composé des points $M_1$ et $M_2$ reliés
				par une tige de longueur $2\ell$.}
			\label{fig:schema2}
		\end{center}
	\end{minipage}
}%

\QR[2]{%
	Exprimer les distances $r_1 = \norm{\vvr{OM_1}}$ et $r_2 =
		\norm{\vvr{OM_2}}$ en fonction de $r_0$, $\ell$ et $\th$
}{%
	On a
	$\vvr{OM_1} \stm[-1]{=} \vvr{OS}+\vvr{SM_1}
		\Lra
		r_1 = \sqrt{r_0^2 + \ell^2 +2r_0\ell \cos(\th)}
	$.
	De même, on trouve $r_2 \stm[-1]{=} \sqrt{r_0^2 + \ell^2 -2r_0\ell \cos(\th)}$
}%

\enonce{
	On rappelle le développement limité à l'ordre 2 suivant~:
	\[
		\frac{1}{\sqrt{1+x}} = 1 - \frac{1}{2}x + \frac{3}{8}x^2 + o(x)
	\]
	De plus, on admet que l'énergie cinétique $\Ec_c$ du satellite s'exprime selon
	$\Ec_c = \frac{1}{2}M_s \ell^2\tp^2$
	% (démonstration peu évidente dans le corrigé).
}%

% \siCorrige{
% 	\textbf{Démonstration~:}
% 	\smallbreak
% 	On a par définition $\Ec_c = \Ec_{c,1}+\Ec_{c,2} = \frac{1}{2}m (v_1^2 + v_2^2)$. De
% 	plus, on sait que $\vf_1 = \dv{\vvr{OM_1}}{t} = \dv{\vvr{OS}}{t}
% 		+\dv{\vvr{SM_1}}{t} = \vf_g + l(\tp+\W) \vec e_{\th'}$
% 	puis que $\vec v_2 = \vec v_g - l(\tp+\W) \vec e_{\th'}$. On en
% 	déduit que~:
% 	\[
% 		\Ec_c =
% 		\frac{1}{2}m (2 v_g^2 + 2 l^2 \tp^2 +0)
% 		\Rightarrow
% 		\Ec_c =
% 		\frac{1}{2}M_S (r_0\W)^2 + \frac{1}{2}M_S l^2(\tp+ \W)^2
% 	\]
% 	\textbf{Remarques~: }
% 	\smallbreak
% 	\begin{itemize}
%     \item Le vecteur $\vvr{SM_1}$ tourne en effet à la vitesse angulaire $\W +
% 			      \tp$ par rapport au repère fixe dans $\Rc $ d'où le
% 		      résultat.
% 		\item Cependant, cette expression de l'énergie cinétique complique beaucoup la
% 		      suite du problème. En utilisant toutefois la conservation du moment
% 		      cinétique $L_0 =\pa{\vvr{OM_1}\wedge m \vf _1 + \vvr{OM_2}\wedge m
% 				      \vf _2}\cdot
% 			      \ez= M_Sr_0^2\W + M_Sl^2 (\W+\tp)$ implique $\dv{L_0}{t} =
% 			      0 \Rightarrow r_0^2 \dot{\W}  = -l^2 (\tpp + \dot{\W})$, on
% 		      peut établir que
% 		      \[
% 			      \dv{\Ec_c}{t} =
% 			      M_S r_0^2 \dot{\W}\W +
% 			      M_S l^2 (\tpp +
% 			      \dot{\W})(\tp +
% 			      \W) =
% 			      M_s l^2 \tp(\tpp+\dot{\W}) \approx
% 			      M_s l^2 \tp \tpp
% 		      \]
% 		      car $\dot{\W} = \frac{\tpp}{1+ (r_0/l)^2} \Rightarrow
% 			      \dot{\W} \ll \tpp$ (toujours via  l'expression du moment
% 		      cinétique).
% 		\item On peut alors ré-integrer cette relation pour obtenir~:
% 		      \[
% 			      \boxed{\Ec_c = \frac{1}{2}M_S l^2 \tp^2}
% 		      \]
% 		\item Ces résultats impliquent que le mouvement du centre de masse du
% 		      satellite n'est pas exactement uniforme. D'après la conservation du
% 		      moment cinétique, une variation de $\tp$ entraine une variation
% 		      de $\W$. Cette variation est toutefois très faible (de l'ordre de
% 		      $(l/r_0)^2$)
% 	\end{itemize}
% }%

\QR[5]{%
	Montrer que l'énergie mécanique du système s'écrit, en procédant aux
	approximations qui s'imposent ($\ell \ll r_0$)~:
	\[
		\Ec_m \approx
		- \frac{GM_TM_S}{r_0}
		\pa{1 +\frac{1}{2} \frac{\ell}{r_0}^2 (3\cos^2(\th) -1)} +
		\frac{1}{2}M_S \ell^2 \tp^2
	\]
}{%
	On commence par s'intéresser aux termes d'énergie potentielle $\Ec_{p,i} =
		-k/r_i$ avec $k = GM_T m$. On obtient ainsi en posant $\ep=\ell/r_0$:
	\begin{align*}
		\Ec_{p,12} & =
		-\frac{k}{r_{12}} =
		-\frac{k}{\sqrt{r_0^2+\ell^2 \pm 2r_0\ell\cos(\th)}} \stm{=}
		-\frac{k}{r_0} \frac{1}{\sqrt{1+\ep^2 \pm 2 \ep \cos(\th)}}
		\\\Leftrightarrow
		\Ec_{p,12} & \stm(un){\stm{=}}
		-\frac{k}{r_0}
		\pa{1 -
		\frac{1}{2} ( \ep^2 \pm 2 \ep\cos(\th) ) +
		\frac{3}{8} ( \ep^2 \pm 2 \ep\cos(\th) )^2} + o(\ep^2)
	\end{align*}
	On peut maintenant ajouter les deux termes d'énergies potentielles (avec
	encore un terme quadratique à développer puis simplifier à droite du terme
	d'énergie potentielle)~:
	\[
		\Ec_{p,1} + \Ec_{p,2} \stm{=}
		-\frac{k}{r_0} \pa{2 -\ep^2 +3 \ep^2 \cos^2(\th)} + o(\ep^2)
	\]
	On combine ensuite ce terme avec l'expression de l'énergie cinétique donnée
	dans l'énoncé~:
	\[
		\Ec_m =
		\Ec_c + \Ec_p
		\Lra
		\boxed{
			\Ec_m \stm{=}
			-\frac{GM_TM_S}{r_0}
			\pa{1 +\frac{1}{2} \frac{\ell}{r_0}^2 \pa{3 \cos^2(\th)-1}} +
			\frac{1}{2}M_S (\ell\tp)^2
		}
	\]
	car $m=M_S/2$, d'où le résultat~!
}%

\QR[8]{%
	En déduire l'équation du mouvement. Indiquer, par un raisonnement sur cette
	équation différentielle, les positions d'équilibre et préciser, pour celle(s)
	qui sont stable(s), la pulsation des petites oscillations autour de ces
	dernières. Conclure.
}{%
	On applique le TPM dans le référentiel géocentrique au satellite qui n'est
	soumis à aucune force non conservative~:
	\begin{gather*}
		\dv{\Ec_m}{t} \stm{=} 0
		\Lra
		- \frac{GM_T}{r_0} \frac{\ell}{r_0}^2
		3\cos(\th)(-\sin(\th))\tp +
		\ell^2 \tp \tpp \stm{=} 0
		\\\Lra
		\tpp +
		\frac{3GM_T}{2r_0^3}\sin(2\th) = 0
		\Lra
		\boxed{\tpp + 3 \W^2 \frac{\sin(2\th)}{2} \stm{=} 0}
	\end{gather*}
	On est à l'équilibre lorsque $\tpp=0$ soit ici pour $\th=p
		\frac{\pi}{2},~p \in \mathbb{N}$. \pt{1}
	\begin{itemize}
		\litem{10pt}{\pt{1}} Pour $\th = 0+x$ avec $x \ll 1$, on a comme équation du
		mouvement $ \xpp + 3\W^2 x = 0$ qui est l'équation de l'oscillateur
		harmonique donc la position d'équilibre est stable et la pulsation des
		petites oscillations vaut $\w_0 = \W\sqrt{3}$

		\litem{10pt}{\pt{1}} Pour $\th = \pi/2+x$, on a maintenant $\xpp- \w_0^2 x$.
		Cette position d'équilibre n'est pas stable.

		\litem{10pt}{\pt{1}} Pour $\th = \pi+x$, on a $\xpp + \w_0^2 x=0$ et on
		retrouve la même équation que pour la première position d'équilibre. Cette
		position d'équilibre est donc aussi stable et de pulsation $\w_0$

		\litem{10pt}{\pt{1}} Pour $\th = 3\pi/2+x$, on obtient au final $\xpp- \w_0^2
			x$~: équilibre instable.
	\end{itemize}
	Ainsi, seules les positions verticale (à l'endroit ou à l'envers) sont
	stables. \pt{1} En cas de léger décalage, le satellite va donc osciller autour
	de la position d'équilibre verticale et donc toujours présenter le même côté
	vers la Terre. \pt{1}
}%
\end{document}
