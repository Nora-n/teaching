\documentclass[../DS10.tex]{subfiles}%
\graphicspath{{./figures/}}%

\begin{document}%
\section[9]"P"{Induction du champ magnétique terrestre dans un téléphone portable}
\enonce{%
	Une expérimentatrice tient son téléphone portable dans sa main. Son bras passe
	rapidement d’une position horizontale à une position verticale afin d’entrer
	en communication. On tient compte de la composante horizontale du champ
	magnétique terrestre d’environ $B = 2.10^{-5} \si{.\tesla}$. On modélise le
	circuit électronique du téléphone par un circuit fermé.
}%

\QR[7]{%
	Évaluer l'ordre de grandeur de la fem induite dans le téléphone lors de son
	déplacement.
}{%
	La surface du circuit électrique contenu dans le téléphone est de l'ordre de~:
	\[
		S \stm{=} \SI{10}{cm} \times \SI{5}{cm} = \SI{50}{cm^2}
		\quad
		\pt{1}
	\]
	Initialement, le téléphone est à plat, donc le champ magnétique est parallèle
	à la surface du téléphone. Son flux est alors nul~:
	\[
		\Phi_i=0
		\quad
		\pt{1}
	\]
	Lorsque le téléphone est au niveau de l'oreille, on peut supposer que le champ
	magnétique terrestre est perpendiculaire à la surface du téléphone. Le flux
	magnétique à travers le téléphone est alors maximal et vaut~:
	\[
		\Phi_f = BS
		\quad
		\pt{1}
	\]
	La durée de cette action est d'environ~:
	\[
		\Delta t = \SI{1}{s}
		\quad
		\pt{1}
	\]
	D'après la loi de \textsc{Faraday}, l'ordre de grandeur de la fem induite est
	alors~:
	\[
		\boxed{|e| \stm{=} \frac{\Phi_f - \Phi_i}{\Delta t} \stm{=} \SI{e-7}{V}}
	\]
}%

\QR[2]{%
	Cette tension induite perturbe-t-elle le fonctionnement du téléphone~?
}{%
	Cette fem est très faible par rapport aux tensions utilisées dans un téléphone
	(de l'ordre du mV). \pt{1} Elle ne va donc pas perturber son fonctionnement.
	\pt{1}
}%

\end{document}
