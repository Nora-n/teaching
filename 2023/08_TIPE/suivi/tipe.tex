\documentclass[a4paper, 11pt, final, garamond]{book}
\usepackage{cours-preambule}
\usepackage[french]{babel}

\raggedbottom

\makeatletter
\renewcommand{\@chapapp}{TIPE -- Jeux, sport}
\renewcommand\thechapter{\!\!}
\makeatother

\let\SavedIndent\indent
\protected\def\indent{%
  \begingroup
    \parindent=\the\parindent
    \SavedIndent
  \endgroup
}
\setlength{\parindent}{0pt}
\renewcommand{\thesection}{\arabic{section}}

\titleformat{\section}{\Large\bfseries}{
  \begin{tikzpicture}[scale=0.55, overlay]
    \draw[line width=1pt, color=darkgray, xshift=1em, yshift=0.6em]
      (1,-0.4) -- (1,0.8) --(0.8,1) -- (-0.8,1) -- (-1,0.8) -- (-1,-0.8) --
      (-0.8,-1) -- (31.75,-1);
    \node[xshift=0.6em, yshift=0.35em] at (0,0)
      {\rmfamily\textcolor{darkgray}{\thesection}};
  \end{tikzpicture}}{0.4em}{\hspace{1.4em}\hyperref[tab:tipe]{$\uparrow$}\,\,}

\setlist[itemize, 1]{leftmargin=10pt, label=$\diamond$}

\newcommand{\ccb}{\cellcolor{brandeisblue!20}\textbf{?}\,}
\newcommand{\ccg}{\cellcolor{limegreen!20}\cmark}
\newcommand{\ccr}{\cellcolor{red!10}\xmark}
\newcommand{\ccy}{\cellcolor{yellow!20}$\nearrow$}
\newcommand{\cco}{\cellcolor{orange!20}$\approx$}
\newcommand{\cbc}{\cellcolor{brandeisblue!20}}
\newcommand{\cgc}{\cellcolor{limegreen!20}}
\newcommand{\crc}{\cellcolor{red!10}}
\newcommand{\cyc}{\cellcolor{yellow!20}}
\newcommand{\coc}{\cellcolor{orange!20}}

\begin{document}
\setcounter{chapter}{0}

\chapter{Suivi avancement MPSI3 -- 2023}
\label{ch:tipe}

\begin{table}[h]
  \centering
  \caption{Avancement global.}
  \label{tab:tipe}
  \begin{threeparttable}
    \makebox[\linewidth]{
    \begin{tabular}{lclM{1.5cm}M{1.5cm}M{1.5cm}M{1.5cm}M{1.5cm}}
      \toprule
      \textbf{Élève} & \multicolumn{2}{l}{\textbf{Sujet}} &
      \textbf{Pbatique} & \textbf{Biblio} & \textbf{Manip} & \textbf{Plan}
                        & \textbf{Diapos}
      \\\midrule
      \hyperref[ch:elisa]{Élisa} &
      \ccg & \cgc Ping-pong & \ccg & \ccg & \ccg & \ccg & \ccy
      \\
      \hyperref[ch:walid]{Walid} &
      \ccg & \cgc Décrochage & \ccg & \ccg & \ccg & \ccg & \cco
      \\
      \hyperref[ch:axel]{Axel} &
      \ccg & \cgc Perche & \ccg & \ccg & \ccg & \ccy & \ccy
      \\
      \hyperref[ch:davidylan]{David \& Ylan} &
      \ccg & \cgc Toupies & \ccg & \ccg & \ccg & \ccg & \ccy
      \\
      \hyperref[ch:serena]{Séréna}\tnote{7} &
      \ccg & \cgc Prothèse athlétisme & \ccg & \ccg & \ccy & \ccy & \ccr
      \\
      \hyperref[ch:eliott]{Eliott}\tnote{5} &
      \ccg & \cgc Voiture inclinée & \ccg & \ccy & \ccr & \ccy & \cco
      \\
      % \hyperref[ch:time]{Timé} &
      % Contrôle & dopage & \ccr & \cco & \ccr & \cco
      % \\
      \hyperref[ch:marie]{Marie}\tnote{7} &
      \ccg & \cgc Vendée Globe & \ccg & \ccg & \ccg & \ccy & \ccr
      \\
      \hyperref[ch:natiao]{Natiao} &
      \ccg & \cgc Redressement bateau & \ccg & \ccg & \ccg & \ccy & \ccy
      \\
      % \hyperref[ch:clement]{Clément}\tnote{2} &
      % ? & & \? & ? & ? & ? & \cco
      % \\
      \hyperref[ch:amineferdinand]{Amine \& Ferdinand}\tnote{6} &
      \ccg & \cgc Boomerang & \ccg & \ccg & \ccy & \ccg & \ccg
      \\
      % \hyperref[ch:keo]{Kéo}\tnote{2} &
      % Boxe & sous l'eau & \ccr & \ccr & \ccr & \cco
      % \\
      \hyperref[ch:leo]{Léo}\tnote{6} &
      \ccg & \cgc Badminton & \ccg & \ccg & \ccy & \ccg & \ccr
      \\
      \hyperref[ch:victor]{Victor}\tnote{1,7} &
      \ccg & \cgc Saut hauteur & \ccy & \ccy & \cco & \ccy & \cco
      \\
      % \hyperref[ch:bruno]{Bruno} &
      % \ccb & \cbc Arc & \ccb & \ccb & \ccb & \ccb
      % \\
      % \hyperref[ch:arsene]{Arsène}\tnote{1,5,7} &
      % \ccb & \cbc & \ccb & \ccb & \ccb & \ccb
      % \\
      \hyperref[ch:evanne]{Évanne} &
      \ccg & \cgc Cyclisme & \ccy & \ccg & \ccg & \ccy & \ccy
      \\
      \hyperref[ch:ambdaloi]{Ambdaloi}\tnote{3,5,6,7,10,11,12,13,14} &
      \ccg & \cgc Saut élastique & \ccg & \ccy & \ccr & \ccy & \ccr
      \\
      \hyperref[ch:benjamin]{Benjamin} &
      \ccg & \cgc Cordage tennis & \ccg & \ccy & \ccg & \ccy & \ccr
      \\
      \hyperref[ch:gaspard]{Gaspard}\tnote{5} &
      \ccg & \cgc Foule marathon & \ccg & \ccg & \ccg & \ccg & \ccy
      \\
      \hyperref[ch:mathisbruno]{Mathis\tnote{11}~~\& Bruno} &
      \ccg & \cgc Arc & \ccy & \ccy & \ccr & \ccy & \cco
      \\
      \hyperref[ch:alize]{Alizé} &
      \ccg & \cgc Course para et pas & \ccy & \ccg & \ccy & \ccy & \ccy
      \\
      \hyperref[ch:samy]{Samy} &
      \ccg & \cgc Volley flottant & \ccg & \ccg & \ccg & \ccg & \ccg
      \\
      \hyperref[ch:maxime]{Maxime} &
      \ccg & \cgc Athlétisme & \ccg & \ccg & \ccg & \ccg & \ccg
      \\
      \bottomrule
    \end{tabular}
    }
    \begin{tablenotes}[flushleft]
        \item[$x$] Élève absent à la $x$\ieme\ séance
    \end{tablenotes}
  \end{threeparttable}
\end{table}

\chapter{Séréna \textsc{Chapelet}}
\label{ch:serena}
\section{Suivi}
\subsection{10 février}
\begin{itemize}
    \litem{Sujet}~: Prothèse sport handicap paralympique
\end{itemize}

\subsection{03 mars}
\begin{itemize}
  \litem{Pbatique}~: Comment optimiser les lames de saut pour les athlètes
    amputæs~? Battre Usain Bolt en para~?
  \litem{Biblio}~: projet de thèse 2024 en partenariat avec l'X
  \litem{Manipulation}~: Prise de photos pour étude du système lame-sol.
    Conversion élan/vitesse~: étude énergétique de élasticité conservation…
\end{itemize}

\subsection{10 mars}
\begin{itemize}
  \litem{Pbatique}~: Comment améliorer la course par prothèse~?
  \item Si handi, gestion emboiture, dissipation énergie lame et membre~?
  \item Choix des matériaux pour la lame~?
  \item Valide avec prothèse bionique~: conservation de l'énergie meilleure
    que leur membre valide.
  \litem{Biblio}~: trouvée mais pas notée. À décortiquer pour voir ce qui est
    caractérisable.
  \litem{Manip}~: analogie ressort, raideur~; imprimante 3D~; rigidité bande
    toute droite.
\end{itemize}

\subsection{17 mars}
\begin{itemize}
    \litem{Manip}~: pb de matériau. Voir pour un patron sur Solidworks
\end{itemize}

\subsection{24 mars}
\begin{itemize}
    \item Recherche de modèles 3D.
\end{itemize}

\subsection{31 mars}
\begin{itemize}
  \item SolidWorks~: template non trouvé, tentative d'usinage à la main~: faire
    une version simple~?
\end{itemize}

\subsection{07 avril}
\begin{center}
  Absente.
\end{center}

\subsection{14 avril}
\begin{itemize}
  \item Modélisation sur SolidWorks~: trouver cotations.
  \item Vacances~: matériaux à utiliser~?
\end{itemize}

\subsection{05 mai}
\begin{itemize}
  \item Voir Jonas pour vérifier matériaux (compatibilité avec imprimante).
  \item Plan~: à aborder~: différences phases de course (analyse mouvement),
    analyse conversion d'énergie. Focalisation sur modélisation lame comme
    ressort.
\end{itemize}

\subsection{12 mai}
\begin{itemize}
  \item Cotations trouvées~! Modélisation sur SW avec Jolivet.
  \item Matériaux~: nylon~? À acheter.
  \item Diapos à commencer.
\end{itemize}

\subsection{26 mai}
\begin{itemize}
  \item Pièce fabriquée~!!
  \item Comment caractériser la lame~?
  \item Présentation~: encore
\end{itemize}

À chercher~: TIPE ramener physiquement qqch~?

\subsection{02 juin}
\begin{itemize}
  \item Avancer les diapos ajd.
  \item Vérifier ressources pour mesures sur la lame
  \item Faire les mesures en mode perche
\end{itemize}

\section{Présentation}

\begin{enumerate}
  \item[2)] Intro chouette~!
  \item[3)] Bien mais petit. Éviter les «~:~» dans les titres. Espace avant
    «~:~». Génial contexte.
  \item[5)] Très progressif, super.
  \item[6)] Très bien~!
  \item[7-10)] Ok. «~L'Homme~» urgh.
  \item[11)] Ok.
  \item[14)] Frottements négligés~?! Le poids s'applique où~?
  \item[15)] Restitution avec un angle~: pourquoi~?
  \item[16-20)] Ok sur la loi de \textsc{Hooke}.
  \item[21)] TB modèlé.
  \item[22)] Super simulation. Ça représente quoi les couleurs~? Comment mesuré
    le module d'\textsc{Young} avec la simulation~?
  \item[24)] Seconde simulation avec direction différente, pas mal.
  \item[25)] Trop petit encore.
  \item[26)] Ccl. Propre.
\end{enumerate}

\paragraph*{Questions}
\begin{itemize}
  \item Coureur très penché, est-ce que ça a une influence sur la course~?
  \item Modélisation plus épaisse en bas, pourquoi~? Doit être adhérente, donc
    voilà.
  \item Force moyenne qu'on applique estimée à combien~? Pas que le poids.
  \item Angle de la lame et force de propulsion~? Normal.
  \item Négliger les frottements~? Bah non.
\end{itemize}

\chapter{Léo \textsc{Hassenforder}}
\label{ch:leo}
\section{Suivi}
\subsection{10 février}
\begin{itemize}
    \litem{Sujet}~: Amortisseur voiture tout terrain~; raytracing, miroir jeux
        vidéos.
    \litem{Pbatique}~: Différentes manières de faire des miroirs en JV~;
        logiciels de JV pour voir l'évolution.
\end{itemize}
    
\subsection{03 mars}
\begin{itemize}
    \litem{Sujet}~: volant de badminton
    \litem{Pbatique}~: \xmark
\end{itemize}

\subsection{10 mars}
\begin{itemize}
    \litem{Sujet}~: cyclisme, col avec pente moyenne~; supposant la pente
        constante.
    \litem{Pbatique}~: Meilleure tactique pour optimiser ses performances sur le
        col~? Bof.
\end{itemize}

\subsection{17 mars}
\begin{itemize}
    \litem{Sujet}~: Badminton~? Bonne idée, masse, friction, etc.
    \litem{Pbatique}~: à trouver mais des idées.
    \litem{Biblio}~: des sources à droite à gauche.
\end{itemize}

\subsection{24 mars}
\begin{itemize}
    \litem{Sujet}~: badminton à confirmer.
    \litem{Biblio}~: Fournie mais à traiter.
    \litem{Pbatique}~: volant
\end{itemize}

\subsection{31 mars}
\begin{center}
  Absent.
\end{center}

\subsection{07 avril}
\begin{itemize}
  \litem{Plan}~:
    \begin{enumerate}[label=\Roman*]
      \item Caractéristiques du volant~: balance, soufflerie du labo.
      \item Modélisation de la trajectoire~: ENS 2015.
      \item Mesure de la trajectoire pour 2 volants différents avec le même
        coup.
    \end{enumerate}
\end{itemize}

\subsection{14 avril}
\begin{itemize}
  \item Voir labo plume/volant.
  \item Faire un pointage pour comparer la trajectoire avec la théorie (ENS).
\end{itemize}

\subsection{05 mai}
\begin{itemize}
  \item Volants acquis. 
  \item Soufflerie~: docu mais pas ouf.
  \item Idée lancer vertical force constante~?
\end{itemize}

\subsection{12 mai}
\begin{itemize}
  \item Expérience soufflerie~: en cours
  \item Lanceur vertical~: moteur perceuse pas assez~: voir directement en
    club~?
\end{itemize}

\subsection{26 mai}
\begin{itemize}
  \item Refaire la vitesse avec autre volant, et avoir vitesse du vent.
  \item Club machine que en semaine
\end{itemize}

\subsection{02 juin}
\begin{itemize}
  \item Tentative de détermination de vitesse de changement
  \item Diapos~: ok ça ira
\end{itemize}

\section{Présentation}
\begin{itemize}
  \item Intro bof. «~Je vais faire une présentation~». No shit.
  \item LES DOIGTS PUNAISE.
  \item Super intro sur différence de vitesse
  \item «~gap~» en anglais.
  \item théorique d'abord~!!
  \item «~expérience~» pour mesure de masse…
  \item Assez lent dans son explication.
  \item $\SI{24\pm 0.1}{m.s^{-1}}$.
  \item Diapo 6~: euuuuuh Python SVP.
  \item Diapo 7~: estimation vient d'où~? C'est quoi les coefficients de
    frottements~?
  \item Théorique enfin~!
  \item Diapo 9~: force de frottements POSITIVE~!
  \item Diapo 10~: UNITÉ LAMBDA~? Postion manque un $i$.
  \item Diapo 11~: Tableau sur Excel mais OMG. Présentation graphique bof.
\end{itemize}

\subsection{Questions}
\label{ssec:q}
\begin{itemize}
  \item Turbulent équivaut à quadratique~?
  \item Prendre la surface de la fin de jupe c'est pertinent pour les
    frottements~?
  \item TB «~est-ce que ça a répondu à ta question~».
  \item Diapo 7~: comment t'as obtenu la courbe~?
\end{itemize}

\chapter{Élisa \textsc{Augier}}
\label{ch:elisa}
\section{Suivi}
\subsection{10 février}
\begin{itemize}
    \litem{Sujet}~: Balle de tennis de table~: rebond, effet, on verra.
\end{itemize}

\subsection{03 mars}
\begin{itemize}
    \litem{Sujet}~: Balle neuve/usée~; club ou ***~; trajectoire selon effet
    \litem{Problématique}~: Contrôle de la balle~: problématique à trouver
\end{itemize}

\subsection{10 mars}
\begin{itemize}
    \litem{Pbatique}~: pas avancée.
    \litem{Manip}~: commencée, test de vidéos pour voir distorsion. Modélisation
        \texttt{python} éventuelle~?
    \litem{Biblio}~: sources à relever. Étude du nombre de \textsc{Reynolds}
        PC/PSI.
\end{itemize}

\subsection{17 mars}
\begin{itemize}
    \litem{Pbatique}~: comment modéliser le mouvement de la balle selon l'effet~?
    \litem{Biblio}~: physique MP/MP$^*$, \textsc{Reynolds} et laminaire,
        \textsc{Magnus} etc. Super.
    \litem{Manip}~: déjà mesure pour parabole, $g = \SI{12}{m.s^{-1}}$. Ajouter
        incertitudes~? Effet lifté à venir.
\end{itemize}

\subsection{24 mars}
\begin{itemize}
    \item Mesure de $g$ par rebonds, super, on continue.
\end{itemize}

\subsection{31 mars}
\begin{itemize}
  \item Résultats sur $g$ cohérents.
  \item Attente des vacances pour mesures.
  \item Étude rebond et perte de vitesse de rotation/\textsc{Varignon}~?
\end{itemize}

\subsection{07 avril}
\begin{itemize}
  \item Pas d'air mais juste rebond et transfert de vitesse~?
\end{itemize}

\subsection{14 avril}
\begin{itemize}
  \item Robot club cassé~: autre club, pas de réponse. À relancer.
  \item Étude rebond vitesse rotation~: envoyer photo
  \item Trouver référence machine club~?
\end{itemize}

\subsection{05 mai}
\begin{itemize}
  \item Modélisations
  \item{Plan}~: super.
  \item Différentes mesures avant-après.
\end{itemize}

\subsection{12 mai}
\begin{itemize}
  \item Vitesse angulaire au carré négative~: aïe. Sinon ça avance très bien.
\end{itemize}

\subsection{26 mai}
\begin{itemize}
  \item Inclinaison marche bizarre
  \item Diapos commencées
  \item Idées sur souci de l'accélération~: condition de roulement glissement.
  \item Coefficient de restitution à venir.
\end{itemize}

\subsection{02 juin}
\begin{itemize}
  \item Diapos bien avancées !
\end{itemize}

\begin{center}
  \huge Regarder résultats et dépendance $h$ ou pas~!
\end{center}

\section{Présentation}

\begin{enumerate}
  \item Ok intro. Bien le pointeur, mais faut savoir trembler le moins possible.
  \item[3)] Ok.
  \item[5)] Ok.
  \item[7)] $6e^{-16}$~! Ah. Mise en forme des incertitudes.
  \item[9)] Décrire l'effet.
  \item[10)] Trop rapide.
  \item[11)] $\gg$ pas >>.
  \item[14)] Ok excellent.
  \item[16)] Enfin des théorèmes. Superbes équations.
  \item[20)] Excellente disjonction de cas.
  \item[24)] Conservation du moment cinétique~?!
  \item[26)] Ok chouette.
  \item[29)] Excellentissime.
  \item[30)] Ok chouette écart normalisé.
  \item[32)] Très bien extro.
\end{enumerate}

\paragraph*{Questions}
\begin{itemize}
  \item Diapo 9~: c'est quoi les deux cas ?
  \item Possible de faire rouler la balle sur la table après un rebond~? Superbe
    analyse diapo 26.
  \item Diapo 26~: Excellente démonstration.
  \item Diapo 24~: Conservation du moment cinétique~! Projeté sur $B$… Mouais.
\end{itemize}

\chapter{David \textsc{Chandellier} et Ylan \textsc{Malherbe}}
\label{ch:davidylan}
\section{Suivi}
\subsection{10 février}
\begin{itemize}
    \litem{Sujet}~: toupies \textsc{Beyblade}
    \litem{Pbatique}~: comment faire la meilleure…
    \litem{Manip}~: 
        \begin{itemize}
            \item Pointe en sphère~: étude frottements
            \item Hauteur
            \item Masse sur les bords
            \item Lanceur
            \item Étude de caractéristiques
            \item Effet accélération anti-horaire
        \end{itemize}
\end{itemize}

\subsection{03 mars}
\begin{itemize}
    \litem{Pbatique}~: Est-ce qu'il est possible de faire une toupie imbattable~?
    \litem{Biblio}~: commencée.
    \litem{Manip}~: vidéos, difficultés FPS
\end{itemize}

\subsection{10 mars}
\begin{itemize}
    \litem{Manip}~: mesure tours/minute ok, fabrication nouveau lanceur et quelle
        est la limite~: bras de levier et crans.
    \litem{Biblio}~: ok.
\end{itemize}

\subsection{17 mars}
\begin{itemize}
    \litem{Manip}~: en avance, connexions pour augmenter le lanceur.
    \litem{Biblio}~: exercice etc.
\end{itemize}

\subsection{24 mars}
\begin{itemize}
    \litem{Plan}~: ouais il s'est fait avec les manips, faut juste le mettre au
        clair.
    \litem{Étude théorique}~: bien avancés. Voir lanceur et éventuellement
        frottements.
\end{itemize}

\subsection{31 mars}
\begin{itemize}
  \item Étude théorique lancer ok, vitesse angulaire trouvée.
  \item Limitations sur les toupies \textsc{Beyblade}. Détermination des valeurs
    min et max de paramètres.
  \item Mesure dynamomètre force utilisée pour lancer.
  \item Documentation de l'approche, hypothèse (endurance meilleure que
    attaque).
\end{itemize}

\subsection{07 avril}
\begin{itemize}
  \item Problème de valeur estimation de vitesse de rotation~: regarder photo
    envoyée par courriel.
  \item Étude de la pointe~: cf courriel envoyé.
\end{itemize}

\subsection{14 avril}
\begin{itemize}
  \item Mesure masse de la toupie~: attention incertitude.
  \item Formule qui marche toujours pas.
  \item Étude pointe toupie.
\end{itemize}

\subsection{05 mai}
\begin{itemize}
  \item Calcul force frottement pointe sur aire disque.
\end{itemize}

\subsection{12 mai}
\begin{itemize}
  \item Mesure RPM dans le temps.
  \item Diapo commencées.
\end{itemize}

\subsection{26 mai}
\begin{itemize}
  \item Vitesse de moyenne de la toupie tracée sur Python~: super. Mesure seconde
    toupie plus plate~?
  \item Mesures effectuées. Regrouper les données pour tracer commun. La toupie
    plus plate dure plus longtemps.
  \item Présentation bien avancée~! LibreOffice ok. Master slides etc.
\end{itemize}

\subsection{02 juin}
\begin{itemize}
  \item Coefficient de fricton~: pas évident.
  \item Fit exponentiel TB.
  \item Essayer de relier à la théorie.
  \item Diapos~: super.
\end{itemize}

\begin{center}
  \large regarder cet exercice !!
\end{center}

\section{Présentation}

\subsection{David}
\begin{enumerate}
  \item Pas mal intro. \textbf{Trop de notes}.
  \item[2-5)] Ok objets
  \item[6)] Définir toupie d'attaque et d'endurance.
  \item [7)] Pas mal image, écrire hypothèses~!
  \item [9)] Tach\ul{y}mètre. Introduire les différentes parties qui sont
    optimisables.
  \item [14)] Ça va trop vite.
  \item [15)] Représenter les grandeurs~!! Ça vient d'où tout ça~?
  \item [16)] On est où dans les bails~? AXES trop petits.
  \item [18)] Unités.
  \item [20)] Ça va trop vite~!
  \item [21)] Résumer les infos. Ok sur analyse temps.
  \item [22)] Ok mais liste à points, majuscules.
\end{enumerate}

\subsection{Questions}
\begin{itemize}
  \item Ambdaloi tiens~! Homogénéité~?
  \item Pourquoi toupie 2 plus rapides que 1~?
  \item Différence attaque/endurance~? Attaque vitesse plus élevée donc
    déstabilise.
  \item Le $\alpha$ c'est quel angle~? «~C'est sur le site~»
\end{itemize}

\subsection{Ylan}
\begin{enumerate}
  \item TP au lieu de TIPE.
  \item Sympa l'intro.
  \item [5)] Super images et description.
  \item [9)] Bien sur la voix.
  \item [10)] Ça veut dire quoi imbattable~?
  \item [12)] Excellent description.
  \item [15)] Euh bonne idée malheureusement.
  \item [17)] Mesure de force ok
  \item [18)] Super explication du tachymètre.
  \item [19)] Votre toupie c'est quoi du coup~?
  \item [21)] Rapport 10~: incertitude sur $\alpha$, sur force~?
  \item [22)] Scripte…
  \item [24)] Ok hypothèses.
  \item [26)] Donc, mesure après veut dire que calcul pas bon~!
  \item [27)] Ok. Axes trop petits. Décrire les axes.
  \item[33)] Unités~! Rapport 10 sur le temps~?
  \item[35)] Ok ccl.
\end{enumerate}

\subsection{Questions}
\begin{itemize}
  \item Pas plus simple de mettre un moteur~? Possible…
  \item Modifier le lanceur~? Pourquoi pas un fil plutôt que des dents~? Idée.
  \item Pourquoi pas de mesure avant le choc~? Saladier trop petit, et pas de
    bords dans vrai stade.
  \item Quelles sont vos conclusions sur les matériaux~? À faire aussi, oui.
\end{itemize}

\chapter{Évanne \textsc{Le Mée}}
\label{ch:evanne}
\section{Suivi}
\subsection{10 février}
\begin{itemize}
  \litem{Sujet}~:
    \begin{itemize}
      \item Basket, étude des trois points~?
      \item Biathlon, prévoir résultat d'une course~? 4 tirs, nombre de
          faute rebat les cartes. Étude tir bal, farts et ski change
          l'avancée.
      \item Cyclisme, frottements positions interdites~?
    \end{itemize}
\end{itemize}

\subsection{03 mars}
\begin{itemize}
    \litem{Sujet}~: Cyclisme frottements.
    \litem{Pbatique}~: Le règlement de l'UCI (union cyclistes internationale) sur
        la position réglementaire des coureurs est-il justifié~? 
    \litem{Manip}~: polystyrène~?
\end{itemize}

\subsection{10 mars}
\begin{itemize}
    \litem{Pbatique}~: frottements machins, bof
    \litem{Biblio}~: ok, voir chaîne optimisation cyclisme
    \litem{Manip}~: à commencer~?
\end{itemize}

\subsection{17 mars}
\begin{itemize}
    \litem{Pbatique}~: tjs en évolution.
    \litem{Biblio}~: fournie
    \litem{Manip}~: commencée !
\end{itemize}

\subsection{24 mars}
\begin{center}
    Manipulations.
\end{center}

\subsection{31 mars}
\begin{itemize}
  \item Relevé en cours.
  \item Étude théorique de l'angle en fonction de la force du vent à faire.
\end{itemize}

\subsection{07 avril}
\begin{itemize}
  \item Calculs pour trouver angle d'équilibre en fonction du coefficient de
    frottement.
\end{itemize}

\subsection{14 avril}
\begin{itemize}
  \item Revoir résultat avec DS pour trouver $\tt_{\eq}$. Comparer avec
    expérience ensuite.
\end{itemize}

\subsection{05 mai}
\begin{itemize}
  \item Ok retour sur DS07~: mesure $\tt_{eq}$ à exploiter.
\end{itemize}

\subsection{12 mai}
\begin{itemize}
  \item 
\end{itemize}

\subsection{26 mai}
\begin{itemize}
  \item Plus de formes. Ficelles. Passer à la soufflerie~?
\end{itemize}

\subsection{02 juin}
\begin{itemize}
  \item Pas de pièce SolidWorks
  \item Diapos commencées, pas mal~!
\end{itemize}

% \section{Bruno \textsc{Idir}}
% \label{ch:bruno}
% \subsection{10 février}
% \begin{itemize}
%     \litem{Sujet}~: épées, exigences physiques dans la conception d'une épée,
%         notamment masse et équilibre. À problématiser
% \end{itemize}
% 
% \subsection{03 mars}
% \begin{itemize}
%     \litem{Pbatique}~: Énergétique et morphologie des épées~?
%     \litem{Biblio}~: en cours
% \end{itemize}
% 
% \subsection{10 mars}
% \begin{itemize}
%     \litem{Manip}~: à trouver en lien avec biblio… peu d'avancée.
% \end{itemize}
% 
% \subsection{17 mars}
% \begin{itemize}
%     \litem{Pbatique}~: Meilleure épée pour un duel~? Niveau de protection.
%     \litem{Sujet}~: escrime bout d'épée~? Bof.
% \end{itemize}
% 
% \subsection{24 mars}
% \begin{itemize}
%     \litem{Sujet}~: À continuer~: baseball pas concluant
% \end{itemize}
% 
% \subsection{31 mars}
% \begin{itemize}
%   \litem{Sujet}~: ricochet~?
%   \litem{Manip}~: à peu près irrélisable.
% \end{itemize}
% 
% \subsection{07 avril}
% \begin{itemize}
%   \item Abandon ricochets. Proposition d'avion en papier, ou travail en groupe
%     avec um autre étudiantx.
% \end{itemize}
% 
% \subsection{14 avril}
% \begin{center}
%   Travail avec Mathis.
% \end{center}

\section{Présentation}

\begin{itemize}
  \item Introduction pas mal.
  \item Numéro de diapos ok.
  \item I/ moins de 2 minutes, ok.
  \item Prolongateur pas défini~?
  \item Casques~: c'est quoi améliorer les performances~? Théorique d'abord~!
  \item Diapo 11 TB.
  \item Trop de détails diapo 17.
  \item Garder que le TMC.
  \item Diapo 30~: $\ang{16}\pm?$
  \item 34 
  \item Trop de tailles différentes dans toutes les diapos
  \item Conclusion amélioration \textbf{après} tout, non ?
\end{itemize}

\subsection{Questions}
\label{ssec:q}
\begin{itemize}
  \item Masses différentes~: c'est grave~?
  \item À l'aise avec les réponses.
\end{itemize}

\chapter{Amine \textsc{Eljaafari} et Ferdinand \textsc{Ledenko}}
\label{ch:amineferdinand}
\section{Suivi}
\subsection{10 février}
\begin{itemize}
    \litem{Sujet}~: boomerang
    \litem{Pbatique}~: récréatif ou de chasse~?
\end{itemize}

\subsection{03 mars}
\begin{itemize}
    \litem{Pbatique}~: Temps de vol en fonction des caractéristiques. Travailler
        la problématique
    \litem{Manip}~: à trouver
\end{itemize}

\subsection{10 mars}
\begin{itemize}
    \litem{Biblio}~: en japonais mais ok avec boomerang symétrique.
    \litem{Manip}~: essai de prise de vidéo pour pointer la trajectoire, mais
        quelle caractéristique testée~? Ok idée mesure masse volumique.
        Potentiellement boomerang impression 3D.
\end{itemize}

\subsection{17 mars}
\begin{center}
  En maths.
\end{center}

\subsection{24 mars}
\begin{center}
  En maths.
\end{center}

\subsection{31 mars}
\begin{center}
  Absence.
\end{center}

\subsection{07 avril}
\begin{itemize}
  \item Détermination du barycentre par le calcul. Prochaine étape~:
    vérification par rotation.
  \litem{Pbatique}~: comment expliquer les mouvements du boomerang, expliquer les
    paramètres et comment les optimiser~?
  \item Vitesse~? Pourquoi revient-il~? Influence de l'angle~?
\end{itemize}

\subsection{14 avril}
\begin{itemize}
  \item Jusqu'où peut aller un boomerang~?
  \litem{Plan}~:
    \begin{enumerate}[label=\Roman*]
      \item Trois temps classique.
    \end{enumerate}
  \item Vacances~: pointage.
  \item Trouver pourquoi ça revient.
\end{itemize}

\subsection{05 mai}
\begin{itemize}
  \item Présentation boomerang~: parler des équations.
  \item Plan~: TB
  \item Manip~: vidéo boomerang comparaison.
\end{itemize}

\subsection{12 mai}
\begin{itemize}
  \item Diapos commencées. Pas mal.
\end{itemize}

\subsection{26 mai}
\begin{itemize}
  \item Présentation~: squelette terminer, à remplir.
  \item Numéro de page à refaire.
  \item Enlever Remerciements.
\end{itemize}

\subsection{02 juin}
\begin{itemize}
  \item Bernouilli à relier au boomerang
  \item Diapos à fournir.
\end{itemize}

\section{Présentation}
\subsection{Amine}
\begin{itemize}
  \item Intro ok mais c'était la mienne mdr.
  \item Liste à puces + des nombres~?
  \item amazon.fr pas ouf.
  \item Titre de la diapo… trop gros.
  \item Diapos non finies.
  \item Précession non expliquée.
  \item Numéros diapos trop petits.
  \item Discussion barycentre et expérience sans support.
  \item Aucune expérience présentée.
  \item À peine \SI{07}{minutes}.
\end{itemize}

\subsection{Ferdinand}
\begin{itemize}
  \litem{Hyper} stressé.
  \item Problématique ok.
  \item À l'oral, catastrophique par contre.
  \item Pareil sur les titres trop grands.
  \item Aucune réelle explication diapo 9 sur l'effet de la traînée.
  \item Mort silencieuse.
\end{itemize}

\subsection{Questions}
\begin{itemize}
  \litem{Natiao}~: d'où la différence de pression~?
    $\lra$ différence de vitesse.
  \litem{Samy}~: d'autres exemples de portance~?
    $\lra$ voitures, dans l'autre sens.
  \litem{Bruno}~: pourquoi la traînée ralentit~? C'est laquelle qui ralentit~?
  \litem{Évanne}~: c'est quoi la coupe~?
  \litem{Séréna}~: j'ai pas eu. Utilité de réduire l'espace~?
  \litem{Maxime}~: NS et Bernoulli c'est quoi~?
    $\lra$ l'écrit au tableau.
  \litem{Bruno}~: espace restreint utile pour la chasse, origine forêt etc.
\end{itemize}

\chapter{Samy \textsc{Ugolini}}
\label{ch:samy}
\section{Suivi}
\subsection{10 février}
\begin{itemize}
    \litem{Sujet}~: pétanque
    \litem{Pbatique}~: Étude du carreau/gravier et possibilité de tir droit en
        fonction de la qualité du sol et du $\pf$/$\Ec_c$
\end{itemize}

\subsection{03 mars}
\begin{itemize}
    \litem{Sujet}~: service flottant volley.
    \litem{Pbatique}~: vitesse limite pour avoir oscillations~?
    \litem{Biblio}~: BU d'Orléans.
    \litem{Manip}~: à trouver.
\end{itemize}

\subsection{10 mars}
\begin{itemize}
    \litem{Manip}~: pointage 2 dimensions club Orléans potentiellement.
    \litem{Pbatique}~: quel service est le plus flottant~? Mouais.
\end{itemize}

\subsection{17 mars}
\begin{itemize}
    \litem{Manip}~: Recherche des règles, atterrissage de la balle, masse etc.
        Choses à mesurer, vitesse, etc.
    \litem{Biblio}~: ok mais la développer la théorie.
    \litem{Pbatique}~: service volant sur Mars~?
\end{itemize}

\subsection{24 mars}
\begin{itemize}
    \litem{Manip}~: pendant les vacances~: voir pour des trépieds. Vitesse dans
        le sens, vitesse/mouvement perpendiculaire, hauteur du lancer, distance
    \litem{Biblio}~: à continuer.
    \item Étude théorique à faire, mais en cours.
\end{itemize}

\subsection{31 mars}
\begin{itemize}
  \item En plein dans la biblio… risqué~?
  \item Il existe des pointages.
\end{itemize}

\subsection{07 avril}
\begin{itemize}
  \litem{Pbatique}~: comment attraper un service flottant efficacement~?
  \litem{Manip}~: tirer toujours au même endroit, voir la probabilité d'impact et
    en faire quelque chose~?
\end{itemize}

\subsection{14 avril}
\begin{itemize}
  \item Changement~? Ok maybe.
\end{itemize}

\subsection{05 mai}
\begin{itemize}
  \item Garde sujet
  \item Présentation commencée, super.
\end{itemize}

\subsection{12 mai}
\begin{itemize}
  \item Continuation présentation.
\end{itemize}

\subsection{26 mai}
\begin{itemize}
  \item Pointage en cours. Points d'arrivée en fonction de l'abscisse~:
    histogramme~?
  \item Ok suite.
\end{itemize}

\subsection{02 juin}
\begin{itemize}
  \item Script Python~: histogramme ok
  \item tentative de smooth non-gaussien~: non.
  \item Faire gaussienne~?
  \item Faire un idéogramme (cf. Mickaël)
  \item Faire des lancers sans effet pour comparer.
  \item Présentation~: plan super. Diapos en cours.
\end{itemize}

\section{Présentation}

Intro classique. Définition service simple mais clair. Diapos qui bougent. On
sur le style. Numéro de pages trop petits. Noir sur gris foncé~: très mauvaise
idée~!

Taille moyenne, pourrait être plus gros. Pas de buzette.

Ok sur les schémas, TB même. Progressif, bien.

Bien sur la problématique.

Laminaire turbulent cool. Référence nouveaux ballons chouette. Modélisation et
rayson d'étude TB.

13 ou 24 ? But = faire ça… bof, protocole = faire ça.
25 ? Pourquoi 16 mètres ?
Programme Python… bof bof quoi. Illisible.

Inverser le sens du protocole~! Présentation ne doit pas être chronologique.

TB sur la comparaison.

TB sur ordres de grandeurs.

\subsection{Questions}
\begin{itemize}
  \item Bruno~: c'est quoi une gaussienne~?
    $\lra$ description, courbe qu'on trouve souvent dans la nature (fission de
    l'uranium).
  \item Différence laminaire/turbulent~: très bien géré.
\end{itemize}

\chapter{Victor \textsc{Hatton}}
\label{ch:victor}
\section{Suivi}
\subsection{10 février}
\begin{center}
    Absent
\end{center}

\subsection{03 mars}
\begin{itemize}
    \litem{Sujet}~:
        \begin{itemize}
            \item Surfeur sur vagues superposées~?
            \item Matériau chaussures athlétisme course~: rebondissement, balle
                de ping-pong~?
        \end{itemize}
    \litem{Pbatique}~: à trouver
\end{itemize}

\subsection{10 mars}
\begin{itemize}
    \litem{Sujet}~: hauteur de saut.
    \litem{Biblio}~: thèse «~Évaluation des propriétés mécaniques de la
        cheville~», à décortiquer. Super.
    \litem{Manip}~: pointage film éventuellement. Voir pourquoi le saut en
        hauteur aux JO se fait en courant.
\end{itemize}

\subsection{17 mars}
\begin{itemize}
    \item Plan~: élasticité musculaire, stockage musculaire et processus
        énergétique.
    \item Dick Fosbury intro
    \litem{Pbatique}~: trouver une problématique
    \litem{Biblio}~: bien mais trop aidante pour une manip.
    \litem{Manip}~: 
\end{itemize}

\subsection{24 mars}
\begin{itemize}
    \litem{Pbatique}~: optimiser sa hauteur de saut.
    \litem{Plan}~:
        \begin{enumerate}[label=\Roman*]
            \item Hypothèses, principe, énergie cinétique etc. Bilan des forces
            \item Modélisation ressort, optimisation, mesures
            \item Exploitation.
        \end{enumerate}
    \litem{Manip}~: modèle à perfectionner, et modèle réduit de jambe. Modèle 2
        tiges rigides + 4 ressorts~?
    \litem{Biblio}~: mieux exploiter, trouver un schéma réaliste.
\end{itemize}

\subsection{31 mars}
\begin{itemize}
  \item Continuer à chercher des ressources sur modélisation du saut.
  \item Article trouvé sur Research Gate! Très bien. En français, modélisation
    foulée.
\end{itemize}

\subsection{07 avril}
\begin{center}
  Absent.
\end{center}

\subsection{14 avril}
\begin{itemize}
  \item Voir labo ressort dur.
  \item Voir théorie rebond ressort~? Exercice transmis.
\end{itemize}

\subsection{05 mai}
\begin{itemize}
  \item Expérience~: masse, ressort.
  \item Caractéristiques physiques d'um sautaire.
  \item Voir labo ressort résistant.
  \item Équation sur Kiné Sport -> à comprendre.
\end{itemize}

\subsection{12 mai}
\begin{itemize}
  \item Pas de ressort entre deux. Essayer de faire une mesure de constante de
    raideur~: LatisPro et Python~?
\end{itemize}

\subsection{26 mai}
\begin{itemize}
  \item Expérience ressort rebond vers le haut~: non.
  \item Expérience élongation ressort tout court~: oui.
  \item Présentation pas commencée.
\end{itemize}

\subsection{02 juin}
\begin{itemize}
  \item Manip linéaire ressort ok
  \item Faire photo
  \item Faire reg lin
  \item Présentation pas commencée~!
\end{itemize}

\section{Présentation}
\begin{itemize}
  \item Ne pas dire pourquoi vous avez choisi ce sujet~!
  \item Problématique… ouais.
  \item Histoire du saut en hauteur trop long.
  \item NUMÉRO DE DIAPO~? 5
  \item Diapo 6~: liste à points décalée…
  \item Bien sur les photos, mettre des indications dessus.
  \item Diapo 9~: schéma~? Homogénéité~? C'est dégueulasse.
  \item Diapo 10~: screen Python faites un effort… on s'en contrefout des
    données, on veut un GRAPHE.
  \item Ccl ouais bon ok…
\end{itemize}

\subsection{Questions}
\begin{itemize}
  \litem{Séréna}~: que le poids sur le système~?
    $\lra$ même l'équilibre du ressort c'est pas clair. Rappel du ressort = une
    tension…
  \litem{Léo}~: c'est quoi le blocage sur le sujet~? Pourquoi pas boule~?
    $\lra$ verticalité
  \litem{Samy}~: ça serait quoi l'expérience~?
  \litem{Mathis}~: pourquoi le Fosbury~? $\lra$ moins de frottements à l'air…
  AH.
  \litem{Léo}~: saut en hauteur
\end{itemize}

\chapter{Natiao \textsc{Dommanget}}
\label{ch:natiao}
\section{Suivi}
\subsection{10 février}
\begin{itemize}
    \litem{Sujet}~:
        \begin{itemize}
            \item Cyclisme, optimisation poids/résistance~; positions
                (voir~\nameref{ch:evanne})~?
            \item Voile pour l'année prochaine
        \end{itemize}
\end{itemize}

\subsection{03 mars}
\begin{itemize}
    \litem{Sujet}~: Roue de vélo. Caractéristiques de la roue~: qu'est-ce qui est
        modifiable~? Poids, rayons apparents ou en disques, lubrifier la
        liaison.
    \litem{Pbatique}~: futur de la roue de vélo~?
    \litem{Biblio}~: un peu.
\end{itemize}

\subsection{10 mars}
\begin{itemize}
    \litem{Manip}~: lancé roue et pointage LatisPro~: voir comment elle ralentit.
    \litem{Biblio}~: énergie cinétique roue et influence sur énergie cinétique
        vélo~: assez avancé sur ce point-là.
    \litem{Pbatique}~: à trouver~!
\end{itemize}

\subsection{17 mars}
\begin{center}
    Plan d'action pour les manips.
\end{center}

\subsection{24 mars}
\begin{center}
    \item Manips~!
\end{center}

\subsection{31 mars}
\begin{itemize}
  \item Vidéo avec et sans pneu, mesure moment d'inertie~?
  \item Transfert exercice roue de vélo pour traitement.
\end{itemize}

\subsection{07 avril}
\begin{itemize}
  \item Étude théorique~: coefficient de frottement, accélération pour freinage
    sce (total) et recherche pour freinage continu.
\end{itemize}

\subsection{14 avril}
\begin{itemize}
  \item Continue théorie.
\end{itemize}

\subsection{05 mai}
\begin{itemize}
  \item Temps de redressement d'un bateau.
  \item{Biblio}~: ok
  \item{Pbatique}~: temps perdu en régate à cause d'un retournement.
  \item{Théorie}~: pas mal avancé.
  \item{Manip}~: bateau en polystyrène, poids pour la coque.
  \item{Plan}~: en cours
\end{itemize}

\subsection{12 mai}
\begin{itemize}
  \item Mise au point de l'équation différentielle~: rajouter frottement fluide
    eau.
  \item Mise au point maquette ensuite~?
\end{itemize}

\subsection{26 mai}
\begin{itemize}
  \item Pas avancé, voir fin d'heure~?
  \item Présentation commencée
  \item Maquette
  \item Vidéo
\end{itemize}

\subsection{02 juin}
\begin{itemize}
  \item Maquette réalisée
  \item Vidéo faite, comparer avec la résolution de l'équation différentielle
  \item Application système réel
  \item Ensuite diapos
\end{itemize}

\section{Présentation}

\begin{enumerate}
  \item Pas vu
  \item Ok. Diapos jolies. Numéro de page bien. Trop sur ses notes~!
  \item C'est quoi les forces~? On représente quoi ici~? Pourquoi y'a un cœur~?
    Y'a pas de titre de partie, on est où dans le récit~?
  \item Bleu sur noir~: on voit rien. $\sin{\theta} \approx \tt$ vraiment~?
  \item On voit rien. Trop petit à gauche. Moches équations.
  \item Pareil pour équations, trop petit trop moche. $K_{\rm eau}$, pas $Keau$
  \item Les accents ça existe. Attention aux $l$ vs $\ell $.
  \item Ok.
  \item C'est quoi cette maquette~?
  \item Geogebra ok, mais AXES~? Expliquer. Elles sont où les valeurs
    expérimentales~?
  \item Euuh ouais beaucoup trop d'oscillations non~? Bien $\pi/8$. Bien
    solutions pour éviter balancements.
\end{enumerate}

\subsection{Questions}
\begin{itemize}
  \item C'est quoi diapo 3~?
  \item C'est quoi la différence entre | \num{1.17} et | \num{1.98}~? $\Ra $ ne
    pas mettre ce qu'on ne commente pas~!
  \item Eau sur voile~? Bof, on prend $\ang{90}$.
  \item Diapo 10~: ok pas beaucoup plus convaincant.
  \item Diapo 4~: poids du mat~?
  \item Différence centre de gravité et flottaison~? CdG = là où tout le poids
    s'applique. CdF~: là où la… poussée d'\textsc{Archimède} s'applique~?
\end{itemize}

\chapter{Eliott \textsc{Dautry}}
\label{ch:eliott}
\section{Suivi}
\subsection{10 février}
\begin{itemize}
    \litem{Sujet}~: Profil aérodynamique idéal pour une voiture.
\end{itemize}

\subsection{03 mars}
\begin{itemize}
    \litem{Sujet}~: adhésion pneu route.
    \litem{Pbatique}~: aucune.
    \litem{Biblio}~: à fournir.
\end{itemize}

\subsection{10 mars}
\begin{itemize}
    \litem{Biblio}~: quelques documents.
    \litem{Pbatique}~: différence pneus hiver/été~? Bof.
    \litem{Manip}~: force, distance de freinage, LEGO~? Ou aquaplanning~?
\end{itemize}

\subsection{17 mars}
\begin{itemize}
    \litem{Sujet}~: aquaplanning
    \litem{Biblio}~: ça avance, mais pas assez scientifique
    \litem{Manip}~: Peut-être rotation voiture dans bac avec eau.
\end{itemize}

\subsection{24 mars}
\begin{itemize}
    \litem{Biblio}~: 2 thèses en français, 1 sur les pneus/voiture dynamique (410
        pages) et un sur les films d'eau sur les pneus (170)~: à décortiquer et
        manip à trouver.
\end{itemize}

\subsection{31 mars}
\begin{center}
  Absent.
\end{center}

\subsection{07 avril}
\begin{itemize}
  \item Perte de motiv', tentative d'exploitation de la thèse~: trouver comment
    mesurer $F_x$ pour un plan incliné avec de l'eau~?
\end{itemize}

\subsection{14 avril}
\begin{itemize}
  \litem{Plan}~:
    \begin{enumerate}[label=\Roman*]
      \litem{Introduction}~: pbatique
      \litem{Expérience}~: hypothèse et description
      \litem{Résultats}~: conclusion.
    \end{enumerate}
  \item Mesure coefficient frottements à plat~?
\end{itemize}

\subsection{05 mai}
\begin{itemize}
  \item Travail sur air plutôt que eau~?
  \item 
\end{itemize}

\subsection{12 mai}
\begin{itemize}
  \item Frottement mouvement voiture~: quelle inclinaison nécessite le frein à
    main~?
  \item Diapos commencées.
\end{itemize}

\subsection{26 mai}
\begin{itemize}
  \item Sujet~: arrêt voiture lancée sur une pente fixe.
  \item Présentation commencée, pas mal.
\end{itemize}

\subsection{02 juin}
\begin{itemize}
  \item Avancée diapos. Simu de Em.
\end{itemize}

\section{Présentation}

\begin{itemize}
  \item Ok numéros de diapos.
  \item Pas mal contexte.
  \item Expérience n'est pas modélisation~!
  \item Schéma un peu dégueu.
  \item Est-ce qu'on peut vraiment prendre un modèle miniature~? Les frottements
    sont-ils scalables~?
  \item Diapo 5~: pas de frottement du sol~?!
  \item Diapo 6~: un point qui est flottant.
  \item Énergie potentielle de position………
  \item Angle de la pente~: ouais ok.
  \item Aurait pu avoir une simulation numérique.
\end{itemize}

\subsection{Questions}
\label{ssec:label}
\begin{itemize}
  \item sur les routes on donne pas des ° mais des pourcents. c'est quoi~?
  \litem{Samy}~: moteur à l'arrêt mais avance~?
  \litem{Élisa}~: c'est quoi $h$~? Pas clair.
  \litem{Ambdaloi}~: tu l'arrêtes quand le sol redevient plat non…~?
\end{itemize}

\chapter{Ambdaloi \textsc{MOHAMED ELHAD}}
\label{ch:ambdaloi}
\section{Suivi}
\subsection{10 février}
\begin{itemize}
    \litem{Sujet}~: saut à l'élastique, mouvement ressort force rappel élastique.
\end{itemize}

\subsection{03 mars}

\begin{itemize}
    \litem{Pbatique}~: limite à la hauteur du saut~?
    \litem{Biblio}~: aucune.
\end{itemize}

\begin{center}
    Très faible avancée.
\end{center}

\subsection{10 mars}
\begin{center}
    Absent.
\end{center}

\subsection{17 mars}
\begin{itemize}
    \litem{Biblio}~: forces, types de saut, sécurité, endroits… blogs et articles
        internet.
    \litem{Manip}~: type de matériau à exploiter~? Élastique ou ressort. Longueur
        à vide en fonction de la hauteur du saut… Modélisation~: masse de 5 à 10
        g. Puis étude énergétique.
    \litem{Biblio}~: comparaison autres sauts.
\end{itemize}

\subsection{24 mars}
\begin{center}
    Absent.
\end{center}

\subsection{31 mars}
\begin{center}
  Absent.
\end{center}

\subsection{07 avril}
\begin{center}
    Absent.
\end{center}

\subsection{14 avril}
\begin{itemize}
  \item Rédaction protocole
  \item Module d'Young pour élastique
  \litem{Pbatique}~: non-rupture de l'élastique~: masse max ou hauteur max.
  \item Voir labo matériel disponible.
\end{itemize}

\subsection{05 mai}
\begin{itemize}
  \item Allongement élastique avec module d'Young.
  \item Pas tarder sur les manips.
\end{itemize}

\subsection{12 mai}
\begin{itemize}
  \item 
\end{itemize}

\subsection{26 mai}
\begin{itemize}
  \item 
\end{itemize}

\subsection{02 juin}
\begin{itemize}
  \item 
\end{itemize}

\section{Présentation}

\begin{itemize}
  \item Pas mal diapo. Moche le rouge.
  \item Contexte ok. Élasticité et limite.
  \item Dynamique des fluides~?! Ridicule et rien dit.
  \item Module d'\textsc{Young} pas du tout détaillé avec image… Les parenthèses
    sont ??? diapo 4.
  \item Démarche expérimentale~: ressort. «~Beaucoup d'imprécision~» mais pas de
    résultats.
  \item Conclusion ridicule.
\end{itemize}

\subsection{Questions}
\begin{itemize}
  \litem{Natiao}~: fluide~?!
    $\lra$ tout est fluide. Mais pas ça.
  \litem{Samy}~: les forces c'est quoi~?
  \litem{Élisa}~: c'était quoi les galères~?
\end{itemize}

\chapter{Marie \textsc{Di Giovanni}}
\label{ch:marie}
\section{Suivi}
\subsection{10 février}
\begin{itemize}
    \litem{Sujet}~:
        \begin{itemize}
            \item Fronde Voyageur et billes magnétiques~? Retrouver simulation
                champ potentiel mouvement bille chargée.
            \item Fond diffus cosmologique~:~?
        \end{itemize}
\end{itemize}

\subsection{03 mars}
\begin{itemize}
    \litem{Sujet}~: Escrime~? Étude du ressort de la tête de pointe~:
        déclenchement.
    \litem{Pbatique}~: aucune
    \litem{Biblio}~: un peu
    \litem{Manip}~: difficile
\end{itemize}

\subsection{10 mars}
\begin{itemize}
    \litem{Sujet}~: envoi marteau JO par Thor
    \litem{Pbatique}~: rattraper sonde Voyageur~?
    \item Extraction attraction gravitationnelle~?
    \litem{Manip}~: visite labo aérospatial~?
\end{itemize}

\subsection{17 mars}
\begin{itemize}
    \litem{Sujet}~: Vendée Globes, constitution des voiliers pour course. Tests
        pour passer le règlement.
    \litem{Biblio}~: multiple, hydraulique, vagues, portance, etc.
    \litem{Manip}~: des TPs à reproduire, un modèle miniature à faire… ?
\end{itemize}

\subsection{24 mars}
\begin{itemize}
    \litem{Plan}~:
        \begin{enumerate}[label=\Roman*]
            \item Vendée Globes présentation
            \item Participer à la course~: jauge IMOCA~; test à \ang{90}, à
                \ang{180}.
        \end{enumerate}
    \litem{Biblio}~: fournie, à détailler.
    \item Manip à trouver. Foils~?
\end{itemize}

\subsection{31 mars}
\begin{itemize}
  \item Bilan des forces sur un voilier~: couples qui s'exercent sur un bateau,
    chavirage etc. pour comprendre l'expérience~; centre de gravité, centre
    d'application des forces, particularité des voiliers du Vendée Globes~:
    foils, intérêt physique.
  \litem{Pbatique}~: bateau insubmersible~?
  \item Intérêt des foils~?
\end{itemize}

\subsection{07 avril}
\begin{center}
    Absente.
\end{center}

\subsection{14 avril}
\begin{itemize}
  \item Prometteur~: moments, couple de redressement.
  \item Sinon faire théorie elle-même.
\end{itemize}

\subsection{05 mai}
\begin{itemize}
  \item Équation couples de redressement.
  \item Biblio~: très bon document. Limite d'angle de gîte.
  \item Calcul du document appliqué aux bateaux.
  \item Plan~: intro ça ira. Ça continue.
\end{itemize}

\subsection{12 mai}
\begin{itemize}
  \item Centre de gravité bateau triangle.
  \item Expérience angle de gîte~: poids sur les côtés.
  \item Super logiciel \texttt{NavalDesigner} clés en main.
\end{itemize}

\subsection{26 mai}
\begin{itemize}
  \item Estimation hauteur de la coque pour centre de gravité.
  \item Mesure coque Neuville tout ça.
  \item Présentation en tête.
\end{itemize}

\subsection{02 juin}
\begin{itemize}
  \item Contact la skipper pour prendre les mesures
  \item Mesurer sur les photos
  \item Simuler bateau en pendule
\end{itemize}

\section{Présentation}

\begin{enumerate}
  \item[1)] Faible intro…
  \item[2)] Belles diapos. Voix bof.
  \item[3)] D'acc' contexte
  \item[4)] Ok tests
  \item[5)] Problématique chouette. «~Euh voilà~» bof.
  \item[6)] Ok.
  \item[7)] Pas de frottement de l'eau, ah quand même.
  \item[8)] Ok c'est clair. $H_g$ plutôt que $Hg$.
  \item[9)] Super description. LES DOIGTS PUNAISE.
  \item[10)] Ok hypothèses. Assez complexe mine de rien
  \item[11)] Ok. Le poids est si bas~?
  \item[12)] Ok, axes un peu petits.
  \item[13)] Obtient pas la même chose~: pourquoi~? «~Ma hauteur~». Ok explique
    un issue possible.
  \item[14)] D'acc' conclusion, et même perspectives année prochaine.
\end{enumerate}

\paragraph*{Questions}
\begin{itemize}
  \item IMOCA c'est quoi~? Norme de bateau.
  \item Différences IMOCA et d'autres normes~? Ok ok.
  \item Le poids diapo 11~? Ok pas réaliste.
  \item La quille~? Ok plus tard.
  \item Frottements de l'eau~? Résistans, à prendre en compte.
\end{itemize}

\chapter{Gaspard \textsc{Pichon-Legroux}}
\label{ch:gaspard}
\section{Suivi}
\subsection{10 février}
\begin{itemize}
    \litem{Sujet}~: Simulation de foule avec mécanique des fluides~: stade de
        foot
\end{itemize}

\subsection{03 mars}
\begin{itemize}
    \litem{Sujet}~: Foule départ marathon
    \litem{Pbatique}~: taille de la route~?
    \litem{Manip}~: fluide/sable dans tube
\end{itemize}

\subsection{10 mars}
\begin{itemize}
    \litem{Manip}~: tubes avec différents solides. Voir si taille max pour
        écoulement correct.
    \litem{Biblio}~: quelques documents. Chute en milieu granulaire.
    \litem{Pbatique}~: à continuer.
\end{itemize}

\subsection{17 mars}
\begin{itemize}
    \litem{Manip}~: ça avance
    \litem{Pbatique}~: quelle taille de route pour éviter une catastrophe comme «
        insérer date ».
    \litem{Biblio}~: en cours.
\end{itemize}

\subsection{24 mars}
\begin{center}
    Absent.
\end{center}

\subsection{31 mars}
\begin{itemize}
  \litem{Question fil rouge}~: comment les paramètres de départ influencent les
    accidents~?
  \litem{Plan}~:
    \begin{enumerate}[label=\Roman*]
      \item Modèle fluide
      \item Modèle granulaire
      \item comparaison des résultats
    \end{enumerate}
  \litem{Biblio}~: avancement sur écoulements, trouver cas spécifique de
    rétrécissement. Granulaire docu trouvée mais pas avancée.
\end{itemize}

\subsection{07 avril}
\begin{itemize}
  \item Let's go manip bientôt~?
\end{itemize}

\subsection{14 avril}
\begin{itemize}
  \litem{Biblio}~: super, manip univ Le Mans simulation numérique~: super
    ressource tout clé en main.
  \item Demander au labo.
\end{itemize}

\subsection{05 mai}
\begin{itemize}
  \item Sable et cailloux récupérés. Tube en carton étanchéifié avec du scotch/
    Entonnoir avec gobelets en plastique.
  \item Expériences ce weekend et mesures avec régression linéaire.
\end{itemize}

\subsection{12 mai}
\begin{itemize}
  \item Essai vidéo avec balle de tennis~: traitement en cours.
\end{itemize}

\subsection{26 mai}
\begin{itemize}
  \item (Fouloscopie regardée)
  \item Diapos commencées
  \item Reprendre vidéos, prendre photos expérience.
\end{itemize}

\subsection{02 juin}
\begin{itemize}
  \item Diapos avancées
  \item Problème LatisPro
  \item latex2png TB
\end{itemize}

\section{Présentation}

\begin{enumerate}
  \item Ok. Sobre.
  \item TB.
  \item Ok. Bizarre d'avoir le template qui change.
  \item Excellent hypothèses.
  \item TB théorique.
  \item TB
  \item TB.
  \item TB.
  \item Axes un peu plus grands~? C'est quoi $y$~?
  \item Ok.
  \item TB.
  \item TB.
  \item Diapos passées trop vites.
\end{enumerate}

\begin{itemize}
  \item Diapo 9~: ok. Deverloo pas tracé, dommage.
  \item Diapo 13~: inutile TB.
  \item Départ différé du coup~?
  \item Diamètre de route~?
\end{itemize}

\chapter{Axel \textsc{Bernuchon}}
\label{ch:axel}
\section{Suivi}
\subsection{10 février}
\begin{itemize}
    \litem{Sujet}~: saut à la perche
\end{itemize}

\subsection{03 mars}
\begin{itemize}
    \litem{Pbatique}~: Limite taille perche etc.
    \litem{Biblio}~: en cours, thèse trouvée~: à décortiquer.
\end{itemize}

\subsection{10 mars}
\begin{itemize}
    \litem{Manip}~: Plein de possibilités. Script \texttt{Python} échange
        énergétique.
    \litem{Pbatique}~: en construction.
\end{itemize}

\subsection{17 mars}
\begin{itemize}
    \litem{Biblio}~: tests flexibilité, perches plus récentes + rigides au milieu
        moins au bout. Trouver le lien entre déformation et caractéristique de
        flexion.
    \litem{Manip}~: des idées,
\end{itemize}

\subsection{24 mars}
\begin{itemize}
    \litem{Plan}~:
        \begin{enumerate}[label=\Roman*]
            \item Énergie théorique et expérimentale avec Python
            \item Rigidité (énergétique aussi)~: constante de rigidité~?
        \end{enumerate}
    \item Afficher l'énergie mécanique sur Python, et l'énergie avant la pose de
        la perche.
    \item Retrouver (?) critère sélection perche.
    \item Trouver la théorie de la rigidité.
\end{itemize}

\subsection{31 mars}
\begin{itemize}
  \litem{Biblio}~: trouvé coefficient de rigidité de flexion, $EI$ avec $E$
    module d'\textsc{Young}.
  \item Convertir Python en équation.
  \item Commencer à voir les manips
\end{itemize}

\subsection{07 avril}
\begin{itemize}
  \item Commencer manip~? Continuer de traiter le Python.
\end{itemize}

\subsection{14 avril}
\begin{itemize}
  \item Livre du code Python, bien expliqué.
  \item Essais avec Python~: à faire pendant les vacances.
  \item Après les vacances~: mesures physiques.
\end{itemize}

\subsection{05 mai}
\begin{itemize}
  \item Difficultés Python.
  \item Mesure rigidité d'une règle. Pour une perche~: 1400-1600~? Règle~:
    \num{6e-3}. Comparer avec valeurs tabulées. Faire une régression linéaire.
    Essayer avec d'autres matériaux.
\end{itemize}

\subsection{12 mai}
\begin{itemize}
  \item Mesure résistance flexion manche à balais~: 4 mesures, difficile parce
    que très rigide (voir avec incertitudes~?).
  \item Exploitation Python~: avancées sur l'exploitation.
  \item Commencer à penser la présentation.
\end{itemize}

\subsection{26 mai}
\begin{itemize}
  \item $d$ en fonction de $m$, et Monte-Carlo sur propagation des incertitudes.
  \item Présentation~: à continuer.
\end{itemize}

\subsection{02 juin}
\begin{itemize}
  \item Avancées sur les diapos, test beamer.
\end{itemize}

La physique de sup en application avec python

\section{Présentation}

\begin{enumerate}
  \item[1)] Investi de ouf dans l'intro~! Faut tenir plus que 2 secondes.
  \item[2)] Génial sur la voix. Zapette pour passer les diapos. Incroyables
    diapos beamer.
  \item[3)] Ok intro. Transition problématique très bien, mais commenter le
    graphique.
  \item[5)] Tb explication.
  \item[6)] Ok
  \item[8)] Ok
  \item[9)] $L-x$ c'est quoi~?
  \item[10)] Très bien.
  \item[11)] Très bien.
  \item[12)] Très bien.
  \item[13)] Axes plus grands~! Super pointage réel, vient d'où~?
  \item[14)] Super représentation de l'impact de la rigidité.
  \item[15)] Attention \si{N.m^2}. Ok sur le reste.
  \item[16)] TB
  \item[18)] C'est quoi l'outil de mesure~?
  \item[19)] Excel mdr. Très bien Python ensuite.
  \item[21)] Excellent résultat.
  \item[22)] Ok ccl. Reprendre diapo initiale.
\end{enumerate}

\paragraph*{Questions}
\begin{itemize}
  \item Diapo 14~: si $EI$ trop élevé, en étant plus proche de la barre ça passe
    non~?
  \item Diapo 13~: source~?
  \item Ça veut dire quoi que \SI{16}{kg} convient pour le manche à balais.
\end{itemize}

\chapter{Benjamin \textsc{Petitgas}}
\label{ch:benjamin}
\section{Suivi}
\subsection{10 février}
\begin{itemize}
    \litem{Sujet}~: F1 ou autre sujet~?
\end{itemize}

\subsection{03 mars}
\begin{itemize}
    \litem{Sujet}~: tennis et cordage
    \litem{Pbatique}~: vitesse pour casser le cordage~? Vibrations amorties.
\end{itemize}

\subsection{10 mars}
\begin{itemize}
    \litem{Biblio}~: Avancée sur types de cordages, matériaux, ok. Différents
        types de matériaux ok. Calcul de la vitesse moyenne d'une balle et la
        force sur la raquette.
    \litem{Manip}~: Extension fil avec masse~; chute balle sur raquette
        spaghettis.
\end{itemize}

\subsection{17 mars}
\begin{itemize}
    \litem{Biblio}~: tous les cordages différents qui existent. Les plus
        résistants, polyester pour les plus costauds. Le plus fragile c'est le
        multi-filaments. Mono-filament~: mélange de deux
    \litem{Pbatique}~: meilleur matériau~?
    \litem{Manip}~: attention centrer sur le cordage. Cerceau avec plein de
        ressorts de $k$ différents~: étude énergétique de la balle avant-après
\end{itemize}

\subsection{24 mars}
\begin{itemize}
    \item Recherche maximale recensée sur terrain~: \SI{73}{m.s^{-1}} =
        \SI{263}{km.h^{-1}}. Comment trouver la limite~?
    \litem{Plan}~: en cours.
    \litem{Manip}~: on pourra faire varier la masse de la balle.
    \item Prendre en compte le fait que nous on ne déplace pas la raquette
        receveuse.
\end{itemize}

\subsection{31 mars}
\begin{itemize}
  \litem{Biblio}~: balle illustrant la déformation de la balle avec le cordage.
  \item Caractérisation résistance max cordage.
\end{itemize}

\subsection{07 avril}
\begin{itemize}
  \item Détermination tension raquette~: lien entre hauteur de rebond et énergie
    transmise selon la tension.
  \item Mesures ce weekend~?
\end{itemize}

\subsection{14 avril}
\begin{itemize}
  \item Pas mesures, balles vides.
  \item Tension entre \SIrange{20}{25}{N}. Ancienne raquette pour expériences.
  \item Trépied et iPhone 12 pour captation.
\end{itemize}

\subsection{05 mai}
\begin{itemize}
  \item Captations vidéos~: raquette au sol, balle à hauteur d'épaule.
  \item Tension à 18 sur la raquette de tennis -> pas de différence majeure avec
    20. Essayer avec une raquette de badminton~?
\end{itemize}

\subsection{12 mai}
\begin{itemize}
  \item Relevés vidéo rebond raquettes.
\end{itemize}

\subsection{26 mai}
\begin{itemize}
  \item Comparer les deux vidéos. Refaire plein de lancers, estimer incertitudes
    et faire la moyenne.
  \item Présentation~: plan mais pas diapos.
\end{itemize}

\subsection{02 juin}
\begin{itemize}
  \item 4 vidéos pour chaque tension pour incertitudes~: nice.
  \item Trouver Epp
  \item Trouver énergie captée.
  \item Présentation~: 
\end{itemize}

\section{Présentation}

\begin{itemize}
  \litem{Diapo 1}~: c'est chaud. Très moche. Espace avant «~:~». Numéros de
  diapos en haut à droite~: bizarre.
  \litem{Diapo 2}~: Liste à points avec des underscores~?!
  \litem{Diapo 3}~: «~:~» partout.
  \litem{Diapo 5}~: Faute. Mais sinon ok sur les hypothèses.
  \litem{Diapo 6}~: à l'horizontal~?
  \litem{Diapo 7}~: maths moches. Unité $\Delta{t}$~? Espace parenthèses.
  \litem{Diapo 8}~: Espace en trop. Ok vitesse max, mais d'autres exemples~?
  \litem{Diapo 9}~: Rester cohérent sur les typos.
  \litem{Diapo 10}~: tension en \si{kg}~?
  \litem{Diapo 13}~: LES AXES. COMMENTER.
  \litem{Diapo 14}~: Polices d'écritures. Tableau plus succinct~: par réécrire
  $\Ec_p$.
  \litem{Diapo 15}~: «~et donc voilà~». UNITÉ SUR $M$. C'est quoi $M$~?
  \litem{Diapo 16}~: et vous auriez pu le faire avec une raquette de badminton~?
  «~Donc euuuuh~». Calculer vraie valeur théorique $\Ra $ oui~!
\end{itemize}

\begin{itemize}
  \item Tension ou masse~? Bof réponse.
  \item Pourquoi pas raquette de badminton~?
  \item Ça servait à quoi de mesurer la différence d'énergie potentielle~?
  \item Ça se casse vraiment une raquette~? $\Ra $ que s'il les tapent lol.
    Sinon à cause du temps, oui, ça arrive.
\end{itemize}

\chapter{Walid \textsc{Belhamid}}
\label{ch:walid}
\section{Suivi}
\subsection{10 février}
\begin{itemize}
    \litem{Sujet}~: Décrochage avion sport figure aérienne
\end{itemize}

\subsection{03 mars}
\begin{itemize}
    \litem{Pbatique}~: ?
    \litem{Biblio}~: en cours
    \litem{Plan}~:
        \begin{enumerate}[label=\protect\fbox{\Roman*}]
            \item Présentation principe du décrochage
            \item Étude mécanique et lien.
        \end{enumerate}
\end{itemize}

\subsection{10 mars}
\begin{itemize}
    \litem{Biblio}~: presque fini~!
    \litem{Manip}~: bah faire des vrilles. Étude vitesse décrochage en fonction
        de la densité de l'air et de la masse de l'avion.
    \litem{Pbatique}~: en cours.
\end{itemize}

\subsection{17 mars}
\begin{itemize}
    \litem{Manip}~: à venir bientôt
    \litem{Plan}~: en progression
\end{itemize}

\subsection{24 mars}
\begin{itemize}
    \item Étude théorique en cours~: BDF, etc.
\end{itemize}

\subsection{31 mars}
\begin{itemize}
  \litem{Pbatique}~: décrochage et vrille arrivent plus au niveau de la piste
    pour atterir (dernier virages pour s'aligner)~: trouver lien plus probant
    que les vrilles avec avion de ligne~?
\end{itemize}

\subsection{07 avril}
\begin{itemize}
  \item Comparaison avion de voltige et de ligne~: simuler un décrochage,
    prendre trajectoire demi arc de cercle avec PFD, et intégrer 2 fois pour
    avoir les équations horaires du mouvement et réinjecter les valeurs de $x$
    dans $y$ pour comparer l'altitude perdue pour les deux avions.
\end{itemize}

\subsection{14 avril}
\begin{itemize}
  \item Résolution numérique~? Voir sur Capytale~?
\end{itemize}

\subsection{05 mai}
\begin{itemize}
  \item Vol effectué. Altitude, vitesse début décrochage et sortie. Comparaison
    avec données expérimentales Boeing 737.
  \item Complétion avec Python (template pendule conique).
  \item Caméra dysfonctionnelle.
\end{itemize}

\subsection{12 mai}
\begin{itemize}
  \item Présentation~: recherche de support image, diapos en cours.
\end{itemize}

\subsection{26 mai}
\begin{itemize}
  \item Équation différentielle Python commencée.
  \item Diapos pas commencées.
\end{itemize}

\subsection{02 juin}
\begin{itemize}
  \item Python avancé, chouette
  \item Attention reste de $v$ dans l'équation différentielle~!
  \item À retravailler.
  \item Diapos pas commencées.
  \item Plan~: TB.
\end{itemize}

\section{Présentation}

\begin{enumerate}
  \item[1)] Intro ouais bon, on va commencer.
  \item[2)] Ok. Voix peu convaincante, mais bon. Propre et posée.
  \item[3)] Superbe transition avec la problématique.
  \item[4)] Définir la portance avant~? Sinon pas mal.
  \item[5)] Ok bonne représentation après. Par contre diapo un peu dégueu, flux
    d'aire invisible. Très bonne explication de la portance~!!
  \item[6)] Super description du but.
  \item[7)] Étude expérimentale la plus stylée. Convertir les pieds quand même.
    Bien la description des images.
  \item[8)] Ça sert à quoi~?
  \item[9)] Flèches sur les vecteurs.
  \item[10)] Programme \textbf{Python} sur Capytale. Équations dégueu. Pas de
    screen, un tableau.
  \item[11)] Ok figures~!
  \item[12)] Bonne interprétation. Ok.
  \item[13)] D'accord bonne conclusion.
\end{enumerate}

\begin{itemize}
  \item Décrochage avion de chasse~?
  \item Looping avion de ligne~? Différentes catégories. Trop de contraintes sur
    l'avion.
  \item Vrille~? Peut-être possible.
  \item Analogie pendule et avion~? Ok.
\end{itemize}

\chapter{Alizé \textsc{Travers}}
\label{ch:alize}
\section{Suivi}
\subsection{10 février}

\begin{itemize}
    \litem{Sujet}~: course para et pas para
    \litem{Manip}~: ressort etc
\end{itemize}

\subsection{03 mars}
\begin{itemize}
    \litem{Pbatique}~: Peut-on améliorer les performances des sprinteurs grâce
        aux sciences
    \litem{Plan}~:
        \begin{enumerate}[label=\protect\fbox{\Roman*}]
            \item Non handi
            \item Handi
        \end{enumerate}
\end{itemize}

\subsection{10 mars}
\begin{itemize}
    \litem{Biblio}~: thèse sur capacités handi fauteuil~; thèse analyse cinétique
        et cinématique starting block, super.
    \litem{Manip}~: à trouver avec synergie biblio.
\end{itemize}

\subsection{17 mars}
\begin{itemize}
    \litem{Biblio}~: masse-ressort cheville. Étude sinusoïdale. Plein de doc pour
        intro et quelques caractéristiques
    \litem{Manip}~: bonne question. Étude couple avec 2 ressorts~?
\end{itemize}

\subsection{24 mars}
\begin{itemize}
    \litem{Manip}~: réflexion sur la mesure à faire, système de disque avec 2
        ressorts.
\end{itemize}

\subsection{31 mars}
\begin{itemize}
  \litem{Pbatique}~: quelle méthode de sprint serait la plus efficace/optimale.
  \litem{Manip}~: cette idée du couple avec 2 ressorts sur une roue~; et idée
    étude des matériaux.
\end{itemize}

\subsection{07 avril}
\begin{itemize}
  \item Manip c'est parti !
\end{itemize}

\subsection{14 avril}
\begin{itemize}
  \item Mesure de couple plus pertinente~?
\end{itemize}

\subsection{05 mai}
\begin{itemize}
  \item Bonne idée de réalisation style bras de micro.
\end{itemize}

\subsection{12 mai}
\begin{itemize}
  \item Vérifier la faisabilité.
\end{itemize}

\subsection{26 mai}
\begin{itemize}
  \item Manip let's go
\end{itemize}

\subsection{02 juin}
\begin{itemize}
  \item Expérience ressort mesure écartement spires.
  \item Diapos mises en page.
\end{itemize}

\section{Présentation}

\begin{enumerate}
  \item Pas vu, mais ok. Parle très très bien.
  \item Différences demi-fond et sprint. Talon ou pointes.
  \item Ok
  \item Euh ok I guess
  \item D'accord. «~?~».
  \item Plus grand. Trop compliqué~: si vous ne pouvez pas l'expliquer, ne pas
    le mettre.
  \item LES SOURCES~!
  \item Très, très bien amené et expliqué. Mais, c'est quoi le but des
    expériences~?
  \item Su-per.
  \item Ok
  \item Ok
  \item TB.
  \item AH bon. Pourquoi on fait ça en fait~?
  \item Ok
  \item Excellent~!
  \item Ok.
\end{enumerate}

\subsection{Questions}
\begin{itemize}
  \item Diapo 16~: handisport c'est quoi~?
  \item Diapo 7~: elle fonctionne comment cette expérience~? Bien.
  \item C'était quoi qui t'intéressait ici~? Corrélation constante de raideur
    des muscles et tendons, et différentes catégories de sportif-ves.
  \item Plateforme de \textsc{Kistler}~? Personne debout sur plateforme.
    Mouvement cheville, mesure forces. Ok.
\end{itemize}

\chapter{Mathis \textsc{Siot} et Bruno \textsc{Idir}}
\label{ch:mathisbruno}
\section{Suivi}
\subsection{10 février}
\begin{itemize}
  \litem{Sujet}~:
    \begin{itemize}
      \item Football ballon, captation données
      \item Basket~: information basket données, shoot parfait, angle etc
    \end{itemize}
\end{itemize}

\subsection{03 mars}
\begin{itemize}
    \litem{Sujet}~: arc et tir à l'arc, énergétique et vitesse différents arcs
    \litem{Pbatique}~: aucune
\end{itemize}

\subsection{10 mars}
\begin{itemize}
    \litem{Biblio}~: en cours, information sur confection, différents matériaux
        et conséquences sur l'arc~: poignée (carbone = précision et facilité à
        tendre ou alu = moins précision) et branche (dépend tireur).
    \litem{Manip}~: aller voir des clubs, essayer dynamomètre~?
\end{itemize}

\subsection{17 mars}
\begin{itemize}
    \item Énergie stockée dans la flexion d'une barre~?
    \litem{Biblio}~: super schéma fonctionnement arc et forces associées. Voir
        \nameref{sec:axel} pour caractérisation flexion/force.
\end{itemize}

\subsection{24 mars}
\begin{itemize}
    \litem{Pbtique}~: en avance.
    \litem{Biblio}~: à avancer. Trouver caractérisation rigidité~?
\end{itemize}

\subsection{31 mars}
\begin{itemize}
  \litem{Pbatique}~: à trouver (littéral).
  \litem{Manip}~: pas avancé.
  \item Avancé sur la friction… ràv.
\end{itemize}

\begin{center}
  Chercher déformation
\end{center}

\subsection{07 avril}
\begin{itemize}
  \item Achat matériau pour voir la rigidité avec manip simple.
  \item Formule de la déformation en fonction de la force appliquée~: recherche
  sur le module d'\textsc{Young} et la rigidité en flexion.
\end{itemize}

\subsection{14 avril}
Travail avec Bruno.
\begin{itemize}
  \litem{Biblio}~: avance bien !
  \litem{Manip}~: en avance aussi.
\end{itemize}

\subsection{05 mai}
\begin{itemize}
  \item Euuuh. Spine de la flèche~?
  \item Comment faire Robin des Bois~?
  \item Expérience à venir.
  \item Plan à continuer.
\end{itemize}

\subsection{12 mai}
\begin{itemize}
  \item Plan détaillé aujourd'hui.
  \item Explication causes oscillations de la flèche~: $T(x+\dd x)$ etc.
\end{itemize}

\subsection{26 mai}
\begin{itemize}
  \item Bruno~: paradoxe de l'archer (oscillations de la flèche)~: ne pas viser
    le centre de la cible. Corde cause vibration.
  \item Présentation pas avancée.
  \item Mathis~: absent
\end{itemize}

\subsection{02 juin}
\begin{itemize}
  \item Bruno~: diapos ok.
  \item Mathis~: diapos en cours.
\end{itemize}

\section{Présentation}

\subsection{Mathis}
\begin{itemize}
  \item Début chaotique~: «~bon bah on va commencer~».
  \item Sommaire mais sans contexte~?
  \item Omg la nuit des temps. «~C'est important de le savoir~»*2 en parlant de
    l'histoire passée et pratiques actuelles.
    TROP PETIT.
  \item Bien pour les numéros de diapos, mais faut pas abuser avec la flèche non
    plus.
  \item Noir sur bleu~: très mauvaise idée~!
  \item Espace avant les deux points.
  \item Problématisation dans le mauvais sens~? Diapo 5 à revoir
  \item 3'20" avant I~: trop long~!
  \item Trop sur ses notes.
  \item Est-ce qu'il vaut mieux une flèche souple ou pas~? Diapo 8.
  \item Diapo 10 en 2s~!!
  \item Expérience pas réalisée diapo 12
  \item Plus une présentation de 3\ieme
  \item Re-bilan après conclusion, et trop petit en plus, woah.
\end{itemize}

\subsection{Bruno}
\begin{itemize}
  \item Très bien numéro de pages. Ok prénom.
  \item Paradoxe de l'archer~: nécessite correspondance spin et puissance.
  \item Précisément~: ne pas viser le centre de la cible mais à côté.
  \item Diapo 3~: illisible.
  \item Pas mal diapo 5.
  \item ENFIN un TMC. Mais équations trop petites~!
  \item Le DOIGT.
  \item TB diapo 8, mais mal exploitée. Décrire l'évidence~!
  \item Pas mal dans l'ensemble, traduit bien la nécessité des plumes.
  \item Diapo 9~: indiquer distances sur diapos.
  \item Bonne présentation diapo 10. Enfin, presque, \SIrange{200}{2000}{mm}~:
    ça fait \SI{2}{m}~!!
  \item Diapo 11~: l'enfer. «~Je sais plus la puissance de l'arc………~»
\end{itemize}

\subsection{Questions}
\begin{itemize}
  \litem{Natiao}~: t'as dit que spine proportionnel à la déformation~:
    $\lra$ en profite pour dire tout ce qu'il sait sur le module
    d'\textsc{Young}…
  \litem{Natiao}~: donc, ma question~: \SI{2000}{m m}~?
    $\lra$ confond en boucle.
  \litem{Samy}~: t'as pas présenté l'abscisse l'ordonnée etc (bien vu~!)
    $\lra$ blabla (vue de dessus~? oui)
  \litem{Séréna}~: comment choisir (???)
\end{itemize}

\chapter{Maxime \textsc{Watroba}}
\label{ch:maxime}
\section{Suivi}
\subsection{10 février}
\begin{itemize}
    \litem{Sujet}~:
        \begin{itemize}
            \item Distance max saut à ski
            \item Record Usain \textsc{Bolt} à battre
        \end{itemize}
\end{itemize}

\subsection{03 mars}
\begin{itemize}
    \litem{Sujet}~: Usain \textsc{Bolt}.
    \litem{Pbatique}~: aucune
\end{itemize}

\begin{center}
    Avancée mineure~: idée de plan.
\end{center}

\subsection{10 mars}
\begin{itemize}
    \litem{Biblio}~: pas trouvée spécifiquement sur Usain \textsc{Bolt}
    \litem{Manip}~: polystyrène pour tester frottements~: Peu convaincant.
\end{itemize}

\begin{center}
    Voir si retour sur saut à ski~?
\end{center}

\subsection{17 mars}
\begin{itemize}
    \item Innovation technologiques~: histoire, législation, chaussures nike
        interdites. Pistes avec rendu énergétique.
    \item Manip à trouver.
\end{itemize}

\subsection{24 mars}
\begin{itemize}
    \item Revêtements sol impact course.
    \item Étude énergétique masse, voir exo TD ch4 méca modélisation sol comme
        ressorts.
    \litem{Biblio}~: étude allemande. Université
\end{itemize}

\subsection{31 mars}
\begin{itemize}
  \litem{Pbatique}~: quelle piste permet les meilleures performances sportives~?
  \litem{Manip}~: jambe bâton, sol constante de raideur et évolution énergie
    mécanique avec $k$~?
  \item Reprendre exemple de la grimpeuse~?
\end{itemize}

\subsection{07 avril}
\begin{itemize}
  \item Chute libre avec frottements~: résolution numérique~? Penser à
    l'approche énergétique.
\end{itemize}

\subsection{14 avril}
\begin{itemize}
  \item Très bonne avancée niveau calculs,
  \litem{Manip}~: différents systèmes plaque-ressort.
\end{itemize}

\subsection{05 mai}
\begin{itemize}
  \item Idée système plaque-ressort
  \item Python après.
  \litem{Plan}~: évolution terre battue -> pistes. Intro impact chaussures et
  sol, concentre sur sol.
\end{itemize}

\subsection{12 mai}
\begin{itemize}
  \item Expérience mise au clair~: hypothèses etc. Souder ressort sur la
    plaque~?
  \item Plan~: évolution piste, chaussures Nike interdiction etc. Prédominance
    du sol sur amélioration.
  \item Présentation expérience etc. Python, frottements quadratiques etc~:
    rendu énergétique.
  \item Utile ou pas~?
\end{itemize}

\subsection{26 mai}
\begin{itemize}
  \item Présentation commencée, pas mal~!
\end{itemize}

\subsection{02 juin}
\begin{itemize}
  \item Diapos bien avancées
  \item Fake graphe à faire.
\end{itemize}

\section{Présentation}

\begin{enumerate}[start=0]
  \item Bof «~je vais présenter~». Bien diapo, joli.
  \item «~Pistes d'athlé.~»…
  \item J'ai pas écouté. Trop de notes.
  \item Pas écrire la pas logique~! Mais ok.
  \item Euh un schéma svpppp.
  \item Idée ok.
  \item Schéma~: bof gris sur vert.
  \item Ok, rapide
  \item D'accord.
  \item D'acc'. Bien schéma~!
  \item $\vv{R}$ pour ressort~: bizarre.
  \item Ok
  \item[16)] Ok TB.
  \item[17)] Ok.
\end{enumerate}

\begin{itemize}
  \item Expérience~: vraiment tige~? Ça va être incontrôlable~! Pourquoi pas
    boule~?
\end{itemize}


% \section{Arsène \textsc{Joffre}}
% \label{ch:arsene}
% \subsection{10 février}
% \begin{center}
%     Absent.
% \end{center}
% 
% \subsection{03 mars}
% \begin{itemize}
%     \litem{Sujet}~: Accéléromètre manette joycon ou infrarouge manette wii,
%     \litem{Pbatique}~: aucune
%     \litem{Biblio}~: début
% \end{itemize}
% 
% \subsection{10 mars}
% \begin{itemize}
%     \litem{Manip}~: appli sur téléphone. Différence infrarouge/accéléromètre.
%         Plutôt se concentrer sur le masse-ressort.
%     \litem{Biblio}~: 4 masses sur ressort
%     \litem{Pbatique}~: comment les accéléromètres ont-ils révolutionné le contact
%         aux consoles~? À revoir.
% \end{itemize}
% 
% \subsection{17 mars}
% \begin{itemize}
%     \litem{Sujet}~: lancers au baseball.
%     \litem{Pbatique}~: 
%     \litem{Biblio}~: étude de l'impact à trouver
%     \litem{Manip}~: à trouver.
% \end{itemize}
% 
% \subsection{24 mars}
% \begin{center}
%     Absent.
% \end{center}
% 
% \subsection{31 mars}
% \begin{itemize}
%   \item \textsc{Rubik}'s cube~? ou mise en groupe physique~?
% \end{itemize}

% \section{Clément \textsc{El Housni}}
% \label{ch:clement}
% \subsection{10 février}
% 
% \begin{itemize}
%     \litem{Sujet}~:
%         \begin{itemize}
%             \item Chaussures en carbone, record, \SI{100}{m} d'Usain
%                 \textsc{Bolt}
%             \item Échecs en info, IA.
%         \end{itemize}
% \end{itemize}
% 
% \subsection{03 mars}
% \begin{center}
%     Absent
% \end{center}
% 
% \subsection{10 mars}
% \begin{center}
%     Passé en maths.
% \end{center}

% \section{Kéo \textsc{Gravelard}}
% \label{ch:keo}
% \subsection{10 février}
% 
% \begin{itemize}
%     \litem{Sujet}~: Boxe, étude frottements fluides et vitesse
%     \litem{Pbatique}~: densité maximale pour porter un coup.
% \end{itemize}
% 
% \subsection{03 mars}
% \begin{center}
%     Absent \smallbreak
%     Aucune avancée
% \end{center}
% 
% \subsection{10 mars}
% \begin{center}
%     Aucune avancée.
% \end{center}
% 
% \subsection{17 mars}
% \begin{center}
%     Parti en maths
% \end{center}

% \section{Yahia \textsc{Ouechtati}}
% \label{ch:yahia}
% \subsection{10 février}
% \begin{itemize}
%     \litem{Sujet}~: Hockey sous glace
% \end{itemize}
% 
% \subsection{03 mars}
% \begin{center}
%     Passé en maths
% \end{center}

% \section{Timé \textsc{De Miranda}}
% \label{ch:time}
% \subsection{10 février}
% 
% \begin{itemize}
%     \litem{Sujet}~:
%         \begin{itemize}
%             \item F1… frottements
%             \item Éventuellement astrophysique~? 
%         \end{itemize}
% \end{itemize}
% 
% \subsection{03 mars}
% \begin{itemize}
%     \litem{Sujet}~: Évolution contrôle dopage.
% \end{itemize}
% 
% \subsection{10 mars}
% \begin{itemize}
%     \litem{Manip}~: chromatographie pour test.
%     \litem{Biblio}~: un document. Peu de documentation sur internet $\LRa$ faire
%         une visite~? Voir avec labo chimie.
% \end{itemize}
% 
% \subsection{17 mars}
% \begin{itemize}
%     \litem{Sujet}~: partir en maths, cryptographie
% \end{itemize}

\end{document}
