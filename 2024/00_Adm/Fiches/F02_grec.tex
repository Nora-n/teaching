\documentclass[a4paper, 12pt, garamond]{book}
\usepackage{cours-preambule}

\makeatletter
\renewcommand{\@chapapp}{Fiches -- numéro}
\makeatother

\begin{document}
\setcounter{chapter}{1}

\chapter{Alphabet grec pour les sciences}

On utilise de nombreuses lettres grecques comme variables en sciences. Il faut
vous familiariser avec leur symbole, leur prononciation et la manière de les
écrire.

\begin{center}
	\begin{tabular}{ccccl}
		\toprule
		\textbf{Minuscule}          & \textbf{Majuscule} & \textbf{Nom} & \textbf{Prononciation} & \textbf{Exemple}                       \\
		\midrule
		$\alpha$                    &                    & alpha        & alfa                   & angle $\alpha$                         \\
		$\beta$                     &                    & beta         & béta                   & angle $\beta$                          \\
		$\gamma$                    & $\Gamma$           & gamma        & gama                   & conductivité $\gamma$, couple $\Gamma$ \\
		$\delta$                    & $\Delta$           & delta        & dèlta                  & variations $\delta$ et $\Delta$        \\
		$\epsilon$ ou $\varepsilon$ &                    & epsilon      & èpsilonne              & petite quantité $\epsilon$             \\
		$\zeta$                     &                    & zeta         & zéta                   & fonction de \textsc{Riemman} $\zeta$   \\
		$\eta$                      &                    & eta          & éta                    & rendement $\eta$                       \\
		$\tt$                       & $\Theta$           & theta        & téta                   & angle $\tt$, température $\Theta$      \\
		$\iota$                     &                    & iota         & (hi)yota               & \textit{pas d'exemple}                 \\
		$\kappa$                    &                    & kappa        & kapa                   & courbure $\kappa$                      \\
		$\lambda$                   & $\Lambda$          & lambda       & lent-mbda              & longueur d'onde $\lambda$              \\
		$\mu$                       &                    & mu           & mu                     & préfixe micro $\mathrm{\mu}$           \\
		$\nu$                       &                    & nu           & nu                     & fréquence $\nu$                        \\
		$\xi$                       & $\Xi$              & xi           & ksi                    & avancement molaire $\xi$               \\
		$\pi$                       & $\Pi$              & pi           & pi                     & nombre $\pi$, produit $\Pi$            \\
		$\rho$                      &                    & rho          & ro                     & masse volumique $\rho$                 \\
		$\sigma$                    & $\Sigma$           & sigma        & sigma                  & conductivité $\sigma$, somme $\Sigma$  \\
		$\tau$                      &                    & tau          & to                     & temps caractéristique $\tau$           \\
		$\upsilon$                  & $\Upsilon$         & upsilon      & upsilonne              & \textit{pas d'exemple}                 \\
		$\phi$ ou $\varphi$         & $\Phi$             & phi          & fi                     & phase $\phi$, flux $\Phi$              \\
		$\chi$                      &                    & chi          & ki                     & compressibilité $\chi$                 \\
		$\psi$                      & $\Psi$             & psi          & psi                    & phase $\psi$                           \\
		$\w$                        & $\Omega$           & omega        & oméga                  & pulsation $\w$, Ohm $\Omega$
		\\
		\bottomrule
	\end{tabular}
\end{center}

\textbf{Remarques}~:
\begin{enumerate}
	\item Les lettres majuscules non indiquées sont identiques au latin.
	\item La lettre grecque \textit{omicron} est indifférentiable du \textit{o}
	      latin, et n'est pas utilisée en sciences.
\end{enumerate}

\end{document}
