% https://tex.stackexchange.com/a/498765
\documentclass{standalone}

\usepackage{tikz}
\usetikzlibrary{calc, positioning, shapes, backgrounds, fit, arrows}
\usepackage{pgf-spectra}

\usepackage{siunitx}

\usepackage{contour}

\begin{document}

\pgfdeclarehorizontalshading{visiblelight}{50bp}{% https://tex.stackexchange.com/a/348492/120853
	color(0bp)=(violet!25);
	color(8.33bp)=(blue!25);
	color(16.67bp)=(cyan!25);
	color(25bp)=(green!25);
	color(33.33bp)=(yellow!25);
	color(41.5bp)=(orange!25);
	color(50bp)=(red!25)
}%

\begin{tikzpicture}[%
		raylabel/.style={font=\scriptsize}
	]
	\def\minexponent{-6}
	\def\maxexponent{8}
	\def\spectrumheight{9em}

	\pgfmathtruncatemacro{\nextminexponent}{\minexponent + 1}
	\pgfmathtruncatemacro{\previousmaxexponent}{\maxexponent - 1}

	% Main foreach loop, drawing the wavelengths as powers of 10 in an alternating fashion: even on top, odd at bottom. Then connects them with help lines
	\foreach [count=\i, remember=\exponent as \previousexponent, evaluate=\i as \currentposition using int(\i/2)] \exponent in {\minexponent, \nextminexponent, ..., \maxexponent}{
			\ifodd\exponent
				\def\height{0}
				\pgfmathtruncatemacro{\currentexponent}{14-\exponent}
				\node[anchor=base] (WAVELENGTH_\exponent) at (\exponent, \height) {\contour{white}{\num{e\currentexponent}}};
			\else
				\def\height{\spectrumheight}
				\pgfmathtruncatemacro{\currentexponent}{\exponent+3}
				\node[anchor=base] (WAVELENGTH_\exponent) at (\exponent, \height) {\contour{white}{\num{e\currentexponent}}};
			\fi

			% Anchor at baseline to get all nodes on same baseline.
			% https://tex.stackexchange.com/questions/133227/how-to-align-text-in-tikz-nodes-by-baseline#comment300863_133227
			% \node[anchor=base] (WAVELENGTH_\exponent) at (\exponent, \height) {\contour{white}{\num{e\exponent}}};

			\ifnum\i > 1
				\ifodd\i
					\node (LABEL_\currentposition)
					at ($(WAVELENGTH_\exponent)!0.5!(WAVELENGTH_\previousexponent)$)
					{};% This is left as a node as opposed to coordinate: fill it out with '\currentposition' for debugging
				\else
					% Do not draw connection at exponent 1:
					% \pgfmathparse{\exponent != 1}% \pgfmathparse stores result (0 or 1) in macro \pgfmathresult
					% \ifnum\pgfmathresult = 1
					\draw[help lines]
					(WAVELENGTH_\previousexponent) --
					(WAVELENGTH_\exponent)
					node[midway] (CONNECTION_\currentposition) {}% This is left as a node as opposed to coordinate: fill it out with '\currentposition' for debugging
					coordinate[at start] (CONNECTION_\currentposition_START)
					coordinate[at end] (CONNECTION_\currentposition_END);
					% \fi
				\fi
			\fi
		}

	% Create an arrow shape that fits around all relevant nodes, but do not draw it.
	% Draw it manually later to leave out the 'bottom' of the arrow.
	% We still need this invisible arrow for lining up of coordinates
	\node[
		double arrow,
		double arrow head extend=0pt,
		double arrow tip angle=150,% Inner angle of arrow tip
		fit={([xshift=-3em]CONNECTION_1_START)(CONNECTION_1_END)(CONNECTION_\previousmaxexponent_START)([xshift=5em]CONNECTION_\previousmaxexponent_END)},
		inner sep=0pt, draw
	]
	(ARROW) {};

	% \node[align=center]
	% (THERM) at
	% ([yshift=3em]WAVELENGTH_1|-ARROW.after tip 2)
	% {Radiation\\thermique};% Only works because exponent 1 is between -1 and 3
	% \draw (THERM) -| ([yshift=-1.5em]WAVELENGTH_-1|-THERM);
	% \draw (THERM) -| ([yshift=-1.5em]WAVELENGTH_3|-THERM);

	% On background layer so already drawn arrow and scale lines cover it up nicely
	\begin{scope}[on background layer]
		\node[
			inner sep=0pt,
			outer sep=0pt,
			fit={([xshift=-2.2em]WAVELENGTH_0|-ARROW.after tip 2)([xshift=-2.2em]WAVELENGTH_1|-ARROW.before tip 2)}, shading=visiblelight]
		(SMALL_VISIBLE_LIGHT) {};

		\shade[
			left color=white,
			right color=violet!25,
			middle color=violet!5,
			outer sep=0pt
		]
		(CONNECTION_3_START) -- (CONNECTION_3_END) -- ([xshift=\pgflinewidth]SMALL_VISIBLE_LIGHT.south west) -- ([xshift=\pgflinewidth]SMALL_VISIBLE_LIGHT.north west) -- cycle;

		\shade[
			left color=red!25,
			right color=white,
			middle color=red!5,
			outer sep=0pt,
		]
		(CONNECTION_5_START) -- (CONNECTION_5_END) -- ([xshift=-\pgflinewidth]SMALL_VISIBLE_LIGHT.south east) -- ([xshift=-\pgflinewidth]SMALL_VISIBLE_LIGHT.north east) -- cycle;
	\end{scope}

	% Some labels can be drawn automatically at the designated label coordinates:
	\foreach [count=\i] \label in {
			{Rayons\\gamma},
			{Rayons\\X},
			{},%Skip this one
			{Infrarouge},
			{Micro\\ondes},
		}{
			\node[raylabel, align=center] at (LABEL_\i) {\label};
		}

	% These do not fit the loop and are drawn manually:
	% \node[raylabel, align=right, anchor=north] at ([yshift=-1em]$(WAVELENGTH_-2)!0.45!(WAVELENGTH_0)$) {Ultra-\\violet};
	% \node[raylabel, fill=white] at (CONNECTION_5) {Infrarouge};

	\node[raylabel, fill=white] at (CONNECTION_7) {Ondes radio};
	% \node[raylabel, align=center] at (LABEL_6) {Ondes radio};

	\node[raylabel, left=0.1em of CONNECTION_1, align=right] {Rayons\\cosmiques};

	\node[
	draw,
	fill=black!20,
	below=4em of SMALL_VISIBLE_LIGHT,
	align=center,
	label=above:{\textbf{Spectre visible}}
	] (FULL_VISIBLE_LIGHT) {%
	\pgfspectra[width=13em,height=0em]\\%pgfspectra also has a builtin axis which of course much better than this terrible approach, but it is in nanometer
		{\hspace{1.8em}\num{380} \hspace{1em}\num{480} \hspace{1em}\num{580}\hspace{1em} \num{680} \hspace{1em}\num{780}} [\si{nm}]\\
	\pgfspectra[width=13em,height=3em]\\%pgfspectra also has a builtin axis which of course much better than this terrible approach, but it is in nanometer
		{[\SI{e14}{Hz}]~\num{7.9} \hspace{1.3em}\num{6.2} \hspace{1.3em}\num{5.2}\hspace{1.0em} \num{4.4} \hspace{1.5em}\num{3.8}\hspace*{3.5em}~}\\
	{\hfill\hspace{2em} bleu \hspace{.8em} vert jaune \hfill rouge \hfill}
	};

	% Draw 'magnifying' trapeze, on background so it is covered by scale labels
	\begin{scope}[on background layer]
		\filldraw[help lines, fill=black!10] (FULL_VISIBLE_LIGHT.north east) -- (SMALL_VISIBLE_LIGHT.south east) -- (SMALL_VISIBLE_LIGHT.south west) -- (FULL_VISIBLE_LIGHT.north west) -- cycle;
	\end{scope}

	% Draw around arrow manually, leaving its tail open
	% \node[right=1em of ARROW.before tip 1] {$\lambda [\si{\micro m}]$};
	% \node[left=1em of ARROW.before tip 2] {$f [\si{s^{-1}}]$};
	\draw[draw, thick]
	(ARROW.after tip 2) --
	% (ARROW.before head) --
	(ARROW.before tip 1) node[above right] {$\lambda [\si{nm}]$} --
	(ARROW.tip 1) --
	(ARROW.after tip 1) --
	% (ARROW.after head) --
	(ARROW.before tip 2) node[below left] {$f [\si{Hz}]$} --
	(ARROW.tip 2) -- cycle;

\end{tikzpicture}
\end{document}
