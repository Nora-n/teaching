\documentclass{standalone}
\usepackage{mintikz}

\begin{document}
\begin{tikzpicture}[]
	%\draw [ultra thin, gray!20] (-6,-6) grid[step=0.5] (6,6);
	%\draw [thin,gray!50] (-6,-6) grid[step=1] (6,6);
	\def \taillehaut{2}; %taille de la lentille
	\def \taillebas{2};%taille de la lentille
	\def \xA{-2};%position de l'objet
	\def \yB{1};%hauteur de l'objet
	%\def \xAA {(\xA*\f)/(\xA+\f)};%position de l'image
	%\def \yBB {(\xAA*\yB)/(\xA)};%hauteur de l'image
	\coordinate (O1) at (0,0);%centre optique de la première lentille
	\coordinate (A) at (\xA,0);%position de l'objet
	\coordinate (B) at (\xA,\yB);%sommet de l'objet
	\def \f{2};%focale de la première lentille
	\coordinate (F') at (\f,0);
	%\coordinate (A') at ({\xAA},0);%position de l'objet
	%\coordinate (B') at ({\xAA},{\yBB});%sommet de l'objet

	\draw[thin,->](-4.5,0)--++(8,0)node[below]{A.O.};
	\draw[|->, Purple!70, thick] (A)node[below left]{A}--(B)node[above]{ B};
	%\draw[|->,dashed, thick] (A')node[below]{ A'}--(B')node[above]{ B'};
	\draw[shift={(O1)},ultra thick,<->,>=latex] (0,-\taillebas)--(0,\taillehaut) node[above]{ $\mathcal{L}$};%lentille convergente
	% \draw[thick,brandeisblue,simple] (B)--(0,\yB);
	% \draw[thick,orange!70,simple] (0,\yB)--(F');
	% \draw[thick,orange!70,dashed] (0,\yB)--++($-1*(O1)+1*(B)$);
	% \draw[thick,brandeisblue,double] (B)--(O1);
	% \draw[thick,orange!70,double] (O1)--++($1*(O1)-1*(B)$);
	% \draw[thick,orange!70,dashed] (B)--++($1*(B)-1*(O1)$);
	%\draw[blue,thick,double] (B)--(0,{\yBB});
	%\draw[blue,thick,double] (0,{\yBB})--++(2,0);
	%\draw[blue,thick,dashed] (0,{\yBB})--(B');
	\foreach \x/\z in {O1/\f}{
			\draw[shift={(\x)}] (0,0) node[below left] { O};
			\draw[shift={(\x)}] (\z,2pt) --++ (0,-4pt) node[below left] { F'};
			\draw[shift={(\x)}] ({-\z},2pt) --++ (0,-4pt) node[below right] { F};

			\oeil[shift={(2.8,0)},rotate=180];
		}
\end{tikzpicture}

\end{document}
