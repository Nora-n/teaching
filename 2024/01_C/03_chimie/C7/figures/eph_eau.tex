\documentclass{standalone}
\usepackage{mintikz}

\begin{document}
\begin{tikzpicture}[xscale=.5, yscale=3]
	\tikzmath{
		\wdth = 12;
		\hgth = 4;
		\emin = -0.85;
		\emax = 1.4;
		\tick = 0.025;
		\phmax = 12;
		\ct = 1;
		\pt = 1;
		\pke = 14;
		\estandod = 1.23;
		% \pksalohc = 33.5;
		% \phalohc = \pke-\pksalohc/3 - log10(\ct)/3;
		% \pbeta4 = -35;
		% \pkaalohc = \pksalohc + \pbeta4;
		% \phalohd = \pkaalohc + \pke + log10(\ct);
		% \estandalc = -1.66;
		% \efrontal = \estandalc + 0.02*log10(\ct);
		% \estandalohc = -2.31;
		% \estandalohd = -2.33;
	}
	\tikzset{declare function={
				% efront3(\x) = \estandalohc + 0.02*log10(\ct) - 0.06*\x;
				% efront4(\x) = \estandalohd + 0.02*log10(\ct) - 0.08*\x;
				efront1(\x) = \estandod - 0.06*(\x);
				efront2(\x) = - 0.06*(\x);
			}}
	\pgfmathparse{efront1(\phmax)}\let\elimod\pgfmathresult
	\pgfmathparse{efront2(\phmax)}\let\elimhd\pgfmathresult
	% print values
	% \node[draw, align=left] at (\wdth/2, \hgth/2)
	% {
	% $E^\circ(\ce{Al^3+/Al}) = \SI{\fpeval{round(\estandalc,2)}}{V}$\\
	% $E^\circ(\ce{Al(OH)3 /Al}) = \SI{\fpeval{round(\estandalohc,2)}}{V}$\\
	% $E^\circ(\ce{Al(OH)4 /Al}) = \SI{\fpeval{round(\estandalohd,2)}}{V}$\\
	% $\pH\ind{lim}(\ce{Al^3+/Al(OH)3}) = \num{\fpeval{round(\phalohc,2)}}$\\
	% $\pH\ind{lim}(\ce{Al(OH)4 /Al(OH)3}) = \num{\fpeval{round(\phalohd,2)}}$\\
	% $E\ind{front}(\ce{Al(OH)4 /Al(OH)3/Al}) = \SI{\fpeval{round(\efrontlim,2)}}{V}$
	% };
	% 0
	\coordinate (O) at (0,0);
	\coordinate (Z) at (\wdth,0);
	% place values
	\coordinate (A) at (0,\estandod);
	\coordinate (B) at (\phmax,\elimod);
	\coordinate (C) at (\phmax,\elimhd);
	\coordinate (Bl) at (B-|O);
	\coordinate (Cl) at (C-|O);
	\draw (Bl) ++ (-6*\tick,0) --++ (12*\tick,0);
	\draw (Cl) ++ (-6*\tick,0) --++ (12*\tick,0);
	\node[above left=-.1cm and .3cm] at (Bl) {\num{\fpeval{round(\elimod,2)}}};
	\node[above left=-.1cm and .3cm] at (Cl) {\num{\fpeval{round(\elimhd,2)}}};
	% \coordinate (F) at (\wdth,\efrontend);
	% \coordinate (G) at (F-|O);
	% fill
	% \fill[red!20] (O) -- (A) -- (D) -- (C);
	% \fill[orange!20] (A) -- (B) -- (E) -- (D);
	% \fill[dgreen!20] (B) -- (Z) -- (F) -- (E);
	\fill[blue!20, opacity=.2]
	(A) --
	node[midway, above right, blue!80, opacity=1] {$\ce{{O_2}_{\rm(g)}}$}
	% node[below left] {$\ce{{H_2O}_{\rm(l)}}$}
	(B) --
	(C) --
	node[midway, below left, blue!80, opacity=1] {$\ce{{H_2}_{\rm(g)}}$}
	(O);
	% Place elements
	\node[blue!80] (H2O) at (barycentric cs:O=1,A=1,B=1,C=1) {$\ce{{H_2O}_{\rm(l)}}$};
	% Horiz
	% \draw[thick]
	% (C) -- (D);
	% Verti
	% \draw[thick]
	% (D) -- (A);
	% \draw[thick]
	% (E) -- (B);
	% Incli
	\draw[thick, dotted] plot[domain=0:\fpeval{\phmax}] (\x,{efront1(\x)});
	\draw[thick, dotted] plot[domain=0:\fpeval{\phmax}] (\x,{efront2(\x)});
	% Scale
	\draw[very thick, -{Stealth}]
	(-.5,0) --
	node[at end, right] {$\pH$}
	(\wdth+1,0);
	\foreach \x in {2, 4, ..., \fpeval{\wdth}}{
			\draw
			(\x,-\tick)
			node[below] {\num{\x}}
			--++ (0,2*\tick)
			;
		}
	\draw[very thick, -{Stealth}]
	(0,\emin) --
	node[at end, left] {$E~(\si{V})$}
	(0,\emax);
	\foreach \y in {-0.8, -0.4, ..., 1.3}{
			\draw
			(-6*\tick,\y)
			node[left=.3cm] {\num{\fpeval{round(\y,1)}}}
			--++ (12*\tick,0);
		}
\end{tikzpicture}

\end{document}
