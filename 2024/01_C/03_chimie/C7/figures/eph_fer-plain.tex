\documentclass{standalone}
\usepackage{mintikz}

\begin{document}
\begin{tikzpicture}[xscale=.5, yscale=3]
	\tikzmath{
		\wdth = 12;
		\hgth = 4;
		\emin = -0.85;
		\emax = 1.4;
		\phmax = 12;
		\tick = 0.025;
		\ct = 0.01;
		\pt = 1;
		\pke = 14;
		\pksfeohb = 15;
		\pksfeohc = 38;
		\phfeohb = \pke - 0.5*\pksfeohb - 0.5*log10(\ct);
		\phfeohc = \pke - \pksfeohc/3 - log10(\ct)/3;
		\estandfefeb = -0.44;
		\estandfebfec = 0.77;
		\efrontfefeb = \estandfefeb + 0.03*log10(\ct);
		\efrontfebfec = \estandfebfec;
	}
	\tikzset{declare function={
				% efront3(\x) = \estandalohc + 0.02*log10(\ct) - 0.06*\x;
				% efront4(\x) = \estandalohd + 0.02*log10(\ct) - 0.08*\x;
				efront1(\x) = \efrontfebfec - 0.18*(\x-\phfeohc);
				efront2(\x) = efront1(\phfeohb) - 0.06*(\x-\phfeohb);
				efront3(\x) = \efrontfefeb - 0.06*(\x-\phfeohb);
			}}
	\pgfmathparse{efront1(\phfeohb)}\let\elima\pgfmathresult
	\pgfmathparse{efront2(\phmax)}\let\elimb\pgfmathresult
	\pgfmathparse{efront3(\phmax)}\let\elimc\pgfmathresult
	% print values
	% \node[draw, align=left] at (\wdth/2, \hgth/2)
	% {
	% 	$E\ind{front}(\ce{Fe^3+/Fe^2+}) = \SI{\fpeval{round(\efrontfebfec,2)}}{V}$\\
	% 	$E\ind{front}(\ce{Fe^2+/Fe}) = \SI{\fpeval{round(\efrontfefeb,2)}}{V}$\\
	% 	$E\ind{lim}(\ce{Fe(OH)_3 /Fe^2+}) = \SI{\fpeval{round(\elima,2)}}{V}$\\
	% 	$\pH\ind{lim}(\ce{Fe^3+/Fe(OH)3}) = \num{\fpeval{round(\phfeohc,2)}}$\\
	% 	$\pH\ind{lim}(\ce{Fe^2+/Fe(OH)2}) = \num{\fpeval{round(\phfeohb,2)}}$
	% };
	% 0
	\coordinate (O) at (0,0);
	\coordinate (Z) at (\phmax,0);
	% place values
	\coordinate (A) at (0,\efrontfebfec);
	\draw (A) ++ (-6*\tick,0) --++ (12*\tick,0);
	% \node[left=.3cm] at (A) {\num{\fpeval{round(\efrontfebfec,2)}}};
	\coordinate (B) at (\phfeohc,\efrontfebfec);
	\coordinate (Bb) at (B|-O);
	\coordinate (Bt) at ($(B)+(0,0.5)$);
	\coordinate (Btl) at (Bt-|O);
	\coordinate (Btr) at (Bt-|Z);
	% \node[below] at (Bb) {\num{\fpeval{round(\phfeohc,2)}}};
	\coordinate (C) at (\phfeohb,\elima);
	\coordinate (Ct) at (C|-O);
	\coordinate (Cl) at (C-|O);
	% \draw (Cl) ++ (-6*\tick,0) --++ (12*\tick,0);
	% \node[above] at (Ct) {\num{\fpeval{round(\phfeohb,2)}}};
	\node[left=.3cm, white] at (Cl) {\num{\fpeval{round(\elima,2)}}};
	% \node[above] at (A) {\num{\fpeval{round(\phalohc,2)}}};
	% \node[above] at (B) {\num{\fpeval{round(\phalohd,2)}}};
	\coordinate (D) at (\phmax,\elimb);
	\coordinate (E) at (0,\efrontfefeb);
	\draw (E) ++ (-6*\tick,0) --++ (12*\tick,0);
	% \node[left=.3cm] at (E) {\num{\fpeval{round(\efrontfefeb,2)}}};
	\coordinate (F) at (\phfeohb,\efrontfefeb);
	\coordinate (G) at (\phmax,\elimc);
	\coordinate (Gl) at (G-|O);
	% \node[left] at (Gl) {\num{\fpeval{round(\elimc,2)}}}
	% fill
	% \fill[red!20] (O) -- (A) -- (D) -- (C);
	% \fill[orange!20] (A) -- (B) -- (E) -- (D);
	% \fill[dgreen!20] (B) -- (Z) -- (F) -- (E);
	% \fill[blue!20] (C) -- (D) -- (E) -- (F) -- (G);
	% Place elements
	% \node (FE3) at (barycentric cs:A=1,B=1,Bt=1,Btl=1) {$\ce{{Fe}^3+_{\rm(aq)}}$};
	% \node (FEOH3) at (barycentric cs:Bt=1,B=1,C=1,D=1,Btr=1) {$\ce{{Fe(OH)_3}_{\rm(s)}}$};
	% \node[rotate=-19] (FEOH2) at (barycentric cs:C=1,D=1,F=1,G=1) {$\ce{{Fe(OH)_2}_{\rm(s)}}$};
	% \node (FE2) at ([shift={(0,.2)}]barycentric cs:A=1,B=1,C=1,F=1,E=1) {$\ce{{Fe}^2+_{\rm(aq)}}$};
	% \node (FE) at (barycentric cs:E=1,F=1,G=1,Gl=1) {$\ce{{Fe}_{\rm(s)}}$};
	% Verti
	\draw[thick]
	(B) -- (Bt);
	% \draw[thick, dotted]
	% (B) -- (Bb);
	\draw[thick]
	(C) -- (F);
	% \draw[thick, dotted]
	% (C) -- (Ct);
	% Horiz
	\draw[thick]
	(A) -- (B);
	% \draw[thick, dotted]
	% (Cl) -- (C);
	\draw[thick]
	(E) -- (F);
	% incli
	\draw[thick] plot[domain=\fpeval{\phfeohc}:\fpeval{\phfeohb}] (\x,{efront1(\x)});
	\draw[thick] plot[domain=\fpeval{\phfeohb}:\fpeval{\phmax}] (\x,{efront2(\x)});
	\draw[thick] plot[domain=\fpeval{\phfeohb}:\fpeval{\phmax}] (\x,{efront3(\x)});
	% Scale
	\draw[very thick, -{Stealth}]
	(-.5,0) --
	node[at end, right] {$\pH$}
	(\wdth+1,0);
	\foreach \x in {2, 4, ..., \fpeval{\wdth}}{
			\draw
			(\x,-\tick)
			node[below] {\num{\x}}
			--++ (0,2*\tick)
			;
		}
	\draw[very thick, -{Stealth}]
	(0,\emin) --
	node[at end, left] {$E~(\si{V})$}
	(0,\emax);
	\foreach \y in {-0.8, -0.4, ..., 1.3}{
			\draw
			(-6*\tick,\y)
			node[left=.3cm] {\num{\fpeval{round(\y,1)}}}
			--++ (12*\tick,0);
		}
\end{tikzpicture}

\end{document}
