\documentclass{standalone}
\usepackage{mintikz}

\begin{document}
\begin{tikzpicture}[xscale=.5, yscale=3]
	\tikzmath{
		\wdth = 14;
		\phmax = 14;
		\hgth = 4;
		\emin = -0.85;
		\emax = 1.4;
		\tick = 0.025;
		\ct = 1;
		\pt = 1;
		\pke = 14;
		\estanda = 1.1625;
		\estandb = 0.60;
		\estandc = 1.10;
		\phlim = 50/6;
	}
	\tikzset{declare function={
				efront1(\x) = \estanda - 0.0675*(\x);
				efront2(\x) = \estandb;
				efront3(\x) = \estandc - 0.06*(\x);
			}}
	\pgfmathparse{efront1(\phmax)}\let\elima\pgfmathresult
	\pgfmathparse{efront2(\phmax)}\let\elimb\pgfmathresult
	\pgfmathparse{efront3(\phmax)}\let\elimc\pgfmathresult
	% 0
	\coordinate (O) at (0,0);
	\coordinate (Z) at (\wdth,0);
  \coordinate (A) at (0,\estanda);
  \coordinate (B) at (0,\estandb);
  \coordinate (Ae) at (\phmax,\elima);
  \coordinate (Be) at (\phmax,\elimb);
	% Incli
	\path[thick, name path=E1,
		mynode={0.25}{above, blue, rotate=-22}{$\boxed{\ce{{IO_3}^-}}$},
		mynode={0.25}{below, blue, rotate=-22}{$\ce{I_2}$},
		mynode={0.95}{above, blue, rotate=-22}{$\ce{{IO_3}^-}$},
		mynode={0.95}{below, blue, rotate=-22}{$\boxed{\ce{I_2}}$},
	]
	plot[domain=0:\fpeval{\phmax}] (\x,{efront1(\x)})
	;
	\path[thick, name path=E2,
		mynode={0.15}{above, red}{$\ce{I_2}$},
		mynode={0.15}{below, red}{$\boxed{\ce{I^-}}$},
		mynode={0.95}{above, red}{$\boxed{\ce{I_2}}$},
		mynode={0.95}{below, red}{$\ce{I^-}$},
	]
	plot[domain=0:\fpeval{\phmax}] (\x,{efront2(\x)})
	;
	% Intersection
	\path[name intersections={of=E1 and E2, by=pho}];
	% Plot \pH0
	\draw[dashed]
	(pho) -- (pho|-O)
	node[below] {$\pH_0$};
	% Before ph0
	\draw[thick, blue]
	(0,\estanda) -- (pho);
	\draw[thick, red]
	(0,\estandb) -- (pho);
	% After ph0
	\draw[thick, dashed, blue]
	(pho) -- (\phmax,\elima);
	\draw[thick, dashed, red]
	(pho) -- (\phmax,\elimb);
	% Final
	% \draw[thick,
	% 	mynode={0.5}{above, blue, rotate=-19}{$\ce{{IO_3}^-}$},
	% 	mynode={0.5}{below, red, rotate=-19}{$\ce{I^-}$},
	% ]
	%  plot[domain=\fpeval{\phlim}:\fpeval{\phmax}] (\x,{efront3(\x)})
	% ;
  % Place nodes for média dismut
	\node (MD) at (barycentric cs:A=1,B=1,pho=1) {};
	\node (DM) at (barycentric cs:Ae=1,Be=1,pho=1) {};
  % Fill
  \path[pattern=north west lines, opacity=.3] (A) -- (B) -- (pho) -- cycle;
  \path[pattern=north east lines, opacity=.3] (Ae) -- (Be) -- (pho) -- cycle;
  % Draw names
  \node[draw] (MDn) at (4,.2) {Médiamutation};
  \node[draw] (DMn) at (11.8,1) {Dismutation};
  % Draw flèches
  \draw[thick, ->] (MD.south east) to[bend left] (MDn.north);
  \draw[thick, ->] (DM.north) to[bend left] (DMn.south);
	% Scale
	\draw[very thick, -{Stealth}]
	(-.5,0) --
	node[at end, right] {$\pH$}
	(\wdth+1,0);
	\foreach \x in {2, 4, 6, 10, 12, 14}{
			\draw
			(\x,-\tick)
			node[below] {\num{\x}}
			--++ (0,2*\tick)
			;
		}
	\draw[very thick, -{Stealth}]
	(0,\emin) --
	node[at end, left] {$E~(\si{V})$}
	(0,\emax);
	\foreach \y in {-0.8, -0.4, ..., 1.3}{
			\draw
			(-6*\tick,\y)
			node[left=.3cm] {\num{\fpeval{round(\y,1)}}}
			--++ (12*\tick,0);
		}
\end{tikzpicture}

\end{document}
