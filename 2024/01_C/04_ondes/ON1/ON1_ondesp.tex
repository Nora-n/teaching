\documentclass[../../main/main.tex]{subfiles}
\graphicspath{{./figures/}}

\dominitoc
\faketableofcontents

% \renewcommand{\mtcSfont}{\small\bfseries}
% \renewcommand{\mtcSSfont}{\footnotesize}
\mtcsettitle{minitoc}{}
\mtcsetrules{*}{off}

\makeatletter
\renewcommand{\@chapapp}{Ondes -- chapitre}
\renewcommand{\chaplett}{ON}
\makeatother

% \toggletrue{student}
% \toggletrue{corrige}
% \renewcommand{\mycol}{black}
% \renewcommand{\mycol}{gray}

\hfuzz=5.003pt

\begin{document}
\setcounter{chapter}{0}

\settype{book}
\settype{prof}
\settype{stud}

\chapter{Ondes progressives}

\vspace*{\fill}

\begin{tcn}(appl)<ctc>"somm"'t'{Sommaire}
	\let\item\olditem
	\vspace{-15pt}
	\minitoc
	\vspace{-25pt}
\end{tcn}

\begin{tcn}[sidebyside, fontupper=\small, fontlower=\small]
	(appl)<ctb>"how"'t'{Capacités exigibles}
	\begin{itemize}[label=\rcheck]
		\item Identifier les grandeurs physiques correspondant à
		      des signaux acoustiques, électriques, électromagnétiques.
		\item Propagation d'un signal dans un milieu illimité, non dispersif et
		      transparent.
		\item Onde progressive dans le cas d'une propagation unidimensionnelle non
		      dispersive.
		\item Modèle de l'onde progressive sinusoïdale unidimensionnelle. Vitesse
		      de phase, déphasage, double périodicité spatiale et temporelle.
		\item Définir un milieu dispersif. Citer des exemples de situations de
		      propagation dispersive et non dispersive.
	\end{itemize}
	\tcblower
	\begin{itemize}[label=\rcheck]
		\item Citer quelques ordres de grandeur de fréquences dans les domaines
		      acoustique, mécanique et électromagnétique.
		\item Écrire les signaux sous la forme $f(x-ct)$ ou $f(x+ct)$, et
		      sous la forme $g(t-x/c)$ ou $g(t+x/c)$.
		\item Prévoir, dans le cas d'une onde progressive, l'évolution temporelle
		      à position fixée et l'évolution spatiale à différents instants.
		\item Établir la relation entre la fréquence, la longueur d'onde et la
		      vitesse de phase.
		\item Relier le déphasage entre les signaux perçus en deux points
		      distincts au retard dû à la propagation.
	\end{itemize}
\end{tcn}

\vspace*{\fill}
\newpage
\vspace*{\fill}

%\vspace{-15pt}
\begin{tcn}[%
		sidebyside, fontupper=\small, fontlower=\small
	](appl)<ctb>"chek"'t'{L'essentiel}
	\begin{tcn}[nsp](defi)<ctc>'t'{Définitions}
		\vspace{-25pt}
		\tcblistof[\subsubsection*]{defi}{\hspace*{4.8pt}}
	\end{tcn}
	% \begin{tcn}[nsp](rapp)<ctc>'t'{Rappels}
	% 	\vspace{-25pt}
	% 	\tcblistof[\subsubsection*]{rapp}{\hspace*{4.8pt}}
	% \end{tcn}
	\begin{tcn}[nsp](prop)<ctc>'t'{Propriétés}
		\vspace{-25pt}
		\tcblistof[\subsubsection*]{prop}{\hspace*{4.8pt}}
		% \tcblistof[\subsubsection*]{loi}{\hspace*{4.8pt}}
		% \tcblistof[\subsubsection*]{theo}{\hspace*{4.8pt}}
	\end{tcn}
	% \begin{tcn}[nsp](prop)<ctc>'t'{Théorèmes}
	% 	\vspace{-25pt}
	% 	% \tcblistof[\subsubsection*]{prop}{\hspace*{4.8pt}}
	% 	% \tcblistof[\subsubsection*]{loi}{\hspace*{4.8pt}}
	% 	\tcblistof[\subsubsection*]{theo}{\hspace*{4.8pt}}
	% \end{tcn}
	% \begin{tcn}[nsp](loi)<ctc>'t'{Lois}
	% 	\tcblistof[\subsubsection*]{loi}{\hspace*{4.8pt}}
	% \end{tcn}
	% \begin{tcn}[nsp](coro)<ctc>'t'{Corollaires}
	%   \tcblistof[\subsubsection*]{coro}{\hspace*{4.8pt}}
	% \end{tcn}
	\begin{tcn}[nsp](demo)<ctc>'t'{Démonstrations}
		\vspace{-25pt}
		\tcblistof[\subsubsection*]{demo}{\hspace*{4.8pt}}
		% \tcblistof[\subsubsection*]{prev}{\hspace*{4.8pt}}
	\end{tcn}
	% \begin{tcn}[nsp](inte)<ctc>'t'{Interprétations}
	% 	\tcblistof[\subsubsection*]{inte}{\hspace*{4.8pt}}
	% \end{tcn}
	% \begin{tcn}[nsp](impl)<ctc>'t'{Implications}
	% 	\vspace{-25pt}
	% 	\tcblistof[\subsubsection*]{impl}{\hspace*{4.8pt}}
	% \end{tcn}
	% \begin{tcn}[nsp](tool)<ctc>'t'{Outils}
	% 	\tcblistof[\subsubsection*]{tool}{\hspace*{4.8pt}}
	% \end{tcn}
	% \begin{tcn}[nsp](nota)<ctc>'t'{Notations}
	% 	\tcblistof[\subsubsection*]{nota}{\hspace*{4.8pt}}
	% \end{tcn}
	% \begin{tcn}[nsp](appl)<ctc>'t'{Applications}
	% 	\vspace{-25pt}
	% 	\tcblistof[\subsubsection*]{appl}{\hspace*{4.8pt}}
	% \end{tcn}
	% \begin{tcn}[nsp](rema)<ctc>'t'{Remarques}
	%   \tcblistof[\subsubsection*]{rema}{\hspace*{4.8pt}}
	% \end{tcn}
	% \begin{tcn}[nsp](exem)<ctc>'t'{Exemples}
	%   \tcblistof[\subsubsection*]{exem}{\hspace*{4.8pt}}
	% \end{tcn}
	% \begin{tcn}[nsp](ror)<ctc>"hart"'t'{Points importants}
	%   \tcblistof[\subsubsection*]{ror}{\hspace*{4.8pt}}
	% \end{tcn}
	% \begin{tcn}[nsp](impo)<ctc>'t'{Erreurs communes}
	%   \tcblistof[\subsubsection*]{impo}{\hspace*{4.8pt}}
	% \end{tcn}
	\tcblower
	% \begin{tcn}[nsp](defi)<ctc>'t'{Définitions}
	%   \tcblistof[\subsubsection*]{defi}{\hspace*{4.8pt}}
	% \end{tcn}
	% \begin{tcn}[nsp](rapp)<ctc>'t'{Rappels}
	%   \tcblistof[\subsubsection*]{rapp}{\hspace*{4.8pt}}
	% \end{tcn}
	% \begin{tcn}[nsp](prop)<ctc>'t'{Propriétés}
	% \tcblistof[\subsubsection*]{prop}{\hspace*{4.8pt}}
	% \tcblistof[\subsubsection*]{loi}{\hspace*{4.8pt}}
	% \tcblistof[\subsubsection*]{theo}{\hspace*{4.8pt}}
	% \end{tcn}
	% \begin{tcn}[nsp](coro)<ctc>'t'{Corollaires}
	%   \tcblistof[\subsubsection*]{coro}{\hspace*{4.8pt}}
	% \end{tcn}
	% \begin{tcn}[nsp](demo)<ctc>'t'{Démonstrations}
	% 	\vspace{-25pt}
	% 	\tcblistof[\subsubsection*]{demo}{\hspace*{4.8pt}}
	% 	% \tcblistof[\subsubsection*]{prev}{\hspace*{4.8pt}}
	% \end{tcn}
	% \begin{tcn}[nsp](inte)<ctc>'t'{Interprétations}
	% 	\tcblistof[\subsubsection*]{inte}{\hspace*{4.8pt}}
	% \end{tcn}
	% \begin{tcn}[nsp](impl)<ctc>'t'{Implications}
	% 	\tcblistof[\subsubsection*]{impl}{\hspace*{4.8pt}}
	% \end{tcn}
	% \begin{tcn}[nsp](tool)<ctc>'t'{Outils}
	% 	\vspace{-25pt}
	% 	\tcblistof[\subsubsection*]{tool}{\hspace*{4.8pt}}
	% \end{tcn}
	% \begin{tcn}[nsp](nota)<ctc>'t'{Notations}
	% 	\tcblistof[\subsubsection*]{nota}{\hspace*{4.8pt}}
	% \end{tcn}
	% \begin{tcn}[nsp](odgr)<ctc>'t'{Ordres de grandeur}
	% 	\tcblistof[\subsubsection*]{odgr}{\hspace*{4.8pt}}
	% \end{tcn}
	\begin{tcn}[nsp](appl)<ctc>'t'{Applications}
		\vspace{-25pt}
		\tcblistof[\subsubsection*]{appl}{\hspace*{4.8pt}}
	\end{tcn}
	% \begin{tcn}[nsp](rema)<ctc>'t'{Remarques}
	%   \tcblistof[\subsubsection*]{rema}{\hspace*{4.8pt}}
	% \end{tcn}
	\begin{tcn}[nsp](exem)<ctc>'t'{Exemples}
		\vspace{-25pt}
		\tcblistof[\subsubsection*]{exem}{\hspace*{4.8pt}}
	\end{tcn}
	\begin{tcn}[nsp](ror)<ctc>"hart"'t'{Points importants}
		\vspace{-25pt}
		\tcblistof[\subsubsection*]{ror}{\hspace*{4.8pt}}
	\end{tcn}
	\begin{tcn}[nsp](impo)<ctc>'t'{Erreurs communes}
		\vspace{-25pt}
		\tcblistof[\subsubsection*]{impo}{\hspace*{4.8pt}}
	\end{tcn}
\end{tcn}

\vspace*{\fill}

\newpage

%Tension et vitesse en fonction du milieu~: \url{https://phet.colorado.edu/sims/html/wave-on-a-string/latest/wave-on-a-string_fr.html}

\section{Introduction}
\subsection{Signal}
\begin{tcn}(defi)<lfnt>{\tiny Définition}
	On appelle \textbf{signal} une grandeur physique mesurable pouvant varier
	dans le temps et qui transporte une information.
\end{tcn}
\begin{tcn}(exem)<lfnt>{Exemples}
	\begin{itemize}
		\item Signal \textbf{sonore}~: voix, instrument de musique~;
		\item Signal sismique~;
		\item Signal électrique…
	\end{itemize}
\end{tcn}
La notion de signal \textbf{dépend de l'observation}. Par exemple, la découverte
des ondes radios était perturbée par le premier signal lumineux de l'Univers, le
fonds diffus cosmologique~: il baigne la totalité de l'Univers et est
fondamental dans la cosmologie, mais peut être parasite selon l'objectif.

\subsection{Perturbation}

\begin{tcn}(defi)<lfnt>{\tiny Définition}
	Une \textbf{perturbation} est une modification locale et temporaire des
	propriétés d'un milieu.
\end{tcn}
\begin{tcn}(exem)<lfnt>{Exemples}
	\begin{itemize}
		\item Jet d'un caillou dans un lac~;
		\item Séisme~;
		\item Déplacement de la membrane d'un haut-parleur…
	\end{itemize}
\end{tcn}
Une perturbation, quand elle est créée, se propage autour d'elle de proche en
proche~: chaque point impacté va subir des modifications temporaires similaires
à celle de la source. Après le passage de cette perturbation, chaque point
retrouve sa position initiale.

\subsection{Onde}
\begin{tcb}(defi){Onde}
	\psw{%
		On appelle \textbf{onde} la propagation d'une perturbation, dans un milieu
		matériel ou dans le vide.
	}
\end{tcb}

Certaines ondes ont besoin d'un milieu matériel pour se propager~: ce sont les
ondes \textbf{mécaniques}. Les ondes sismiques, les ondes dans la corde ou les
ondes sonores en sont des exemples. Certaines ondes peuvent se propager dans le
vide, comme les ondes \textbf{électromagnétiques}. Les infrarouges, la lumière
visible ou les micro-ondes sont des exemples d'ondes électromagnétiques.

\begin{tcb}[breakable](exem)<lftt>{Ondes}
	\begin{itemize}
		\item Lorsqu'on secoue l'extrémité d'une corde tendue,
		      les positions des différents points sont modifiées. Une fois l'onde
		      passée, les points retrouvent leur position initiale.
		\item Le caillou dans le lac forme des \textbf{rides qui s'éloignent} du point
		      d'impact, mais il n'y a pas de mouvement d'ensemble du
		      fluide.
		\item La membrane du haut-parleur, lors de son déplacement elle provoque
		      une brève \textbf{compression-dilatation} de l'air qui la touche.
		      Cette propagation se déplace ensuite dans l'air~: ce sont les ondes
		      sonores. Elles peuvent aussi se déplacer dans les liquides et dans
		      les solides.
	\end{itemize}
\end{tcb}

\subsection{Perturbation et propagation}

% La perturbation se propageant peut être soit parallèle, soit perpendiculaire à
% la direction de propagation. On distingue donc~:

\begin{tcb*}(defi){Transversale et longitudunale}
	\begin{itemize}
		\item[b]{Onde transversale}\footnote{\url{https://phyanim.sciences.univ-nantes.fr/Ondes/general/onde_transversale.php}}~:
		      \psw{%
			      la perturbation est perpendiculaire à la direction de propagation~;
		      }
		\item[b]{Onde longitudunale}\footnote{\url{https://phyanim.sciences.univ-nantes.fr/Ondes/general/onde_longitudinale.php}}~:
		      \psw{%
			      la perturbation est parallèle à la direction de propagation.
		      }
	\end{itemize}
\end{tcb*}

\begin{tcb}(exem)<lftt>'l'{Transversales et longitudunales}
	\begin{minipage}{0.50\linewidth}
		\begin{itemize}
			\item[b]{Longitudinales}~:
			      \psw{%
				      \begin{itemize}
					      \item Certaines ondes sismiques~;
					      \item Contraction-élongation d'un ressort.
				      \end{itemize}
			      }
		\end{itemize}
	\end{minipage}
	\hfill
	\begin{minipage}{0.48\linewidth}
		\begin{itemize}
			\item[b]{Transversales}~:
			      \psw{%
				      \begin{itemize}
					      \item Mouvement d'une corde secouée~;
					      \item Vagues sur l'eau.
				      \end{itemize}
			      }
		\end{itemize}
	\end{minipage}
	\vspace{-15pt}
\end{tcb}

\section{Onde progressive à une dimension}
\subsection{Définition}
\begin{tcb*}(defi){Onde progressive à une dimension}
	\begin{itemize}
		\item[b]{Progressive}~:
		      \psw{%
			      sa propagation ne se fait que \textbf{dans un seul sens depuis sa source}.
			      À un instant ultérieur, on retrouve la perturbation \textit{à l'identique
				      plus loin}.
		      }
		\item[b]{À une dimension}~:
		      \begin{itemize}
			      \item
			            \psw{%
				            une onde qui se propage dans un milieu matériel à une dimension~;
			            }
			      \item
			            \psw{%
				            ou une onde qui se propage dans un milieu matériel à deux ou trois
				            dimensions, avec une direction de propagation unique.
			            }
		      \end{itemize}
	\end{itemize}
\end{tcb*}

\begin{tcb}(exem)<lftt>'l'{Ondes et dimensions}
	\begin{itemize}
		\item[b]{1D}~:
		      \psw{%
			      onde le long d'une corde, compression le long d'un ressort~;
		      }
		\item[b]{2D}~:
		      \psw{%
			      vagues sur l'eau~;
		      }
		\item[b]{3D}~:
		      \psw{%
			      son, lumière.
		      }
	\end{itemize}
	\vspace{-15pt}
\end{tcb}

\subsection{Représentation spatiale et célérité des ondes}

\begin{tcb}[sidebyside, lefthand ratio=.45](ror){Représentation spatiale}
	Dans une représentation \textbf{spatiale}, on regarde à un \textbf{temps
		fixé} la perturbation dans \textbf{tout l'espace}~; on parle également de
	représentation \textbf{photographique}%
	\footnote{\url{https://phyanim.sciences.univ-nantes.fr/Ondes/general/retard.php}}.
	\tcblower
	\begin{center}
		\sswitch{
			\includegraphics[width=\linewidth, draft=true]{rep_spa}
		}{
			\includegraphics[width=\linewidth]{rep_spa}
		}
		\vspace{-15pt}
		\captionof{figure}{Exemple représentation spatiale}
	\end{center}
\end{tcb}

Lorsqu'une onde se propage, on peut définir une \textbf{vitesse de propagation
	de la perturbation}. Pour la distinguer de la vitesse d'un point matériel, on
emploi plutôt le terme \textbf{célérité}. Par convention, celle-ci est toujours
positive.
\smallbreak
\noindent
\begin{minipage}[c]{.65\linewidth}
	\begin{tcb}(defi){Célérité}
		La célérité $c$ d'une onde est le quotient de la distance $d$ parcourue
		par la perturbation, sur l'intervalle de temps $\Delta t$ que dure ce
		parcours~:
		\psw{%
			\[\boxed{c = \frac{d}{\Delta t}}\]
		}
		\vspace{-15pt}
	\end{tcb}
\end{minipage}
\noindent
\begin{minipage}[c]{.45\linewidth}
	\begin{tcb}(exem)<lftt>'r'{Célérité}
		Sur le schéma précédent,
		\psw{%
			\[
				\boxed{c = \frac{x_2-x_1}{t_2-t_1}}
			\]
		}%
		\vspace{-15pt}
	\end{tcb}
\end{minipage}
\smallbreak

% \begin{tcbraster}[raster equal height=rows, raster columns=8]
% 	\begin{tcb}[raster multicolumn=5](defi){Célérité}
% 		La célérité $c$ d'une onde est le quotient de la distance $d$ parcourue par la
% 		perturbation, sur l'intervalle de temps $\Delta t$ que dure ce parcours~:
% 		\psw{%
% 			\[\boxed{c = \frac{d}{\Delta t}}\]
% 		}
% 		\vspace{-15pt}
% 	\end{tcb}
% 	\begin{tcb}[raster multicolumn=3](exem)<lftt>'r'{Célérité}
% 		Sur le schéma précédent,
% 		\psw{%
% 			\[
% 				\boxed{c = \frac{x_2-x_1}{t_2-t_1}}
% 			\]
% 		}%
% 		\vspace{-15pt}
% 	\end{tcb}
% \end{tcbraster}

En première approximation, la célérité ne dépend pas de la perturbation mais
seulement de la nature et des propriétés du \textbf{milieu}.
\begin{tcb*}(odgr)<lftt>{}
	\begin{center}
		\captionof{table}{Ordres de grandeur de célérité à connaître}
		\label{tab:ctoknow}
		\begin{tabularx}{.6\linewidth}{lY}
			\toprule
			Signal                   & Célérité
			\\\midrule
			Ondes électromagnétiques & \psw{$\SI{3.0e8}{m.s^{-1}}$}
			\\
			Son dans l'air (\SI{20}{\degreeCelsius},
			\SI{1}{bar})             & \psw{$\approx \SI{340}{m.s^{-1}}$}
			\\
			Son dans les métaux      & \psw{quelques $\si{km.s^{-1}}$}
			\\
			Son dans l'eau           & \psw{$\SI{1.5}{km.s^{-1}}$}
			\\\bottomrule
		\end{tabularx}
	\end{center}
\end{tcb*}


\begin{tcb*}(appl)<lftt>{Vague en représenta$^\circ$ spatiale}
	% Un mascaret est une vague solitaire remontant un fleuve au voisinage de son
	% estuaire, et provoqué par une interaction entre son écoulement et la marée
	% montante.
	On considère ici une vague solitaire qui se déplace à la vitesse $\boxed{c =
			\SI{18}{km.h^{-1}}}$ le long d'un fleuve rectiligne, et on définit un axe
	$(Ox)$ dans la direction du sens de sa propagation.
	\smallbreak
	À l'instant $t=0$, le profil du niveau de l'eau du fleuve a l'allure
	suivante~:
	\begin{center}
		\includegraphics[width=0.8\linewidth]{rep_spa-masc_a}
	\end{center}
	Faire un schéma du profil du fleuve à $\tau = \SI{1}{min}$ en supposant que
	l'onde se propage sans déformation.
	\tcblower
	\psw{%
		La queue de la vague est à $x_{q,0} = \SI{50}{m}$. À $\tau =
			\SI{1}{min}$, elle est en $x_{q,1}$. Par définition de la célérité,
		\begin{gather*}
			\frac{x_{q,1}-x_{q,0}}{\tau-0} = c
			\Lra
			x_{q,1} = x_{q,0} + c\tau = \SI{350}{m}
		\end{gather*}
		On procèderait de même pour repérer le haut de la vague et sa tête~: en
		réalité, chaque point du mascaret se déplace de $c\tau = \SI{300}{m}$
		vers la droite.}
	\begin{center}
		\sswitch{
			\includegraphics[width=0.8\linewidth, draft=true]{rep_spa-masc_b}
		}{
			\includegraphics[width=0.8\linewidth]{rep_spa-masc_b}
		}
		\captionof{figure}{Vague solitaire à $\tau = \SI{1}{min}$}
	\end{center}
\end{tcb*}

\subsection{Représentation temporelle et retard}

\begin{tcb}(ror){Représentation temporelle}
	Dans une représentation \textbf{temporelle}, on regarde à un
	\textbf{endroit fixé} la perturbation \textbf{sur sa durée}. Voir cette
	animation\footnote{\url{https://phyanim.sciences.univ-nantes.fr/Ondes/general/evolution_temporelle.php}}.
\end{tcb}

\begin{tcb}(defi){Retard d'une onde}
	La grandeur $\tau$ est le \textbf{retard} du point M' par rapport au point
	M~:
	\psw{%
		\[\tau = \frac{\rm MM'}{c}\]
	}%
	avec $c$ la célérité de l'onde.
\end{tcb}

En effet, en créant à l'instant $t = 0$ une déformation à un endroit M, cette
perturbation se propage le long de l'axe corde avec une célérité $c$. Elle
parvient donc en un point M' après le temps $\tau$.

\begin{tcb*}(appl)<lftt>{Vague en représenta$^\circ$ temporelle}
	On reprend l'exemple de la vague précédente.
	\begin{enumerate}
		\item À quel instant la vague arrive-t-elle au point d'abscisse $x_1 =
			      \SI{2.2}{km}$~?
		      \smallbreak
		      \psw{%
			      À $t=0$, la tête de la vague est à $x_0 = \SI{400}{m}$. Elle
			      arrive en $x_1$ avec un retard~:
			      \[t = \frac{x_1-x_0}{c} = \SI{6}{min}\]
		      }
		      \vspace{-15pt}
		\item Un détecteur fixe, enregistrant la hauteur du fleuve en fonction
		      du temps, est placé à l'abscisse $x_d = \SI{1.6}{km}$. Dessiner
		      l'allure des variations $y(x_d,t)$ en fonction du temps à cette
		      abscisse.
		      \smallbreak
		      \psw{%
			      Les différents éléments de la vagues arrivent avec les retards~:
			      \[
				      \tau_{\text{tête}} = \frac{x_d - x_{t,0}}{c} = \SI{4}{min}
				      \quad
				      \tau_{\text{haut}} = \frac{x_h - x_{h,0}}{c} =
				      \SI{4}{min}\SI{10}{s}
				      \quad
				      \tau_{\text{queue}} = \frac{x_q - x_{q,0}}{c} =
				      \SI{5}{min}\SI{10}{s}
			      \]
		      }
	\end{enumerate}
	\begin{center}
		\sswitch{
			\includegraphics[width=0.8\linewidth, draft=true]{rep_temp_masc}
		}{
			\includegraphics[width=0.8\linewidth]{rep_temp_masc}
		}
		\captionof{figure}{Vague solitaire en représentation temporelle.}
	\end{center}
\end{tcb*}

\subsection{Lien entre les représentations}
Nous avons vu deux représentations graphiques différentes, une selon l'espace et
une selon le temps. En réalité, le signal d'une onde est une fonction de
\textbf{deux} variables~:
\[y(x,t)\]
Pour obtenir l'une au l'autre des représentations, on fixe l'une des variables.
Une animation sur les représentations temporelles et spatiales est disponible au
lien suivant~: \url{https://www.geogebra.org/m/RkmRF9M6}

% \subsection{Formes mathématiques des représentations}
% \subsubsection{À partir de la représentation spatiale}
\begin{tcb*}(demo){Lien entre représentations}
	\tcbsubtitle{\fatbox{\textbf{Depuis représentation spatiale}}}
	L'onde observée à $t=0$ se déplace vers la droite. À l'instant $t$, elle est
	décalée vers la droite de $\setlength{\fboxsep}{3mm} \boxed{\psw{\delta =
				ct}}$~: la valeur de $y(x,t)$ en $x$ et à l'instant $t$ était en $x-ct$
	à l'instant $t=0$, soit~:
	\psw{%
		\[
			y(x,t) = y(x-ct,0)
		\]
	}%
	On note alors $f(x) = y(x,0)$~: c'est la représentation spatiale de l'onde à
	$t=0$. On a alors
	\psw{%
		\[\boxed{
				y(x,t) = f(x-ct)
			}\]
	}%
	\vspace{-15pt}
	\tcblower
	\tcbsubtitle{\fatbox{\textbf{Depuis représentation temporelle}}}
	Lorsqu'une onde se propage sans atténuation ni déformation, les valeurs
	observées en $x = 0$ au cours du temps sont aussi observées en $x > 0$ mais
	avec un retard $\setlength{\fboxsep}{3mm}\boxed{\psw{\tau = x/c}}$ lié à la
	propagation. La valeur de $y(x, t)$ en $x$ à l'instant $t$ était en $x = 0$
	plus tôt, à l'instant $t - x/c$. Ainsi,
	\psw{%
		\[
			y(x,t) = y\left(0,t-\frac{x}{c}\right)
		\]
	}%
	La fonction $y(0,t)$ est la hauteur de la perturbation en $x=0$ à l'instant
	$t$~: c'est la perturbation imposée par la source. On la note alors $g(t) =
		y(0,t)$~: c'est la représentation temporelle de l'onde à $x=0$. On a alors
	\psw{%
		\[\boxed{
				y(x,t) = g\left(t-\frac{x}{c}\right)
			}\]
	}%
\end{tcb*}

% \subsubsection{À partir de la représentation temporelle}

\begin{tcb}[sidebyside](ror){Lien entre représentations}
	La représentation \textbf{spatiale en $\mathbf{t_0}$} est le graphique de la
	fonction~:
	\psw{%
		\[y~:~x \mapsto y(x,t_0) = f(x-ct_0) = g\left(t_0 - \frac{x}{c}\right)\]
	}%
	\vspace{-15pt}
	\tcblower
	La représentation \textbf{temporelle en} $\mathbf{x_0}$ est le graphique de la
	fonction~:
	\psw{%
		\[y~:~t \mapsto y(x_0,t) = f(x_0-ct) = g\left(t - \frac{x_0}{c}\right)\]
	}%
	\vspace{-15pt}
\end{tcb}

\begin{tcb*}(impo){Vers la droite ou vers la gauche~?}
	Vous ferez bien attention, à défaut de travailler votre intuition pour
	comprendre que $f (x-ct)$ est une onde se propageant vers la droite, à ne pas
	penser «~signe moins donc vers la gauche~»~!
	%  Il faudrait refaire le raisonnement, et on arrive à~:
	\smallbreak
	\begin{isd}[interior hidden](impo)
		% \tcbsubtitle{\fatbox{\textbf{Vers la gauche}}}
		% TODO: Vérifier espacement
		\vspace{-15pt}
		\begin{gather*}
			\beforetext{\textcolor{impo}{\fatbox{\textbf{Vers la gauche}}}}
			\hspace{80pt}
			\psw{%
				f (x+ct) = g \left( t + \frac{x}{c} \right)
			}
		\end{gather*}
		\vspace{-25pt}
		\tcblower
		% \tcbsubtitle{\fatbox{\textbf{Vers la droite}}}
		\vspace{-15pt}
		\begin{gather*}
			\beforetext{\textcolor{impo}{\fatbox{\textbf{Vers la droite}}}}
			\hspace{80pt}
			\psw{%
				f (x-ct) = g \left( t - \frac{x}{c} \right)
			}
		\end{gather*}
		\vspace{-25pt}
	\end{isd}
\end{tcb*}

\section{Onde progressive sinusoïdale}

\begin{tcb}(defi){Onde progressive sinusoïdale}
	Une onde progressive est dite \textbf{sinusoïdale} si la source impose une
	\textbf{perturbation sinusoïdale}~:
	\psw{%
		\[
			\boxed{s (0,t) = g(t) = A \cos(\wt + \f_0)}
		\]
	}%
	\vspace{-15pt}
\end{tcb}

% Si on relie un haut-parleur à un GBF délivrant une tension sinusoïdale, on
% observe le mouvement périodique dont est animé la membrane de l'air, qui génère
% une perturbation périodique de l'air.

% L'exemple du diapason discuté chapitre E7 est également une onde progressive
% sinusoïdale~: lorsque l'on frappe le diapason, celui-ci vibre a une fréquence
% bien déterminée, qui se propage ensuite dans l'air pour parvenir à nos oreilles.

\subsection{Double périodicité spatiale et temporelle}
% \subsubsection{Observations sur l'animation \texttt{Geogebra}}
\begin{tcb}(obsv)<lftt>{Animation \texttt{Geogebra}}
	\begin{enumerate}
		\item \psw{%
			      Lorsque l'on impose une excitation temporelle sinusoïdale, la
			      représentation spatiale est aussi sinusoïdale.
		      }
		\item \psw{%
			      À célérité constante, lorsque la fréquence de l'excitation augmente
			      (la période diminue), la période spatiale diminue.
		      }
		\item \psw{%
			      À fréquence de l'excitation constante, si on augmente la célérité, la
			      période spatiale diminue.
		      }
	\end{enumerate}
\end{tcb}

\begin{tcb*}[sidebyside, sidebyside align=top](defi){Périodicités}
	\tcbsubtitle{\fatbox{\textbf{Périodicité temporelle}}}
	Si la perturbation créée en S est sinusoïdale avec une période $T$, alors
	l'onde en M l'est également (il n'y a qu'un retard entre les deux dû à la
	propagation).
	\tcblower
	\tcbsubtitle{\fatbox{\textbf{Périodicité spatiale}}}
	Au moment de l'émission du deuxième maximum, le premier maximum a déjà
	parcouru une distance $cT$. L'écart spatial entre deux maximum successifs est
	la période spatiale.
\end{tcb*}

\begin{tcb}(ror){Longueur d'onde d'une OPS}
	Une onde progressive sinusoïdale présente à la fois une périodicité spatiale,
	nommée \textbf{longueur d'onde} et notée $\lambda$, et une périodicité
	temporelle, notée $T$. Elles sont reliées par la relation
	\psw{%
		\[\boxed{\lambda = cT = \frac{c}{f}}\]
	}
	\vspace{-15pt}
	% avec $c$ la célérité de l'onde.
\end{tcb}

Cette relation est celle donnant les périodes spatiales et temporelles des ondes
électromagnétiques~:
\vspace{-25pt}
\begin{center}
	\includegraphics[width=.95\linewidth]{full_spectre}
\end{center}

\subsection{Expression mathématique de l'onde progressive sinusoïdale}

\begin{tcb*}(prop){Forme générale d'une OPS}
	L'expression générale d'une onde progressive sinusoïdale se propageant
	\textit{vers la droite} sans déformation ni atténuation est~:
	\psw{%
		\begin{gather*}
			\boxed{s(x,t) = A\cos(\wt - kx + \f_0)}
			%    \\\Lra
			% s(x,t) = A\cos \left( \frac{2\pi}{T}t - \frac{2\pi}{\lambda}x + \f
			% \right)
		\end{gather*}
	}
	\vspace{-10pt}
\end{tcb*}

\begin{tcb*}(defi){Vecteur d'onde}
	Comme pour la fréquence et la pulsation, on relie la longueur d'onde à une
	autre grandeur permettant une expression simple dans une fonction
	sinusoïdale~: le \textbf{vecteur d'onde} $k$, tel que
	\[
		\psw{\boxed{k = \frac{2\pi}{\lambda} = \frac{\w}{c}}
			\qavec
			k\quad\text{en}\quad\boxed{\si{rad.m^{-1}}}}
	\]
\end{tcb*}

\begin{tcb*}(impo){Faux-ami}
	Sous cette forme, le vecteur d'onde n'est pas un vecteur~!! C'est vraiment un
	vecteur quand on travaille en trois dimensions d'espace.
\end{tcb*}

\begin{tcb*}(demo){Forme générale d'une OPS}
	On s'intéresse à un mouvement vers la droite. Par définition, la perturbation
	$g(t)$ imposée en $x=0$ est un signal sinusoïdal~:
	\psw{%
		\begin{align*}
			g(t) = s(0,t) & = A\cos(\wt+\f_0)
			\\\Ra
			s(x,t)~       &
			= A\cos(\w\left(t - \frac{x}{c}\right) +\f_0)
			\\\Lra
			s(x,t)~       &
			= A\cos(\wt - \frac{\w}{c}x +\f_0)
			\\\Lra
			s(x,t)~       &
			= A\cos(\wt - \frac{2\pi}{cT}x +\f_0)
			\qed
		\end{align*}
	}%
	\vspace{-15pt}
\end{tcb*}

\begin{tcb*}(appl)<lftt>{Double périodicité d'une OPS}
	Soit un signal $s (x,t)$ double-périodique. Montrer que $s (x+\lambda,t) = s
		(x,t)$.
	\tcblower
	% \vspace{-15pt}
	\begin{isd}[sidebyside align=top]
		\vspace{-15pt}
		\psw{%
			\begin{align*}
				s(x+\lambda,t) & = A\cos(\wt - k(x+\lambda) + \f_0)
				\\\Lra
				s(x+\lambda,t) & = A\cos\left(\wt - kx - \frac{2\pi}{\lambda}\lambda +
				\f_0\right)
			\end{align*}
		}
		\vspace{-15pt}
		\tcblower
		\vspace{-15pt}
		\psw{%
			\begin{align*}
				\Lra
				s(x+\lambda,t) & = A\cos(\wt - kx -2\pi +\f_0)
				\\\Lra
				s(x+\lambda,t) & = A\cos(\wt - kx +\f_0)
				\\\Lra
				\Aboxed{%
				s(x+\lambda,t) & = s(x,t)
				}%
			\end{align*}
		}
		\vspace{-15pt}
	\end{isd}
	\vspace{-20pt}
\end{tcb*}

\subsection{Vitesse de phase}

Soit une onde progressive sinusoïdale. La \textbf{phase} de l'onde est, par
définition, le terme à l'intérieur de la fonction~: $\wt-kx+\f$. Cette phase
varie spatialement \textit{et} temporellement, de manières corrélées.

\begin{tcb}[sidebyside, sidebyside align=top, righthand ratio=.2](defi){Vitesse de phase}
	Si on trouve une phase mesurée en $x_1$ à l'instant $t_1$, le signal aura la
	même phase en $x_2$ à un instant $t_2$ donnés par la \textbf{vitesse de
		phase}, notée $v_\f$, telle que~:
	\psw{\[\boxed{v_\f = \frac{x_2-x_1}{t_2-t_1}}\]}
	\tcblower
	\tcbsubtitle{\fatbox{\textbf{Unité}}}
	\psw{%
		\[
			\si{m.s^{-1}}
		\]
	}%
	\vspace{-15pt}
\end{tcb}

\begin{tcb*}(prop){Vitesse de phase}
	\vspace{-15pt}
	\begin{gather*}
		\beforetext{Pour une OPS, on a}
		\psw{%
			\boxed{v_{\f} = \frac{\w}{k} = c}
		}%
	\end{gather*}
	\vspace{-15pt}
\end{tcb*}

% \vspace{-10pt}
\begin{tcb*}(demo){Vitesse de phase}
	\vspace{-15pt}
	\psw{%
		\begin{gather*}
			\wt_2-kx_2+\f = \wt_1-kx_1+\f\\
			\Lra
			\w(t_2-t_1) = k(x_2-x_1)\\
			\Lra
			\boxed{v_\f = \frac{x_2-x_1}{t_2-t_1} = \frac{\w}{k}}
		\end{gather*}
	}
	\vspace{-10pt}
\end{tcb*}

\section{Milieux dispersifs}

\begin{tcb*}(defi){Milieu dispersif}
	Un milieu est dit \textbf{dispersif} si
	\psw{la célérité $c$ dépend de la fréquence ou de la longueur d'onde.}
	\bigbreak
	Si c'est le cas, les différentes composantes spectrales d'un signal ne vont
	pas à la même vitesse, et donc le signal peut se déformer lors de la
	propagation.
\end{tcb*}

\begin{tcb}[breakable](exem)<lftt>{Milieux dispersifs ou non}
	\tcbsubtitle{\fatbox{Propagation non-dispersive}}
	\begin{itemize}
		\item
		      \psw{%
			      Propagation des ondes acoustiques dans un fluide. La célérité
			      est~:
			      \[c = \frac{1}{\sqrt{\rho_0\chi_0}}\]
			      avec $\rho_0$ la masse volumique du fluide au repos et $\chi_0$ sa
			      compressibilité.
		      }
		\item
		      \psw{%
			      Propagation des ondes électromagnétiques dans le vide~:
			      \[c = \SI{299792458}{m.s^{-1}}\]
			      C'est une des constantes fondamentales de la physique.
		      }
	\end{itemize}
	\tcblower
	\tcbsubtitle{\fatbox{Propagation dispersive}}

	\begin{itemize}
		\item
		      \psw{%
			      Propagation des ondes à la surface de l'eau. On a
			      \[\w^2 = gk
				      \qsoit
				      v_\f(\lambda) = \sqrt{\frac{g}{k}} = \sqrt{\frac{\lambda g}{2\pi}}
			      \]
			      Ainsi, la vitesse de phase dépend de $k$, et donc de la longueur
			      d'onde.
		      }%
		\item
		      \psw{%
			      Propagation des ondes électromagnétiques dans le verre~:
			      \[v_\f(\lambda) = \frac{c}{n(\lambda)}\]
			      avec $n(\lambda) = A + \frac{B}{\lambda^2}$. C'est la dispersion qui
			      cause la décomposition spectrale de la lumière par un prisme.
		      }
	\end{itemize}
	\vspace{-10pt}
\end{tcb}

% \sidecaptionvpos{figure}{c}
\begin{figure}[h!]
	\centering
	\includegraphics[width=\linewidth]{dispersion}
	%\captionsetup{justification=centering}
	\caption[Dispersion d'une onde à la surface de l'eau]{Propagation dispersive
		d'une onde à la surface de l'eau. On observe nettement que les
		composantes sinusoïdales de hautes fréquences se propagent avec une
		moins grande vitesse que les composantes de basses fréquences. En
		ordonnée, l'unité de la hauteur d'eau est arbitraire.}
	\label{fig:disp_eau}
\end{figure}

\end{document}
