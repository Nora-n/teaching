\documentclass[../../main/main.tex]{subfiles}
\graphicspath{{./figures/}}

\dominitoc
\faketableofcontents

% \renewcommand{\mtcSfont}{\small\bfseries}
% \renewcommand{\mtcSSfont}{\footnotesize}
\mtcsettitle{minitoc}{}
\mtcsetrules{*}{off}

\makeatletter
\renewcommand{\@chapapp}{Mécanique -- chapitre}
\renewcommand{\chaplett}{M}
\makeatother

% \toggletrue{student}
% \toggletrue{corrige}
% \renewcommand{\mycol}{black}
% \renewcommand{\mycol}{gray}

\hfuzz=5.002pt

\begin{document}
\setcounter{chapter}{2}

\settype{book}
\settype{prof}
\settype{stud}

\chapter{Mouvements courbes}

\vspace*{\fill}

\begin{tcn}(appl)<ctc>"somm"'t'{Sommaire}
	\let\item\olditem
	\vspace{-15pt}
	\minitoc
	\vspace{-25pt}
\end{tcn}

\begin{tcn}[sidebyside, fontupper=\small, fontlower=\small]
	(appl)<ctb>"how"'t'{Capacités exigibles}
	\begin{itemize}[label=\rcheck]
		\item Identifier les degrés de liberté d'un mouvement. Choisir un système
		      de coordonnées adapté au problème.

		      % \item Systèmes de coordonnées cartésiennes, cylindriques et sphériques.
		      %
		\item Vitesse et accélération dans le repère de Frenet pour une
		      trajectoire plane.

		\item Coordonnées cylindriques~: exprimer à partir d'un schéma le
		      déplacement élémentaire, construire le trièdre local associé et en
		      déduire géométriquement les composantes du vecteur vitesse.

		\item Établir les expressions des composantes des vecteurs position,
		      déplacement élémentaire, vitesse et accélération en coordonnées
		      cylindriques.
	\end{itemize}
	\tcblower
	\begin{itemize}[label=\rcheck]
		\item Mouvement circulaire uniforme et non uniforme~: exprimer les
		      composantes du vecteur position, du vecteur vitesse et du vecteur
		      accélération en coordonnées polaires planes.

		\item Exploiter les liens entre les composantes du vecteur accélération,
		      la courbure de la trajectoire, la norme du vecteur vitesse et sa
		      variation temporelle.

		\item Établir l’équation du mouvement du pendule simple. Justifier
		      l’analogie avec l'oscillateur harmonique dans le cadre de
		      l'approximation linéaire.
	\end{itemize}
\end{tcn}

\vspace*{\fill}
\newpage
\vspace*{\fill}

%\vspace{-15pt}
\begin{tcn}[%
		sidebyside, fontupper=\small, fontlower=\small
	](appl)<ctb>"chek"'t'{L'essentiel}
	\tce{defi}
	% \tce{rapp}
	% \tce{loi}
	\tce{prop}
	\tce{demo}
	% \tce{theo}
	% \tce{prev}
	% \tce{coro}
	% \tce{inte}
	% \tce{impl}
	% \tce{tool}
	% \tce{nota}
	% \tce{appl}
	% \tce{rema}
	% \tce{exem}
	% \tce{ror}
	% \tce{impo}
	\tcblower
	% \tce{defi}
	% \tce{rapp}
	% \tce{prop}
	% \tce{theo}
	% \tce{loi}
	% \tce{coro}
	% \tce{demo}
	% \tce{inte}
	% \tce{impl}
	% \tce{nota}
	\tce{appl}
	\tce{rema}
	\tce{exem}
	\tce{tool}
	% \tce{ror}
	\tce{impo}
\end{tcn}

\vspace*{\fill}

\newpage

\section{Mouvement courbe dans un plan}
\subsection{Position en coordonnées polaires}

\begin{tcb*}[sidebyside, righthand ratio=.37](defi){Coordonnées polaires}
	Le repère polaire est constitué d'une origine O autour de laquelle sont
	définis deux vecteurs $\ur$ et $\ut$ tels que~:
	\begin{itemize}
		\item \psw{$\ur$ dans la direction $\OM$}
		\item \psw{$\ut \perp \ur$ dans le sens direct}
		\item[m][15] \psw{%
			      \[
				      \boxed{\OM(t) = r(t)\ur}
				      \qet
				      \boxed{\norm{\OM}(t) = r(t)}
			      \]
		      }%
		      \vspace{-15pt}
	\end{itemize}
	\begin{center}
		\fatbox{\textbf{$\ur$ et $\ut$ dépendent de $\th(t)$ donc du temps}}
	\end{center}
	\tcblower
	\begin{center}
		\sswitch{%
			\includegraphics[width=\linewidth]{pos_pol_stud}
		}{%
			\includegraphics[width=\linewidth]{pos_pol_prof}
		}%
		\vspace{-15pt}
		\captionof{figure}{Polaires}
	\end{center}
\end{tcb*}

\begin{tcb}(impo)<lftt>{Variables vs.\ coordonnées}
	Il faut opérer la distinction entre les \textbf{variables} servant à repérer
	le point et les \textbf{coordonnées} dans la base de projection. Ici, les
	variables sont $r(t)$ et $\th(t)$, mais dans la base $(\ur,\ut)$, on a
	\[
		\OM(t) = \mqty(r(t)\\0) = r(t)\ur
		\qMath{\textbf{ET PAS}}
		\dcancel{%
		\OM(t) = \mqty(r(t)\\\th(t)) = r(t)\ur +
		\underbracket[1pt]{\th(t)\ut}_{\mathclap{\text{pas homogène~!}}}
		}
	\]
	\vspace{-15pt}
\end{tcb}

\begin{tcb*}(prop){Lien polaires/cartésiennes}
	Les vecteurs $\ur$ et $\ut$ variables se décomposent sur $\ux$ et $\uy$
	fixes tels que
	\[
		\psw{\boxed{\ur = \cos(\th(t))\ux + \sin(\th(t))\uy}}
		\qet
		\psw{\boxed{\ur = -\sin(\th(t))\ux + \cos(\th(t))\uy}}
	\]
	d'où en cartésiennes pour un point M~:
	\[
		\psw{%
			x(t) = r(t)\cos(\th(t))
		}%
		\qet
		\psw{%
			y(t) = r(t)\sin(\th(t))
		}%
		\qso
		\psw{%
			\norm{\OM}(t) = r(t) = \sqrt{x(t)^2 + y(t)^2}
		}%
	\]
\end{tcb*}

\begin{tcb}(demo)<lftt>{Lien polaires/cartésiennes}
	On projette les vecteurs de la base polaire sur la base cartésienne en
	appliquer la méthode de vraisemblance ou par définition du produit
	scalaire, d'où la propriété. On a alors~:
	\psw{%
		\begin{gather*}
			\OM(t) = r(t)\ur
			\Lra
			\OM(t) =
			\underbracket[1pt]{r(t)\cos(\th(t))}_{=x(t)}\ux +
			\underbracket[1pt]{r(t)\sin(\th(t))}_{=y(t)}\uy
			\\\Ra
			\norm{\OM}(t) =
			\sqrt{x(t)^2 + y(t)^2} =
			\sqrt{r(t)^2\underbracket[1pt]{\pa{\cos^2(\th(t)) + \sin^2(\th(t))}}_{=1}}
			= r(t) \qed
		\end{gather*}
	}%
	\vspace{-25pt}
\end{tcb}

\vspace{-15pt}
\subsection{Variation temporelle des vecteurs de base}
\begin{tcb*}[breakable](tool){Dérivée composée en physique}
	En physique, on pense les dérivées comme des fractions. Ainsi, on peut traiter
	la dérivée d'une composition en faisant intervenir d'autres dérivées par une
	écriture fractionnaire. Par exemple~:

	\psw{%
		\begin{gather*}
			\dv{t}\/(\cos(\theta(t))) =
			\psw[orchid]{\overbracket[1pt]{\psw{\dv{\th}{t}}}^{\tikzmark{MT}}}
			\psw[cornflowerblue]{\overbracket[1pt]{\psw{\dv{\th}(\cos(\theta(t)))}}^{\tikzmark{MC}}}
			=
			\psw[orchid]{\overbracket[1pt]{\psw{\tp(t)}}^{\tikzmark{NT}}}
			\psw[cornflowerblue]{\overbracket[1pt]{\psw{(-\sin(\theta(t)))}}^{\tikzmark{NC}}}
			= - \tp(t) \sin(\th(t))
		\end{gather*}
	}%
	\tikz[remember picture, overlay]
	\draw[-stealth, thick, color=\sswitch{white}{orchid}]
	(pic cs:MT) --++ (0,6pt) -|
	(pic cs:NT)
	; \tikz[remember picture, overlay]
	\draw[-stealth, thick, color=\sswitch{white}{cornflowerblue}]
	(pic cs:MC) --++ (0,3pt) -|
	(pic cs:NC)
	;
	\vspace{-15pt}
\end{tcb*}

\begin{tcb*}(prop){Dérivées de $\protect\ur$ et $\protect\ut$}
	La variation temporelle des vecteurs de la base polaire est~:
	\[
		\psw{%
			\boxed{\dv{\ur}{t} = \tp(t)\ut}
		}%
		\qet
		\psw{%
			\boxed{\dv{\ut}{t} = -\tp(t)\ur}
		}%
	\]
	\vspace{-15pt}
\end{tcb*}

\begin{tcb}[breakable](demo)<lftt>{Dérivées de $\protect\ur$ et $\protect\ut$}
	% On peut travailler géométriquement ou par le calcul, en repassant en
	% coordonnées cartésiennes.
	\tcbsubtitle{\fatbox{\textbf{Géométriquement}}}
	\begin{isd}[interior hidden, righthand ratio=.3](demo)
		On représente les deux vecteurs après un petit temps $\dd{t}$, c'est-à-dire
		augmentés d'un angle $\dd{\th}$~:
		\begin{align*}
			\psw{%
				\dd{\ur} = \underbracket[1pt]{\norm{\ur}}_{=1}\cdot \dd{\th} \ut
			}%
			\qeta
			\psw{%
				\dd{\ut} = \underbracket[1pt]{\norm{\ut}}_{=1}\cdot \dd{\th} (-\ur)
			}%
			\\\beforetext{Soit}
			\psw{%
				\dv{\ur}{t} = \dv{\th}{t}\ut
			}%
			\qeta
			\psw{%
				\dv{\ut}{t} = -\dv{\th}{t}\ur
			}%
			\qed
		\end{align*}
		\tcblower
		\begin{center}
			\sswitch{%
				\includegraphics[width=\linewidth]{pol_dur_stud}
			}{%
				\includegraphics[width=\linewidth]{pol_dur_prof}
			}%
			\vspace{-15pt}
			\captionsetup{justification=centering}
			\captionof{figure}{\\$\dd\protect\ur$ et $\dd\protect\ut$}
		\end{center}
	\end{isd}
	\tcblower
	\tcbsubtitle{\fatbox{\textbf{Mathématiquement}}}
	On part des décompositions dans la base cartésienne et on dérive~:
	\smallbreak
	\begin{isd}[interior hidden](demo)
		\tcbsubtitle{\fatbox{\!$\ur$}}
		\vspace{-15pt}
		\psw{%
			\small
			\begin{DispWithArrows*}[fleqn, mathindent=0pt]
				\ur
				& =
				\cos(\th)\ux + \sin(\th)\uy
				\CArrow{$\dv{t}\/(\cdot)$}
				\\\Lra
				\dv{\ur}{t}
				& =
				\dv{\cos(\th)}{t}\ux + \dv{\sin(\th)}{t}\uy
				\Arrow{Composée}
				\\\Lra
				\dv{\ur}{t}
				& =
				-\tp\sin(\th)\ux + \tp\cos(\th)\uy
				\Arrow{Factorisa$^\circ$}
				\\\Lra
				\dv{\ur}{t}
				& =
				\tp \underbracket[1pt]{\left(-\sin(\th)\ux + \cos(\th)\uy\right)}_{= \ut}
				\Arrow{Identifica$^\circ$}
				\\\Lra
				\dv{\ur}{t}
				& =
				\tp\ut
				\qed
			\end{DispWithArrows*}
		}%
		\vspace{-15pt}
		\tcblower
		\tcbsubtitle{\fatbox{\!$\ut$}}
		\vspace{-15pt}
		\psw{%
			\small
			\begin{DispWithArrows*}[fleqn, mathindent=0pt]
				\ut
				& =
				-\sin(\th)\ux + \cos(\th)\uy
				\CArrow{$\dv{t}\/(\cdot)$}
				\\\Lra
				\dv{\ut}{t}
				& =
				\dv{(-\sin(\th))}{t}\ux + \dv{\cos(\th)}{t}\uy
				\Arrow{Composée}
				\\\Lra
				\dv{\ut}{t}
				& =
				-\tp\cos(\th)\ux - \tp\sin(\th)\uy
				\Arrow{Facto.}
				\\\Lra
				\psw{\dv{\ut}{t}}
				& =
				-\tp \underbracket[1pt]{\left(\cos(\th)\ux + \sin(\th)\uy\right)}_{= \ur}
				\Arrow{Identif.}
				\\\Lra
				\dv{\ut}{t}
				& =
				-\tp\ur
				\qed
			\end{DispWithArrows*}
		}%
		\vspace{-15pt}
	\end{isd}
\end{tcb}

\subsection{Déplacement élémentaire en polaires}
\begin{tcb*}(prop){Déplacement élémentaire polaire}
	En coordonnées polaires, le déplacement élémentaire s'exprime
	\psw{\[\boxed{\dd\OM = \dd r\ur + r(t)\dd\th\ut}\]}
	\vspace{-15pt}
\end{tcb*}

% On a toujours $\dd\OM = \OM(t+\dt) - \OM(t)$. On trouve son expression
% géométriquement~:

\begin{tcb}[sidebyside, righthand ratio=.40](demo)<lftt>{Déplacement élémentaire polaire}
	On trouve la composante de $\dd{\OM}$ sur $\ur$ en \xul{fixant $\th$} et on
	\xul{incrémente la variable $r$ de $\dd{r}$}.
	\begin{center}
		La distance ainsi obtenue est \xul{\psw{$\dd{r}$ sur $\ur$.}}
	\end{center}
	\bigbreak
	On trouve la composante de $\dd{\OM}$ sur $\ut$ en \xul{fixant $r$} et on
	\xul{incrémente la variable $\th$ de $\dd{\th}$}.
	\begin{center}
		La distance ainsi obtenue est \xul{\psw{$r(t)\dd{\th}$ sur $\ut$.}}
	\end{center}
	\tcblower
	\begin{center}
		\sswitch{%
			\includegraphics[width=\linewidth]{pol_dom_stud}
		}{%
			\includegraphics[width=\linewidth]{pol_dom_prof}
		}%
		\vspace{-15pt}
		% \captionsetup{justification=centering}
		\captionof{figure}{$\dd{\protect\OM}$ polaire}
	\end{center}
\end{tcb}

\subsection{Vitesse en coordonnées polaires}

\begin{tcb*}(prop){Vitesse en polaires}
	La vitesse en coordonnées polaires s'écrit
	\psw{\[\boxed{\vf(t) = \rp(t)\ur + r(t)\tp(t)\ut}\]}
	\vspace{-15pt}
\end{tcb*}

\begin{tcb}(demo)<lftt>{Vitesse en polaires}
	Ici aussi, il y a deux manières d'obtenir l'expression de la vitesse.
	\tcbsubtitle{\fatbox{\textbf{Dérivée}}}
	\vspace{-15pt}
	\psw{%
		\begin{gather*}
			\vf(t) =
			\dv{\OM}{t} =
			\dv{(r(t)\ur)}{t}
			\Lra
			\vf(t) = \rp(t)\ur + r(t)\dv{\ur}{t}
			\Lra
			\vf(t) = \rp(t)\ur + r(t)\tp(t)\ut
			\qed
		\end{gather*}
	}%
	\vspace{-15pt}
	\tcblower
	\tcbsubtitle{\fatbox{\textbf{Rapport}}}
	\psw{%
		\[
			\vf(t) =
			\dv{\OM}{t} =
			\frac{\dd{r}\ur + r(t) \dd{\th}\ut}{\dd{t}} =
			\dv{r}{t} \ur + r(t) \dv{\th}{t}\ut
			\qed
		\]
	}%
	\vspace{-15pt}
\end{tcb}

\subsection{Accélération}
\begin{tcb*}(demo)<lftt>{Accélération en polaires}
	Par définition,
	\psw{%
		\begin{gather*}
			\af = \dv{\vf}{t} =
			\dv{\tikzmark{DV}}{t}
			\left(
			\rp\tikzmark{RP}(t)\ur\tikzmark{UR} +
			r\tikzmark{R}(t)\tp\tikzmark{TP}(t)\ut\tikzmark{UT}
			\right)
			\\
			\Lra
			\af =
			\psw[orchid]{\rpp(t)}\ur +
			\rp(t) \underbracket[1pt]{\psw[cornflowerblue]{\dv{\ur}{t}}}_{= \tp(t)\ut} +
			\psw[limegreen]{\rp(t)}\tp(t)\ut +
			r(t)\psw[orange]{\tpp(t)}\ut +
			r(t)\tp(t) \underbracket[1pt]{\psw[firebrick]{\dv{\ut}{t}}}_{= -\tp(t)\ur}
			\qed
		\end{gather*}
	}%
	\vspace{-35pt}

	\tikz[remember picture, overlay]
	\draw[-stealth, thick, transform canvas={yshift=6pt}, color=\sswitch{white}{orchid}]
	(pic cs:DV) to [out=90, in=90] ([shift={(-3pt,3pt)}]pic cs:RP)
	;
	\tikz[remember picture, overlay]
	\draw[-stealth, thick, transform canvas={yshift=6pt}, color=\sswitch{white}{cornflowerblue}]
	(pic cs:DV) to [out=90, in=90] ([shift={(-6pt,6pt)}]pic cs:UR)
	;
	\tikz[remember picture, overlay]
	\draw[-stealth, thick, transform canvas={yshift=6pt}, color=\sswitch{white}{limegreen}]
	(pic cs:DV) to [out=90, in=90] ([shift={(-3pt,3pt)}]pic cs:R)
	;
	\tikz[remember picture, overlay]
	\draw[-stealth, thick, transform canvas={yshift=6pt}, color=\sswitch{white}{orange}]
	(pic cs:DV) to [out=90, in=90] ([shift={(-3pt,6pt)}]pic cs:TP)
	;
	\tikz[remember picture, overlay]
	\draw[-stealth, thick, transform canvas={yshift=6pt}, color=\sswitch{white}{firebrick}]
	(pic cs:DV) to [out=90, in=90] ([shift={(-6pt,6pt)}]pic cs:UT)
	;

\end{tcb*}

\begin{tcb*}(prop){Accélération en polaires}
	Finalement, la vitesse en coordonnées polaires s'écrit
	\psw{
		\[
			\boxed{
				\af = \left( \rpp -r\tp^2 \right)\ur + \left( 2\rp\tp+r\tpp \right)\ut
			}
		\]
	}
	\vspace{-15pt}
\end{tcb*}

\section{Exemples de mouvements plans}
\subsection{Mouvement circulaire}

\begin{tcb*}(defi)<lftt>{Mouvement circulaire}
	Un mouvement est dit \textbf{circulaire} s'il se fait dans un plan, à une
	distance de l'axe de rotation $r$ constante, soit
	\psw{%
		\[
			r(t) = \cte = R
		\]
	}%
	\vspace{-25pt}
\end{tcb*}

\begin{tcb}(impl)<lftt>{Mouvement circulaire}
	Dans ce cas-là, on a
	\[
		\psw{\OM(t) = R\ur}
		\qet
		\psw{\rp(t) = 0 = \rpp(t)}
	\]
	En notant $\w(t) = \tp(t)$ la vitesse angulaire, la vitesse et l'accélération
	donnent
	\[
		\psw{\boxed{\vf(t) = R\w(t)\ut}}
		\qet
		\psw{\boxed{\af(t) = -R\w^2(t)\ur + R\wp(t)\ut}}
	\]
	\vspace{-25pt}
\end{tcb}

% \begin{tcb*}(impo){$\w$ en mécanique vs.\ $\w$ en filtrage}
% 	Bien que les symboles des variables soient les mêmes, les deux grandeurs
% 	décrites n'ont \textbf{rien à avoir} entre elles~:
% 	\begin{itemize}
% 		\item $\w$ en filtrage est la \textit{pulsation}~;
% 		\item $\w$ en mécanique est la \textit{vitesse angulaire}.
% 	\end{itemize}
% 	Cependant, les deux ont la \textbf{même unité}, les $\si{rad.s^{-1}}$,
% 	puisqu'elles décrivent bien la variation d'un angle/d'une phase dans le temps.
% \end{tcb*}

\vspace{-20pt}

\subsection{Mouvement circulaire uniforme}

\vspace{-15pt}

\begin{tcbraster}[raster equal height=rows, raster columns=4]
	\begin{tcolorbox}[blankest, raster multicolumn=3, space to=\myspace]
		\begin{tcb*}(defi)<lftt>{Mouvement circulaire uniforme}
			Un mouvement est dit \textbf{circulaire \textit{uniforme}} si c'est un
			mouvement circulaire ($r(t) = \cte$) à \textit{vitesse angulaire
				constante}, soit
			\psw{%
				\[
					r(t) = R
					\qet
					\tp(t) = \w_0
				\]
			}%
			\vspace{-25pt}
		\end{tcb*}
		\begin{tcb}(impl)<lftt>{Mouvement circulaire uniforme}
			Dans ce cas, $\rp = 0 = \rpp$ mais également $\tpp = 0$, donc la vitesse et
			l'accélération donnent
			\[
				\psw{\boxed{\vf(t) = R\w_0\ut}}
				\qet
				\psw{\boxed{\af(t) = -R\w_0{}^2\ur}}
			\]
			\vspace{-25pt}
		\end{tcb}
	\end{tcolorbox}
	\begin{tcb}[list entry={\lte\theexem~:~Mouvement circulaire uniforme}](exem)<lftt>'r'{}
		\vspace{-25pt}
		\begin{center}
			\sswitch{%
				\includegraphics[width=\linewidth, draft=true]{mvt_circ_uni}
			}{%
				\includegraphics[width=\linewidth]{mvt_circ_uni}
			}%
			\vspace{-15pt}
			\captionsetup{justification=centering}
			\captionof{figure}{\\Mvt.\ circ.\ unif.}
		\end{center}
		\vspace*{-45pt}
	\end{tcb}
\end{tcbraster}

% \begin{tcb*}(ror){Mouvement circulaire uniforme}
% 	Dans le cas du mouvement circulaire uniforme,
% 	\begin{itemize}
% 		\item Le vecteur vitesse est selon $\ut$ et est de norme constante,
% 		      égale à $R\w_0$~;
% 		\item Le vecteur accélération pointe vers le centre et est de norme
%       constance, égale à $\DS R\w_0{}^2 = \frac{v^2}{R}$.
% 	\end{itemize}
% \end{tcb*}

% \begin{tcb}*(expe)<itc>"trans"{Transition}
% 	Si la trajectoire d'un objet change de courbure, il peut être fastidieux de
% 	travailler avec les coordonnées polaires~: on utilisera alors un repère
% 	attaché à l'objet.
% \end{tcb}

\vspace*{-20pt}

\subsection{Repère de \textsc{Frenet}}

\begin{tcb*}[sidebyside, righthand ratio=.4](defi){Repère de \textsc{Frenet}}
	Soit un point M sur une trajectoire courbe d'origine O. On approxime sa
	trajectoire à un instant $t$ par son \textbf{cercle osculateur}, de
	\textbf{rayon de courbure} $R(t)$. D'où le repère mobile de \textsc{Frenet}
	attaché à M~:
	\begin{itemize}
		\item \psw{$\vv{u_T}$ tangent à la trajectoire en M~;}
		\item \psw{%
			      $\vv{u_N} \perp \vv{u_T}$ dirigé vers le centre.
		      }%
	\end{itemize}
	$\gamma(t) = 1/R(t)$ s'appelle la \textbf{courbure} de
	la trajectoire.
	\tcblower
	\begin{center}
		\sswitch{%
			\includegraphics[width=\linewidth]{frenet_stud}
		}{%
			\includegraphics[width=\linewidth]{frenet_prof}
		}%
		\vspace{-15pt}
		\captionof{figure}{\textsc{Frenet}}
	\end{center}
\end{tcb*}

\begin{tcb*}(prop){Vitesse et accélération \textsc{Frenet}}
	La vitesse et l'accélération dans le repère mobile de \textsc{Frenet}
	s'expriment~:
	\[
		\psw{\boxed{\vf(t) = v(t) \vv{u_T}}}
		\qet
		\psw{\boxed{\af(t) = \vp(t) \vv{u_T} + \frac{v(t)^2}{R(t)}\vv{u_N}}}
	\]
\end{tcb*}

\begin{tcb}(demo)<lftt>{Vitesse et accélération \textsc{Frenet}}
	Soit $s(t)$ la distance parcourue sur la courbe de la trajectoire $\Cc$
	depuis l'origine O. On l'appelle \textbf{abscisse curviligne}, telle que
	\[
		s(t) = \int_{\Cc} \dd{s}
	\]
	\tcbsubtitle{\fatbox{\textbf{Vitesse}}}
	\vspace{-15pt}
	\psw{%
		\begin{gather*}
			\dd{\OM} =
			\OM (t + \dd{t}) - \OM(t) =
			\OM(t+\dd{t}) + \vv{\Mr(t)\Or} =
			\vv{\Mr(t)\Mr(t+\dd{t})} = \dd{s} \vv{u_T}
			\\\Lra
			\vv{u_T} = \dv{\OM}{s}
			\Ra
			\vf(t) =
			\dv{\OM}{t} =
			\underbracket[1pt]{\dv{s}{t}}_{= v(t)}
			\underbracket[1pt]{\dv{\OM}{s}}_{= \vv{u_T}}
			\qed
		\end{gather*}
	}%
	\vspace{-15pt}
	\tcbsubtitle{\fatbox{\textbf{Accélération}}}
	\vspace{-15pt}
	\begin{isd}[interior hidden, righthand ratio=.3](demo)
		\begin{gather*}
			\beforetext{On a}
			\psw{%
				\af(t) = \dv{\vf}{t} = \vp(t)\vv{u_T} + v(t) \dv{\vv{u_T}}{t}
			}%
			\\
			\beforetext{Or,}
			\psw{%
				\dd{\vv{u_T}} =
				\dd{\th} \vv{u_N}
				\qet
				\dd{s} = R(t) \dd{\th}
				\qso
				\dd{\vv{u_T}} = \frac{\dd{s}}{R(t)}\vv{u_N}
			}%
			\\
			\psw{%
				\Lra
				\dv{\vv{u_T}}{t} = \frac{1}{R(t)} \dv{s}{t} \vv{u_N}
				\Lra
				\boxed{\dv{\vv{u_T}}{t} = \frac{v(t)}{R(t)}\vv{u_N}}
			}%
			\\
			\beforetext{D'où}
			\psw{%
				\boxed{\af(t) = \vp(t)\vv{u_T} + \frac{v(t)^2}{R(t)}\vv{u_N}}
				\qed
			}%
		\end{gather*}
		\tcblower
		\begin{center}
			\sswitch{%
				\includegraphics[width=\linewidth]{frenet_demo_stud}
			}{%
				\includegraphics[width=\linewidth]{frenet_demo_prof}
			}%
			\vspace{-15pt}
			\captionof{figure}{$\dd{\protect\vv{u_T}}$}
		\end{center}
	\end{isd}
\end{tcb}

\begin{tcb}(rema)<lftt>{Cas limites repère de \textsc{Frenet}}
	\begin{itemize}
		\item On retrouve le mouvement rectiligne uniforme avec $R = +\infty \Lra
			      \gamma = 0$, puisqu'on a alors $\af = \dv{v}{t}\vv{u_T}$ avec
		      $\vv{u_T}$ dans le sens de la trajectoire.

		\item On retrouve également le mouvement circulaire puisque dans ce cas la
		      trajectoire \textbf{est} le cercle osculateur, donc $\vv{u_T} = \ut$ et
		      $\vv{u_N} = -\ur$.
	\end{itemize}
	\vspace{-15pt}
\end{tcb}

\section{Application~: pendule simple}

\subsection{Tension d'un fil}
\begin{tcb*}[sidebyside, righthand ratio=.18](defi){Tension d'un fil}
	Un point matériel M accroché à un fil tendu subit de la part de ce fil une
	force appelée \textbf{tension du fil} et notée $\Tf$ telle que
	\psw{%
		\[\boxed{\Tf = -T \,\ur}\]
	}%
	avec $\ur$ un vecteur unitaire dirigé \textbf{du point d'accroche vers M}.
	\tcblower
	\begin{center}
		\sswitch{%
			\includegraphics[width=\linewidth]{tension_stud}
		}{%
			\includegraphics[width=\linewidth]{tension_prof}
		}%
		\vspace{-15pt}
		\captionof{figure}{}
	\end{center}
\end{tcb*}

\begin{tcb*}[bld](impl){Condition de tension}
	\begin{center}
		\psw{La \textbf{condition de tension} est $\norm{\Tf} > 0$.}
	\end{center}
\end{tcb*}

\subsection{Pendule simple}
% Et si je vous disais qu'on peut mesurer l'attraction de la pesanteur… avec un
% bout de ficelle et une masse~?
\begin{tcb*}(prop){Mouvement d'un pendule simple}
	Soit un point matériel M de masse $m$ accrochée au bout d'un fil de longueur
	$\ell = \cte$ dans le champ de pesanteur $\gf$. On l'écarte de la verticale
	d'un angle $\th_0 \neq 0$ avec une vitesse initiale nulle. Pour
	$\th_0$ suffisamment faible, on obtient
	\smallbreak
	\begin{isd}[interior hidden](prop)
		\tcbsubtitle{\fatbox{\textbf{Équation différentielle}}}
		\vspace{-15pt}
		\psw{%
			\[
				\boxed{\tpp(t) + \w_0{}^2\th(t) = 0}
				\qav
				\boxed{\w_0 = \sqrt{\frac{g}{\ell}}}
			\]
		}%
		\vspace{-15pt}
		\tcblower
		\tcbsubtitle{\fatbox{\textbf{Solution}}}
		\psw{%
			\[
				\boxed{\th(t) = \th_0 \cos(\w_0t)}
			\]
		}%
		\vspace{-15pt}
	\end{isd}
\end{tcb*}

\begin{tcb*}[breakable](demo)<lftt>{Mouvement pendule simple}
	\begin{isd}[righthand ratio=.3, sidebyside align=top]
		\begin{enumerate}[label=\sqenumi]
			\item[b]{Système}~: \{masse\} repérée par M
			\item[b]{Schéma.}
			\item[b]{Modélisation.}
			      \begin{itemize}
				      \item[b]{Référentiel}~: \psw{$\Rc\ind{labo}$ supposé
					            galiléen}
				      \item[b]{Repère}~: \psw{%
					            $(\Or,\ur,\ut)$.
				            }%
			      \end{itemize}
		\end{enumerate}
		\tcblower
		\begin{center}
			\vspace{-12pt}
			\sswitch{%
				\includegraphics[width=.9\linewidth]{pendule_stud}
			}{%
				\includegraphics[width=.9\linewidth]{pendule_prof}
			}%
			\vspace{-15pt}
			\captionof{figure}{Schéma.}
		\end{center}
	\end{isd}
	% \vspace{-55pt}
	\begin{enumerate}[label=\sqenumi, start=4]
		\item{}[]
		      \vspace{-35pt}
		      \begin{itemize}
			      \item[b]{Repérage}~:
			            \vspace{-20pt}
			            \psw{%
				            \begin{align*}
					            \OM(t) & = \underbracket[1pt]{\ell}_{\mathclap{= \cte}}\ur
					            \\
					            % \quad ; \quad
					            \vf(t) & = \ell \tp(t)\ut
					            \\
					            % \quad ; \quad
					            \af(t) & = -\ell \tp^2(t)\ur + \ell \tpp(t) \ut
				            \end{align*}
			            }%
			            \vspace{-25pt}
			      \item%
			            \leftcenters{%
				            \textbf{Conditions initiales}~:
			            }{%
				            \psw{%
					            $\th(0) = \th_0
						            \qet
						            \vf(0) = \of
						            \Lra
						            \tp(0) = 0$
				            }%
			            }%
		      \end{itemize}
		\item[b]{Bilan des forces.}
		      \psw{%
			      \begin{align*}
				      \beforetext{\textbf{Poids}}
				      \Pf & = m\gf = mg \pa{\cos(\th(t))\ur - \sin(\th(t))\ut}
				      \\ \beforetext{\textbf{Tension}}
				      \Tf & = -T \ur
			      \end{align*}
		      }%
		      \vspace{-25pt}
		\item \leftcenters{\textbf{PFD.}}{\psw{$m\af(t) = \Pf + \Tf$}}
		\item[b]{Équations scalaires.}~On projette le PFD sur les axes~:
		      \psw{%
			      \[
				      \left\{
				      \begin{array}{rcl}
					      -m\ell\tp^2(t) & = & mg\cos(\th(t)) - T
					      \qMath{\xul{ignorée}}
					      \\
					      m\ell\tpp(t)   & = & -mg\sin(\th(t))
				      \end{array}
				      \right.
				      \quad \Ra \quad
				      \boxed{\tpp(t) + \frac{g}{\ell}\sin(\th(t)) = 0}
			      \]
		      }%
		      qui constitue l'équation du mouvement du pendule. Sous cette forme, elle
		      est \textbf{non-linéaire} donc non résoluble analytiquement~; elle peut
		      l'être numériquement, voir
		      \texttt{Capytale}\footnote{\url{
				      https://capytale2.ac-paris.fr/web/c/a7c5-1241282}}.
		      En revanche, dans l'approximation des petits angles, on a $\sin(\th)\approx\th$,
		      et ainsi on obtient~:
		      \psw{
			      \[
				      \tpp(t) + \frac{g}{\ell}\th(t) = 0
				      \Lra
				      \boxed{\tpp(t) + \w_0{}^2\th(t) = 0}
				      \qav
				      \w_0 = \sqrt{\frac{g}{\ell}}
			      \]
		      }
		      \textbf{C'est \psw{l'équation différentielle d'un oscillateur harmonique}~!}
		\item[b]{Résolution.}~On a la solution générale homogène~:
		      \begin{gather*}
			      \psw{%
				      \th(t) = A\cos(\w_0t) + B\sin(\w_0t)
			      }%
			      \qor
			      \psw{%
				      \th(0) = 0
				      \Lra
				      \boxed{A = \th_0}
			      }%
			      \qet
			      \psw{%
				      \tp(0) = 0
				      \Lra
				      \boxed{B = 0}
			      }%
			      \\\beforetext{Finalement}
			      \psw{%
				      \boxed{\th(t) = \th_0\cos(\w_0t)}
			      }%
			      \vspace{-15pt}
		      \end{gather*}
	\end{enumerate}
\end{tcb*}

\begin{tcb}(rema)<lftt>{Pendule simple grands angles}
	Dans cette approximation, la période ne dépend \textbf{ni de la masse, ni de
		l'angle initial}. En réalité, si on s'écarte beaucoup de la verticale
	($\abs{\theta} > \pi/4$), la période change et n'est plus celle que l'on a aux
	petits angles. Voir le changement sur le graphique ci-dessous et
	\href{http://www.sciences.univ-nantes.fr/sites/genevieve_tulloue/Meca/Oscillateurs/periode_pendule.php}{en ligne}.
	\smallbreak
	\begin{minipage}{0.45\linewidth}
		\begin{center}
			\includegraphics[width=\linewidth]{pendule_sol}
			\captionof{figure}{$\th (t)$ pour petits angles.}
		\end{center}
	\end{minipage}
	\hfill
	\begin{minipage}{0.45\linewidth}
		\begin{center}
			\includegraphics[width=\linewidth]{pendule_gdang}
			\captionof{figure}{Évolution de $T$ selon $\th_0$.}
		\end{center}
	\end{minipage}
	\vspace{-15pt}
\end{tcb}

\begin{tcb*}(appl)<lftt>{Mesure de $g$ par un pendule}
	Déterminer l'accélération de la pesanteur $g$ par l'expérience du pendule
	simple.
	\tcblower
	\psw{%
	Le pendule oscille à la pulsation $\w_0$ et à la période $T_0$ telles que
	\[
		\w_0 = \sqrt{\frac{g}{\ell}}
		\qdc
		T_0 = 2\pi \sqrt{\frac{\ell}{g}}
		\qso
		\boxed{g = \frac{4\pi^2\ell}{T_0{}^2}}
	\]
	Avec un fil de longueur $\ell = \SI{0.84\pm0.06}{cm}$, on mesure $T_0 =
		\psw{\SI{1.84\pm0.1}{s}}$~; ainsi
	\[
		\boxed{g = \psw{\SI{9.75}{m.s^{-2}}}}
	\]
	}%
	\vspace{-35pt}
\end{tcb*}

\section{Mouvement courbe dans l'espace}
\subsection{Coordonnées cylindriques}

% La manière la plus simple de passer du plan à l'espace est de prendre les
% coordonnées polaires et d'y ajouter la coordonnée cartésienne $z$~: on définit
% ainsi les coordonnées \textbf{cylindriques}.

\begin{tcb*}[sidebyside, righthand ratio=.35](defi){Coordonnées cylindriques}
	Le repère cylindrique est constitué d'une origine O autour de laquelle sont
	définis trois vecteurs $(\!\ur,\ut,\uz)$, avec
	\begin{itemize}
		\item \psw{$(\ur,\ut)$ la base polaire}
		\item \psw{%
			      $\uz$ le vecteur de base cartésienne tel que $\ur \wedge \ut = \uz$
		      }%
		\item[m][15]
		      \[
			      \psw{\boxed{\OM(t) = \vvr{OH}(t) + \vvr{HM}(t) = r(t)\ur + z(t)\uz}}
		      \]
		\item[m][15]
		      \[
			      \psw{\boxed{\norm{\OM}(t) = \sqrt{r(t)^2 + z(t)^2}}}
		      \]
	\end{itemize}
	\tcblower
	\begin{center}
		\sswitch{
			\includegraphics[width=\linewidth]{pos_cyl_stud}
		}{
			\includegraphics[width=\linewidth]{pos_cyl_prof}
		}
		\vspace{-15pt}
		% \captionsetup{justification=centering}
		\captionof{figure}{Cylindriques.}
	\end{center}
\end{tcb*}

La détermination de la vitesse et de l'accélération est la même qu'en polaires,
il suffit d'ajouter les dérivées de $z$ puisque $\uz$ est fixe dans le temps.
Ainsi,

\begin{tcb*}(prop){Bilan~: coordonnées cylindriques}
	\begin{itemize}[itemsep=-10pt]
		\item \leftcenters{\textbf{Variables~:}}
		      {\psw{$(r,\th,z)$}}
		\item \leftcenters{\textbf{Vecteurs de base~:}}
		      {\psw{$(\ur,\ut,\uz)$}}
		\item \leftcenters{\textbf{Position~:}}
		      {\psw{$\OM = r\ur + z\uz$}}
		\item \leftcenters{\textbf{Vitesse~:}}
		      {\psw{$\vf = \rp\ur + r\tp\ut + \zp\uz$}}
		\item \leftcenters{\textbf{Déplacement élém.~:}}
		      {\psw{$\dd\OM = \dd{r}\ur + r\dd{\th}\ut + \dd{z}\uz$}}
		\item \leftcenters{\textbf{Accélération~:}}
		      {\psw{
				      $\DS \af =
					      \left( \rpp -r\tp^2 \right)\ur +
					      \left( 2\rp\tp + r\tpp \right)\ut +
					      \zpp\uz$
			      }}
	\end{itemize}
\end{tcb*}

\begin{tcb}(rema)<lftt>{Volume élémentaire cylindriques}
	\begin{isd}[interior hidden](rema)
		Une conséquence fondamentale du déplacement élémentaire est de pouvoir définir
		une surface et un volume infinitésimaux suivant une variation infinitésimale
		des trois coordonnées.
		\smallbreak
		En effet, pour une petite variation $(\dd{r}, \dd{\th}, \dd{z})$,
		on se déplace de $\dd{r}$ dans la direction $\ur$, de $\dd{z}$ dans la
		direction $\uz$ et l'arc de cercle formé par la variation d'angle $\dd{\th}$
		est de longueur $r\dd{\th}$.
		\tcblower
		\begin{minipage}{0.49\linewidth}
			\begin{center}
				\includegraphics[width=\linewidth]{cyl_vol}
				\captionsetup{justification=centering}
				\captionof{figure}{\\$\dd{V}$ cylindriques.}
			\end{center}
		\end{minipage}
		\hfill
		\begin{minipage}{0.49\linewidth}
			\begin{center}
				\includegraphics[height=2cm]{zoom_cyl_lgn}
			\end{center}
			\begin{center}
				\includegraphics[height=2cm]{zoom_cyl_sfc}
				\captionsetup{justification=centering}
				\captionof{figure}{\\Zoom volume.}
			\end{center}
		\end{minipage}
	\end{isd}
	Le volume élémentaire est alors le \textbf{produit des trois composantes de}
	$\dd{\OM}$~:
	\psw{%
		\[
			\boxed{\dd{V} = r \dd{r} \dd{\th} \dd{z}}
		\]
	}%
	On trouve le volume d'un cylindre de rayon $R$ et de hauteur $h$ en
	intégrant sur les trois coordonnées~:
	\[
		\psw{%
			V\ind{cyl} = \iiint_{r,\th,z} \dd{V} =
			\int_{r'=0}^{R} r'\dd{r'}
			\int_{\th'=0}^{2\pi} \dd{\th'}
			\int_{z'=0}^{h} \dd{z'} =
			\frac{1}{2}R^2\times 2\pi \times h = \boxed{h\pi R^2}
		}%
	\]
	C'est l'aire d'un disque multiplié par la hauteur~!
\end{tcb}

\begin{tcb*}(impo){Choix des coordonnées}
	Dans un problème de mécanique, on choisit les coordonnées judicieusement en
	fonction des symétries du système. \textbf{Sauf proposition de l'énoncé}, on
	utilisera les coordonnées \textbf{cylindriques} pour les mouvements de
	\textbf{rotation}. On utilisera les coordonnées cartésiennes sinon.
\end{tcb*}

\subsection{Coordonnées sphériques}
La manière la plus complète de décrire un mouvement général dans l'espace repose
sur un dernier système de coordonnées, les coordonnées \textbf{sphériques}.

\begin{tcb*}[sidebyside, righthand ratio=.35](defi){Repère sphérique}
	Le repère sphérique est constitué d'une origine O autour de laquelle sont
	définis trois vecteurs, $(\!\ur, \ut, \uf)$, tels que
	\begin{gather*}
		\psw{\boxed{\OM(t) = r(t)\ur}}
		\\\beforetext{avec}
		\psw{\boxed{\th(t) = \widehat{(\uz,\OM)}}}
		\qet
		\psw{\boxed{\f(t) = \widehat{(\ux,\vvr{OH})}}}
	\end{gather*}
	où $\widehat{(\cdot, \cdot)}$ est l'\textbf{angle orienté}, et H$(t)$ le
	projeté orthogonal de M$(t)$ sur le plan polaire.
	\smallbreak
	\fatbox{\iconimpo~\textbf{$\f(t)$ correspond à $\th(t)$ des coordonnées polaires.}}
	\tcblower
	\begin{center}
		\sswitch{%
			\includegraphics[width=\linewidth]{pos_sph_stud}
		}{%
			\includegraphics[width=\linewidth]{pos_sph_prof}
		}%
		\vspace{-15pt}
		% \captionsetup{justification=centering}
		\captionof{figure}{Sphériques.}
	\end{center}
\end{tcb*}

\begin{tcb}(rema)<lftt>{Lien sphériques/cartésiennes}
	On peut inverser les définitions. En effet,
	\begin{gather*}
		\psw{{\rm OH}(t) = r(t)\sin(\th(t))}
		\qet
		\psw{{\rm HM}(t) = r(t)\cos(\th(t))}
		\\
		\psw{\boxed{x(t) = r(t)\sin(\th(t))\cos(\f(t))}}
		\qet
		\psw{\boxed{y(t) = r(t)\sin(\th(t))\sin(\f(t))}}
		\qet
		\psw{\boxed{z(t) = r(t)\cos(\th(t))}}
	\end{gather*}
\end{tcb}

\begin{tcb}(nota)<lftt>{Coordonnées sphériques}
	\begin{itemize}[itemsep=-5pt]
		\item $\th \in [0~;~\pi]$ est nommé \textbf{colatitude} ($\lb =
		      \abs{\pi/2 - \th}$ la latitude)~, et respecte
		      \psw{%
			      \[
				      \tan(\th(t))
				      = \frac{\rm OH(t)}{z(t)}
				      \Lra \th
				      = \arctan(\frac{\sqrt{x^2(t) + y^2(t)}}{z(t)})
			      \]
		      }%
		      \vspace{-15pt}
		\item $\f \in [0~;~2\pi]$ est nommé \textbf{longitude}, et respecte
		      \psw{$\DS \f(t) = \arctan(\frac{y(t)}{x(t)})$}
		\item Une courbe $\th(t) = \cte$ est appelée \textbf{parallèle}~; le
		      \textbf{rayon} d'un parallèle est \xul{\psw{$r(t)\sin(\th(t))$}}.
		\item Une courbe $\f(t) = \cte$ est appelée \textbf{méridien}~; le
		      \textbf{rayon} d'un méridien est \xul{\psw{$r(t)$}}.
	\end{itemize}
\end{tcb}

\begin{tcb}(exem)<lftt>{Repérage sphérique sur Terre}
	Le repérage sur la Terre utilise la latitude et la longitude. Par
	exemple, le lycée \textsc{Pothier} se situe à \ang{47.90}N,
	\ang{1.90;;}E~; on a donc
	\psw{%
		\[
			\th\ind{\textsc{Pothier}} = \ang{42.1}
			\qet
			\f\ind{\textsc{Pothier}} = \ang{1.90}
		\]
	}%
	\vspace{-25pt}
	% \tcblower
	% \begin{center}
	% 	\includegraphics[width=\linewidth]{sph_terre}
	% \end{center}
\end{tcb}

\begin{tcb*}[sidebyside, righthand ratio=.5](prop)
	{Déplacement élémentaire sphérique}
	\begin{itemize}
		\item Variation $\dd{r}$ $\Ra$ déplace\mnt\ $\dd{r}\ur$~;
		\item Variation $\dd{\th}$ $\Ra$ déplace\mnt\ $r\dd{\th}\ut$~;
		\item Variation $\dd{\f}$ $\Ra$ déplace\mnt\ $r\sin\th\dd{\f}\uf$.
	\end{itemize}
	\psw{\[\boxed{\dd\OM = \dd{r}\ur + r\dd{\th}\ut + r\sin\th\dd{\f}\uf}\]}
	\tcblower
	\noindent
	\begin{minipage}{.59\linewidth}
		\begin{center}
			\includegraphics[width=\linewidth]{sphe_vol}
			\captionsetup{justification=centering}
			\captionof{figure}{\\$\dd{\protect\OM}$ sphériques}
		\end{center}
	\end{minipage}
	\begin{minipage}{.39\linewidth}
		\begin{center}
			\includegraphics[height=1.5cm]{zoom_sph_lgn}
		\end{center}
		\begin{center}
			\includegraphics[height=1.5cm]{zoom_sph_sfc}
			\captionsetup{justification=centering}
			\captionof{figure}{Zoom volume.}
		\end{center}
	\end{minipage}
\end{tcb*}

\begin{tcb}(rema)<lftt>{Volume élémentaire sphériques}
	On trouve de la même manière le volume élémentaire~:
	\psw{%
		\[
			\boxed{\dd{V} = r^{2}\sin(\theta)\dd{r}\dd{\th}\dd{\f}}
		\]
	}%
	Il permet de déterminer le volume d'une boule~:
	\psw{%
		\[
			V\ind{boule}
			= \iiint_{r,\th,\f}
			= \int_{r'=0}^{R} r'^2\dd{r'}
			\int_{\th' = 0}^{\pi} \sin\th'\dd{\th'}
			\int_{\f'=0}^{2\pi} \dd{\f}
			= \int_{r'=0}^{R} 4\pi r'^2 \dd{r}
			= \boxed{\frac{4}{3}\pi R^3}
		\]
	}%
	\vspace{-15pt}
\end{tcb}

\end{document}
