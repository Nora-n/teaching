\documentclass{standalone}
\usepackage{mintikz}

\renewcommand{\mycol}{black}

\settype{prof}
\settype{stud}

\begin{document}
\begin{tikzpicture}[auto,
	node distance=2,
	start chain=A going right,
	block/.style={draw, thick, rounded corners, align=center,
			minimum height=11mm, minimum width=12mm,
			text opacity=1, draw opacity=1,
			on chain=#1, join},
	block/.default=A,
	meca/.style={%
			fill=mygreen, opacity=.5,
			block,
		},
	elec/.style={%
			fill=myblue, opacity=.5,
			block,
		},
	sum/.style n args={5}{circle, draw, thick, minimum size=9mm,
			append after command={\pgfextra{\let\LN\tikzlastnode}
					(\LN.north west) edge[thick] (\LN.south east)
					(\LN.south west) edge[thick] (\LN.north east)
					node[left]     at (\LN.center) {$#2$}
					node[above]    at (\LN.center) {$#3$}
					node[right]    at (\LN.center) {$#4$}
					node[below]    at (\LN.center) {$#5$}
				},
			node contents={},
			on chain=#1, join},
	sum/.default={A}{+}{+}{}{},
	arr/.style={-{Stealth[scale=1]}, semithick},
	every join/.style={-{Stealth[scale=1]}, semithick}
	]
	\begin{scope}
		\tikzset{node distance={1cm}}
		\node[elec, minimum width=20mm]
		% {\psw(mygreen!50){Puissance}\\\psw(mygreen!50){mécanique}}; % A-3
		{\psw(myblue!50){Puissance}\\\psw(myblue!50){électrique}};% A-1
		\node[block, minimum width=20mm] {\psw{Moteur}}; % A-2
		\node[meca, minimum width=20mm]
		% {\psw(myblue!50){Puissance}\\\psw(myblue!50){électrique}};% A-1
		{\psw(mygreen!50){Puissance}\\\psw(mygreen!50){mécanique}}; % A-3
	\end{scope}

	\node[start chain=B going below, block=B, below=.5 of A-2,
		dashed,
		join=with A-2
	] {\psw{Pertes}};
\end{tikzpicture}

\end{document}
