\documentclass[../TDN1.tex]{subfiles}%

\begin{document}
\section[s]"1"{Vitesse du son}

\QR{%
	Donner l'expression de la célérité $c$ du son dans un fluide en fonction de la
	masse volumique du $\rho$ du fluide et du coefficient d'incompressibilité
	$\chi$, homogène à l'inverse d'une pression.
}{%
	\ifprof{
		\vspace{-15pt}
	}
	\begin{tcn}(data){Données}
		$c$ est une vitesse, $\rho$ une masse volumique et $\chi$ une grandeur
		relative à la pression. On nous donne $\dim{\chi} = \dim{P}^{-1}$ avec $P$
		une pression.
	\end{tcn}
	\begin{tcn}(ques){Résultat attendu}
		On cherche $c$ en fonction de $\rho$ et $\chi$, soit
		\[
			\boxed{c = \rho^\alpha\chi^\beta}
		\]
		avec $\alpha$ et $\beta$ à déterminer.
	\end{tcn}
	\begin{tcn}(tool){Outil}
		Une pression est une force surfacique, c'est-à-dire une force répartie
		sur une surface. On a donc
		\[
			\dim{P} = \frac{\dim{F}}{\rm L^2}
		\]
		De plus, la force de pesanteur s'exprime $F = mg$, avec $g$
		l'accélération de la pesanteur~: ainsi,
		\[
			\dim{F} = \dim{m}\cdot \dim{g} = \rm M\cdot L\cdot T^{-2}
		\]
	\end{tcn}
	\begin{tcn}[sidebyside](appl){Application}
		On commence par déterminer la dimension de $c$. En tant que vitesse, on a
		\[\dim{c} = \rm L\cdot T^{-1}\]
		On exprime ensuite les dimensions de $\rho$ et $\chi$. D'une part,
		\[\dim{\rho} = \rm M\cdot L^{-3}\]
		D'autre part,
		\begin{align*}
			\dim{\chi} & = \DS\frac{\rm L^2}{\dim{F}}
			\\\Lra
			\dim{\chi} & = \DS\frac{\rm L^{\cancel{2}}}
			{\rm M\cdot \cancel{\rm L}\cdot T^{-2}}
			\\\Lra
			\dim{\chi} & = \rm L\cdot M^{-1}\cdot T^2
		\end{align*}
		\tcblower
		L'expression recherchée revient à résoudre
		\[\rm L\cdot T^{-1} = (M\cdot L^{-1})^\alpha(L\cdot M^{-1}\cdot T^2)^\beta\]
		En développant, on trouve un système de 3 équations à 2 inconnues~:
		\[ \left\{
			\begin{array}{rcl}
				1  & = & -3\alpha + \beta \\
				-1 & = & 2\beta           \\
				0  & = & \alpha - \beta   \\
			\end{array}
			\right. \Longleftrightarrow \left\{
			\begin{array}{rcl}
				\beta  & = & - \frac{1}{2} \\
				\alpha & = & - \frac{1}{2}
			\end{array}
			\right.\]
		Ainsi, on peut exprimer $c$ tel que
		\begin{empheq}[box=\fbox]{equation*}
			c = \frac{1}{\sqrt{\rho\chi}}
		\end{empheq}
	\end{tcn}
}%

\end{document}
