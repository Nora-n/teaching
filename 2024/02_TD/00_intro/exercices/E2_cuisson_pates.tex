\documentclass[../TDN1.tex]{subfiles}%

\begin{document}

\section[s]"2"{Faire cuire des pâtes}
\enonce{%
	Sur une facture d'électricité, on peut lire sa consommation d'énergie
	électrique exprimée en \si{kWh} (kilowatt-heure).
}%
\QR{%
	Quelle est l'unité SI associée~? Que vaut \SI{1}{kWh} dans cette unité
	SI~?
}{%
	\begin{tcbraster}[raster columns=2]
		\begin{tcn}(data){Donnée}
			Consommation électrique en \si{kWh}.
		\end{tcn}
		\begin{tcn}(ques)'r'{Résultat attendu}
			Unité associée en unités SI.
		\end{tcn}
	\end{tcbraster}
	\begin{tcn}(tool){Outil}
		Toute énergie s'exprime en joules (J), et les \textbf{puissances}
		sont des \textbf{énergies par unité de temps}. Notamment pour les
		watts on a $\SI{1}{W} = \SI{1}{J.s^{-1}}$.
	\end{tcn}
	\begin{tcn}(appl){Application}
		On a directement
		\[ \SI{1}{kWh} = \SI{1e3}{J.s^{-1}.h}\]
		Avec l'évidence que $ \SI{1}{h} = \SI{3600}{s}$, finalement
		\[\xul{}{\SI{1}{kWh} = \SI{3.6e6}{J}}\]
	\end{tcn}
}%

\QR{%
Sachant que la capacité thermique massique de l'eau est $c =
	\SI{4.18}{J.g^{-1}.K^{-1}}$ et que le prix du kilowatt-heure est de
\SI{0.16}{\EUR}, évaluer le coût du chauffage électrique permettant de
faire passer \SI{1}{L} d'eau de \SI{20}{\degreeCelsius} à
\SI{100}{\degreeCelsius}.
}{%
\ifprof{%
	\vspace{-15pt}
}%
\begin{tcbraster}[raster columns=2, raster equal height=rows]
	\begin{tcn}(data){Données}
		Notre objet d'étude est l'eau. On a~:
		\begin{itemize}
			\item $V_{\rm eau} = \SI{1}{L}$~;
			\item $T_{\rm i} = \SI{20}{\degreeCelsius}$~;
			\item $T_{\rm f} = \SI{100}{\degreeCelsius}$~;
			\item $c = \SI{4.18}{J.g^{-1}.K^{-1}}$
		\end{itemize}
		De plus, on nous donne
		\begin{itemize}
			\item $ \SI{1}{kWh} = \SI{1}{\EUR}$.
		\end{itemize}
	\end{tcn}
	\begin{tcolorbox}[blankest, raster multicolumn=1, space to=\myspace]
		\begin{tcbraster}[raster columns=1]
			\begin{tcn}[add to natural height=\myspace](ques)'r'{Résultat
				attendu}
				On cherche à monter \SI{1}{L} d'eau de 20 à
				\SI{100}{\degreeCelsius} et d'en calculer le coût en
				euros.
			\end{tcn}
			\begin{tcn}(tool)'r'{Outil}
				On doit donc trouver le coût en énergie et le convertir
				en euro. On cherche pour ça une loi reliant l'énergie
				consommée avec les données du problème, sachant que
				\textbf{pour l'eau}, \SI{1}{L} = \SI{1}{kg}.
			\end{tcn}
		\end{tcbraster}
	\end{tcolorbox}
\end{tcbraster}
\begin{tcn}(appl){Application}
	L'énergie à apporter $Q$ se déduit de la dimension de la capacité
	thermique massique~: $\dim{c} = \dim{Q}\rm\cdot M^{-1}\cdot
		\Theta^{-1}$. En appelant $m$ la masse du volume d'eau, par cette
	analyse dimensionnelle on a
	\[\boxed{Q = mc\Delta T}\]
	On a donc
	\[Q = \SI{3.3e5}{J}\quad\text{avec}\quad \left\{
		\begin{array}{rcl}
			m        & = & \SI{1}{kg}                    \\
			c        & = & \SI{4.18}{J.g^{-1}.K^{-1}}    \\\Lra
			c        & = & \SI{4.18e3}{J.kg^{-1}.K^{-1}} \\
			\Delta T & = & \SI{80}{K}
		\end{array}
		\right.\]
	et pour utiliser le coût en euros, on la converti en \si{kWh}~:
	\[\xul{Q = \SI{9.3e-2}{kWh} = \SI{1.5e-2}{\EUR}}\]
\end{tcn}
}%

\QR{%
	Si la plaque chauffe avec une puissance de $P = \SI{1200}{W}$, combien
	de temps faudra-t-il pour chauffer ce litre d'eau~?
}{%
	\ifprof{
		\vspace{-15pt}
	}
	\begin{tcbraster}[raster columns=2, raster equal height=rows]
		\begin{tcn}(data){Données}
			On utilise une plaque chauffante de puissance $P =
				\SI{1200}{W}$.
		\end{tcn}
		\begin{tcn}(ques)'r'{Résultat attendu}
			On cherche la durée que cette plaque prendrait pour
			transférer l'énergie calculée précédemment.
		\end{tcn}
	\end{tcbraster}
	\begin{tcbraster}[raster columns=2, raster equal height=rows]
		\begin{tcn}(tool){Outil}

			Une puissance est une énergie par unité de temps, et
			\SI{1}{W} = \SI{1}{J.s^{-1}}.

		\end{tcn}
		\begin{tcn}(appl)'r'{Application}
			On en déduit
			\begin{gather*}
				P = \frac{Q}{\Delta t}
				\qav
				\left\{
				\begin{array}{rcl}
					Q & = & \SI{3.3e5}{J}            \\
					P & = & \SI{1200}{J\cdot s^{-1}}
				\end{array}
				\right.
				\\
				\text{d'où}\quad
				\boxed{\Delta t = \frac{Q}{P} = \xul{\SI{2.8e2}{s}}}
			\end{gather*}
		\end{tcn}
	\end{tcbraster}
}%

\end{document}
