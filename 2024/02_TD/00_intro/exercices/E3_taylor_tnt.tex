\documentclass[../TDN1.tex]{subfiles}%

\begin{document}
\section[s]"3"{\textsc{Taylor} meilleur que James \textsc{Bond}~?}
\enonce{%
	À l'aide d'un film sur bande magnétique et en utilisant l'analyse
	dimensionnelle, le physicien Geoffrey \textsc{Taylor} a réussi en 1950 à
	estimer l'énergie $\Ec$ dégagée par l'explosion nucléaire du test
	\textsc{Trinity} de 1945, valeur pourtant évidemment classifiée. Le film
	permet d'avoir accès à l'évolution du rayon $R(t)$ du «~nuage~» de l'explosion
	au cours du temps. Nous supposons que les grandeurs influant sur ce rayon sont
	le temps $t$, l'énergie $\Ec$ de l'explosion et la masse volumique $\rho$ de
	l'air.
}%

\QR{%
	Quelles sont les dimensions de ces grandeurs~?
}{%
	On a directement \fbox{$\dim{R} = \rm L$}, \fbox{$\dim{t} = \rm T$},
	\fbox{$\dim{\rho} = \rm M\cdot L^{-3}$} et \fbox{$\dim{\Ec} = \rm M\cdot
			L^2\cdot T^{-2}$}.
}%

\QR{%
	Chercher une expression de $R$ sous la forme $R = k\times
		\Ec^{\alpha}t^\beta\rho^\gamma$, avec $k$ une constante adimensionnée.
}{%
	\ifprof{%
		\vspace{-15pt}
	}%
	\begin{tcbraster}[raster columns=2, raster equal height=rows]
		\begin{tcn}(data){Données}
			On nous donne la formule $R = k\times
				\Ec^{\alpha}t^\beta\rho^\gamma$ et que $\dim{k} = 1$.
		\end{tcn}
		\begin{tcn}(ques)'r'{Résultat attendu}
			On cherche $\alpha$, $\beta$ et $\gamma$ tels que $R =
				k\times \Ec^{\alpha}t^\beta\rho^\gamma$
		\end{tcn}
	\end{tcbraster}

	\begin{tcn}(tool){Outils}
		\begin{tasks}[label=\bdmd](3)
			\task $\dim{\Ec} = \rm M\cdot L^2\cdot T^{-2}$~;
			\task $\dim{t} = \rm T$~;
			\task $\dim{\rho} = \rm M\cdot L^{-3}$.
		\end{tasks}
	\end{tcn}

	\begin{tcn}(appl){Application}
		\begin{gather*}
			\beforetext{$\dim{R} = L$, donc on a}
			L =
			\left( {\rm  M}\cdot {\rm L}^2\cdot {\rm T}^{-2} \right)^\alpha
			{\rm T}^\beta
			\left( {\rm M}\cdot {\rm L}^{-3} \right)^\gamma
			\\
			\beforetext{Soit}
			\left\{
			\begin{array}{rcl}
				1 & = & 2\alpha - 3\gamma \\
				0 & = & -2\alpha + \beta  \\
				0 & = & \alpha + \gamma
			\end{array}
			\right.
			\Lra
			\left\{
			\begin{array}{rcl}
				\alpha & = & -\gamma \\
				\alpha & = & \beta/2 \\
				\alpha & = & 1/5
			\end{array}
			\right.
			\\
			\beforetext{Ainsi,}
			\left\{
			\begin{array}{rcl}
				\alpha & = & 1/5  \\
				\gamma & = & -1/5 \\
				\beta  & = & 2/5
			\end{array}
			\right.
			\\
			\beforetext{Soit}
			\boxed{R = k\times \Ec^{1/5}t^{2/5}\rho^{-1/5}}
		\end{gather*}
	\end{tcn}
}%

\QR{%
	L'analyse du film montre que le rayon augmente au cours du temps comme
	$t^{2/5}$. Exprimer alors $\Ec$ en fonction de $R$, $\rho$ et $t$.
}{%
	On isole simplement en mettant la relation à la puissance 5~: \fbox{$\Ec
			= k^{-5}R^5 t^{-2}\rho$}.
}%

\QR{%
	En estimant que $R\approx \SI{70}{m}$ après $t = \SI{5}{ms}$, sachant
	que la masse volumique de l'air vaut $\rho\approx \SI{1.2}{kg.m^{-3}}$
	et en prenant $k\approx 1$, calculer la valeur de $\Ec$ en joules puis en
	kilotonnes de TNT (une tonne de TNT libère \SI{4.18e9}{J}).
}{%
	On fait une simple application numérique~:
	\[\Ec = \SI{8.0e13}{J}\quad\text{avec}\quad \left\{
		\begin{array}{rcl}
			k    & = & 1                   \\
			R    & = & \SI{70}{m}          \\
			t    & = & \SI{5e-3}{s}        \\
			\rho & = & \SI{1.2}{kg.m^{-3}}
		\end{array}
		\right.\]
	En équivalent tonne de TNT, on trouve~:
	\[\xul{\Ec = \SI{19}{kT}\text{ de TNT}}\]
	\begin{tcn}(rema){Remarque}
		En réalité, l'explosion aura été estimée à $\SIrange{21}{24}{kT}\text{ de
				TNT}$ des années plus tard, connaissant la composition de la bombe.
		Cependant, l'analyse dimensionnelle marche très bien, même si le $k \approx
			1$ n'était pas évident. Voir le billet de blog de
		\href{https://scienceetonnante.com/2021/10/01/analyse-dimensionnelle/}{Science
			Étonnante} joint à sa vidéo sur l'analyse dimensionnelle -- que je vous
		recommande aussi.
	\end{tcn}
}%

\end{document}
