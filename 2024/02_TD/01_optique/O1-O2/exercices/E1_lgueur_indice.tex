\documentclass[../TDO1-O2.tex]{subfiles}%

\begin{document}
\section[s]"1"{Fréquence, longueur d'onde et indice}
\enonce{%
	La lumière visible possède des longueurs d'onde dans le vide comprises entre
	\SIrange{400}{800}{nm}.
}%
\QR{%
	À quel intervalle de fréquences cela correspond-il~?
}{%
	On définit $\lambda_{r,0} = \SI{800}{nm}$ et $\lambda_{b,0} =
		\SI{400}{nm}$ les longueurs d'ondes dans le vide correspondant au
	extrémités bleue et rouge du spectre de la lumière visible.\bigbreak
	On sait que
	\begin{equation*}
		f = \frac{c}{\lambda_0}
	\end{equation*}
	On aura donc\smallbreak
	\begin{minipage}{0.45\linewidth}
		\begin{empheq}[box=\fbox]{align*}
			f_b &= \frac{c}{\lambda_{b,0}}\\
			f_r &= \frac{c}{\lambda_{r,0}}
		\end{empheq}
	\end{minipage}
	\begin{minipage}{0.45\linewidth}
		\begin{equation*}
			\text{avec}\quad
			\left\{
			\begin{array}{rcl}
				c             & = & \SI{3.00e8}{m.s^{-1}}          \\
				\lambda_{b,0} & = & \SI{400}{nm} = \SI{4.00e-7}{m} \\
				\lambda_{r,0} & = & \SI{800}{nm} = \SI{8.00e-7}{m}
			\end{array}
			\right.
		\end{equation*}
	\end{minipage}\smallbreak
	L'application numérique donne
	\begin{empheq}[box=\fbox]{align*}
		f_b &= \SI{7.50e14}{Hz} = \SI{750}{THz}\\
		f_r &= \SI{3.80e14}{Hz} = \SI{380}{THz}
	\end{empheq}
}%

\QR{%
	Que deviennent ces longueurs d'ondes
	\begin{enumerate}[label=\alph* --]
		\item dans l'eau d'indice $n_1 = \num{1.33}$~?
		\item dans un verre d'indice $n_2 = \num{1.5}$~?
	\end{enumerate}
}{%
	Dans un milieu TLHI, la longueur d'onde change de valeur selon
	\begin{equation*}
		\lambda = \frac{\lambda_0}{n}
	\end{equation*}
	Ainsi,
	\begin{enumerate}[label=\alph* --]
		\item dans l'eau d'indice $n_1 = \num{1.33}$,
		      \begin{empheq}[box=\fbox]{align*}
			      \lambda_{b,\rm eau} &= \SI{300}{nm} \\
			      \lambda_{r,\rm eau} &= \SI{602}{nm}
		      \end{empheq}
		\item dans un verre d'indice $n_2 = \num{1.5}$,
		      \begin{empheq}[box=\fbox]{align*}
			      \lambda_{b,\rm eau} &= \SI{267}{nm} \\
			      \lambda_{r,\rm eau} &= \SI{533}{nm}
		      \end{empheq}
	\end{enumerate}
	Leur couleur ne change cependant pas puisque \textbf{la couleur d'une
		lumière est définie par sa fréquence/longueur d'onde dans le vide}.
}%

\QR{%
	Calculer la valeur de la vitesse de la lumière dans un verre d'indice
	$n = \num{1.5}$.
}{%
	Dans un milieu TLHI, la vitesse de la lumière se calcule avec
	\begin{equation*}
		v = \frac{c}{n}
	\end{equation*}
	Avec $n = \num{1.5}$, on a donc
	\begin{empheq}[box=\fbox]{equation*}
		v = \SI{2.00e8}{m.s^{-1}}
	\end{empheq}

}%

\end{document}
