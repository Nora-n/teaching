\documentclass[../TDO1-O2.tex]{subfiles}%

% TODO: Schéma !

\begin{document}
\section[s]"1"{Incidence de \textsc{Brewster}}
\QR{%
	Un dioptre plan sépare l'air d'un milieu d'indice $n$. Pour quelle valeur de
	l'angle d'incidence le rayon réfléchi est-il perpendiculaire au rayon
	réfracté~?
}{%
	Les rayons réfléchis et réfractés sont perpendiculaires si $r+i =
		\frac{\pi}{2} \Leftrightarrow r = \frac{\pi}{2} -i$. En venant de l'air, on
	a $\sin i = n\sin r$, soit $\sin i = n\cos i$~; autrement dit
	\begin{equation*}
		\boxed{\tan i = n}
	\end{equation*}
}%

\end{document}
