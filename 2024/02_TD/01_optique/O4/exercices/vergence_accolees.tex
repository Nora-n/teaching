\documentclass[../TDO4.tex]{subfiles}%

\begin{document}
\section[s]"1"{Vergence et grandissement de lentilles accolées}
\enonce{%
	Soit le système de deux lentilles $\Lc_1$ et $\Lc_2$, de centres optiques
	$O_1$ et $O_2$ et de vergences $V_1$ et $V_2$ qui sont \textit{accolées}
	(c'est-à-dire de même axe optique et de centres optiques confondus~: dans la
	pratique, on veut $|\obar{O_1O_2}| \ll |f_1'|$ et $\ll |f_2'|$ simultanément).

}%

\QR{%
	Montrer qu'il est équivalent à une lentille $\Lc$ de vergence $V = V_1
		+ V_2$.
}{%
	Dans cette situation, on a le système $\rm A \opto{\Lc_1}{\rm O_1} A_1
		\opto{\Lc_2}{O_2} A'$, avec $\rm O_1 = O_2 = O$. Une lentille équivalente à ce
	système ferait passer directement de A à A' et aurait une distance
	focale $\OFp$ telle que
	\begin{equation}\label{eq:vaccol}
		\frac{1}{\OFp} = \frac{1}{\OAp} - \frac{1}{\OA}
	\end{equation}
	Pour faire apparaître les vergences des lentilles une et deux, on
	peut~:
	\begin{enumerate}
		\item Écrire les relations de conjugaison pour les deux lentilles~:
		      \begin{equation*}
			      \boxed{\frac{1}{\obarr{OF'_1}} =
				      \frac{1}{\obarr{OA_1}} -
				      \frac{1}{\OA}}
			      \qet
			      \boxed{\frac{1}{\obarr{OF'_2}} =
				      \frac{1}{\OAp} -
				      \frac{1}{\obarr{OA_1}}}
		      \end{equation*}

		\item Ou directement dans~\ref{eq:vaccol} ajouter et retirer
		      $\frac{1}{\obarr{OA_1}}$ dans le terme de droite.
	\end{enumerate}
	Quoiqu'il en soit, on trouve rapidement
	\begin{align*}
		\frac{1}{\OFp} & =
		\frac{1}{\obarr{OF'_1}} + \frac{1}{\obarr{OF'_2}}
		\\\Lra
		V              & = V_1 + V_2
	\end{align*}
}%

\QR{%
	Préciser le grandissement de l'ensemble en fonction du grandissement
	de chaque lentille.
}{%
	Le grandissement de l'ensemble est $\gamma =
		\dfrac{\OAp}{\OA}$. Or, $\gamma_1 =
		\dfrac{\obarr{OA_1}}{\OA}$ et
	$\gamma_2 = \dfrac{\OAp}{\obarr{OA_1}}$~; on a donc
	\[
		\boxed{
			\gamma = \gamma_1\gamma_2
		}
	\]
	\begin{tcn}[label=rema:prod_gamma](prop){Produit des grandissements}
		Si le théorème des vergences (c'est son nom) ne vaut que pour des lentilles
		accolées (une version plus générique s'écrit $V = V_1 + V_2 -
			\obarr{O_1O_2}V_1V_2$), l'expression du grandissement \textbf{vaut pour toute
			association}.
	\end{tcn}
}%

\end{document}
