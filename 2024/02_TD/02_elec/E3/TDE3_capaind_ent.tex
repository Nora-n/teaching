\documentclass[../../main/main.tex]{subfiles}
\graphicspath{{./figures/}}

\makeatletter
\renewcommand{\@chapapp}{Électrocinétique -- chapitre}
\renewcommand{\chaplett}{E}
\makeatother

\def\sspace{50}

\hfuzz=5.003pt

% \toggletrue{student}
% \toggletrue{corrige}
% \renewcommand{\mycol}{black}
% \renewcommand{\mycol}{gray}

\begin{document}
\setcounter{chapter}{2}

\settype{enon}
\settype{solu_prof}
\settype{solu_stud}

\chapter{\cswitch{Correction du TD d'entraînement}{TD entraînement~: capacités et inductances}}

\resetQ
\section[s]"2"{Circuit RL -- RC}
\subimport{/home/nora/Documents/Enseignement/Prepa/bpep/exercices/TD/circuit_RL_RC/}{sujet.tex}

\resetQ
\subfile{exercices/RC_2mailles.tex}

\resetQ
\section[s]"2"{Régime transitoire d’un circuit RC}
\subimport{/home/nora/Documents/Enseignement/Prepa/bpep/exercices/TD/regime_transitoire_RC/}{sujet.tex}

\end{document}
