\documentclass[../TDE6_rsf.tex]{subfiles}%

\begin{document}
\section[s]"2"{Exploitation d'un oscillogramme en RSF}
\enonce{%
	On considère le circuit ci-dessous. On pose $e(t) = E_m\cos(\wt)$ et $u(t) =
		U_m\cos(\wt+\f)$. La figure ci-dessous représente un oscillogramme réalisé à la
	fréquence $f = \SI{1.2e3}{Hz}$, avec $R = \SI{1.0}{k\Omega}$ et $C =
		\SI{0.10}{\micro F}$.
	\begin{center}
		\includegraphics[width=\linewidth]{exo3_plain}
	\end{center}
}
\QR{%
	Déduire de cet oscillogramme les valeurs expérimentales de $E_m$,
	$U_m$ et $\f$.
}{%
	On lit l'amplitude de $e(t)$ à son maximum pour avoir \fbox{$E_m =
			\SI{10}{V}$}. On lit l'amplitude de $u(t)$ à son maximum pour avoir
	\fbox{$U_m = \SI{6}{V}$}. Pour la phase \textbf{à l'origine des temps},
	on regarde le signal à $t = 0$~: on lit $u(0) = U_m\cos(\f) =
		\SI{-3}{V}$, soit
	\begin{gather*}
		\boxed{\cos(\f) = \frac{u(0)}{U_m}}
		\qavec
		\left\{
		\begin{array}{rcl}
			u(0) & = & \SI{-3}{V} \\
			U_m  & = & \SI{6}{V}
		\end{array}
		\right.\\
		\mathrm{A.N.~:}\quad
		\boxed{\f = \frac{2\pi}{3}\si{rad}}
	\end{gather*}
}

\QR{%
	Exprimer $U_m$ et $\f$ en fonction des composants du circuit.
}{%
	On utilise un pont diviseur de tension pour avoir l'amplitude
	complexe~:
	\begin{gather*}
		\Uu
		= \frac{\Zu_L}{\Zu_R + \Zu_C + \Zu_L}E_m
		= \frac{1}{\frac{\Zu_R}{\Zu_L} + \frac{\Zu_C}{\Zu_L}
			+ \frac{\Zu_L}{\Zu_L}}E_m
		\Lra
		\Uu
		= \frac{1}{1 + \frac{R}{\jj L\w} + \frac{1}{\jj^2\w^2CL}}E_m\\
		\Lra
		\boxed{
			\Uu
			= \frac{1}{1 -\jj \frac{R}{L\w} - \frac{1}{\w^2LC}}E_m
		}
	\end{gather*}
	On peut en vérifier l'homogénéité en se souvenant des résultats des
	chapitres précédents~:
	\begin{gather*}
		\w_0{}^2 = \frac{1}{LC}
		\qdonc
		\w^2LC \text{ adimensionné}
		\qet
		\frac{R}{L} = \tau^{-1}
		\qdonc
		\frac{R}{L\w} \text{ adimensionné}
	\end{gather*}
	D'une manière générale, on exprimera les résultats de la sorte, avec une
	fraction dont le numérateur est homogène à la quantité exprimée alors
	que le dénominateur est adimensionné. \bigbreak
	On trouve l'amplitude réelle en prenant le module de cette expression~:
	\begin{gather*}
		U_m
		= \abs{ \Uu }
		\Lra
		\boxed{U_m
			= \frac{E}{\sqrt{\left(1 - \frac{1}{LC\w^2}\right)^2 +
					\frac{R^2}{L^2\w^2}}}
		}
	\end{gather*}
	On trouve la phase en en prenant l'argument~:
	\begin{gather*}
		\f
		= \arg{\Uu}
		= \underbracket[1pt]{\cancel{\arg{E}}}_{=0}
		- \arg*{1 - \frac{1}{LC\w^2} - \jj \frac{R}{L\w}}
		\\
		\Lra
		\tan(\f)
		= - \left(-\frac{R}{L\w}\times \frac{1}{1 - \frac{1}{LC\w^2}}\right)
		= \frac{R}{L\w - \frac{1}{C\w}}
		\Lra
		\boxed{\tan(\f)
			= \frac{RC\w}{LC\w^2 - 1}
		}
	\end{gather*}
	Ici, il n'est pas évident de prendre l'arctangente de la tangente~: la
	partie réelle de l'argument calculé n'est pas forcément positif (il
	l'est si $\w^2 > \frac{1}{LC}$).
}

\QR{%
	En déduire la valeur numérique de l'inductance $L$ de la bobine.
}{%
	Il paraît évidemment plus simple de calculer $L$ à partir de la phase,
	sachant qu'on a déterminé $\f$ à la première question~:
	\begin{gather*}
		LC\w^2 - 1
		= \frac{RC\w}{\tan(\f)}
		\Lra
		LC\w^2 = 1 + \frac{RC\w}{\tan(\f)}\\
		\Lra
		\boxed{L = \frac{1}{C\w^2} + \frac{R}{\w\tan(\f)}}
		\qavec
		\left\{
		\begin{array}{rcl}
			C  & = & \SI{0.10}{\micro F}    \\
			\w & = & 2\pi f                 \\
			f  & = & \SI{1.2e3}{Hz}         \\
			R  & = & \SI{1}{k\Omega}        \\
			\f & = & \frac{2\pi}{3}\si{rad}
		\end{array}
		\right.\\
		\mathrm{A.N.~:}\quad
		\boxed{L = \SI{9.9e-2}{H}}
	\end{gather*}
}
\end{document}
