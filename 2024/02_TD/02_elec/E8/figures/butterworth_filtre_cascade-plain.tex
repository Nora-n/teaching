\documentclass{standalone}
\usepackage{mintikz}

\begin{document}
\begin{circuitikz}[line width=.7pt]
	\draw
	(0,0)
	node [fourport] (F1) {}
	(3,0)
	node [fourport] (F2) {}
	(F1.port1) --++ (-1,0) node [pos=.8] (bleft) {}
	(F1.port4) --++ (-1,0) node [pos=.8] (tleft) {}
	(F1.port2) --
	node[ground, midway] (M) {}
	node[pos=.3] (bmid) {}
	(F2.port1)
	(F1.port3) --
	node[pos=.3] (tmid) {}
	(F2.port4)
	(F2.port2) --++ (1,0) node [pos=.8] (bright) {}
	(F2.port3) --++ (1,0) node [pos=.8] (tright) {};
	\draw[color=red!70]
	(bleft) --
	node [left] {$v_1$}
	(tleft)
	node[currarrow, sloped, anchor=tip, pos=1, allow upside down] {}
	(bmid) --
	node [right] {$v_2$}
	(tmid)
	node[currarrow, sloped, anchor=tip, pos=1, allow upside down] {}
	(bright) --
	node [left] {$v_3$}
	(tright)
	node[currarrow, sloped, anchor=tip, pos=1, allow upside down] {}
	;
	\node[] at (F1.center) {Ordre 1};
	\node[] at (F2.center) {Ordre 2};
\end{circuitikz}
\end{document}
