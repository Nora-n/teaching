\documentclass[../TDTM1.tex]{subfiles}%

\begin{document}
\section[s]"1"{Concentration en soluté apporté}
\begin{tcn}(data)<lftt>{}
	\[
		M(\ce{Mg}) = \SI{24.3}{g.mol^{-1}}
		\qet
		M(\ce{Cl}) = \SI{35.5}{g.mol^{-1}}
	\]

\end{tcn}

\QR{
Identifier les ions présents dans l'acide sulfurique $\ce{H_2SO_4}$. Écrire
l'équation de dissolution.
}{
Ce sont les ions $\ce{H^{+}}$ et $\ce{SO_4^{2-}}$. L'équation de la
dissolution s'écrit
\[
	\ce{H_2SO_4\sol{} -> 2H^{+}\aqu{} + SO_4^{2-}\aqu{}}
\]
}
\QR{
On ajoute une quantité de matière $n_{\rm app} = \SI{2e-2}{mol}$ en acide
sulfurique dans de l'eau distillée. Déterminer les quantités de matière de
chaque ion dans la solution formée.
}{
D'après l'équation de dissolution, une molécule de solide libère deux ions
$\ce{H^{+}}$ et un ion $\ce{SO_4^{2-}}$. On en déduit $n_{\ce{H^{+}}} =
	2n_{\rm app} = \SI{4e-2}{mol}$ et $n_{\ce{SO_4^{2-}}} = n_{\rm app} =
	\SI{2e-2}{mol}$.
}
\QR{
La solution des questions précédentes a un volume $V = \SI{200}{mL}$. Calculer
la concentration en soluté apporté, puis les concentrations des ions dans la
solution après dissolution.
}{
$C_{\rm app} = \frac{n_{\rm app}}{V} = \SI{0.1}{mol.L^{-1}}$~; $[\ce{H+}] =
	\SI{0.2}{mol.L^{-1}}$ et $[\ce{SO_4^{2-}}] = \SI{0.1}{mol.L^{-1}}$.
}
\QR{
	On considère une solution de chlorure de chrome $\ce{CrCl_3}$ de concentration
	en soluté apporté $c = \SI{5e-3}{mol.L^{-1}}$. Déterminer les concentrations
	des ions dans la solution.
}{
	L'équation de dissolution s'écrit
	\[
		\ce{CrCl_3\sol{} -> Cr^{3+}\aqu{} + 3Cl^{-}\aqu{}}
	\]
	On en déduit
	\[
		\ce{[Cr^{3+}]} = c = \SI{5e-3}{mol.L^{-1}}
		\qet
		[\ce{Cl^{-}}] = 3c = \SI{1.5e-2}{mol.L^{-1}}
	\]
}
\QR{
	On dissout $m = \SI{6.0}{g}$ de chlorure de magnésium $\ce{MgCl_2}$ dans
	\SI{200}{mL} d'eau distillée. Calculer la concentration en soluté apporté,
	puis les concentrations des ions dans la solution
}{
	Raisonnons sur la quantité de matière apportée~:
	\[
		n_{\rm app} = \frac{m}{M_{\ce{Mg}} + 2M_{\ce{Cl}}}
		\qdc
		C_{\rm app} = \frac{m}{(M_{\ce{Mg}} + 2M_{\ce{Cl}})V} = \SI{0.315}{mol.L^{-1}}
	\]
	L'équation de dissolution s'écrit
	\[
		\ce{MgCl_2\sol{} -> Mg^{2+}\aqu{} + 2Cl^{-}\aqu{}}
	\]
	Ainsi,
	\[
		[\ce{Mg^{2+}}] = c\ind{app} = \SI{0.32}{mol.L^{-1}}
		\qqet
		[\ce{Cl^-}] = 3c\ind{app} = \SI{0.96}{mol.L^{-1}}
	\]
}

\end{document}
