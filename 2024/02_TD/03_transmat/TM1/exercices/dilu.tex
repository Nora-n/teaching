\documentclass[../TDTM1.tex]{subfiles}%

\begin{document}
\section[s]"1"{Dilution et mélange}

On dispose d'une solution de sulfate de cuivre contenant les ions $\ce{Cu^{2+}}$
et les ions sulfate $\ce{SO_4^{2-}}$ à la même concentration $C_0 =
	\SI{1e-2}{mol.L^{-1}}$. On en prélève à la pipette jaugée un volume $V_0 =
	\SI{10}{mL}$ que l'on verse dans une fiole jaugée de volume $V_1 = \SI{50}{mL}$.
On remplit la fiole d'eau distillée jusqu'au trait de jauge.

\QR{
Quelle est la concentration $C_1$ en ions $\ce{Cu^{2+}}$ et en ions
$\ce{SO_4^{2-}}$ dans la fiole~?
}{
On note $n_0$ la quantité de matière prélevée. Attention, $V_1$ est le volume
\textbf{total} de la fiole, différent du volume d'eau ajouté. Ainsi,
\[
	C_1 = \frac{n_0}{V_1} = \frac{C_0V_0}{V_1} = \SI{2e-3}{mol.L^{-1}}
\]
}

On verse le contenu de cette fiole dans un bécher. On y ajoute un volume $V_2 =
	\SI{20}{mL}$ d’une solution de sulfate de magnésium, contenant les ions
$\ce{Mg^{2+}}$ et les ions $\ce{SO_4^{2-}}$ à la même concentration $C_2 =
	\SI{2e-2}{mol.L^{-1}}$.

\QR{
	Calculer les concentrations des trois ions après le mélange.
}{
	Les ions cuivre ne viennent que de la solution 1, les ions magnésium que de la
	solution 2, mais les ions sulfate sont apportés par les deux solutions.
	\begin{align*}
		[\ce{Cu^{2+}}]   & = \frac{n_{\ce{Ce^{2+}},1}}{V_{\tot}} = \frac{C_1V_1}{V_1+V_2}
		= \SI{1.4e-3}{mol.L^{-1}}
		\\
		[\ce{Mg^{2+}}]   & = \frac{n_{\ce{Mg^{2+}},2}}{V_{\tot}} = \frac{C_2V_2}{V_1+V_2}
		= \SI{5.7e-3}{mol.L^{-1}}
		\\
		[\ce{SO_4^{2-}}] & = \frac{n_{\ce{SO_4^{2-}},1} + n_{\ce{SO_4^{2-}},2}}{V_{\tot}}
		= \frac{C_1V_1 + C_2V_2}{V_1+V_2} = \SI{7.1e-3}{mol.L^{-1}}
	\end{align*}
}

\end{document}
