\documentclass[../TDTM1.tex]{subfiles}%

\begin{document}
\section[s]"1"{Calculs de quantités de matière}

\begin{tcn}(data)<lftt>{Données}
	\[
		M(\ce{Fe}) = \SI{55.8}{g.mol^{-1}}
		\qet
		M(\ce{Cu}) = \SI{63.5}{g.mol^{-1}}
	\]
\end{tcn}

\QR{
	On verse dans un bécher une masse $m = \SI{350}{mg}$ de poudre de fer
	métallique. Quelle est la quantité de matière $n_{\ce{Fe}}$ correspondante~?
}{
	$n_{\ce{Fe}} = \frac{m}{M_{\ce{Fe}}} = \SI{6.3e-3}{mol} = \SI{6.3}{mmol}$
}

\QR{
	On dispose d'un flacon contenant $V_0 = \SI{800}{mL}$ de solution de sulfate
	de cuivre contenant les ions $\ce{Cu^{2+}}$ à la concentration $C =
		\SI{0.50}{mol.L^{-1}}$. Quelle est la quantité de matière correspondante~?
}{
	$n_0 = CV_0 = \SI{0.4}{mol}$
}

\QR{
	On prélève $V = \SI{50}{mL}$ de cette solution. Quelle est la concentration du
	prélèvement~? Quelle est la quantité de matière $n_{\ce{Cu^{2+}}}$ prélevée~?
}{
	Le prélèvement est à la même concentration $C$ que la solution mère~:
	\[
		n_{\ce{Cu^{2+}}} = CV = \SI{2.5e-2}{mol} = \SI{25}{mmol}
	\]
}

Le prélèvement est versé dans le bécher~; une transformation chimique a lieu.

\QR{
	À l'issue de cette transformation, on obtient du cuivre métallique en quantité
	de matière $n_f = \SI{4.8}{mmol}$. Quelle est la masse correspondante~?
}{
	$m_{\ce{Cu}} = n_f M_{\ce{Cu}} = \SI{0.30}{g} = \SI{300}{mg}$
}

\QR{
	On obtient également la même quantité de matière $n_f$ d'ions $\ce{Fe^{2+}}$.
	Quelle est la concentration correspondante~?
}{
	$[\ce{Fe^{2+}}]_f = \frac{n_f}{V} = \SI{9.6e-2}{mol.L^{-1}}$
}
\end{document}
