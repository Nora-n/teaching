\documentclass[../TDTM2.tex]{subfiles}%

\begin{document}
\section[s]"2"{Coefficient de dissociation}
\enonce{%
	On considère l'équilibre de l'eau en phase gazeuse~:
	\[
		\ce{2H2O = 2H2_{\gaz} + O2_{\gaz}}
	\]
}

\QR{%
	On se place à \SI{400}{K} sous une pression constante $P = \SI{1.00}{bar}$.
	Sous quelle forme se trouve l'eau~?
}{%
	Pour connaître l'état de l'eau, on détermine la température en degrés
	pour interpréter par des connaissances élémentaires si elle est solide
	(glace), liquide, ou vapeur~: \SI{400}{K} = \SI{127}{\degreeCelsius}, et
	la pression est de \SI{1}{bar}, c'est-à-dire presque la pression
	habituelle. À cette température, l'eau est sous forme vapeur.
}
\QR{%
	La valeur de la constante vaut $K^\circ(\SI{400}{K}) = \num{3.12e-59}$. Conclure
	sur la stabilité de l'eau dans ces conditions.
}{%
	La constante est extrêmement petite~: $K^\circ \ll 10^{\num{-4}}$, donc la
	réaction est \textbf{quasi-nulle} dans ce sens~: l'eau ne se dissocie
	pratiquement pas de cette manière et est par conséquent très stable.
}
\QR{%
	En supposant que l'on introduit de l'eau pure, calculer le coefficient
	de dissociation de l'eau.\\
	\textit{Rappel}~: le coefficient de dissociation $\alpha$ est
	égal au rapport de la quantité ayant été dissociée sur la quantité
	initiale.
}{%
	Si on introduit de l'eau pure, on n'a pas les autres composants au
	départ. Soit $n_0$ la quantité de matière d'eau pure introduite~: on
	dresse le tableau d'avancement~:
	\begin{center}
		\def\rhgt{0.35}
		\centering
		\begin{tabularx}{\linewidth}{|l|c||YdYdY||Y|}
			\hline
			\multicolumn{2}{|c||}{
				$\xmathstrut{\rhgt}$
			\textbf{Équation}}  &
			$2\ce{H_2O_{\gaz}}$ & $=$                  &
			$2\ce{H_2_{\gaz}}$  & $+$                  &
			$\ce{O_2_{\gaz}}$   &
			$n\ind{tot, gaz}$                            \\
			\hline
			$\xmathstrut{\rhgt}$
			Initial             & $\xi = 0$            &
			$n_0$               & \vline               &
			$0$                 & \vline               &
			$0$                 &
			$n_0$                                        \\
			\hline
			$\xmathstrut{\rhgt}$
			Final               & $\xi_f = \xi_{\eql}$ &
			$n_0 - 2\xi_{\eql}$ & \vline               &
			$2\xi_{\eql}$       & \vline               &
			$\xi_{\eql}$        &
			$n_0 + \xi_{\eql}$                           \\
			\hline
		\end{tabularx}
	\end{center}
	Le coefficient de dissociation correspond à la quantité d'eau
	transformée sur la quantité initiale, c'est-à-dire
	\[\alpha = \frac{2\xi_{\eql}}{n_0}\]
	On va donc exprimer la constante d'équilibre en fonction des quantités
	de matière pour introduire $\xi_{\eql}$ et faire apparaître $\alpha$, à l'aide
	de l'activité d'un gaz, de la loi de \textsc{Dalton} puis de la
	définition de la fraction molaire~:
	\begin{gather*}
		K^\circ = \frac{p_{\ce{H2}}{}^2p_{\ce{O2}}}
		{p_{\ce{H2O}}{}^2p^\circ}
		\Lra
		K^\circ = \frac{n_{\ce{H2}}{}^2n_{\ce{O2}}}
		{n_{\ce{H2O}}{}^2n\ind{tot, gaz}}
		\overbracket[1pt]{\frac{p}{p^\circ}}^{=1}
		\\\Lra
		K^\circ = \frac{4\xi_{\eql}^3}{(n_0-2\xi_{\eql})^2(n_0+\xi_{\eql})}
		\Lra
		K^\circ = \underbracket[1pt]{\cancel{\frac{n_0{}^2}{n_0{}^3}}}_{=1}
		\frac{4 \left(\frac{\xi_{\eql}}{n_0}\right)^3}
		{\left( 1- \frac{2\xi_{\eql}}{n_0} \right)^2
			\left( 1+ \frac{\xi_{\eql}}{n_0} \right)}
		\\\Lra
		K^\circ = \frac{4 \left(\frac{\alpha}{2}\right)^3}
		{\left( 1- \alpha \right)^2
			\left( 1+ \frac{\alpha}{2} \right)}
		\Lra
		\boxed{
			K^\circ = \frac{\alpha^3}
			{\left( 1- \alpha \right)^2
				\left( 2+ \alpha \right)}
		}
	\end{gather*}
	Une résolution numérique (\texttt{Python} ou calculatrice) donne
	\[\boxed{\alpha = \num{3.97e-20} \ll 1}\]
	Ceci est en accord avec le très faible avancement de la réaction.
}
\QR{%
	À \SI{3000}{K}, toujours sous une pression de \SI{1}{bar}, le coefficient de
	dissociation vaut $\alpha = \num{0.30}$. Calculer $K^\circ(\SI{3000}{K})$.
	Conclure sur la stabilité de l'eau dans ces conditions.
}{%
	En prenant $\alpha = \num{0.30}$, cela veut dire que 30\% de l'eau se dissocie,
	l'eau ne serait plus stable dans ces conditions. La valeur de $K^\circ$
	correspondant est $K^\circ = \num{2.4e-2}$, ce qui est peu favorisé dans le
	sens direct mais pas quasi-nulle.
}
\end{document}
