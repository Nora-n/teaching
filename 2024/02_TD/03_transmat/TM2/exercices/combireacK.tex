\documentclass[../TDTM2.tex]{subfiles}%

\begin{document}
\section[s]"1"{Combinaisons de réactions et constantes d'équilibre}
\QR{%
	Montrer que, pour des réactions numérotées (1) et (2), de constantes de
	réactions $K_1^\circ$ et $K_2^\circ$ respectivement, alors une réaction $(3) =
		\alpha(1) + \beta(2)$ a pour constante
	\[
		K_3^\circ = K_1^\circ{}^{\alpha} \times K_2^\circ{}^{\beta}
	\]
}{%
	Dans cet exercice, on introduit le lien entre relation sur les équations-bilan
	et les constantes d'équilibre associées. En effet, on a vu dans le cours que
	\[a{\ce{A}} + b{\ce{B}} = c{\ce{C}} + d{\ce{D}}\]
	a pour constante d'équilibre
	\[K_1^\circ = \prod_i a({\ce{X}}_i)^{\nu_{i, 1}}\]
	Si on inverse la réaction pour avoir
	\[c{\ce{C}} + d{\ce{D}} = a{\ce{A}} + b{\ce{B}}\]
	alors on prend l'opposé de chaque coefficient stœchiométrique~: $\nu_{i, 2} =
		- \nu_{i,1}$, ce qui fait que cette réaction a pour constante d'équilibre
	\[
		K_2^\circ
		= \prod_i a({\ce{X}}_i)^{\nu_{i,2}}
		= \prod_i a({\ce{X}}_i)^{-\nu_{i,1}}
		= \left(\prod_i a({\ce{X}}_i)^{\nu_{i,1}}\right)^{-1}
		= \left( K_1^\circ \right)^{-1}
	\]

	Le même raisonnement tient pour montrer que
	\[2a{\ce{A}} + 2b{\ce{B}} = 2c{\ce{C}} + 2d{\ce{D}}\]
	a pour constante d'équilibre
	\[K_3^\circ = K_1^\circ{}^2\]

	On étend le raisonnement pour montrer que si on ajoute deux réactions (1) et
	(2) pour avoir une équation (3), alors on aura $K_3^\circ = K_1^\circ\times
		K_2^\circ$, et que si on a $(3) = \alpha(1) + \beta(2)$, alors on a bien
	\begin{gather*}
		K_3^\circ = K_1^\circ{}^{\alpha}\times K_2^\circ{}^{\beta}
		\qed
	\end{gather*}
}%

\QR{%
On considère les réactions numérotées $(1)$ et $(2)$ ci-dessous~:
\[\ce{4Cu_{\sol} + O2_{\gaz} = 2Cu2O_{\sol}}\quad K_1^\circ
	\qqet
	\ce{2Cu2O_{\sol} + O2_{\gaz} = 4CuO_{\sol}}\quad K_2^\circ\]
Exprimer les constantes d'équilibre des trois réactions ci-dessous en fonction
de $K_1^\circ$ et $K_2^\circ$~:
\[
	\ce{2Cu_{\sol} + O2_{\gaz} = 2CuO_{\sol}}
	\quad ; \quad
	\ce{8Cu_{\sol} + 2O2_{\gaz} = 4Cu2O_{\sol}}
	\quad ; \quad
	\ce{2Cu_2O_{\sol} = 4Cu_{\sol} + O2_{\gaz}}
\]
}{%
Ainsi, dans cet exercice il suffit de trouver les relations entre les
équations (3), (4), (5) et les équations (1) et (2) de constantes respectives
$K_1^\circ$ et $K_2^\circ$. On trouve alors~:
\begin{enumerate}[leftmargin=20pt, label=\alph* --]
	\item[m]
	      \begin{gather*}
		      (3)
		      = \frac{(1)+(2)}{2}
		      \Lra
		      K_3^\circ
		      = \left(K_1^\circ\times K_2^\circ\right)^{1/2}
		      = \sqrt{K_1^\circ K_2^\circ}
	      \end{gather*}
	\item[m]
	      \begin{gather*}
		      (4) = 2(1)
		      \Lra
		      K_4^\circ
		      = \left( K_1^\circ \right)^2
	      \end{gather*}
	\item[m]
	      \begin{gather*}
		      (5)
		      = -(1)
		      \Lra
		      K_5^\circ
		      = \left( K_1^\circ \right)^{-1}
	      \end{gather*}
\end{enumerate}
Tout ceci se vérifie bien sûr en écrivant les constantes de chacune des
réactions~:
\[
	K_1^\circ = \frac{p^\circ}{p_{\ce{O2}}}
	\quad ; \quad
	K_2^\circ = \frac{p^\circ}{p_{\ce{O2}}}
	\quad ; \quad
	K_3^\circ = \frac{p^\circ}{p_{\ce{O2}}}
	\quad ; \quad
	K_4^\circ = \frac{p^\circ{}^2}{p_{\ce{O2}}{}^2}
	\quad ; \quad
	K_5^\circ = \frac{p_{\ce{O2}}}{p^\circ}
\]
}
\end{document}
