\documentclass[../TDTM2.tex]{subfiles}%

\begin{document}
\section[s]"1"{Équilibre… ou pas~!}
\enonce{%
La dissociation du peroxyde de baryum sert à l'obtention de dioxygène avant la
mise au point de la liquéfaction de l'air, selon l'équation

\centersright{%
	$\ce{2BaO2_{\sol} <=> 2BaO_{\sol} + O2_{\gaz}}$%
}{%
	$K^\circ(\SI{795}{\degreeCelsius}) = \num{0.50}$%
}

Le volume de l'enceinte, initialement vide de tout gaz, vaut $V = \SI{10}{L}$.
On rappelle que $R = \SI{8.314}{J.K^{-1}.mol^{-1}}$.
}

\begin{blocQR}
	\item
	\QR{%
		Exprimer la constante d'équilibre $K^\circ$ en fonction de la pression
		partielle à l'équilibre $p_{\ce{O2}, \eql}$.
	}{%
		Par définition, $K^\circ = Q_{r, \eql}$. On exprime donc le
		quotient de réaction avec les activités à l'équilibre~:
		\[
			\boxed{K^\circ = \frac{p_{\ce{O2,\eql}}}{p^\circ}}
		\]
	}

	\QR{%
		En déduire la valeur numérique de $p_{\ce{O2}, \eql}$.
	}{%
		On a la valeur de $K^\circ$ et la valeur de $p^\circ$~: de
		l'équation précédente on isole $p_{\ce{O2}, \eql}$~:
		\begin{gather*}
			\boxed{p_{\ce{O2},\eql} = K^\circ p^\circ}
			\qavec
			\left\{
			\begin{array}{rcl}
				K^\circ & = & \num{0.50}     \\
				p^\circ & = & \SI{1.00}{bar}
			\end{array}
			\right.\\
			\AN
			\xul{p_{\ce{O2},\eql} = \SI{0.50}{bar} = \SI{5.0e4}{Pa}}
		\end{gather*}
	}
	\QR{%
		Calculer le nombre de moles de dioxygène qui permet d'atteindre
		cette pression dans l'enceinte.
	}{%
		Avec la loi des gaz parfaits, on a
		\begin{gather*}
			p_{\ce{O2},\eql}V = n_{\ce{O2}, \eql}RT
			\Lra
			\boxed{n_{\ce{O2}, \eql} = \frac{p_{\ce{O2},\eql}V}{RT}}
			\qavec
			\left\{
			\begin{array}{rcl}
				V & = & \SI{10e-3}{m^3} \\
				T & = & \SI{1068.15}{K}
			\end{array}
			\right.\\
			\AN
			\xul{n_{\ce{O2}, \eql} = \SI{0.056}{mol}}
		\end{gather*}
	}
\end{blocQR}
\begin{blocQR}
	\item Cas 1~:
	\enonce{
		\begin{center}
			\def\rhgt{0.50}
			\centering
			\begin{tabularx}{\linewidth}{|l|c||YdYdY||Y|}
				\hline
				\multicolumn{2}{|c||}{
					$\xmathstrut{\rhgt}$
				\textbf{Équation}}   &
				$2\ce{BaO_2_{\sol}}$ & $\rightleftharpoons$ &
				$2\ce{BaO_{\sol}}$   & $+$                  &
				$\ce{O_2_{\gaz}}$    &
				$n\ind{tot, gaz}$                             \\
				\hline
				$\xmathstrut{\rhgt}$
				Initial (\si{mol})   & $\xi = 0$            &
				$\num{0.20}$         & \vline               &
				$\num{0.00}$         & \vline               &
				$\num{0.00}$         &
				$\num{0.00}$                                  \\
				\hline
				$\xmathstrut{\rhgt}$
				Interm. (\si{mol})   & $\xi$                &
				                     & \vline               &
				                     & \vline               &
				                     &
				\\
				\hline
				$\xmathstrut{\rhgt}$
				Final  (\si{mol})    & $\xi = \xi_f$        &
				                     & \vline               &
				                     & \vline               &
				                     &
				\\
				\hline
			\end{tabularx}
		\end{center}
	}
	\ifcorrige{
		\begin{center}
			\def\rhgt{0.50}
			\centering
			\begin{tabularx}{\linewidth}{|l|c||YdYdY||Y|}
				\hline
				\multicolumn{2}{|c||}{
					$\xmathstrut{\rhgt}$
				\textbf{Équation}}   &
				$2\ce{BaO_2_{\sol}}$ & $\rightleftharpoons$ &
				$2\ce{BaO_{\sol}}$   & $+$                  &
				$\ce{O_2_{\gaz}}$    &
				$n\ind{tot, gaz}$                             \\
				\hline
				$\xmathstrut{\rhgt}$
				Initial (\si{mol})   & $\xi = 0$            &
				$\num{0.20}$         & \vline               &
				$\num{0.00}$         & \vline               &
				$\num{0.00}$         &
				$\num{0.00}$                                  \\
				\hline
				$\xmathstrut{\rhgt}$
				Interm. (\si{mol})   & $\xi$                &
				$\num{0.20} - 2\xi$  & \vline               &
				$2\xi$               & \vline               &
				$\xi$                &
				$\xi$                                         \\
				\hline
				$\xmathstrut{\rhgt}$
				Final  (\si{mol})    & $\xi = \xi_f$        &
				$\num{0.088}$        & \vline               &
				$\num{0.112}$        & \vline               &
				$\num{0.056}$        &
				$\num{0.056}$                                 \\
				\hline
			\end{tabularx}
		\end{center}
	}
	\QR{%
		Calculer le quotient de réaction initial $Q_{r,0}$ et en déduire le sens
		d'évolution du système.
	}{%
		On change juste $p_{\ce{O2},\eql}$ de la première question en
		$p_{\ce{O2}, 0}$~; sachant qu'on commence sans gaz dans l'enceinte,
		cette pression est nulle~:
		\[\boxed{Q_{r,0} = \frac{p_{\ce{O2}, 0}}{p^\circ} = 0}\]
		On a donc $Q_{r,0} < K$, et l'évolution se fait en sens direct.
	}
	\QR{%
		Remplir le tableau d'avancement et remplir la ligne intermédiaire dans le
		tableau en fonction de $\xi$.
	}{%
		Voir tableau.
	}
	\QR{%
		Déterminer $\xi_f$ en précisant si l'équilibre est atteint ou pas. On
		rappelle que l'équilibre correspond à la coexistence de toutes les
		espèces.
	}{%
		~
		\vspace{-15pt}
		\begin{tcb}(tool){État d'équilibre}
			Pour trouver l'état final dans cette situation, \textbf{on
				détermine $\xi_{\eql}$ s'il y avait équilibre, et on regarde
				si c'est compatible avec $\xi_{\max}$ si la réaction était
				totale}.
		\end{tcb}
		S'il y a équilibre, ça veut dire que $n_{\ce{O2}, \eql} =
			\SI{0.056}{mol}$ comme déterminé au début. Or, le tableau nous
		indique que $n_{\ce{O2}, f} = \xi_f$, donc si c'est un équilibre
		\xul{$\xi_{\eql} = \SI{0.056}{mol}$}.
		\bigbreak
		L'avancement est maximal si \ce{BaO2} est limitant~: on trouve donc
		$\xi_{\max}$ en résolvant $\num{0.20} - 2\xi_{\max} = 0$,
		c'est-à-dire \xul{$\xi_{\max} = \SI{0.1}{mol}$}.
		\bigbreak
		La valeur est finale $\xi_f$ est la plus petite valeur (en valeur
		absolue) de $\xi_{\eql}$ et $\xi_{\max}$~; or ici $\xi_{\eql} <
			\xi_{\max}$~: il y a donc bien équilibre, et on a
		\[\xul{\xi_f = \xi_{\eql} = \SI{0.056}{mol}}\]
	}
	\QR{%
		Remplir la dernière ligne du tableau d'avancement.
	}{%
		Voir tableau.
	}
\end{blocQR}
\QR{%
	Mêmes questions dans le cas 2~:
	\begin{center}
		\def\rhgt{0.50}
		\centering
		\begin{tabularx}{\linewidth}{|l|c||YdYdY||Y|}
			\hline
			\multicolumn{2}{|c||}{
				$\xmathstrut{\rhgt}$
			\textbf{Équation}}   &
			$2\ce{BaO_2_{\sol}}$ & $\rightleftharpoons$ &
			$2\ce{BaO_{\sol}}$   & $+$                  &
			$\ce{O_2_{\gaz}}$    &
			$n\ind{tot, gaz}$                             \\
			\hline
			$\xmathstrut{\rhgt}$
			Initial (\si{mol})   & $\xi = 0$            &
			$\num{0.10}$         & \vline               &
			$\num{0.00}$         & \vline               &
			$\num{0.00}$         &
			$\num{0.00}$                                  \\
			\hline
			$\xmathstrut{\rhgt}$
			Interm. (\si{mol})   & $\xi$                &
			                     & \vline               &
			                     & \vline               &
			                     &
			\\
			\hline
			$\xmathstrut{\rhgt}$
			Final  (\si{mol})    & $\xi = \xi_f$        &
			                     & \vline               &
			                     & \vline               &
			                     &
			\\
			\hline
		\end{tabularx}
	\end{center}
}{%
	Cas 2~:
	\begin{center}
		\def\rhgt{0.50}
		\centering
		\begin{tabularx}{\linewidth}{|l|c||YdYdY||Y|}
			\hline
			\multicolumn{2}{|c||}{
				$\xmathstrut{\rhgt}$
			\textbf{Équation}}       &
			$2\ce{BaO_2_{\sol}}$     & $\rightleftharpoons$ &
			$2\ce{BaO_{\sol}}$       & $+$                  &
			$\ce{O_2_{\gaz}}$        &
			$n\ind{tot, gaz}$                                 \\
			\hline
			$\xmathstrut{\rhgt}$
			Initial (\si{mol})       & $\xi = 0$            &
			$\num{0.10}$             & \vline               &
			$\num{0.00}$             & \vline               &
			$\num{0.00}$             &
			$\num{0.00}$                                      \\
			\hline
			$\xmathstrut{\rhgt}$
			Interm. (\si{mol})       & $\xi$                &
			$\num{0.10} - 2\xi$      & \vline               &
			$2\xi$                   & \vline               &
			$\xi$                    &
			$\xi$                                             \\
			\hline
			$\xmathstrut{\rhgt}$
			Final  (\si{mol})        & $\xi = \xi_f$        &
			$\ifcorrige{\num{0.00}}$ & \vline               &
			$\ifcorrige{\num{0.10}}$ & \vline               &
			$\ifcorrige{\num{0.05}}$ &
			$\ifcorrige{\num{0.05}}$                          \\
			\hline
		\end{tabularx}
	\end{center}
	\begin{enumerate}[leftmargin=20pt, label=\alph* --]
		\item On a toujours aucun gaz au départ, donc ici aussi
		      \[\boxed{Q_{r,0} = \frac{p_{\ce{O2}, 0}}{p^\circ} = 0}\]
		      et la réaction est en sens direct.
		\item Voir tableau.
		\item Même procédé~: \textbf{on détermine $\xi_{\eql}$ s'il y avait
			      équilibre, et on regarde si c'est compatible avec $\xi_{\max}$
			      si la réaction était totale}.
		      \bigbreak
		      S'il y a équilibre, ça veut dire que $n_{\ce{O2}, \eql} =
			      \SI{0.056}{mol}$ comme déterminé au début. Or, le tableau nous
		      indique que $n_{\ce{O2}, f} = \xi_f$, donc si c'est un équilibre
		      \xul{$\xi_{\eql} = \SI{0.056}{mol}$}.
		      \bigbreak
		      L'avancement est maximal si \ce{BaO2} est limitant~: on trouve donc
		      $\xi_{\max}$ en résolvant $\num{0.10} - 2\xi_{\max} = 0$,
		      c'est-à-dire \xul{$\xi_{\max} = \SI{0.050}{mol}$}.
		      \bigbreak
		      La valeur est finale $\xi_f$ est la plus petite valeur (en valeur
		      absolue) de $\xi_{\eql}$ et $\xi_{\max}$~; or ici $\xi_{\eql} >
			      \xi_{\max}$~: il n'y a donc \textbf{pas équilibre}, et on a
		      \[\xul{\xi_f = \xi_{\max} = \SI{0.050}{mol}}\]
		\item Voir tableau.
	\end{enumerate}
}
\QR{%
	Mêmes questions dans le cas 3~:
	\begin{center}
		\def\rhgt{0.50}
		\centering
		\begin{tabularx}{\linewidth}{|l|c||YdYdY||Y|}
			\hline
			\multicolumn{2}{|c||}{
				$\xmathstrut{\rhgt}$
			\textbf{Équation}}   &
			$2\ce{BaO_2_{\sol}}$ & $\rightleftharpoons$ &
			$2\ce{BaO_{\sol}}$   & $+$                  &
			$\ce{O_2_{\gaz}}$    &
			$n\ind{tot, gaz}$                             \\
			\hline
			$\xmathstrut{\rhgt}$
			Initial (\si{mol})   & $\xi = 0$            &
			$\num{0.10}$         & \vline               &
			$\num{0.050}$        & \vline               &
			$\num{0.10}$         &
			$\num{0.10}$                                  \\
			\hline
			$\xmathstrut{\rhgt}$
			Interm. (\si{mol})   & $\xi$                &
			                     & \vline               &
			                     & \vline               &
			                     &
			\\
			\hline
			$\xmathstrut{\rhgt}$
			Final  (\si{mol})    & $\xi = \xi_f$        &
			                     & \vline               &
			                     & \vline               &
			                     &
			\\
			\hline
		\end{tabularx}
	\end{center}
}{%
	Cas 3~:
	\begin{center}
		\def\rhgt{0.50}
		\centering
		\begin{tabularx}{\linewidth}{|l|c||YdYdY||Y|}
			\hline
			\multicolumn{2}{|c||}{
				$\xmathstrut{\rhgt}$
			\textbf{Équation}}   &
			$2\ce{BaO_2_{\sol}}$ & $\rightleftharpoons$ &
			$2\ce{BaO_{\sol}}$   & $+$                  &
			$\ce{O_2_{\gaz}}$    &
			$n\ind{tot, gaz}$                             \\
			\hline
			$\xmathstrut{\rhgt}$
			Initial (\si{mol})   & $\xi = 0$            &
			$\num{0.10}$         & \vline               &
			$\num{0.050}$        & \vline               &
			$\num{0.10}$         &
			$\num{0.10}$                                  \\
			\hline
			$\xmathstrut{\rhgt}$
			Interm. (\si{mol})   & $\xi$                &
			$\num{0.10} - 2\xi$  & \vline               &
			$\num{0.050} + 2\xi$ & \vline               &
			$\num{0.10} + \xi$   &
			$\num{0.10} + \xi$                            \\
			\hline
			$\xmathstrut{\rhgt}$
			Final  (\si{mol})    & $\xi = \xi_f$        &
			$\num{0.15}$         & \vline               &
			$\num{0.00}$         & \vline               &
			$\num{0.075}$        &
			$\num{0.075}$                                 \\
			\hline
		\end{tabularx}
	\end{center}
	\begin{enumerate}[leftmargin=20pt, label=\alph* --]
		\item On a cette fois du gaz au départ, donc ici
		      \begin{gather*}
			      Q_{r,0}
			      = \frac{p_{\ce{O2}, 0}}{p^\circ}
			      \Lra
			      \boxed{
				      Q_{r,0}
				      = \frac{n_{\ce{O2}, 0}RT}{Vp^\circ}
			      }
			      \qavec
			      \left\{
			      \begin{array}{rcl}
				      n_{\ce{O2}, 0} & = & \SI{0.10}{mol}  \\
				      T              & = & \SI{1069.15}{K} \\
				      V              & = & \SI{10e-3}{m^3}
			      \end{array}
			      \right.\\
			      \AN
			      \xul{Q_{r,0} = \num{0.89}}
		      \end{gather*}
		      Cette fois, $Q_{r,0} > K$ donc la réaction se fait en sens
		      indirect.
		\item Voir tableau.
		      \begin{tcb}(impo){Tableau sens indirect}
			      Le procédé de remplissage du tableau \textbf{ne doit pas changer}
			      même si la réaction se fait dans le sens indirect~: les coefficients
			      stœchiométriques de la réaction n'ont pas changé, donc les facteurs
			      devant des $\xi(t)$ non plus.
			      \smallbreak
			      Certes, on aura $\xi < 0$ mais il est plus naturel et moins
			      perturbant de garder la forme de base du remplissage du tableau
			      plutôt que de s'embêter à repenser l'écriture du tableau.
		      \end{tcb}
		\item Même procédé~: \textbf{on détermine $\xi_{\eql}$ s'il y
			      avait équilibre, et on regarde si c'est compatible avec
			      $\xi_{\max}$ si la réaction était totale}.
		      \bigbreak
		      S'il y a équilibre, ça veut dire que $n_{\ce{O2}, \eql} =
			      \SI{0.056}{mol}$ comme déterminé au début. Or, le tableau nous
		      indique que $n_{\ce{O2}, f} = \num{0.10} + \xi_f$, donc si c'est
		      un équilibre \xul{$\xi_{\eql} = -\SI{0.044}{mol}$}.
		      \bigbreak
		      L'avancement est maximal si \ce{BaO} ou \ce{O2} sont limitant~:
		      on résout donc
		      \begin{gather*}
			      \left\{
			      \begin{array}{rcl}
				      n_{\ce{BaO},0}+2\xi_{\max} & = & 0 \\
				      n_{\ce{O_2},0}+\xi_{\max}  & = & 0
			      \end{array}
			      \right.
			      \Lra
			      \left\{
			      \begin{array}{rcl}
				      \xi_{\max} & = & -\SI{0.025}{mol} \\
				      \xi_{\max} & = & -\SI{0.050}{mol}
			      \end{array}
			      \right.
		      \end{gather*}
		      Le seul $\xi_{\max}$ possible est le plus petit \textbf{en
			      valeur absolue}, c'est-à-dire \xul{$\xi_{\max} = -\SI{0.025}{mol}$}.
		      \bigbreak
		      La valeur est finale $\xi_f$ est la plus petite valeur \textbf{en
			      valeur absolue} de $\xi_{\eql}$ et $\xi_{\max}$~; or ici
		      $\abs{\xi_{\eql}} > \abs{\xi_{\max}}$~: il n'y a donc \textbf{pas
			      équilibre}, et on a
		      \[\xul{\xi_f = \xi_{\max} = -\SI{0.025}{mol}}\]
		\item Voir tableau.
	\end{enumerate}
}

\end{document}
