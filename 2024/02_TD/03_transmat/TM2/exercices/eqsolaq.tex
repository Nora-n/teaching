\documentclass[../TDTM2.tex]{subfiles}%

\begin{document}
\section[s]"1"{Équilibre en solution aqueuse}
\enonce{%
	Considérons un système de volume \SI{20}{mL} évoluant selon la réaction
	d'équation bilan~:

	\centersright{%
		$\ce{CH3COOH_{\aqu} + F^{-}_{\aqu} <=> CH3COO^{-}_{\aqu} + HF_{\aqu}}$
	}{%
		$K^\circ(\SI{25}{\degreeCelsius}) = 10^{\num{-1.60}}$
	}

	Déterminer le sens d'évolution du système et l'avancement à l'équilibre en
	partant des deux situations initiales suivantes~:
}

\QR{%
$[\ce{CH3COOH}]_0 = [\ce{F^{-}}]_0 = c = \SI{0.1}{mol.L^{-1}}$ et
$[\ce{CH3COO^{-}}]_0 = [\ce{HF}]_0 = 0$
}{%
Pour déterminer le sens d'évolution du système, on calcule $Q_{r,0}$
et on le compare à $K^\circ$~:
\[\boxed{
	Q_{r,0} = \frac{[\ce{CH3COO^{-}}]_0[\ce{HF}]_0}
	{[\ce{CH3COOH}]_0[\ce{F^{-}}]_0} = 0 < K^\circ}
\]
La réaction évoluera donc \textbf{dans le sens direct}.
\bigbreak
Pour trouver l'avancement à l'équilibre, on dresse le tableau
d'avancement, que l'on peut directement faire en concentrations puisque
le volume ne varie pas (ce qui est toujours le cas cette année)~:
\begin{center}
	\def\rhgt{0.35}
	\centering
	\begin{tabularx}{\linewidth}{|l|c||YdYdYdY|}
		\hline
		\multicolumn{2}{|c||}{
			$\xmathstrut{\rhgt}$
		\textbf{Équation}}        &
		$\ce{CH_3COOH_{\aqu}}$    & $+$              &
		$\ce{F^{-}_{\aqu}}$       & $=$              &
		$\ce{CH_3COO^{-}_{\aqu}}$ & $+$              &
		$\ce{HF_{\aqu}}$                               \\
		\hline
		$\xmathstrut{\rhgt}$
		Initial                   & $x = 0$          &
		$c$                       & \vline           &
		$c$                       & \vline           &
		$0$                       & \vline           &
		$0$                                            \\
		\hline
		$\xmathstrut{\rhgt}$
		Final                     & $x_f = x_{\eql}$ &
		$c - x_{\eql}$            & \vline           &
		$c - x_{\eql}$            & \vline           &
		$x_{\eql}$                & \vline           &
		$x_{\eql}$                                     \\
		\hline
	\end{tabularx}
\end{center}
D'après la loi d'action des masses, on a
\begin{gather*}
	K^\circ = \frac{x_{\eql}{}^2}{(c-x_{\eql})^2}
	\Lra
	\sqrt{K^\circ} = \frac{x_{\eql}}{c-x_{\eql}}
	\Lra
	x_{\eql} = \sqrt{K^\circ}(c-x_{\eql})\\
	\Lra
	\boxed{x_{\eql} = \frac{\sqrt{K^\circ}}{1+ \sqrt{K^\circ}}c}
	\qavec
	\left\{
	\begin{array}{rcl}
		K^\circ & = & 10^{\num{-1.60}}     \\
		c       & = & \SI{0.1}{mol.L^{-1}}
	\end{array}
	\right.\\
	\mathrm{A.N.~:}\quad
	\xul{x_{\eql} = \SI{1.4e-2}{mol.L^{-1}}}
\end{gather*}
En encadrant le résultat, on vérifie la cohérence physico-chimique de la
réponse~: ici c'est bien cohérent de trouver $x_{\eql} > 0$ puisqu'on
avait déterminé que la réaction se faisait dans le sens direct.
}
\QR{%
$[\ce{CH3COOH}]_0 = [\ce{F^{-}}]_0 = [\ce{CH3COO^{-}}]_0 =
	[\ce{HF}]_0 = c = \SI{0.1}{mol.L^{-1}}$
}{%
De la même manière, pour déterminer le sens d'évolution du système, on
calcule $Q_{r,0}$ et on le compare à $K$~:
\[\boxed{
	Q_{r,0} = \frac{[\ce{CH3COO^{-}}]_0[\ce{HF}]_0}
	{[\ce{CH3COOH}]_0[\ce{F^{-}}]_0}
	= \frac{c^2}{c^2} = 1 > K^\circ}
\]
La réaction évoluera donc \textbf{dans le sens indirect}.
\bigbreak
On effectue un bilan de matière grâce à un tableau d'avancement~:
\begin{center}
	\def\rhgt{0.35}
	\centering
	\begin{tabularx}{\linewidth}{|l|c||YdYdYdY|}
		\hline
		\multicolumn{2}{|c||}{
			$\xmathstrut{\rhgt}$
		\textbf{Équation}}        &
		$\ce{CH_3COOH_{\aqu}}$    & $+$              &
		$\ce{F^{-}_{\aqu}}$       & $=$              &
		$\ce{CH_3COO^{-}_{\aqu}}$ & $+$              &
		$\ce{HF_{\aqu}}$                               \\
		\hline
		$\xmathstrut{\rhgt}$
		Initial                   & $x = 0$          &
		$c$                       & \vline           &
		$c$                       & \vline           &
		$c$                       & \vline           &
		$c$                                            \\
		\hline
		$\xmathstrut{\rhgt}$
		Final                     & $x_f = x_{\eql}$ &
		$c - x_{\eql}$            & \vline           &
		$c - x_{\eql}$            & \vline           &
		$c + x_{\eql}$            & \vline           &
		$c + x_{\eql}$                                 \\
		\hline
	\end{tabularx}
\end{center}
D'après la loi d'action des masses, on a
\begin{gather*}
	K^\circ = \frac{(c+x_{\eql})^2}{(c-x_{\eql})^2}
	\Lra
	\sqrt{K^\circ} = \frac{c+x_{\eql}}{c-x_{\eql}}
	\Lra
	c+x_{\eql} = \sqrt{K^\circ}(c-x_{\eql})\\
	\Lra
	\boxed{x_{\eql} = \frac{\sqrt{K^\circ}-1}{\sqrt{K^\circ}+1}c}
	\qavec
	\left\{
	\begin{array}{rcl}
		K^\circ & = & 10^{\num{-1.60}}     \\
		c       & = & \SI{0.1}{mol.L^{-1}}
	\end{array}
	\right.\\
	\mathrm{A.N.~:}\quad
	\xul{x_{\eql} = -\SI{5.3e-2}{mol.L^{-1}}}
\end{gather*}
De même que précédemment, on vérifie qu'il est logique de trouver
$x_{\eql} < 0$~: la réaction se fait bien dans le sens indirect.
}


\end{document}
