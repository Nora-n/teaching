\documentclass[../TDTM2.tex]{subfiles}%

\begin{document}
\section[s]"2"{Équilibre avec des solides}
\enonce{%
	La chaux vive, solide blanc de formule $\ce{CaO_{\sol}}$, est un des produits
	de chimie industrielle les plus communs. Utilisée depuis l'Antiquité,
	notamment dans le domaine de la construction, elle est aujourd'hui utilisée
	comme intermédiaire en métallurgie. Elle est obtenue industriellement par
	dissociation thermique du calcaire dans un four à $T = \SI{1100}{K}$. On
	modélise cette transformation par la réaction d'équation~:

	\centersright{%
		$\ce{CaCO3_{\sol} = CaO_{\sol} + CO2 _{\gaz}}$
	}{%
		$K^\circ(\SI{1100}{K}) = \num{0.358}$
	}
}

\QR{%
	Dans un récipient de volume $V = \SI{10}{L}$ préalablement vide, on
	introduit \SI{10}{mmol} de calcaire à température constante $T =
		\SI{1100}{K}$. Déterminer le sens d'évolution du système chimique.
}{%
	Comme on ne part que de calcaire, la réaction \textbf{ne peut avoir
		lieu que dans le sens direct}. On vérifie cette intuition en calculant
	$Q_{r,0}$ pour le comparer à $K$, sachant qu'on part d'un récipient vide
	de gaz au début~:
	\[\boxed{Q_{r,0} = \frac{p_{\ce{CO2}, 0}}{p^\circ} = 0 < K^\circ}\]
	La réaction se fait bien dans le sens direct.
}

\QR{%
	Supposons que l'état final est un état d'équilibre. Déterminer la
	quantité de matière de calcaire qui devrait avoir réagi. Conclure sur
	l'hypothèse faite.
}{%
	Si l'état final est un état d'équilibre, alors avec l'équation
	précédente on aura
	\begin{gather*}
		p_{\ce{CO2}, \eql}
		= K^\circ p^\circ\\
		\Lra
		n_{\ce{O2}, \eql}
		= \frac{p_{\ce{O2},\eql}V}{RT}
		= \frac{K^\circ p^\circ V}{RT}
	\end{gather*}
	Or, un tableau d'avancement donne que $n_{\ce{O2}, \eql} = \xi_{\eql}$~; on
	trouve donc
	\begin{gather*}
		\boxed{\xi_{\eql} = \frac{K^\circ p^\circ V}{RT}}
		\qavec
		\left\{
		\begin{array}{rcl}
			K^\circ & = & \num{0.358}              \\
			V       & = & \SI{10e-3}{m^3}          \\
			T       & = & \SI{1100}{K}             \\
			p^\circ & = & \SI{1}{bar}              \\
			R       & = & \SI{8.314}{J.mol.K^{-1}}
		\end{array}
		\right.\\
		\mathrm{A.N.~:}\quad
		\xul{\xi_{\eql} = \SI{39}{mmol}}
	\end{gather*}
	Pour savoir si cette valeur est réalisable, on calcule $\xi_{\max}$ que
	l'on trouverait si le calcaire était limitant, c'est-à-dire en résolvant
	$\num{10} - \xi_{\max} = 0$~: on trouve naturellement \xul{$\xi_{\max}
			= \SI{10}{mmol}$}.
	\bigbreak
	On sait que la valeur de $\xi_f$ est la plus petite valeur absolue entre
	$\xi_{\eql}$ et $\xi_{\max}$. Or, ici on trouve $\xi_{\eql} >
		\xi_{\max}$, ce qui veut dire que \textbf{l'équilibre ne peut être
		atteint} et qu'on aura ainsi
	\[\xul{\xi_f = \xi_{\max} = \SI{10}{mmol}}\]
	c'est-à-dire que \textbf{la réaction est totale}. On peut donc remplir
	la dernière ligne du tableau d'avancement.
	\begin{center}
		\def\rhgt{0.35}
		\centering
		\begin{tabularx}{\linewidth}{|l|c||YdYdY||Y|}
			\hline
			\multicolumn{2}{|c||}{
				$\xmathstrut{\rhgt}$
			\textbf{Équation}}   &
			$\ce{CaCO_3_{\sol}}$ & $=$                  &
			$\ce{CaO_{\sol}}$    & $+$                  &
			$\ce{CO_2_{\gaz}}$   &
			$n\ind{tot, gaz}$                             \\
			\hline
			$\xmathstrut{\rhgt}$
			Initial (\si{mmol})  & $\xi = 0$            &
			$10$                 & \vline               &
			$0$                  & \vline               &
			$0$                  &
			$0$                                           \\
			\hline
			$\xmathstrut{\rhgt}$
			Interm. (\si{mmol})  & $\xi$                &
			$10 - \xi$           & \vline               &
			$\xi$                & \vline               &
			$\xi$                &
			$\xi$                                         \\
			\hline
			$\xmathstrut{\rhgt}$
			Final  (\si{mmol})   & $\xi_f = \xi_{\max}$ &
			$0$                  & \vline               &
			$10$                 & \vline               &
			$10$                 &
			$10$                                          \\
			\hline
		\end{tabularx}
	\end{center}
}
\QR{%
Si on part de \SI{50}{mmol} de calcaire, quelle est la quantité de chaux
obtenue~? Comment faire pour augmenter la quantité de chaux produite~?
}{%
En ne partant que de calcaire, dès que $\xi_f$ atteint $\xi_{\eql}$ la réaction
s'arrêtera puisqu'on aura atteint l'équilibre. Mettre plus de calcaire ne
formera pas plus de chaux, l'excédent de réactif initial ne réagira simplement
pas. Ainsi, \textbf{la quantité de matière de calcaire maximale qui puisse
	être transformée est de $\xi_{\max} = \SI{39}{mmol}$}.
\bigbreak
Pour déplacer l'équilibre dans le sens direct, il faut diminuer la quantité de
$\ce{CO_2_{\gaz}}$~: on diminue alors $Q_r$ qui peut repasser en-dessous de
$K^\circ$. Il suffit pour ça de \textbf{travailler en volume ouvert} ou
\textbf{d'aspirer le $\ce{CO_2_{\gaz}}$}.
}

\end{document}
