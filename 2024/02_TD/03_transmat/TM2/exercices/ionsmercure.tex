\documentclass[../TDTM2.tex]{subfiles}%

\begin{document}
\section[s]"1"{Ions mercure}
\enonce{%
	Les ions mercure (II) $\ce{Hg^{2+}}$ peuvent réagir avec le métal liquide
	(insoluble dans l'eau) mercure \ce{Hg} pour donner les ions mercure (I)
	$\ce{Hg2^{2+}}$ selon l'équilibre chimique ci-dessous~:

	\centersright{%
		$\ce{Hg^{2+}_{\aqu} + Hg_{\liq} = Hg2^{+2}_{\aqu}}$
	}{%
		$K^\circ(\SI{25}{\degreeCelsius}) = 91$}
}

\QR{%
	Dans quel sens évolue un système obtenu en mélangeant du mercure
	liquide en large excès avec $V_1 = \SI{40.0}{mL}$ d'une solution de
	chlorure de mercure (I) à $c_1 = \SI{1.0e-3}{mol.L^{-1}}$ et $V_2 =
		\SI{10.0}{mL}$ d'une solution de chlorure de mercure (II) à $c_2 =
		\SI{2.0e-3}{mol.L^{-1}}$~?
}{%
	On détermine les concentrations en mercure (I) et (II)~:
	\begin{gather*}
		[\ce{Hg^{2+}_{\aqu}}] = \frac{c_2V_2}{V_1 + V_2} = c_2' =
		\SI{0.4}{mmol.L^{-1}}
		\qqet
		[\ce{Hg_2^{2+}_{\aqu}}] = \frac{c_1V_1}{V_1 + V_2} = c_1' =
		\SI{0.8}{mmol.L^{-1}}
	\end{gather*}
	On peut donc calculer le quotient de réaction initial, avec
	$a(\ce{Hg_{\liq}}) )= 1$~:
	\[\boxed{Q_{r,0} = \frac{c_1'}{c_2'} = 2 < K}
		\quad\Longrightarrow\quad
		\text{évolution sens direct}\]
}
\QR{%
	Déterminer la composition finale de la solution.
}{%
	On dresse le tableau d'avancement en concentration~:
	\begin{center}
		\def\rhgt{0.35}
		\centering
		\begin{tabularx}{\linewidth}{|l|c||YdYdY|}
			\hline
			\multicolumn{2}{|c||}{
				$\xmathstrut{\rhgt}$
			\textbf{Équation}}    &
			$\ce{Hg^{2+}_{\aqu}}$ & $+$              &
			$\ce{Hg_{\liq}}$      & $=$              &
			$\ce{Hg_2^{2+}_{\aqu}}$                    \\
			\hline
			$\xmathstrut{\rhgt}$
			Initial               & $x = 0$          &
			$c_2'$                & \vline           &
			excès                 & \vline           &
			$c_1'$                                     \\
			\hline
			$\xmathstrut{\rhgt}$
			Interm.               & $x$              &
			$c_2' - \xi$          & \vline           &
			excès                 & \vline           &
			$c_1' + \xi$                               \\
			\hline
			$\xmathstrut{\rhgt}$
			Final                 & $x_f = x_{\eql}$ &
			$c_2' - \xi_f$        & \vline           &
			excès                 & \vline           &
			$c_1' + \xi_f$                             \\
			\hline
		\end{tabularx}
	\end{center}
	Par la loi d'action des masses, on trouve en effet
	\begin{gather*}
		K^\circ = \frac{c_1'+x_{\eql}}{c_2'-x_{\eql}}
		\Lra
		\boxed{x_{\eql} = \frac{K^\circ c_2' - c_1'}{K^\circ + 1} =
			\SI{0.387}{mmol.L^{-1}}}
	\end{gather*}
	ce qui est bien inférieur à $x_{\max} = c_2' = \SI{0.4}{mmol.L^{-1}}$~:
	l'équilibre est atteint.
}
\end{document}
