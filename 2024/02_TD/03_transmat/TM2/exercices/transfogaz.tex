\documentclass[../TDTM2.tex]{subfiles}%

\begin{document}
\section[s]"2"{Transformations de gaz}
\QR{%
On considère l'équilibre suivant~:

\centersright{%
$\DS\ce{H2S_{\gaz} + \frac{3}{2}O2_{\gaz} = H2O_{\gaz} + SO2_{\gaz}}$
}{%
$K_1^\circ$
}

Donner l'expression de la constante d'équilibre $K_1^\circ$. En
supposant les réactifs introduits dans les proportions stœchiométriques,
faire un bilan de matière à l'équilibre. Exprimer $K_1^\circ$ en
fonction de $\xi_{\eql}$.
}{%
Par la loi d'action des masses, on a
\begin{gather*}
	K_1^\circ =
	\frac{p_{\ce{SO2}}p_{\ce{H2O}}}{p_{\ce{O2}}^{3/2}p_{\ce{H2S}}}
	\underbracket[1pt]{\frac{p^\circ{}^{5/2}}{p^\circ{}^2}}_{= p^\circ{}^{1/2}}
	\Leftrightarrow
	K_1^\circ =
	\frac{n_{\ce{H2O}}n_{\ce{SO2}}n\ind{tot,
		gaz}^{1/2}}{n_{\ce{H2S}}n_{\ce{O2}}{}^{3/2}}
	\left(\frac{p^\circ}{p}\right)^{1/2}
\end{gather*}
Soit $n_0$ la quantité de matière de $\ce{H2S}$ introduite
initialement~: pour que $\ce{O2}$ soit introduit dans les proportions
stœchiométriques, on relie sa quantité initiale à celle de $\ce{H2S}$
\textit{via} les coefficients stœchiométriques tel que $n_{\ce{O2},0} =
	\frac{3}{2}n_{\ce{H2S},0} = \frac{3}{2}n_0$. D'où le tableau
d'avancement~:
\begin{center}
	\def\rhgt{0.35}
	\centering
	\begin{tabularx}{\linewidth}{|l|c||YdYdYdY||Y|}
		\hline
		\multicolumn{2}{|c||}{
			$\xmathstrut{\rhgt}$
		\textbf{Équation}}                       &
		$\ce{H_2S_{\gaz}}$                       & $+$                  &
		$\frac{3}{2}\ce{O_2_{\gaz}}$             & $=$                  &
		$\ce{H_2O_{\gaz}}$                       & $+$                  &
		$\ce{SO_2_{\gaz}}$                       &
		$n\ind{tot, gaz}$                                                 \\
		\hline
		$\xmathstrut{\rhgt}$
		Initial                                  & $\xi = 0$            &
		$n_0$                                    & \vline               &
		$\frac{3}{2}n_0$                         & \vline               &
		$0$                                      & \vline               &
		$0$                                      &
		$\frac{5}{2}n_0$                                                  \\
		\hline
		$\xmathstrut{\rhgt}$
		Final                                    & $\xi_f = \xi_{\eql}$ &
		$n_0 - \xi_{\eql}$                       & \vline               &
		$\frac{3}{2}n_0 - \frac{3}{2}\xi_{\eql}$ & \vline               &
		$\xi_{\eql}$                             & \vline               &
		$\xi_{\eql}$                             &
		$\frac{5}{2}n_0 - \frac{1}{2}\xi_{\eql}$                          \\
		\hline
	\end{tabularx}
\end{center}
On peut donc remplacer les quantités de matière de l'expression de
$K_1^\circ$ par les expressions avec $\xi_{\eql}$~:
\[\boxed{
		K_1^\circ =
		\frac{\xi_{\eql}^2( \frac{5}{2}n_0- \frac{1}{2}\xi_{\eql})^{1/2}}
		{(n_0-\xi_{\eql})( \frac{3}{2}n_0 - \frac{3}{2}\xi_{\eql})^{3/2}}
		\left(\frac{p^\circ}{p}\right)^{1/2}}
\]
}

\QR{%
	On considère l'équilibre suivant~:

	\centersright{%
		$\DS\ce{2H2S_{\gaz} + SO2_{\gaz} = 2H2O_{\gaz} + 3S_{\liq}}$
	}{%
		$K_2^\circ$}

	Donner l'expression de la constante d'équilibre $K_2^\circ$. On
	introduit les réactifs avec des quantités quelconques. Faire un bilan de
	matière à l'équilibre. Exprimer $K_2^\circ$ en fonction de $\xi_{\eql}$.
}{%
	Par la loi d'action des masses, on a
	\begin{gather*}
		K_2^\circ =
		\frac{p_{\ce{H2O}}{}^2p^\circ}
		{p_{\ce{SO2}}p_{\ce{H2S}}{}^2}
		\Lra
		K_2^\circ =
		\frac{n_{\ce{H2O}}{}^2n\ind{tot, gaz}}
		{n_{\ce{SO2}}n_{\ce{H2S}}{}^2}
		\frac{p^\circ}{p}
	\end{gather*}
	Soit $n_1$ la quantité de matière de $\ce{H2S}$ introduite initialement,
	et $n_2$ la quantité de matière initiale en $\ce{SO2}$~:
	\begin{center}
		\def\rhgt{0.35}
		\centering
		\begin{tabularx}{\linewidth}{|l|c||YdYdYdY||Y|}
			\hline
			\multicolumn{2}{|c||}{
				$\xmathstrut{\rhgt}$
			\textbf{Équation}}  &
			$2\ce{H_2S_{\gaz}}$ & $+$                  &
			$\ce{SO_2_{\gaz}}$  & $=$                  &
			$2\ce{H_2O_{\gaz}}$ & $+$                  &
			$3\ce{S_{\liq}}$    &
			$n\ind{tot, gaz}$                            \\
			\hline
			$\xmathstrut{\rhgt}$
			Initial             & $\xi = 0$            &
			$n_1$               & \vline               &
			$n_2$               & \vline               &
			$0$                 & \vline               &
			$0$                 &
			$n_1+n_2$                                    \\
			\hline
			$\xmathstrut{\rhgt}$
			Final               & $\xi_f = \xi_{\eql}$ &
			$n_1 - 2\xi_{\eql}$ & \vline               &
			$n_2 - \xi_{\eql}$  & \vline               &
			$0 + 2\xi_{\eql}$   & \vline               &
			$0 + 3\xi_{\eql}$   &
			$n_1+n_2-\xi_{\eql}$                         \\
			\hline
		\end{tabularx}
	\end{center}
	On peut donc remplacer les quantités de matière de l'expression de
	$K_2^\circ$ par les expressions avec $\xi_{\eql}$~:
	\[\boxed{
			K_2^\circ =
			\frac{4\xi_{\eql}^2(n_1+n_2-\xi_{\eql})}
			{(n_1-2\xi_{\eql})^2(n_2-\xi_{\eql})}
			\frac{p^\circ}{p}}
	\]
}

\QR{%
	On fait brûler du méthane dans de l'oxygène~:
	\[
		\ce{\ldots CH4_{\gaz} + \ldots O2_{\gaz}
			-> \ldots CO2_{\gaz} + \ldots H2O_{\liq}}
	\]
	Équilibrer l'équation de la réaction. Elle peut être considérée
	comme totale. On introduit les réactifs de façon à consommer la moitié
	du dioxygène. Décrire l'état final du système.
}{%
	L'équation bilan équilibrée est~:
	\[\ce{CH4_{\gaz} + 2O2_{\gaz} \rightarrow CO2_{\gaz} + 2H2O_{\liq}}\]
	Soit $n_0$ la quantité initiale en dioxygène. Si la moitié seulement est
	consommée, alors que la réaction est totale, c'est que le méthane est
	limitant. On trouve la quantité de $\ce{CH4}$ à introduire initialement
	en dressant le tableau d'avancement pour que $n_{\ce{H2O},f} =
		\frac{1}{2}n_0$, c'est-à-dire $n_0 - 2\xi_{\max} = 0$~: on obtient
	\[\boxed{\xi_{\max} = \frac{1}{4}n_0}
		\qor
		n_{\ce{CH4},0} - \xi_{\max} = 0
		\qdonc
		\boxed{n_{\ce{CH4},0} = \frac{1}{4}n_0}
	\]
	\begin{center}
		\def\rhgt{0.35}
		\centering
		\begin{tabularx}{\linewidth}{|l|c||YdYdYdY||Y|}
			\hline
			\multicolumn{2}{|c||}{
				$\xmathstrut{\rhgt}$
			\textbf{Équation}}     &
			$\ce{CH_4_{\gaz}}$     & $+$                  &
			$2\ce{O_2_{\gaz}}$     & $\ra$                &
			$\ce{CO_2_{\gaz}}$     & $+$                  &
			$2\ce{H_2O_{\liq}}$    &
			$n\ind{tot, gaz}$                               \\
			\hline
			$\xmathstrut{\rhgt}$
			Initial                & $\xi = 0$            &
			$\frac{1}{4}n_0$       & \vline               &
			$n_0$                  & \vline               &
			$0$                    & \vline               &
			$0$                    &
			$\frac{5}{4}n_0$                                \\
			\hline
			$\xmathstrut{\rhgt}$
			Interm.                & $\xi$                &
			$\frac{1}{4}n_0 - \xi$ & \vline               &
			$n_0 - 2\xi$           & \vline               &
			$\xi$                  & \vline               &
			$2\xi$                 &
			$\frac{5}{4}n_0 - 2\xi$                         \\
			\hline
			$\xmathstrut{\rhgt}$
			Final                  & $\xi_f = \xi_{\max}$ &
			$0$                    & \vline               &
			$\frac{1}{2}n_0$       & \vline               &
			$\frac{1}{4}n_0$       & \vline               &
			$\frac{1}{2}n_0$       &
			$\frac{3}{4}n_0$                                \\
			\hline
		\end{tabularx}
	\end{center}
}
\end{document}
