\documentclass[../TDTM2.tex]{subfiles}%

\begin{document}
\section[s]"1"{Transformations totales}
\enonce{%
	Compléter les tableaux suivants. Les gaz seront supposés parfaits. Dans la
	ligne intermédiaire, on demande d'exprimer la quantité de matière en fonction
	de l'avancement molaire $\xi(t)$ à un instant $t$ quelconque.
}
\QR{%
Réaction de l'oxydation du monoxyde d'azote en phase gazeuse, à $T =
	\SI{25}{\degreeCelsius}$ dans un volume $V = \SI{10.0}{L}$~:
\begin{center}
	\def\rhgt{0.50}
	\centering
	\begin{tabularx}{\linewidth}{|l|c||YdYdY||Y|Y|}
		\hline
		\multicolumn{2}{|c||}{
			$\xmathstrut{\rhgt}$
		\textbf{Équation}}       &
		$\ldots\ce{NO_{\gaz}}$   & $+$                        &
		$\ldots\ce{O_2_{\gaz}}$  & $\ra$                      &
		$\ldots\ce{NO_2_{\gaz}}$ &
		$n\ind{tot, gaz} $       &
		$P_{\tot} (\si{bar})$                                                      \\
		\hline
		$\xmathstrut{\rhgt}$
		Initial (\si{mol})       & $\xi = 0$                  &
		$\num{1.00}$             & \vline                     &
		$\num{2.00}$             & \vline                     &
		$\num{0.00}$             &
		                         &                                                 \\
		\hline
		$\xmathstrut{\rhgt}$
		Interm. (\si{mol})       & $\xi$                      &
		                         & \vline                     &   & \vline &  &  & \\
		\hline
		$\xmathstrut{\rhgt}$
		Final (\si{mol})         & $\xi_f = \wht{\num{0.50}}$ &
		                         & \vline                     &   & \vline &  &  & \\
		\hline
	\end{tabularx}
\end{center}
}{%
Pour la quantité totale de gaz, il suffit de sommer les quantité de matière de
chacun des gaz~: ici, initialement on a $n_0(\ce{NO_{\gaz}}) +
	n_0(\ce{O_2_{\gaz}}) = \SI{3.00}{mol}$ de gaz. Ensuite, pour la pression
totale
on utilise l'équation d'état des gaz parfaits~:
\begin{tcb}(rapp){gaz parfait}
	\vspace{-15pt}
	\begin{gather*}
		\boxed{pV = nRT}
		\qavec
		\left\{
		\begin{array}{ll}
			p & \text{en Pa }                 \\
			V & \text{en m}^3                 \\
			n & \text{en mol}                 \\
			T & \text{en \textbf{Kelvin} (K)}
		\end{array}
		\right.\\
		\text{et }
		\boxed{R = \SI{8.314}{J.mol^{-1}.K^{-1}}}\\
		\text{est la constante des gaz parfaits}
	\end{gather*}
\end{tcb}
Il faut donc convertir le volume en $\si{m^3}$. Pour cela, il suffit
d'écrire
\[\SI{10.0}{L} = \SI{10.0}{dm^3} = \num{10.0}(\SI{e-1}{m})^3 =
	\SI{10.0e-3}{m^3}\]
Il est très courant d'oublier les puissances sur les conversions du genre~:
n'oubliez pas les parenthèses. Il nous faut de plus convertir la température en
Kelvins, attention à ne pas vous tromper de sens~: il
faut ici \textbf{ajouter} \SI{273.15}{K} à la température en degrés
Celsius, ce qui donne $T = \SI{298.15}{K}$. On peut donc faire
l'application numérique pour $P_{\tot}$ initial.
\bigbreak
On remplit la deuxième ligne du tableau avec les coefficients stœchiométriques
algébriques des constituants en facteur de chaque $\xi$, et on somme les
quantités de matière de gaz pour $n\ind{tot,
		gaz}$. En réalité, il est plus simple de partir de la valeur totale de
la première ligne et de compter algébriquement le nombre de $\xi$~: on en perd 3
avec les réactifs pour en gagner 2 avec les produits, donc en tout la quantité
de matière totale de gaz décroit de 1$\xi$. On ne peut
pas calculer précisément la valeur de $P_{\tot}$ ici, il faudrait
l'exprimer en fonction de $\xi$ (ça viendra dans d'autres exercices).
\bigbreak
Enfin, pour trouver le réactif limitant, on résout~:
\begin{gather*}
	\left\{
	\begin{array}{rcl}
		n_0(\ce{NO_{\gaz}})-2\xi_f & = & 0 \\
		n_0(\ce{O_2_{\gaz}})-\xi_f & = & 0
	\end{array}
	\right.
	\Lra
	\left\{
	\begin{array}{rcl}
		\xi_f & = & \SI{0.50}{mol}              \\
		\xi_f & = & \SI{1.00}{mol} > \xi_{\max}
	\end{array}
	\right.
\end{gather*}
La seule valeur possible est la plus petite, $\xi_f = \SI{0.50}{mol}$~:
si on prenait \SI{1.00}{mol} on trouverait une quantité négative de
\ce{NO} à l'état final, ce qui, vous en conviendrez, est une absurdité.
Même travail qu'initialement pour $n\ind{tot, gaz}$ et $P_{\tot}$.
D'où le tableau~:
\begin{center}
	\def\rhgt{0.50}
	\centering
	\begin{tabularx}{\linewidth}{|l|c||YdYdY||Y|Y|}
		\hline
		\multicolumn{2}{|c||}{
			$\xmathstrut{\rhgt}$
		\textbf{Équation}}  &
		$2\ce{NO_{\gaz}}$   & $+$                  &
		$\ce{O_2_{\gaz}}$   & $\ra$                &
		$2\ce{NO_2_{\gaz}}$ &
		$n\ind{tot, gaz} $  &
		$P_{\tot} (\si{bar})$                        \\
		\hline
		$\xmathstrut{\rhgt}$
		Initial (\si{mol})  & $\xi = 0$            &
		$\num{1.00}$        & \vline               &
		$\num{2.00}$        & \vline               &
		$\num{0.00}$        &
		$\num{3.00}$        &
		$\num{7.40}$                                 \\
		\hline
		$\xmathstrut{\rhgt}$
		Interm. (\si{mol})  & $\xi$                &
		$\num{1.00} - 2\xi$ & \vline               &
		$\num{2.00} - \xi$  & \vline               &
		$2\xi$              &
		$\num{3.00} - \xi$  &
		--                                           \\
		\hline
		$\xmathstrut{\rhgt}$
		Final (\si{mol})    & $\xi_f = \num{0.50}$ &
		\num{0.00}          & \vline               &
		\num{1.50}          & \vline               &
		\num{1.00}          &
		\num{2.50}          &
		\num{6.20}                                   \\
		\hline
	\end{tabularx}
\end{center}
}
\QR{%
Réaction de combustion de l'éthanol dans l'air. Les réactifs sont introduits
dans les proportions stœchiométriques. Le dioxygène provient
de l'air, qui contient 20\% de $\ce{O2}$ et 80\% de $\ce{N2}$ en
fraction molaire.
\begin{center}
	\def\rhgt{0.50}
	\centering
	\begin{tabularx}{\linewidth}{|l|c||YdYdYdY||Y|Y|}
		\hline
		\multicolumn{2}{|c||}{
			$\xmathstrut{\rhgt}$
		\textbf{Équation} (\si{mol})} &
		$\ce{C_2H_5OH_{\liq}}$        & $+$                        &
		$3\ce{O_2_{\gaz}}$            & $\ra$                      &
		$2\ce{CO_2_{\gaz}}$           & $+$                        &
		$3\ce{H_2O_{\gaz}}$           &
		$n_{\ce{N_2}}$                &
		$n\ind{tot, gaz}$                                                                                         \\
		\hline
		$\xmathstrut{\rhgt}$
		Initial                       & $\xi = 0$                  &
		$\num{2.00}$                  & \vline                     &
		                              & \vline                     &   & \vline &  &        &                     \\
		\hline
		$\xmathstrut{\rhgt}$
		Interm.                       & $\xi$                      &   & \vline &  & \vline &   & \vline &   &  &
		\\
		\hline
		$\xmathstrut{\rhgt}$
		Final                         & $\xi_f = \wht{\num{2.00}}$ &
		                              & \vline                     &   & \vline &  & \vline &   &        &        \\
		\hline
	\end{tabularx}
\end{center}
}{%
Pour une réaction $a{\ce{A}} + b{\ce{B}} = c{\ce{C}} + d{\ce{D}}$, le fait
que les réactifs soient introduits dans les proportions stœchiométriques se
traduit par
\[
	\frac{n_{\ce{A},0}}{a} = \frac{n_{\ce{B},0}}{b}
	\Lra
	n_{\ce{B},0} = \frac{b}{a}n_{\ce{A},0}
\]
Ici, on a donc $n_{\ce{O2},0} = 3n_{\ce{C2H2OH},0}$, c'est-à-dire
$n_{\ce{O2},0} = \SI{6.00}{mol}$. On peut donc remplir cette
case.
\bigbreak
On suppose qu'on commence sans $\ce{CO2}$ ou $\ce{H2O}$ initialement,
puisque rien n'est indiqué~; en revanche, on sait qu'il y a déjà du diazote dans
le milieu puis que le dioxygène vient de l'air, comme c'est
indiqué. Comme il y a 80\% de \ce{N2} pour 20\% de \ce{O2}, cela veut
dire qu'il y a 4 fois plus de diazote que de dioxygène, donc
\SI{24.00}{mol}. Ici, la colonne $n\ind{tot, gaz}$ n'a pas grande
utilité puisqu'il n'y a qu'un gaz, mais c'est une bonne pratique à ne pas
oublier.
\bigbreak
Le reste du remplissage est le même que pour la question 1. On trouve
évidemment $\xi_f = \SI{2.00}{mol}$ avec les deux réactifs limitant,
c'est le principe des proportions stœchiométriques.
\begin{center}
	\def\rhgt{0.50}
	\centering
	\begin{tabularx}{\linewidth}{|l|c||YdYdYdY||Y|Y|}
		\hline
		\multicolumn{2}{|c||}{
			$\xmathstrut{\rhgt}$
		\textbf{Équation} (\si{mol})} &
		$\ce{C_2H_5OH_{\liq}}$        & $+$                  &
		$3\ce{O_2_{\gaz}}$            & $\ra$                &
		$2\ce{CO_2_{\gaz}}$           & $+$                  &
		$3\ce{H_2O_{\gaz}}$           &
		$n_{\ce{N_2}}$                &
		$n\ind{tot, gaz}$                                               \\
		\hline
		$\xmathstrut{\rhgt}$
		Initial                       & $\xi = 0$            &
		$\num{2.00}$                  & \vline               &
		$\num{6.00}$                  & \vline               &
		$\num{0.00}$                  & \vline               &
		$\num{0.00}$                  &
		$\num{24.00}$                 &
		$\num{30.00}$                                                   \\
		\hline
		$\xmathstrut{\rhgt}$
		Interm.                       & $\xi$                &
		$\num{2.00} - \xi$            & \vline               &
		$\num{6.00} - 3\xi$           & \vline               &
		$2\xi$                        & \vline               & $3\xi$ &
		$\num{24.00}$                 &
		$\num{30.00} + 2\xi$                                            \\
		\hline
		$\xmathstrut{\rhgt}$
		Final                         & $\xi_f = \num{2.00}$ &
		$\num{0.00}$                  & \vline               &
		$\num{0.00}$                  & \vline               &
		$\num{4.00}$                  & \vline               &
		$\num{6.00}$                  &
		$\num{24.00}$                 &
		$\num{34.00}$                                                   \\
		\hline
	\end{tabularx}
\end{center}
}
\end{document}
