\documentclass[../TDM1-M2.tex]{subfiles}%

\begin{document}
\section[s]"1"{Collision entre deux voitures}

\enonce{%
	Pendant le GP explorer organisé par Squeezie en octobre 2022, Pierre suit Xari
	de près en vue de le dépasser. On considère ici que les deux voitures se
	suivent sur une ligne droite à la vitesse de $v_0 = \SI{30}{m.s^{-1}}$ à une
	distance $d = \SI{20}{m}$ l'une de l'autre. À la date $t=0$, la première
	freine avec une décélération constante $a_1 = \SI{-20,0}{m.s^{-2}}$. Celle qui
	suit commence son freinage $\tau = \SI{1}{s}$ plus tard (à cause du temps de
	réaction du conducteur), avec une accélération de $a_2 =
		\SI{-10,0}{m.s^{-2}}$ (décélération).
}

\QR{%
	En prenant pour origine du repère spatial la position de la seconde
	voiture à la date $t=0$, établir les équations horaires du mouvement des
	deux véhicules.
}{%
	Notons M$_1$ et M$_2$ les points matériels représentant chacun une des
	deux voitures. On se limite au mouvement unidimensionnel selon l'axe $x$
	et on notera $x_1(t)$ et $x_2(t)$ les positions respectives de M$_1$ et
	M$_2$ selon cet axe. Initialement, $x_1(t=0) = d = \SI{20}{m}$ et
	$x_2(t=0) = 0$. \bigbreak

	La voiture M$_1$ de Xari subit l'accélération (qui est négative donc
	c'est une décélération) constante $a_1$. Ainsi, par intégration
	successive,
	\[
		x_1(t) = \frac{1}{2}a_1 t^2 + \alpha t + \beta
	\]
	Avec $\alpha$ et $\beta$ deux constantes d'intégration. En considérant
	par ailleurs une vitesse initiale $v_0$ et une position initiale $d$, on
	obtient~:
	\[
		x_1(t) = \frac{1}{2}a_1 t^2 + v_0 t + d
	\]
	Pour le second véhicule, il faut décomposer le mouvement en deux étapes
	successives~:
	\begin{itemize}
		\item pour $t\in \SIrange{0}{1}{s}$, $a = 0$. La position initiale
		      étant par ailleurs nulle et la vitesse initiale étant égale à
		      $v_0$, il vient, pour $t\in \SIrange{0}{1}{s}$~:
		      \[
			      x_2(t) = v_0t
		      \]
		\item
		      pour $t>1$, l'accélération vaut $a_2$ constante. Notons par
		      ailleurs $t_2 = \SI{1}{s}$. On a par intégration~:
		      \[
			      v_2(t) = a_2 t + \gamma
		      \]
		      Avec $\gamma$ une constante à déterminer. Or, par continuité de
		      la vitesse, $v_2(t=t_2) = v_0$. Ainsi,
		      \[
			      v_2(t) = a_2 (t-t_2) + v_0
		      \]
		      Intégrons une nouvelle fois, avec $\delta$ une nouvelle
		      constante d'intégration~:
		      \[
			      x_2(t) = \frac{1}{2} a_2 (t-t_2)^2 + v_0t + \delta
		      \]
		      En utilisant le fait que $x(t_2) = v_0t_2$, il vient finalement
		      \[
			      x_2(t) = \frac{1}{2} a_2 (t-t_2)^2 + v_0t
		      \]
	\end{itemize}
}

\QR{%
	Déterminer la position $x_c$ et la date $t_c$ du contact. Pierre
	avait-il le temps d'esquiver Xari~?
}{%
	Il y a contact à l'instant $t_c$ tel que
	\[
		x_1(t_c) = x_2(t_c)
	\]
	Supposons d'abord le contact sur l'intervalle $t\in\SIrange{0}{1}{s}$.
	Il faut alors résoudre~:
	\begin{gather*}
		\frac{1}{2}a_1 {t_c}^2 + \cancel{v_0t_c} + d = \cancel{v_0t_c}
		\\
		\Leftrightarrow
		\boxed{t_c = \sqrt{\frac{-2d}{a_1}}}
		\qavec
		\left\{
		\begin{array}{rcl}
			d   & = & \SI{20}{m}           \\
			a_1 & = & -\SI{30.0}{m.s^{-2}}
		\end{array}
		\right.\\
		\AN
		\boxed{t_c = \SI{1.41}{s} > \SI{1}{s}}
	\end{gather*}

	Cette solution est donc exclue puisqu'elle n'est pas en accord avec
	notre hypothèse initiale $t\in\SIrange{0}{1}{s}$. \bigbreak

	Supposons maintenant $t_c>\SI{1}{s}$. Il faut résoudre :
	\begin{align*}
		\frac{1}{2}a_1 {t_c}^2 + \cancel{v_0t_c} + d
		 & = \frac{1}{2} a_2 (t_c-t_2)^2 + \cancel{v_0t_c}
		\\
		\Leftrightarrow
		\frac{1}{2}a_1t_c{}^2 + d
		 & = \frac{1}{2}a_2 \left( t_c{}^2 - 2t_2t_c + t_2{}^2\right)
		\\
		\Leftrightarrow
		\frac{1}{2} \left( a_1 - a_2 \right)t_c{}^2 + a_2t_2t_c + d -
		\frac{1}{2}a_2t_2{}^2
		 & = 0
	\end{align*}

	C'est un polynôme de degré 2 dont le discriminant $\Delta$ est tel que
	\begin{gather*}
		\boxed{\D = (a_2t_2)^2 - 2(a_1-a_2)\left(d -
			\frac{1}{2}a_2t_2{}^2\right)}
		\qavec
		\left\{
		\begin{array}{rcl}
			d   & = & \SI{20}{m}           \\
			a_1 & = & -\SI{30.0}{m.s^{-2}} \\
			a_2 & = & -\SI{20.0}{m.s^{-2}} \\
			t_2 & = & \SI{1}{s}
		\end{array}
		\right.\\
		\AN
		\boxed{\D = \SI{600}{m.s^{-2}}}\\
		\text{D'où}\quad
		t_{c,\pm} = \frac{-a_2t_2 \pm \sqrt{\D}}{(a_1-a_2)}\\
		\Leftrightarrow
		t_{c,+} = \SI{-3.45}{s}
		\qou
		t_{c,-} = \SI{1.45}{s}
	\end{gather*}
	La solution négative étant exclue, on trouve finalement
	\[
		\boxed{t_c = \SI{1.45}{s}}
		\qet
		\boxed{x_1(t_c) = \SI{42.5}{m}}
	\]
	Il était donc pratiquement impossible que Pierre esquive Xari, étant
	donné qu'en freinant au plus tôt il n'a eu que \SI{0.45}{s} avant de
	rentrer en collision avec lui, laissant peu de marge à un autre temps de
	réaction et à une autre manœuvre évasive.
}

\end{document}
