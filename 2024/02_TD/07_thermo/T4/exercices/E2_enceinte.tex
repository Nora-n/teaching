\documentclass[../TDT4.tex]{subfiles}%

\begin{document}
\section[s]"1"{Équilibre d'une enceinte à deux compartiments}
\enonce{%
	Une enceinte indéformable aux parois calorifugées est séparée en deux
	compartiments par une cloison étanche, diatherme et mobile sans frottement.
	Les deux compartiments contiennent un même gaz parfait. Dans l'état initial,
	la cloison est maintenue au milieu de l'enceinte. Le gaz du compartiment 1 est
	dans l'état $(T_0,P_0,V_0)$ et le gaz du compartiment 2 dans l'état
	$(T_0,2P_0,V_0)$. On laisse alors la cloison bouger librement jusqu'à ce que
	le système atteigne un état d'équilibre.
}%
\QR{%
	Exprimer les quantités de matière $n_1,n_2$ dans chaque compartiment en
	fonction de $n_0 = P_0V_0/RT_0$.
}{%
	Pas d'échange de matière~:
	\[
		\boxed{n_1 = n_0}
		\qqet
		2P_0V_0 = n_2RT_0
		\Lra
		\boxed{n_2 = 2n_0}
	\]
}%
\QR{%
	Exprimer la température, le volume et la pression du gaz de chaque
	compartiment dans l'état final, en fonction de $n_0,T_0$ et $V_0$.
}{%
	On étudie l'ensemble des gaz~: la transformation est donc \textbf{isochore} et
	\textbf{adiabatique}. On exploite l'équilibre thermodynamique~:
	\begin{itemize}
		\item[b]{Équilibre thermique}: parois diathermanes $\Ra$ échange de chaleur,
		pas de matière donc
		\begin{gather*}
			\boxed{T_1 = T_2}
			\tag{1}
			\label{eq:eqth}
		\end{gather*}
		\item[b]{Équilibre mécanique}: les forces sur la paroi mobile se compensent,
		soit $P_1S = P_2S$, soit
		\begin{gather*}
			\boxed{P_1 = P_2}
			\tag{2}
			\label{eq:eqm}
		\end{gather*}
		\item[b]{Premier principe}:
		\leavevmode\vspace*{-20pt}\relax
		\begin{gather*}
			\Delta{U} = W+Q
			\Lra
			C_V (T_1-T_0) + C_V (T_2 - T_0) = 0
			\\\Lra
			T_1 + T_2 = 2T_0
			\\\beforetext{Or $T_1 \stc{=}{\eqref{eq:eqth}} T_2$}
			\Lra
			\boxed{T_1 = T_2 = T_0}
		\end{gather*}
		\item[b]{Conservation du volume}:
		\leavevmode\vspace*{-20pt}\relax
		\begin{gather*}
			\boxed{V_1+V_2 = 2V_0}
			\tag{3}
			\label{eq:consv}
		\end{gather*}
		\item[b]{Équation d'état gaz parfait}:
		\leavevmode\vspace*{-20pt}\relax
		\begin{gather*}
			\left\{
			\begin{array}{ll}
				P_1V_1 & = n_0RT_0
				\\
				P_2V_2 & = 2n_0RT_0
			\end{array}
			\right.
			\Lra
			P_2V_2 = 2P_1V_1
			\\\beforetext{Or $P_1 \stc{=}{\eqref{eq:eqm}} P_2$}
			\Lra
			\boxed{V_2 = 2V_1}
			\tag{4}
			\label{eq:vdbl}
			\\
			\eqref{eq:vdbl} \text{ et } \eqref{eq:consv}
			\Lra
			3V_1 = 2V_0
			\Lra
			\left\{
			\begin{array}{ll}
				\Aboxed{V_1 & = \frac{2}{3}V_0}
				\\
				\Aboxed{V_2 & = \frac{4}{3}V_0}
			\end{array}
			\right.
			\\\beforetext{Enfin,}
			P_1 \frac{2}{3}V_0 = n_0RT_0
			\Lra
			\boxed{P_1 = \frac{3n_0RT_0}{2V_0} = P_2}
		\end{gather*}
	\end{itemize}
}%
\QR{%
	Exprimer l'entropie créée en fonction de $n_0$. Conclure.
}{%
	Avec $\Delta{S}\sup{G.P.} = C_V \ln \frac{T_f}{T_i} + nR \ln \frac{V_f}{V_i}$,
	on a~:
	\begin{align*}
		\Delta{S}_1       & =
		C_V \underbracket[1pt]{\ln \frac{T_0}{T_0}}_{=0} +
		n_0R \ln \frac{V_1}{V_0} =
		n_0R \ln \frac{2}{3}
		\\
		\Delta{S}_2       & = 2n_0R \ln \frac{4}{3}
		\\\beforetext{$S$ additive, donc}
		\Delta{S}         & = \Delta{S_1} + \Delta{S_2}
		\\\Lra
		\Delta{S}         & = n_0R \left( \ln \frac{2}{3} + 2 \ln \frac{4}{3}\right)
		\\\Lra
		\Delta{S}         & = n_0R \left( \ln \frac{2}{3} + \ln \frac{16}{9} \right)
		\\\Lra
		\Delta{S}         & = n_0R \ln \frac{32}{27}
		\\\beforetext{Or $Q = 0 \Ra S\ind{ech} = 0 \Ra$}
		\Aboxed{S\ind{cr} & = n_0R \ln \frac{32}{27}} > 0
	\end{align*}
	On a donc une transformation \textbf{irréversible} par inhomogénéité de
	pression/matière/concentration.
}%
\end{document}
