\documentclass[../TDT4.tex]{subfiles}%

\begin{document}
\section[s]"2"{Effet \textsc{Joule}}
\enonce{%
	Considérons une masse $m = \SI{100}{g}$ d'eau, dans laquelle plonge un
	conducteur de résistance $R = \SI{20}{\ohm}$. L'ensemble forme un système
	$\Sigma$, de température initiale $T_0 = \SI{20}{\degreeCelsius}$. On impose
	au travers de la résistance un courant $I = \SI{1}{A}$ pendant une durée $\tau
		= \SI{10}{s}$. L'énergie électrique dissipée dans la résistance peut être
	traitée du point de vue de la thermodynamique comme un transfert thermique
	$Q\ind{élec}$ reçu par $\Sigma$.
	\begin{tcn}(defi)<lftt>{Données}
		\begin{itemize}
			\item Capacité thermique de la résistance~: $C_R = \SI{8}{J.K^{-1}}$
			\item Capacité thermique massique de l'eau~: $c\ind{eau} =
				      \SI{4.18}{J.g^{-1}.K^{-1}}$.
		\end{itemize}
	\end{tcn}
}%

\QR{%
	La température de l'ensemble est maintenue constante. Quelle est la
	variation d'entropie du système~? Quelle est l'entropie créée~?
}{%
	On modélise l'eau et la résistance par deux phases condensées idéales. Leur
	entropie ne dépend donc que de la température, et comme la transformation est
	isotherme à $T_0 = \SI{293}{K}$, on a
	\[
		\boxed{\Delta{S} = \Delta{S}\ind{eau} + \Delta{S}_R = 0}
	\]
	Pour calculer l'entropie créée, il faut d'abord calculer l'entropie échangée
	par $\Sigma$, donc le transfert thermique reçu. Comme la transformation est
	isotherme, c'est qu'il y a échange de transfert thermique en plus du transfert
	thermique d'origine électrique. Pour le déterminer, appliquons le premier
	principe à $\Sigma$ de transformation isobare~:
	\[
		\Delta{H} = Q\ind{élec} + Q\ind{autre}
	\]
	Compte-tenu de la modélisation de $\Sigma$ par deux phases condensées idéales,
	son enthalpie ne dépend que de la température (seconde loi de \textsc{Joule}),
	et comme la transformation est isotherme on a $\Delta{U} = 0$. Ainsi,
	\[
		Q\ind{autre} = -Q\ind{élec} = -RI^2\tau
	\]
	Comme $\Delta{S} = 0$, on en déduit
	\[
		S\ind{cr} = -S\ind{ech} = -\frac{Q\ind{autre}}{T}
		\quad \Lra \quad
		\boxed{S\ind{cr} = \frac{RI^2\tau}{T}}
		\Ra
		\xul{S\ind{cr} = \SI{0.68}{J.K^{-1}}}
	\]
	\begin{tcn}(inte){Travail ou chaleur~?}
		Le statut donné à $Q\ind{élec}$ est ambigu~: faut-il considérer qu'il est
		associé à un échange d'entropie~? sous quelle forme~? L'usage est plutôt de
		considérer l'entropie échangée par le système comme celle échangée avec
		l'extérieur, soit uniquement le transfert thermique $Q\ind{autre}$.
		\smallbreak
		Une autre façon équivalent de le dire est de parler de «~travail
		électrique~» plutôt que de transfert thermique électrique. Cela est sans
		doute moins intuitif mais plus cohérent~: la résistance fait partie
		intégrante du système, et reçoit de l'énergie de la part du générateur… donc
		il est clair qu'il ne fournit pas de transfert thermique~!
		\smallbreak
		Finalement, le plus explicite serait peut-être d'écrire le premier principe
		sous la forme $\Delta{U} = W\ind{méca} + Q + \Ec\ind{élec}$, avec
		$\Ec\ind{élec}$ l'énergie électrique reçue par le système, qui prend une
		forme différente d'un travail mécanique $W\ind{méca}$ ou d'un transfert
		thermique $Q$… cependant, on ne le voit presque jamais écrit comme tel.
	\end{tcn}
}%
\QR{%
	Commenter le signe de l'entropie créée. Que peut-on en déduire à propos du
	signe d'une résistance~?
}{%
	L'entropie créée est positive, respectant le second principe (ouf~!) et
	indiquant que la transformation est irréversible. Le signe de la résistance
	contraint le signe de la variation d'entropie~: toutes les autres grandeurs
	intervenant sont positives. Il vient donc qu'\textbf{une résistance électrique
		est forcément positive}.
}%
\QR{%
	Le même courant passe dans le même conducteur pendant la même durée, mais
	cette fois $\Sigma$ est isolé thermiquement. Calculer sa variation d'entropie
	et l'entropie créée.
}{%
	La transformation est désormais adiabatique~:
	\smallbreak
	\begin{isd}[sidebyside align=top]
		\tcbsubtitle{\fatbox{1\ier{} ppe.}}
		\vspace{-15pt}
		\begin{align*}
			\Delta{H}                    & = Q\ind{élec} + 0
			\\\Lra
			(mc\ind{eau} + C_R)\Delta{T} & = RI^2\tau
			\\\Lra
			T_f                          & = T_i + \frac{RI^2\tau}{mc\ind{eau} + C_R}
		\end{align*}
		\tcblower
		\tcbsubtitle{\fatbox{2\up{d} ppe.}}
		\vspace{-15pt}
		\begin{align*}
			\Delta{S} = S\ind{cr} & = (mc\ind{eau} + C_R) \ln \frac{T_f}{T_i}
			\\\Lra
			\Aboxed{
			\Delta{S} = S\ind{cr} & =
				(mc\ind{eau} + C_R) \ln (1 + \frac{RI^2\tau}{(mc\ind{eau} + C_R)T_i})
			}
			\\
			\makebox[0pt][l]{$\phantom{\AN}\xul{\phantom{S\ind{cr} =
							\SI{6.8}{J.K^{-1}}}}$}
			\AN
			S\ind{cr}             & = \SI{6.8}{J.K^{-1}}
		\end{align*}
	\end{isd}
	Il y a donc un facteur 10 d'écart avec la transformation précédente.
}%
\end{document}
