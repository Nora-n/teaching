\documentclass[../TDT4.tex]{subfiles}%

\begin{document}
\section[s]"2"{Corps en contact avec $n$ thermostats quasi-statiques}
\enonce{%
Un métal de capacité thermique $C_P$ passe de la température initiale $T_0$ à
la température finale $T_f = T_N$ par contacts successifs avec une suite $N$
thermostats de températures $T_i$ étagées entre $T_0$ et $T_f$. On prendra le
rapport $T_i/T_{i+1} = \alpha$ constant.
}%
\QR{%
	Exprimer pour chaque étape la variation d'entropie du corps $\Delta S_i$ en
	fonction de $m, c$ et $\alpha$.
}{%
	\leavevmode\vspace*{-20pt}\relax
	\begin{gather*}
		\Delta{S}_i = C_P \ln \frac{T_{i+1}}{T_i} = -mc \ln \alpha
	\end{gather*}
}%
\QR{%
	Calculer le transfert thermique reçu par le métal sur une étape en fonction de
	$T_{i+1}$ et $T_i$, puis l'entropie échangée $S\ind{ech}$ en fonction de $m,
		c$ et $\alpha$.
}{%
	\leavevmode\vspace*{-20pt}\relax
	\begin{gather*}
		\Delta{U}_i = \boxed{mc (T_{i+1}-T_i) = Q_i}
		\qet
		\boxed{S_{\mathrm{ech},i} = \frac{Q_i}{T_{i+1}} = mc(1-\alpha)}
	\end{gather*}
}%
\QR{%
	Calculer la variation d'entropie du corps $\Delta S$, l'entropie échangée
	$S\ind{ech}$ ainsi que l'entropie créée $S_c$ sur l'ensemble en fonction de
	$C_P, \alpha$ et $N$.
}{%
	\leavevmode\vspace*{-20pt}\relax
	\begin{align*}
		\Aboxed{\Delta{S}  & = \sum_i \Delta{S}_i = -NC_P \ln \alpha}
		\\
		\Aboxed{S\ind{ech} & = \sum_i S_{\mathrm{ech},i} = NC_P (1-\alpha)}
		\\
		\Aboxed{S\ind{cr}  & = \Delta{S} - S\ind{ech} = NC_P (\alpha-1-\ln \alpha)}
	\end{align*}
}%
\QR{%
	Étudier $S\ind{cr}$ pour $N \ra \infty$. On exprimera $\alpha$ en fonction de
	$T_f, T_i$ et $N$, et on utilisera le développement limité $\exp(x) =
		1+x+x^2/2$ pour $x$ petit devant 1. Conclure.
}{%
	On voit vite qu'on peut exprimer $\alpha^N$ grâce à une technique de
	télescopie~:
	\begin{align*}
		\alpha =
		\frac{T_{N-1}}{T_N} =
		\frac{T_{N-2}}{T_{N-1}} =
		\ldots =
		\frac{T_0}{T_1}
		          & \Ra
		\alpha^N =
		\frac{\cancel{T_{N-1}}}{T_N}\frac{\bcancel{T_{N-2}}}{\cancel{T_{N-1}}}
		\ldots
		\frac{\dcancel{T_1}}{\bcancel{T_2}}\frac{T_0}{\dcancel{T_1}}
		\\\Lra
		\alpha^N = \frac{T_0}{T_f}
		          & \Lra
		\boxed{\alpha = \left( \frac{T_0}{T_f} \right)^{1/N}}
		\\\beforetext{Ainsi,}
		S\ind{cr} & =
		C_P N \left(
		\left( \frac{T_0}{T_f} \right)^{1/N} -
		1 -
		\ln (\frac{T_0}{T_f})^{1/N}
		\right)
		\\\Lra
		S\ind{cr} & =
		C_P \left(
		N \left( \frac{T_0}{T_f} \right)^{1/N} -
		N -
		\ln (\frac{T_0}{T_f})
		\right)
		\\\Lra
		S\ind{cr} & =
		C_P \left(
		N \exp (\frac{1}{N} \ln (\frac{T_0}{T_f})) -
		N -
		\ln (\frac{T_0}{T_f})
		\right)
		\intertext{Or, comme $\ln (\frac{T_0}{T_f}) = \cte$, on a $\DS \frac{1}{N}
				\ln (\frac{T_0}{T_f}) \opto{}{N \to \infty} 0$. On peut effectuer le
			développement limité $\exp (x) \Sim_{x \to 0} 1+x+\frac{x^2}{2}$, soit}
		S\ind{cr} & \Sim_{N \to 0}
		C_P \left(
		N \left(
			1 +
			\frac{1}{N} \ln (\frac{T_0}{T_f}) +
			\frac{1}{2N^2} \left[ \ln (\frac{T_0}{T_f})\right]^2
			\right) -
		N -
		\ln (\frac{T_0}{T_f})
		\right)
		\\\Lra
		S\ind{cr} & \Sim_{N \to 0}
		C_P \left(
		N +
		\cancel{\ln (\frac{T_0}{T_f})} +
		\frac{1}{2N} \left[ \ln (\frac{T_0}{T_f})\right]^2 -
		N -
		\cancel{\ln (\frac{T_0}{T_f})}
		\right)
		\\\Lra
		S\ind{cr} & \Sim_{N \to 0}
		\frac{C_P}{2N} \left[ \ln (\frac{T_0}{T_f}) \right]^2
		\\\Lra
		\Aboxed{
		S\ind{cr} & \opto{}{N \to \infty} 0
		}
	\end{align*}
	Ainsi, en effectuant des transformations \textbf{infinitésimales}, on trouve
	bien une entropie créée nulle, donc une transformation \textbf{réversible}~!
}%

\end{document}
