\documentclass[../TDT5.tex]{subfiles}%

\begin{document}
\section{Rafraîchir sa cuisine en ouvrant son frigo}
\enonce{%
	Un réfrigérateur est une machine thermique à écoulement, dans laquelle un
	fluide subit une série de transformations thermodynamiques cyclique. À chaque
	cycle, le fluide extrait de l'intérieur du frigo un transfert thermique
	$\abs{Q\ind{int}}$, cède un transfert thermique $\abs{Q\ind{ext}}$ à la pièce
	dans laquelle se trouve le frigo et reçoit un travail $\abs{W}$ fourni par un
	moteur électrique.
	\smallbreak
	On fait l'hypothèse que l'intérieur du réfrigérateur et l'air ambiant
	constituent deux thermostats aux températures respectives $T\ind{int} =
		\SI{268}{K}$ et $T\ind{ext} = \SI{293}{K}$, et qu'en dehors des échanges avec
	ces thermostats les transformations sont adiabatiques.
}%
\QR{%
	Quel est le signe des énergies échangées~?
}{%
	Un frigo est une machine réceptrice, qui prélève du transfert thermique à la
	source froide ($Q\ind{int} > 0$ car le transfert thermique est fourni au
	fluide) pour le céder à la source chaude ($Q\ind{ext} < 0$). Cela demande de
	fournir de l'énergie sous forme de travail ($W > 0$).
}%
\QR{%
	Lorsqu'il fait très chaud en été, est-ce une bonne idée d'ouvrir la porte de
	son frigo pour refroidir sa cuisine~? Pourquoi~?
}{%
	Pour pouvoir refroidir sa cuisine en ouvrant son frigo, il faudrait que
	globalement le transfert thermique prélevé à l'intérieur du frigo
	$\abs{Q\ind{int}}$ (qui serait finalement prélevé à l'air de la cuisine,
	puisque la porte est ouverte) soit \textbf{plus grand} que celui cédé à la
	source chaude $\abs{Q\ind{ext}}$, qui n'est autre que l'air de la cuisine. Or,
	avec le premier principe,
	\[
		W + \abs{Q\ind{int}} - \abs{Q\ind{ext}} = 0
		\Lra
		\abs{Q\ind{int}} - \abs{Q\ind{ext}} = -W < 0
	\]
	On en déduit qu'il est \textbf{impossible} d'avoir $\abs{Q\ind{int}} >
		\abs{Q\ind{ext}}$ comme on l'aurait aimé~: laisser son frigo ouvert \textbf{ne
		peut que conduire à réchauffer l'air de la cuisine}.
}%
\QR{%
	Quelle est la différence avec un climatiseur~?
}{%
	Un climatiseur est relié à l'\textbf{extérieur} de la maison, qui joue le rôle
	de source chaude. Ainsi, quand on climatise sa maison ou sa voiture par une
	journée de canicule, on réchauffe l'air extérieur.
}%

\end{document}
