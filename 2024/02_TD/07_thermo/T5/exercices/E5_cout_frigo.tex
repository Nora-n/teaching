\documentclass[../TDT5.tex]{subfiles}%

\begin{document}
\section{Coût énergétique d'un goûter}
\enonce{%
	Pour préparer le goûter de fin d'année avec vos professeures, vous achetez six
	bouteilles de \SI{1}{L} de différents jus que vous rangez dans votre
	réfrigérateur. Une heure plus tard, elles sont à la température du frigo.
	\begin{tcn}(defi)<lftt>{Données}
		\begin{itemize}
			\item L'efficacité thermodynamique du réfrigérateur vaut $70\%$ de
			      l'efficacité de \textsc{Carnot}~;
			\item L'isolation imparfaite du réfrigérateur se traduit par des fuites
			      thermiques de puissance $\SI{10}{W}$~;
			\item Tarifs électricité~: $\SI{1}{kWh}$ coûte $\SI{0.25}{\text{\euro}}$.
		\end{itemize}
	\end{tcn}
}%
\QR{%
	Combien vous coûte ce refroidissement~?
}{%
	Supposons que les bouteilles de jus de fruit sont à température initiale $T_I
		= \SI{25}{\degreeCelsius}$, et que la température finale (celle du frigo) vaut
	$T_F = \SI{5}{\degreeCelsius}$. Commençons par calculer l'énergie nécessaire
	au refroidissement.
	\begin{itemize}
		\item[b]{Système}: \{contenu du frigo\}
		\item[b]{Bilan des échanges}:
		\begin{itemize}
			\item Transfert thermique reçu de la part du fluide frigorigène~:
			      $Q\ind{frigo} < 0$ que l'on cherche à déterminer~;
			\item transfert thermique de fuite~: $Q\ind{fuite} =
				      +\Pc\ind{fuite}\Delta{t} > 0$ avec $P\ind{fuite} = \SI{10}{W}$ et
			      $\Delta{t} = \SI{1}{h} = \SI{3.6e3}{s}$~: \textbf{attention au signe},
			      compte-tenu de la différence de température, c'est le \textbf{contenu
				      du frigo} qui reçoit effectivement de l'énergie.
		\end{itemize}
		\item[b]{Variation d'énergie interne}: on assimile les jus à de l'eau du
		point de vue thermique, soit
		\begin{gather*}
			\Delta{U} = m\ind{jus}c\ind{eau} (T_F-T_I) = Q\ind{frigo} + Q\ind{fuite}
			\\\Lra
			Q\ind{frigo} m\ind{jus}c\ind{eau} (T_F - T_I) - \Pc\ind{fuite}\Delta{t}
			= \SI{-5.4e5}{J}
		\end{gather*}
	\end{itemize}
	Calculons maintenant le coût en énergie électrique du refroidissement. On fait
	l'hypothèse que l'énergie électrique fournie au frigo ne sert qu'à faire
	tourner le moteur. Par définition de l'efficacité d'un frigo, $e =
		\abs{Q\ind{froid}/W}$ où les échanges sont ceux du fluide. Ici, on a donc
	\[
		e = \abs{Q\ind{frigo}}/\Ec\ind{élec}
	\]
	Par ailleurs, l'efficacité de \textsc{Carnot} d'un frigo vaut $e_C =
		T\ind{frigo}/(T\ind{ext} - T\ind{frigo})$. En combinant, on en déduit
	\[
		e =
		\frac{\abs{Q\ind{frigo}}}{\Ec\ind{élec}} =
		\num{0.7}\frac{T\ind{frigo}}{T\ind{ext}-T\ind{frigo}} \approx 10
		\qqdc
		\Ec\ind{élec} = \frac{\abs{Q\ind{frigo}}}{e} = \SI{5e4}{J}
	\]
	Enfin, calculons le prix en euros de cette énergie, sachant que $\SI{1}{kWh} =
		\SI{1e3}{W}\times \SI{3.6e3}{s} = \SI{3.6e6}{J}$. On trouve
	\[
		\xul{
			p =
			\frac{\SI{5e4}{J}}{\SI{3.6e6}{J}}\times \SI{0.25}{\text{\euro}} \approx
			\SI{0.33}{\text{\euro}}
		}
	\]
}%

\end{document}
