\documentclass{standalone}
\usepackage{mintikz}

\colorlet{mylightblue}{blue!20}
\colorlet{myblue}{blue!80!black}
\colorlet{mydarkblue}{blue!30!black}
\colorlet{mylightred}{red!10}
\colorlet{myred}{red!80!black}
\colorlet{mydarkred}{red!60!black}
\colorlet{mydarkgreen}{green!30!black}

\tikzset{
  midarr/.style={decoration={markings,mark=at position #1 with
  {\arrow[scale=1.3]{stealth}}},postaction={decorate}},
  midarr/.default=0.5
}

\begin{document}
% PV diagram - Otto cycle
\def\N{40} % number of plot samples
\def\xmax{4}
\def\ymax{3}
\begin{tikzpicture}
	\def\Ch{2.5}
	\def\Cc{0.9}
	\def\N{40}
	\def\gam{1.4}
	\def\adiabatic#1#2{{ #2/(#1)^(\gam) }}
	\def\xA{.24*\xmax}
	\def\xB{.90*\xmax}
	\coordinate (A) at ({\xA},{\adiabatic{\xA}{\Ch}});
	\coordinate (B) at ({\xB},{\adiabatic{\xB}{\Ch}});
	\coordinate (C) at ({\xB},{\adiabatic{\xB}{\Cc}});
	\coordinate (D) at ({\xA},{\adiabatic{\xA}{\Cc}});
	\coordinate (E) at (C-|A);

	% WORK
	\fill[gray, opacity=.3, domain={\xA:\xB},samples=\N]
	plot (\x,\adiabatic{\x}{\Ch}) --
	plot[domain={\xB:\xA}](\x,\adiabatic{\x}{\Cc}) -- cycle;
	% \node[below=-5,blue,scale=.9] at ($(D)!.4!(B)$) {$W$};

	\node[darkgray, scale=1.3] (W) at
	([shift={(0,-.50)}]barycentric cs:A=1,B=1,C=1,D=1)
	{$W$};

	% ADIABATIC TRANSFORMATION
	\draw[myred,thick,midarr=.18,midarr=.75,
		mynode={.20}{left=-1.2}{isoV},
		mynode={.80}{above=.2}{adia},
		domain={\xA:\xB},samples=\N]
	(D) -- (A)
	plot (\x,\adiabatic{\x}{\Ch});
	\draw[blue,thick,midarr=.1,midarr=.62,
		mynode={.00}{above right=.45 and -.3}{isoV},
		mynode={.80}{below=.05}{adia},
		domain={\xB:\xA},samples=\N]
	(B) -- plot(\x,\adiabatic{\x}{\Cc});
	% \draw[blue!80!white,thick,midarr=.30]
	% (C)++(0,.02) -- ($(E)+(0,.02)$);
	% \draw[blue!80!black,thick,midarr=.25]
	% (E)++(0,-.03) -- ($(C)+(0,-.03)$);
	% \node[blue,above=2,right,scale=0.75] at (B) {exhaust};
	% \node[blue!80!black, scale=0.75] at ([shift={(1.95,-.20)}]E) {échappe\mnt};
	% \node[blue!80!black, scale=0.75] at ([shift={(.5,-.20)}]E) {admission};
	% \node[isoT, above right=.4 and -.00, scale=0.75, rotate=90] at (D) {isoV};

	% HEAT
	\draw[>={LaTeX[width=5,length=4]},->,line width=2, red, opacity=.5]
	($(D)!.20!(A)$)++(-.2,0) --++ (.5,0)
	node[at start,left,scale=.8] {$Q_C$};

	\draw[>={LaTeX[width=5,length=4]},->,line width=2,blue, opacity=.5]
	($(B)!.72!(C)$)++(-.2,0) --++ (.5,0) node[at end, right=.00,scale=.8] {$Q_F$};

	% POINTS
	\fill[black]
	(A) circle(0.05) node[above] {3}
	(B) circle(0.05) node[above] {4}
	(C) circle(0.05) node[below right] {1}
	(D) circle(0.05) node[left] {2};
	% (E) circle(0.05) node[above] {A};

	% % AXIS
	% \draw[->,thick] (0,-0.1*\ymax) -- (0,\ymax+0.1)
	% node[anchor=north east,inner sep=4,scale=1] {$P$};
	% \draw[->,thick] (-0.1*\xmax,0) -- (\xmax+0.1,0)
	% node[anchor=north east,inner sep=4,scale=1] {$V$};

	% AXIS
	\draw[-{Stealth[scale=1.3]},thick]
	(0,-0.2*\ymax) -- (0,\ymax+0.1)
	node[anchor=north east] {$P$};
	\draw[-{Stealth[scale=1.3]},thick]
	(-0.1*\xmax,-0.1*\ymax) -- (\xmax+0.1,-0.1*\ymax)
	node[anchor=north east] {$V$};

\end{tikzpicture}
\end{document}
