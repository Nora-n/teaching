\documentclass[../main/main.tex]{subfiles}
\graphicspath{{./figures/}}
\usepackage{pdfpages}

\makeatletter
\renewcommand{\@chapapp}{Travaux pratiques -- TP}
\makeatother

% \toggletrue{student}
% \HideSolutionstrue
% \toggletrue{corrige}
% \renewcommand{\mycol}{black}
\renewcommand{\mycol}{gray}

\begin{document}
\setcounter{chapter}{14}

\chapter{\cswitch{Correction du TP}{Analyses spectrales de signaux \'electriques}}

\enonce{
\begin{prgm}
	\begin{tcb}*(ror)"how"{Savoir-faire}
		\begin{itemize}
			\item Mettre en œuvre un dispositif expérimental illustrant l’utilité
			      des fonctions de transfert pour un système linéaire à un ou
			      plusieurs étages.
			\item Étudier le filtrage linéaire d’un signal non sinusoïdal à partir
			      d’une analyse spectrale.
			\item Détecter le caractère non linéaire d’un système par l’apparition
			      de nouvelles fréquences.
		\end{itemize}
	\end{tcb}
\end{prgm}
\vspace{-10pt}
\section{Objectifs}
\begin{itemize}
	\item Mettre en œuvre un dispositif expérimental illustrant l'utilité des
	      fonctions de transfert pour  un système linéaire à un ou plusieurs
	      étages.
	\item Étudier le filtrage linéaire d'un signal non  sinusoïdal à partir
	      d'une analyse spectrale.
	\item Choisir un modèle de filtre en fonction d'un cahier des charges.
\end{itemize}

\section{S'approprier~: analyse spectrale}

\subsection{Décomposition en série de Fourier}

\begin{tcb}(prop){Décomposition en série de Fourier}
	Toute fonction périodique peut se décomposer en série de Fourier, c'est-à-dire
	en une somme de fonctions sinusoïdales de pulsations différentes. Soit $y$ une
	fonction périodique de période $T$ et de pulsation $\w = 2\pi/T$. La
	décomposition en série de Fourier de $y$ est~:

	\[
		y(t) = y_0 + \sum_{n=1}^{\infty} a_n \cos(n\wt + \f)
	\]

	Avec $a_n$ et $\f_n$ respectivement l'amplitude et la phase de l'harmonique de
	rang $n$.
\end{tcb}

% Un logiciel tel que LatisPro (que l'on va utiliser ici) permet de réaliser
% numériquement la décomposition en série de Fourier (en réalité une transformée
% de Fourier numérique) d'un signal et de fournir une représentation graphique
% exploitable.

\begin{tcb}[sidebyside](exem){Exemple}
	\begin{center}
		\includegraphics[height=4.5cm]{tp15_spectre-temp}
		\captionof{figure}{Représentation temporelle d'un signal périodique.}
	\end{center}
	\tcblower
	\begin{center}
		\includegraphics[height=4.5cm]{tp15_spectre-f}
		\captionof{figure}{Spectrogramme du même signal périodique.}
	\end{center}
\end{tcb}

\subsection{Vocabulaire}

\begin{tcb}(defi){Vocabulaire}
	\begin{itemize}
		\item Spectre~: représentation de l'amplitude de chacune des composantes
		      spectrales d'un signal en fonction de leurs pulsations ou de leurs
		      fréquences.
		\item $y_0 = \moy{y}$ est la valeur moyenne du signal $y$, c'est-à-dire sa
		      \textbf{composante continue}~;
		\item $a_1 \cos(\wt + \f_1)$ est appelé
		      \textbf{fondamental}~;
		\item $a_i \cos(n\wt + \f_i)$ est l'\textbf{harmonique} de
		      rang $i$.
	\end{itemize}
\end{tcb}

\begin{tcb}(rema)<lftt>'l'{Remarques}
	\begin{enumerate}
		\item Le fondamental est aussi l'harmonique de rang $1$.
		\item Le spectre d'un signal temporel pair ne contient que des
		      harmoniques de rang pair ($n=2p, p\in\Nb$)
		\item Le spectre d'un signal temporel impair ne contient que des
		      harmoniques de rang impair ($n=2p+1, p\in\Nb$)
	\end{enumerate}
\end{tcb}

\subsection{Durée d'enregistrement et fréquence d'échantillonnage}

\begin{tcb}(prop){Échantillonnage}
	Le critère de \textsc{Shannon} (vu en seconde année) impose que la
	\textbf{fréquence d'échantillonnage} (fréquence de calcul) soit
	\textbf{supérieure à deux fois la fréquence maximale du signal} étudié.
	\bigbreak
	Par ailleurs, le temps d'acquisition total $T_{\rm acq\, tot}$ doit être égal à
	un multiple entier de fois la période du signal étudié~: $T_{\rm acq\, tot} =
		nT$ avec $n\in \Zb$. Si ce n'est pas possible, il faut que la durée
	d'acquisition soit longue, sachant que le pas fréquentiel du spectre vaudra~:
	\[
		\Delta f = \frac{1}{T_{\rm acq\, tot}}
	\]
	}
\end{tcb}

\setcounter{section}{2}
\section{Réaliser et valider}
\subsection{Analyses spectrales de signaux périodiques de différentes formes}
\subsubsection{Signal sinusoïdal}

\setlist[blocQR,1]{leftmargin=20pt, label=\clenumi}
\begin{tcb}[breakable](expe)<itc>{}
	\underline{Réaliser une acquisition}~:
	\begin{enumerate}
		\item Connecter le générateur basses fréquences (GBF) à l'interface SYSAM
		      entre les voies EA0 et la masse.
		\item Ouvrir le logiciel Latispro en suivant le chemin~: programmes
		      $\rightarrow$ discipline $\rightarrow$ physique-chimie $\rightarrow$
		      latispro.
		\item Allumer le GBF, choisir un signal sinusoïdal de fréquence
		      $\SI{500}{Hz}$ et d'amplitude moyenne ($5-\SI{10}{V}$ par exemple).
		\item Pour faire une acquisition~: cliquer sur le bouton
		      \includegraphics[height=.5cm]{bouton_acq}
		      \begin{itemize}
			      \item \textit{Pour activer la voie EA0}~: Dans le cadre entrées
			            analogiques, cliquer sur les boutons des entrées à activer (EA0
			            ici~!).
			      \item \textit{Pour paramétrer l'acquisition}~: Dans le cadre
			            acquisition, onglet temporel, mode normal, entrer le nombre de
			            points de mesure et la durée totale de l'acquisition. On
			            choisira~:
			            \begin{itemize}
				            \item Nombre de points~: $\num{10 000}$~;
				            \item Acquisition temporelle~;
			            \end{itemize}
			            \QR{%
				            Durée totale d'acquisition $T_{\rm acq\, tot}$ à
				            choisir. Justifier ce choix succintement.
			            }{%
				            solu
			            }
			            \begin{itemize}
				            \item Fin des réglages, vous êtes prêt-e à faire vos
				                  enregistrements.
			            \end{itemize}
			      \item Lancer l'acquisition en cliquant sur
			            \includegraphics[height=.5cm]{bouton_lancer}
		      \end{itemize}
	\end{enumerate} \bigbreak

	\underline{Tracer le spectre}~:
	\begin{enumerate}
		\item Aller dans traitements $\rightarrow$ calculs spécifiques $\rightarrow$
		      analyse de Fourier.
		\item Accéder à la liste des courbes gràce à
		      \includegraphics[height=.5cm]{bouton_courbe}
		\item Glisser la courbe et cliquer sur calcul.
	\end{enumerate} \bigbreak
\end{tcb}

\setlist[blocQR,1]{leftmargin=10pt, label=\sqenumi}
\QR{%
	Quelle est l'allure du spectre~? Observez-vous des harmoniques~?
}{%
	solu
}
\QR{%
	En cliquant droit sur le graphe, prendre la loupe + pour zoomer,
	plusieurs fois si nécessaire ou utiliser le calibrage. Relever la
	fréquence fondamentale grâce à la fonction réticule (toujours en
	cliquant droit sur le graphe) et la comparer à celle indiquée par le
	GBF. Commenter l'éventuelle différence.
}{%
	solu
}

\subsubsection{Signaux triangulaires et carrés}

Changer la forme du signal délivré par le GBF en gardant la même fréquence
fondamentale et recommencer le même protocole. \bigbreak

\QR{%
	Quelle est l'allure de chacun des spectres (signal triangulaire et
	carré)~? Observez-vous des harmoniques~?
}{%
	solu
}
\QR{%
	Quelle est la particularité de ces deux spectres~? Quelles sont leurs
	différences~?
}{%
	solu
}

\subsection{Étude du spectre obtenu en sortie du filtre de Rauch}

On reprend le filtre de \textsc{Rauch} de la semaine précédente afin de filtrer
le signal carré~:
\begin{gather*}
	\boxed{
		\Hu(\jj \w) = \frac{\su}{\eu} = \frac{H_0}{1+\jj Q
			\left(\dfrac{\w}{\w_0}-\dfrac{\w_0}{\w}\right)}
	}
	\\
	\beforetext{Avec}
	Q = \sqrt{\frac{\alpha+1}{2\alpha}},
	\quad
	H_0 = -1
	\qet
	\w_0 = \sqrt{\frac{\alpha+1}{2\alpha}} \, \frac{1}{RC}
\end{gather*}
Notre objectif est d'obtenir à partir de ce signal un signal
sinusoïdal de \textbf{fréquence fondamentale double}.

\begin{tcb}(expe)<itc>{Manipulation amplificateur}
	\begin{enumerate}
		\item Connecter la borne \SI{+15}{V} du boitier à la sortie \SI{+15}{V} d'un
		      générateur de tension continue,
		\item Connecter la borne \SI{-15}{V} du boitier à la sortie
		      \SI{-15}{V} du générateur
		\item Connecter le point milieu du boitier à la masse du
		      générateur.
		      \begin{center}
			      \begin{tcb}[cnt,bld,width=.9\linewidth](impo){Attention}
				      À la fin de la séance, on coupe le signal du GBF avant les
				      alimentations de l'amplificateur opérationnel qui doivent être
				      coupées en dernier.
			      \end{tcb}
		      \end{center}
		\item Réalisez ensuite le montage en prenant $C=\SI{1}{nF}$ (cavalier prêt à
		      être connecté sur la boite) et $\alpha R$ avec une boite de résistances
		      variables.
	\end{enumerate}
\end{tcb}

On réalise le montage en prenant $C=\SI{1}{nF}$ (cavalier prêt à être connecté
sur la boite) et $\alpha R$ est une boite de résistances variables. Le filtre a
été fabriqué avec $R= \SI{100}{k\Omega}$.
\bigbreak
On s'intéresse tout d'abord au cas où $\alpha= 1$~: On prend donc $\alpha R =
	\SI{100}{k\Omega}$. On injecte à l'entrée du filtre un signal créneau de
fréquence fondamentale $f_0$.
\QR{%
	Comment choisir $f_0$ \textit{a priori} afin d'obtenir à partir de ce
	signal un signal sinusoïdal de \textbf{fréquence fondamentale double}~?
}{%
	solu
}

Choisir une durée d'enregistrement telle que $T_{\rm acq\, tot} = 2 T$, ou une
durée d'enregistrement très grande pour que l'analyse spectrale soit de bonne
qualité. Faire l'analyse spectrale du signal à l'entrée et à la sortie du
filtre.

\QR{%
	Qu'observez-vous~? Quelle est l'allure du signal de sortie~?
}{%
	solu
}

Refaire le même protocole pour $\alpha=\num{e-2}$~: on prend donc $\alpha R =
	\SI{1000}{\Omega}$. On rappelle que la fréquence de résonance trouvée la semaine
précédente dans ce cas est différente.

\QR{%
	Quelle valeur faut-il alors choisir pour la fréquence fondamentale du
	créneau~? En déduire la valeur à donner à $T_{\rm acq tot}$.
}{%
	solu
}

\section{Conclure}

\QR{%
	Comparez les deux spectres de sortie. Interprétez les différences
	obtenues. Quel filtre permet d'atteindre l'objectif que l'on s'est
	initialement fixé~?
}{%
	solu
}

\end{document}
