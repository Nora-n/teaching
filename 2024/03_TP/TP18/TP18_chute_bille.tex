\documentclass[../main/main.tex]{subfiles}
\graphicspath{{./figures/}}
\usepackage{pdfpages}

\makeatletter
\renewcommand{\@chapapp}{Travaux pratiques -- TP}
\makeatother

% \toggletrue{student}
\toggletrue{corrige}
% \renewcommand{\mycol}{black}
\renewcommand{\mycol}{gray}

\begin{document}
\setcounter{chapter}{17}

% \settype{enon}
% \settype{solu_prof}
% \settype{solu_stud}

\chapter{\cswitch{Correction du TP}{\'Etude de la chute d'une bille en fluide
	  visqueux}}

\begin{pycode}
	import numpy as np

	m = 10.4e-3    # kg
	tau = 73.5e-3  # s
	R = 1.0e-2     # m
	eta = m/(6*np.pi*tau*R)
	etat = 1.5
	taut = m/(6*np.pi*etat*R)
\end{pycode}

\enonce{
	\begin{prgm}
		\begin{tcb}*(ror)"how"{Savoir-faire}
			\begin{itemize}
				\item Choisir de façon cohérente la fréquence d’échantillonnage et la
				      durée totale d’acquisition.
				\item Évaluer, par comparaison à un étalon, une longueur (ou les
				      coordonnées d’une position) sur une image numérique et en estimer la
				      précision.
				\item Enregistrer un phénomène à l’aide d’une caméra numérique et
				      repérer la trajectoire à l’aide d’un logiciel dédié, en déduire la
				      vitesse et l’accélération.
				\item Mettre en œuvre un protocole expérimental de mesure de
				      frottements fluides.
			\end{itemize}
		\end{tcb}
	\end{prgm}
	\vspace{-10pt}
	\section{Objectifs}
	\begin{itemize}
		\item Reconnaître si le mouvement du centre d'inertie est rectiligne
		      uniforme ou non.
		\item Reconnaitre le régime transitoire et le régime permanent.  Déterminer
		      la vitesse limite.
		\item Évaluer le temps caractéristique $\tau$ par deux méthodes.
		\item Trouver un ordre de grandeur de la viscosité $\eta$ de l'huile de
		      silicone.
	\end{itemize}
	\section{S'approprier}
	\begin{tcb}(mate){Matériel}
		\begin{itemize}
			\item Éprouvette contenant l'huile de silicone et des billes
			\item \textit{Webcam}
			\item Ordinateur et logiciel \texttt{Latispro}
			\item Chronomètre
		\end{itemize}
	\end{tcb}
	\begin{tcb}(data){Données}
		\begin{itemize}
			\item Bille orange~: $R = \SI{1,0}{cm}$ et $m = \SI{10,4}{g}$.
			\item Masse volumique de l'huile de silicone~: $\rho_0 =
				      \SI{970}{kg.m^{-3}}$.
			\item $g = \SI{9,8}{m.s^{-2}}$.
		\end{itemize}
	\end{tcb}
	\subsection{Principe}
	Une éprouvette contenant un liquide visqueux sert de support à l'étude de la
	chute d'une bille d'acier. La bille, qui constitue le système matériel étudié,
	est lâchée sans vitesse initiale à l'instant $t = 0$.
	\bigbreak
	Le système d'acquisition vidéo est assuré par une \textit{webcam} couplée à un
	ordinateur et réglée de manière à enregistrer 20 images par seconde. La
	position instantanée $y$ du centre de gravité G de la bille est repérée par
	l'axe vertical $(\Or y)$ orienté vers le bas, de vecteur unitaire $\uy$.
}

\setcounter{section}{2}
\section{Analyser}

On étudie le mouvement de translation d'une bille de rayon $R$ et de masse
volumique $\rho$ dans de l'huile de silicone de viscosité $\eta$. On admettra
que les actions de frottement exercées par le liquide sur la bille en mouvement
à la vitesse $\vv{v}$ sont modélisables par une force de frottement
$\vv{f}$ telle que
\[
	\vv{f} = -6\pi \,\eta \, R \, \vv{v}
\]
\bigbreak
On dépose la bille en O sans vitesse initiale dans l'huile de silicone contenue
dans une grande éprouvette. On exprimera  toutes les expressions littérales en
fonction de $\rho_0$, $\eta$, $R$, $m$ et $g$.

\setlist[blocQR,1]{leftmargin=10pt, label=\clenumi}
\QR{%
	Donner les caractéristiques de la poussée d'Archimède $\vv{\Pi}$
	exercée sur la bille plongeant dans l'huile de silicone.
}{%
	C'est une force verticale, orientée vers le haut, de module égal au
	\textbf{poids du fluide qui serait occupé par le volume de l'objet}~:
	\[
		\vv{\Pi} = -\rho_0\frac{4}{3}\pi R^{3}\gf
	\]
}

\QR{%
	Faire le bilan des forces exercées sur la bille plongeant dans l'huile
	de silicone en précisant le référentiel de travail.
}{%
	\noindent
	\begin{minipage}[t]{.70\linewidth}
		On établit le système d'étude~:
		\begin{itemize}
			\bitem{Système}~:
			\{bille\} dans $\Rc\ind{labo}$ supposé galiléen
			\bitem{Schéma}~: cf. Figure~\ref{fig:eprouv}.
			\bitem{Modélisation}~:
			repère $(\Or, \uy)$,
			repérage~: $\OM = y\uy$, $\vf = \yp\uy$, $\af = \ypp\uy$.
			\bitem{Bilan des forces~:}
			\[
				\begin{array}{ll}
					\textbf{Poids}               &
					\Pf = mg\uy                               \\
					\textbf{Poussée d'Archimède} &
					\vv{\Pi} = -\frac{4\pi\rho_0 R^3}{3}g \uy \\
					\textbf{Frottement fluide}   &
					\vv{f} = -6\pi\eta R v \uy
				\end{array}
			\]
		\end{itemize}
	\end{minipage}
	\hfill
	\begin{minipage}[t]{.25\linewidth}
		\vspace{0pt}
		\begin{center}
			\includegraphics[width=.7\linewidth]{schema_eprouv}
			\captionof{figure}{Schéma}
			\label{fig:eprouv}
		\end{center}
	\end{minipage}
}

\QR{%
	Établir l'équation différentielle que vérifie la valeur de la vitesse
	$\vv{v}$ du centre d'inertie de la bille, sous la forme~:
	\[
		\dv{v}{t} + \frac{6\pi \eta R}{m}v = g\pa{1-\frac{4\pi \rho_0 \,
				R^3}{3m}}
	\]
}{%
	On applique le PFD à la bille~:
	\begin{DispWithArrows*}
		m\dv{\vv{v}}{t} &= \vv{P} + \vv{\Pi} +\vv{f}
		\CArrow{$\cdot \uy$}
		\\\Ra
		m\dv{v}{t} &=  g\left(m-\rho_0 \frac{4}{3}\pi R^{3}\right) -6\pi\eta R v
		\Arrow{Forme canonique}
		\\\Lra
		\Aboxed{
			\dv{v}{t} + \frac{6\pi\eta R}{m}v &= g\left(1-\frac{4\pi\rho_0R^3}{3m}\right)
		}
	\end{DispWithArrows*}
}

\QR{%
	Montrer que la vitesse de la bille tend vers une vitesse limite
	$v\ind{limTheo}$ telle que~:
	\[
		v\ind{limTheo} = \frac{g}{6\pi \eta R} \pa{m-\frac{4\pi \rho_0 \, R^3}{3}}
	\]
}{%
	On trouve $v\ind{lim}$ en l'injectant dans l'équation différentielle, avec
	$\dv{v\ind{lim}}{t} = 0$, d'où le résultat immédiat.
}

\QR{%
	Donner l'expression de la constante de temps $\tau\ind{theo}$ du
	mouvement en fonction de $m$, $\eta$ et $R$.
}{%
	On identifie le terme devant $v$ comme étant $1/\tau\ind{theo}$ par analyse
	dimensionnelle, soit
	\[
		\boxed{\tau\ind{theo} = \frac{m}{6\pi\eta R}}
	\]
}

\QR{%
	En déduire la forme de la solution de l'équation différentielle en $v(t)$.
}{%
	\leftcenters{%
		Ainsi,
	}{%
		$\DS
			\boxed{
				\dv{v}{t} + \frac{v}{\tau\ind{theo}} = \frac{v\ind{lim}}{\tau\ind{theo}}
			}
		$
	}
}

\enonce{%
	\section{Réaliser}
	\subsection{Enregistrement vidéo}

	\begin{tcb}[breakable](expe)<itc>{Préréglages de la \textit{webcam} et de la prise de vue}
		\begin{enumerate}
			\item Ouvrir le logiciel \texttt{Amcap3} dans Bureau $\ra$ Programmes
			      Physique Chimie $\ra$ \texttt{Amcap3}.
			\item Dans le menu \texttt{Options}~:
			      \begin{itemize}
				      \item \texttt{Preview}, pour visualiser ce qu'on voit dans la
				            caméra. Régler la distance caméra-éprouvette pour voir
				            uniquement la moitié supérieure de l'éprouvette. Le trait noir
				            sur l'éprouvette aide à régler l'horizontal de la caméra.
				            Régler l'objectif de la caméra pour que l'image soit nette
				            (faire le point sur l'éprouvette).
				      \item \texttt{Video capture pin}~: taille de sortie choisir,
				            \texttt{640*360}. Fréquence image~: 20. Format~: MJPG.
				            $\ra$ Appliquer $\ra$ OK.
				      \item \texttt{Video capture filter}~: luminosité, netteté et
				            contraste en positions intermédiaires. $\ra$ Appliquer
				            $\ra$ OK.
			      \end{itemize}
			\item Puis dans le menu \texttt{Capture}~:
			      \begin{itemize}
				      \item \texttt{Set Frame Rate}~: nombre d'images par seconde~: 20.
				            $\ra$ Cocher~: \texttt{use} $\ra$ OK.
				      \item \texttt{Set time limit}~: \SI{5}{s} $\ra$ Cocher~:
				            \texttt{use} $\ra$ OK.
				      \item \texttt{Start capture}~: dossier élève~: choisir où vous
				            mettez vos dossiers.
			      \end{itemize}
		\end{enumerate}
	\end{tcb}

	\begin{tcb}(expe)<itc>{Enregistrement de la vidéo}
		Aller sur~: \texttt{Capture}~; \texttt{Start Capture}~: Cliquer sur \texttt{OK},
		puis lâcher la bille juste après.
	\end{tcb}

	\begin{tcb}[breakable](expe)<itc>{Traitement vidéo de la chute de la bille}
		\begin{enumerate}
			\item Ouvrir le logiciel \texttt{Latispro} dans programmes $\ra$
			      discipline $\ra$ physique-chimie $\ra$ Eurosmart.
			\item Cliquer sur la 5\ieme\ icône~: lecture de séquence AVI (ressemblant à
			      Google chrome).
			\item Fichiers $\ra$ ouvrir le fichier \texttt{avi}.
			\item Revenir à zéro pour exploiter (4\ieme\ icône en bas).
			\item Grâce à >, choisir le début de la vidéo à exploiter (première image
			      quand la bille commence à descendre).
			\item Puis cliquer sur sélection de l'origine et pointer la bille, grâce à
			      la loupe à droite.
			\item Sélection de l'étalon~: sélection de haut en bas sur la partie visible
			      de l'éprouvette.
			\item Indiquer la distance associée (ne pas mettre l'unité qui est le
			      \si{m}). $\ra$ correspond à la hauteur de l'éprouvette graduée.
			\item Sens des axes~: \fbox{$\downarrow \ra$}. Sélection manuelle des
			      points.
			\item Viser la cible et pointer grâce au zoom à droite~: pointer alors ainsi
			      une quarantaine de positions de la bille.
			      \begin{center}
				      \begin{tcb}*[width=.8\linewidth](prop)"bomb"{Attention}
					      Il faut \textbf{absolument} prendre plusieurs points pendant que
					      la balle ne bouge pas~; vous choisirez de couper les points
					      inutiles plus tard.
				      \end{tcb}
			      \end{center}
			\item Terminer la sélection manuelle, quand il y a assez de points.
			\item Fermer la fenêtre. Pour exploiter, cliquer sur le signal sinusoïdal
			      vert. Icône~: \fbox{$\sim$}.
		\end{enumerate}
	\end{tcb}

	\subsection{Modélisation des données}

	\begin{tcb}(expe)<itc>{Tracé de la vitesse verticale en fonction du temps}
		\begin{itemize}
			\item Traitements~: $\ra$ Calculs spécifiques. $\ra$ Dérivée. $\ra$ Faire
			      glisser $y$ pour obtenir $v = \dv{y}{t}$
			\item Pour visualiser $v = f(t)$, faire glisser la fonction $v$ sur le
			      graphe. On affichera uniquement les points sans les relier.
		\end{itemize}
	\end{tcb}

	\begin{tcb}[breakable](expe)<itc>{Modélisation de la vitesse}
		\begin{itemize}
			\item Cliquer modéliser.
			\item Choisir le modèle sous forme $A(1-\exr^{-t/\tau})$ (forme théorique
			      attendue pour la loi de vitesse d'après la partie s'approprier). On
			      forcera $V_0 = 0$ sur le modèle. Le calculer.
			\item Le glisser sur la courbe en superposition.
			\item Pour afficher la modélisation~: Copier dans le presse papier $\ra$
			      Fermer $\ra$ Créer un commentaire $\ra$ Coller après avoir choisi une
			      fenêtre.
		\end{itemize}
	\end{tcb}
}

\setcounter{section}{4}
\section{Valider et conclure}

\subsection{Détermination de la vitesse limite}

\setlist[blocQR,1]{leftmargin=10pt, label=\sqenumi}
\QR<[start=1]>{%
	Imprimer la courbe, puis déterminer la valeur expérimentale de la vitesse
	limite.
}{%
	On lit la valeur finale~: $v\ind{lim} \approx \SI{26}{cm.s^{-1}}$
}

\subsection{Détermination de la constante de temps $\tau$ par deux méthodes}

\subsubsection{Utilisation du temps de montée}

\setlist[blocQR,1]{leftmargin=10pt, label=\clenumi}
\QR<[start=7]>{%
	Quelle est la valeur de la vitesse (en pourcentage de la vitesse
	limite $v\ind{lim}$) lorsque $t=\tau$~? En déduire une méthode de
	détermination de $\tau$ (qui est celle du cours).
}{%
	On a $v(\tau) = \num{0.63}v\ind{lim}$. On peut donc trouver $\tau$ en lisant
	l'abscisse pour laquelle $v = \num{0.63}v\ind{lim}$.
}

\setlist[blocQR,1]{leftmargin=10pt, label=\sqenumi}
\QR<[start=2]>{%
	Relever $\tau\ind{exp1}$.
}{%
	\leftcenters{%
		On trouve
	}{%
		\xul{$\tau\ind{exp1} = \SI{162\pm 10}{ms}$}
	}
}

\subsubsection{Utilisation de la modélisation de la vitesse}

\QR{%
	Déduire de la modélisation de la vitesse l'ordre de grandeur de la
	constante de temps $\tau\ind{exp2}$, disponible dans la fenêtre de
	modélisation.
}{%
	\leftcenters{%
		On trouve
	}{%
		$\xul{\tau\ind{exp2} = \SI{160}{ms}}$
	}
}

\QR{%
	Calculer l'écart normalisé entre les deux valeurs expérimentales de
	$\tau$. Conclure et discuter des limites de la mesure.
}{%
	\makebox[\linewidth]{$\DS
			\begin{gathered}
				\boxed{
					E_N = \frac{\abs{\tau\ind{exp1} - \tau\ind{exp2}}}{u(\tau\ind{exp1})}
				}
				\Lra
				\xul{E_N = \num{1}}
			\end{gathered}
		$}
	\smallbreak
	Les deux mesures sont compatibles, malgré l'absence d'incertitude sur la
	valeur de la modélisation Latispro.
}

\subsection{Détermination de la viscosité $\eta$ de l'huile de silicone}

\QR{%
	Grâce à l'étude théorique de la partie S'approprier et à la valeur
	moyenne de $\tau\ind{exp}$, déterminer la valeur expérimentale de la
	viscosité $\eta\ind{exp}$ en précisant son unité et son incertitude (revoir la
	fiche incertitudes).
}{%
	\makebox[\linewidth]{$\DS
			\begin{gathered}
				\tau\ind{exp} = \frac{\tau\ind{exp1} + \tau\ind{exp2}}{2} = \SI{73.5}{ms}
				\\
				\eta\ind{exp} = \frac{m}{6\pi\tau\ind{exp}R}
				\qav
				\left\{
				\begin{array}{rcl}
					m             & = & \SI{10.4e-3}{kg}
					\\
					\tau\ind{exp} & = & \SI{161e-3}{s}
					\\
					R             & = & \SI{1.0e-2}{m}
				\end{array}
				\right.\\
				\AN
				\xul{
					\eta\ind{exp} = \SI{0.343}{Pa.s}
				}
			\end{gathered}
		$}
	De plus,
	\begin{gather*}
		\frac{u(\eta)}{\eta} = \frac{u(\tau)}{\tau}
		\Lra
		\boxed{u(\eta) = \frac{\eta}{\tau} u(\tau)}
		\Ra
		\xul{u(\eta) = \SI{0.021}{Pa.s}}
		\\\Ra
		\boxed{\eta\ind{exp} = \SI{0.343\pm 0.021}{Pa.s}}
	\end{gather*}
}

\QR{%
	La valeur théorique de la viscosité et $\eta\ind{theo} =
		\SIrange{0.32}{0.35}{SI}$. L'écrire sous forme de valeur centrale $\pm$ son
	incertitude. Comparer alors votre résultat expérimental avec la valeur
	théorique à l'aide d'un écart normalisé.
}{%
	On a $\eta\ind{theo} = \SI{0.3366\pm0.0084}{Pa.s}$.
	Ainsi,
	\[
		\boxed{
			E_N = \frac{\abs{\eta\ind{theo} - \eta\ind{exp}}}
			{\sqrt{u(\eta\ind{exp})^{2} + u(\eta\ind{theo})^{2}}}
		}
		\Lra
		\xul{E_N = \num{0.27}}
	\]
	Les résultats sont bien cohérents.
}

\subsection{Détermination plus rapide de la vitesse limite sans enregistrement vidéo}

\QR{%
	Justifier que le régime transitoire ait une durée négligeable devant
	la durée totale de chute de la bille.
}{%
	On trouve en effet, que ce soit sur le chronogramme ou par étude de la valeur
	de $\tau$, que la vitesse limite est atteinte très rapidement~: au maximum
	après $5\tau \approx \SI{0.4}{s}$ pour un temps de chute de plusieurs
	secondes.
}
\QR{%
	Proposer alors puis réaliser un protocole expérimental (sans
	enregistrement) qui permettrait de déterminer la vitesse limite en
	répétant les mesures 5 à 6 fois.
}{%
	On mesure au chronomètre le temps de trajet entre deux points repérés par
	l'échelle derrière le tube.
}

\QR{%
	Partagez vos résultats de mesure avec le reste de la classe.
}{%
	Non corrigé.
}

\QR{%
	En déduire une nouvelle valeur de $v\ind{lim}$, en tenant compte de vos
	différents mesurages et de ceux des autres groupes de la classe. Vous
	présenterez le résultat avec l'incertitude de type A.
}{%
	Non corrigé.
}

\QR{%
	Comparer à la valeur trouvée avec la modélisation.
}{%
	Non corrigé.
}

\end{document}
