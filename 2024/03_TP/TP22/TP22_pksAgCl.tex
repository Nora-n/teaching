\documentclass[../main/main.tex]{subfiles}
\graphicspath{{./figures/}}

\makeatletter
\renewcommand{\@chapapp}{Travaux pratiques -- TP}
\makeatother

% \toggletrue{student}
% \toggletrue{corrige}
% \renewcommand{\mycol}{black}
\renewcommand{\mycol}{gray}

\begin{document}
\setcounter{chapter}{21}

\settype{enon}
\settype{solu_prof}
\settype{solu_stud}

\chapter{\cswitch{%
	  Correction du TP
  }{%
	  Détermination de $\pk[s](\ce{AgCl})$ par colorimétrie \& potentiométrie
  }
 }

\enonce{%
\begin{prgm}
	\begin{tcb}*(ror)"how"{Savoir-faire}
		\begin{itemize}
			\item Identifier et exploiter la réaction support du titrage
			      (recenser les espèces présentes dans le milieu au cours du
			      titrage, repérer l'équivalence, justifier qualitativement
			      l'allure de la courbe ou le changement de couleur observé).
			\item Exploiter une courbe de titrage pour déterminer la
			      concentration d'une espèce dosée.
			\item Choisir et utiliser un indicateur coloré de fin de
			      titrage~; distinguer l'équivalence et le virage d'un indicateur
			      coloré de fin de titrage.
			\item Dosage par précipitation (technique de \textsc{Mohr})
		\end{itemize}
	\end{tcb}
\end{prgm}
\vspace{-10pt}

\section{Objectifs}

\begin{itemize}
	\item Réaliser des dosages colorimétriques (méthode de \textsc{Mohr}) et
	      potentiométrique.
	\item Modéliser les courbes obtenues.
	\item Exploiter les courbes pour déterminer le $\pk[s]$ du chlorure
	      d'argent.
\end{itemize}

\section{S'approprier}
\subsection{Introduction}

Le chlorure d'argent $\ce{AgCl_{\rm(s)}}$ est une solide blanc photosensible
présent naturellement dans certains affleurements de filons d'argent. Il est,
entre autres, utilisé dans les cathodes des batteries $\ce{AgCl-Mg}$ servant
de dispositif de propulsion des torpilles sous-marines fonctionnant à l'eau de
mer.
\smallbreak
On veut doser une solution de chlorure de sodium $(\ce{Na^+};\ce{Cl^-})$ de
concentration $c_0$ par une solution de nitrate d'argent
$(\ce{Ag^+},\ce{NO_3^-})$ de concentration $c_1$, puis déterminer le produit de
solubilité $K_s$ du chlorure d'argent afin de le comparer à la valeur fournie
par la littérature. En raison de la faible valeur du produit de solubilité du
chlorure d'argent, la mise en présence des ions argent $\ce{Ag^+}$ et chlorure
$\ce{Cl^-}$ provoque la précipitation du chlorure d'argent, selon l'équation
bilan
\begin{gather}
	\ce{{Ag^+}_{\rm(aq)} + {Cl^-}_{\rm(aq)} = AgCl_{\rm(s)}}
	\label{eq:prepAgCl}
\end{gather}
% avec $K = K_s^{-1}$ et $K_s = \num{1.8e-10}$ à \SI{298}{K}.

\subsection{Dosage colorimétrique des ions \ce{Cl^-} par la méthode de
	\textsc{Mohr}}
La méthode de \textsc{Mohr} est une méthode de dosage colorimétrique des ions
chlorure $\ce{Cl^-}$ par les ions argent $\ce{Ag^+}$ en présence d'ions
chromate $\ce{CrO_4^{2-}}$. En présence des ions $\ce{Ag^+}$, les ions
chlorure peuvent former un précipité blanc (qui noircit à la lumière), tandis
que les ions chromate peuvent former avec les ions argent un précipité rouge
brique. En solution, les ions chromate sont jaunes.
\smallbreak
\begin{tcb}(data)<lfnt>{Données}
	À \SI{298}{K}~:
	$\pk[s](\ce{AgCl}) = \num{9.8}$ et
	$\pk[s](\ce{Ag_2CrO_4}) = \num{11.9}$
\end{tcb}

\subsection{Dosage potentiométrique~: détermination de $\pk[s](\ce{AgCl})$}
\subsubsection{Principe du dosage}
Pour suivre ce dosage, on construit une pile d'oxydoréduction~: dans $V_0 =
	\SI{10}{mL}$ de la solution de chlorure de sodium, on introduit une électrode
de mesure constituée d'un fil d'argent de potentiel $E$. L'autre pôle est une
électrode de référence au sulfate mercureux. Le potentiel de cette électrode
par rapport à l'électrode standard à hydrogène est égal à $E\ind{ref} =
	\SI{651}{mV}$ à \SI{25}{\degreeCelsius}. Ainsi, en notant $U$ la tension lue
au millivoltmètre haute impédance entre l'électrode de mesure et l'électrode
de référence, et $E$ le potentiel de la solution lue par l'électrode de
mesure, on a
\[
	U = E - E\ind{ref}
	\Lra
	\boxed{E = U + E\ind{ref}}
\]
\begin{tcb}(impo){Volume d'eau ajouté}
	Pour que les électrodes trempent correctement, on ajoute $V\ind{eau} =
		\SI{10}{mL}$ d'eau distillée, mesurée à l'aide d'une pipette ou fiole
	jaugée.
\end{tcb}
Ce potentiel d'électrode est par ailleurs fourni par la formule de
\textsc{Nernst}~:
\[
	E = E(\ce{Ag^+/Ag}) = E^\circ (\ce{Ag^+/Ag}) + \num{0.06} \log ([\ce{Ag^+}])
\]
Ainsi, \textbf{la mesure de $U$ permet de suivre l'avancement du dosage}. On
notera $V$ le volume de solution de nitrate d'argent versé et $V\ind{eqv}$ le
volume versé à l'équivalence.

\subsubsection{Relations quantitatives}
La réaction~\eqref{eq:prepAgCl} est pratiquement totale dans le sens direct. À
l'équivalence, on a versé en solution autant d'ions argent que d'ions chlorure,
soit $c_1V\ind{eq} = c_0V_0$ \textbf{compte-tenu de la stœchiométrie}. Avant,
\ce{Ag^+} est limitant, après, \ce{Cl^-} est limitant. Le tableau d'avancement
en \textbf{quantité de matière} est donc~:
\begin{center}
	\def\rhgt{0.35}
	\centering
	\begin{tabularx}{\linewidth}{|l|c||YdYdY|}
		\hline
		\multicolumn{2}{|c||}{
			$\xmathstrut{\rhgt}$
		\textbf{Équation}}                     &
		$\ce{{Ag}^+_{\rm(aq)}}$                & $+$                         &
		$\ce{{Cl-}_{\rm(aq)}}$                 & $\ra$                       &
		$\ce{AgCl_{\rm(s)}}$                                                   \\
		\hline
		$\xmathstrut{\rhgt}$
		Initial                                & $\xi = 0$                   &
		$0$                                    & \vline                      &
		$c_0V_0$                               & \vline                      &
		$0$                                                                    \\
		\hline
		$\xmathstrut{\rhgt}$
		Avant l'équi.                          & $\xi = \xi\ind{max} = c_1V$ &
		$c_1V - c_1V \approx \ep_1$            & \vline                      &
		$c_0V_0 - c_1V$                        & \vline                      &
		$c_1V$                                                                 \\
		\hline
		$\xmathstrut{\rhgt}$
		À l'équi.                              & $\xi_f = \xi_{\equ}$        &
		$0$                                    & \vline                      &
		$c_0V_0 - c_1V\ind{eqv} = 0$           & \vline                      &
		$c_1V\ind{eqv}$                                                        \\
		\hline
		$\xmathstrut{\rhgt}$
		Après l'équi.                          & $\xi_f = \xi_{\equ}$        &
		$c_1V - c_1V\ind{eqv}$                 & \vline                      &
		$c_0V_0 - c_1V\ind{eqv} \approx \ep_2$ & \vline                      &
		$c_1V\ind{eqv}$                                                        \\
		\hline
	\end{tabularx}
\end{center}
On appelle $\ep_1$ et $\ep_2$ les quantité de matière infinitésimales restantes
en ions \ce{Ag^+} avant l'équivalence, et en ions \ce{Cl^-} après l'équivalence,
respectivement.

\subsubsection{Expression de $U$ au cours du dosage}
\begin{itemize}
	\bitem{À l'équivalence}, $V = V\ind{eqv}$ et les ions chlorure et argent ont
	été introduits en \textbf{proportions stœchiométriques}, et on est à
	l'équilibre soit
	\begin{gather*}
		[\ce{Ag+}]\ind{eq} = [\ce{Cl-}]\ind{eq} = c^\circ \sqrt{K_s}
		\\\beforetext{D'où}
		U\ind{eqv} = E - E\ind{ref} =
		E^\circ (\ce{Ag+/Ag}) + \num{0.03}\log K_s
		\Lra
		\boxed{
			U\ind{eqv} =
			E^\circ (\ce{Ag+/Ag}) - \num{0.03}\pk[s] - E\ind{ref}
		}
	\end{gather*}
	\bitem{Avant l'équivalence}, $V < V\ind{eqv}$ et d'après le tableau
	d'avancement avec la relation d'équivalence $c_0V_0 = c_1V\ind{eqv}$~:
	\vspace{-15pt}
	\begin{gather*}
		[\ce{Cl^-}]\ind{eq} = \frac{c_0V_0 - c_1V}{V_0+V+V\ind{eau}}
		= \frac{c_1 (V\ind{eq} - V)}{V_0+V+V\ind{eau}}
		\Lra
		[\ce{Ag^+}]\ind{eq} = K_s \frac{V_0+V+V\ind{eau}}{c_1 (V\ind{eqv}-V)}
		\\\Lra
		\boxed{
			U\ind{avant} =
			E^\circ (\ce{Ag^+/Ag}) - \num{0.06} \pk[s] +
			\num{0.06}\log (\frac{V_0+V+V\ind{eau}}{c_1 (V\ind{eq} - V)}) - E\ind{ref}
		}
	\end{gather*}
	\bitem{Après l'équivalence}, $V > V\ind{eqv}$ et d'après le tableau
	d'avancement avec la relation d'équivalence $c_0V_0 = c_1V\ind{eqv}$~:
	\vspace{-15pt}
	\begin{gather*}
		[\ce{Ag^+}]\ind{eq} = \frac{c_1(V - V\ind{eqv})}{V_0+V+V\ind{eau}}
		\\\Lra
		\boxed{
			U\ind{après} =
			E^\circ (\ce{Ag^+/Ag}) +
			\num{0.06}\log (\frac{c_1 (V - V\ind{eq})}{V_0+V+V\ind{eau}}) - E\ind{ref}
		}
	\end{gather*}
\end{itemize}

L'évolution rapide de la courbe de $U$ à l'équivalence est exploitée pour
déterminer le point équivalent.
}

\setcounter{section}{2}
\section{Analyser}
\subsection{Condition de précipitation}
\enonce{%
	On dispose d'une solution de chlorure de sodium de volume $V_0 =
		\SI{10.0}{mL}$ et de concentration $c_0 \approx \SI{7.5e-2}{mol.L^{-1}}$, et
	d'une solution de nitrate d'argent de concentration $c_1 \approx
		\SI{5.0e-2}{mol.L^{-1}}$~; ce sont des ordres de grandeur des concentrations.
}%

\setlist[blocQR,1]{leftmargin=10pt, label=\clenumi}
\QR{%
	Montrer que le précipité apparaît dès la première goutte de nitrate d'argent
	versée. $K_s$ étant très faible, on pourra négliger le volume de la goutte en
	ions argent $V_L$ à verser devant $V_0$.
}{%
	solu
}%
\QR{%
	Déterminer l'ordre de grandeur du volume $V\ind{eqv}$ de la solution de
	nitrate d'argent nécessaire pour que tous les ions aient précipité.
}{%
	solu
}%
\enonce{%
	Dans toute la suite on supposera qu'il y a équilibre hétérogène (coexistence
	solide-liquide).
}%

\subsection{Étude théorique de la méthode de \textsc{Mohr}}
\QR{%
	En exploitant les deux réactions de précipitation, prévoir lequel des deux
	précipités est formé en premier lorsque l'on ajoute une solution d'ions $Ag^+$ à
	une solution obtenue en mélangeant $V_0 = \SI{10.0}{mL}$ d'une solution d'ions
	\ce{Cl^-} de concentration voisine de $c_0 \approx \SI{7.5e-2}{mol.L^{-1}}$ et
	$V_2 = \SI{1.0}{mL}$ d'une solution de chromate de potassium de concentration
	$c_2 = \SI{0.20}{mol.L^{-1}}$. Pour cela, on tracera les domaines d'existence
	des deux précipités en fonction de $\prm Ag$.
}{%
	solu
}%
\QR{%
	En déduire quelle sera la couleur de la solution initiale.
}{%
	solu
}%
\QR{%
	Comment repérer l'équivalence~?
}{%
	solu
}%

\section{Réaliser}
% \subsection{Dosage colorimétrique}

\resetQ
\setlist[blocQR,1]{leftmargin=10pt, label=\sqenumi}
\enonce{%
	\begin{tcb}*[breakable](expe)<itc>"chem"{Dosage colorimétrique}
		\begin{enumerate}
			\item Réaliser le dosage colorimétrique correspondant à la méthode de
			      \textsc{Mohr} avec les quantités précisées. On placera sous le bécher un
			      morceau de \textbf{papier blanc} pour bien repérer le changement de couleur.

			\item Penser à faire un \textbf{témoin colorimétrique} en préparant dans un
			      second bécher la même solution initiale (mélange d'ions chlorure et
			      d'ions chromate) qui, elle, ne sera \textbf{pas dosée}.
		\end{enumerate}
	\end{tcb}
}%


\QR{%
	Noter la couleur du premier précipité formé, puis du second.
}{%
	solu
}%
\QR{%
	Déterminer le volume équivalent $V\ind{eqv}$ et en déduire la concentration
	$c_0$ de la solution de chlorure de sodium.
}{%
	solu
}%

% \subsection{Dosage potentiométrique}
\enonce{%
	\begin{tcb}*(expe)<itc>"chem"{Dosage potentiométrique}
		\begin{enumerate}
			\item Réaliser la pile correspondante et préparer le dosage.
			      \begin{center}
				      \begin{tcb}*[width=.8\linewidth](prop)"bomb"{Attention}
					      Il ne faut surtout pas ajouter de chromate de potassium~!
				      \end{tcb}
			      \end{center}
			      N'oublier pas d'ajouter le volume $V\ind{eau} = \SI{10}{mL}$ d'eau
			      distillée pour que les électrodes soient correctement plongées.
			\item Créer un tableau de valeurs sur \textbf{Régressi} (important pour la
			      modélisation ensuite) avec deux colonnes~: $U$ et $V$. Resserrer les
			      valeurs autour de l'équivalence. La mesure de $U$ n'est valable que si
			      les deux électrodes trempent dans la solution.
		\end{enumerate}
	\end{tcb}
}%


\section{Analyser}
\QR{%
	Réaliser la méthode des tangentes automatique sur Régressi pour déterminer le
	volume équivalent.
}{%
	solu
}%
\QR{%
	Modéliser la courbe en deux parties grâce aux curseurs~:
	\begin{gather*}
		\beforetext{Avant}
		U = a + b \log (\frac{V_0+V+V\ind{eau}}{c_1(V\ind{eqv}-V)})
		\\\beforetext{Après}
		U = c + d \log (\frac{c_1(V-V\ind{eqv})}{V_0+V+V\ind{eau}})
	\end{gather*}
	Les coefficients $a$, $b$, $c$ et $d$ sont déterminés par l'ordinateur, vous
	devez remplacer les autres variables par leurs valeurs manuellement.
}{%
	solu
}%
\QR{%
	Grâce aux expressions de $U$ données dans la partie S'approprier, vérifier
	qu'on obtient $\pk[s]$ grâce à la relation~:
	\[
		\pk[s] = \frac{c-a}{b} = \frac{c-a}{d}
	\]
}{%
	solu
}%

\section{Conclure}
\QR{%
	Donner alors la valeur de $\pk[s]$ pour chacun de deux calculs précédent,
	ainsi que la valeur moyenne. Estimez l'incertitude en faisant varier les
	valeurs de $V\ind{eqv}$ dans les modélisations, et calculer l'écart normalisé.

}{%
	solu
}%

\end{document}
