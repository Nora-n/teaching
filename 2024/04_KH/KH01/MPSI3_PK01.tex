\documentclass[a4paper, 12pt, final, garamond]{book}
\usepackage{cours-preambule}

\raggedbottom

\makeatletter
\renewcommand{\@chapapp}{Programme de kh\^olle -- semaine}
\makeatother

\begin{document}
\setcounter{chapter}{0}

\chapter{Du 16 au 19 septembre}

\section{Cours et exercices}

\subsection(O2){Base de l'optique géométrique}
\begin{enumerate}[label=\Roman*]
	\item[b]{Propriétés générales}: optique non géométrique~: diffraction,
	approximation de l'optique géométrique~: notion de rayon lumineux, propriétés
	d'un rayon lumineux, limites.
	\item[b]{Lois de \textsc{Snell-Descartes}}: changement de milieu, lois
	de \textsc{Snell-Descartes} pour la réflexion et la réfraction, phénomène
	de réflexion totale.
	\item[b]{Généralités sur les systèmes optiques}: système, rayons,
	faisceaux~; objets et images réelles ou virtuelles, conjugaison et
	schématisation $\rm A \opto{\rm S}{}A'$, objet étendu et grandissement
	transversal, foyers principaux et secondaires d'un S.O.\ et propriétés
	associées.
	\item[b]{Approximation de \textsc{Gauss}}: définition stigmatisme,
	aplanétisme, rigoureux ou approché, rayons paraxiaux, conditions et
	approximation de \textsc{Gauss}.
\end{enumerate}

\subsection(O3){Miroir plan et lentilles minces}
\begin{enumerate}[label=\Roman*]
	\item[b]{Miroir plan}: définition, stigmatisme et aplanétisme
	rigoureux, construction pour objet réel et virtuel, relation de
	conjugaison (démonstration), grandissement transversal (démonstration).
	\item[b]{Lentilles minces}: définition lentille, minces, convergentes
	et divergentes, stigmatisme et aplanétisme, centre optique et propriété,
	distance focale image, vergence, construction rayons parallèles à l'axe
	optique pour divergente et convergente, règles primaires des
	constructions géométriques, cas simples pour lentille convergente et
	divergente, cas divers, \textbf{relation de conjugaison} et
	grandissement transversal
	\item[b]{Quelques applications}: condition de netteté, champ de vision dans
	un miroir.
\end{enumerate}

\vspace*{\fill}

{\Large Pas d'association de dispositifs optiques en exercice cette semaine.}

\vspace*{\fill}

\newpage

\section{Questions de cours possibles}
\begin{enumerate}[label=\sqenumi]
	\subsection(O2){Base de l'optique géométrique}
	\item %
	      Énoncer les lois de \textsc{Snell-Descartes} pour la réflexion et la
	      réfraction \textit{avec un schéma} (P.O2.4), énoncer les conditions de
	      réflexion totale \textit{avec un schéma}, donner et démontrer
	      l'expression de l'angle limite $i\ind{lim}$ en fonction de $n_2$ et
	      $n_1$ (P.O2.5, Dm.O2.1)~;
	\item %
	      Définir la notion de stigmatisme et d'aplanétisme (Df.O2.15 et 16), de
	      rayons paraxiaux (Df.O2.17) et l'approximation de \textsc{Gauss}
	      (P.O2.7).
	      Schéma demandé pour le stigmatisme, mais non demandé pour l'aplanétisme.

	      \item[s]"3"%
	      Présenter la fibre optique à saut d'indice avec un schéma (TDO3.ent.I).
	      Démontrer l'expression de l'angle du cône d'acceptance en fonction des
	      indices optiques de la fibre, puis déterminer l'expression de la
	      dispersion intermodale.

	      \subsection(O3){Miroir plan et lentilles minces}
	\item Construire l'image d'un objet (point ou étendu, réel ou virtuel) par un
	      miroir plan, donner et démontrer la relation de conjugaison d'un miroir
	      plan (P.O3.1, 2 et 3)~;

	\item Définir le grandissement transversal (Df.O2.13), donner et démontrer
	      (schématiquement au moins) sa valeur pour un miroir plan (P.O3.4, Dm.O3.1),
	      donner ses expressions pour une lentille.

	\item Plusieurs tracés peuvent être demandés parmi~:
	      \begin{enumerate}
		      \item Construire l'image d'un objet étendu réel ou virtuel par une
		            lentille quelconque en présentant les règles primaires et en
		            précisant la nature de l'objet et de l'image (I.O3.1, A.O3.2 et
		            3)~;
		      \item Construire le rayon émergent d'un rayon quelconque en présentant
		            les règles de construction secondaires et nommant tous les points
		            d'intérêt (I.O3.2, A.O3.4)
	      \end{enumerate}

	\item Savoir établir et connaître la relation de conjugaison de
	      \textsc{Descartes} et le grandissement (P.O3.6 et 7, Dm.O3.2)~;

	\item Savoir établir et connaître la relation de conjugaison de
	      \textsc{Newton} et le grandissement (P.O3.6 et 7, Dm.O3.2)~;

	\item Savoir refaire la démonstration de la condition de netteté (O3.III/A)
	      pour l'image réelle d'un objet réel d'une lentille convergente ($D \geq
		      4f'$)~; les conditions du système seront redonnées~;

	\item \leavevmode%
	      (O3.III/B) Une personne dont les yeux se situent à $h = \SI{1.70}{m}$ du
	      sol observe une mare gelée (équivalente à un miroir plan) de largeur $l
		      = \SI{5.00}{m}$ et située à $d = \SI{2.00}{m}$ d'elle.
	      \begin{enumerate}
		      \item Peut-elle voir sa propre image~? Quelle est la nature de
		            l'image~?
		      \item Quelle est la hauteur maximale $H$ d'un arbre situé de l'autre
		            côté de la mare (en bordure de mare) qu'elle peut voir par
		            réflexion dans la mare~? On notera $D = l+d$.
	      \end{enumerate}
\end{enumerate}

% \section{Consignes}
% \begin{enumerate}
% 	\item Une question de cours non connue entraîne une note sous la moyenne~;
% 	\item Les schémas des questions de cours sont obligatoires~: s'ils manquent,
% 	      la question ne saurait être notée au-dessus de 5~;
% 	\item Chacune des règles suivantes doit être respectée~:
% 	      \begin{enumerate}
% 		      \item Les schémas optiques doivent comporter le sens de comptage
% 		            algébrique des distances et des angles~;
% 		      \item Les rayons lumineux doivent avoir un sens de propagation~;
% 		      \item Les angles doivent être orientés.
% 	      \end{enumerate}
% \end{enumerate}

\end{document}
