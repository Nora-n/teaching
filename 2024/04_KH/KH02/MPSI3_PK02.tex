\documentclass[a4paper, 12pt, final, garamond]{book}
\usepackage{cours-preambule}

\raggedbottom

\makeatletter
\renewcommand{\@chapapp}{Programme de kh\^olle -- semaine}
\makeatother

\begin{document}
\setcounter{chapter}{1}

\chapter{Du 23 au 26 septembre}

\section{Exercices uniquement}
\subsection(O2){Base de l'optique géométrique}

\section{Cours et exercices}

\subsection(O3){Miroir plan et lentilles minces}
\begin{enumerate}[label=\Roman*]
	\item[b]{Miroir plan}: définition, stigmatisme et aplanétisme
	rigoureux, construction pour objet réel et virtuel, relation de
	conjugaison (démonstration), grandissement transversal (démonstration).
	\item[b]{Lentilles minces}: définition lentille, minces, convergentes
	et divergentes, stigmatisme et aplanétisme, centre optique et propriété,
	distance focale image, vergence, construction rayons parallèles à l'axe
	optique pour divergente et convergente, règles primaires des
	constructions géométriques, cas simples pour lentille convergente et
	divergente, cas divers, \textbf{relation de conjugaison} et
	grandissement transversal
	\item[b]{Quelques applications}: condition de netteté, champ de vision dans
	un miroir.
\end{enumerate}

\subsection(O4){Dispositifs optiques}
\begin{enumerate}[label=\Roman*]
	\item[b]{L'œil}~: présentation et modélisation, accommodation et
	focales minimales et maximales, réglage d'un instrument optique, résolution
	angulaire et vocabulaire sur les défauts.
	\item[b]{La loupe}~: présentation de l'effet loupe, définition
	grossissement général et propriété $G = d_m/f'$ pour la loupe avec
	démonstration.
	\item[b]{Appareil photo}~: description, modélisation simple, champ et
	influence de la focale et de la taille du capteur, distance de mise au point,
	profondeur de champ et influence de la distance de mise au point, de la focale
	et de l'ouverture.
	\item[b]{Systèmes optiques à plusieurs lentilles}~: association
	quelconque, notion de microscope, définition lunettes astronomiques
	\textsc{Kepler} et \textsc{Galilée}, définition système afocal, calcul
	d'encombrement, grossissement $G=-f'_1/f'_2$ et démonstration, cercle
	oculaire.
\end{enumerate}

\section{Questions de cours possibles}
\begin{enumerate}
	\subsection(O3){Miroir plan et lentilles minces}
	\item %
	      Énoncer les lois de \textsc{Snell-Descartes} pour la réflexion et la
	      réfraction \textit{avec un schéma} (P.O2.4), énoncer les conditions de
	      réflexion totale \textit{avec un schéma}, donner et démontrer
	      l'expression de l'angle limite $i\ind{lim}$ en fonction de $n_2$ et
	      $n_1$ (P.O2.5, Dm.O2.1).
	      % \item %
	      %       Définir la notion de stigmatisme et d'aplanétisme (Df.O2.15 et 16), de
	      %       rayons paraxiaux (Df.O2.17) et l'approximation de \textsc{Gauss}
	      %       (P.O2.7).
	      %       Schéma demandé pour le stigmatisme, mais non demandé pour l'aplanétisme.

	      \item[s]"3"%
	      Présenter la fibre optique à saut d'indice avec un schéma (TDO3.ent.I).
	      Démontrer l'expression de l'angle du cône d'acceptance en fonction des
	      indices optiques de la fibre, puis déterminer l'expression de la
	      dispersion intermodale.

	\item Construire l'image d'un objet (point ou étendu, réel ou virtuel) par un
	      miroir plan, donner et démontrer la relation de conjugaison d'un miroir
	      plan (P.O3.1, 2 et 3).

	      % \item Définir le grandissement transversal (Df.O2.13), donner et démontrer
	      %       (schématiquement au moins) sa valeur pour un miroir plan (P.O3.4, Dm.O3.1),
	      %       donner ses expressions pour une lentille.

	\item Plusieurs tracés \textbf{doivent} être demandés parmi~:
	      \begin{enumerate}
		      \item Construire l'image d'un objet étendu réel ou virtuel par une
		            lentille quelconque en présentant les règles primaires et en
		            précisant la nature de l'objet et de l'image (I.O3.1, A.O3.2 et
		            3)~;
		      \item Construire le rayon émergent d'un rayon quelconque en présentant
		            les règles de construction secondaires et nommant tous les points
		            d'intérêt (I.O3.2, A.O3.4).
	      \end{enumerate}

	\item Donner les relations de conjugaison de \textsc{Descartes} et de
	      \textsc{Newton} ainsi que les grandissements associés. Démontrer-les
	      toutes (P.O3.6 et 7, Dm.O3.2).

	      % \item Savoir établir et connaître la relation de conjugaison de
	      %       \textsc{Descartes} et le grandissement (P.O3.6 et 7, Dm.O3.2)~;
	      %
	      % \item Savoir établir et connaître la relation de conjugaison de
	      %       \textsc{Newton} et le grandissement (P.O3.6 et 7, Dm.O3.2)~;

	\item Savoir refaire la démonstration de la condition de netteté (O3|III/A)
	      pour l'image réelle d'un objet réel d'une lentille convergente ($D \geq
		      4f'$)~; les conditions du système seront redonnées.

	\item \leavevmode%
	      (O3|III/B) Une personne dont les yeux se situent à $h = \SI{1.70}{m}$ du
	      sol observe une mare gelée (équivalente à un miroir plan) de largeur $l
		      = \SI{5.00}{m}$ et située à $d = \SI{2.00}{m}$ d'elle.
	      \begin{enumerate}
		      \item Peut-elle voir sa propre image~? Quelle est la nature de
		            l'image~?
		      \item Quelle est la hauteur maximale $H$ d'un arbre situé de l'autre
		            côté de la mare (en bordure de mare) qu'elle peut voir par réflexion
		            dans la mare~? On notera $D = l+d$.
	      \end{enumerate}
	      \subsection(O4){Dispositifs optiques}
	\item Décrire les caractéristiques d'un œil et donner son modèle en optique
	      géométrique (Df.O4.1). Définir la plage d'accommodation et les valeurs pour
	      un œil emmétrope (Df.O4.2), le pouvoir de résolution \textbf{avec un
		      schéma} et un ordre de grandeur (Df.O4.3, Odgr.O4.1). Décrire les
	      principaux défauts et la manière de les corriger (Df.O4.4, Ap.O4.2).
	      % Refaire l'exercice~:
	      % \begin{tcb}(appl){Exercice~:}
	      % Quelles sont les valeurs maximale et minimale de la focale du
	      % cristallin pour un œil emmétrope~? On rappelle que la distance
	      % cristallin-rétine est $d \approx \SI{22.3}{mm}$.
	      % \end{tcb}
	\item Décrire l'effet loupe dans les deux cas d'accommodation ou non
	      (Df.O4.5). Montrer qu'on ne peut pas modifier la taille
	      \textbf{perçue} d'une image vue au travers d'une loupe (Ap.O4.3),
	      définir le grossissement et démontrer sa formule pour une loupe
	      (Df.O4.6, P.O4.1, Dm.O4.1).
	\item Décrire un modèle simple de l'appareil photographique (Df.O4.7 et 8).
	      Quelle la différence avec un œil (At.O4.1)~? Définir le champ, la mise au
	      point et la profondeur de champ d'un appareil photo (O4|III/B, C
	      et D). \textbf{Donner et démontrer} la manière dont un paramètre de
	      l'appareil (focale, position capteur, taille du capteur et diaphragme)
	      modifie une caractéristique photographique (profondeur de champ, champ,
	      mise au point), au choix de l'interrogataire (Ip.O4.4, et tout le
	      III/).
	\item Tracer l'image d'une association quelconque de 2 lentilles donnée par
	      l'examinataire (Ap.O4.4 et 5). Qu'est-ce qu'un microscope~? Le représenter
	      par une représentation optique ($\rm A \opto{\Lc}{\rm O} A'$) et
	      indiquer quels points du système doivent être confondus (Df.O4.11).
	\item Définir ce qu'est une lunette astronomique et les deux types classiques
	      de lunette, avec schéma et
	      représentation optique ($\rm A \opto{\Lc}{\rm O} A'$) pour la lunette de
	      \textsc{Kepler} (Df.O4.12). Définir un système afocal (Df.O4.13).
	      Exprimer leur encombrement en fonction de $V_1$ et $V_2$ les vergences
	      des lentilles (Ap.O4.6). Établir la formule du grossissement (P.O4.2,
	      Dm.O4.2).
	\item \textbf{Faire un schéma} puis démontrer le théorème des vergences pour
	      les lentilles accolées. Démontrer ensuite la relation du grandissement d'une
	      association quelconque de lentilles en fonction du grandissement de chacune des
	      lentilles (TDO4.app.I).
\end{enumerate}

\end{document}

