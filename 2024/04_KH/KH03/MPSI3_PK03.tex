\documentclass[a4paper, 12pt, final, garamond]{book}
\usepackage{cours-preambule}

\raggedbottom

\makeatletter
\renewcommand{\@chapapp}{Programme de kh\^olle -- semaine}
\makeatother

\begin{document}
\setcounter{chapter}{2}

\chapter{Du 02 au 05 octobre}

\section{Exercices uniquement}

\section*{Optique chapitre 3 -- Miroir plan et lentilles minces}
\begin{enumerate}[label=\Roman*]
	\bitem{Miroir plan}~: définition, stigmatisme et aplanétisme
	rigoureux, construction pour objet réel et virtuel, relation de
	conjugaison (démonstration), grandissement transversal (démonstration).
	\bitem{Lentilles minces}~: définition lentille, minces, convergentes
	et divergentes, stigmatisme et aplanétisme, centre optique et propriété,
	distance focale image, vergence, construction rayons parallèles à l'axe
	optique pour divergente et convergente, règles primaires et secondaires
	des constructions géométriques, tous les cas pour lentilles convergentes
	et divergentes, relations de conjugaison + démonstration, grandissement
	transversal.
	\bitem{Quelques applications}~: condition de netteté (méthode de
	Bessel, $D \geq 4f'$), champ de vision à travers un miroir plan et
	hauteur d'un arbre.
\end{enumerate}

\section{Cours et exercices}

\section*{Optique chapitre 4 -- Dispositifs optiques}
\begin{enumerate}[label=\Roman*]
	\bitem{L'œil}~: présentation et modélisation, accommodation et
	focales minimales et maximales, réglage d'un instrument optique,
	résolution angulaire et vocabulaire sur les défauts.
	\bitem{La loupe}~: présentation de l'effet loupe, définition
	grossissement général et propriété $G = d_m/f'$ pour la loupe avec
	démonstration.
	\bitem{Appareil photo}~: description, modélisation simple, champ et
	influence de la focale et de la taille du capteur, distance de mise au
	point, profondeur de champ et influence de la distance de mise au point,
	de la focale et de l'ouverture.
	\bitem{Systèmes optiques à plusieurs lentilles}~: association
	quelconque, convergente+convergente en cours, notion de microscope,
	définition lunettes astronomiques Kepler et Galilée, définition système
	afocal, calcul d'encombrement, grossissement $G=-f'_1/f'_2$ et
	démonstration.
\end{enumerate}

\section{Cours uniquement}

\section*{Électrocinétique ch. 1 -- Circuits électriques dans l'ARQS}
\begin{enumerate}[label=\Roman*]
	\bitem{Courant électrique et intensité}~: charge électrique, courant
	électrique, sens conventionnel.
	\bitem{Tension et potentiel}~: définition, additivité, masse, analogie
	électro-hydraulique.
	\bitem{Vocabulaire des circuits électriques}~: circuit, schéma,
	dipôle, nœud, branche, maille~; conventions générateur et récepteur,
	dipôles en série ou dérivation, mesures de tensions et d'intensités.
	\bitem{Lois fondamentales des circuits électriques dans l'ARQS}~:
	approximation, application, loi des branches et nœuds, loi des mailles,
	puissance électrocinétique, fonctionnement générateur et récepteur, et
	conservation de l'énergie.
\end{enumerate}

\section*{Électrocinétique chapitre 2 -- Résistances et sources}
\begin{enumerate}[label=\Roman*]
	\bitem{Généralité sur les dipôles}~: caractéristique courant-tension,
	vocabulaire associé.
	\bitem{Résistance}~: définition et schéma, association en série
	\textbf{et démonstration}, association en parallèle \textbf{et
		démonstration}
	\bitem{Sources}~: sources idéale et réelle de tension, sources idéale
	et réelle de courant, résistances de sortie.
	\bitem{Les ponts diviseurs}~: pont diviseur de tension \textbf{et
		démonstration}, pont diviseur de courant \textbf{et démonstration}.
\end{enumerate}

\section{Questions de cours possibles}
\begin{enumerate}
	\item[] \textbf{Optique~: chapitre 4}
	\item Décrire les caractéristiques d'un œil et donner son modèle en optique
	      géométrique. Définir la plage d'accommodation, le pouvoir de résolution
	      et donner des ordres de grandeur. Décrire les principaux défauts.
	      Quelles sont les valeurs maximale et minimale de la focale du
	      cristallin pour un œil emmétrope~? On rappelle que la distance
	      cristallin-rétine est $d \approx \SI{22.3}{mm}$~;
	\item Présenter le défaut d'un œil hypermétrope \textbf{avec un schéma},
	      comment corriger ce défaut et les points caractéristique du verre
	      correcteur et de l'œil qui doivent être confondus pour corriger la
	      vision de loin. Une représentation optique (du type $AB \opto{\Lc}{O}
		      A'B'$) et un schéma sont nécessaires~;
	\item Décrire un modèle simple de l'appareil photographique avec un schéma.
	      Définir le champ, la mise au point et la profondeur de champ d'un
	      appareil photo~: 3 schémas de mise en situation sont attendus.
	      Connaître, si demandé, la manière dont un paramètre de l'appareil
	      (focale, position capteur, taille du capteur et diaphragme) modifie une
	      caractéristique photographique (profondeur de champ, champ, mise au
	      point)~;
	\item Savoir comment se modélise un microscope et construire l'image d'un
	      objet avant le foyer objet de la première lentille. Définir alors le
	      grossissement et \textbf{donner et démontrer} son expression, en donner
	      un ordre de grandeur et commenter son signe~;
	\item Savoir comment se modélise une lunette de \textbf{Kepler} et
	      construire le chemin de deux rayons parallèles quelconques. Définir
	      alors le grossissement, \textbf{donner et démontrer} son expression, en
	      donner un ordre de grandeur et commenter son signe~;
	\item[] \textbf{Électrocinétique~: chapitre 1}
	\item Énoncer et expliquer les conditions de l'ARQS, donner des exemples
	      d'application et non-application avec des valeurs numériques~;
	\item[] \textbf{Électrocinétique~: chapitre 2}
	\item Démontrer puis utiliser la loi des mailles pour trouver l'intensité
	      dans un circuit simple~;
	\item Démontrer les relations des associations séries et parallèles de
	      résistances \textbf{et} déterminer la résistance équivalente d'une
	      portion de circuit donné par l'examinataire~;
	\item Donner et démontrer les relations des ponts diviseurs de tension et de
	      courant. Application très simple du pont divisuer de tension sur un
	      circuit à une maille proposé par l'examinataire~;
	\item Présenter les sources réelles de tension et de courant. Comment
	      s'appellent ces modèles~? À l'aide de relations de ponts diviseurs,
	      démontrer dans quelles conditions on peut les considérer comme idéales.
\end{enumerate}

\end{document}
