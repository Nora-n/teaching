\documentclass[a4paper, 11pt, final, garamond]{book}
\usepackage{cours-preambule}

\raggedbottom

\makeatletter
\renewcommand{\@chapapp}{Programme de kh\^olle -- semaine}
\makeatother

\begin{document}
\setcounter{chapter}{2}

\chapter{Du 30 septembre au 03 octobre}

\section{Exercices uniquement}
\subsection(O3){Miroir plan et lentilles minces}

\section{Cours et exercices}

\subsection(O4){Dispositifs optiques}
\begin{enumerate}[label=\Roman*]
	\item[b]{L'œil}: présentation et modélisation, accommodation et
	focales minimales et maximales, réglage d'un instrument optique, résolution
	angulaire et vocabulaire sur les défauts.
	\item[b]{La loupe}: présentation de l'effet loupe, définition
	grossissement général et propriété $G = d_m/f'$ pour la loupe avec
	démonstration.
	\item[b]{Appareil photo}: description, modélisation simple, champ et
	influence de la focale et de la taille du capteur, distance de mise au point,
	profondeur de champ et influence de la distance de mise au point, de la focale
	et de l'ouverture.
	\item[b]{Systèmes optiques à plusieurs lentilles}: association
	quelconque, notion de microscope, définition lunettes astronomiques
	\textsc{Kepler} et \textsc{Galilée}, définition système afocal, calcul
	d'encombrement, grossissement $G=-f'_1/f'_2$ et démonstration, cercle
	oculaire.
\end{enumerate}

\section{Cours uniquement}

\subsection(E1){Circuits électriques dans l'ARQS}
\begin{enumerate}[label=\Roman*]
	\item[b]{Courant électrique et intensité}: charge électrique, courant
	électrique, sens conventionnel.
	\item[b]{Tension et potentiel}: définition, additivité, masse, analogie
	électro-hydraulique.
	\item[b]{Vocabulaire des circuits électriques}: circuit, schéma,
	dipôle, nœud, branche, maille~; conventions générateur et récepteur, dipôles
	en série ou dérivation, mesures de tensions et d'intensités.
	\item[b]{Lois fondamentales des circuits électriques dans l'ARQS}:
	approximation, application, lois de \textsc{Kirchhoff} (des branches et nœuds,
	des mailles), puissance électrocinétique, fonctionnement générateur et
	récepteur, conservation de l'énergie.
\end{enumerate}

\subsection(E2){Dipôles et associations}
\begin{enumerate}[label=\Roman*]
	\item[b]{Généralité sur les dipôles}: caractéristique courant-tension,
	vocabulaire associé.
	\item[b]{Résistance}: définition et schéma, association en série
	\textbf{et démonstration}, association en parallèle \textbf{et
		démonstration}, ponts diviseurs de tension et de courants.
	\item[b]{Sources}: sources idéale et réelle de tension, sources idéale
	et réelle de courant, résistances de sortie~; entraînement de ponts.
	% \item[b]{Condensateur et bobine}: présentation du condensateur, relations
	%   fondamentales ($q = Cu$ et RCT), conditions limites, associations,
	%   condensateur réel et énergie stockée~;
	%   présentation de la bobine, RCT, conditions limites, assocations, bobine
	%   réelle, énergie stockée.
\end{enumerate}

\newpage

\section{Questions de cours possibles}
\begin{enumerate}
	\subsection(O4){Dispositifs optiques}
	\item Décrire les caractéristiques d'un œil et donner son modèle en optique
	      géométrique (Df.O4.1). Définir la plage d'accommodation et les valeurs pour
	      un œil emmétrope (Df.O4.2), le pouvoir de résolution \textbf{avec un
		      schéma} et un ordre de grandeur (Df.O4.3, Odgr.O4.1).
	      Quelles sont les valeurs maximale et minimale de la focale du
	      cristallin pour un œil emmétrope~? On rappelle que la
	      distance cristallin-rétine est $d \approx \SI{22.3}{mm}$ (Ap.O4.1).

	\item Décrire simplement les principaux défauts et la manière de les corriger
	      (Df.O4.4).
	      Présenter le défaut d'un œil hypermétrope \textbf{avec un schéma},
	      comment corriger ce défaut et les points caractéristique du verre
	      correcteur et de l'œil qui doivent être confondus pour corriger la
	      vision de loin. Un schéma de principe (du type $\rm AB \opto{\Lc}{O}
		      A'B'$) et un schéma de pour correction (verre de lunette + œil) sont
	      nécessaires (TDO4.app.III)
	\item Décrire un modèle simple de l'appareil photographique (Df.O4.7 et 8).
	      Quelle la différence avec un œil (At.O4.1)~? Définir le champ, la mise au
	      point et la profondeur de champ d'un appareil photo (O4|III/B, C
	      et D). \textbf{Donner et démontrer} la manière dont un paramètre de
	      l'appareil (focale, position capteur, taille du capteur et diaphragme)
	      modifie une caractéristique photographique (profondeur de champ, champ,
	      mise au point), au choix de l'interrogataire (Ip.O4.4, et tout le III/).
	      % \item Savoir comment se modélise un microscope et construire l'image
	      % d'un
	      % objet avant le foyer objet de la première lentille. Définir alors le
	      % grossissement et \textbf{donner et démontrer} son expression, en
	      % donner
	      % un ordre de grandeur et commenter son signe~;
	\item Définir ce qu'est une lunette astronomique et les deux types classiques
	      de lunette avec schéma de principe ($\rm A \opto{\Lc}{\rm O}
		      A'$), et schéma entier pour la lunette de \textsc{Kepler} (Df.O4.12).
	      Définir un système afocal (Df.O4.13). Exprimer leur encombrement en
	      fonction de $V_1$ et $V_2$ les vergences des lentilles (Ap.O4.6).
	      Établir la formule du grossissement (Pt.O4.2, Dm.O4.2).
	      \subsection(E1){Circuits électriques dans l'ARQS}
	\item Énoncer et expliquer les conditions de l'ARQS (L.E1.1, Itp.E1.1), donner
	      des exemples d'application et non-application avec des valeurs numériques
	      (Ap.E1.3)~;
	\item Énoncer les lois de \textsc{Kirchhoff} (branche, nœud, maille) et
	      expliquer leur origine (L.E1.2, 3 et 4). Application sur un schéma donné
	      par l'interrogataire (Ap.E1.4). Présenter les conventions générateur et
	      récepteur (Df.E1.9), et établir le signe de la puissance selon le dipôle
	      et la convention choisie (Df.E1.13, Ipt.E1.3).
	      \subsection(E2){Dipôles et associations}
	\item Présenter le résistor et donner sa relation courant-tension pour les
	      deux conventions (Df.E2.3, At.E2.1), en déduire sa puissance en convention
	      récepteur (Ipl.E2.1). Tracer sa caractéristique et y associer le vocabulaire
	      pertinent (Ex.E2.2, Df.E2.2). Indiquer alors comment traiter les cas des
	      interrupteurs ouvert et fermé avec un schéma pour chacun (Pt.E2.1).
	\item Démontrer les relations des associations séries et parallèles de
	      résistances \textbf{et} déterminer la résistance équivalente d'une
	      portion de circuit donné par l'examinataire (Pt.E2.2 et 3, Dm.E2.1 et 2,
	      Ap.E2.1).
	\item Donner et démontrer les relations des ponts diviseurs de tension et de
	      courant (Pt.E2.4 et 5, Dm.E2.3 et 4). Application très simple de
	      \textbf{chaque pont} sur un circuit proposé par l'examinataire
	      (Ap.E2.2).
	\item Présenter les sources réelles de tension et de courant \textit{via} les
	      modèles de \textsc{Thévenin} et \textsc{Norton} ainsi que leur relation
	      courant-tension à l'aide de schémas (Df.E2.5 et 7), puis tracer leurs
	      caractéristiques (Ex.E2.3 et 4). À l'aide de relations de ponts
	      diviseurs, démontrer dans quelles conditions on peut les considérer
	      comme idéales (Pt.E2.6 et 7, Dm.E2.5 et 6).
\end{enumerate}

\end{document}
