\documentclass[a4paper, 12pt, final, garamond]{book}
\usepackage{cours-preambule}

\raggedbottom

\makeatletter
\renewcommand{\@chapapp}{Programme de kh\^olle -- semaine}
\makeatother

\begin{document}
\setcounter{chapter}{3}

\chapter{Du 07 au 10 octobre}

\section{Cours et exercices}

\subsection(E1){Circuits électriques dans l'ARQS}
\begin{enumerate}[label=\Roman*]
	\item[b]{Courant électrique et intensité}: charge électrique, courant
	      électrique, sens conventionnel.
	\item[b]{Tension et potentiel}: définition, additivité, masse, analogie
	      électro-hydraulique.
	\item[b]{Vocabulaire des circuits électriques}: circuit, schéma,
	      dipôle, nœud, branche, maille~; conventions générateur et récepteur, dipôles
	      en série ou dérivation, mesures de tensions et d'intensités.
	\item[b]{Lois fondamentales des circuits électriques dans l'ARQS}:
	      approximation, application, lois de \textsc{Kirchhoff} (des branches et nœuds,
	      des mailles), puissance électrocinétique, fonctionnement générateur et
	      récepteur, conservation de l'énergie.
\end{enumerate}

\subsection(E2){Dipôles et associations}
\begin{enumerate}[label=\Roman*]
	\item[b]{Généralité sur les dipôles}: caractéristique courant-tension,
	      vocabulaire associé.
	\item[b]{Résistance}: définition et schéma, association en série
	      \textbf{et démonstration}, association en parallèle \textbf{et
		      démonstration}, ponts diviseurs de tension et de courants.
	\item[b]{Sources}: sources idéale et réelle de tension, sources idéale
	      et réelle de courant, résistances de sortie~; entraînement de ponts.
	\item[b]{Condensateur et bobine}: présentation du condensateur, relations
	      fondamentales ($q = Cu$ et RCT), conditions limites, associations,
	      condensateur réel et énergie stockée~;
	      présentation de la bobine, RCT, conditions limites, assocations, bobine
	      réelle, énergie stockée.
\end{enumerate}

\section{Cours uniquement}

\subsection(E3){Circuits du premier ordre}
\begin{enumerate}[label=\Roman*]
	\item[b]{Circuits RC série}
	      \begin{enumerate}[label=\Alph*]
		      \item[b]{Échelon montant}:
		            présentation RC série en charge, équation différentielle sur
		            $u$, unité de $RC$, résolution avec méthode, représentation
		            graphique, détermination constante de temps et temps de réponse
		            à 99\%~; détermination de l'intensité, bilan de puissance et
		            d'énergie.
		      \item[b]{Échelon descendant}: idem sans bilan, puis méthode à
		            plusieurs mailles.
	      \end{enumerate}
	\item[b]{Circuits RL série}
	      \begin{enumerate}[label=\Alph*]
		      \item[b]{Échelon montant}:
		            idem RC en charge, sans bilan.
		      \item[b]{Échelon descendant}: idem RC en décharge.
	      \end{enumerate}
\end{enumerate}

\newpage
\section{Questions de cours possibles}
\begin{enumerate}
	\subsection(E1){Circuits électriques dans l'ARQS}
	\item Énoncer et expliquer les conditions de l'ARQS (L.E1.1, Itp.E1.1), donner
	      des exemples d'application et non-application avec des valeurs numériques
	      (Ap.E1.3)~;
	\item Énoncer les lois de \textsc{Kirchhoff} (branche, nœud, maille) et
	      expliquer leur origine (L.E1.2, 3 et 4). Application sur un schéma donné
	      par l'interrogataire (Ap.E1.4). Présenter les conventions générateur et
	      récepteur (Df.E1.9), et établir le signe de la puissance selon le dipôle
	      et la convention choisie (Df.E1.13, Ipt.E1.3).
	      \subsection(E2){Dipôles et associations}
	\item Présenter le résistor et donner sa relation courant-tension pour les
	      deux conventions (Df.E2.3, At.E2.1), en déduire sa puissance en convention
	      récepteur (Ipl.E2.1). Tracer sa caractéristique et y associer le vocabulaire
	      pertinent (Ex.E2.2, Df.E2.2). Indiquer alors comment traiter les cas des
	      interrupteurs ouvert et fermé avec un schéma pour chacun (Pt.E2.1).
	\item Démontrer les relations des associations séries et parallèles de
	      résistances \textbf{et} déterminer la résistance équivalente d'une
	      portion de circuit donné par l'examinataire (Pt.E2.2 et 3, Dm.E2.1 et 2,
	      Ap.E2.1).
	\item Donner et démontrer les relations des ponts diviseurs de tension et de
	      courant (Pt.E2.4 et 5, Dm.E2.3 et 4). Application très simple de
	      \textbf{chaque pont} sur un circuit proposé par l'examinataire
	      (Ap.E2.2).
	\item Présenter les sources réelles de tension et de courant \textit{via} les
	      modèles de \textsc{Thévenin} et \textsc{Norton} ainsi que leur relation
	      courant-tension à l'aide de schémas (Df.E2.5 et 7), puis tracer leurs
	      caractéristiques (Ex.E2.3 et 4). À l'aide de relations de ponts
	      diviseurs, démontrer dans quelles conditions on peut les considérer
	      comme idéales (Pt.E2.6 et 7, Dm.E2.5 et 6).
	\item Présenter et démontrer les caractéristiques d'un condensateur et d'une
	      bobine (E2|IV/A) et B)~: schémas, relations courant-tension (sans
	      démonstration pour la bobine), continuité, dipôle équivalent en régime
	      permanent, énergie stockée.
	\item Démontrer les relations des associations séries et parallèles
	      d'un condensateur \textbf{et} d'une bobine (E2|IV/A)4- et B)4-).
	      \subsection(E3){Circuits du premier ordre}
	\item Présenter le schéma et la condition initiale, donner et démontrer
	      l'équation différentielle, \textbf{justifier l'unité de $\tau$}, établir
	      la solution et la tracer pour un des quatre circuits suivants~:
	      \begin{tasks}[label=\protect\fbox{\Alph*}, label-width=4ex](4)
		      \task RC en charge
		      \task RC en décharge
		      \task RL montant
		      \task RL régime libre
	      \end{tasks}
	\item Faire un bilan de puissance, éventuellement un bilan d'énergie,
	      démontrer comment trouver graphiquement la constante de temps et établir
	      le temps de réponse à 99\% pour un des circuits suivants~:
	      \begin{tasks}[label=\protect\fbox{\Alph*}, label-width=4ex](2)
		      \task Circuit RC en charge (Pt.E3.5 et 6, Dm.E3.5 et 6, Ipl.E3.1,
		      Pt.E3.3 et Dm.E3.3)
		      \task Circuit RL échelon montant (Pt.E3.15 et Dm.E3.14, Ipl.E3.3,
		      Pt.E3.13 et Dm.E3.3)
	      \end{tasks}
\end{enumerate}

\end{document}
