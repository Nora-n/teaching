\documentclass[a4paper, 12pt, final, garamond]{book}
\usepackage{cours-preambule}

\raggedbottom

\makeatletter
\renewcommand{\@chapapp}{Programme de kh\^olle -- semaine}
\makeatother

\begin{document}
\setcounter{chapter}{3}

\chapter{Du 09 au 12 octobre}

\section{Cours et exercices}

\section*{Électrocinétique ch.\ 1 -- Circuits électriques dans l'ARQS}
\begin{enumerate}[label=\Roman*]
	\bitem{Courant électrique et intensité}~: charge électrique, courant
	électrique, sens conventionnel.
	\bitem{Tension et potentiel}~: définition, additivité, masse,
	analogie électro-hydraulique.
	\bitem{Vocabulaire des circuits électriques}~: circuit, schéma,
	dipôle, nœud, branche, maille~; conventions générateur et récepteur,
	dipôles en série ou dérivation, mesures de tensions et d'intensités.
	\bitem{Lois fondamentales des circuits électriques dans l'ARQS}~:
	approximation, application, loi des branches et nœuds, loi des mailles,
	puissance électrocinétique, fonctionnement générateur et récepteur, et
	conservation de l'énergie.
\end{enumerate}

\section*{Électrocinétique chapitre 2 -- Résistances et sources}
\begin{enumerate}[label=\Roman*]
	\bitem{Généralité sur les dipôles}~: caractéristique courant-tension,
	vocabulaire associé.
	\bitem{Résistance}~: définition et schéma, association en série
	\textbf{et démonstration}, association en parallèle \textbf{et
		démonstration}, pont diviseur de tension \textbf{et démonstration}, pont
	diviseur de courant \textbf{et démonstration}.
	\bitem{Sources}~: sources idéale et réelle de tension, sources idéale
	et réelle de courant, résistances de sortie.
\end{enumerate}

\section{Cours uniquement}

\section*{Électrocinétique ch.\ 3 -- Capacités et inductances~: circuits du
  1\ier{} ordre}
\begin{enumerate}[label=\Roman*]
	\bitem{Condensateur}~: présentation, relation fondamentale,
	relation courant-tension, continuité et régime permanent, associations
	série et parallèle, condensateur réel, énergie stockée.

	\bitem{Bobine}~: présentation, relation courant-tension, continuité
	et régime permanent, associations série et parallèle, bobine réelle, énergie
	stockée.
	\bitem{Circuits RC série}~: échelon montant~: définition, présentation RC
	série en charge, équation différentielle, unité de $RC$, résolution avec
	méthode, représentation graphique, détermination constante de temps et temps
	de réponse régime permanent.
\end{enumerate}

\newpage
\section{Questions de cours possibles}
\begin{enumerate}
	\item[] \textbf{Chapitre 1}
	\item Énoncer et expliquer les conditions de l'ARQS, donner des exemples
	      d'application et non-application~;
	\item[] \textbf{Chapitre 2}
	\item Démontrer puis utiliser la loi des mailles pour trouver l'intensité
	      dans un circuit simple (deux mailles possible)~;
	\item Démontrer les relations des associations séries et parallèles des
	      résistances \textbf{et} déterminer la résistance équivalente d'une
	      portion de circuit donné par l'examinataire~;
	\item Démontrer les relations des ponts diviseurs de tension et de courant
	      et en utiliser sur un schéma donné par l'examinataire~;
	\item Présenter les sources réelles de tension et de courant. Comment
	      s'appellent ces modèles~? À l'aide de relations de ponts diviseurs,
	      démontrer dans quelles conditions on peut les considérer comme idéales.
	\item[] \textbf{Chapitre 3}
	\item Présenter et démontrer les caractéristiques d'un condensateur et d'une
	      bobine~: relation courant-tension (sans démonstration pour la bobine),
	      continuité, régime permanent, énergie stockée.
	\item Démontrer les relations des associations séries et parallèles
	      d'un condensateur \textbf{et} d'une bobine.
	\item Présenter le circuit RC en charge sous un échelon de tension $E$
	      (schéma et condition initiale), donner et démontrer l'équation
	      différentielle sur $u_C$, donner la solution et la tracer. Démontrer
	      comment trouver graphiquement la constante de temps et le temps de
	      réponse à 99\%.
\end{enumerate}

\end{document}
