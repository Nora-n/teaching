\documentclass[a4paper, 11pt, final, garamond]{book}
\usepackage{cours-preambule}

\raggedbottom

\makeatletter
\renewcommand{\@chapapp}{Programme de kh\^olle -- semaine}
\makeatother

\begin{document}
\setcounter{chapter}{4}

\chapter{Du 14 au 17 octobre}

\section{Exercices uniquement}

\subsection(E2){Dipôles et associations}
% \begin{enumerate}[label=\Roman*]
% 	\item[b]{Généralité sur les dipôles}: caractéristique courant-tension,
% 	      vocabulaire associé.
% 	\item[b]{Résistance}: définition et schéma, association en série
% 	      \textbf{et démonstration}, association en parallèle \textbf{et
% 		      démonstration}, ponts diviseurs de tension et de courants.
% 	\item[b]{Sources}: sources idéale et réelle de tension, sources idéale
% 	      et réelle de courant, résistances de sortie~; entraînement de ponts.
% 	\item[b]{Condensateur et bobine}: présentation du condensateur, relations
% 	      fondamentales ($q = Cu$ et RCT), conditions limites, associations,
% 	      condensateur réel et énergie stockée~;
% 	      présentation de la bobine, RCT, conditions limites, assocations, bobine
% 	      réelle, énergie stockée.
% \end{enumerate}

\section{Cours et exercices}

\subsection(E3){Circuits du premier ordre}
\begin{enumerate}[label=\Roman*]
	\item[b]{Circuits RC série}
	      \begin{enumerate}[label=\Alph*]
		      \item[b]{Échelon montant}:
		            présentation RC série en charge, équation différentielle sur
		            $u$, unité de $RC$, résolution avec méthode, représentation
		            graphique, détermination constante de temps et temps de réponse
		            à 99\%~; détermination de l'intensité, bilan de puissance et
		            d'énergie.
		      \item[b]{Échelon descendant}: idem sans bilan, puis méthode à
		            plusieurs mailles.
	      \end{enumerate}
	\item[b]{Circuits RL série}
	      \begin{enumerate}[label=\Alph*]
		      \item[b]{Échelon montant}:
		            idem RC en charge, sans bilan.
		      \item[b]{Échelon descendant}: idem RC en décharge.
	      \end{enumerate}
\end{enumerate}

\section{Cours uniquement}
\subsection(E4){Oscillateur harmonique}
\begin{enumerate}[label=\Roman*]
	\item[b]{Introduction}: signal sinusoïdal général, introduction signal
	      complexe, équation différentielle générale, changement de variable,
	      étude expérimentale avec simulation numérique.
	\item[b]{Circuit RC régime libre}: présentation, équation différentielle,
	      résolution et graphique, représentation dans l'espace des phases, bilan
	      énergétique.
	\item[b]{Exemple harmonique mécanique~: ressort horizontal libre}: force
	      rappel, présentation, équation différentielle et solution, analogie
	      ressort-LC, bilan énergétique, espace des phases.
	\item[b]{Complément LC montant}: présentation, équation différentielle et
	      solution, simulation numérique.
\end{enumerate}

\subsection(E5){Oscillateur amorti}
\begin{enumerate}[label=\Roman*]
	\item[b]{Introduction}: exemple avec simulation numérique, tracé des
	      différentes réponses en fonction de $R_c$, équation différentielle générale,
	      équation caractéristique et régimes de solutions.
	\item[b]{RLC série libre}: présentation, bilan énergétique et justification
	      évolution amortie, équation différentielle et identification facteur de
	      qualité.
	      %~; résolutions dans chaque cas : solution, tracé, espace des phases et
	      %régime transitoire à 95\%.
	      % \item[b]{Ressort amorti}: présentation, équation différentielle, analogie RLC,
	      %   bilan énergétique. Solutions laissées en démonstration libre, voir RLC.
\end{enumerate}

\newpage

\section{Questions de cours possibles}
\begin{enumerate}
	\subsection(E3){Circuits du premier ordre}
	\item Présenter le schéma et la condition initiale, donner et démontrer
	      l'équation différentielle, \textbf{justifier l'unité de $\tau$}, établir
	      la solution et la tracer pour un des quatre circuits suivants~:
	      \begin{tasks}[label=\protect\fbox{\Alph*}, label-width=4ex](4)
		      \task RC en charge
		      \smallbreak
		      (E3|I/A)
		      \task RC en décharge
		      \smallbreak
		      (E3|I/B)
		      \task RL montant
		      \smallbreak
		      (E3|II/A)
		      \task RL régime libre
		      \smallbreak
		      (E3|II/B)
	      \end{tasks}
	\item Faire un bilan de puissance, éventuellement un bilan d'énergie,
	      démontrer comment trouver graphiquement la constante de temps et établir le
	      temps de réponse à 99\% pour un des circuits suivants~:
	      \begin{tasks}[label=\protect\fbox{\Alph*}, label-width=4ex](2)
		      \task Circuit RC en charge
		      \smallbreak
		      (Pt.E3.5 et 6, Dm.E3.5 et 6, Ipl.E3.1, Pt.E3.3 et Dm.E3.3)
		      \task Circuit RL échelon montant
		      \smallbreak
		      (Pt.E3.15 et Dm.E3.14, Ipl.E3.3, Pt.E3.13 et Dm.E3.3)
	      \end{tasks}
	      \subsection(E4){Oscillateur harmonique}
	\item Donner la forme générale d'un signal sinusoïdal en détaillant les
	      paramètres, expliquer ce qu'est la pulsation et exprimer la période en
	      fonction de la pulsation (Df.E4.1). Bonus~: comment visualiser l'amplitude
	      et la phase d'un signal sinusoïdal par passage aux complexes (Itp.E4.1)~?

	\item Donner l'équation différentielle générale d'un oscillateur harmonique et
	      les deux formes de solutions associées (Pt.E4.1 et 2). Expliquer le principe
	      du changement de variable avec cette équation comme exemple, et résoudre
	      l'équation du RL montant avec cette méthode (Pt.E4.3 et Apl.E4.1).

	\item Présenter le schéma et les conditions initales, établir l'équation
	      différentielle, \textbf{justifier l'unité de $\w_0$}, établir les
	      solutions de $u_C(t)$ et $i(t)$ (ou $\ell(t)|x(t)$ et $v(t)$) et les
	      tracer en fonction du temps \textbf{puis} dans l'espace des phases sans
	      tenir compte des constantes mutiplicatives pour un des sytèmes
	      suivants~:
	      \begin{tasks}[label=\protect\fbox{\Alph*}, label-width=4ex](3)
		      \task LC libre
		      \smallbreak
		      (Df.E4.2, Dm.E4.1 et Pt.E4.4, Apl.E4.2, Dm.E4.2, Pt.E4.5 et Itp.E4.2)
		      \task Ressort libre sans frottements
		      \smallbreak
		      (Df.E4.4, Dm.E4.4 et Pt.E4.7, Fig.4.6, Itp.E4.3)
		      \task LC montant
		      \smallbreak
		      (Df.E4.6, Pt.E4.9 et Dm.E4.6, cf.\ \texttt{url} simulation)
	      \end{tasks}

	\item Faire un \textbf{bilan de puissance} pour démontrer la
	      conservation de l'énergie totale, vérifier la conservation à l'aide des
	      solutions analytiques, et tracer la forme des graphiques pour un des
	      systèmes suivants~:
	      \begin{tasks}[label=\protect\fbox{\Alph*}, label-width=4ex](2)
		      \task LC libre
		      \smallbreak
		      (Dm.E4.3 et Pt.E4.6, Rmq.E4.1)
		      \task Ressort libre sans frottements
		      \smallbreak
		      (Dm.E4.5 et Pt.E4.8, Rmq.E4.2)
	      \end{tasks}

	\item Faire l'analogie complète entre les deux systèmes harmoniques LC libre
	      et ressort sans frottement (E4|II/ et E4|III/ en général)~: présentation,
	      conditions initiales, équations différentielles \textbf{sans démonstration},
	      correspondance entre les grandeurs (Ipt.E4.2), tracé des solutions
	      \textbf{dans l'espace des phases} sans résolution et commenter sur la
	      conservation de l'énergie visible dans le graphique.

	      \subsection(E5){Oscillateur amorti}
	\item Présenter l'équation différentielle générale d'un oscillateur amorti
	      (Pt.E5.1), démontrer la dimension et interpréter le facteur de qualité
	      (Rmq.E5.1), démontrer l'expression de l'équation caractéristique et son
	      discriminant (Df.E5.1), démontrer l'existence des régimes de solutions en
	      fonction de la valeur de $Q$, les nommer et les décrire (Ipl.E5.1), et
	      représenter les formes de solutions vues pour le RLC série pour chaque
	      régime (E5|I/A).

	\item Présenter le RLC libre et ses conditions initiales (Df.E5.2, CI dans
	      Pt.E5.3), faire le bilan de puissance et expliquer la différence avec le
	      système harmonique (Dm.E5.1, Pt.E5.2 et Ipt.E5.2), déterminer l'équation
	      différentielle du circuit en identifiant les expressions de $\w_0$ et $Q$
	      (Dm.E5.2, Pt.E5.3).
\end{enumerate}

\end{document}
