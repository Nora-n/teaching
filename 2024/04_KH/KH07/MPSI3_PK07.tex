\documentclass[a4paper, 12pt, final, garamond]{book}
\usepackage{cours-preambule}

\raggedbottom

\makeatletter
\renewcommand{\@chapapp}{Programme de kh\^olle -- semaine}
\makeatother

\begin{document}
\setcounter{chapter}{6}

\chapter{Du 12 au 14 novembre}

\section{Exercices uniquement}

\subsection(E4){Oscillateur harmonique}
% \begin{enumerate}[label=\Roman*]
% 	\item[b]{Introduction}: signal sinusoïdal général, introduction signal
% 	      complexe, équation différentielle générale, changement de variable,
% 	      étude expérimentale avec simulation numérique.
% 	\item[b]{Circuit RC régime libre}: présentation, équation différentielle,
% 	      résolution et graphique, représentation dans l'espace des phases, bilan
% 	      énergétique.
% 	\item[b]{Exemple harmonique mécanique~: ressort horizontal libre}: force
% 	      rappel, présentation, équation différentielle et solution, analogie
% 	      ressort-LC, bilan énergétique, espace des phases.
% 	\item[b]{Complément LC montant}: présentation, équation différentielle et
% 	      solution, simulation numérique.
% \end{enumerate}

\subsection(E5){Oscillateur amorti}
% \begin{enumerate}[label=\Roman*]
% 	\item[b]{Introduction}: exemple avec simulation numérique, tracé des
% 	      différentes réponses en fonction de $R_c$, équation différentielle
% 	      générale, équation caractéristique et régimes de solutions.
% 	\item[b]{RLC série libre}: présentation, bilan énergétique et justification
% 	      évolution amortie, équation différentielle et identification facteur de
% 	      qualité~; résolutions dans chaque cas : solution, tracé, espace des
% 	      phases et régime transitoire à 95\%.
% 	\item[b]{Ressort amorti}: présentation, équation différentielle, analogie RLC,
% 	      bilan énergétique. Solutions laissées en démonstration libre, voir
% 	      RLC.
% \end{enumerate}

\section{Cours uniquement}
\subsection(TM1){Introduction aux transformations}
\begin{enumerate}[label=\Roman*]
	\item[b]{Vocabulaire général}~: atomes et molécules, classification par
	      composition, états de la matière et systèmes physico-chimiques,
	      transformations de la matière.
	\item[b]{Quantification des systèmes}~: mole, masse molaire, fractions
	      molaire et massique, masse volumique, concentrations molaire et
	      massique, dilution~; pression d'un gaz, modèle du gaz parfait, volume
	      molaire, pression partielle et loi de \textsc{Dalton}~; intensivité et
	      extensivité, activité d'un élément chimique.
\end{enumerate}

\subsection(TM2){Transformation et équilibre chimique}
\begin{enumerate}[label=\Roman*]
	\item[b]{Avancement d'une réaction}~: présentation, avancements molaire
	      et volumique, tableau d'avancement et $n\ind{tot, gaz}$, coefficients
	      stœchiométriques algébriques.
	\item[b]{États finaux d'un système chimique}~: types d'avancements~; réaction
	      totale~; réaction limitée~: quantifications de l'avancement (taux de
	      conversion, coefficient de dissociation, rendement), quotient de
	      réaction, constante d'équilibre et combinaison de réactions, réactions
	      quasi-nulles et quasi-totales.
	\item[b]{Évolution d'un système chimique}~: quotient réactionnel et
	      sens d'évolution, cas des ruptures d'équilibre, résumé pratique de
	      résolution.
\end{enumerate}

\vspace{\fill}

\begin{center}
	\begin{framed}
		\Large
		\bfseries
		Plusieurs questions simples sont possibles cette semaine
	\end{framed}
\end{center}

\vspace{\fill}

\newpage

\section{Questions de cours possibles}

\begin{enumerate}
	%       \subsection(E4){Oscillateur harmonique}
	% \item Donner l'équation différentielle générale d'un oscillateur harmonique et
	%       les deux formes de solutions associées (Pt.E4.1 et 2). Expliquer le principe
	%       du changement de variable avec cette équation comme exemple, et résoudre
	%       l'équation du RL montant avec cette méthode (Pt.E4.3 et Apl.E4.1).
	%
	% \item Présenter le schéma et les conditions initales, établir l'équation
	%       différentielle, \textbf{justifier l'unité de $\w_0$}, établir les
	%       solutions de $u_C(t)$ et $i(t)$ (ou $\ell(t)|x(t)$ et $v(t)$) et les
	%       tracer en fonction du temps \textbf{puis} dans l'espace des phases sans
	%       tenir compte des constantes mutiplicatives pour un des sytèmes
	%       suivants~:
	%       \begin{tasks}[label=\protect\fbox{\Alph*}, label-width=4ex](3)
	% 	      \task LC libre
	% 	      \smallbreak
	% 	      (Df.E4.2, Dm.E4.1 et Pt.E4.4, Apl.E4.2, Dm.E4.2, Pt.E4.5 et Itp.E4.2)
	% 	      \task Ressort libre sans frottements
	% 	      \smallbreak
	% 	      (Df.E4.4, Dm.E4.4 et Pt.E4.7, Fig.4.6, Itp.E4.3)
	% 	      \task LC montant
	% 	      \smallbreak
	% 	      (Df.E4.6, Pt.E4.9 et Dm.E4.6, cf.\ \texttt{url} simulation)
	%       \end{tasks}
	%
	% \item Faire un \textbf{bilan de puissance} pour démontrer la
	%       conservation de l'énergie totale, vérifier la conservation à l'aide des
	%       solutions analytiques, et tracer la forme des graphiques pour un des
	%       systèmes suivants~:
	%       \begin{tasks}[label=\protect\fbox{\Alph*}, label-width=4ex](2)
	% 	      \task LC libre
	% 	      \smallbreak
	% 	      (Dm.E4.3 et Pt.E4.6, Rmq.E4.1)
	% 	      \task Ressort libre sans frottements
	% 	      \smallbreak
	% 	      (Dm.E4.5 et Pt.E4.8, Rmq.E4.2)
	%       \end{tasks}
	%
	% \item Faire l'analogie complète entre les deux systèmes harmoniques LC libre
	%       et ressort sans frottement (E4|II/ et E4|III/ en général)~: présentation,
	%       conditions initiales, équations différentielles \textbf{sans démonstration},
	%       correspondance entre les grandeurs (Ipt.E4.2), tracé des solutions
	%       \textbf{dans l'espace des phases} sans résolution et commenter sur la
	%       conservation de l'énergie visible dans le graphique.

	% \subsection(E5){Oscillateur amorti}
	% \item Présenter l'équation différentielle générale d'un oscillateur amorti
	%       (Pt.E5.1), démontrer la dimension et interpréter le facteur de qualité
	%       (Rmq.E5.1), démontrer l'expression de l'équation caractéristique et son
	%       discriminant (Df.E5.1), démontrer l'existence des régimes de solutions en
	%       fonction de la valeur de $Q$, les nommer et les décrire (Ipl.E5.1), et
	%       représenter les formes de solutions vues pour le RLC série pour chaque
	%       régime (E5|I/A).
	%
	% \item Présenter le RLC libre et ses conditions initiales (Df.E5.2, CI dans
	%       Pt.E5.3), faire le bilan de puissance et expliquer la différence avec le
	%       système harmonique (Dm.E5.1, Pt.E5.2 et Ipt.E5.2), déterminer l'équation
	%       différentielle du circuit en identifiant les expressions de $\w_0$ et $Q$
	%       (Dm.E5.2, Pt.E5.3).
	%
	% \item Faire l'analogie complète entre les deux systèmes amortis RLC libre et
	%       ressort avec frottement fluide (E5|II/ et E5|III/ en général)~:
	%       présentation, conditions initiales, équations différentielles \textbf{sans
	% 	      démonstration}, correspondance entre les grandeurs (Ipt.E5.3), tracer
	%       de solutions \textbf{dans l'espace des phases} selon différentes
	%       valeurs de $Q$ sans résolution et commenter sur l'évolution de
	%       l'énergie visible dans le graphique.
	%
	% \item Résoudre l'équation différentielle d'un oscillateur amorti de conditions
	%       initiales données par l'interrogataire pour l'un des trois régimes
	%       possibles~:
	%       \begin{tasks}[label=\protect\fbox{\Alph*}, label-width=4ex](3)
	% 	      \task Pseudo-pér.\
	% 	      (E5|II/D)1-1.)
	% 	      \task Critique
	% 	      (E5|II/D)2-1.)
	% 	      \task Apériodique
	% 	      (E5|II/D)3-1.)
	%       \end{tasks}
	%
	% \item Déterminer la durée du régime transitoire à 95\% pour l'un des trois
	%       régimes suivants~:
	%       \begin{tasks}[label=\protect\fbox{\Alph*}, label-width=4ex](3)
	% 	      \task \strs{1}
	% 	      Pseudo-pér.\
	% 	      (E5|II/D)1-2.)
	% 	      \task \strs{1}
	% 	      Critique
	% 	      (E5|II/D)2-2.)
	% 	      \task \strs{3}
	% 	      Apériodique
	% 	      (E5|II/D)3-2.)
	%       \end{tasks}
	%       On rappelle le développement asymptotique $\sqrt{1+x} \Sim_{x \ll 1}
	% 	      1+x/2$ et l'approximation numérique $\ln (20) \approx \pi$.
	%
	\subsection(TM1){Introduction aux transformations}
	\item{} (Ap.TM1.5) L'air est constitué, en quantité de matière, à 80\% de
	      \ce{N2} et à 20\% de dioxygène \ce{O2}.
	      On a
	      $M(\ce{N_2}) = \SI{28.0}{g.mol^{-1}}$ et
	      $M(\ce{O_2}) = \SI{32.0}{g.mol^{-1}}$.
	      En déduire les fractions molaires puis les fractions massiques.
	\item{} (Ap.TM1.7 et 8) On dissout une masse $m = \SI{2.00}{g}$ de sel
	      NaCl$\sol$ dans $V = \SI{100}{mL}$ d'eau.
	      \smallbreak
	      On donne
	      $M(\ce{NaCl}) = \SI{58.44}{g.mol^{-1}}$ et
	      $M(\ce{Na}) = \SI{22.99}{g.mol^{-1}}$.
	      \smallbreak
	      Déterminer les concentrations molaire et massique en \ce{Na^{+}} dans la
	      solution.
	\item{} (Ap.TM1.9) On considère une seringue cylindrique de \SI{10}{cm} le long
	      et de \SI{2.5}{cm} de diamètre, contenant \SI{0.250}{g} de diazote de
	      masse
	      molaire $M({\rm N}_2) = \SI{28.01}{g.mol^{-1}}$ à la température $T =
		      \SI{20}{\degreeCelsius}$. Calculer la pression exercée par le diazote dans la seringue
	\item{} (Ap.TM1.12)
	      Soit un mélange de gaz nobles contenu dans une enceinte de \SI{100}{L} à
	      la
	      température $T = \SI{298.3}{K}$, avec \SI{2}{mol} d'hélium He,
	      \SI{5}{mol}
	      d'argon Ar et \SI{10}{mol} de néon Ne.
	      \smallbreak
	      Calculer la pression totale dans l'enceinte aussi que la partielle de
	      chacun des gaz.
	      \subsection(TM2){Transformation et équilibre chimique}
	\item Introduire et expliquer les différents types d'avancements (Df.TM2.4).
	      Refaire l'exemple du cours sur la combustion totale du méthane (TM2|II/B)
	      avec $n_{\ce{CH4},0} = \SI{2}{mol}$ et $n_{\ce{O2},0} = \SI{3}{mol}$. Que
	      vérifient des réactifs introduits en proportions stœchiométriques (Df.TM2.5,
	      Pt.TM2.1 et Dm.TM2.1)~?

	\item Donner les différentes expressions de l'activité d'un constituant selon
	      sa nature (Ipt.TM1.1), exprimer le quotient de réaction d'une
	      équation-bilan générale $0=\sum_i \nu_i{\ce{X}}_i$ ou
	      $\alpha_1{\ce{R}}_1 +
		      \alpha_2{\rm
			      R}_2 + … = \beta_1{\ce{P}}_1 + \beta_2{\ce{P}}_2 + …$ (Df.TM2.7) et
	      la constante d'équilibre associée (Df.TM2.8), et exprimer $Q_r$ pour les
	      réactions suivantes (Ap.TM2.2)~:
	      \begin{enumerate}
		      \item $\ce{2I^-_{\aqu} + {S_2O_8}^{2-}_{\aqu} = I2_{\aqu}
			            +2{SO_4}^{2-}_{\aqu}}$
		      \item $\ce{Ag^+_{\aqu} + Cl^-_{\aqu} = AgCl_{\sol}}$
		      \item $\ce{2FeCl3_{\gaz} = Fe2Cl6_{\gaz}}$
	      \end{enumerate}

	\item Réaction et avancement~: \textbf{définir le taux de conversion,
		      le coefficient de dissociation et le rendement} (Df.TM2.6), indiquer
	      ce qu'est la loi d'action de masse (Df.TM2.8) et refaire l'application
	      de réaction limitée (Ap.TM2.3) suivante. Soit la réaction de l'acide
	      éthanoïque avec l'eau~:
	      \[
		      \ce{CH3COOH_{\aqu} + H2O_{\liq} = CH3COO^{-}_{\aqu} + H3O^{+}_{\aqu}}
	      \]
	      de constante $K^\circ = \num{1.78e-5}$. On introduit
	      $c = \SI{1.0e-1}{mol.L^{-1}}$ d'acide éthanoïque et on note $V$ le
	      volume de solution. Déterminer la composition à l'état final.
	      \textbf{On utilisera le trinôme \xul{puis} la simplification}.

	\item Indiquer comment prévoir le sens d'évolution d'un système (Pt.TM2.4).
	      Soit la synthèse de l'ammoniac (Ap.TM2.4)~:
	      \smallbreak
	      \centersright{$\ce{N2\gaz{} + 3H2\gaz{} = 2NH3\gaz{}}$}{$K = \num{0.5}$}
	      \smallbreak
	      On introduit \SI{3}{mol} de diazote, \SI{5}{mol} de dihydrogène et
	      \SI{2}{mol} d'ammoniac sous une pression de \SI{200}{bars}.
	      \textbf{Dresser le tableau d'avancement}, puis
	      \textbf{déterminer les pressions partielles des gaz} et \textbf{indiquer
		      dans quel sens se produit la réaction}.

	      % \item Considérons la dissolution du chlorure de sodium, de masse molaire
	      %       $M(\ce{NaCl}) = \SI{58.44}{g.mol^{-1}}$~:
	      %       \smallbreak
	      %       \centersright{$\ce{NaCl\sol{} =
	      % 			      Na^{+}\aqu{} + Cl^{-}\aqu{}}$}{$K=33$}
	      %       \smallbreak
	      %       On introduit \SI{2.0}{g} de sel dans \SI{100}{mL} d'eau.
	      %       \textbf{Déterminer l'état d'équilibre}.
	      %
	      % \item Déterminer, à l'aide d'un tableau d'avancement, la \textbf{composition à
	      % 	      l'état final} de la réaction totale de la combustion de \SI{2.00}{mol}
	      %       d'éthanol dans l'air. On précise que les réactifs sont introduits dans
	      %       les proportions stœchiométriques, et que le dioxygène provient de
	      %       l'air (20\% \ce{O2} et 80\% \ce{N2} en mole). Quelle est la
	      %       \textbf{pression finale} pour $V = \SI{1.00}{m^3}$ et $T =
	      % 	      \SI{293}{K}$, $R = \SI{8.314}{J.K^{-1}.mol^{-1}}$~?
	      %       \[
	      % 	      \ce{C_2H_5OH\liq{} + 3O_2\gaz{} -> 2CO_2\gaz{} + 3H_2O\gaz{}}
	      %       \]
\end{enumerate}
\end{document}

