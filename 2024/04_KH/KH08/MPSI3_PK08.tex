\documentclass[a4paper, 12pt, final, garamond]{book}
\usepackage{cours-preambule}

\raggedbottom

\makeatletter
\renewcommand{\@chapapp}{Programme de kh\^olle -- semaine}
\makeatother

\begin{document}
\setcounter{chapter}{7}

\chapter{Du 18 au 21 novembre}

\section{Cours et exercices}
\subsection(TM1){Introduction aux transformations}
\begin{enumerate}[label=\Roman*]
	\item[b]{Vocabulaire général}~: atomes et molécules, classification par
	      composition, états de la matière et systèmes physico-chimiques,
	      transformations de la matière.
	\item[b]{Quantification des systèmes}~: mole, masse molaire, fractions
	      molaire et massique, masse volumique, concentrations molaire et
	      massique, dilution~; pression d'un gaz, modèle du gaz parfait, volume
	      molaire, pression partielle et loi de \textsc{Dalton}~; intensivité et
	      extensivité, activité d'un élément chimique.
\end{enumerate}

\subsection(TM2){Transformation et équilibre chimique}
\begin{enumerate}[label=\Roman*]
	\item[b]{Avancement d'une réaction}~: présentation, avancements molaire
	      et volumique, tableau d'avancement et $n\ind{tot, gaz}$, coefficients
	      stœchiométriques algébriques.
	\item[b]{États finaux d'un système chimique}~: types d'avancements~; réaction
	      totale~; réaction limitée~: quantifications de l'avancement (taux de
	      conversion, coefficient de dissociation, rendement), quotient de
	      réaction, constante d'équilibre et combinaison de réactions, réactions
	      quasi-nulles et quasi-totales.
	\item[b]{Évolution d'un système chimique}~: quotient réactionnel et
	      sens d'évolution, cas des ruptures d'équilibre, résumé pratique de
	      résolution.
\end{enumerate}

\section{Cours uniquement}
\subsection(TM3){Cinétique chimique}
\begin{enumerate}[label=\Roman*]
	\item[b]{Introduction}~: réactions lentes et rapides, méthodes de
	      suivi, exemple de suivi cinétique.
	\item[b]{Facteurs cinétiques}~: présentation, loi d'\textsc{Arrhénius} et
	      utilisations.
	\item[b]{Vitesse(s) de réaction}~: hypothèses de travail, vitesse de
	      réaction, vitesses de formation/disparition.
	\item[b]{Concentration et ordre de réaction}~: ordre d'une réaction,
	      ordre initial et courant, cas particulier des réactions simples loi de
	      \textsc{Van't Hoff}, cas particulier dégénérescence de l'ordre et
	      proportions stœchiométriques.
	      % \item[b]{Méthodes de résolution}~: temps de demi-réaction, ordres 0, 1
	      % et 2 par rapport à un réactif~: hypothèse de départ, unité de $k$,
	      % équation différentielle, résolution et $t_{1/2}$~; résumé méthodes en
	      % pratique et résumé.
	      % \item[b]{Méthodes de suivi cinétique expérimental}~: dosage par
	      % titrage et trempe chimique, dosage par étalonnage~: loi de
	      % \textsc{Beer-Lambert} et loi de \textsc{Kohlrausch}.
\end{enumerate}

\newpage

\section{Questions de cours possibles}
\begin{enumerate}
	\subsection(TM2){Transformation et équilibre chimique}
	\item Introduire et expliquer les différents types d'avancements (Df.TM2.4).
	      Refaire l'exemple du cours sur la combustion totale du méthane (TM2|II/B)
	      avec $n_{\ce{CH4},0} = \SI{2}{mol}$ et $n_{\ce{O2},0} = \SI{3}{mol}$. Que
	      vérifient des réactifs introduits en proportions stœchiométriques (Df.TM2.5,
	      Pt.TM2.1 et Dm.TM2.1)~?

	\item Donner les différentes expressions de l'activité d'un constituant selon
	      sa nature (Ipt.TM1.1), exprimer le quotient de réaction d'une
	      équation-bilan générale $0=\sum_i \nu_i{\ce{X}}_i$ ou
	      $\alpha_1{\ce{R}}_1 +
		      \alpha_2{\rm
			      R}_2 + … = \beta_1{\ce{P}}_1 + \beta_2{\ce{P}}_2 + …$ (Df.TM2.7) et
	      la constante d'équilibre associée (Df.TM2.8), et exprimer $Q_r$ pour les
	      réactions suivantes (Ap.TM2.2)~:
	      \smallbreak
	      a)
	      $\ce{2I^-_{_{\aqu}} + {S_2O_8}^{2-}_{_{\aqu}} = I2_{_{\aqu}} +2{SO_4}^{2-}_{_{\aqu}}}$
	      \hspace*{\fill}
	      b)
	      $\ce{Ag^+_{_{\aqu}} + Cl^-_{_{\aqu}} = AgCl_{_{\sol}}}$
	      \hspace*{\fill}
	      c)
	      $\ce{2FeCl3_{_{\gaz}} = Fe2Cl6_{_{\gaz}}}$
	      % \begin{enumerate}
	      %  \item $\ce{2I^-_{_{\aqu}} + {S_2O_8}^{2-}_{_{\aqu}} = I2_{_{\aqu}}
	      %         +2{SO_4}^{2-}_{_{\aqu}}}$
	      %  \item $\ce{Ag^+_{_{\aqu}} + Cl^-_{_{\aqu}} = AgCl_{_{\sol}}}$
	      %  \item $\ce{2FeCl3_{_{\gaz}} = Fe2Cl6_{_{\gaz}}}$
	      % \end{enumerate}

	\item Réaction et avancement~: \textbf{définir le taux de conversion,
		      le coefficient de dissociation et le rendement} (Df.TM2.6), indiquer
	      ce qu'est la loi d'action de masse (Df.TM2.8) et refaire l'application
	      de réaction limitée (Ap.TM2.3) suivante. Soit la réaction de l'acide
	      éthanoïque avec l'eau~:
	      \hspace*{\fill}
	      $\ce{CH3COOH_{\aqu} + H2O_{\liq} = CH3COO^{-}_{\aqu} + H3O^{+}_{\aqu}}$
	      \hspace*{\fill}
	      $K^\circ = \num{1.78e-5}$
	      \smallbreak
	      On introduit $c = \SI{1.0e-1}{mol.L^{-1}}$ d'acide éthanoïque dans
	      $V$ le volume de solution. Déterminer la composition à l'état final.
	      \textbf{On utilisera le trinôme \xul{puis} la simplification} (Pt.TM2.3).

	\item Indiquer comment prévoir le sens d'évolution d'un système (Pt.TM2.4).
	      Soit la synthèse de l'ammoniac (Ap.TM2.4)~:
	      \hspace*{80pt}
	      $\ce{N2_{\gaz} + 3H2_{\gaz} = 2NH3_{\gaz}}$
	      \hspace*{\fill}
	      $K = \num{0.5}$
	      \smallbreak
	      On introduit \SI{3}{mol} de diazote, \SI{5}{mol} de dihydrogène et
	      \SI{2}{mol} d'ammoniac sous une pression de \SI{200}{bars}.
	      \textbf{Dresser le tableau d'avancement}, puis
	      \textbf{déterminer les pressions partielles des gaz} et \textbf{indiquer
		      dans quel sens se produit la réaction}.

	\item{} (Ap.TM2.5) Considérons la dissolution du chlorure de sodium, de
	      masse molaire $M(\ce{NaCl}) = \SI{58.44}{g.mol^{-1}}$~:
	      \hspace*{\fill}
	      $\ce{NaCl_{\sol} = Na^{+}_{\aqu} + Cl^{-}_{\aqu}}$
	      \hspace*{\fill}
	      $K=33$
	      \smallbreak
	      On introduit \SI{2.0}{g} de sel dans \SI{100}{mL} d'eau.
	      \textbf{Déterminer l'état d'équilibre}.

	      % \item Déterminer, à l'aide d'un tableau d'avancement, la
	      % \textbf{composition
	      % à
	      % l'état final} de la réaction totale de la combustion de \SI{2.00}{mol}
	      % d'éthanol dans l'air. On précise que les réactifs sont introduits dans
	      % les proportions stœchiométriques, et que le dioxygène provient de
	      % l'air (20\% \ce{O2} et 80\% \ce{N2} en mole). Quelle est la
	      % \textbf{pression finale} pour $V = \SI{1.00}{m^3}$ et $T =
	      % \SI{293}{K}$, $R = \SI{8.314}{J.K^{-1}.mol^{-1}}$~?
	      %       \[
	      % 	      \ce{C_2H_5OH_{\liq} + 3O_2_{\gaz} -> 2CO_2_{\gaz} + 3H_2O_{\gaz}}
	      %       \]
	      \subsection(TM3){Cinétique chimique}
	\item Présenter les facteurs cinétiques d'une réaction (Pt.TM3.1) et le lien
	      des deux premiers avec la notion de choc efficace (Df.TM3.3). Énoncer alors
	      la loi d'\textsc{Arrhénius} (L.TM3.1) et représenter l'énergie
	      d'activation sur un graphe réactifs/produits (Fig.TM2). Indiquer
	      finalement les deux manières d'utiliser la loi d'\textsc{Arrhénius}
	      (Oti.TM3.1~: deux températures ou succession de températures).
	\item Définir la vitesse d'une réaction, de formation d'un produit, de
	      disparition d'un réactif (Df.TM3.5 et 6). Donner et \textbf{démontrer}
	      le lien entre vitesse de réaction et variation de la concentration d'un
	      constituant en fonction de son nombre stœchiométrique algébrique
	      (Pt.TM3.2 et Dm.TM3.1), puis exprimer $v$ en fonction des concentrations
	      (Ap.TM3.1) pour la réaction
	      \[
		      \ce{6H^{+}_{\aqu} + 5Br^{-}_{\aqu} + BrO3^{-}_{\aqu}
			      =
			      3Br2_{\aqu} + 2H2O_{\liq}}
	      \]
	\item Donner la loi de vitesse d'une réaction $a\ce{A} + b\ce{B} = c\ce{C}
		      + d\ce{D}$ admettant un ordre (Df.TM3.7). Expliquer en vos mots ce
	      qu'est une réaction simple (Df.TM3.9) et énoncer la loi de
	      \textsc{Van't Hoff} (L.TM3.2). Présenter puis démontrer ensuite
	      l'intérêt de la dégénérescence de l'ordre et des proportions
	      stœchiométriques (Ipt.TM3.3 et 4, Oti.TM3.2 et 3).
	      % \item À partir d'une loi de vitesse d'ordre \textbf{choisi par
	      % l'interrogataire} par rapport à un unique réactif $[\ce{A}]$, donner
	      % l'unité de $k$, démontrez l'équation différentielle vérifiée par
	      % $[\ce{A}]$ et la solution associée, indiquer quelle régression
	      % linéaire
	      % pourrait permettre de vérifier cette loi et donner le temps de
	      % demi-réaction.
\end{enumerate}
\end{document}
