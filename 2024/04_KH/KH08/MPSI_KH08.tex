\documentclass[a4paper, 11pt]{book}
\usepackage{/home/nora/Documents/Enseignement/Prepa/bpep/fichiers_utiles/preambule}

\makeatletter
\renewcommand{\@chapapp}{Kh\^olles MPSI3 -- semaine}
\makeatother
\renewcommand\thechapter{8}

% \toggletrue{corrige}

\begin{document}

\resetQ
\newpage

\chapter{Sujet 1\siCorrige{\!\!-- corrig\'e}}

\enonce{%
	On étudie en phase gazeuse l'équilibre de dimérisation de \ce{FeCl3}, de
	constante d'équilibre $K\degree(T)$ à une température $T$ donnée et
	d'équation-bilan
	\[\ce{2FeCl3\gaz{} = Fe2Cl6\gaz{}}\]

	La réaction se déroule sous une pression totale constante $p_{\tot} =
		2p\degree = \SI{2}{bars}$. À la température $T_1 = \SI{750}{K}$, la constante
	d'équilibre vaut $K\degree(T_1) = \num{20.8}$. Le système est maintenu à la
	température $T_1 = \SI{750}{K}$. Initialement le système contient $n_0$ moles
	de \ce{FeCl3} et de \ce{Fe2Cl6}. Soit $n_{\tot}$ la quantité totale de matière
	d'espèces dans le système.
}

\QR{%
	Exprimer la constante d'équilibre en fonction des pressions partielles
	des constituants à l'équilibre et de $p\degree$.
}{%
	On peut dresser le tableau d'avancement initial dans cette situation~:
	\begin{center}
		\def\rhgt{0.35}
		\centering
		\begin{tabularx}{.7\linewidth}{|l|c||YdY||Y|}
			\hline
			\multicolumn{2}{|c||}{
				$\xmathstrut{\rhgt}$
			\textbf{Équation}}    &
			$2\ce{FeCl_3\gaz{}}$  & $=$       &
			$\ce{Fe_2Cl_6\gaz{}}$ &
			$n_{\tot, gaz}$                     \\
			\hline
			$\xmathstrut{\rhgt}$
			Initial               & $\xi = 0$ &
			$n_0$                 & \vline    &
			$n_0$                 &
			$2n_0$                              \\
			\hline
		\end{tabularx}
	\end{center}
	Par la loi d'action des masses et les activités de constituants
	gazeux~:
	\[
		\boxed{
			K\degree = \frac{p_{\ce{Fe2Cl6}}p\degree}{p_{\ce{FeCl3}}{}^2}
		}
	\]
}
\QR{%
Exprimer le quotient de réaction $Q_r$ en fonction de la quantité de
matière de chacun des constituants, de la pression totale $p_{\tot}$
et de $p\degree$. Calculer la valeur initial $Q_{r,0}$ du quotient de
réaction.
}{%
Pour passer des pressions partielles aux quantités de matière, on
utilise la loi de \textsc{Dalton}~:
\begin{tcb}(rapp){Rappel~: loi de \textsc{Dalton}}
	Soit un mélange de gaz parfaits de pression $P$. Les pressions
	partielles $P_i$ de chaque constituant $\mathrm{X}_i$ s'exprime
	\[\boxed{P_i = x_iP}\]
	avec $x_i$ la fraction molaire du constituant~:
	\[\boxed{x_i = \frac{n_i}{n_{\tot}}}\]
\end{tcb}
On écrit donc
\[
	p_{\ce{Fe2Cl6}} = \frac{n_{\ce{Fe2Cl6}}}{n_{\tot}}\times p_{\tot}
	\qquad
	p_{\ce{FeCl3}} = \frac{n_{\ce{FeCl3}}}{n_{\tot}}\times p_{\tot}
\]

Pour simplifier l'écriture, on peut séparer les termes de pression
totale des termes de matière en comptant combien vont arriver «~en
haut~» et combien «~en bas~»~: 1 en haut contre 2 en bas, on se
retrouvera avec $p_{\tot}$ au dénominateur, ce qui est logique par
homogénéité vis-à-vis de $p\degree$ qui reste au numérateur. Comme
$n_{\tot}$ apparaît le même nombre de fois que $p_{\tot}$ mais
avec une puissance -1, on sait aussi qu'il doit se retrouver au
numérateur, là aussi logiquement pour avoir l'homogénéité vis-à-vis de
la quantité de matière. Ainsi,

\[\boxed{
	Q_r = \frac{n_{\ce{Fe2Cl6}}/\cancel{n_{\tot}}
	\times\bcancel{p_{\tot}}}
	{n_{\ce{FeCl3}}{}^2/n_{\tot}^{\cancel{2}}
	\times p_{\tot}^{\bcancel{2}}} p\degree
	= \frac{n_{\ce{Fe2Cl6}} n_{\tot}}
	{n_{\ce{FeCl3}}{}^2} \frac{p\degree}{p_{\tot}}
	}
\]
Avec $p_{\tot} = 2p\degree$ et $n_{\ce{Fe2Cl6}} = n_0 =
	n_{\ce{FeCl3}}$, on a $n_{\tot} = 2n_0$ (cf.\ tableau d'avancement),
d'où

\[
	Q_{r,0} = \frac{n_0\times2n_0}{n_0{}^2} \frac{1}{2}
	\Leftrightarrow
	\xul{Q_{r,0} = 1}
\]
}
\QR{%
	Le système est-il initialement à l'équilibre thermodynamique~?
	Justifier la réponse. Si le système n'est pas à l'équilibre, dans quel
	sens se produira l'évolution~?
}{%
	Le système serait à l'équilibre si $Q_{r,0} = K\degree$~; or, ici
	$Q_{r,0} \neq K\degree$, donc l'équilibre n'est pas atteint. De plus, $Q_{r,0}
		< K\degree$ donc le système évoluera dans le sens direct.
}
\enonce{%
	On considère désormais une enceinte indéformable, de température
	constante $T_1 = \SI{750}{K}$, initialement vide. On y introduit une
	quantité $n$ de \ce{FeCl3} gazeux et on laisse le système évoluer de
	telle sorte que la pression soit maintenu constante et égale à $p =
		2p\degree = \SI{2}{bars}$. On désigne par $\xi$ l'avancement de la
	réaction.
}
\QR{%
	Calculer à l'équilibre la valeur du rapport $z = \xi/n$.
}{%
	On dresse le tableau d'avancement pour effectuer un bilan de matière
	dans cette nouvelle situation~:
	\begin{center}
		\def\rhgt{0.35}
		\centering
		\begin{tabularx}{.7\linewidth}{|l|c||YdY||Y|}
			\hline
			\multicolumn{2}{|c||}{
				$\xmathstrut{\rhgt}$
			\textbf{Équation}}    &
			$2\ce{FeCl_3\gaz{}}$  & $=$           &
			$\ce{Fe_2Cl_6\gaz{}}$ &
			$n_{\tot, gaz}$                         \\
			\hline
			$\xmathstrut{\rhgt}$
			Initial               & $\xi = 0$     &
			$n$                   & \vline        &
			$0$                   &
			$n$                                     \\
			\hline
			$\xmathstrut{\rhgt}$
			Final                 & $\xi = \xi_f$ &
			$n-2\xi$              & \vline        &
			$\xi$                 &
			$n-\xi$                                 \\
			\hline
		\end{tabularx}
	\end{center}
	On reprend l'expression du quotient réactionnel initial en
	remplaçant les quantités de matière par leur expression selon $\xi$ pour
	déterminer l'avancement à l'équilibre, décrit par $K\degree$~:
	\begin{gather*}
		K\degree = \frac{\xi(n-\xi)}{(n-2\xi)^2}
		\underbrace{\frac{p\degree}{p_{\tot}}}_{=\frac{1}{2}}
		\Leftrightarrow
		K\degree = \underbrace{\cancel{\frac{n^2}{n^2}}}_{=1}
		\frac{\xi/n(1-\xi/n)}
		{\left( 1 - 2\xi/n \right)^2} \frac{1}{2}
	\end{gather*}
	Pour simplifier les calculs, posons $z = \frac{\xi}{n}$. L'équation
	précédente devient~:
	\begin{align*}
		K\degree                                                & = \frac{1}{2} \frac{z(1-z)}{\left( 1 - 2z \right)^2}
		\\\Lra
		2K\degree(1-2z)^2                                       & = z(1-z)
		\\\Lra
		2K\degree(1-4z + 4z^2)                                  & = z - z^2
		\\\Lra
		\Aboxed{z^2(8K\degree +1) - z(8K\degree +1) + 2K\degree & = 0}
	\end{align*}
	On trouve un polynôme du second degré. Soit $\Delta$ son discriminant~:
	\begin{align*}
		\Delta         & = \left( 8K\degree+1 \right)^{2+} -
		4\left(8K\degree+1\right)\times 2K\degree
		\\\Lra
		\Delta         & = \left( 8K\degree+1 \right)\left(
		\cancel{8K\degree}+1-\cancel{8K\degree} \right)
		\\\Lra
		\Aboxed{\Delta & = 8K\degree+1}
		\qav
		\left\{
		\begin{array}{rcl}
			K\degree & = & \num{20.8}
		\end{array}
		\right.                                              \\
		\makebox[0pt][l]{$\phantom{\AN}\xul{\phantom{\Delta = \num{167.4}}}$}
		\AN
		\Delta         & = \num{167.4}
	\end{align*}
	Les racines sont
	$\DS\left\{
		\begin{array}{rcl}
			z_1 & = & \num{0.54} \\
			z_2 & = & \num{0.46}
		\end{array}
		\right.$.
	\bigbreak
	Étant donné qu'on part de $\xi = 0$ et que $\xi$ augmente, la valeur que
	prendrait $z_{\equ}$ serait $z_{\equ} = \num{0.46}$. On doit cependant
	vérifier que cette valeur est bien possible, en déterminant $z_{\max}$~:
	pour cela, on résout $n-2\xi = 0$, ce qui donne $z_{\max} = \num{0.5}$.
	On a bien $z_{\equ} < z_{\max}$, donc \textbf{l'équilibre est atteint} et
	on a \xul{$\xi/n = \num{0.46}$}.
}

\resetQ
\newpage

\chapter{Sujet 2\siCorrige{\!\!-- corrigé}}

\subimport{/home/nora/Documents/Enseignement/Prepa/bpep/exercices/TD/utilisation_quotient_reaction/}{sujet.tex}

\resetQ
\newpage

\chapter{Sujet 3\siCorrige{\!\!-- corrigé}}

\subimport{/home/nora/Documents/Enseignement/Prepa/bpep/exercices/Colle/regime_transitoire_ordre2/}{sujet.tex}

\end{document}
