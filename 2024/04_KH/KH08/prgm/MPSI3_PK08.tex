\documentclass[a4paper, 12pt, final, garamond]{book}
\usepackage{cours-preambule}

\raggedbottom

\makeatletter
\renewcommand{\@chapapp}{Programme de kh\^olle -- semaine}
\makeatother

\begin{document}
\setcounter{chapter}{7}

\chapter{Du 20 au 23 novembre}

\section{Exercices seulement}
\section*{Électrocinétique ch.\ 4 -- Oscillateurs harmonique et amorti}

\begin{enumerate}[label=\Roman*, start=2]
	\bitem{Oscillateurs amortis}~:
	\begin{enumerate}[label=\Alph*]
		\bitem{Introduction amorti}~: exemple expérimental RLC, vocabulaire, équation
		différentielle générale, dimension de $Q$, équation caractéristique et
		régimes de solutions, présentation des solutions générales.
		\bitem{Oscillateur amorti RLC libre}~: présentation, bilan de puissance,
		équation différentielle, et solutions dans tous les régimes avec tracé et
		transitoire à 95\%~; extrapolation à $Q \ra \infty$ et $Q \ra 0$~; tracés
		dans l'espace des phases.
		\bitem{Ressort horizontal amorti libre}~:
		schéma et situation initiale, équation différentielle, analogie RLC-ressort
		amorti, bilan de puissance.
	\end{enumerate}
\end{enumerate}

\section{Cours et exercices}
\section*{Chimie chapitre 1 -- Introduction}
\begin{enumerate}[label=\Roman*]
	\bitem{Vocabulaire général}~: atomes et molécules, classification par
	composition, états de la matière et systèmes physico-chimiques,
	transformations de la matière.
	\bitem{Quantification des systèmes}~: mole, masse molaire, fractions
	molaire et massique, masse volumique, concentrations molaire et
	massique, dilution~; pression d'un gaz, modèle du gaz parfait, volume
	molaire, pression partielle et loi de \textsc{Dalton}, intensivité et
	extensivité, activité d'un élément chimique.
\end{enumerate}

\section*{Chimie chapitre 2 -- Transformation et équilibre chimique}
\begin{enumerate}[label=\Roman*]
	\bitem{Avancement d'une réaction}~: présentation, avancements molaire
	et volumique, tableau d'avancement, coefficients stœchiométriques
	algébriques.
	\bitem{États final et d'équilibre d'un système chimique}~: réactions
	totales et limitées et exercice d'application, quantifications de
	l'avancement~: taux de conversion, coefficient de dissociation,
	rendement~; quotient de réaction et exercice d'application, constante
	d'équilibre et exercice d'application, réactions quasi-nulles et
	quasi-totales.
	\bitem{Évolution d'un système chimique}~: quotient réactionnel et
	évolution et exercice d'application, cas des ruptures d'équilibre,
	résumé pratique de résolution.
\end{enumerate}

\section{Questions de cours possibles}
\begin{enumerate}
	\item[] \textbf{Chapitre 1}
	\item L'air est constitué, en quantité de matière, à 80\% de diazote \ce{N2} et
	      à 20\% de dioxygène \ce{O2}.
	      \smallbreak
	      On a
	      $M(\ce{N_2}) = \SI{28.0}{g.mol^{-1}}$ et
	      $M(\ce{O_2}) = \SI{32.0}{g.mol^{-1}}$.
	      \smallbreak
	      En déduire les \textbf{fractions molaires} puis les \textbf{fractions
		      massiques}.
	\item On dissout une masse $m = \SI{2.00}{g}$ de sel NaCl$\sol$ dans $V =
		      \SI{100}{mL}$ d'eau.
	      \smallbreak
	      On donne
	      $M(\ce{NaCl}) = \SI{58.44}{g.mol^{-1}}$ et
	      $M(\ce{Na}) = \SI{22.99}{g.mol^{-1}}$.
	      \smallbreak
	      Déterminer les \textbf{concentrations molaire et massique} en
	      \ce{Na^{+}} dans la solution.
	\item On considère une seringue cylindrique de \SI{10}{cm} le long et de
	      \SI{2.5}{cm} de diamètre, contenant \SI{0.250}{g} de diazote de masse
	      molaire $M({\ce{N}}_2) = \SI{28.01}{g.mol^{-1}}$ à la température $T =
		      \SI{20}{\degreeCelsius}$. \textbf{Calculer la pression exercée par le
		      diazote dans la seringue.}
	\item Savoir ajuster l'équation suivante \textbf{\underline{et}} une autre
	      équation proposée par l'interrogataire~:
	      \begin{gather*}
		      \ce{\ldots I^-\aqu{} + \ldots Cr2O7^{2-}\aqu{} + \ldots H^{+}\aqu}
		      =
		      \ce{\ldots Cr^{+3}\aqu{} + \ldots I2\gaz{} + \ldots H2O\liq{}}
	      \end{gather*}
	\item[] \textbf{Chapitre 2}
	\item Réaction et avancement~: \textbf{définir le taux de conversion,
		      le coefficient de dissociation et le rendement} et refaire l'exemple
	      du cours sur la combustion totale du méthane $\ce{CH4\gaz{} +
			      2O2\gaz{} \rightarrow CO2\gaz{} + 2H2O\gaz{}}$ avec $n_{\ce{CH4}}^0 =
		      \SI{2}{mol}$ et $n_{\ce{O2}}^0 = \SI{3}{mol}$.
	\item Donner les différentes expressions de l'activité d'un constituant
	      selon sa nature, exprimer le quotient de réaction d'une équation-bilan
	      générale $0=\sum_i \nu_i{\ce{X}}_i$ ou $\alpha_1{\ce{R}}_1 + \alpha_2{\rm
			      R}_2 + … = \beta_1{\ce{P}}_1 + \beta_2{\ce{P}}_2 + …$ et la constante
	      d'équilibre associée, et exprimer $Q_r$ pour les réactions~:
	      \begin{enumerate}
		      \item $\ce{2I^-\aqu{} + S2O8^{2-}\aqu{} = I2\aqu{} +2SO4^{2-}\aqu}$
		      \item $\ce{Ag+\aqu{} + Cl^-\aqu{} = AgCl\sol}$
		      \item $\ce{2FeCl3\gaz{} = Fe2Cl6\gaz{}}$
	      \end{enumerate}
	\item Soit la réaction de l'acide éthanoïque avec l'eau~:
	      \[\ce{CH3COOH\aqu{} + H2O\liq{} = CH3COO^{-}\aqu{} + H3O^{+}\aqu{}}\]
	      de constante $K = \num{1.78e-5}$. On introduit $c =
		      \SI{1.0e-1}{mol.L^{-1}}$ d'acide éthanoïque et on note $V$ le volume de
	      solution. \textbf{Déterminer la composition à l'état final}.

	\item Indiquer comment prévoir le sens d'évolution d'un système.
	      Soit la synthèse de l'ammoniac~:
	      \smallbreak
	      \centersright{$\ce{N2\gaz{} + 3H2\gaz{} = 2NH3\gaz{}}$}{$K = \num{0.5}$}
	      \smallbreak
	      On introduit \SI{3}{mol} de diazote, \SI{5}{mol} de dihydrogène et
	      \SI{2}{mol} d'ammoniac sous une pression de \SI{200}{bars}.
	      \textbf{Déterminer les pressions partielles des gaz} et \textbf{indiquer
		      dans quel sens se produit la réaction}.
	\item Considérons la dissolution du chlorure de sodium, de masse molaire
	      $M(\ce{NaCl}) = \SI{58.44}{g.mol^{-1}}$~:
	      \smallbreak
	      \centersright{$\ce{NaCl\sol{} =
				      Na^{+}\aqu{} + Cl^{-}\aqu{}}$}{$K=33$}
	      \smallbreak
	      On introduit \SI{2.0}{g} de sel dans \SI{100}{mL} d'eau.
	      \textbf{Déterminer l'état d'équilibre}.

	\item Déterminer, à l'aide d'un tableau d'avancement, la \textbf{composition à
		      l'état final} de la réaction totale de la combustion de \SI{2.00}{mol}
	      d'éthanol dans l'air. On précise que les réactifs sont introduits dans
	      les proportions stœchiométriques, et que le dioxygène provient de
	      l'air (20\% \ce{O2} et 80\% \ce{N2} en mole). Quelle est la
	      \textbf{pression finale} pour $V = \SI{1.00}{m^3}$ et $T =
		      \SI{293}{K}$, $R = \SI{8.314}{J.K^{-1}.mol^{-1}}$~?
	      \[
		      \ce{C_2H_5OH\liq{} + 3O_2\gaz{} -> 2CO_2\gaz{} + 3H_2O\gaz{}}
	      \]
\end{enumerate}
\end{document}
