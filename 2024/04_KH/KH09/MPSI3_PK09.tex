\documentclass[a4paper, 12pt, final, garamond]{book}
\usepackage{cours-preambule}

\raggedbottom

\makeatletter
\renewcommand{\@chapapp}{Programme de kh\^olle -- semaine}
\makeatother

\begin{document}
\setcounter{chapter}{8}

\chapter{Du 25 au 28 novembre}

\section{Cours et exercices}
% \subsection(TM2){Transformation et équilibre chimique}
% \begin{enumerate}[label=\Roman*]
% 	\item[b]{Avancement d'une réaction}~: présentation, avancements molaire
% 	      et volumique, tableau d'avancement et $n\ind{tot, gaz}$, coefficients
% 	      stœchiométriques algébriques.
% 	\item[b]{États finaux d'un système chimique}~: types d'avancements~; réaction
% 	      totale~; réaction limitée~: quantifications de l'avancement (taux de
% 	      conversion, coefficient de dissociation, rendement), quotient de
% 	      réaction, constante d'équilibre et combinaison de réactions, réactions
% 	      quasi-nulles et quasi-totales.
% 	\item[b]{Évolution d'un système chimique}~: quotient réactionnel et
% 	      sens d'évolution, cas des ruptures d'équilibre, résumé pratique de
% 	      résolution.
% \end{enumerate}

\subsection(TM3){Cinétique chimique}
\begin{enumerate}[label=\Roman*]
	\item[b]{Introduction}~: réactions lentes et rapides, méthodes de
	      suivi, exemple de suivi cinétique.
	\item[b]{Facteurs cinétiques}~: présentation, loi d'\textsc{Arrhénius} et
	      utilisations.
	\item[b]{Vitesse(s) de réaction}~: hypothèses de travail, vitesse de
	      réaction, vitesses de formation/disparition.
	\item[b]{Concentration et ordre de réaction}~: ordre d'une réaction,
	      ordre initial et courant, cas particulier des réactions simples loi de
	      \textsc{Van't Hoff}, cas particulier dégénérescence de l'ordre et
	      proportions stœchiométriques.
	\item[b]{Méthodes de résolution}~: méthode différentielle, méthode
	      intégrale et méthode du temps de demi-réaction pour les ordres 0, 1
	      et 2 par rapport à un réactif.
	\item[b]{Méthodes de suivi cinétique expérimental}~: dosage par
	      titrage et trempe chimique, dosage par étalonnage~: loi de
	      \textsc{Beer-Lambert} et loi de \textsc{Kohlrausch}.
\end{enumerate}

\section{Cours uniquement}
\subsection(E6){Circuits électriques en RSF}

\begin{enumerate}[label=\Roman*]
	\item[b]{Présentation du régime forcé}~: définition, réponse d'un système en
	      RSF (même pulsation), notion de signaux périodiques (période, moyenne,
	      signal efficace), passage en complexes~: outils mathématiques.
	\item[b]{Circuits électriques en RSF}~: lois des l'électrocinétique (LdN,
	      LdM), exemple RC série en RSF (\textbf{uniquement depuis l'équation
		      différentielle réelle})~: amplitude complexe, module et argument.
\end{enumerate}

\newpage

\section{Questions de cours possibles}
\begin{enumerate}
	\subsection(TM3){Cinétique chimique}
	\item Présenter les facteurs cinétiques d'une réaction (Pt.TM3.1) et le lien
	      des deux premiers avec la notion de choc efficace (Df.TM3.3). Énoncer alors
	      la loi d'\textsc{Arrhénius} (L.TM3.1) et représenter l'énergie
	      d'activation sur un graphe réactifs/produits (Fig.TM2). Indiquer
	      finalement les deux manières d'utiliser la loi d'\textsc{Arrhénius}
	      (Oti.TM3.1~: deux températures ou succession de températures).
	\item Définir la vitesse d'une réaction, de formation d'un produit, de
	      disparition d'un réactif (Df.TM3.5 et 6). Donner et \textbf{démontrer}
	      le lien entre vitesse de réaction et variation de la concentration d'un
	      constituant en fonction de son nombre stœchiométrique algébrique
	      (Pt.TM3.2 et Dm.TM3.1), puis exprimer $v$ en fonction des concentrations
	      (Ap.TM3.1) pour la réaction
	      \[
		      \ce{6H^{+}_{\aqu} + 5Br^{-}_{\aqu} + BrO3^{-}_{\aqu}
			      =
			      3Br2_{\aqu} + 3H2O_{\liq}}
	      \]
	\item Donner la loi de vitesse d'une réaction $a\ce{A} + b\ce{B} = c\ce{C}
		      + d\ce{D}$ admettant un ordre (Df.TM3.7). Expliquer en vos mots ce
	      qu'est une réaction simple (Df.TM3.9) et énoncer la loi de
	      \textsc{Van't Hoff} (L.TM3.2). Présenter puis démontrer ensuite
	      l'intérêt de la dégénérescence de l'ordre et des proportions
	      stœchiométriques (Ipt.TM3.3 et 4, Oti.TM3.2 et 3).
	\item \textbf{Méthode intégrale}~: à partir d'une loi de vitesse d'ordre
	      \textbf{choisi par l'interrogataire} par rapport à un unique réactif
	      $[\ce{A}]$, donner l'unité de $k$, démontrez l'équation différentielle
	      vérifiée par $[\ce{A}]$ et la solution associée, indiquer quelle
	      régression linéaire pourrait permettre de vérifier cette loi et donner
	      le temps de demi-réaction.
	      \subsection(E6){Circuits électriques en RSF}
	\item Présenter ce qu'on appelle le régime sinusoïdal forcé (Df.E6.1 et 2) et
	      la forme de la réponse d'un système en RSF (Pt.E6.1) et quel est ainsi
	      l'objectif du chapitre (Ipt.E6.1).
	\item Justifier pourquoi un signal sinusoïdal a une moyenne nulle (pas de
	      calcul nécessaire), et présenter ce qu'est la valeur efficace d'un signal
	      périodique. Démontrer sa valeur pour un signal $s(t) = A \cos(\wt)$.
	\item Présenter le passage en complexes ainsi que la manière de représenter
	      une amplitude complexe (Oti.E6.1), et l'intérêt que cela comporte pour la
	      dérivation et l'intégration (Rap.E6.2). Application au circuit RC série
	      \textbf{en partant de l'équation différentielle
		      réelle}~:
	      amplitude complexe, module et argument (Dm.E6.2). Expliquer, sans
	      justification, pourquoi on peut prendre l'arctangente de la tangente.
	\item{} (At.E6.3) Indiquer dans quel intervalle s'expriment les angles en
	      physique. Expliquer précisément comment déterminer l'argument d'une
	      amplitude complexe $\Yu$ en connaissant sa partie réelle et sa partie
	      imaginaire~: quelle fonction trigonométrique applique-t-on à
	      $\arg*{\Yu}$~? comment remonter à $\arg*{\Yu}$ ensuite~? quelles sont
	      alors les précautions à appliquer~? Au moins un schéma est attendu.
\end{enumerate}
\end{document}
