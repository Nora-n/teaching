\documentclass[a4paper, 12pt, final, garamond]{book}
\usepackage{cours-preambule}

\raggedbottom

\makeatletter
\renewcommand{\@chapapp}{Programme de kh\^olle -- semaine}
\makeatother

\begin{document}
\setcounter{chapter}{9}

\chapter{Du 2 au 5 décembre}

\section{Cours et exercices}

\subsection(TM3){Cinétique chimique}
\begin{enumerate}[label=\Roman*]
	\item[b]{Introduction}~: réactions lentes et rapides, méthodes de
	      suivi, exemple de suivi cinétique.
	\item[b]{Facteurs cinétiques}~: présentation, loi d'\textsc{Arrhénius} et
	      utilisations.
	\item[b]{Vitesse(s) de réaction}~: hypothèses de travail, vitesse de
	      réaction, vitesses de formation/disparition.
	\item[b]{Concentration et ordre de réaction}~: ordre d'une réaction,
	      ordre initial et courant, cas particulier des réactions simples loi de
	      \textsc{Van't Hoff}, cas particulier dégénérescence de l'ordre et
	      proportions stœchiométriques.
	\item[b]{Méthodes de résolution}~: méthode différentielle, méthode
	      intégrale et méthode du temps de demi-réaction pour les ordres 0, 1
	      et 2 par rapport à un réactif.
	\item[b]{Méthodes de suivi cinétique expérimental}~: dosage par
	      titrage et trempe chimique, dosage par étalonnage~: loi de
	      \textsc{Beer-Lambert} et loi de \textsc{Kohlrausch}.
\end{enumerate}

\subsection(E6){Circuits électriques en RSF}
\begin{enumerate}[label=\Roman*]
	\item[b]{Présentation du régime forcé}~: définition, réponse d'un système en
	      RSF (même pulsation), notion de signaux périodiques (période, moyenne,
	      signal efficace), passage en complexes~: outils mathématiques.
	\item[b]{Circuits électriques en RSF}~: lois des l'électrocinétique (LdN,
	      LdM), exemple RC série en RSF (\textbf{uniquement depuis l'équation
		      différentielle réelle})~: amplitude complexe, module et argument~;
	      impédances, admittances, associations et ponts diviseurs, exercices et
	      résumé.
	\item[b]{Mesure de déphasages}~: définition, lecture, valeurs particulières.
\end{enumerate}

\section{Cours uniquement}
\subsection(E7){Oscillateurs en RSF}
\begin{enumerate}[label=\Roman*]
	\item[b]{Exemple RLC série}~: présentation, étude de l'intensité et de la
	      tension (amplitude complexe, amplitude réelle et phase, comportements à la
	      résonance, bande passante pour l'intensité).
	\item[b]{Exemple mécanique}~: présentation ressort horizontal amorti.
\end{enumerate}

\vspace{\fill}

\begin{center}
	\begin{framed}
		\bfseries
		\large
		Tout exercice de circuits en RSF, même ceux donnant une amplitude complexe
		résonante, mais aucune \textit{étude} de résonance. Pas d'oscillateurs
		mécaniques en exercices.
	\end{framed}
\end{center}

\vspace{\fill}

\newpage

\section{Questions de cours possibles}
\begin{enumerate}
	\subsection(TM3){Cinétique chimique}
	\item \textbf{Méthode intégrale}~: à partir d'une loi de vitesse d'ordre
	      \textbf{choisi par l'interrogataire} par rapport à un unique réactif
	      $[\ce{A}]$, donner l'unité de $k$, démontrez l'équation différentielle
	      vérifiée par $[\ce{A}]$ et la solution associée, indiquer quelle
	      régression linéaire pourrait permettre de vérifier cette loi et donner
	      le temps de demi-réaction.
	      \vspace{-15pt}
	      \subsection(E6){Circuits électriques en RSF}
	\item[s]"1" Présenter ce qu'on appelle le régime sinusoïdal forcé (Df.E6.1 et
	      2) et la forme de la réponse d'un système en RSF (Pt.E6.1) et quel est ainsi
	      l'objectif du chapitre (Ipt.E6.1). Présenter alors le passage en complexes
	      ainsi que la manière de représenter une amplitude complexe (Oti.E6.1), et
	      l'intérêt que cela comporte pour la dérivation et l'intégration (Rap.E6.2).

	\item{} (At.E6.3) Indiquer dans quel intervalle s'expriment les angles en
	      physique. Expliquer précisément comment déterminer l'argument d'une
	      amplitude complexe $\Yu$ en connaissant sa partie réelle et sa partie
	      imaginaire~: quelle fonction trigonométrique applique-t-on à
	      $\arg*{\Yu}$~? comment remonter à $\arg*{\Yu}$ ensuite~? quelles sont
	      alors les précautions à appliquer~? Au moins un schéma est attendu.
	      % \item Justifier pourquoi un signal sinusoïdal a une moyenne nulle (pas de
	      % calcul nécessaire), et présenter ce qu'est la valeur efficace d'un signal
	      % périodique. Démontrer sa valeur pour un signal $s(t) = A \cos(\wt)$.
	\item Indiquer comment se définit une impédance complexe (Df.E6.7), puis
	      donner \textbf{et démontrer} les impédances complexes d'une résistance,
	      d'une bobine et d'un condensateur (Df.E6.9), indiquer et justifier leurs
	      comportements limites si elles en ont (Pt.E6.7).

	\item Donner (Ipt.E6.3) \textbf{et démontrer} (Dm.E\xul{2}.1 et 2) les
	      associations en série et en parallèle d'impédances complexes, \textbf{ainsi}
	      que les relations de ponts diviseurs de tension et de courant en complexes
	      (Dm.E\xul{2}.3 et 4). Éventuellement déterminer l'impédance équivalente
	      d'une association donnée par l'examinataire, si le temps le permet.

	\item Circuit RC série en RSF~: présenter le système (Df.E6.6), déterminer
	      l'amplitude complexe sur la tension du condensateur ainsi que son amplitude
	      réelle et sa phase (Dm.E6.2) \textbf{à partir d'un pont diviseur de
		      tension} (voir exemples du TDE6 et exemple d'application du cours).
	      \vspace{-15pt}
	      \subsection(E7){Oscillateurs en RSF}
	\item Étude de la résonance en intensité pour le circuit RLC série en RSF~:
	      présenter le système réel et complexe (Df.E7.1), établir l'expression de
	      $\Iu(x)$ (Dm.E7.1), donner son amplitude réelle $I(x)$ et sa phase $\f_i(x)$
	      en justifiant son domaine d'appartenance (Dm.E7.2). Déterminer sa pulsation
	      de résonance et sa phase à la résonance (Dm.E7.3) et tracer $I(x)$ et
	      $\f_i(x)$ (Ipt.E7.1).
	\item[s]"2" À partir de $I(x) = \frac{E_0/R}{\sqrt{1+Q^2 \pa{x-\frac{1}{x}}^2}}$,
	      déterminer les pulsations réduites de coupure $x_1$ et $x_2$ donnant les
	      limites de la bande passante, et exprimer la largeur de la bande
	      passante en fonction du facteur de qualité (Dm.E7.4).
	\item Étude de la résonance en tension pour le circuit RLC série en RSF~:
	      présenter le système réel et complexe (Df.E7.1), établir l'expression de
	      l'amplitude complexe $\Uu(x)$ (Dm.E7.5), donner son amplitude réelle $U(x)$
	      et sa phase $\f_u(x)$ en justifiant son domaine d'appartenance (Dm.E7.6).
	      Les tracer (Ipt.E7.2).
	\item[s]"3" À partir de $U(x) = \frac{E_0}{\sqrt{\pa{1-x^2}^2 +
				      \pa{\frac{x}{Q}}^2}}$, démontrer la condition de résonance ainsi
	      que l'amplitude à la résonance (Dm.E7.7).
\end{enumerate}

\end{document}
