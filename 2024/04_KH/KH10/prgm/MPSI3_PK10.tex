\documentclass[a4paper, 12pt, final, garamond]{book}
\usepackage{cours-preambule}

\raggedbottom

\makeatletter
\renewcommand{\@chapapp}{Programme de kh\^olle -- semaine}
\makeatother

\begin{document}
\setcounter{chapter}{9}

\chapter{Du 4 au 7 d\'ecembre}

\section{Cours et exercices}

\section*{Chimie chapitre 3 -- Cinétique chimique}
\begin{enumerate}[label=\Roman*]
	\bitem{Introduction}~: réactions lentes et rapides, méthodes de
	suivi, exemple de suivi cinétique, facteurs cinétiques.
	\bitem{Vitesse(s) de réaction}~: hypothèses de travail, vitesse de
	réaction, vitesses de formation/disparition.
	\bitem{Concentration et ordre de réaction}~: ordre d'une réaction,
	ordre initial et courant, cas particulier des réactions simples loi de
	\textsc{Van't Hoff}, cas particulier dégénérescence de l'ordre et
	proportions stœchiométriques.
	\bitem{Méthodes de résolution}~: temps de demi-réaction, ordres 0, 1
	et 2 par rapport à un réactif~: hypothèse de départ, unité de $k$,
	équation différentielle, résolution et $t_{1/2}$~; résumé méthodes en
	pratique et résumé.
	\bitem{Température et loi d'\textsc{Arrhénius}}~: phénoménologie, expression
	de $k(T)$, exemple d'utilisation pour deux températures et pour une
	sucession de températures
	\bitem{Méthodes de suivi cinétique expérimental}~: dosage par
	titrage et trempe chimique, dosage par étalonnage~: loi de
	\textsc{Beer-Lambert} et loi de \textsc{Kohlrausch}.
\end{enumerate}

\section*{Électrocinétique chapitre 5 -- Circuits électriques en RSF}
\begin{enumerate}[label=\Roman*]
	\bitem{Circuit RC série en RSF}~: présentation, réponse d'un système
	en RSF, passage en complexes.
	\bitem{Circuits électriques en RSF}~: lois de l'électrocinétique (loi
	des nœuds, loi des mailles), impédance et admittance complexes
	(définition, impédances de bases, comportements limites), associations
	d'impédances et ponts diviseurs.
	\bitem{Mesure de déphasages}~: définition, valeurs particulières,
	lecture d'un déphasage, déphasage des impédances.
\end{enumerate}

\section{Cours uniquement}
\section*{Électrocinétique chapitre 6 -- Oscillateurs en RSF}
\begin{enumerate}[label=\Roman*]
	\bitem{Introduction}~: rappel oscillateurs, méthode des complexes,
	notion de résonance et bande passante.
	\bitem{Exemple électrique~: circuit RLC série en RSF}~: présentation,
	étude de l'intensité (amplitude complexe, amplitude réelle et maximum,
	phase).
\end{enumerate}

\section{Questions de cours possibles}
\subsection{C3~: cinétique chimique}
\begin{enumerate}
	%\nitem{9}%
	\item Définir la vitesse d'une réaction, de formation d'un produit, de
	      disparition d'un réactif et le lien entre vitesse de réaction et
	      variation de la concentration d'un constituant en fonction de son nombre
	      stœchiométrique algébrique, puis exprimer $v$ en fonction des
	      concentrations pour la réaction
	      \[
		      \ce{6H^{+}\aqu{} + 5Br^{-}\aqu{} + BrO3^{-}\aqu{}
			      =
			      3Br2\aqu{} + 2H2O\liq{}}
	      \]
	      %\nitem{9}%
	\item Donner la loi de vitesse d'une réaction $a\ce{A} + b\ce{B} = c\ce{C}
		      + d\ce{D}$ admettant un ordre, la loi de vitesse de la même réaction
	      si elle est simple, montrer l'intérêt de la dégénérescence de l'ordre
	      et des proportions stœchiométriques.
	      %\nitem{9}%
	\item À partir d'une loi de vitesse d'ordre \textbf{choisi par
		      l'interrogataire} par rapport à un unique réactif $[\ce{A}]$, donner
	      l'unité de $k$, démontrez l'équation différentielle vérifiée par
	      $[\ce{A}]$ et la solution associée, indiquer quelle régression linéaire
	      pourrait permettre de vérifier cette loi et donner le temps de
	      demi-réaction.
	      %\nitem{9}%
	\item Énoncer la loi d'\textsc{Arrhénius}, indiquer une manière d'utiliser
	      deux constantes de vitesse à deux températures différentes pour
	      déterminer l'énergie d'activation, et une autre manière d'utiliser
	      plusieurs constantes de vitesse à différentes températures pour
	      déterminer l'énergie d'activation.
\end{enumerate}
\subsection{E5~: circuits en RSF}
\begin{enumerate}[resume]
	%\nitem{9}%
	\item Méthode des complexes en RSF~: donner la forme de réponse d'un système
	      en RSF, les relations entre les grandeurs réelle et complexe associée,
	      l'intérêt pour la dérivation et le lien entre une équation
	      différentielle réelle et l'équation algébrique complexe associée.
	      %\nitem{9}%
	\item Rappeler comment se définit une impédance complexe, puis donner
	      \textbf{et démontrer} les impédances complexes d'une résistance, d'une
	      bobine et d'un condensateur, indiquer et justifier leurs comportements
	      limites si elles en ont.
	      %\nitem{9}%
	\item Donner \textbf{et démontrer} les associations en série et en parallèle
	      d'impédances complexes, et déterminer l'impédance équivalente d'une
	      association donnée par l'examinataire.
	      %\nitem{9}%
	\item Donner \textbf{et démontrer} les relations des ponts diviseur de tension
	      et diviseur de courant en complexes.
	      %\nitem{9}%
	\item Circuit RC série en RSF~: présenter le système réel, le système en
	      complexes, déterminer l'amplitude complexe sur la tension du
	      condensateur ainsi que son amplitude réelle et sa phase.
\end{enumerate}

\subsection{E6~: Oscillateurs en RSF}
\begin{enumerate}[resume]
	%\nitem{10}%
	\item Étude de la résonance en intensité pour le circuit RLC série en RSF~:
	      établir l'expression de $\Iu$, donner son amplitude réelle $I(\w)$.
	      Déterminer sa pulsation de résonance et tracer $I(\w)$. Étudier sa
	      phase.
\end{enumerate}

\end{document}
