\documentclass[a4paper, 12pt, final, garamond]{book}
\usepackage{cours-preambule}
\graphicspath{{../figures/}}

\raggedbottom

\makeatletter
\renewcommand{\@chapapp}{Programme de kh\^olle -- semaine}
\makeatother

\begin{document}
\setcounter{chapter}{10}

\chapter{Du 9 au 12 d\'ecembre}

\section{Cours et exercices}
\subsection(E6){Circuits électriques en RSF}
\begin{enumerate}[label=\Roman*]
	\item[b]{Présentation du régime forcé}~: définition, réponse d'un système en
	      RSF (même pulsation), notion de signaux périodiques (période, moyenne,
	      signal efficace), passage en complexes~: outils mathématiques.
	\item[b]{Circuits électriques en RSF}~: lois des l'électrocinétique (LdN,
	      LdM), exemple RC série en RSF (\textbf{uniquement depuis l'équation
		      différentielle réelle})~: amplitude complexe, module et argument~;
	      impédances, admittances, associations et ponts diviseurs, exercices et
	      résumé.
	\item[b]{Mesure de déphasages}~: définition, lecture, valeurs particulières.
\end{enumerate}


\subsection(E7){Oscillateurs en RSF}
\begin{enumerate}[label=\Roman*]
	\item[b]{Exemple RLC série}~: présentation, étude de l'intensité et de la
	      tension (amplitude complexe, amplitude réelle et phase, comportements à la
	      résonance, bande passante pour l'intensité).
	\item[b]{Exemple mécanique}~: présentation ressort horizontal amorti, étude de
	      l'élongation et de la vitesse (amplitude complexe pour les deux, amplitude
	      réelle et phase pour l'élongation, comportements à la résonance).
	\item[b]{Synthèse}~: résumé et comportements à $\w_0$.
\end{enumerate}

\section{Cours uniquement}
\subsection(E8){Filtrage linéaire}
\begin{enumerate}[label=\Roman*]
	\item[b]{Décomposition en série de Fourier}~: théorème de
	      \textsc{Fourier}, analyse spectrale, relation de \textsc{Parseval}.
	\item[b]{Filtrage linéaire}~: introduction, notion de filtre et de fonction de
	      transfert, exemple filtre RC sur C~; effet d'un filtre sur un signal
	      périodique composé.
	\item[b]{Description d'un filtre}~: gain et gain en décibel, échelle
	      logarithmique~; lien entre amplitude et gain en décibel, application RC sur
	      C~; diagramme de Bode, définition et exemple, diagramme asymptotique et
	      application RC sur C~; lecture d'un diagramme de Bode~; types de filtres
	      (moyenneur, intégrateur, dérivateur).
\end{enumerate}

\newpage
\section{Questions de cours possibles}
\begin{enumerate}
	\subsection(E6){Circuits électriques en RSF}
	% \item[s]"1" Présenter ce qu'on appelle le régime sinusoïdal forcé (Df.E6.1 et
	%       2) et la forme de la réponse d'un système en RSF (Pt.E6.1) et quel est ainsi
	%       l'objectif du chapitre (Ipt.E6.1). Présenter alors le passage en complexes
	%       ainsi que la manière de représenter une amplitude complexe (Oti.E6.1), et
	%       l'intérêt que cela comporte pour la dérivation et l'intégration (Rap.E6.2).

	% \item{} (At.E6.3) Indiquer dans quel intervalle s'expriment les angles en
	% physique. Expliquer précisément comment déterminer l'argument d'une
	% amplitude complexe $\Yu$ en connaissant sa partie réelle et sa partie
	% imaginaire~: quelle fonction trigonométrique applique-t-on à
	% $\arg*{\Yu}$~? comment remonter à $\arg*{\Yu}$ ensuite~? quelles sont
	% alors les précautions à appliquer~? Au moins un schéma est attendu.
	\item Justifier pourquoi un signal sinusoïdal a une moyenne nulle (pas de
	      calcul nécessaire). Présenter ce qu'est la valeur efficace d'un signal
	      périodique et son lien avec l'énergie (Df.E6.5). Démontrer sa valeur
	      pour un signal $s(t) = A \cos(\wt)$ (Ap.E6.3).

	\item Indiquer comment se définit une impédance complexe (Df.E6.7), puis
	      donner \textbf{et démontrer} les impédances complexes d'une résistance,
	      d'une bobine et d'un condensateur (Df.E6.9), indiquer et justifier leurs
	      comportements limites si elles en ont (Pt.E6.7).

	\item Donner (Ipt.E6.3) \textbf{et démontrer} (Dm.E\xul{2}.1 et 2) les
	      associations en série et en parallèle d'impédances complexes,
	      \textbf{ainsi} que les relations de ponts diviseurs de tension et de
	      courant en complexes (Dm.E\xul{2}.3 et 4). Éventuellement déterminer
	      l'impédance équivalente d'une association donnée par l'examinataire, si
	      le temps le permet.

	\item Circuit RC série en RSF~: présenter le système (Df.E6.6), déterminer
	      l'amplitude complexe sur la tension du condensateur ainsi que son amplitude
	      réelle et sa phase (Dm.E6.2) \textbf{à partir d'un pont diviseur de
		      tension} (voir exemples du TDE6 et exemple d'application du cours).
	      \vspace{-15pt}
	      \subsection(E7){Oscillateurs en RSF}

	\item Étude de la résonance en intensité pour le circuit RLC série en RSF~:
	      présenter le système réel et complexe (Df.E7.1), établir l'expression de
	      $\Iu(x)$ (Dm.E7.1), donner son amplitude réelle $I(x)$ et sa phase
	      $\f_i(x)$ en justifiant son domaine d'appartenance (Dm.E7.2).
	      Déterminer sa pulsation de résonance et sa phase à la résonance
	      (Dm.E7.3) et tracer $I(x)$ et $\f_i(x)$ (Ipt.E7.1).

	\item À partir de $I(x) = \frac{E_0/R}{\sqrt{1+Q^2 \pa{x-\frac{1}{x}}^2}}$,
	      déterminer les pulsations réduites de coupure $x_1$ et $x_2$ donnant les
	      limites de la bande passante, et exprimer la largeur de la bande
	      passante en fonction du facteur de qualité (Dm.E7.4).

	\item Présenter le système, établir l'amplitude complexe, l'amplitude
	      réelle et la phase en justifiant le domaine d'appartenance et tracer
	      ces dernières pour l'un des système suivants~:
	      \begin{tasks}[label=\protect\fbox{\Alph*}](2)
		      \task $\quad u_C(t)$ RLC série
		      \smallbreak
		      (Df.E7.1, Dm.E7.5, Dm.E7.6 et Ipt.E7.2)
		      \task $\quad x(t)$ ressort amorti
		      \smallbreak
		      (Df.E7.4, Dm.E7.8, Dm.E7.6 adaptée au ressort)
	      \end{tasks}

	      % \item Étude de la résonance de type tension ou élongation pour le
	      % circuit RLC série en RSF~: présenter le système réel et complexe
	      % (Df.E7.1 ou Df.E7.4), établir l'expression de l'amplitude complexe
	      % $\Uu(x)$ (Dm.E7.5), donner son amplitude réelle $U(x)$ et sa phase
	      % $\f_u(x)$ en justifiant son domaine d'appartenance (Dm.E7.6). Les
	      % tracer (Ipt.E7.2).

	\item À partir de $U(x) = \frac{E_0}{\sqrt{\pa{1-x^2}^2 +
				      \pa{\frac{x}{Q}}^2}}$, démontrer la condition de résonance ainsi
	      que l'amplitude à la résonance (Dm.E7.7).
	      \subsection(E8){Filtrage linéaire}
	\item Énoncer le théorème de \textsc{Fourier} (Th.E8.1). Expliquer ce qu'il
	      décrit. Qu'est-ce que l'analyse spectrale~? Décrire le vocabulaire
	      introduit (Df.E8.1). Tracer le signal somme d'une fréquence de
	      \SI{50}{Hz} d'amplitude 1 avec un signal de \SI{500}{Hz} d'amplitude
	      \num{0.2} dans le domaine temporel, et tracer sa décomposition
	      fréquentielle (Ex.E8.1, Fig.E8.2).

	\item Filtre RC sur C~: démontrer l'expression de la fonction de transfert
	      (Dm.E8.1), calculer son gain et sa valeur maximale, son gain en décibels et
	      sa valeur maximale, sa bande passante et sa phase (Ap.E8.1)~; déterminer
	      ses asymptotes en gain et en phase (Ap.E8.2) et tracer son diagramme de
	      \textsc{Bode} (Fig.E8.10).
\end{enumerate}

\end{document}
