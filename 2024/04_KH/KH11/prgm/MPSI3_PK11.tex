\documentclass[a4paper, 12pt, final, garamond]{book}
\usepackage{cours-preambule}
\graphicspath{{../figures/}}

\raggedbottom

\makeatletter
\renewcommand{\@chapapp}{Programme de kh\^olle -- semaine}
\makeatother

\begin{document}
\setcounter{chapter}{10}

\chapter{Du 11 au 14 d\'ecembre}

\section{Cours et exercices}
\ssubsection{E5}{Circuits électriques en RSF}
\begin{enumerate}[label=\Roman*]
	\bitem{Circuit RC série en RSF}~: présentation, réponse d'un système
	en RSF, passage en complexes.
	\bitem{Circuits électriques en RSF}~: lois de l'électrocinétique (loi
	des nœuds, loi des mailles), impédance et admittance complexes
	(définition, impédances de bases, comportements limites), associations
	d'impédances et ponts diviseurs.
	\bitem{Mesure de déphasages}~: définition, valeurs particulières,
	lecture d'un déphasage, déphasage des impédances.
\end{enumerate}

\ssubsection{E6}{Oscillateurs en RSF}
\begin{enumerate}[label=\Roman*]
	\bitem{Introduction}~: rappel oscillateurs, méthode des complexes,
	notion de résonance et bande passante.
	\bitem{Exemple électrique~: circuit RLC série en RSF}~: présentation,
	étude de l'intensité (amplitude complexe, amplitude réelle et maximum,
	phase, influence de $Q$), étude de la tension (amplitude complexe,
	amplitude réelle et condition de résonance, phase).
	\bitem{Exemple mécanique~: ressort horizontal en RSF}~: présentation,
	étude de l'élongation (amplitude d'élongation complexe, amplitude réelle
	et condition de résonance), résonance en vitesse.
\end{enumerate}

\section{Cours uniquement}
\ssubsection{E7}{Filtrage linéaire}
\begin{enumerate}[label=\Roman*]
	\bitem{Signaux périodiques}~: période, moyenne, valeur efficace.
	\bitem{Décomposition en série de Fourier}~: théorème de
	\textsc{Fourier}, analyse spectrale, relation de \textsc{Parseval}.
	\bitem{Filtrage linéaire}~: introduction.
\end{enumerate}

\newpage
\section{Questions de cours possibles}
\ssubsection{E5}{Circuits électriques en RSF}
\begin{enumerate}
	\item Méthode des complexes en RSF~: donner la forme de réponse d'un système
	      en RSF, les relations entre les grandeurs réelle et complexe associée,
	      l'intérêt pour la dérivation et le lien entre une équation
	      différentielle réelle et l'équation algébrique complexe associée.
	\item Rappeler comment se définit une impédance complexe, puis donner
	      \textbf{et démontrer} les impédances complexes d'une résistance, d'une
	      bobine et d'un condensateur, indiquer et justifier leurs comportements
	      limites si elles en ont.
	\item Donner \textbf{et démontrer} les associations en série et en parallèle
	      d'impédances complexes, et déterminer l'impédance équivalente d'une
	      association donnée par l'examinataire.
	\item Donner \textbf{et démontrer} les relations des ponts diviseur de tension
	      et diviseur de courant en complexes. Application sur un circuit proposé
	      par l'interrogataire.
	\item Circuit RC série en RSF~: présenter le système réel, le système en
	      complexes, déterminer l'amplitude complexe sur la tension du
	      condensateur \textbf{à l'aide d'un pont diviseur de tension} ainsi que
	      son amplitude réelle et sa phase.
	\item Expliquer, à l'aide de graphiques représentant les fonctions tangente,
	      arctangente et du plan complexe, quand est-ce qu'on peut prendre
	      $\arctan(\tan{\tt}) = \tt$, quand est-ce qu'on ajoute $\pi$ et quand est-ce
	      qu'on ajoute $-\pi$.
\end{enumerate}

\ssubsection{E6}{Oscillateurs en RSF}
\begin{enumerate}[resume]
	\item Étude de la résonance en intensité pour le circuit RLC série en RSF~:
	      établir l'expression de $\Iu$, donner son amplitude réelle $I(\w)$.
	      Déterminer sa pulsation de résonance et tracer $I(\w)$. Étudier sa
	      phase, tracer $\arg*{\Iu(\w)}$.
	\item À partir de $\Iu(\w) =
		      \frac{E_0/R}{\sqrt{1 + Q^2\left( \frac{\w}{\w_0} - \frac{\w_0}{\w}
				      \right)^2}}$, déterminer les valeurs $\w_1$ et $\w_2$ donnant
	      les limites de la bande passante et exprimer la largeur de la
	      bande passante en fonction de facteur de qualité.
	\item Étude de la résonance en \textbf{élongation} pour le ressort
	      horizontal en RSF~: introduire le système, déterminer l'amplitude
	      complexe $\Xu(\w)$ sous forme canonique, déterminer l'amplitude réelle
	      $X(\w)$ et la pulsation de résonance en explicitant la condition de
	      résonance, tracer l'allure de l'amplitude réelle.
\end{enumerate}

\ssubsection{E7}{Filtrage linéaire}
\begin{enumerate}[resume]
	\item Définir la valeur efficace d'un signal périodique. Déterminer sa valeur
	      pour un signal $s(t) = A \cos(\wt)$. Que représente la valeur efficace~?
	      Donner un exemple.
	\item Énoncer le théorème de \textsc{Fourier}. Expliquer ce qu'il décrit.
	      Qu'est-ce que l'analyse spectrale~? Décrire le vocabulaire introduit. Tracer
	      le signal somme d'une fréquence de \SI{50}{Hz} d'amplitude 1 avec un signal
	      de \SI{500}{Hz} d'amplitude \num{0.2} dans le domaine temporel, et tracer sa
	      décomposition fréquentielle\ftn{Version couleur du graphique projeté en
		      cours sur
		      \href{https://cahier-de-prepa.fr/mpsi3-pothier/download?id=2816}{Cahier de
			      Prépa}.}.
\end{enumerate}

\end{document}
