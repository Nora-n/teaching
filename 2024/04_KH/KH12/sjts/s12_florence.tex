\documentclass[a4paper, 11pt]{book}
\usepackage{/home/nicolas/Documents/Enseignement/Prepa/bpep/fichiers_utiles/preambule}

\newcommand{\dsNB}{12}
\makeatletter
\renewcommand{\@chapapp}{Kh\^olles MPSI2 -- semaine \dsNB}
\makeatother

\toggletrue{corrige}  % décommenter pour passer en mode corrigé

\begin{document}

\resetQ
\newpage

\chapter{Sujet 1\siCorrige{\!\!-- corrigé}}
\section{Question de cours}
Tracé du diagramme de \textsc{Bode} du circuit RC avec $R$ en sortie.

\subimport{/home/nicolas/Documents/Enseignement/Prepa/bpep/exercices/Colle/filtre_passe_haut_ordre_2/}{sujet.tex}

\resetQ
\newpage

\chapter{Sujet 2\siCorrige{\!\!-- corrigé}}
\section{Question de cours}
Domaines intégrateur et dérivateur des filtres du 1er ordre.

\subimport{/home/nicolas/Documents/Enseignement/Prepa/bpep/exercices/TD/diagramme_Bode_composition/}{sujet.tex}

\resetQ
\newpage

\chapter{Sujet 3\siCorrige{\!\!-- corrigé}}
\section{Question de cours}
Exercice d'application sur le filtrage de sinaux avec un passe-bas du 1er ordre.

\subimport{/home/nicolas/Documents/Enseignement/Prepa/bpep/exercices/TD/filtre_RL/}{sujet.tex}

\resetQ
\subimport{/home/nicolas/Documents/Enseignement/Prepa/bpep/exercices/Colle/circuit_RLC_resonance/}{sujet.tex}

\end{document}
