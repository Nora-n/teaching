\documentclass[a4paper, 12pt, final, garamond]{book}
\usepackage{cours-preambule}
\graphicspath{{../figures/}}

\raggedbottom

\makeatletter
\renewcommand{\@chapapp}{Programme de kh\^olle -- semaine}
\makeatother

\begin{document}
\setcounter{chapter}{13}

\chapter{Du 13 au 16 janvier}

\section{Exercices uniquement}
\subsection(E8){Filtrage linéaire}
\subsection(ON1){Ondes progressives}

\section{Cours et exercices}
\subsection(ON2){Interférences à deux ondes}
\begin{enumerate}[label=\Roman*]
	\item[b]{Introduction}~: approximation par une onde plane, phase spatiale et
	      déphasage, rappel valeurs particulières, différence de marche et
	      correspondance valeurs particulières pour $\Delta{\f_0} = 0$.
	\item[b]{Superposition d'ondes sinusoïdales de mêmes fréquences}~:
	      préesntation, signaux de même amplitude, signaux d'amplitudes différentes,
	      bilan, exercice d'application.
	\item[b]{Interférences lumineuses}~: cohérence, intensité, formule de
	      \textsc{Fresnel}, chemin optique.
	\item[b]{Expérience des trous d'\textsc{Young}}~: introduction,
	      présentation, détermination de l'interfrange.
\end{enumerate}

\section{Cours uniquement}
\subsection(M1){Cinématique du point}
\begin{enumerate}[label=\Roman*]
	\item[b]{Description et paramétrage du mouvement}~: système et point matériel,
	      notion de référentiel, relativité du mouvement, exemples de référentiels,
	      outils mathématiques, projection de vecteurs.
	\item[b]{Position, vitesse et accélération}~: position et déplacement
	      élémentaire, équations horaires et trajectoires~; vitesse et vitesse
	      instantanée, notation pointée~; accélération et accélération instantanée.
	\item[b]{Exemples de mouvements}~: rectiligne uniforme, rectiligne
	      uniformément accéléré, courbe uniformément accéléré.
\end{enumerate}

\subsection(M2){Dynamique du point}
\begin{enumerate}[label=\Roman*]
	\item[b]{Introduction}~: inertie et quantité de mouvement, forces
	      fondamentales.
	\item[b]{Trois lois de \textsc{Newton}}~: principe d'inertie, principe
	      fondamental de la mécanique, loi des actions réciproques.
	\item[b]{Ensembles de points}~: centre d'inertie, quantité de mouvement
	      d'un ensemble de points, théorème de la résultante cinétique, méthode
	      générale de résolution.
	\item[b]{Forces usuelles}~: poids, chute libre avec angle initial~; poussée
	      d'\textsc{Archimède}~; frottements fluides, chute avec frottements
	      linéaires et quadratique, résolution par adimensionnement
	      %  ~; frottements
	      % solides~; force de rappel d'un ressort et longueur d'équilibre vertical.
\end{enumerate}

\newpage

\section{Questions de cours possibles}
\subsection(ON2){Interférences à deux ondes}
\begin{enumerate}
	% \item Démontrer le lien entre déphasage et différence de marche (Df.ON2.2,
	%       Pt.ON2.2, Dm.ON2.1). Démontrer les valeurs particulières de différence de
	%       marche en précisant la condition pour les exprimer ainsi (Dm.ON2.2). Définir
	%       et démontrer le chemin optique d'un rayon lumineux, et donner le lien entre
	%       entre déphasage et chemin optique (Df.ON2.5, Dm.ON.9, Pt.ON2.11).
	%
	% \item{}(Ap.ON2.1)
	%       Soient 2 émetteurs envoyant une onde progressive sinusoïdale de même
	%       fréquence, amplitude et phase à l'origine. Le premier est fixé à
	%       l'origine du repère, l'émetteur 2 est mobile et à une distance $d$ du
	%       premier, et un microphone est placé à une distance fixe $x_0$ de
	%       l'émetteur 1 et est aligné avec les deux émetteurs. On néglige
	%       l'influence de l'émetteur 2 sur l'émetteur 1 et toute atténuation.
	%       \begin{enumerate}[label=\sqenumi]
	% 	      \item \textbf{Faire un schéma}.
	% 	      \item On part de $d=0$ et on augmente $d$ jusqu'à ce que le signal
	% 	            enregistré soit nul. Ceci se produit pour $d_1 =
	% 		            \SI{6.0}{cm}$. \textbf{Expliquer cette extinction} et en
	% 	            déduire la longueur d'onde du signal.
	% 	      \item Pour $d_2 = \SI{12.0}{cm}$, quelle sera
	% 	            l'amplitude du signal enregistré~?
	%       \end{enumerate}
	\item Expliquer ce qu'est la cohérence (Df.ON2.4) et pourquoi on ne fait des
	      interférence qu'avec une unique source pour des signaux lumineux (Pt.ON2.8).
	      Donner et justifier/démontrer l'expression de l'intensité d'un signal en
	      général et pour une OPS (Pt.ON2.9, Dm.ON2.7). Démontrer la formule de
	      \textsc{Fresnel} pour deux signaux sinusoïdaux de même fréquence et
	      d'amplitudes différentes. La simplifier pour des signaux de même
	      amplitude (Pt.ON2.10, Dm.ON2.8).

	\item Trous d'\textsc{Young}~: présenter l'expérience (Df.ON2.7) et démontrer
	      l'expression de l'intensité relevée dans le cas de signaux de même
	      intensité. En déduire l'expression de l'interfrange (Pt.ON2.12,
	      Dm.ON2.10).
	      \smallbreak
	      On donne le développement limité suivant~:
	      \[\sqrt{1+\ep} \mathop{=}\limits_{\ep \ll 1} 1 + \ep/2 + o(\ep)\]

	      \subsection(M1){Cinématique du point}
	\item Présenter les 3 référentiels fondamentaux (Df.M1.3) et la condition pour
	      les supposer galiléen (Pt.\textbf{M2}.1). Présenter le repère cartésien
	      (Df.M1.5) ainsi que la méthode de projection vectorielle en 2D (Ot.M1.1).

	\item[b]{Coordonnées cartésiennes}~: Présenter comment s'écrit la position
	      avec un schéma (Df.M1.6). Donner l'expression du vecteur déplacement
	      élémentaire (Df.M1.7). Démontrer alors l'expression du vecteur vitesse par
	      deux approches différentes (Dm.M1.1).

	\item{}(Ap.M1.2) Déterminer les équations horaires pour un mouvement
	      uniformément accéléré caractérisé par $\af(t) = -g \uy$ avec des conditions
	      initiales nulles ($\OM(0) = \of$ et $\vf(0) = \of$).

	      \subsection(M2){Dynamique du point}
	\item Énoncer les trois lois de \textsc{Newton} (L.M2.1, 2 et 3). Définir le
	      centre d'inertie d'un ensemble de points (Df.M2.3), le vecteur quantité
	      de mouvement d'un ensemble de points et son lien avec le centre
	      d'inertie (Df.M2.4, Pt et Dm.M2.2), énoncer et démontrer le théorème de
	      la résultante cinétique (Th et Pr.M2.1).

	\item{}(Dm.M2.3) Déterminer les \textbf{équations horaires} ainsi que la
	      \textbf{trajectoire} du lancer d'une masse avec une vitesse initiale
	      $\vf_0$ faisant un angle $\alpha$ avec l'horizontale. Une attention
	      particulière sera portée à l'établissement du système d'étude.
	      Déterminer alors la portée, la flèche du tir ainsi que le temps de vol,
	      au choix (potentiellement multiple) de l'interrogataire

	\item{}(Ap.M2.2) Déterminer la proportion immergée d'un glaçon. On donne
	      $\rho_{\rm eau} = \SI{1.00e3}{km.m^{-3}}$ et $\rho_{\rm glace} =
		      \SI{9.17e2}{kg.m^{-3}}$.

	\item{}(Dm.M2.4, Ap.M2.3) Déterminer la vitesse limite et le temps
	      caractéristique du mouvement pour une chute libre sans vitesse initiale avec
	      frottements \textbf{linéaires}. Les approches d'adimensionnement d'équation
	      différentielle, de solution particulière ou de résolution totale directe
	      sont possibles.

	\item{}(Dm.M2.5) Déterminer la vitesse limite et le temps caractéristique du
	      mouvement pour une chute libre sans vitesse initiale avec frottements
	      \textbf{quadratiques} par une approche d'adimensionnement. Une
	      attention particulière sera portée à l'établissement du système (cf.\
	      Dm.M2.4).
\end{enumerate}

\end{document}
