\documentclass[a4paper, 12pt, final, garamond]{book}
\usepackage{cours-preambule}

\raggedbottom

\makeatletter
\renewcommand{\@chapapp}{Programme de kh\^olle -- semaine}
\makeatother

\begin{document}
\setcounter{chapter}{14}

\chapter{Du 20 au 23 janvier}

\section{Exercices uniquement}
\subsection(ON2){Interférences à deux ondes}

\section{Cours et exercices}
\subsection(M1){Cinématique du point}
\begin{enumerate}[label=\Roman*]
	\item[b]{Description et paramétrage du mouvement}~: système et point matériel,
	      notion de référentiel, relativité du mouvement, exemples de référentiels,
	      outils mathématiques, projection de vecteurs.
	\item[b]{Position, vitesse et accélération}~: position et déplacement
	      élémentaire, équations horaires et trajectoires~; vitesse et vitesse
	      instantanée, notation pointée~; accélération et accélération instantanée.
	\item[b]{Exemples de mouvements}~: rectiligne uniforme, rectiligne
	      uniformément accéléré, courbe uniformément accéléré.
\end{enumerate}

\subsection(M2){Dynamique du point}
\begin{enumerate}[label=\Roman*]
	\item[b]{Introduction}~: inertie et quantité de mouvement, forces
	      fondamentales.
	\item[b]{Trois lois de \textsc{Newton}}~: principe d'inertie, principe
	      fondamental de la mécanique, loi des actions réciproques.
	\item[b]{Ensembles de points}~: centre d'inertie, quantité de mouvement
	      d'un ensemble de points, théorème de la résultante cinétique, méthode
	      générale de résolution.
	\item[b]{Forces usuelles}~: poids, chute libre avec angle initial~; poussée
	      d'\textsc{Archimède}~; frottements fluides, chute avec frottements
	      linéaires et quadratique, résolution par adimensionnement~; frottements
	      solides~; force de rappel d'un ressort et longueur d'équilibre vertical.
\end{enumerate}

\section{Cours uniquement}
\subsection(M3){Mouvements courbes}
\begin{enumerate}[label=\Roman*]
	\item[b]{Mouvement courbe dans le plan}~: position, variation des vecteurs de
	      base, déplacement élémentaire, vitesse, accélération en coordonnées
	      polaires.
	\item[b]{Exemples de mouvements plans}~: mouvement circulaire,
	      circulaire uniforme, repère de \textsc{Frenet}, démonstration.
	\item[b]{Application~: pendule simple}~: tension d'un fil, pendule simple.
	\item[b]{Mouvement courbe dans l'espace}~: coordonnées cylindriques,
	      coordonnées sphériques.
\end{enumerate}

\newpage
\section{Questions de cours possibles}
\subsection(M2){Dynamique du point}
\begin{enumerate}
	\item Énoncer les trois lois de \textsc{Newton} (L.M2.1, 2 et 3). Définir le
	      centre d'inertie d'un ensemble de points (Df.M2.3), le vecteur quantité
	      de mouvement d'un ensemble de points et son lien avec le centre
	      d'inertie (Df.M2.4, Pt et Dm.M2.2), énoncer et démontrer le théorème de
	      la résultante cinétique (Th et Pr.M2.1).

	\item{}(Dm.M2.3) Déterminer les \textbf{équations horaires} ainsi que la
	      \textbf{trajectoire} du lancer d'une masse avec une vitesse initiale
	      $\vf_0$ faisant un angle $\alpha$ avec l'horizontale. Une attention
	      particulière sera portée à l'établissement du système d'étude.
	      Déterminer alors la portée, la flèche du tir ainsi que le temps de vol,
	      au choix (potentiellement multiple) de l'interrogataire

	\item{}(Ap.M2.2) Déterminer la proportion immergée d'un glaçon. On donne
	      $\rho_{\rm eau} = \SI{1.00e3}{km.m^{-3}}$ et $\rho_{\rm glace} =
		      \SI{9.17e2}{kg.m^{-3}}$.

	\item{}(Dm.M2.4 et 5, Ap.M2.3)
	      Déterminer la vitesse limite et le temps caractéristique du
	      mouvement pour une chute libre sans vitesse initiale avec frottements
	      \textbf{linéaires ou quadratiques}, avec l'approche désirée.

	\item%
	      Présenter les lois du frottement de \textsc{Coulomb} (Df.M2.9, Pt.M2.7,
	      Ex.M2.3), et refaire l'exercice (TDM2.app|III/2)~:
	      \smallbreak
	      On considère un plan incliné d'un angle $\alpha$ par rapport à
	      l'horizontale. Une brique de masse $m$ est lancée depuis le bas du plan
	      vers le haut, avec une vitesse $v_0$.
	      On suppose qu'\textbf{il existe des frottements solides},
	      avec $f$ le coefficient de frottements solides tel que $f =
		      \num{0.20}$.
	      \begin{enumerate}[label=\sqenumi]
		      \item Établir l'équation horaire du mouvement de la brique lors de sa
		            montée.
		      \item Déterminer la date à laquelle la brique s'arrête, ainsi que la
		            distance qu'elle aura parcourue. Commenter l'expression littérale.
	      \end{enumerate}
	\item{}(Dm.M2.6)
	      Position d'équilibre d'un ressort vertical~: présenter le système,
	      déterminer la longueur d'équilibre, déterminer l'équation différentielle
	      sur la position de la masse, solution pour des conditions initiales
	      données par l'interrogataire.

	      \subsection(M3){Mouvements courbes}
	\item%
	      Présenter les coordonnées \textbf{cylindriques} avec un schéma
	      introduisant la base et indiquant l'expression de $\OM(t)$ dans cette
	      base (Df.M3.6). Démontrer les expressions de $\dd{\ur}$ et $\dd{\ut}$,
	      en déduire à l'aide d'un schéma l'expression de $\dd{\OM}$ en polaires
	      et justifier son expression en cylindriques, puis démontrer $\vv{v}(t)$
	      et $\af(t)$ en cylindriques (Pt.M3.8, Dm.M3.2, 3, 4 et 5).

	\item%
	      Établir succinctement la vitesse et l'accélération pour un mouvement
	      circulaire de rayon $R$, puis pour le mouvement circulaire uniforme de
	      rayon $R$ (Ipl.M3.1 et 2). Un schéma est attendu.
	\item%
	      Présenter la base de \textsc{Frenet} sur une trajectoire
	      quelconque (Df.M3.4), démontrer les expressions de $\vf(t)$ et $\af(t)$
	      dans cette base (Dm.M3.6).

	\item{}(Dm.M3.7)
	      Étude du pendule simple~: mise en situation, équation différentielle,
	      linéarisation, résolution. Que se passe-t-il pour de grands angles~?
	      Tracer l'allure de l'évolution de $T/T_0$ en fonction de $\th_0$.
	      % \item%
	      % Présenter les coordonnées sphériques avec un schéma introduisant la
	      % base
	      % et indiquant les coordonnées, donner l'expression de $\OM$ dans cette
	      % base,
	      % donner \textbf{et démontrer} le déplacement
	      % élémentaire. Déterminer le volume d'une boule. Indiquer comment on se
	      % repère sur Terre en donnant les coordonnées du lycée \textsc{Pothier}.
\end{enumerate}
\end{document}
