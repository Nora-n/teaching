\documentclass[a4paper, 10pt, final, garamond]{book}
\usepackage{cours-preambule}

\raggedbottom

\makeatletter
\renewcommand{\@chapapp}{Programme de kh\^olle -- semaine}
\makeatother

\begin{document}
\setcounter{chapter}{14}

\chapter{Du 22 au 25 janvier}

\section{Exercices uniquement}
\ssubsection{ON2}{Interférences à deux ondes}

\section{Cours et exercices}
\ssubsection{M1}{Cinématique du point}
\begin{enumerate}[label=\Roman*]
	\bitem{Système et point matériel}~: définition système, point
	matériel.
	\bitem{Description et paramétrage du mouvement}~: notion de
	référentiel, relativité du mouvement, exemples de référentiels, vecteur
	base de projection et repère, projection de vecteurs.
	\bitem{Position, vitesse et accélération}~: position et déplacement
	élémentaire, équations horaires et trajectoires~; vitesse et vitesse
	instantanée, notation pointée~; accélération et accélération
	instantanée.
	\bitem{Exemples de mouvements}~: rectiligne uniforme, rectiligne
	uniformément accéléré, courbe uniformément accéléré.
\end{enumerate}

\ssubsection{M2}{Dynamique du point}
\begin{enumerate}[label=\Roman*]
	\bitem{Introduction}~: inertie et quantité de mouvement, forces
	fondamentales.
	\bitem{Trois lois de \textsc{Newton}}~: principe d'inertie, principe
	fondamental de la mécanique, loi des actions réciproques.
	\bitem{Ensembles de points}~: centre d'inertie, quantité de mouvement
	d'un ensemble de points, théorème de la résultante cinétique, méthode
	générale de résolution.
	\bitem{Forces usuelles}~: poids, chute libre avec angle initial~; poussé
	d'\textsc{Archimède}~; frottements fluides, chute libre avec frottements
	linéaires et quadratique, résolution par adimensionnement~; frottements
	solides~; force de rappel d'un ressort et longueur d'équilibre vertical.
\end{enumerate}

\section{Cours uniquement}
\ssubsection{M3}{Mouvements courbes}
\begin{enumerate}[label=\Roman*]
	\bitem{Mouvement courbe dans le plan}~: position, vitesse,
	déplacement élémentaire, accélération en coordonnées polaires.
	\bitem{Exemples de mouvements plans}~: mouvement circulaire,
	circulaire uniforme, repère de \textsc{Frenet}.
	\bitem{Application~: pendule simple}~: tension d'un fil, pendule simple.
	\bitem{Mouvement courbe dans l'espace}~: coordonnées cylindriques,
	coordonnées sphériques.
\end{enumerate}

\newpage
\section{Questions de cours possibles}
\ssubsection{M2}{Dynamique du point}
\begin{enumerate}
	\litem{17pt}{\strr}%
	Énoncer les trois lois de \textsc{Newton}, définir le centre d'inertie
	d'un ensemble de points, le vecteur quantité de mouvement d'un
	ensemble de points et son lien avec le centre d'inertie, énoncer et
	démontrer le théorème de la résultante cinétique.

	\litem{17pt}{\strrr}%
	Déterminer les \textbf{équations horaires} ainsi que la
	\textbf{trajectoire} du lancer d'une masse avec une vitesse initiale
	$\vf_0$ faisant un angle $\alpha$ avec l'horizontale. Une attention
	particulière sera portée à l'établissement du système d'étude.
	Déterminer alors la portée, la flèche du tir ainsi que le temps de vol,
	au choix (potentiellement multiple) de l'interrogataire.

	\litem{17pt}{\str}%
	Déterminer la proportion immergée d'un glaçon. On donne $\rho_{\rm
			eau} = \SI{1.00e3}{km.m^{-3}}$ et $\rho_{\rm glace} =
		\SI{9.17e2}{kg.m^{-3}}$.

	\litem{17pt}{\strr}%
	Déterminer la vitesse limite et le temps caractéristique du
	mouvement pour une chute libre sans vitesse initiale avec frottements
	\textbf{linéaires ou quadratiques}, avec l'approche désirée.

	\litem{17pt}{\strr}%
	Présenter les lois du frottement de \textsc{Coulomb}, et refaire
	l'exercice~:
	\smallbreak
	On considère un plan incliné d'un angle $\alpha$ par
	rapport à l'horizontale. Une brique de masse $m$ est
	lancée depuis le bas du plan vers le haut, avec une vitesse $v_0$.
	On suppose qu'il existe des frottements solides,
	avec $f$ le coefficient de frottements solides tel que $f =
		\num{0.20}$.
	\begin{enumerate}[label=\sqenumi]
		\item Établir l'équation horaire du mouvement de la brique lors de
		      sa montée.
		\item Déterminer la date à laquelle la brique s'arrête, ainsi que la
		      distance qu'elle aura parcourue. Commenter l'expression littérale.
	\end{enumerate}
	\litem{17pt}{\strr}%
	Position d'équilibre d'un ressort vertical~: présenter le système,
	déterminer l'équation différentielle sur la position de la masse,
	déterminer la longueur d'équilibre, solution pour des conditions
	initiales données par l'interrogataire.
\end{enumerate}

\ssubsection{M3}{Mouvements courbes}
\begin{enumerate}[resume]
	\litem{17pt}{\strrr}%
	Présenter les coordonnées cylindriques avec un schéma introduisant la
	base et indiquant les coordonnées, donner l'expression de $\OM$ dans
	cette base, donner \textbf{et démontrer} l'expression de la vitesse, du
	déplacement élémentaire et de l'accélération en coordonnées cylindriques.
	\litem{17pt}{\str}%
	Établir succinctement la vitesse et l'accélération pour un mouvement
	circulaire de rayon $R$, puis pour le mouvement circulaire uniforme de rayon
	$R$.
	\litem{17pt}{\strr}%
	Présenter succinctement la base de \textsc{Frenet} sur une trajectoire
	quelconque (cercle osculateur), écrire les vecteurs vitesse et
	accélération dans cette base en fonction de $v$, $R$ et $\dot{v}$.
	\litem{17pt}{\strr}%
	Étude du pendule simple~: mise en situation, équation différentielle,
	linéarisation, résolution. Que se passe-t-il pour de grands angles~? Tracer
	l'allure de l'évolution de $T/T_0$ en fonction de $\tt_0$.
	% \litem{17pt}{\strrr}%
	% Présenter les coordonnées sphériques avec un schéma introduisant la base
	% et indiquant les coordonnées, donner l'expression de $\OM$ dans cette base,
	% donner \textbf{et démontrer} le déplacement
	% élémentaire. Déterminer le volume d'une boule. Indiquer comment on se
	% repère sur Terre en donnant les coordonnées du lycée \textsc{Pothier}.
\end{enumerate}
\end{document}
