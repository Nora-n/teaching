\documentclass[a4paper, 12pt, final, garamond]{book}
\usepackage{cours-preambule}

\raggedbottom

\makeatletter
\renewcommand{\@chapapp}{Programme de kh\^olle -- semaine}
\makeatother

\begin{document}
\setcounter{chapter}{17}

\chapter{Du 12 au 15 f\'evrier}

\section{Exercices uniquement}
\ssubsection{M4}{Approche énergétique}
% \begin{enumerate}[label=\Roman*]
% 	\bitem{Notions énergétiques}~: énergie, conservation, puissance.
% 	\bitem{Énergie cinétique et travail d'une force constante}~: définitions,
% 	exemples, travail du poids.
% 	\bitem{Puissance d'une force et travail élémentaire}~: définitions, TPC,
% 	TEC, et applications, comment choisir~?
% 	\bitem{Énergie potentielle}~: forces conservatives ou
% 	non, travail d'une force conservative, gradient d'une fonction scalaire,
% 	opérateur différentiel, lien à l'énergie potentielle
% 	\bitem{Énergie mécanique}~: définition, TEM et TPM et applications.
% 	\bitem{Énergie potentielle et équilibres}~: notion d'équilibre, lien
% 	avec $\Ec_p$, équilibres stables et instables, lien avec
% 	$\dv[2]{\Ec_p}{x}$ , étude générale autour d'un point d'équilibre
% 	stable~: oscillateur harmonique.
% 	\bitem{Énergie potentielle et trajectoire}~: détermination
% 	qualitative d'une trajectoire, état lié et diffusion~; cas du pendule
% 	simple, étude mouvement selon $\Ec_p$ et $\Ec_m$.
% \end{enumerate}

\section{Cours et exercices}
\ssubsection{M5}{Mouvement de particules chargées}
\begin{enumerate}[label=\Roman*]
	\bitem{Champs électrique et magnétique}~: définitions, exemples condensateur
	et bobine.
	\bitem{Force de \textsc{Lorentz}}~: définition, comparaison au poids,
	remarque produit vectoriel, puissance de la force de \textsc{Lorentz},
	potentiel électrostatique.
	\bitem{Mouvement dans un champ électrique}~: situation générale, accélération
	pour $\vfo\parr\Ef$, déviation pour $\vfo\perp\Ef$, angle de déviation,
	applications (accélérateur linéaire, oscilloscope analogique).
	\bitem{Mouvement dans un champ magnétique}~: mise en équation, cas
	$\vfo\parr\Bf$, cas $\vfo\perp\Bf$~: trajectoire et équations horaires
	cyclotron~; cas général (mouvement hélicoïdal), applications
	(spectromètre de masse, cyclotron, effet \textsc{Hall})
\end{enumerate}

\section{Cours uniquement}
\ssubsection{AM1}{Structure des entités chimiques}
\begin{enumerate}[label=\Roman*]
	\bitem{Niveaux d'énergie d'un électron dans un atome}~: nombres quantiques et
	orbitales atomiques (HP), niveaux d'énergie (HP), électrons de cœur et de
	valence.
	\bitem{Tableau périodique}~: construction et blocs, analyse par période,
	analyse par famille.
	\bitem{Structure électronique des molécules}~: représentation de
	\textsc{Lewis} des atomes, liaison covalente, notation de \textsc{Lewis}
	des molécules, écarts à la règle de l'octet.
	\bitem{Géométrie et polarité des entités chimiques}~: modèle VSEPR, polarité
	des liaisons et des molécules, polarisabilité.
\end{enumerate}

% \ssubsection{AM2}{Propriétés physico-chimiques macroscopiques}
% \begin{enumerate}[label=\Roman*]
% 	\bitem{Interactions de \textsc{Van der Waals}}~: \textsc{Keesom}
% 	permanent/permanent, \textsc{Debye} permanent/induit, \textsc{London}
% 	induit/induit, bilan et remarque répulsion.
% 	\bitem{Températures de changement d'état}~: influence du moment dipolaire,
% 	influence de la polarisabilité.
% 	\bitem{Liaison hydrogène}~: introduction expérimentale, définition et
% 	exemples.
% 	\bitem{Solubilité, miscibilité}~: classement des solvants, solubilité, mise
% 	en solution d'espèces ioniques, miscibilité.
% \end{enumerate}

\newpage
\section{Questions de cours possibles}
\ssubsection{M5}{Mouvement de particules chargées}
\begin{enumerate}
	\item Définir la force de \textsc{Lorentz}~; comparer les ordres de
	      grandeurs des forces électriques et magnétiques au poids~; déterminer la
	      puissance de la force de \textsc{Lorentz} et discuter des conséquences.
	      Démontrer qu'elle est conservative et déterminer l'expression de
	      l'énergie potentielle associée.
	\item Action de $\Ef$ uniforme entre deux grilles chargées sur une particule
	      chargée avec $\vfo\parr\Ef$~: présenter la situation, faire un bilan
	      énergétique pour calculer la vitesse de sortie en fonction de la
	      différence de potentiel $U$.
	\item Action de $\Ef$ uniforme entre deux grilles chargées sur une particule
	      chargée avec $\vfo\perp\Ef$~: présenter la situation, déterminer le
	      temps de vol et l'angle de déviation en fonction de $U$.
	\item Action de $\Bf$ uniforme sur une particule chargée avec
	      $\vfo\perp\Bf$~: présenter la situation, et prouver que le mouvement est
	      uniforme, plan et circulaire. On déterminera l'équation de la
	      trajectoire en introduisant le rayon et la pulsation cyclotron, ainsi
	      que les équations scalaires.
\end{enumerate}
\ssubsection{AM1}{Structure des entités chimiques}
\begin{enumerate}[resume]
	\item Savoir comment construire (pas connaître par cœur) les 4 premières
	      lignes du tableau périodique. Définir et placer les blocs $s$, $p$ et
	      $d$. Préciser les colonnes des familles des gaz rares, des halogènes,
	      des métaux alcalins et alcalino-terreux. Placer les métaux et
	      non-métaux. Placer un élément ($Z \leq 36$) sur le tableau à partir de
	      son numéro atomique \textbf{\xul{et/ou}} déterminer son numéro atomique
	      à partir de sa position~; dans tous les cas donner son nombre
	      d'électrons de valence et son schéma de \textsc{Lewis} (bloc $s$ ou
	      $p$).
	\item Établir (pas «~juste~» donner) les représentations de \textsc{Lewis}
	      de molécules simples (\ce{CO2}, \ce{CH4}, \ce{H2O}, \ce{NH3}…) et
	      indiquer leurs représentations spatiales liées à la méthode VSEPR en
	      donnant un ordre de grandeur des angles.
	\item Établir les représentations de \textsc{Lewis} et les charges formelles
	      de $\ce{HO^-, CN^-, NO3^-}$.
	\item Définir l'électronégativité d'un élément et donner (en le justifiant)
	      son évolution par colonne, par famille et globalement dans le tableau.
	      Définir le moment dipolaire d'une liaison, d'une molécule et la
	      polarisabilité, et déterminer le moment dipolaire de \ce{H2O}
	      connaissant $p_{\ce{HO}} = \SI{1.51}{D}$ et $\widehat{({\rm HOH})} =
		      \ang{104.45}$.
\end{enumerate}

\end{document}
