\documentclass[a4paper, 12pt, final, garamond]{book}
\usepackage{cours-preambule}

\raggedbottom

\makeatletter
\renewcommand{\@chapapp}{Programme de kh\^olle -- semaine}
\makeatother

\begin{document}
\setcounter{chapter}{19}

\chapter{Du 11 au 14 mars}

\section{Cours et exercices}
\ssubsection{M6}{Moment cinétique d'un point matériel}
\begin{enumerate}[label=\Roman*]
    \bitem{Moment d'une force}~: par rapport à un point, définition et exemples~;
        par rapport à un axe orienté~: définition et exemples~; bras de levier
        d'une force~: propriété, méthode et application~; exemples de calcul de
        moments.
    \bitem{Moment cinétique}~: par rapport à un point, définition et exemples~;
        par rapport à un axe orienté, définition et exemples.
    \bitem{Théorème du moment cinétique}~: par rapport à un point fixe, énoncé et
        démonstration~; par rapport à un axe orienté fixe~: énoncé et
        démonstration.
    \bitem{Exemple du pendule simple}~: équation du mouvement par TMC.
\end{enumerate}

\ssubsection{M7}{Mouvement à force centrale conservative}
\begin{enumerate}[label=\Roman*]
    \bitem{Forces centrales conservatives}~: définition force centrale,
        définition force centrale conservative et exemples.
    \bitem{Quantités conservées}~: moment cinétique, loi des aires, énergie
        mécanique et énergie potentielle effective.
    \bitem{Champs de force newtoniens}~: cas général, cas attractif, cas
    répulsif.
    \bitem{Mécanique céleste}~: ellipse, lois de \textsc{Kepler}, mouvement
    circulaire.
    \bitem{Satellite en orbite terrestre}~: vitesses cosmiques, satellites
        artificiels~: géostationnaire, positionnement, circumpolaires.
\end{enumerate}

\section{Cours uniquement}
\ssubsection{M8}{Mécanique du solide}
\begin{enumerate}[label=\Roman*]
    \bitem{Système de points matériels}~: systèmes discret et continu, centre
    d'inertie, mouvements d'un solide indéformable~: translation, rotation.
    \bitem{Rappel~: TRC}~: quantité de mouvement d'un ensemble de points, forces
    intérieures et extérieures, théorème de la résultante cinétique.
    % \bitem{Énergétique des systèmes de points}~: énergie cinétique, puissances
    % intérieures et extérieures, théorèmes énergétiques.
    % \bitem{Moments pour un système de points}~: moment cinétique et moment
    % d'inertie, moments intérieurs et extérieurs, théorème du moment cinétique,
    % énergétique d'un solide en rotation.
    % \bitem{Cas particuliers et application}~: notion de couple, liaison pivot,
    % pendule pesant.
\end{enumerate}

\newpage

\section{Questions de cours possibles}
\begin{enumerate}
  \ssubsection{M6}{Moment cinétique d'un point matériel}
	\item Définir le moment cinétique d'un point matériel par rapport à un point
	      et à un axe, et le moment d'une force par rapport à un point et à un
	      axe. Expliquer ce qu'est le bras de levier \textbf{avec un schéma}, et
	      énoncer le lien entre moment d'une force et bras de levier.
	      Démonstration \textbf{pour $\Ff \perp$ à l'axe} (dans le plan de
	      rotation).
	\item Énoncer et démontrer le théorème du moment cinétique par rapport à un
	      point et à un axe~; application au pendule simple pour retrouver
	      l'équation du mouvement \textbf{avec ou sans bras de levier} (au choix
	      de l'interrogataire).

  \ssubsection{M7}{Mouvement à force centrale conservative}
  \item Présenter ce qu'est une force centrale, démontrer que le moment
      cinétique se conserve, prouver que le mouvement est plan, déterminer
      l'expression de la constante des aires, et démontrer la loi des aires.
  \item En utilisant la constante des aires, déterminer l'expression de
      l'énergie potentielle effective pour un mouvement à force centrale
      conservative. \textbf{Démontrer} $\Ec_p$ pour un champ de force
      newtonien. Représenter alors $\Ec_{p,\rm eff}(r)$ dans les cas attractif
      et répulsif, discuter de la nature du mouvement en fonction de
      l'énergie mécanique totale et représenter les types de trajectoires
      possibles.
  \item Présenter les propriétés d'une ellipse avec un schéma~: construction
      mathématique, demi-grand axe, péricentre et apocentre, et vitesses en ces
      points. Énoncer les trois lois de \textsc{Kepler}, démontrer la troisième
      loi de \textsc{Kepler} pour le cas spécifique de l'orbite circulaire~:
      vitesse, période et énergie mécanique.
  \item Définir et démontrer les expressions des vitesses cosmiques en
      justifiant les valeurs d'énergie mécanique à atteindre à l'aide du
      schéma de l'énergie potentielle effective.
  \item Présenter les différents types de satellites terrestres. Détailler les
    conditions pour les satellites géostationnaires, trouver la vitesse
    angulaire correspondante ainsi que le rayon/l'altitude de ces satellites
    et leur vitesse.

  \ssubsection{M8}{Mécanique du solide}
  \item Présenter les systèmes discrets et continus. Démontrer l'équivalence
    entre les 2 définitions du centre d'inertie d'un solide. Définir un solide
    en translation et donner des exemples, un solide en rotation avec un
    exemple.
  \item Donner le lien entre quantité de mouvement d'un système et le centre
      d'inertie d'un solide. Démontrer que la résultante des forces
      intérieures d'un solide est nulle, et démontrer le théorème de la
      résultante cinétique.
    % \item Définir le moment d'inertie d'un solide, donner et démontrer la
    %     relation entre moment cinétique scalaire et moment d'inertie d'un
    %     solide. Retrouver le TMC pour un solide en rotation, en supposant acquis
    %     que la somme des moments intérieurs est nulle. Définir un couple, une
    %     liaison pivot et une liaison pivot parfaite.
    % \item Établir l'équation différentielle du mouvement pour le pendule
    %     \textbf{pesant} grâce au TMC scalaire.
    % \item Donner l'expression de l'énergie cinétique d'un solide en translation
    %     \textbf{et} dans le cas particulier d'un solide en rotation autour d'un
    %     axe fixe. Exprimer les théorèmes énergétiques pour les solides. Donner
    %     l'expression de la puissance des forces extérieures pour un solide en
    %     rotation en fonction du moment des forces extérieures. Démonstration
    %     pour une force $\Ff$ dans le sens de $\ut$.
\end{enumerate}

\end{document}
