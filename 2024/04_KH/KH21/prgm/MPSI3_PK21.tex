\documentclass[a4paper, 12pt, final, garamond]{book}
\usepackage{cours-preambule}

\raggedbottom

\makeatletter
\renewcommand{\@chapapp}{Programme de kh\^olle -- semaine}
\makeatother

\begin{document}
\setcounter{chapter}{20}

\chapter{Du 18 au 21 mars}

\section{Exercices uniquement}
\ssubsection{M6}{Moment cinétique d'un point matériel}

\section{Cours et exercices}
\ssubsection{M7}{Mouvement à force centrale conservative}
\begin{enumerate}[label=\Roman*]
	\bitem{Forces centrales conservatives}~: définition force centrale,
	définition force centrale conservative et exemples.
	\bitem{Quantités conservées}~: moment cinétique, loi des aires, énergie
	mécanique et énergie potentielle effective.
	\bitem{Champs de force newtoniens}~: cas général, cas attractif, cas
	répulsif.
	\bitem{Mécanique céleste}~: ellipse, lois de \textsc{Kepler}, mouvement
	circulaire.
	\bitem{Satellite en orbite terrestre}~: vitesses cosmiques, satellites
	artificiels~: géostationnaire, positionnement, circumpolaires.
\end{enumerate}

\ssubsection{M8}{Mécanique du solide}
\begin{enumerate}[label=\Roman*]
	\bitem{Système de points matériels}~: systèmes discret et continu, centre
	d'inertie, mouvements d'un solide indéformable~: translation, rotation.
	\bitem{Rappel~: TRC}~: quantité de mouvement d'un ensemble de points, forces
	intérieures et extérieures, théorème de la résultante cinétique.
	\bitem{Moments pour un système de points}~: moment cinétique et moment
	d'inertie, moments de forces (couple, liaison pivot), théorème du moment
	cinétique, application pendule pesant
	\bitem{Énergétique des systèmes de points}~: énergie cinétique, puissances,
	théorèmes énergétiques, application pendule pesant et intégrale première du
	mouvement
\end{enumerate}

\section{Cours uniquement}
\ssubsection{C4}{Réactions acido-basiques}
\begin{enumerate}[label=\Roman*]
	\item{Acides et bases}~: couples, pH.
	\item{Réactions acido-basiques}~: constantes
	d'acidité, autoprotolyse de l'eau, réactions entre couples et calculs de
	constantes.
	\item{Distribution des espèces d'un couple}~: lien pH et concentration
	(relation de \textsc{Henderson}), diagramme de prédominance (force des
	acides et échelle des $\pk[A]$), diagramme
	de distribution.
	\item{Méthode de détermination d'un pH}
	% \item{Titrages acido-basiques}~: définition et exemple, méthodes de suivi.
\end{enumerate}

\section{Questions de cours possibles}
\begin{enumerate}[label=\sqenumi]
	\ssubsection{M7}{Mouvement à force centrale conservative}
	% \item Présenter ce qu'est une force centrale, démontrer que le moment
	%       cinétique se conserve, prouver que le mouvement est plan, déterminer
	%       l'expression de la constante des aires, et démontrer la loi des aires.
	\item En utilisant la constante des aires, déterminer l'expression de
	      l'énergie potentielle effective pour un mouvement à force centrale
	      conservative. \textbf{Démontrer} $\Ec_p$ pour un champ de force
	      newtonien. Représenter alors $\Ec_{p,\rm eff}(r)$ dans les cas attractif
	      et répulsif, discuter de la nature du mouvement en fonction de
	      l'énergie mécanique totale et représenter les types de trajectoires
	      possibles.
	\item Présenter les propriétés d'une ellipse avec un schéma~: construction
	      mathématique, demi-grand axe, péricentre et apocentre, et vitesses en ces
	      points. Énoncer les trois lois de \textsc{Kepler}, démontrer la troisième
	      loi de \textsc{Kepler} pour le cas spécifique de l'orbite circulaire~:
	      vitesse, période et énergie mécanique.
	      % \item Définir et démontrer les expressions des vitesses cosmiques en
	      %       justifiant les valeurs d'énergie mécanique à atteindre à l'aide du
	      %       schéma de l'énergie potentielle effective.
	      % \item Présenter les différents types de satellites terrestres. Détailler les
	      %       conditions pour les satellites géostationnaires, trouver la vitesse
	      %       angulaire correspondante ainsi que le rayon/l'altitude de ces satellites
	      %       et leur vitesse.
	      %
	      \ssubsection{M8}{Mécanique du solide}
	      % \item Donner le lien entre quantité de mouvement d'un système et le centre
	      %       d'inertie d'un solide. Démontrer que la résultante des forces
	      %       intérieures d'un solide est nulle, et démontrer le théorème de la
	      %       résultante cinétique.
	\item Donner et démontrer la
	      relation entre moment cinétique scalaire et moment d'inertie d'un
	      solide. Interpréter physiquement le moment d'inertie.
	      Retrouver le TMC pour un solide en rotation, en supposant acquis
	      que la somme des moments intérieurs est nulle. Définir un couple, une
	      liaison pivot et une liaison pivot parfaite.
	\item Démontrer l'expression de la puissance des forces extérieures pour un
	      solide en rotation par calcul direct. Établir le tableau de comparaison
	      entre le point en translation et le solide en rotation et rappeler les
	      théorèmes énergétiques.
	\item Établir l'équation différentielle du mouvement pour le pendule
	      \textbf{pesant} grâce au TMC scalaire par \textbf{bras de levier}.
	\item Établir l'équation différentielle du mouvement pour le pendule
	      \textbf{pesant} grâce au TPC utilisant le \textbf{bras de levier}.
	      Trouver l'intégrale première du mouvement et la relier à une quantité
	      conservée~; conclure sur une méthode permettant de retrouver l'équation
	      différentielle directement à partir de cette quantité.

	      \ssubsection{C4}{Réactions acido-basiques}
	\item Définir le pH, la constante d'acidité d'un couple acide/base,
	      l'autoprotolyse de l'eau et le produit ionique de l'eau. Justifier les
	      valeurs de $\pk[A](\ce{H_3O+}/\ce{H_2O}) = 0$ et
	      $\pk[A](\ce{H_2O}/\ce{HO-}) = \pk[e] = 14$.
	\item Connaître nom, formule et équation entre acide et base des couples
	      contenant~: acide sulfurique, acide nitrique, acide chlorhydrique, acide
	      phosphorique, acide éthanoïque, acide carbonique, ion ammonium, ion
	      hydroxyde. À partir du lien entre pH et pK$_a$ d'un couple acide-base,
	      justifier et tracer un diagramme de prédominance.
	\item Tracer qualitativement le diagramme de distribution de l'acide
	      carbonique \ce{H2CO3}. Identifier les espèces sur le schéma, indiquer
	      comment lire le pK$_a$ des couples, et le lien entre les concentrations
	      des espèces des couples quand $\pH = \pk[A]$.
	\item On mélange $V_0 = \SI{50}{mL}$ d'une solution d'acide éthanoïque à $c_0
		      = \SI{0.10}{mol.L^{-1}}$, et le même volume d'une solution de nitrite de
	      sodium $\left( \ce{Na+};\ce{NO_2-} \right)$ à la même concentration. On
	      donne
	      \[
		      \pk[A,1] = \pk(\ce{CH_3COOH}/\ce{CH_3COO-}) = \num{4.74}
		      \qet
		      \pk[A,2] = \pk(\ce{HNO_2}/\ce{NO_2-}) = \num{3.2}
	      \]
	      \textbf{Déterminer les concentrations des espèces à l'équilibre et le
		      pH}
\end{enumerate}

\end{document}
