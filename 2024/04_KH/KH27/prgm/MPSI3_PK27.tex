\documentclass[a4paper, 12pt, final, garamond]{book}
\usepackage{cours-preambule}
\usepackage[french]{babel}

\raggedbottom

\makeatletter
\renewcommand{\@chapapp}{Programme de kh\^olle -- semaine}
\makeatother

\begin{document}
\setcounter{chapter}{26}

\chapter{Du 20 au 23 mai}

\section{Cours et exercices}

\subsection(T3){Premier principe de la thermodynamique}
\begin{enumerate}[label=\Roman*]
	\item[b]{Énoncé du premier principe}~: énoncé général, première loi de
	\textsc{Joule}, cas particuliers, premier principe entre deux états voisins.
	\item[b]{Transformation monobare et enthalpie}~: enthalpie et premier
	principe, seconde loi de \textsc{Joule}, capacités thermiques et relations
	associées, calorimétrie.
\end{enumerate}

\subsection(T4){Second principe de la thermodynamique}
\begin{enumerate}[label=\Roman*]
	\item[b]{L'entropie}~: statistique et entropie de \textsc{Boltzmann},
	irréversibilité et causes.
	\item[b]{Second principe}~: énoncé, cas particuliers (cyclique, adiabatique,
	monotherme, polytherme, isentropique)
	\item[b]{Expressions de l'entropie}~: (HP) identités thermodynamique et
	expressions de $\dd{S}$, phases condensées et application au mélange, gaz
	parfait, loi de \textsc{Laplace}.
	\item[b]{Applications}~: méthode de bilans d'entropie, application isochore
	monotherme et isobare monotherme.
\end{enumerate}

\section{Cours uniquement}

\subsection(T5){Machines thermiques}
\begin{enumerate}[label=\Roman*]
	\item[b]{Introduction}~: définition et performance, fonctionnement général et
	inégalité de \textsc{Clausius}, machines monothermes.
	\item[b]{Machines dithermes}~: diagramme de \textsc{Raveau}, moteur ditherme,
	machines frigorigiques et pompes à chaleur, théorèmes de \textsc{Carnot}.
	% \item[b]{Applications}~: cogénération, cycle moteur de \textsc{Carnot} 
\end{enumerate}

% \section*{Thermodynamique chapitre 4 -- Changements d'états}
% \begin{enumerate}[label=\Roman*]
%   \litem{Introduction}~: vocabulaire, observation changement température fixée
%     pour pression fixée, hypothèses de travail.
%   \litem{Diagramme $(P,T)$}~: cas fréquent, cas rare, vocabulaire courbes
%     d'équilibre diphasé, point triple, point critique, pression de vapeur
%     saturante.
%   \litem{Diagramme de \textsc{Clapeyron}}~: expérience compression d'un gaz et
%     changement d'état liquide, forme d'une courbe en $(P,V)$, et bilan
%     isothermes d'\textsc{Andrews}~: courbes de rosée et d'ébullition.
% \end{enumerate}

\newpage


\section{Questions de cours possibles}

\begin{enumerate}[label=\sqenumi]
	\subsection(T3){Premier principe de la thermodynamique}
	\item[s]"2" Énoncer le premier principe de la thermodynamique, en version
	intégrale et différentielle, en détaillant les termes. Préciser lesquels sont
	des fonctions d'état, lesquels ne le sont pas. Étudier les cas particuliers
	des transformations adiabatique, isochore et cyclique. \textbf{Expliquer la
		différence entre adiabatique et isotherme}.

	\item[s]"2" Définir l'enthalpie d'un corps et ses propriétés. Démontrer ensuite
	l'expression du premier principe enthalpique en indiquant ses conditions
	d'application.

	\item[s]"2" Présenter les deux lois de \textsc{Joule}. Rappeler les expressions
	de $U$ et $C_V$ pour un gaz parfait mono- et diatomique. Démontrer la seconde
	loi de \textsc{Joule} pour les phases condensées avec un calcul et pour les
	gaz parfait en donnant l'expression de l'enthalpie molaire en fonction du
	nombre de degré de liberté du gaz.

	\item[s]"1" Définir les capacités thermiques à volume et pression constantes
	dans le cas général. Les relier aux variations $\Delta{U}$ et $\Delta{H}$ pour
	un gaz parfait, et présentez la différence entre ces variations sur un
	diagramme de \textsc{Watt} présentant deux isothermes. Définir le coefficient
	adiabatique $\gamma$, démontrer la relation de \textsc{Mayer} et établir les
	expressions de $C_V$ et de $C_P$.

	\item[s]"1" Calorimétrie~: dans un calorimètre parfaitement isolé de masse en
	eau $m_0 = \SI{24}{g}$, on place $m_1 = \SI{150}{g}$ d'eau à $T_1 =
		\SI{298}{K}$. On ajoute $m_2 = \SI{100}{g}$ de cuivre à $T_2 = \SI{353}{K}$.
	Sachant que $c_{\rm Cu} = \SI{385}{J.K^{-1}.kg^{-1}}$ et $c_{\rm eau} =
		\SI{4185}{J.K^{-1}.kg^{-1}}$, déterminer $T_f$.

	\subsection(T4){Second principe de la thermodynamique}
	\item[s]"2" Définir macro-état, micro-état et nombre de configuration. À
	partir de l'exemple des particules dans l'expérience de \textsc{Joule
		Gay-Lussac}, présenter l'origine statistique de l'irréversibilité en
	traçant l'évolutionde la probabilité des macro-états. Donner la formule de
	\textsc{Boltzmann} et l'interpréter.

	\item[s]"1" Présenter ce qu'on appelle une transformation réversible et
	irréversible et donner des exemples. Énoncer le second principe de la
	thermodynamique. Appliquer le second principe dans les cas particuliers
	des transformations cyclique, adiabatique, mono- et polytherme. Qu'est-ce
	qu'une transformation isentropique~?

	\item[s]"2" On définit $T = \eval{\pdv{U}{S}}_V$ et $P = -\pdv{U}{V}_S$.
	Démontrer alors la première identité de la thermodynamique, puis
	l'expression de la variation d'entropie à partir de la première identité.
	Démontrer l'expression de $\Delta{S}\sup{cond}$ et une expression de
	$\Delta{S}\sup{G.P.}$.

	\item[s]"2" Énoncer les 3 lois de \textsc{Laplace} en précisant leurs
	conditions d'application. Comment qualifier ces transformations en terme
	d'entropie~? À partir d'une expression de l'entropie pour un GP (rappelée
	par l'interrogataire), démontrer l'une d'entre elle. Retrouver les deux
	autres à partir de celle-ci.

	\item[s]"2" Soit un gaz parfait passant de l'état initial $I$ à l'état final
	$F$ en contact avec un thermostat à $T\ind{ext} = T_f$. Pour une
	transformation isochore, déterminer l'entropie créée et tracer
	l'expression obtenue avec $x = \frac{T_i}{T_f}$, et conclure sur la nature
	de la transformation.

	\subsection(T5){Machines thermiques}
	\item[s]"1" Présenter le principe général des machines thermiques grâce à un
	schéma de fonctionnement, et démontrer  les deux relations utiles pour les
	machines à partir du premier et du second principe (inégalité de
	\textsc{Clausius}). Pourquoi ne peut-on pas réaliser de moteur monotherme~?
	Construire le diagramme de \textsc{Raveau} pour les machines dithermes, en
	précisant les domaines des moteurs et des réfrigérateurs.

	\item[s]"2" Présenter le moteur ditherme, le réfrigérateur ou la pompe à
	chaleur (au choix de l'interrogataire), en différenciant les sens
	conventionnel et réel des échanges. Définir son coefficient de performance
	thermodynamique, et établir l'expression du théorème de \textsc{Carnot}
	associé puis un ordre de grandeur des valeurs idéales et réelles.

	% \item Cycle de \textsc{Carnot}~: définir les transformations, traduire le
	%   vocabulaire associé, le dessiner dans un diagramme $(P,V)$, trouver le
	%   travail total (on admet l'expression du travail pour une isotherme
	%   quasi-statique~: $W_{\rm isoT} = nRT_{\rm iso} \ln \left(
	%   V_{i}/V_{f} \right)$), la chaleur échangée et l'expression finale
	%   du rendement.

\end{enumerate}

\end{document}
