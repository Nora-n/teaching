\documentclass[a4paper, 11pt, final, garamond]{book}
\usepackage{cours-preambule}

\raggedbottom

\makeatletter
\renewcommand{\@chapapp}{Programme de kh\^olle -- semaine}
\makeatother

\begin{document}
\setcounter{chapter}{27}

\chapter{Du 27 au 30 mai}

\section{Exercices uniquement}

\subsection(T3){Premier principe de la thermodynamique}
% \begin{enumerate}[label=\Roman*]
% 	\item[b]{Énoncé du premier principe}~: énoncé général, première loi de
% 	\textsc{Joule}, cas particuliers, premier principe entre deux états voisins.
% 	\item[b]{Transformation monobare et enthalpie}~: enthalpie et premier
% 	principe, seconde loi de \textsc{Joule}, capacités thermiques et relations
% 	associées, calorimétrie.
% \end{enumerate}

\section{Cours et exercices}
\subsection(T4){Second principe de la thermodynamique}
\begin{enumerate}[label=\Roman*]
	\item[b]{L'entropie}: statistique et entropie de \textsc{Boltzmann},
	irréversibilité et causes.
	\item[b]{Second principe}: énoncé, cas particuliers (cyclique, adiabatique,
	monotherme, polytherme, isentropique)
	\item[b]{Expressions de l'entropie}: (HP) identités thermodynamique et
	expressions de $\dd{S}$, phases condensées et application au mélange, gaz
	parfait, loi de \textsc{Laplace}.
	\item[b]{Applications}: méthode de bilans d'entropie, application isochore
	monotherme et isobare monotherme.
\end{enumerate}

\subsection(T5){Machines thermiques}
\begin{enumerate}[label=\Roman*]
	\item[b]{Introduction}: définition et performance, fonctionnement général et
	inégalité de \textsc{Clausius}, machines monothermes.
	\item[b]{Machines dithermes}: diagramme de \textsc{Raveau}, moteur ditherme,
	machines frigorifiques et pompes à chaleur, théorèmes de \textsc{Carnot}.
	\item[b]{Applications}: cogénération, cycle moteur de \textsc{Carnot},
	présentation moteur à explosion (cycle de \textsc{Beau de Rochas})
\end{enumerate}

\section{Cours uniquement}
\subsection(T6){Changements d'états}
\begin{enumerate}[label=\Roman*]
	\item[b]{Équilibres diphasés}: rappels états de la matière et vocabulaire des
	transitions de phase, diagramme $(P,T)$ et systèmes monovariants + pression
	de vapeur saturante, diagramme $(P,v)$~: construction d'une isotherme
	d'\textsc{Andrews}, présentation du diagramme, théorème des moments,
	application au stockage des fluides.
	\item[b]{Thermodynamique des transitions de phase}: enthalpies de changement
	d'état, représentation $(T,Q)$ du chauffage d'une masse de glace, méthode de
	résolution et application à la calorimétrie~; entropie de changement d'état
	(démonstration) et application à la calorimétrie.
	\item[b]{Application aux machines thermiques}: présentation de l'intérêt,
	description d'une machine frigorifique et d'une pompe à chaleur (pas de
	calcul).
\end{enumerate}

\begin{center}
	\begin{framed}
		\Large
		Les machines thermiques avec changements d'états simples ou guidés dans leur
		résolution sont autorisés et même \textbf{bienvenus}. En revanche, pas
		d'exercice spécifique sur les équilibres diphasés cette semaine.
	\end{framed}
\end{center}

\newpage

\section{Questions de cours possibles}

\begin{enumerate}[label=\sqenumi]
	\subsection(T4){Second principe de la thermodynamique}
	\item[s]"2" Définir macro-état, micro-état et nombre de configuration. À
	partir de l'exemple des particules dans l'expérience de \textsc{Joule
		Gay-Lussac}, présenter l'origine statistique de l'irréversibilité en
	traçant l'évolutionde la probabilité des macro-états. Donner la formule de
	\textsc{Boltzmann} et l'interpréter.

	\item[s]"1" Présenter ce qu'on appelle une transformation réversible et
	irréversible et donner des exemples. Énoncer le second principe de la
	thermodynamique. Appliquer le second principe dans les cas particuliers
	des transformations cyclique, adiabatique, mono- et polytherme. Qu'est-ce
	qu'une transformation isentropique~?

	\item[s]"2" On définit $T = \eval{\pdv{U}{S}}_V$ et $P = -\eval{\pdv{U}{V}}_S$.
	Démontrer alors la première identité de la thermodynamique, puis
	l'expression de la variation d'entropie à partir de la première identité.
	Démontrer l'expression de $\Delta{S}\sup{cond}$ et une expression de
	$\Delta{S}\sup{G.P.}$.

	\item[s]"2" Énoncer les 3 lois de \textsc{Laplace} en précisant leurs
	conditions d'application. Comment qualifier ces transformations en terme
	d'entropie~? À partir d'une expression de l'entropie pour un GP (rappelée
	par l'interrogataire), démontrer l'une d'entre elle. Retrouver les deux
	autres à partir de celle-ci.

	\item[s]"2" Soit un gaz parfait passant de l'état initial $I$ à l'état final
	$F$ en contact avec un thermostat à $T\ind{ext} = T_f$. Pour une
	transformation isochore, déterminer l'entropie créée et tracer
	l'expression obtenue avec $x = \frac{T_i}{T_f}$, et conclure sur la nature
	de la transformation.

	\subsection(T5){Machines thermiques}
	\item[s]"1" Présenter le principe général des machines thermiques grâce à un
	schéma de fonctionnement, et démontrer les deux relations utiles pour les
	machines à partir du premier et du second principe (inégalité de
	\textsc{Clausius}). Pourquoi ne peut-on pas réaliser de moteur monotherme~?
	Construire le diagramme de \textsc{Raveau} pour les machines dithermes, en
	précisant les domaines des moteurs et des réfrigérateurs.

	\item[s]"2" Présenter le moteur ditherme, le réfrigérateur \textbf{ET} la pompe à
	chaleur, en différenciant les sens conventionnel et réel des échanges. Définir
	les coefficients de performance thermodynamique, et établir l'expression du
	théorème de \textsc{Carnot} pour l'\textbf{UNE} d'entre elle, donner un ordre
	de grandeur des valeurs idéales et réelles pour \textbf{TOUTES} les machines.

	\item[s]"3" Cycle de \textsc{Carnot}~: définir les transformations, traduire le
	vocabulaire associé, le dessiner dans un diagramme $(P,V)$ en précisant et
	justifiant $Q\ind{ch}$ et $Q\ind{fr}$, tracer le schéma de la machine.
	Définir le rendement, exprimer les travaux et transferts thermiques en
	fonction de $\alpha = V\ind{B}/V\ind{A}$, en déduire l'expression finale du
	rendement. Montrer ensuite à l'aide des expressions données de l'entropie
	que ce cycle est réversible par un bilan d'entropie.

	\subsection(T6){Changements d'états}
	\item[s]"2" Présenter les diagrammes $(P,T)$. Pour celui de l'eau, tracer et
	expliquer une transformation isobare à $P = \SI{1}{bar}$ à partir de $T =
		\SI{0}{K}$, et une transformation isotherme à $T = \SI{50}{\degreeCelsius}$.
	Qu'est-ce qu'un système monovariant~? Construire une isotherme
	d'\textsc{Andrews} du diagramme $(P,v)$ en présentant l'expérience de cours,
	et présenter le diagramme $(P,v)$ complet d'un équilibre liquide-gaz.

	\item[s]"1" Énoncer et démontrer le théorème des moments. Présenter
	l'application et les précautions à prendre dans le cas du stockage de
	fluides.

	\item[s]"2" Présenter les variations d'enthalpie et d'entropie lors d'une
	transition de phase. Démontrer la relation concernant l'entropie. Refaire la
	figure de l'évolution de la température d'une masse d'eau commençant à
	$\SI{-20}{\degreeCelsius}$ en fonction de l'énergie apportée. Application~:
	On place $m_0 = \SI{40}{g}$ de glaçons à $T_0 = \SI{0}{\degreeCelsius}$ dans
	$m_1 = \SI{300}{g}$ d'eau à $T_1 = \SI{20}{\degreeCelsius}$ à l'intérieur
	d'un calorimètre de capacité $C = \SI{150}{J.K^{-1}}$. Déterminer la
	température d'équilibre $T_f$, sachant que $c\ind{eau} =
		\SI{4185}{J.K^{-1}.kg^{-1}}$ et $\Delta{h}\ind{fus} = \SI{330}{kJ.kg^{-1}}$,
	puis l'entropie créée. On donne $\Delta{S}\sup{cond} = mc\ln(T_f/T_i)$.
\end{enumerate}
\vspace{-15pt}
\end{document}
