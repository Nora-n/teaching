\documentclass[a4paper, 11pt, final, garamond]{book}
\usepackage{cours-preambule}

\raggedbottom

\makeatletter
\renewcommand{\@chapapp}{Programme de kh\^olle -- semaine}
\makeatother

\begin{document}
\setcounter{chapter}{28}

\chapter{Du 03 au 06 juin}

\section{Exercices uniquement}
\subsection(T4){Second principe}

\section{Cours et exercices}
\subsection(T5){Machines thermiques}
\begin{enumerate}[label=\Roman*]
	\item[b]{Introduction}: définition et performance, fonctionnement général et
	inégalité de \textsc{Clausius}, machines monothermes.
	\item[b]{Machines dithermes}: diagramme de \textsc{Raveau}, moteur ditherme,
	machines frigorifiques et pompes à chaleur, théorèmes de \textsc{Carnot}.
	\item[b]{Applications}: cogénération, cycle moteur de \textsc{Carnot},
	présentation moteur à explosion (cycle de \textsc{Beau de Rochas})
\end{enumerate}

\subsection(T6){Changements d'états}
\begin{enumerate}[label=\Roman*]
	\item[b]{Équilibres diphasés}: rappels états de la matière et vocabulaire des
	transitions de phase, diagramme $(P,T)$ et systèmes monovariants + pression
	de vapeur saturante, diagramme $(P,v)$~: construction d'une isotherme
	d'\textsc{Andrews}, présentation du diagramme, théorème des moments,
	application au stockage des fluides.
	\item[b]{Thermodynamique des transitions de phase}: enthalpies de changement
	d'état, représentation $(T,Q)$ du chauffage d'une masse de glace, méthode de
	résolution et application à la calorimétrie~; entropie de changement d'état
	(démonstration) et application à la calorimétrie.
	\item[b]{Application aux machines thermiques}: présentation de l'intérêt,
	description d'une machine frigorifique et d'une pompe à chaleur (pas de
	calcul).
\end{enumerate}

\section{Cours uniquement}

\subsection(AM3){Solides cristallins}
\begin{enumerate}[label=\Roman*]
	\item[b]{Description d'un cristal parfait}: variétés de solides
	(allotropiques), solides amorphes et cristallins~; modèle du cristal parfait
	et sphères dures~: description d'un cristal (motif, réseau, maille),
	condition de contact des sphères dures et limites du modèle.
	\item[b]{Caractérisation des mailles classiques}: vocabulaire de
	caractérisation (population, coordinence, rayon, compacité, masse
	volumique)~; empilements non compacts (CS et CC)~; empilements compacts (HC
	rapidement, CFC).
	\item[b]{Sites interstitiels}: présentation, sites T, sites O et
	habitabilités.
	\item[b]{Différents types de cristaux}: propriétés macroscopiques à décrire,
	cristaux métalliques et alliages, cristaux ioniques et stabilité (exemples
	\ce{NaCl}, \ce{ZnS}, \ce{CsCl}), cristaux covalents, cristaux moléculaires.
\end{enumerate}

\newpage

\section{Questions de cours possibles}

\begin{enumerate}[label=\sqenumi]
	\subsection(T5){Machines thermiques}
	\item[s]"1" Présenter le principe général des machines thermiques grâce à un
	schéma de fonctionnement, et démontrer les deux relations utiles pour les
	machines à partir du premier et du second principe (inégalité de
	\textsc{Clausius}). Pourquoi ne peut-on pas réaliser de moteur monotherme~?
	Construire le diagramme de \textsc{Raveau} pour les machines dithermes, en
	précisant les domaines des moteurs et des réfrigérateurs.

	\item[s]"2" Présenter le moteur ditherme, le réfrigérateur \textbf{ET} la pompe à
	chaleur, en différenciant les sens conventionnel et réel des échanges. Définir
	les coefficients de performance thermodynamique, et établir l'expression du
	théorème de \textsc{Carnot} pour l'\textbf{UNE} d'entre elle, donner un ordre
	de grandeur des valeurs idéales et réelles pour \textbf{TOUTES} les machines.

	\item[s]"3" Cycle de \textsc{Carnot}~: définir les transformations, traduire le
	vocabulaire associé, le dessiner dans un diagramme $(P,V)$ en précisant et
	justifiant $Q\ind{ch}$ et $Q\ind{fr}$, tracer le schéma de la machine.
	Définir le rendement, exprimer les travaux et transferts thermiques en
	fonction de $\alpha = V\ind{B}/V\ind{A}$, en déduire l'expression finale du
	rendement. Montrer ensuite à l'aide des expressions données de l'entropie
	que ce cycle est réversible par un bilan d'entropie.

	\subsection(T6){Changements d'états}
	\item[s]"2" Présenter les diagrammes $(P,T)$. Pour celui de l'eau, tracer et
	expliquer une transformation isobare à $P = \SI{1}{bar}$ à partir de $T =
		\SI{0}{K}$, et une transformation isotherme à $T = \SI{50}{\degreeCelsius}$.
	Qu'est-ce qu'un système monovariant~? Construire une isotherme
	d'\textsc{Andrews} du diagramme $(P,v)$ en présentant l'expérience de cours,
	et présenter le diagramme $(P,v)$ complet d'un équilibre liquide-gaz.

	\item[s]"1" Énoncer et démontrer le théorème des moments. Présenter
	l'application et les précautions à prendre dans le cas du stockage de
	fluides.

	\item[s]"2" Présenter les variations d'enthalpie et d'entropie lors d'une
	transition de phase. Démontrer la relation concernant l'entropie. Refaire la
	figure de l'évolution de la température d'une masse d'eau commençant à
	$\SI{-20}{\degreeCelsius}$ en fonction de l'énergie apportée. Application~:
	On place $m_0 = \SI{40}{g}$ de glaçons à $T_0 = \SI{0}{\degreeCelsius}$ dans
	$m_1 = \SI{300}{g}$ d'eau à $T_1 = \SI{20}{\degreeCelsius}$ à l'intérieur
	d'un calorimètre de capacité $C = \SI{150}{J.K^{-1}}$. Déterminer la
	température d'équilibre $T_f$, sachant que $c\ind{eau} =
		\SI{4185}{J.K^{-1}.kg^{-1}}$ et $\Delta{h}\ind{fus} = \SI{330}{kJ.kg^{-1}}$,
	puis l'entropie créée. On donne $\Delta{S}\sup{cond} = mc\ln(T_f/T_i)$.
	\subsection(AM3){Solides cristallins}
	\item[s]"1" Présenter le modèle du cristal parfait de sphères dures (condition de
	tangence). Réaliser alors la caractérisation des mailles cubiques simple et
	centrée (population, coordinence, rayon atomique, compacité, masse
	volumique).

	\item[s]"3" Présenter la maille cubique faces centrées, puis réaliser sa
	caractérisation. Présenter et justifier alors l'existence des sites
	interstitiels. Donner les positions et la population des sites T et O de la
	structure CFC, et déterminer leurs habitabilités.
	%  \textbf{Application}~: le fer $\gamma$ est une
	% variété allotropique du fer, cristallisant dans une structure CFC. Sa masse
	% volumique vaut $\rho = \SI{8.21e3}{kg.m^{-3}}$. Déterminer le paramètre de
	% la maille $a$ et de rayon $r$ des atomes de fer dans la structure. On donne
	% $M_{\ce{Fe}} = \SI{56}{g.mol^{-1}}$ et $\mathcal{N}_A =
	%  \SI{6.02e23}{mol^{-1}}$.

	\item[s]"2" Présenter la définition, puis les propriétés microscopiques et leur
	correspondance macroscopiques de \textbf{deux types de cristaux parmi} les
	suivants~:
	\begin{tasks}[label=\bdmd](4)
		\task Cristal métallique~;
		\task Cristal ionique~;
		\task Cristal covalent~;
		\task Cristal moléculaire.
	\end{tasks}

	\item[s]"2" Donner le critère de stabilité des cristaux ioniques. Donner la
	population/formule chimique, la coordinence, et démontrer le critère de
	stabilité d'un \textbf{ou plusieurs} des cristaux suivants~:
	\begin{tasks}[label=\bdmd](3)
		\task Le chlorure de césium~;
		\task Le chlorure de sodium~;
		\task La blende (sulfure de zinc).
	\end{tasks}
\end{enumerate}

\end{document}
