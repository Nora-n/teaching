\documentclass[a4paper, 12pt, final, garamond]{book}
\usepackage{cours-preambule}
\usepackage[french]{babel}

\raggedbottom

\makeatletter
\renewcommand{\@chapapp}{Khôlles}
\renewcommand\thechapter{\!\!}
\makeatother

\begin{document}
\setcounter{chapter}{0}

\chapter*{Consignes des khôlles}

\Large
\begin{itemize}
	\item Une question de cours non connue/complètement ratée implique une note
	      sous la moyenne. \textbf{Il faudra alors me présenter vos fiches des
		      questions de cours du chapitre concerné par votre question de la semaine
		      concernée sur le coin de votre table à l'interrogation du lundi suivant}.
	\item Vous devez avoir votre calculatrice \textbf{personnelle} pendant les
	      khôlles.
	\item Vous devez avoir votre \textbf{feuille de khôlle} avec vous~! Un malus
	      pourra être appliqué sinon.
	\item Prévenez vos absences tant que possible.
	\item Faites en sorte de rattraper au plus vite vos khôlles. Les courriels des
	      khôllaires sont disponibles sur Cahier de Prépa. \textbf{Mettez-moi en
		      copie} du courriel. Donnez directement vos prochaines disponibilités.
	      Répondez rapidement et jusqu'au bout de l'échange.
\end{itemize}

\begin{tcn}[cnt, bld](prop)"bomb"{Téléphones en khôlles}
	Tout-e étudiant-e qui aurait son téléphone ou une montre connectée sur ellui
	ou pire, dans sa main/allumée pendant la khôlle, sera considéré-e comme ayant
	tenté de frauder.
	\bigbreak
	Conséquence immédiate~: arrêt de la khôlle, 0/20 et rapport à l'ensemble du
	corps éducatif.
\end{tcn}

\end{document}
