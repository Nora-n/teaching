\documentclass[a4paper, 10pt, final, garamond]{book}
\usepackage{cours-preambule}
\graphicspath{{./figures/}}
\addto\captionsfrench{\renewcommand{\figurename}{Fig.}}

\makeatletter
\renewcommand{\@chapapp}{Contrôle de connaissances}
\makeatother

% \toggletrue{student}
% \toggletrue{corrige}
\renewcommand{\mycol}{black}
% \renewcommand{\mycol}{gray}

\hfuzz=5.002pt

\begin{document}
\setcounter{chapter}{10}

\settype{enon}
\settype{solu}

\chapter{Électrocinétique en RSF~: oscillateurs et filtrage\ifstudent{~(15')}}

\begin{enumerate}[label=\sqenumi]
	\item[n]{10} %
	      À partir de $U(x) = \frac{E_0}{\sqrt{\pa{1-x^2}^2 + \pa{\frac{x}{Q}}^2}}$,
	      démontrer la condition de résonance ainsi que l'amplitude à la résonance.
	      \vspace{-15pt}
	      \begin{tcb}(demo)<non>""{Résonance en tension du RLC série}
		      \begin{center}
			      \fatbox{\textbf{Condition de résonance}}
		      \end{center}
		      \vspace{-15pt}
		      \begin{isd}[interior hidden]
			      \psw{%
				      \begin{DispWithArrows*}[groups, fleqn, mathindent=-25pt]
					      U(x_r) &\stm{=} U_{\max}
					      \Arrow{$X=x^2$}
					      \\\Lra
					      f(X_r) &\stm{=} \left(1 - X_r \right)^2 + \dfrac{X_r}{Q^2}
					      \quad
					      \text{min.}
					      % \Arrow{$X=x^2$\\$f(X) = \pa{1-X}^2 + \frac{X}{Q^2}$}
					      \\\Lra
					      f'(X_r) &= 0
					      \Arrow[new-group]{On dérive}
					      \\\Lra
					      -2 \left( 1-X_r \right) + \frac{1}{Q^2} &\stm{=} 0
				      \end{DispWithArrows*}
			      }%
			      \tcblower
			      \psw{%
				      \begin{DispWithArrows*}[groups]
					      \Lra
					      X_r-1 = - \frac{1}{2Q^2}
					      & \Lra
					      X_r = 1 - \frac{1}{2Q^2}
					      \Arrow{$X_r = x^{2}$}
					      \\\Lra
					      x_r = \sqrt{1 - \frac{1}{2Q^2}}
					      & \Lra
					      \boxed{
						      x_r \stm{=} \frac{1}{Q} \sqrt{Q^{2} - \frac{1}{2}}
					      }
					      \Arrow{$x_r \in \Rb$}
					      \\\Ra
					      x_r > 0 & \Lra \boxed{Q \stm{>} \frac{1}{\sqrt{2}}}
				      \end{DispWithArrows*}
			      }
		      \end{isd}
		      \tcblower
		      \begin{center}
			      \fatbox{\textbf{Amplitude de résonance}}
		      \end{center}
		      \noindent
		      \begin{isd}[righthand ratio=.4]
			      \vspace{-15pt}
			      \psw{%
				      \begin{DispWithArrows*}[fleqn, mathindent=5pt, groups]
					      f(X_r) &\stm{=}
					      \left( \cancel{1} -
					      \left( \cancel{1} - \frac{1}{2Q^{2}} \right)
					      \right)^{2} +
					      \frac{1}{Q^{2}} \left( 1 - \frac{1}{2Q^{2}} \right)
					      % \Arrow{On simplifie et développe}
					      \\\Lra
					      f(X_r) &=
					      \left( \frac{1}{2Q^{2}} \right)^{2} + \frac{1}{Q^{2}} -
					      \frac{1}{2Q^{4}}
					      \Arrow{Même dénom.}
					      \\\Lra
					      f(X_r) &\stm{=}
					      \frac{1}{4Q^{4}} - \frac{2}{4Q^{4}} + \frac{1}{Q^{2}}
					      \Arrow{On calcule}
					      \\\Lra
					      f(X_r) &=
					      \frac{1}{Q^{2}} - \frac{1}{4Q^{4}}
					      \Arrow{On factorise}
					      \\\Lra
					      \Aboxed{%
						      f(X_r) &\stm{=}
						      \frac{1}{Q^{2}}\left( 1 - \frac{1}{4Q^{2}} \right)
					      }
				      \end{DispWithArrows*}
			      }
			      \vspace{-15pt}
			      \tcblower
			      \vspace{-15pt}
			      \psw{%
				      \begin{DispWithArrows*}[fleqn, mathindent=5pt]
					      \Ra
					      U(x_r) &\stm{=} \frac{E_0}{\sqrt{f(X_r)}}
					      \CArrow{$\sqrt{(\cdot)}$}
					      \\\Lra
					      U(x_r) &= \frac{E_0}{
						      \frac{1}{Q}\sqrt{1 - \frac{1}{4Q^{2}}}
					      }
					      \Arrow{Simplifie}
					      \\\Lra
					      \Aboxed{
						      U(x_r) &\stm{=} \frac{QE_0}{\sqrt{1 - \frac{1}{4Q^{2}}}}
					      }
				      \end{DispWithArrows*}
			      }
		      \end{isd}
		      \vspace{-15pt}
	      \end{tcb}
	      \vspace*{\fill}
	\item[n]{5} %
	      Montrer que diviser l'amplitude par 10 revient à réduire le gain en
	      décibel de \SI{20}{dB}. Montrer ensuite qu'on trouve la bande passante
	      d'un filtre en trouvant les $\w$ tels que $G\ind{dB}(\w) \geq
		      G\ind{dB,max} - \SI{3}{dB}$.
	      \smallbreak
	      \noindent
	      \begin{isd}
		      \vspace{-15pt}
		      \psw{%
			      \begin{align*}
				      \abs{\Hu(\w_2)}                        & \stm{=} \frac{\abs{\Hu(\w_1)}}{10}
				      \\\Lra
				      20 \log \left( \abs{\Hu(\w_2)} \right) & =
				      20 \log (\frac{\abs{\Hu(\w_1)}}{10})
				      \\\Lra
				      20 \log \left( \abs{\Hu(\w_2)} \right) & =
				      20 \log (\abs{\Hu(\w_1)}) - 20 \log (10)
				      \\\Lra
				      G\ind{dB}(\w_2)                        & \stm{=} G\ind{dB}(\w_1) - \SI{20}{dB}
				      \qed
			      \end{align*}
		      }%
		      \vspace{-15pt}
		      \tcblower
		      \vspace{-15pt}
		      \psw{%
			      \small
			      \begin{align*}
				      \abs{\Hu(\w)}           & \stm{\geq} \frac{\abs{\Hu}_{\max}}{\sqrt{2}}
				      \\\Lra
				      20 \log (\abs{\Hu(\w)}) & \geq 20 \log (\frac{\abs{\Hu}_{\max}}{\sqrt{2}})
				      \\\Lra
				      G\ind{dB}(\w)           & \stm{\geq}
				      \underbracket[1pt]{20 \log (\abs{\Hu}_{\max})}_{= G\ind{dB, max}} -
				      \underbracket[1pt]{20 \log (\sqrt{2})}_{\approx \SI{3}{dB} \pt{1}}
				      \qed
			      \end{align*}
			      \vspace{-15pt}
		      }%
	      \end{isd}
	\item[n]{5} %
	      À partir de $\Hu(x) = \frac{1}{1+\jx}$, déterminer les asymptotes de
	      $G\ind{dB}(x)$ et $\f(x)$.
	      \smallbreak
	      \vspace{-15pt}
	      \psw{%
		      \begin{DispWithArrows*}[fleqn, mathindent=40pt]
			      \Hu(x) \Sim_{x\to 0}^{\pt{1}} \frac{1}{1 + 0} = 1
			      \quad & \text{et} \quad
			      \Hu(x) \Sim_{x\to \infty}^{\pt{1}} \frac{1}{\jx}
			      \Arrow{$G\ind{dB}(x) = 20 \log \abs{\Hu(x)}$}
			      \Arrow[jump=2,tikz-code=
					      {\draw
						      (#1) --++
						      (4.0cm, 0) |-
						      node[pos=0.75, anchor=south] {#3}
						      (#2) ;
					      }]{$\f(x) = \arg*{\Hu(x)}$}
			      \\\Ra
			      G\ind{dB}(x) \Sim_{x\to 0}^{\pt{1}} 20 \log (1) = 0
			      \quad & \text{et} \quad
			      G\ind{dB}(x) \Sim_{x\to\infty}^{\pt{1}}
			      20 \log \abs{\frac{1}{\jx}} = -20 \log x
			      \\\Ra
			      \f(x) \Sim_{x\to 0} \arg*{1} = 0
			      \quad & \stm{\text{et}} \quad
			      \f(x) \Sim_{x\to \infty} \arg*{\frac{1}{\jx}} = -\frac{\pi}{2}
		      \end{DispWithArrows*}
	      }%
\end{enumerate}

\ifstudent{
	\begin{tikzpicture}[remember picture, overlay]
		\node[anchor=north west, align=left]
		at ([shift={(1.4cm,0)}]current page.north west)
		{\\[5pt]\Large\bfseries \textsc{Nom}~:\\[10pt]\Large\bfseries Prénom~:};
		\node[anchor=north east, align=right]
		at ([shift={(-1.5cm,-17pt)}]current page.north east)
		{\Large\bfseries Note~:\hspace{1cm}/20};
	\end{tikzpicture}
}
\end{document}
