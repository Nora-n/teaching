\documentclass[a4paper, 10pt, final, garamond]{book}
\usepackage{cours-preambule}
\graphicspath{{./figures/}}

\makeatletter
\renewcommand{\@chapapp}{Contr\^ole de connaissances}
\makeatother

\toggletrue{student}
% \HideSolutionstrue
% \toggletrue{corrige}
\renewcommand{\mycol}{black}

\begin{document}
\setcounter{chapter}{7}

\chapter{Cinétique chimique \ifstudent{ (10')}}

\begin{enumerate}[label=\sqenumi, leftmargin=10pt]
	\nitem{1}%
	\leavevmode\vspace*{-\dimexpr\baselineskip+\abovedisplayskip\relax-10pt}
	\begin{gather*}
		\beforetext{Soit}
		\ce{\alpha_1R_1 + \alpha_2R_2} +…
		=
		\ce{\beta_1P_1 +\beta_2P_2}+…
		\Lra
		0 = \sum_{i}^{N}\nu_i \ce{X_i}
	\end{gather*}
	Donne l'expression de la vitesse de réaction, la vitesse de formation d'un
	produit et de disparition d'un réactif, et donner le lien entre vitesse de
	réaction et la concentration d'un constituant quelconque $[\ce{X_i}]$.
	\smallbreak
	\vspace{-15pt}
	\wsw{
		\begin{gather*}
			\boxed{v = \dv{x}{t}}
			\qquad ; \qquad
			\boxed{v_{f,\ce{P}} = \dv{[\ce{P}]}{t}}
			\qet
			\boxed{v_{f,\ce{R}} = -\dv{[\ce{R}]}{t}}
			\qquad \Ra \qquad
			\boxed{v = \frac{1}{\nu_i} \dv{[\ce{X}_i]}{t}}
		\end{gather*}
	}
	\nitem{2}%
	Qu'est-ce qu'une réaction admettant un ordre~? Donne un exemple pour une
	réaction $\ce{aA + bB -> cC + dD}$. Détailler les termes de vocabulaire
	nécessaires. Définir en une phrase ce qu'est une réaction simple. Que se
	passe-t-il pour si la réaction précédente est simple~?
	\smallbreak
	\wsw{
	Une réaction admet un ordre si sa vitesse peut s'écrire sous la forme
	\[
		\boxed{v = k[\ce{A}]^p[\ce{B}]^q}
	\]
	avec $k$ la constante de vitesse, $p$ et $q$ sont les \textbf{ordres partiels}
	de la réaction, $m = p + q$ est \textbf{l'ordre global} de la réaction.
	\smallbreak
	Une réaction est simple si elle décrit un unique acte chimique. Dans ce cas,
	les ordres partiels de la réaction sont les coefficients stœchiométriques
	arithmétiques~:
	\[
		\boxed{v = k [\ce{A}]^{a} [\ce{B}]^{b}}
	\]
	}
	\nitem{5}%
	Décrire en une phrase ce qu'est la dégénérescence de l'ordre. Démontrer son
	intérêt pour l'étude d'une loi de vitesse sur l'exemple $\ce{aA + bB -> cC +
			dD}$. Démontrer l'intérêt de mettre les réactifs en proportions
	stœchiométriques.
	\smallbreak
	\wsw{
	La dégénérescence de l'ordre consiste à mettre tous les réactifs en excès
	sauf un. Par exemple, si \ce{A} est en excès, alors $[\ce{A}](t) \approx
		[\ce{A}]_0$~; ainsi
	\[
		\boxed{
		v =
		k[\ce{A}]^{p}[\ce{B}]^{q} =
		\underbracket[1pt]{k[\ce{A}]_0{}^{p}}_{= \cte}[\ce{B}]^{q} =
		k\ind{app}[\ce{B}]^{q}
		}
	\]
	et on peut trouver l'ordre partiel en \ce{B}. Si les réactifs sont en
	proportions stœchiométriques, on aura
	\[
		[{\ce{A}}]_0 = ac_0
		\qet
		[{\ce{B}}]_0 = bc_0
		\Ra
		\boxed{
			[{\ce{A}}] = a (c_0 - x)
			\qet
			[{\ce{B}}] = b (c_0 - x)
		}
	\]
	On peut donc factoriser la loi de vitesse~:
	\begin{gather*}
		v =
		k \left( a (c_0 - x)\right)^{p}\left( b (c_0 - x)\right)^{q}
		\Lra
		v =
		ka^{p}b^{q} \left(c_0 - x\right)^{p + q}
		\Lra
		\boxed{v = k_{\rm app} (c_0 -x)^{m}}
	\end{gather*}
	avec $m = p + q$ l'ordre global, et $k\ind{app} = ka^{p}b^{q}$ la constante
	apparente. On a donc accès à l'ordre global.
	}
	\vfill
	\nitem{2}%
	Donner l'équation différentielle d'une réaction $\ce{aA + bB -> cC + dD}$
	d'ordre 2 par rapport au réactif \ce{A}. La résoudre sous la forme
	$1/[\ce{A}]$.
	\smallbreak
	\vspace{-15pt}
	\wsw{
		\begin{gather*}
			\dv{[{\ce{A}}]}{t} = -av
			\Lra
			\boxed{\dv{[{\ce{A}}]}{t} = -ka[{\ce{A}}]^2}
			\\\Lra
			- \frac{\dd{[\ce{A}]}}{[{\ce{A}}]^2} = ka \dd{t}
			\Ra
			\frac{1}{[{\ce{A}}]} = kat + \frac{1}{[\ce{A}]_0}
		\end{gather*}
	}
\end{enumerate}

\ifstudent{
	\begin{tikzpicture}[remember picture, overlay]
		\node[anchor=north west, align=left]
		at ([shift={(1.4cm,0)}]current page.north west)
		{\\[5pt]\Large\bfseries Nom~:\\[10pt]\Large\bfseries Prénom~:};
		\node[anchor=north east, align=right]
		at ([shift={(-1.5cm,-17pt)}]current page.north east)
		{\Large\bfseries Note~:\hspace{1cm}/10};
	\end{tikzpicture}
}

\end{document}
