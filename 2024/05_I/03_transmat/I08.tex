\documentclass[a4paper, 10pt, final, garamond]{book}
\usepackage{cours-preambule}
\graphicspath{{./figures/}}

\makeatletter
\renewcommand{\@chapapp}{Contr\^ole de connaissances}
\makeatother

\toggletrue{student}
% \HideSolutionstrue
% \toggletrue{corrige}
\renewcommand{\mycol}{black}

\begin{document}
\setcounter{chapter}{7}

\chapter{Transformation et équilibre chimique\ifstudent{ (10')}}

\begin{enumerate}[label=\sqenumi, leftmargin=10pt]
	\nitem{1.5}%
	Expliquer en trois phrases succinctes la différence entre avancement final,
	avancement à l'équilibre et avancement maximal. Aucune comparaison
	mathématique sur des $\xi$ n'est attendue.
	\smallbreak
	\wsw{
		Toute réaction chimique atteint un état final, caractérisé par un avancement
		final. Dans certains cas, on atteint un équilibre~: les réactions en sens
		direct et indirect se compensent et il reste des réactifs et des produits.
		Si à l'état final on a consommé en totalité un (ou plusieurs) réactifs, on
		atteint l'avancement \textbf{maximal}~: la réaction ne peut plus avancer mais
		ce n'est pas un équilibre.
	}
	% \nitem{1}%
	% Déterminer, à l'aide d'un tableau d'avancement, la composition à l'état final
	% de la réaction totale de la combustion de \SI{2.00}{mol} d'éthanol dans l'air.
	% On précise que les réactifs sont introduits dans les proportions
	% stœchiométriques, et que le dioxygène provient de l'air (20\% \ce{O2} et 80\%
	% \ce{N2} en mole). Quelle est la pression finale\ftn{On rappelle que $R =
	% 	\SI{8.314}{J.K^{-1}.mol^{-1}}$.} pour $V = \SI{1.00}{m^3}$ et $T =
	% 	\SI{293}{K}$~?
	% \[
	% 	\ce{C_2H_5OH\liq{} + 3O_2\gaz{} -> 2CO_2\gaz{} + 3H_2O\gaz{}}
	% \]
	% \wsw{
	% On a $n_{\ce{O_2}}^{0} = \frac{3}{1}n_{\ce{C_2H_5OH}}^{0} = \SI{6.00}{mol}$.
	% On a également $n_{\ce{N_2}} =
	% 	\frac{x_{\ce{N_2}}}{x_{\ce{O_2}}}n_{\ce{O_2}}^{0} = \SI{24.00}{mol}$. On part
	% sans $\ce{CO2\gaz{}}$ ou $\ce{H2O\gaz{}}$.
	% \begin{center}
	% 	\def\rhgt{0.35}
	% 	\centering
	% 	\begin{tabularx}{\linewidth}{|l|c||YdYdYdY||Y|Y|}
	% 		\hline
	% 		\multicolumn{2}{|c||}{
	% 			$\xmathstrut{\rhgt}$
	% 		\textbf{Équation} (\si{mol})} &
	% 		$\ce{C_2H_2OH\liq{}}$         & $+$                       &
	% 		$3\ce{O_2\gaz{}}$             & $\ra$                     &
	% 		$2\ce{CO_2\gaz{}}$            & $+$                       &
	% 		$3\ce{H_2O\gaz{}}$            &
	% 		$n_{\ce{N_2}}$                &
	% 		$n_{\tot, gaz}$                                             \\
	% 		\hline
	% 		$\xmathstrut{\rhgt}$
	% 		Initial                       & $\xi = 0$                 &
	% 		$\num{2.00}$                  & \vline                    &
	% 		$\num{6.00}$                  & \vline                    &
	% 		$\num{0.00}$                  & \vline                    &
	% 		$\num{0.00}$                  &
	% 		$\num{24.00}$                 &
	% 		$\num{30.00}$                                               \\
	% 		\hline
	% 		$\xmathstrut{\rhgt}$
	% 		Interm.                       & $\xi$                     &
	% 		$\num{2.00} - \xi$            & \vline                    &
	% 		$\num{6.00} - 3\xi$           & \vline                    &
	% 		$2\xi$                        & \vline                    &
	% 		$3\xi$                        &
	% 		$\num{24.00}$                 &
	% 		$\num{30.00} + 2\xi$                                        \\
	% 		\hline
	% 		$\xmathstrut{\rhgt}$
	% 		Final                         & $\xi_{\max} = \num{2.00}$ &
	% 		$\num{0.00}$                  & \vline                    &
	% 		$\num{0.00}$                  & \vline                    &
	% 		$\num{4.00}$                  & \vline                    &
	% 		$\num{6.00}$                  &
	% 		$\num{24.00}$                 &
	% 		$\num{34.00}$                                               \\
	% 		\hline
	% 	\end{tabularx}
	% \end{center}
	% \begin{gather*}
	% 	\boxed{p_f = \frac{n_{\tot, gaz}RT}{V}}
	% 	\Lra
	% 	\xul{
	% 		p_f = \SI{8.28e-4}{Pa} = \SI{0.828}{bar}
	% 	}
	% \end{gather*}
	% }
	\vfill
	\nitem{8.5}%
	On travaille dans une enceinte initialement vide de tout gaz, et de volume $V
		= \SI{10}{L}$. On insère $n_{\ce{BaO_2}}^{0} = \SI{0.10}{mol}$ de peroxyde de
	baryum qui suit la réaction de dissociation suivante~:
	\smallbreak
	\centersright{%
		$\ce{2BaO2\sol{} <=> 2BaO\sol{} + O2\gaz{}}$%
	}{%
		$K\degree(\SI{795}{\degreeCelsius}) = \num{0.50}$%
	}
	\begin{enumerate}
		\item Dresser un tableau d'avancement et remplir les deux premières lignes.
		\item Déterminez la quantité de matière $n_{\ce{O2}, \eq}$ qui permet
		      d'atteindre l'équilibre\ftn{On rappelle que $R =
			      \SI{8.314}{J.K^{-1}.mol^{-1}}$.}.
		\item Déterminer le sens d'évolution du système.
		\item Déterminer $\xi_f$ et remplir la dernière ligne du tableau. Comment
		      s'appelle cette situation finale~?
	\end{enumerate}
	\wsw{
		\begin{enumerate}
			\item
			      \begin{center}
				      \def\rhgt{0.50}
				      \centering
				      \begin{tabularx}{\linewidth}{|l|c||YdYdY||Y|}
					      \hline
					      \multicolumn{2}{|c||}{
						      $\xmathstrut{\rhgt}$
					      \textbf{Équation}}  &
					      $2\ce{BaO_2\sol{}}$ & $\rightleftharpoons$ &
					      $2\ce{BaO\sol{}}$   & $+$                  &
					      $\ce{O_2\gaz{}}$    &
					      $n_{\tot, gaz}$                              \\
					      \hline
					      $\xmathstrut{\rhgt}$
					      Initial (\si{mol})  & $\xi = 0$            &
					      $\num{0.10}$        & \vline               &
					      $\num{0.00}$        & \vline               &
					      $\num{0.00}$        &
					      $\num{0.00}$                                 \\
					      \hline
					      $\xmathstrut{\rhgt}$
					      Interm. (\si{mol})  & $\xi$                &
					      $\num{0.10} - 2\xi$ & \vline               &
					      $2\xi$              & \vline               &
					      $\xi$               &
					      $\xi$                                        \\
					      \hline
					      $\xmathstrut{\rhgt}$
					      Final  (\si{mol})   & $\xi_f = \xi_{\max}$ &
					      $\num{0.00}$        & \vline               &
					      $\num{0.10}$        & \vline               &
					      $\num{0.05}$        &
					      $\num{0.05}$                                 \\
					      \hline
				      \end{tabularx}
			      \end{center}
			\item Avec la loi des gaz parfaits et la loi d'action des masses, on a
			      \begin{DispWithArrows*}
				      p_{\ce{O2},\eq}V = n_{\ce{O2}, \eq}RT
				      &\Lra
				      n_{\ce{O2}, \eq} = \frac{p_{\ce{O2},\eq}V}{RT}
				      \Arrow{$K\degree = \frac{P_{\ce{O_2}, \eq}}{p\degree}$}
				      \\\Lra
				      \boxed{n_{\ce{O_2}, \eq} = \frac{K\degree p\degree V}{RT}}
				      &\qavec
				      \left\{
				      \begin{array}{rcl}
					      K\degree & = & \num{0.50}      \\
					      p\degree & = & \SI{1.00e5}{Pa} \\
					      V        & = & \SI{10e-3}{m^3} \\
					      T        & = & \SI{1068.15}{K}
				      \end{array}
				      \right.\\
				      \makebox[0pt][l]{%
					      $\phantom{\AN}\xul{\phantom{n_{\ce{O2}, \eq} = \SI{0.056}{mol}}}$%
				      }
				      \AN
				      n_{\ce{O2}, \eq} = \SI{0.056}{mol}
			      \end{DispWithArrows*}
			      \vspace{10pt}
			      \mitem \[
				      \boxed{
					      Q_{r,0} = \frac{\overbracket[1pt]{p_{\ce{O2}, 0}}^{=0}}
					      {p\degree} = 0}
			      \]
			      On a donc $\xul{Q_{r,0} < K\degree}$, et l'évolution se fait en
			      \xul{sens direct}.
			\item S'il y a équilibre, ça veut dire que $n_{\ce{O2}, \eq} =
				      \SI{0.056}{mol}$ comme déterminé au début. Or, le tableau nous
			      indique que $n_{\ce{O2}, f} = \xi_f$, donc si c'est un équilibre
			      \xul{$\xi_{\eq} = \SI{0.056}{mol}$}.
			      \smallbreak
			      L'avancement est maximal si \ce{BaO2} est limitant~: on trouve donc
			      $\xi_{\max}$ en résolvant $\num{0.10} - 2\xi_{\max} = 0$,
			      c'est-à-dire \xul{$\xi_{\max} = \SI{0.050}{mol}$}.
			      \smallbreak
			      La valeur finale $\xi_f$ est la plus petite valeur (en valeur
			      absolue) de $\xi_{\eq}$ et $\xi_{\max}$~; or ici $\xi_{\eq} >
				      \xi_{\max}$~: il n'y a donc \textbf{pas équilibre}, et on a
			      \[\xul{\xi_f = \xi_{\max} = \SI{0.050}{mol}}\]
		\end{enumerate}
	}
\end{enumerate}
\vfill

\ifstudent{
	\begin{tikzpicture}[remember picture, overlay]
		\node[anchor=north west, align=left]
		at ([shift={(1.4cm,0)}]current page.north west)
		{\\[5pt]\Large\bfseries Nom~:\\[10pt]\Large\bfseries Prénom~:};
		\node[anchor=north east, align=right]
		at ([shift={(-1.5cm,-17pt)}]current page.north east)
		{\Large\bfseries Note~:\hspace{1cm}/10};
	\end{tikzpicture}
}

\end{document}
