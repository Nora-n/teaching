\documentclass[a4paper, 10pt, final, garamond]{book}
\usepackage{cours-preambule}
\graphicspath{{./figures/}}
\addto\captionsfrench{\renewcommand{\figurename}{Fig.}}

\makeatletter
\renewcommand{\@chapapp}{Contr\^ole de connaissances}
\makeatother

% \toggletrue{student}
% \toggletrue{corrige}
\renewcommand{\mycol}{black}
% \renewcommand{\mycol}{gray}

\begin{document}
\setcounter{chapter}{7}

\settype{enon}
\settype{solu}

\chapter{Transformation et cinétique chimiques\ifstudent{~(15')}}

\begin{enumerate}[label=\sqenumi, leftmargin=10pt]
	\item[n]{3}%
	      Expliquer en trois phrases succinctes la différence entre avancement
	      final, maximal et à l'équilibre. Aucune comparaison mathématique sur des
	      $\xi$ n'est attendue.
	      \smallbreak
	      \psw{%
		      Toute réaction chimique atteint un état final, caractérisé par un
		      avancement final. Si au moins un réactif est complètement consommé,
		      alors l'avancement final est maximal~: $\xi_f = \xi\ind{max}$. Sinon,
		      si réactifs et produits coexistent, l'avancement final est celui
		      d'équilibre~: $\xi_f = \xi\ind{eq}$.
	      }%
	\item[n]{8}%
	      \leftcentersright{%
	      \hspace{-12pt}
	      Soit la synthèse de l'ammoniac~:
	      }{%
	      $\ce{N_2_{\gaz} + 3H_2_{\gaz} = 2NH_3_{\gaz}}$
	      }{%
	      $K^\circ = \num{0.5}$
	      }%
	      \smallbreak
	      \vspace{-15pt}
	      avec $n_{\ce{N_2},0} = \SI{3}{mol}$, $n_{\ce{H_2},0} = \SI{5}{mol}$ et
	      $n_{\ce{NH_3},0} = \SI{2}{mol}$. \textbf{Dresser le tableau
		      d'avancement} dans les états initial et intermédiaire.
	      \textbf{Exprimer le quotient réactionnel initial} en passant d'abord par
	      les activités. Indiquer comment on pourrait déterminer le sens d'évolution
	      de la réaction.
	      \smallbreak
	      \psw{%
	      \def\rhgt{0.50}
	      \begin{tabularx}{.8\linewidth}{|l|c|YdYdYdY|}
		      \hline
		      \multicolumn{2}{|c|}{
			      $\xmathstrut{\rhgt}$
		      \textbf{Équation}} \pt{1}+\pt{1} &
		      $\ce{N_2_{\gaz}}$                & $+$       &
		      $3\ce{H_2_{\gaz}}$               & $=$       &
		      $2\ce{NH_3_{\gaz}}$              & \vline    &
		      $n\ind{tot,gaz}$
		      \pt{1}                                         \\
		      \hline
		      $\xmathstrut{\rhgt}$
		      Initial (\si{mol})               & $\xi = 0$ &
		      $3$                              & \vline    &
		      $5$                              & \vline    &
		      $2$                              & \vline    &
		      $10$
		      \tikzmark{MA}                                  \\
		      \hline
		      $\xmathstrut{\rhgt}$
		      Interm.  (\si{mol})              & $\xi$     &
		      $3 - \xi$                        & \vline    &
		      $5 - 3\xi$                       & \vline    &
		      $2 + 2\xi$                       & \vline    &
		      $10 - 2\xi$
		      \tikzmark{MB}                                  \\
		      \hline
	      \end{tabularx}
	      \smallbreak
	      \[
		      Q_{r,0} \stm{=}
		      \frac{%
		      a(\ce{NH_3_{\gaz}})_0{}^2
		      }{%
		      a(\ce{H_2_{\gaz}})_0{}^3 \cdot a (\ce{N_2_{\gaz}})_0
		      }
		      \stm{=}
		      \frac{%
			      p_{\ce{NH3},0}{}^2 \cdot p^\circ{}^2
		      }{%
			      p_{\ce{N2},0} \cdot p_{\ce{H2},0}{}^3
		      }%
	      \]
	      Si $Q_{r,0} < K^\circ$, la réaction se produit dans le sens
	      \textbf{direct}~; et indirect sinon. \pt{1}
	      }%
	      \tikz[remember picture, overlay]
	      \node[above right=-8pt and 35pt of pic cs:MA] {\pt{1}}
	      ;
	      \tikz[remember picture, overlay]
	      \node[above right=-8pt and 24pt of pic cs:MB] {\pt{1}}
	      ;
	\item[n]{4}%
	      \leftcenters{%
		      \hspace{-12pt}
		      Soit
	      }{%
		      $
			      \ce{\alpha_1R_1 + \alpha_2R_2} +…
			      =
			      \ce{\beta_1P_1 +\beta_2P_2}+…
			      \Lra
			      0 = \sum_{i}\nu_{\ce{X_i}} \ce{X_i}
		      $
	      }%
	      \smallbreak
	      \vspace{-15pt}
	      Démontrer le lien entre la vitesse de réaction et la concentration d'un
	      constituant $[\ce{X_i}]$.
	      \smallbreak
	      \vspace{-15pt}
	      \psw{%
		      \begin{DispWithArrows*}[]
			      n_{\ce{X_i}}(t) &\stm{=} n_{\ce{X_i},0} + \nu_{\ce{X_i}}\xi(t)
			      \Arrow{On isole}
			      \\\Lra
			      \xi(t) &\stm{=} \frac{n_{\ce{X_i}}(t) -
				      n_{\ce{X_i},0}}{\nu_{\ce{X_i}}}
			      \Arrow{$\mdiv V$}
			      \\\Lra
			      x(t) &= \frac{1}{\nu_{\ce{X_i}}} \frac{n_{\ce{X_i}}(t) -
				      n_{\ce{X_i},0}}{V}
			      \Arrow{$\DS\dv{\cdot}{t}$, $\frac{n_{\ce{X_i}}(t)}{V} =
					      [\ce{X_i}](t)$ et $v(t) = \dv{x}{t}$ \pt{1}}
			      \\\Lra
			      \Aboxed{v(t) &\stm{=} \frac{1}{\nu_{\ce{X_i}}} \dv{[\ce{X_i}]}{t}}
		      \end{DispWithArrows*}
	      }%
	\item[n]{5}%
	      Décrire en une phrase ce qu'est la dégénérescence de l'ordre. Démontrer
	      l'expression de la loi de vitesse sur l'exemple $\ce{aA + bB -> cC +
			      dD}$ dans ce cas, et donner l'expression de $k\ind{app}$ dans ce
	      cas. De même avec les proportions stœchiométriques.
	      \smallbreak
	      \psw{%
	      La dégénérescence de l'ordre consiste à mettre tous les réactifs en
	      excès sauf un \pt{1}. Par exemple, si \ce{A} est en excès, alors
	      $[\ce{A}](t) \approx
		      [\ce{A}]_0$~; ainsi
	      \[
		      \boxed{
		      v \stm(un){=}
		      k[\ce{A}]^{p}[\ce{B}]^{q} =
		      \underbracket[1pt]{k[\ce{A}]_0{}^{p}}_{= \cte}[\ce{B}]^{q} \stm(un){=}
		      k\ind{app}[\ce{B}]^{q}
		      }
	      \]
	      et on peut trouver l'ordre partiel en \ce{B}. Si les réactifs sont en
	      proportions stœchiométriques, on aura
	      \[
		      [{\ce{A}}]_0 = ac_0
		      \qet
		      [{\ce{B}}]_0 = bc_0
		      \quad \Ra \quad
		      \boxed{
			      [{\ce{A}}] = a (c_0 - x)
			      \qet
			      [{\ce{B}}] = b (c_0 - x)
		      }
		      \hspace{12pt}
		      \pt{1}
	      \]
	      On peut donc factoriser la loi de vitesse~:
	      \begin{gather*}
		      v =
		      k \left( a (c_0 - x)\right)^{p}\left( b (c_0 - x)\right)^{q}
		      \Lra
		      v =
		      ka^{p}b^{q} \left(c_0 - x\right)^{p + q}
		      \Lra
		      \boxed{v = k_{\rm app} (c_0 -x)^{m}}
		      \pt{1}
	      \end{gather*}
	      avec $m = p + q$ l'ordre global, et $k\ind{app} = ka^{p}b^{q}$ la
	      constante apparente. On a donc accès à l'ordre global.
	      }%
\end{enumerate}

\ifstudent{
	\begin{tikzpicture}[remember picture, overlay]
		\node[anchor=north west, align=left]
		at ([shift={(1.4cm,0)}]current page.north west)
		{\\[5pt]\Large\bfseries Nom~:\\[10pt]\Large\bfseries Prénom~:};
		\node[anchor=north east, align=right]
		at ([shift={(-1.5cm,-17pt)}]current page.north east)
		{\Large\bfseries Note~:\hspace{1cm}/20};
	\end{tikzpicture}
}

\end{document}
