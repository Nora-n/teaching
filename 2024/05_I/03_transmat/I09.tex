\documentclass[a4paper, 10pt, final, garamond]{book}
\usepackage{cours-preambule}
\graphicspath{{./figures/}}
\addto\captionsfrench{\renewcommand{\figurename}{Fig.}}

\makeatletter
\renewcommand{\@chapapp}{Contrôle de connaissances}
\makeatother

% \toggletrue{student}
% \toggletrue{corrige}
\renewcommand{\mycol}{black}
% \renewcommand{\mycol}{gray}

\begin{document}
\setcounter{chapter}{8}

\settype{enon}
\settype{solu}

\chapter{Cinétique chimique et circuits en RSF\ifstudent{~(15')}}

\begin{enumerate}[label=\sqenumi, leftmargin=10pt]
	\item[n]{7}%
	      Décrire en une phrase ce qu'est la dégénérescence de l'ordre. Démontrer
	      l'expression de la loi de vitesse sur l'exemple $\ce{aA + bB -> cC +
			      dD}$ dans ce cas, et donner l'expression de $k\ind{app}$ dans ce
	      cas. De même avec les proportions stœchiométriques.
	      \smallbreak
	      \psw{%
	      La dégénérescence de l'ordre consiste à mettre tous les réactifs en
	      excès sauf un \pt{1}. Par exemple, si \ce{A} est en excès, alors
	      $[\ce{A}](t) \stm(un){\approx}\/ [\ce{A}]_ $~; ainsi
	      \[
		      \boxed{
		      v \stm(un){=}
		      k[\ce{A}]^{p}[\ce{B}]^{q} =
		      \underbracket[1pt]{k[\ce{A}]_0{}^{p}}_{= \cte}[\ce{B}]^{q} \stm(un){=}
		      k\ind{app}[\ce{B}]^{q}
		      }
	      \]
	      et on peut trouver l'ordre partiel en \ce{B}. Si les réactifs sont en
	      proportions stœchiométriques, on aura
	      \[
		      [{\ce{A}}]_0 = ac_0
		      \qet
		      [{\ce{B}}]_0 = bc_0
		      \quad \Ra \quad
		      \boxed{
			      [{\ce{A}}] = a (c_0 - x)
			      \qet
			      [{\ce{B}}] = b (c_0 - x)
		      }
		      \hspace{12pt}
		      \pt{1}
	      \]
	      On peut donc factoriser la loi de vitesse~:
	      \begin{gather*}
		      v =
		      k \left( a (c_0 - x)\right)^{p}\left( b (c_0 - x)\right)^{q}
		      \Lra
		      v =
		      ka^{p}b^{q} \left(c_0 - x\right)^{p + q}
		      \Lra
		      \boxed{v = k_{\rm app} (c_0 -x)^{m}}
		      \pt{1}
	      \end{gather*}
	      avec $m = p + q$ l'ordre global, et $k\ind{app} = ka^{p}b^{q}$ \pt{1} la
	      constante apparente. On a donc accès à l'ordre global.
	      }%
	\item[n]{8}%
	      Démontrer l'équation différentielle d'une réaction $\ce{aA + bB -> cC +
			      dD}$ d'ordre 2 par rapport au réactif \ce{A}. La résoudre sous la forme
	      $1/[\ce{A}]$. Déterminer alors l'expression du temps de demi-réaction.
	      \smallbreak
	      \vspace{-15pt}
	      \psw{
		      \begin{enumerate}
			      \item[b]{Équation différentielle}:
			            \leavevmode\vspace*{-25pt}\relax
			            \[
				            v \stm{=} \frac{1}{-\abs{\nu_{\ce{A}}}} \dv{[\ce{A}]}{t}
				            \quad \ste{\Longleftrightarrow}{v \stm{=} k[\ce{A}]^2} \quad
				            \boxed{\dv{[\ce{A}]}{t} \stm{=} -\abs{\nu_{\ce{A}}} k [\ce{A}]^2}
			            \]
			      \item[b]{Résolution}:
			            on \textbf{sépare les variables} et on utilise la dérivée de
			            la fonction inverse~:
			            \begin{DispWithArrows*}
				            % \dv{[{\ce{A}}]}{t} &= -\abs{\nu_{\ce{A}}}k[{\ce{A}}]^2
				            % \Arrow{Séparation des variables}
				            % \\\Lra
				            - \frac{\dd{[\ce{A}]}}{[{\ce{A}}]^2} &\stm{=}
				            \abs{\nu_{\ce{A}}}k \dd{t}
				            \Arrow{%
					            $\DS\int$ et
					            $\DS\left( \frac{1}{f} \right)' = - \frac{f'}{f^2}$
				            }
				            % \\\Lra
				            % \int_{[\ce{A}]_0}^{[\ce{A}]}
				            % -\frac{\dd{[\ce{A}]}}{[{\ce{A}}]^2}
				            % &=
				            % \int_{t=0}^{t} \abs{\nu_{\ce{A}}}k \dd{t}
				            % \Arrow{$\DS\left( \frac{1}{f} \right)' = -
				            % \frac{f'}{f^2}$}
				            \\\Lra
				            \int_{[\ce{A}]_0}^{[\ce{A}](t)}
				            \dd{\left( \frac{1}{[\ce{A}](t)} \right)}
				            &\stm{=}
				            \abs{\nu_{\ce{A}}}k \cdot \int_{t=0}^{t} \dd{t}
				            \Arrow{$\int_a^b \dd{(\cdot )} = [(\cdot )]_a^b$}
				            \\\Lra
				            \frac{1}{[{\ce{A}}]} - \frac{1}{[\ce{A}]_0} &=
				            \abs{\nu_{\ce{A}}}k \cdot t
				            \\\Lra
				            \Aboxed{\frac{1}{[{\ce{A}}]} &\stm{=}
					            \frac{1}{[\ce{A}]_0} + \abs{\nu_{\ce{A}}}kt}
			            \end{DispWithArrows*}
			      \item[b]{Temps de demi-réaction}: par définition, $[\ce{A}](t_{1/2})
				            =
				            \frac{[\ce{A}]_0}{2}$, soit~:
			            \begin{gather*}
				            \frac{1}{[\ce{A}](t_{1/2})} \stm{=} \frac{2}{[\ce{A}]_0} =
				            \frac{1}{[\ce{A}]_0} + \abs{\nu_{\ce{A}}}k \cdot t_{1/2}
				            \Lra
				            \boxed{%
					            t_{1/2} \stm{=}
					            \frac{1}{\abs{\nu_{\ce{A}}}k \cdot [\ce{A}]_0}
				            }
			            \end{gather*}
		      \end{enumerate}
	      }
	\item[n]{5}%
	      Sous quelle forme mathématique s'exprime le signal d'un système en RSF~?
	      Présenter alors le passage en complexes et l'intérêt de cette forme pour
	      la dérivation et l'intégration.
	      \smallbreak
	      \begin{isd}
		      \vspace{-15pt}
		      \psw{%
			      \begin{DispWithArrows*}[groups]
				      y(t) &\stm{=} Y_0 \cos(\wt+\f)
				      \Arrow{Passage $\Cb$}
				      \\\Lra
				      \yu(t) &\stm{=} Y_0 \exr^{\jj (\wt+\f)}
				      \Arrow{Séparation de $t$}
				      \\\Lra
				      \yu(t) &= \underbracket[1pt]{Y_0\exr^{\jj
						      \f}}_{=\cte}\cdot\exr^{\jwt}
				      \Arrow{Réécriture}
				      \\\Lra
				      \Aboxed{\yu(t) &\stm{=} \Yu\exr^{\jwt}}
				      \\\text{avec} \quad
				      \Aboxed{\Yu &= Y_0\exr^{\jj \f}}
				      \Ra
				      \left\{
				      \begin{array}{ll}
					      Y_0 & = \abs{\Yu}
					      \\
					      \f  & = \arg*{\Yu}
				      \end{array}
				      \right.
			      \end{DispWithArrows*}
		      }%
		      \tcblower
		      \psw{%
			      \begin{gather*}
				      \dv{\yu}{t} = \dv{\Yu\exr^{\jwt}}{t} = \jw \cdot \Yu\exr^{\jwt}
				      \Lra
				      \boxed{\dv{\yu}{t} \stm{=} \jw \yu(t)}
				      \\
				      \int \yu(t) \dt = \int \Yu\exr^{\jwt} \dt = \frac{\Yu\exr^{\jwt}}{\jw}
				      \Lra
				      \boxed{\int \yu(t) \dt \stm{=} \frac{\yu(t)}{\jw}}
			      \end{gather*}
		      }%
	      \end{isd}
\end{enumerate}
\vspace{-25pt}

\ifstudent{
	\begin{tikzpicture}[remember picture, overlay]
		\node[anchor=north west, align=left]
		at ([shift={(1.4cm,0)}]current page.north west)
		{\\[5pt]\Large\bfseries Nom~:\\[10pt]\Large\bfseries Prénom~:};
		\node[anchor=north east, align=right]
		at ([shift={(-1.5cm,-17pt)}]current page.north east)
		{\Large\bfseries Note~:\hspace{1cm}/20};
	\end{tikzpicture}
}

\end{document}
