\documentclass[a4paper, 10pt, final, garamond]{book}
\usepackage{cours-preambule}
\graphicspath{{./figures/}}

\makeatletter
\renewcommand{\@chapapp}{Contr\^ole de connaissances}
\makeatother

% \toggletrue{student}
% \HideSolutionstrue
\toggletrue{corrige}
\renewcommand{\mycol}{black}

\begin{document}
\setcounter{chapter}{12}

\chapter{Ondes progressives et interférences\ifstudent{ (12')}}

\begin{enumerate}[label=\sqenumi]
	\nitem{5}%
	Soit $g (t) = A \cos(\wt+\f)$ la perturbation en $x=0$ d'un milieu 1D. L'onde
	progressive allant vers la droite, démontrer l'expression du signal $s (x,t)$
	en fonction de $\w$, $t$, $k$, $x$ et $\f$. Comment s'appelle $k$~? L'exprimer
	en fonction de $\lambda$.
	\smallbreak
	\psw{
		L'onde en $x$ est la même qu'en $x=0$, mais avec un retard dû à son
		déplacement. Ainsi,
		\[
			s(x,t) \stm{=} g \left(t-\frac{x}{c}\right)
			= A \cos(\w \left( t-\frac{x}{c} \right) + \f)
			\stm{=} A \cos(\wt - \frac{\w}{c}x + \f)
		\]
		Avec $k \stm{=} \frac{\w}{c} = \frac{2\pi}{\lambda}$ le vecteur \stk{}\
		d'onde, on obtient
		\[
			\boxed{s (x,t) \stm[-1]{=} A \cos(\wt - kx +\f)}
		\]
	}
	\nitem{7}%
	Qu'est-ce que l'approximation par une onde plane~? Répondre en français.
	Démontrer alors le lien entre déphasage et différence de marche en un point M
	recevant le signal somme de deux sources sphériques S$_1$ et S$_2$ de même
	fréquence dans le cadre de cette approximation. Détaillez les expressions de
	$\Delta{L}$ et $\Delta\f$.
	\smallbreak
	\psw{
		Toute vibration en un point M de l'espace peut s'approximer par une onde
		plane si la source est suffisamment \pt{1} éloignée. Pour deux sources
		S$_1$ et S$_2$ et un point M loin d'elles, on aura~:
		\begin{gather*}
			s_1(\Mr,t) \stm{=} A_1\cos(\wt - k\SaMr + \f_{01})
			\qet
			s_2(\Mr,t) = A_2\cos(\wt - k\SbMr + \f_{02})
			\\\Ra
			\f_1(\Mr) \stm{=} -k\SaMr + \f_{01}
			\qet
			\f_2(\Mr) = -k\SaMr + \f_{02}
			\\\Ra
			\D\f_{1/2}(\Mr) =
			\f_1(\Mr) - \f_2(\Mr) \stm{=}
			-k(\SaMr - \SbMr) + \f_{01} - \f_{02}
			\\\Lra
			\boxed{\D\f_{1/2}(\Mr) \stm{=} -k\D L_{1/2}(\Mr) + \D\f_0}
			\qav k=\frac{2\pi}{\lambda}
		\end{gather*}
		avec $\Delta{L_{1/2}(\Mr) \stm{=} \SaMr - \SbMr}$ la différence de marche et
		$\Delta\f_0 \stm{=} \f_{01} - \f_{02}$ la différence de phase à l'origine.
	}
	\nitem{6}%
	Quelles sont les conditions pour avoir interférence entre deux ondes~? Pour
	quelles valeurs de $\Delta\f_{1/2} (M)$ une superposition de signaux donne des
	interférences constructives~? destructives~? Répondre en utilisant
	l'\textbf{ordre d'interférence}. Pour $\Delta{\f_0}=0$, à quelles valeurs de
	$\Delta{L_{1/2}}$ cela correspond~?
	\smallbreak
	\psw{
		Il faut des ondes de même fréq\stk{u}ence et de même nat\stk{u}re. On
		obtient
		\smallbreak
		\begin{isd}[sidebyside align=top]
			\tcbsubtitle{\fatbox{\textbf{Interférences constructives}}}
			\psw{
				\begin{gather*}
					\Delta\f_{1/2}(\Mr) \stm{=} 2p\pi
					\Lra
					\boxed{\Delta{L}_{2/1}(\Mr) \stm{=} p\lambda}
				\end{gather*}
			}
			\tcblower
			\tcbsubtitle{\fatbox{\textbf{Interférences destructives}}}
			\psw{
				\begin{gather*}
					\Delta\f_{1/2}(\Mr) \stm{=} (2p+1)\pi
					\Lra
					\boxed{\Delta{L}_{2/1}(\Mr) \stm{=} \left(p+\frac{1}{2}\right)\lambda}
				\end{gather*}
			}
		\end{isd}
	}
	\nitem{2}%
	Pourquoi fait-on des interférences \textbf{lumineuses} avec une unique
	source~? Comment s'exprime l'intensité d'un signal $s (\Mr,t)$~?
	\smallbreak
	\psw{
		Les sources lumineuses changent de  phase à l'origine très fréquemment
		($\tau_c \approx \SI{3e-15}{s}$ pour le Soleil). Or, pour interférer deux
		ondes doivent être cohé\stk[-1](un){r}entes, c'est-à-dire avoir $\Delta\f_0$
		constant.
		\smallbreak
		L'intensité d'un signal est proportionnel à la moyenne du carré de son
		amplitude~:
		\[
			\boxed{I(\Mr) \stm[-1]{=} K \moy{s^{2}(\Mr,t)}}
		\]
	}

	\ifstudent{
		\begin{tikzpicture}[remember picture, overlay]
			\node[anchor=north west, align=left]
			at ([shift={(1.4cm,0)}]current page.north west)
			{\\[5pt]\Large\bfseries Nom~:\\[10pt]\Large\bfseries Prénom~:};
			\node[anchor=north east, align=right]
			at ([shift={(-1.5cm,-17pt)}]current page.north east)
			{\Large\bfseries Note~:\hspace{1cm}/20};
		\end{tikzpicture}
	}
\end{enumerate}
\end{document}
