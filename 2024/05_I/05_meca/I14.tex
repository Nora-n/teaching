\documentclass[a4paper, 10pt, final, garamond]{book}
\usepackage{cours-preambule}
\graphicspath{{./figures/}}
\addto\captionsfrench{\renewcommand{\figurename}{Fig.}}

\makeatletter
\renewcommand{\@chapapp}{Contr\^ole de connaissances}
\makeatother

% \toggletrue{student}
% \toggletrue{corrige}
\renewcommand{\mycol}{black}
% \renewcommand{\mycol}{gray}

\hfuzz=5.002pt

\begin{document}
\setcounter{chapter}{13}

\settype{enon}
\settype{solu}

\chapter{Cinématique et dynamique du point\ifstudent{~(15')}}

\begin{enumerate}[label=\sqenumi]
	\item[n]{2}%
	      Donner les valeurs de $\Delta{\f_{1/2}(\Mr)}$ et de
	      $\Delta{L_{2/1}(\Mr)}$ donnant des interférences constructives et destructives
	      pour $\Delta{\f_0}=0$.
	      \smallbreak
	      \vspace{-30pt}
	      \psw{
		      \begin{gather*}
			      \Delta\f_{1/2}(\Mr) = 2p\pi
			      \stm{\Lra}
			      \boxed{\Delta{L}_{2/1}(\Mr) = p\lambda}
			      \qqet
			      \Delta\f_{1/2}(\Mr) = (2p+1)\pi
			      \stm{\Lra}
			      \boxed{\Delta{L}_{2/1}(\Mr) = \left(p+\frac{1}{2}\right)\lambda}
		      \end{gather*}
	      }
	      \vspace{-15pt}
	\item[n]{2}%
	      Soient deux points A et B de masses respectives $m_A$ et $m_B$. Exprimer et
	      représenter la force d'attraction gravitationnelle de A sur B.
	      \smallbreak
	      \noindent
	      \begin{minipage}{.6\linewidth}
		      \psw{
			      \vspace{-15pt}
			      \[
				      \Ff_{g,\rm A\ra B} \stm{=} -\Gc \frac{m_Am_B}{\rm AB^2}\ur
				      \qavec
				      \ur = \frac{\vvr{AB}}{\rm AB}
			      \]
			      \vspace{-15pt}
		      }
	      \end{minipage}
	      \hfill
	      \noindent
	      \begin{minipage}{.39\linewidth}
		      \begin{center}
			      \sswitch{
				      \includegraphics[width=.80\linewidth]{fg_stud}
			      }{
				      \includegraphics[width=.80\linewidth]{fg_prof}
			      }
			      \captionof{figure}{Interaction gravitationnelle\protect\pt{1}.}
		      \end{center}
	      \end{minipage}
	      \vspace{-15pt}
	\item[n]{3}%
	      Énoncer les trois lois de \textsc{Newton}. On travaille avec un système
	      ouvert.
	      \smallbreak
	      \vspace{-15pt}
	      \psw{
		      \begin{enumerate}
			      \item[l][20]{\protect\pt{1}}
			            $\exists \Rc$ galiléens $: (\forall \Mr ~|~ \sum \Ff_{\ext\to\Mr} =
				            \of)$, $\Mr$ est soit au repos, soit en translation rectiligne
			            uniforme~;
			      \item[l][20]{\protect\pt{1}} $\dv{\pf\Rg(\Mr)}{t} = \sum \Ff_{\ext\to\Mr}$~;
			      \item[l][20]{\protect\pt{1}} $\forall (\Mr_1,\Mr_2), \Ff_{1\to2} = -\Ff_{2\to 1}$.
		      \end{enumerate}
	      }
	\item[n]{4}%
	      Donner les \textbf{deux expressions} donnant la position du centre
	      d'inertie d'un ensemble de points. Démontrer le lien entre la quantité de
	      mouvement d'un ensemble de points et la vitesse du centre d'inertie. Pourquoi
	      applique-t-on le PFD avec uniquement les forces extérieures au système~?
	      Répondre en français.
	      \smallbreak
	      \psw{
		      \[
			      \boxed{\vv{\rm OG} = \sum_i \frac{m_i}{m_{\tot}} \vv{{\OMr}_i}}
			      \stm{\Lra}
			      \boxed{\sum_i m_i \vv{\rm GM_i} = \of }
		      \]
		      \begin{gather*}
			      \beforetext{Or,}
			      \pf\Rg(\Sc) \stm{=} \sum_i \pf\Rg(\Mr_i)
			      \qet
			      \vf\Rg(\Gr) = \dv{\vv{\rm OG}}{t} = \frac{1}{m_{\tot}}
			      \sum_i m_i \dv{{\OM}_i}{t}
			      \Lra
			      \boxed{\pf\Rg(\Sc) \stm[-1]{=} m_{\tot}\vf\Rg(\Gr)}
		      \end{gather*}
		      Les forces intérieures se compensent d'après la troisième loi de
		      \textsc{Newton} \pt{1}.
	      }
	\item[n]{9}%
	      Soit une balle lancée avec une vitesse $\vfo$ faisant un angle $\alpha$ avec
	      l'horizontale. On néglige toute autre force que le poids. Faire un schéma puis
	      déterminer les équations horaires des composantes sur $\ux$ et $\uy$ du
	      mouvement, et déterminer l'équation de la trajectoire. Portez une attention
	      particulière à l'établissement du système.
	      \smallbreak
	      \psw{
		      \noindent
		      \begin{minipage}[c]{.60\linewidth}
			      \begin{enumerate}[label=\sqenumi]
				      \item[b]{\ltm[20]{\protect\pt{1}}Système}~:
				            \{balle\} dans $\Rc\ind{labo}$ supposé galiléen
				      \item[b]{\ltm[20]{\protect\pt{1}}Schéma}~:
				            cf.\ figure.
				      \item[b]{\ltm[20]{\protect\pt{1}}Modélisation}~:
				            repère $(\!\ux,\uy,\uz)$ (cf.\ schéma),
				            \smallbreak
				            repérage $\OM = x\ux +y\uy$, $\vf = \xp\ux+\yp\uy$, $\af =
					            \xpp\ux+\ypp\uy$.
				      \item[b]{\ltm[20]{\protect\pt{1}}Conditions initiales}~:
				            $\OM (0) = \of$ et
				            \smallbreak
				            $\vf(0) = v_0\cos(\alpha)\ux+ v_0\sin(\alpha)\uy$
				      \item[b]{\ltm[20]{\protect\pt{1}}BdF}~: $\Pf = -mg\uy$
			      \end{enumerate}
		      \end{minipage}
		      \begin{minipage}[c]{0.39\linewidth}
			      \begin{center}
				      \sswitch{
					      \includegraphics[width=\linewidth]{cl_ssf_stud}
				      }{
					      \includegraphics[width=\linewidth]{cl_ssf_prof}
				      }
				      \vspace{-15pt}
				      \captionof{figure}{Chute libre.}
			      \end{center}
		      \end{minipage}
		      \begin{enumerate}[label=\sqenumi, start=6]
			      \item[b]{PFD}~:
			            \vspace{-15pt}
			            \[m\af \stm{=} \Pf
				            \Lra
				            \left\{
				            \begin{array}{l}
					            \ddot{x}(t) \stm{=} 0 \\
					            \ddot{y}(t) = -g
				            \end{array}
				            \right.
				            \Lra
				            \left\{
				            \begin{array}{l}
					            \dot{x}(t) = v_0\cos\alpha \\
					            \dot{y}(t) = -gt + v_0 \alpha
				            \end{array}
				            \right.
				            \Lra
				            \boxed{
					            \left\{
					            \begin{array}{l}
						            x(t) = v_0t\cos\alpha \\
						            y(t) = \DS -\frac{1}{2}gt^2 + v_0t\sin\alpha
					            \end{array}
					            \right.}
				            \pt{1}
			            \]
			            \begin{gather*}
				            \beforetext{Ainsi,}
				            t            = \frac{x}{v_0\cos\alpha}
				            \Ra
				            y(x)         = - \frac{1}{2}g \frac{x^2}{v_0{}^2\cos^2\alpha} +
				            v_0\sin\alpha \frac{x}{v_0\cos\alpha}
				            \\\Lra
				            \boxed{y(x) \stm[-1]{=} - \frac{g}{2v_0{}^2\cos^2\alpha}x^2 + x\tan\alpha}
			            \end{gather*}
		      \end{enumerate}
	      }
	      \ifstudent{
		      \begin{tikzpicture}[remember picture, overlay]
			      \node[anchor=north west, align=left]
			      at ([shift={(1.4cm,0)}]current page.north west)
			      {\\[5pt]\Large\bfseries \textsc{Nom}~:\\[10pt]\Large\bfseries Prénom~:};
			      \node[anchor=north east, align=right]
			      at ([shift={(-1.5cm,-17pt)}]current page.north east)
			      {\Large\bfseries Note~:\hspace{1cm}/20};
		      \end{tikzpicture}
	      }
\end{enumerate}
\end{document}
