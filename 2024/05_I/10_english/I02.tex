\documentclass[a4paper, 11pt, final, garamond]{book}
\usepackage{cours-preambule}
% \usepackage[english]{babel}
\graphicspath{{./figures/}}
\addto\captionsfrench{\renewcommand{\figurename}{Fig.}}
% \addto\captionsfrench{\renewcommand{\chaptername}{Lecture}}

\makeatletter
\renewcommand{\@chapapp}{Physics video of the week, number}
\makeatother

% \toggletrue{student}
% \toggletrue{corrige}
\renewcommand{\mycol}{black}
% \renewcommand{\mycol}{gray}

\begin{document}
\setcounter{chapter}{1}

\settype{enon}
\settype{solu}

\chapter{You don't know how mirrors work \url{https://youtu.be/rYLzxcU6ROM}}

\section{True or false\!?}

\textbf{Circle the correct answer and justify if it's false}
\begin{enumerate}[label=\sqenumi, leftmargin=10pt]
  \item[l]{T/\cswitch{\circled{F}}{F}} %
	Mirrors flip images left-to-right.
	\smallbreak
	\sswitch{%
		\dotfill
	}{%
		False, they only reflect light.
	}%
	\smallbreak
	\dotfill
	\item[l]{T/\cswitch{\circled{F}}{F}} %
    In the painting shown in the video, the girl is looking at herself in a
    mirror.
	\smallbreak
	\sswitch{%
		\dotfill
	}{%
		False, she's looking at us because the mirror is at an angle.
	}%
	\smallbreak
	\dotfill
	\item[l]{\cswitch{\circled{T}}{T}/F} %
    The law of reflection can be explained by modeling light as a wave.
	\smallbreak
	\sswitch{%
		\dotfill
	}{%
		True.
	}%
	\smallbreak
	\dotfill
	\item[l]{T/\cswitch{\circled{F}}{F}} %
    Each silver atom hit by a lightwave radiates its own light in a specific
    direction.
	\smallbreak
	\sswitch{%
		\dotfill
	}{%
		False, they emit light in all directions. It's called the
    \textsc{Huygens-Fresnel} principle.
	}%
	\smallbreak
	\dotfill
	\item[l]{\cswitch{\circled{T}}{T}/F} %
    The law of reflection can be explained by modeling light as a stream of
    photons.
	\smallbreak
	\sswitch{%
		\dotfill
	}{%
		True.
	}%
	\smallbreak
	\dotfill
	\item[l]{T/\cswitch{\circled{F}}{F}} %
    Each individual photon follows the law of reflection to arrive to the
    detector.
	\smallbreak
	\sswitch{%
		\dotfill
	}{%
		False, a photon can take any path to the detector.
	}%
	\smallbreak
	\dotfill
	\item[l]{T/\cswitch{\circled{F}}{F}} %
    Diffraction can be explained using light rays.
	\smallbreak
	\sswitch{%
		\dotfill
	}{%
		False, it's described by quantum mechanics using photons and probabilities.
	}%
	\smallbreak
	\dotfill
\end{enumerate}

\section{Translate into English}
\begin{itemize}
  % \item Loi de la réflexion~: \psw{law of reflection}
	\item Champ~: \psw{field}
  \item Perturbation~: \psw{disturbance}
  \item Onde de probabilité~: \psw{probability wave}
  \item Chemin~: \psw{path}
  \item Bord~: \psw{edge}
\end{itemize}

\ifstudent{
	\begin{tikzpicture}[remember picture, overlay]
		\node[anchor=north west, align=left]
		at ([shift={(1.4cm,0)}]current page.north west)
		{\\[5pt]\Large\bfseries SURNAME\!:\\[10pt]\Large\bfseries Name\!:};
		\node[anchor=north east, align=right]
		at ([shift={(-1.5cm,-17pt)}]current page.north east)
		{\Large\bfseries Mark\!:\hspace{1cm}/12};
	\end{tikzpicture}
}
\end{document}
