\documentclass[a4paper, 11pt, final, garamond]{book}
\usepackage{cours-preambule}
% \usepackage[english]{babel}
\graphicspath{{./figures/}}
\addto\captionsfrench{\renewcommand{\figurename}{Fig.}}
% \addto\captionsfrench{\renewcommand{\chaptername}{Lecture}}

\makeatletter
\renewcommand{\@chapapp}{Physics video of the week, number}
\makeatother

% \toggletrue{student}
% \toggletrue{corrige}
\renewcommand{\mycol}{black}
% \renewcommand{\mycol}{gray}

\hfuzz=5.003pt

\begin{document}
\setcounter{chapter}{2}

\settype{enon}
\settype{solu}

\chapter{The Thermoelectric Effect \url{https://youtu.be/O6waiEeXDGo}}

\section{True or false\!?}

\textbf{Circle the correct answer and justify \xul{if it's false}}
\begin{enumerate}[label=\sqenumi, leftmargin=10pt]
	\item[l]{\crls{T}/F} %
	      The speaker believed that connecting a wire to a battery would trap the
	      current in a loop.
	      \smallbreak
	      \sswitch{%
		      \dotfill
	      }{%
		      True -- As a child, the speaker hoped to catch the current in a loop by
		      quickly connecting the wire ends.
	      }%
	      \smallbreak
	      \dotfill
	\item[l]{\crls{T}/F} %
	      The thermoelectric effect can create a voltage from a temperature difference.
	      \smallbreak
	      \sswitch{%
		      \dotfill
	      }{%
		      True -- A temperature difference can create a voltage, which is part of the
		      thermoelectric effect.
	      }%
	      \smallbreak
	      \dotfill
	\item[l]{\crls{T}/F} %
	      The speaker explains that heating one end of a wire causes electrons to move
	      apart.
	      \smallbreak
	      \sswitch{%
		      \dotfill
	      }{%
		      True -- Heating one end causes electrons to move apart due to increased
		      thermal energy.
	      }%
	      \smallbreak
	      \dotfill
	\item[l]{T/\crls{F}} %
	      The \textsc{Seebeck} effect is pronounced in all metals equally.
	      \smallbreak
	      \sswitch{%
		      \dotfill
	      }{%
		      False -- The \textsc{Seebeck} effect is more pronounced in some metals and
		      less in others.
	      }%
	      \smallbreak
	      \dotfill
	\item[l]{T/\crls{F}} %
	      A thermocouple is created by joining two similar metals together.
	      \smallbreak
	      \sswitch{%
		      \dotfill
	      }{%
		      False -- A thermocouple is created by joining two dissimilar metals together.
	      }%
	      \smallbreak
	      \dotfill
	\item[l]{T/\crls{F}} %
	      The speaker uses a voltmeter to show that a small current is flowing in the
	      wire loop.
	      \smallbreak
	      \sswitch{%
		      \dotfill
	      }{%
		      False -- The speaker uses a multimeter in ammeter mode to show a small current
		      flowing in the loop.
	      }%
	      \smallbreak
	      \dotfill
	\item[l]{\crls{T}/F} %
	      The \textsc{Peltier} effect is the opposite of the \textsc{Seebeck} effect.
	      \smallbreak
	      \sswitch{%
		      \dotfill
	      }{%
		      True -- It states that applying a current between dissimilar metals can
		      create a change in temperature.
	      }%
	      \smallbreak
	      \dotfill
\end{enumerate}

\section{Translate into English}
\begin{itemize}
	% \item Loi de la réflexion~: \psw{law of reflection}
	\item Fil électrique~: \psw{wire}
	\item Alliage~: \psw{allow}
	\item Chauffer~: \psw{to heat (up)}
	\item Refroidir~: \psw{to cool (down)}
	\item Dans le sens des aiguilles d'une montre~: \psw{clockwise}
	\item Un chauffe-eau~: \psw{a boiler}
	\item Une coupure de courant~: \psw{a power cut}
\end{itemize}

\ifstudent{
	\begin{tikzpicture}[remember picture, overlay]
		\node[anchor=north west, align=left]
		at ([shift={(1.4cm,0)}]current page.north west)
		{\\[5pt]\Large\bfseries \textsc{Surname}\!:\\[10pt]\Large\bfseries Name\!:};
		\node[anchor=north east, align=right]
		at ([shift={(-1.5cm,-17pt)}]current page.north east)
		{\Large\bfseries Mark\!:\hspace{1cm}/14};
	\end{tikzpicture}
}
\end{document}
