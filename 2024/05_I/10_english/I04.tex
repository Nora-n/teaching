\documentclass[a4paper, 11pt, final, garamond]{book}
\usepackage{cours-preambule}
% \usepackage[english]{babel}
\graphicspath{{./figures/}}
\addto\captionsfrench{\renewcommand{\figurename}{Fig.}}
% \addto\captionsfrench{\renewcommand{\chaptername}{Lecture}}

\makeatletter
\renewcommand{\@chapapp}{Physics video of the week, number}
\makeatother

% \toggletrue{student}
% \toggletrue{corrige}
\renewcommand{\mycol}{black}
% \renewcommand{\mycol}{gray}

\hfuzz=5.003pt

\begin{document}
\setcounter{chapter}{3}

\settype{enon}
\settype{solu}

\chapter{Making fuming nitric acid \url{https://youtu.be/QmCdrDLyNXQ}}

\section{True or false\!? \small Circle the correct answer and justify \xul{if it's false}}
\begin{enumerate}[label=\sqenumi, leftmargin=10pt]
	% \item[l]{T/\crls{F}} %
	%       Normal nitric acid is composed of 68\% of water.
	%       \smallbreak
	%       \sswitch{%
	%         \dotfill
	%       }{%
	%         False -- It's composed of 68\% of actual nitric acid while the rest of
	%         it is water.
	%       }%
	%       \smallbreak
	%       \dotfill
	\item[l]{\crls{T}/F} %
	      Fuming nitric acid can light common lab gloves on fire.
	      \smallbreak
	      \sswitch{%
		      \dotfill
	      }{%
		      True.
	      }%
	      \smallbreak
	      \dotfill
	\item[l]{T/\crls{F}} %
	      To make fuming nitric acid, the speaker uses hydrochloric acid.
	      \smallbreak
	      \sswitch{%
		      \dotfill
	      }{%
		      False -- The speaker uses \textit{sulfuric} acid.
	      }%
	      \smallbreak
	      \dotfill
	\item[l]{T/\crls{F}} %
	      To speed up the reaction, the speaker cools down the reactants.
	      \smallbreak
	      \sswitch{%
		      \dotfill
	      }{%
		      False -- The reaction is sped up by \textit{heating} the reactants,
		      according to the \textsc{Arrhenius} equation.
	      }%
	      \smallbreak
	      \dotfill
	\item[l]{\crls{T}/F} %
	      The orange color comes from nitrogen dioxide gas.
	      \smallbreak
	      \sswitch{%
		      \dotfill
	      }{%
		      True -- Excess heat decomposes the nitric acid to form
		      $\ce{NO2_{\gaz}}$.
	      }%
	      \smallbreak
	      \dotfill
	\item[l]{T/\crls{F}} %
	      The process used to cool down the gas to a liquid form is called
	      crystallization.
	      \smallbreak
	      \sswitch{%
		      \dotfill
	      }{%
		      False -- It's a \textit{distillation}, a common process in chemical
		      reactions that form gases and a means of purification.
	      }%
	      \smallbreak
	      \dotfill
	\item[l]{\crls{T}/F} %
	      The speaker determines the concentration of his product by measuring its
	      density.
	      \smallbreak
	      \sswitch{%
		      \dotfill
	      }{%
		      True -- Most physical properties of mixtures depend on the relative
		      proportions of each compound.
	      }%
	      \smallbreak
	      \dotfill
	\item[l]{T/\crls{F}} %
	      Using gloves in a lab is something one must always do without thinking.
	      \smallbreak
	      \sswitch{%
		      \dotfill
	      }{%
		      False -- Every safety measure is well thought of beforehand.
	      }%
	      \smallbreak
	      \dotfill
\end{enumerate}

\section{Translate into English}
\begin{itemize}
	% \item Loi de la réflexion~: \psw{law of reflection}
	\item Oxydant~: \psw{oxidizer}
	\item Fusée~: \psw{rocket}
	\item Faire bouillir~: \psw{to boil}
	\item Colonne réfrigérante~: \psw{condenser column}
	\item Feuille d'aluminium~: \psw{aluminium foil}
	\item Sous vide~: \psw{under vacuum}
	\item Curieusement (expression idomatique)~: \psw{oddly enough}
\end{itemize}

\ifstudent{
	\begin{tikzpicture}[remember picture, overlay]
		\node[anchor=north west, align=left]
		at ([shift={(1.4cm,0)}]current page.north west)
		{\\[5pt]\Large\bfseries \textsc{Surname}\!:\\[10pt]\Large\bfseries Name\!:};
		\node[anchor=north east, align=right]
		at ([shift={(-1.5cm,-17pt)}]current page.north east)
		{\Large\bfseries Mark\!:\hspace{1cm}/14};
	\end{tikzpicture}
}
\end{document}
