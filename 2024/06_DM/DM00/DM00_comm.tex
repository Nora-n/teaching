\documentclass[a4paper, 11pt, final, garamond]{book}
\usepackage{cours-preambule}

\makeatletter
\renewcommand{\@chapapp}{Devoir maison -- chapitre}
\makeatother

\begin{document}
\setcounter{chapter}{-1}

\chapter{Commentaires sur le DM}

\begin{tcn}(prop)"bomb"{Malus}
	\begin{minipage}{0.50\linewidth}
		\begin{itemize}
			\item A~: mauvaise application numérique~;
			\item N~: numéro de copie manquant~;
			\item P~: manque prénom sur copies~;
			\item E~: manque d'encadrement des réponses~;
			\item M~: marge non laissée ou trop grande~;
			\item V~: confusion ou oubli de vecteurs~;
			\item D~: doublon avec autre copie~;
		\end{itemize}
	\end{minipage}
	\begin{minipage}{0.50\linewidth}
		\begin{itemize}
			\item Q~: question mal ou non indiquée~;
			\item H~: homogénéité non respectée~;
			\item C~: copie grand carreaux~;
			\item $\f$~: loi physique fondamentale brisée~;
			\item O~: \xul{orthographe}~;
			\item U~: valeur numérique sans unité~;
			\item S~: chiffres significatifs non cohérents.
		\end{itemize}
	\end{minipage}
\end{tcn}

\begin{center}
	\Huge Un devoir maison se traite entièrement~!
	\smallbreak
	«~AR~» à côté d'une question = «~À revoir~»
\end{center}

\section{Mise en page générale}
\begin{itemize}
	\item{}%
	      {\Large Laissez une marge, $\approx \SI{4}{cm}$, pas beaucoup plus,
	            pas beaucoup moins~;}
	\item \textbf{Laissez un espace commentaires et revoyez les consignes pour le
		      cadre}~;
	\item Utilisez des petits carreaux~;
	\item \textbf{Sautez une ligne} à chaque fois~;
	\item Pas de feuilles simples (encore moins des feuilles volantes~!!)~: copies
	      doubles uniquement~;
	\item Pas de sujets dans les copies, pas d'agrafes~;
	\item Numérotez les \textbf{copies} (pas les pages) \textbf{en bas à droite}~;
	\item Écrivez votre nom, prénom, date sur \textit{chaque} copie en
	      \textbf{haut à gauche}~;
	\item Mettez un titre \textbf{pertinent} sur chaque copie (DM00 de physique,
	      puis DM00 – suite par exemple)~;
	\item Pas de pochettes.
\end{itemize}

\section{Rédaction}
\begin{itemize}
	\item Écrivez de manière lisible~;
	\item Achetez et \textbf{utilisez une règle}~;
	\item \textit{N'écrivez surtout pas au crayon} (sauf les schémas)~;
	\item On n'écrit pas «~exo~» sur un rendu~;
	\item Sachez écrire les lettres grecques~: $\rho$ n'est pas $P$, pas $e$ ou
	      autre~;
	\item $k \neq K$ et $\SI{1}{tonne} = \SI{1}{T} = \SI{1e3}{kg}$~;
	\item Séparez bien les exercices en \textbf{soulignant les titres}~;
	\item Encadrez les résultats littéraux~;
	\item Soulignez vos résultats numériques~;
	\item Pas de mélange français/maths~!
	\item Si vous inventez une notation, introduisez-la~;
	\item Tout calcul est \textbf{d'abord sous forme littérale avant application
		      numérique}~;
	\item ÉCRIVEZ VOS APPLICATIONS NUMÉRIQUES AVEC UNE UNITÉ~;
	\item N'écrivez rien d'inutile, et n'écrivez pas trop (ça vous évite de faire
	      des fautes… dures à lire)~;
	\item \textbf{Utilisez des mots de liaison} et des \textbf{connecteurs
		      mathématiques}~: pas d'égalité dans le vide, pas de succession de
	      calculs sans $\Ra$ ou $\Lra$~;
	\item On n'écrit pas «~$\times$~», «~X~» ou «~$\alpha$~» pour écrire «~fois~»
	      dans les expressions littérales (et on n'écrit pas d'expressions
	      numériques)~;
	\item On ne rend \textbf{jamais} un brouillon. S'il est rendu, il ne sera
	      pas lu~;
	\item On écrit $\approx$ («~numériquement à peu près égal à~») et pas $\simeq$
	      («~asymptotiquement égal à~») pour une valeur numérique~;
	\item On n'écrit \textbf{ni} $\sqrt[5]{E}$ \textbf{ni}
	      $\sqrt[\frac{1}{5}]{E}$~;
	\item On n'écrit pas 2 questions côte à côte~! Il faut que je puisse écrire
	      les points correspondant au numéro de la question dans la marge, sans
	      ambiguïté.
\end{itemize}

\begin{tcn}(ror){Règles d'application numérique}
	\vspace*{-10pt}
	\begin{minipage}{0.45\linewidth}
		\begin{gather*}
			\boxed{n = \frac{PV}{RT}}
			\qav
			\left\{
			\begin{array}{rcl}
				p & = & \SI{1.0e5}{Pa}                \\
				V & = & \SI{1.0e-3}{m^3}              \\
				R & = & \SI{8.314}{J.mol^{-1}.K^{-1}} \\
				T & = & \SI{300}{K}
			\end{array}
			\right.\\
			\mathrm{A.N.~:}\quad
			\xul{n = \SI{5.6e-4}{mol}}
		\end{gather*}
	\end{minipage}
	\hfill
	\cancel{\bcancel{
			\begin{minipage}{0.45\linewidth}
				\begin{gather*}
					n = \frac{PV}{RT} = \frac{\num{e5}\cdot\num{1}}{8.32\cdot300}
					= 0.56
				\end{gather*}
			\end{minipage}
		}}
	\smallbreak
	Avec ces règles de mise en page doivent venir des réflexes~:
	\smallbreak
	\begin{isd}
		\tcbsubtitle{\fatbox{\textbf{Encadrer}}}
		Encadrer implique d'avoir vérifié~:
		\begin{enumerate}
			\item La cohérence mathématique~;
			\item L'homogénéité de la formule proposée.
		\end{enumerate}
		\tcblower
		\tcbsubtitle{\fatbox{\textbf{Souligner}}}
		Souligner implique d'avoir vérifié~:
		\begin{enumerate}
			\item La cohérence physique de la grandeur~;
			\item Les chiffres significatifs à utiliser.
		\end{enumerate}
	\end{isd}
\end{tcn}

\section{Sur le DM}

\begin{itemize}
	\item J'avais demandé à ce que vous recopiez \textbf{tous les exercices}~!
	\item J'avais demandé à ce que vous découpiez les réponses en Données,
	      Résultat attendu, Outils, Application~! Écoutez les consignes données en
	      classe~!!
	\item Ne confondez pas l'analyse \textbf{dimensionnelle} avec les équations
	      entre \textbf{grandeurs}~;
	\item Ne confondez pas \textbf{unités} et \textbf{dimensions}~;
	\item Soyez \textbf{critiques sur les valeurs} trouvées~: $\Ec =
		      \SI{1.7e-2}{J}$ c'est à peine une mouche. Très bien quand vous
	      indiquez que le résultat est illogique.
\end{itemize}

\end{document}
