\documentclass[a4paper, 11pt, final, garamond]{book}
\usepackage{cours-preambule}

\makeatletter
\renewcommand{\@chapapp}{Devoir maison -- num\'ero}
\makeatother

\begin{document}
\setcounter{chapter}{3}

\def\lspace{25}

\chapter{Commentaires sur le DM n\degree04}

\section{Pompage optimal}
Pensez à vérifier la cohérence physique de vos résultats numériques~: ici, un
travail négatif n'est pas normal~!

\begin{enumerate}
	\item Pensez à faire des schémas ou représentations des états donnés pour
	      lister les données plus visuellement (cf.\ correction). Bien pour la suite
	      et notamment l'application numérique.
	\item Il faut justifier dans le bon sens. Attention, il faut comprendre que la
	      seconde méthode comporte \textbf{deux étapes}, et \textbf{seule la première
		      est adiabatique}.
	\item Globalement bien. Attention à bien \textbf{donner la formule générale du
		      travail}, \textit{puis} de \textbf{développer en écrivant clairement les
		      hypothèses}. Notamment, on doit voir
	      \begin{DispWithArrows*}
		      W &= -\int_{V_1}^{V_2} P\ind{ext} \dd{V}
		      \Arrow{Mécaniquement réversible}
		      \\\Lra
		      W &= -\int_{V_1}^{V_2} P \dd{V}
		      \Arrow{Gaz parfait~: $P = nRT/V$\\Isotherme~: $T = T_0$}
		      \\\Lra
		      W &= - \int_{V_1}^{V_2} nRT_0\frac{\dd{V}}{V}
	      \end{DispWithArrows*}
	      etc. De plus, quasi-statique donne $P\ind{ext} = P$, mais \textbf{QS ne
		      donne pas isobare}~! Beaucoup de $P\ind{ext} = P_2$, alors que $P$
	      varie. Attention aux définitions fondamentales~!
	\item
	      \begin{enumerate}
		      \item \leavevmode\relax
		            {\Large \textbf{Rapide donc adiabatique~!}} Pas le temps pour
		            les transferts thermiques, qui sont plus lents à atteindre un
		            équilibre que les actions mécaniques.
		      \item Bien. Même si on vient de vous les écrire, il faut
		            \textbf{énoncer les lois de \textsc{Laplace}} pour justifier sur
		            quelle transformation vous l'utilisez. Il est très commun de vouloir
		            l'utiliser «~à tout va~» sur un cycle (par exemple pour une isobare,
		            isochore) alors qu'il n'y a que pour une seule qu'elle est valable.
		            \item[l]{\iconhart} L'énoncé n'était pas tout à fait clair dans les variables à
		            utiliser, donc c'était libre. Il faut savoir réduire le nombre de
		            variables à uniquement 2, comme $(P,V)$ ou $(P,T)$. Entraînez-vous
		            également à faire le calcul avec l'intégrale de $\dd{V}/V^{\gamma}$,
		            mais retenez l'astuce d'utiliser le premier principe.
		            \smallbreak
		            {\Large C'est une question hyper importante qu'il faut savoir faire~!}
	      \end{enumerate}

	\item Il faut respecter l'énoncé~: tracé \textbf{sur un même diagramme}. De
	      plus, il faut impérativement le \textbf{sens de parcours}, et il faut
	      absolument \textbf{savoir tracer isotherme et adiabatique}. Revoir le
	      commentaire/analyse.
\end{enumerate}


\end{document}
