\documentclass[../DS01.tex]{subfiles}%
\graphicspath{{./figures/}}%

% \subimport{/home/nora/Documents/Enseignement/Prepa/bpep/exercices/DS/lentille_app_photo/}{sujet.tex}

\begin{document}%
\section[75]"P"{Module photographique d’un smartphone \ifcorrige{~\small\textit{(D'après
		CCS MPI 2023)}}}

\enonce{%
	En 1973, Martin Cooper invente, avec son équipe, le premier téléphone
	cellulaire. Depuis, l’usage du téléphone portable n’a cessé de croître, ce qui
	en fait l’un des biens de consommation les plus répandus de la planète.

	\begin{tcn}(data)<lfnt>{}
		\textit{Appareil photographique d’un téléphone portable}
		\begin{center}
			\begin{tabular}{lcl}
				Résolution du capteur                   &  & $4000\times 3000$ pixels \\
				Nombre d’ouverture                      &  & $N=2,2$                  \\
				Diagonale du capteur                    &  & $d=1/3$ de pouce         \\
				Indice de réfraction de l’air           &  & $n\ind{air}=1,00$        \\
				Indice de réfraction de la lentille     &  & $n=1,52$                 \\
				Rayon de courbure de la lentille        &  & $R=\SI{4.0}{mm}$         \\
				Diamètre réel la lentille               &  & $\Phi=\SI{5.0}{mm}$      \\
				Distance focale effective de l’objectif &  & $f'=\SI{9.0}{mm}$
			\end{tabular}
		\end{center}

		\textit{Propriétés de l'\oe il humain}
		\begin{center}
			\begin{tabular}{lcl}
				Distance normale d’observation d’un téléphone portable &  & \SI{30}{cm}               \\
				Pouvoir de résolution d’un oeil \og{}normal\fg{}       &  & $\alpha=\SI{3.0e-4}{rad}$
			\end{tabular}
		\end{center}

		\textit{Conversion }%
		\begin{center}
			\begin{tabular}{c}
				$\SI{1}{pouce}=\SI{25.4}{mm}$ \\
			\end{tabular}
		\end{center}

		\textit{Estimation d'une incertitude-type composée}%

		Si la grandeur $y$ calculée est un produit ou un quotient du type $y=x_1x_2$
		ou $y=x_1/x_2$, alors l’incertitude-type sur $y$, notée $u(y)$ est reliée à
		l’incertitude-type sur chacun des facteurs par
		\eq{%
			\pa{\dfrac{u(y)}{y}}^2=\pa{\dfrac{u(x_1)}{x_1}}^2+\pa{\dfrac{u(x_2)}{x_2}}^2
		}%
	\end{tcn}

	\begin{center}
		\begin{framed}
			\Large
			Les trois parties sont indépendantes.
		\end{framed}
	\end{center}
}

\partie{\'Etude du capteur de l’appareil photographique}%
\enonce{%
	On considère dans un premier temps les paramètres géométriques du capteur principal, afin de vérifier si le phénomène de diffraction dégrade ou non l’image formée sur le capteur. Tous les pixels considérés sont carrés de coté $a$.
	% \figCap{0.4}{fig_capteur}{Module photographique d’un smartphone (source : wikimediacommons)}
}

\QR[5]{%
	À partir des données du téléphone portable fournies, déterminer la longueur $L$ et la largeur $l$ ($L>l$) du capteur de ce téléphone. En déduire la taille $a$ d’un pixel du capteur.
}{%
	Soit $L=4000a$ la longueur du capteur et $l=3000a$ sa largeur, avec $a$ la taille d'un pixel supposé carré. On a donc $l=3L/4$. \pt{1}

	La diagonale du capteur $d$ vaut $1/3$ de pouce soit $d=1/3\times 25,4=\SI{8.47}{mm}$. \pt{1}

	On en déduit
	\eq{%
		d^2=L^2+l^2=L^2\pa{1+\dfrac{3^2}{4^2}}=\pa{\dfrac{5}{4}L}^2
		\qdonc
		\boxed{L=\dfrac{4}{5}d=\SI{6.77}{mm}} \pt{1}
	}%

	On trouve alors
	\eq{%
		\boxed{l=\dfrac{3}{4}L\stm{=}\SI{5.08}{mm}
			\qet
			a=\dfrac{L}{4000}\stm{=}\SI{1.69}{\micro m}}
	}%
}

\enonce{%
	Le nombre d’ouverture $N$ de l’appareil photographique est défini par la formule ci-après, où $D$ est le diamètre de l’ouverture et $f'$ la distance focale de l’objectif,
	\eq{%
		N=\dfrac{f'}{D}
	}%

	On souhaite estimer la taille caractéristique de la tache de diffraction visible sur le capteur. Pour cela on  considère que  l’objectif est éclairé par un \og{}point unique\fg{} $A$ situé à l’infini. Les rayons issus de $A$ sont diffractés par le diaphragme de diamètre $D$. Les rayons extrêmes après diffraction sont caractérisés par l'angle $\theta$.

	% \figsvgCap{fig_diffraction.pdf_tex}{Tracé de rayons lors de la diffraction par le diaphragme d'ouverture $D$.}


}

\QR[4]{%
	Préciser où doit être placé le capteur pour former l'image de $A$, puis compléter la figure~\ref{annexe:diffraction} du document réponse en traçant les rayons émergents de la lentille issus des deux rayons incidents déjà tracés. Placer le rayon $R\ind{diff}$ de la tâche centrale de diffraction perçue sur le capteur.
}{%
	L'objet étant à l'infini sur l'axe optique, son image (en l'absence de diffraction) est $F'$. \pt{1} Donc on place le capteur dans le plan focal image. \pt{1}

	Pour déterminer le rayon émergent, on trace le rayon incident incliné de $\theta$ par rapport à l'axe optique et passant par $O$. Ce rayon n'est pas dévié et passe par $\phi'$. \pt{1}

	On fait de même pour l'autre rayon. On pose alors $R\ind{diff}$, \pt{1} distance entre $F'$ et $\phi'$.

	\figsvg{fig_diffractioncorr.pdf_tex}

}

\QR[6]{\label{Q:diffraction}
	Après avoir justifié la longueur d’onde lumineuse choisie, déterminer $R\ind{diff}$. Commenter.
}{%
	On choisit $\lambda=\SI{500}{nm}$, \pt{1} longueur dans le visible, proche du maximum de sensibilité de l'\oe il.

	La demi-ouverture angulaire de la tâche centrale de diffraction vaut $\theta\approx\dfrac{\lambda}{D}$. \pt{1}

	D'après la figure précédente, $\tan\theta=R\ind{diff}/f'$. La lentille étant utilisée dans les conditions de Gauss, on a
	\eq{%
		\tan\theta\approx\theta \stm{=}\dfrac{R\ind{diff}}{f'}
		\qdonc
		\boxed{R\ind{diff}=\dfrac{\lambda f'}{D}\stm{=}\lambda N}
	}%

	L'application numérique donne $\boxed{R\ind{diff}\approx\SI{1.1}{\micro m}}$.
	\pt{1}

	On obtient une valeur du même ordre de grandeur que $a$, la taille d'un pixel, donc on est à la limite de résolution \pt{1} de l'appareil liée au phénomène de diffraction.


}

\QR[3]{%
	Proposer une justification du choix de l’entreprise de configurer par défaut la prise d’image en full HD ($1920\times  1080$ pixels) au lieu de la résolution 4K ($3840 \times 2160$ pixels).
}{%
	Déterminons la taille des pixels dans les deux cas
	\eq{%
		a\ind{HD}=\dfrac{L}{1920}=\SI{3.5}{\micro m}>R\ind{diff}
		\stm{\qet}
		a\ind{4K}=\dfrac{L}{3840}=\SI{1.7}{\micro m}\approx R\ind{diff}
	}%

	Ainsi en full HD, on ne sera pas sensible à la diffraction \pt{1}. De plus, l'enregistrement de l'image nécessitera moins de capacité de mémoire \pt{1} qu'en résolution 4K sans que cela soit visible sur la qualité de l'image pour un photographe amateur.
}


\partie{Étude des aberrations géométriques dues à la lentille de l’appareil
	photographique}
\enonce{%
	La lentille de forme plano-convexe est constituée de silice fondue associée à
	du quartz. On peut la modéliser comme une demi-boule de rayon $R=\SI{4.0}{mm}$
	et d’indice de réfraction $n=1,52$, plongée dans l’air dont l’indice de
	réfraction est pris égal à 1. Un faisceau lumineux cylindrique, de rayon
	$r_m<R$, arrive sous incidence normale sur la face plane de cette lentille. On
	note $C$ l’intersection de la face plane de la demi-boule avec l’axe optique
	$(Ox)$ et $S$ l’intersection de la face hémisphérique avec ce même axe. On
	s’intéresse au rayon lumineux incident qui arrive parallèle à l’axe optique et
	à une distance $r$ de cet axe.
}

\sousPartie{ Condition de traversée de la lentille }%
\QR[10]{\label{Q:r0}%
	Compléter sur la figure~\ref{annexe:demiboule} du document réponse le trajet
	de ce rayon lumineux lors de son passage à travers la lentille.
	\begin{itemize}
		\item On notera $I$ le point d'incidence sur le dioptre verre-air et $J$ son
		      projeté orthogonal sur l'axe $x$.
		\item On notera $i$ l’angle d'incidence \textbf{orienté} sur le dioptre
		      verre-air et $t$ l’angle de réfraction correspondant.
	\end{itemize}
	Déterminer l’expression de $r_0$, valeur limite du rayon du faisceau à
	respecter si l’on souhaite que tous les rayons incidents émergent de la
	lentille. Calculer numériquement la valeur de $r_0$. En réalité, le
	constructeur a choisi un diamètre de la lentille $\Phi = \SI{5,0}{ mm}$.
	Justifier ce choix.
}{%
	Le rayon arrive normalement au dioptre air-verre, donc il n'est pas dévié. Il
	arrive avec un angle d'incidence $i$ sur le dioptre verre-air et est réfracté
	dans une direction caractérisée par l'angle de réfraction $t>i$ car
	$n\ind{air}<n$.

	\def\svgwidth{0.9\linewidth}
	\figsvgCap{fig_demiboulecorr.pdf_tex}{\protect \pt{1} pour incidence normale,
		\protect \pt{1} pour angle qui s'écarte de la normale, \protect \pt{1} pour
		les sens de comptage, \protect\pt{1} pour les flèches sur les rayons.}

	Le rayon réfracté existe tant que $t<\pi/2$, \pt{1} i.e.\ par application de la relation de Snell-Descartes
	\eq{%
		n\sin i<1
		\pt{1}
		\qavec
		\sin i=\dfrac{r}{R} \pt{1}
	}%

	On en déduit
	\eq{%
		r<\boxed{\dfrac{R}{n}\stm{=}r_0}
		\quad;\quad
		\xul{r_0\stm{=}\SI{2.6}{mm}}
	}%

	Le constructeur choisit un rayon $\Phi/2=\SI{2.5}{mm}<r_0<R$ afin de n'avoir que des rayons pouvant émerger. \pt{1} Néanmoins il risque d'y avoir des aberrations géométriques car il y a des rayons très éloignés de l'axe optique (cf la suite).
}


\sousPartie{ Étude des rayons proches de l’axe optique }%
\QR[5]{%
	On note $A'$ la position de l’intersection du rayon incident avec l’axe optique après son passage au travers de la lentille. Le placer sur le schéma annexe, puis montrer que la distance algébrique $\obarr{CA'}$ vérifie
	\eq{%
		\obarr{CA'}=R\cos i+\dfrac{R\sin i}{\tan (t-i)}
	}%
}{%
	\pt{1} Pour le schéma et l'angle $t-i$.
	\smallbreak
	On a $\obarr{CA'}=\obarr{CJ}+\obarr{JA'}$ \pt{1}. Dans les triangles $ICJ$ et $IJA'$ :
	\eq{%
		\obarr{CJ}=R\cos i \pt{1}
		\qet
		\obarr{JA'}=\dfrac{\obarr{JI}}{\tan(t-i)} \pt{1}
		\qavec
		\obarr{JI}=R\sin i
	}%
	On obtient alors
	\eq{%
		\boxed{\obarr{CA'}\stm{=}R\cos i+\dfrac{R\sin i}{\tan (t-i)}}
	}%
}


\QR[5]{%
	En déduire, en fonction de $R$ et $n$, l’expression de la limite $\obarr{CF'}$ de $\obarr{CA'}$ lorsque la distance $r$ tend vers 0. Calculer numériquement $\obarr{CF'}$.
}{%
	Si $r\rightarrow 0$, alors les angles $i$ et $t$ sont faibles, donc
	\eq{%
		\cos i\approx 1
		\qet
		\sin i\approx i
		\qet
		\tan (t-i)\approx t-i
		\qquad \pt{1}
	}%

	De plus, la relation de Snell-Descartes devient
	\eq{%
		ni\stm{\approx} t
		\qdonc
		\obarr{CA'}\approx R+R\times \dfrac{i}{ni-i}
		\qsoit
		\obarr{CA'}\stm{\approx} R\pa{1+\dfrac{1}{n-1}}
	}%

	On a alors
	\eq{%
		\boxed{\obarr{CF'}=\lim_{r\rightarrow 0}\obarr{CA'}\stm{=}\dfrac{nR}{n-1}}
	}%

	L'AN donne $\boxed{\obarr{CF'}=\SI{12}{mm}}$. \pt{1}
}

\QR[2]{%
	Dans quelles conditions peut-on considérer que le point $F'$ est
	stigmatiquement conjugué d’un point source situé à l’infini sur l’axe ?
	Comment peut-on nommer le point $F'$ ?
}{%
	Le point $F'$ est approximativement stigmatiquement conjugué d'un point source
	situé à l'infini sur l'axe optique dans le cas des rayons paraxiaux, i.e.\
	dans les conditions de \textsc{Gauss}. \pt{1} $F'$ est alors le foyer
	principal image \pt{1} de la lentille demi-boule utilisée dans les conditions
	de \textsc{Gauss}.
}

\sousPartie{ Étude de l’aberration sphérique de la lentille }%
\enonce{%
	Le capteur de l’appareil photographique est placé dans le plan focal image de la lentille (figure~\ref{fig:aberration}). On s’intéresse à présent au rayon lumineux qui arrive parallèle à l’axe optique à la distance $r<r_0$ de l’axe (question~\ref{Q:r0}).
	\def\svgscale{.8}
	\figsvgCap{fig_aberration.pdf_tex}{Aberration sphérique \label{fig:aberration}}

}

\QR[6]{%
	On définit la distance TSA (\textit{transversal spherical aberration}) comme
	la distance entre $F'$ et le point où le rayon émergent extrême correspondant
	à $r=r_0$ rencontre le capteur. Faire un schéma en faisant apparaitre la
	distance TSA. On notera $i_0$ l'angle d'incidence au niveau du dioptre
	sphérique verre-air.
}{%
	Pour $r=r_0$, $t=\pi/2$.
	\def\svgwidth{0.9\linewidth}
	\figsvgCap{fig_TSA.pdf_tex}{\protect \pt{1} Pour l'axe optique, \protect\pt{1}
		pour les flèches sur les rayons, \protect\pt{1} pour $t = \pi/2$,
		\protect\pt{1} pour les sens de comptage, \protect\pt{1} pour l'écran,
		\protect\pt{1} pour la TSA}
}

\QR[10]{%
	Montrer que la distance TSA peut se mettre sous la forme
	\eq{%
		TSA=\dfrac{r_0}{\sin^2i_0}\pa{\dfrac{\obarr{CF'}}{R}\cos i_0-1}
	}%

}{%
	D'après le théorème de Thalès :
	\eq{%
		\dfrac{\obarr{F'K}}{\obarr{JI}}\stm{=}\dfrac{\obarr{A'F'}}{\obarr{A'J}}
		\qavec
		\obarr{JI}=r_0\stm{=}R/n
		\quad ;\quad
		\obarr{A'F'}=\obarr{CF'}-\obarr{CA'}\stm{=}\obarr{CF'}-\pa{R\cos i_0+\dfrac{R\sin i_0}{\tan(\pi/2-i_0)}}
	}%

	Or $\dfrac{1}{\tan(\pi/2-i_0)}\stm{=}\tan i_0$, d'où
	\eq{%
		\obarr{A'F'}=\obarr{CF'}-\pa{R\cos i_0+R\sin i_0\tan i_0}\stm{=}\obarr{CF'}-\underbracket[1pt]{\pa{R\cos i_0+R\dfrac{\sin^2 i_0}{\cos i_0}}}_{=\dfrac{R}{\cos i_0} \pt{1}}>0
	}%

	De plus $\obarr{JA'}\stm{=}R\dfrac{\sin^2 i_0}{\cos i_0}>0$, on en déduit
	\eq{%
		TSA\stm{=}|\obarr{F'K}|=r_0\times\dfrac{\obarr{CF'}/R-\dfrac{1}{\cos  i_0}}{\dfrac{\sin^2 i_0}{\cos i_0}}
		\qsoit
		TSA \stm{=}r_0\pa{\dfrac{\obarr{CF'}}{R}\,\dfrac{\cos i_0}{\sin^2 i_0} -\dfrac{1}{\sin^2i_0}}
	}%

	On obtient alors la forme donnée dans l'énoncé
	\eq{%
		\boxed{TSA\stm{=}\dfrac{r_0}{\sin^2i_0}\pa{\dfrac{\obarr{CF'}}{R}\cos i_0-1}}
	}%

}

\QR[5]{%
	Exprimer alors  la distance TSA en fonction de $R$ et $n$. Faire l'application numérique. Comparer la distance TSA à la taille de la tache de diffraction obtenue en question~\ref{Q:diffraction} et à la dimension d’un pixel. Conclure sur l’adéquation de cette lentille au téléphone portable considéré.
}{%
	On a montré que $\sin i_0=1/n$ \pt{1}, on en déduit
	\eq{%
		\cos i_0=\sqrt{1-\sin^2 i_0}\stm{=}\dfrac{1}{n}\sqrt{n^2-1}
	}%
	De plus $r_0=R/n$ et $\obarr{CF'}=\dfrac{nR}{n-1}$. En remplaçant dans l'expression de la distance TSA obtenue à la question précédente, on obtient
	\eq{%
		TSA\stm{=}\dfrac{n^2R}{n}\pa{\dfrac{n}{n-1}\times  \dfrac{1}{n}\sqrt{n^2-1}-1}
		\qsoit
		\boxed{TSA\stm{=}nR\pa{\dfrac{\sqrt{n^2-1}}{n-1}-1}}
	}%

	L'AN donne $\boxed{TSA=\SI{7.3}{mm}}$. \pt{1}

	On a $TSA\gg R\ind{diff}$ et $TSA\gg a$. La lentille demi-boule est extrêmement sensible aux aberrations géométriques. \pt{1}
}

\partie{Estimation de la taille d’un pixel de l’écran d’un smartphone}
\enonce{%
	Depuis l’apparition des premiers smartphones, la qualité des écrans a fait des progrès considérables. Un célèbre constructeur de téléphones affirme que \og{}la densité de pixels des écrans est si élevée qu’à l’oeil nu et à une
	distance normale, il est impossible de discerner les pixels individuels\fg{}.

	L’objectif de cette partie est de vérifier si l’écran de l’objet d’étude vérifie ou non ce critère.
}

\QR[6]{%
	Montrer qu’il existe une taille de pixel maximale pour satisfaire à la
	description précédente. Calculer sa valeur numérique.
}{%
	L'\oe il pourra discerner deux pixels voisins s'ils sont vus sous un angle
	$\beta$ supérieur au pouvoir de résolution de l'\oe il
	$\alpha=\SI{3.0e-4}{rad}$. \pt{1}

	Un pixel de taille $a$, vu à une distance $D$ est observé sous un angle $\beta$ tel que
	\eq{%
		\tan \beta \stm{=}\dfrac{a}{D}
	}%

	On se place dans le cas où $\beta$ est faible (proche de $\alpha$ qui est faible), donc $\tan\beta\approx\beta$. \pt{1} On prend $D=\SI{30}{cm}$, on a alors
	\eq{%
		\dfrac{a}{D}>\alpha
		\qsoit
		\boxed{a \stm{>} D\alpha=\SI{9.0e-5}{m}=\SI{90}{\micro m}} \pt{1}
	}%

	Cette limite est très grande devant la taille d'un pixel qui est de l'ordre du
	micromètre. Donc on ne peut en effet pas distinguer deux pixels à l'\oe il nu.
	\pt{1}
}

\QR[8]{%
	Estimer la dimension $a$ d’un pixel de l’écran présenté dans la
	figure~\ref{fig:taillePixel}, ainsi que l’incertitude $u(a)$ de cette mesure,
	et conclure sur la véracité du propos du constructeur.
	\figCapRaw{fig_taillePixel}{Écran de téléphone, grossit 10 fois (source
		wikimediacommons)\label{fig:taillePixel}}
}{%
	Je mesure la hauteur du chiffre 4 : $h=\SI{1.7}{cm}$ pour $N=\SI{18}{pixels}$.
	\pt{1}

	J'évalue les incertitudes maximales~:
	\begin{itemize}
		\item $\Delta h=\SI{0.05}{cm}$ soit une demi-graduation \pt{1}
		\item $\Delta N=1$ \pt{1}
	\end{itemize}
	On choisit une loi uniforme pour calculer les incertitudes types correspondantes
	\eq{%
		u(h)=\dfrac{\Delta h}{\sqrt{3}}
		\stm{\qet}
		u(N)=\dfrac{\Delta N}{\sqrt{3}}
	}%

	La taille d'un pixel est donnée par la relation (en n'oubliant pas le facteur 10 du grossissement) :
	\eq{%
		a\stm{=}\dfrac{h}{10N}=\SI{94.4}{\micro m}
		\qet
		u(a)\stm{=}a\sqrt{\pa{\dfrac{u(h)}{h}}^2 + \pa{\dfrac{u(N)}{N}}^2}=\SI{3.4}{\micro m}
	}%

	Le résultat de la mesure est donc
	\[
		\boxed{a = \SI{94.4\pm 3.4}{\micro m}} \pt{1}
	\]
	On rappelle que l'on donne 2 chiffres significatifs sur l'incertitude-type et
	autant que nécessaire sur la valeur moyenne pour avoir la même précision que
	celle de l'incertitude-type.

	On constate que la valeur de $a$ est assez élevée. Elle n'est pas cohérente
	avec les caractéristiques de l'appareil photo données en début d'énoncé. \pt{1}
}

% \enonce{%
% 	\clearpage
% 	\lhead{Nom :}
% 	\rhead{Classe :\hspace*{3cm}}
%
% 	\begin{center}
% 		{\large \textbf{Document réponse}}
% 	\end{center}
%
% 	\figsvgCap{fig_diffraction.pdf_tex}{Tracé de rayons lors de la diffraction par le diaphragme d'ouverture $D$.\label{annexe:diffraction}}
%
% 	\figsvgCap{fig_demiboule.pdf_tex}{Modèle de la lentille demi-boule.\label{annexe:demiboule}}
%
% }

\end{document}%
