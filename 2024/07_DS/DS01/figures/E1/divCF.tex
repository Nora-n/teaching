\documentclass{standalone}
\usepackage{mintikz}

\begin{document}
\begin{tikzpicture}[]
	% Sens de comptage
	\node[] (Cp) at (-3.5,2.5) {$\oplus$};
	\draw[<->] (Cp.north west) |- (Cp.south east);
	% Sens de comptage
	\node[] (Cp) at (-2.5,2.5) {$\oplus$};
	\circledarrow{black}{(Cp)}{0.3}
  \def\xtr{4.5}
	% grid
	\draw[lightgray!10, ultra thin, help lines]
	(-\xtr,-3) grid (\xtr,3);
	\draw[lightgray!10, ultra thin, help lines, dotted, step=.5cm]
	(-\xtr,-3) grid (\xtr,3);
	% Axe optique
	\draw[thin, ->](-\xtr,0)--(\xtr,0)node[below]{A.O.};
	% ---------------------------------------------------------------------
	% Lentille 
	\coordinate (O) at (0,0);
	\def \fa{-2}
	\def \lsiz{2}
	\coordinate (F) at ($(O)+(-\fa,0)$);
	\coordinate (Fp) at ($(O)+(\fa,0)$);

	\foreach \x/\z in {O/\fa}{
	\draw[shift={(\x)}] (0,0)
	node[below right] {O$_{}$};
	\draw[shift={(\x)}] (\z,2pt) --++ (0,-4pt)
	node[below left] {F$'_{}$};
	\draw[shift={(\x)}] ({-\z},2pt) --++ (0,-4pt)
	node[below left] {F$_{}$};
	}

	\draw[shift={(O)}, thick,
		>-<, >=latex, name path=L] (0,-\lsiz)--(0,\lsiz)
	node[above]{$\Lc_{}$};
	% ---------------------------------------------------------------------
	% Objet 1
	\def \pos{1}
	\def \siz{1}
	\coordinate (A1) at ($(O)+(\pos,0)$);
	\coordinate (B1) at ($(A1)+(0,\siz)$);
	\draw[->, Purple!70, thick] (A1)
	node[below] {A$_1$}
	-- (B1)
	node[above] {B$_1$};
	% Rayons pour objet virtuel d'une lentille divergente
	% Lengths
	\len{(A1)}{(O)}{\posi}
	\len{(A1)}{(B1)}{\size}
	\len{(O)}{(F)}{\foc}
	% Rayon 1
	\draw[brandeisblue, simple] ($2*(O)-(B1)$) -- (O);
	\draw[orange!70, simple, name path=OB1p] (O) --++ ($2*(B1)-2*(O)$);
	% Rayon 2
	\def \mut{3}
	\draw[brandeisblue, double] ($(B1)-(\mut*\posi,0)$)
	--++ (\mut*\posi-\posi,0) coordinate (I);
	\draw[brandeisblue, dashed, double] (I) --++ (2*\posi,0);
	\def \mut{1.2}
	\draw[orange!70, doublerev, dashed] (I)
	--++ ($\mut*(Fp)-\mut*(I)$);
	\draw[orange!70, double, name path=IB1p] (I)
	--++ ($-\mut*(Fp)+\mut*(I)$);
	% Rayon 3
	\path[name path=tripleB1] (F) --++ ($3*(B1)-3*(F)$);
	\path[name intersections={of=L and tripleB1, by=E}];
	\def \mut{1.5}
	\draw[brandeisblue, triplerev] (E) --++ ($\mut*(B1)-\mut*(F)$);
	\draw[brandeisblue, dashed, triple] (E) --++ ($\mut*(F)-\mut*(E)$);
	\draw[orange!70, triple, name path=EB1p] (E) --++ (\mut*\foc,0);
	% Image
	\path[name intersections={of=OB1p and IB1p, by=B1\bp}];
	\len{(O)}{(E)}{\intsiz}
	\coordinate (A1\bp) at ($(B1\bp)+(0,-{\intsiz})$);
	\draw[->, Red!70, thick] (A1\bp)
	node[below] {A$_1$\bp}
	-- (B1\bp)
	node[below, right] {B$_1$\bp};
	% ---------------------------------------------------------------------
	% Objet 2
	\def \pos{4}
	\def \siz{1.5}
	\coordinate (A2) at ($(O)+(\pos,0)$);
	\coordinate (B2) at ($(A2)+(0,\siz)$);
	\draw[->, Purple!70, thick] (A2)
	node[below] {A$_2$}
	-- (B2)
	node[above] {B$_2$};
	% Rayons pour objet virtuel après F d'une lentille divergente
	% Lengths
	\len{(A2)}{(O)}{\posi}
	\len{(A2)}{(B2)}{\size}
	\len{(O)}{(F)}{\foc}
	% Rayon 1
	\draw[ForestGreen, simple, name path=OB2p] ($2*(O)-1.2*(B2)$) -- (O);
	\def \mut{1.5}
	\draw[firebrick!70, simple] (O) --++ ($\mut*(B2)-\mut*(O)$);
	% Rayon 2
	\def \mut{2.2}
	\draw[ForestGreen, double] ($(B2)-(\mut*\posi,0)$)
	--++ (\mut*\posi-\posi,0) coordinate (I);
	\draw[ForestGreen, dashed, double] (I) --++ (1.5*\posi,0);
	\def \mut{2.2}
	\draw[firebrick!70, doublerev, dashed, name path=IB2p] (I)
	--++ ($\mut*(Fp)-\mut*(I)$);
	\def \mut{1.2}
	\draw[firebrick!70, double] (I)
	--++ ($-\mut*(Fp)+\mut*(I)$);
	% Rayon 3
	\path[name path=tripleB2] (B2) --++ ($3*(F)-3*(B2)$);
	\path[name intersections={of=L and tripleB2, by=E}];
	\def \mut{1.2}
	\draw[ForestGreen, triplerev] (E) --++ ($\mut*(F)-\mut*(B2)$);
	\def \mut{2.2}
	\draw[ForestGreen, dashed, triple] (E) --++ ($\mut*(F)-\mut*(E)$);
	\draw[firebrick!70, triple] (E) --++ (\mut*\foc,0);
	\draw[firebrick!70, triplerev,
		dashed, name path=EB2p] (E) --++ (-\mut*\foc,0);
	% Image
	\path[name intersections={of=OB2p and IB2p, by=B2\bp}];
	\len{(O)}{(E)}{\intsiz}
	\coordinate (A2\bp) at ($(B2\bp)+(0,{\intsiz})$);
	\draw[->, Red!70, thick] (A2\bp)
	node[above] {A$_2$\bp}
	-- (B2\bp)
	node[below] {B$_2$\bp};
\end{tikzpicture}

\end{document}
