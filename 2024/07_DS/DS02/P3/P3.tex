\documentclass[../DS02.tex]{subfiles}%
\graphicspath{{./figures/}}%

% \subimport{/home/nora/Documents/Enseignement/Prepa/bpep/exercices/DS/lentille_app_photo/}{sujet.tex}

\begin{document}%
\section[75]"P"{Guirlandes électriques}

\enonce{%
	Dans ce problème, on cherche à optimiser l'alimentation électrique d'un
	système comportant deux guirlande électriques \(G_1\) et \(G_2\),
	chacune étant modélisée par un conducteur ohmique de résistance
	identique \(R_1 = R_2 = R\).

	La première guirlande est dédiée à un fonctionnement continu. La seconde
	est associée avec un interrupteur \(S\) en série qui bascule de manière
	périodique afin de produire un clignotement.

	On supposera dans ce problème que la puissance lumineuse fournie par ces
	guirlandes est proportionnelle à la puissance électrique qu'elles
	reçoivent.
}%

\subsection{Système de base}

\enonce{%
	On considère dans un premier temps le circuit ci-contre alimenté par un
	générateur réel de f.e.m. \(E\) et de résistance interne \(r\).
	\textbf{Les expressions demandées ne feront intervenir que \(E,r\) et
		\(R\).}

	FIGURE

	\emph{On considère que l'interrupteur \(S\) est ouvert.}
}%

\QR{%
	Quelle est la puissance reçue \(\mathcal{P}_{2,o}\) par la seconde
	guirlande \(G_2\)?
}{%
	solu
}%

\QR{%
	Établir l'expression du courant \(i_o\) passant à travers le
	générateur puis l'expression de la puissance électrique
	\(\mathcal{P}_{1,o}\) reçue par la guirlande \(G_1\).
}{%
	solu
}%

\enonce{%
	\emph{On considère désormais que l'interrupteur \(S\) est fermé.}
}%

\QR{%
	Établir l'expression du courant \(i_f\) passant à travers le
	générateur.
}{%
	solu
}%

\QR{%
	À l'aide d'un pont diviseur de courant, déterminer les expressions de
	\(i_{1,f}\) et \(i_{2,f}\).
}{%
	solu
}%

\QR{%
	Quelles sont alors les puissances \(\mathcal{P}_{1,f}\) et
	\(\mathcal{P}_{2,f}\) reçues par les deux guirlandes?
}{%
	solu
}%

\enonce{%
	\emph{Comparaisons des 2 situations.}
}%

\QR{%
	La puissance reçue par la première guirlande est-elle identique dans
	les deux situations étudiées (\(S\) ouvert et fermé)? Sachant qu'elle
	ne doit pas clignoter, est-ce un problème? Expliquer.
}{%
	solu
}%

\QR{%
	Comment doit-on choisir \(r\) par rapport à \(R\) pour limiter le
	problème? Cette condition est-elle vérifiée pour
	\(r = R = 1\ \Omega\)?
}{%
	solu
}%

\subsection{Système amélioré}

\enonce{%
On considère maintenant le circuit ci-dessous afin de limiter la
variation de puissance électrique reçue par la première guirlande, donc
la variation du courant \(i_1\).

Une bobine d'inductance \(L\) a donc été ajoutée en série avec la
première guirlande. L'interrupteur \(S\) est ouvert de manière
périodique pour \(t \in \left[0 ; \frac{T}{2} \right[\) et fermé pour
\(t \in \left[ \frac{T}{2} ; T\right[\).

FIGURE.
}%

\QR{%
	En régime stationnaire (permanent continu), donner le schéma
	équivalent du nouveau montage.
}{%
	solu
}%

\enonce{%
	\emph{On se place juste avant la fermeture de l'interrupteur},
	c'est-à-dire en \(t = \frac{T}{2}^-\), et on admet que le régime
	stationnaire a été atteint.
}%

\QR{%
	Déterminer la valeur de \(i_1\left(\frac{T}{2}^-\right)\). En déduire
	la valeur de \(i_1\left(\frac{T}{2}^+\right)\).
}{%
	solu
}%

\QR{%
	Déterminer les valeurs de \(i_2\left(\frac{T}{2}^-\right)\) et
	\(i_2\left(\frac{T}{2}^+\right)\).
}{%
	solu
}%

\enonce{%
\emph{On considère l'intervalle \(\left[0 ; \frac{T}{2} \right[\),
lorsque l'interrupteur est ouvert.}
}%

\QR{%
Établir l'équation différentielle dont \(i_1\) est solution sur
l'intervalle \(\left[0 ; \frac{T}{2} \right[\). On fera apparaître un
temps caractéristique \(\tau_o\).
}{%
solu
}%

\enonce{%
\emph{On s'intéresse maintenant à l'intervalle
\(\left[ \frac{T}{2} ; T\right[\), lorsque l'interrupteur est fermé.}
}%

\QR{%
	Montrer que \(i_1\) est solution de l'équation différentielle
	suivante:
	\[\dv{i_1}{t} + \dfrac{i_1}{\tau_f} =
		\dfrac{E}{L\left(1 + \dfrac{r}{R}\right)}
		\qav
		\tau_f = \dfrac{L\left(1 + \dfrac{r}{R}\right)}{2r + R}
	\]
}{%
	solu
}%

\QR{%
	Donner la forme générale \(i_1(t)\) de la solution de cette équation
	différentielle. On ne cherchera pas à déterminer la constante
	intervenant dans la solution.
}{%
	solu
}%

\enonce{%
	On étudie expérimentalement les variations du courant \(i_1(t)\) en
	mesurant la tension aux bornes de la guirlande \(G_1\) à l'aide d'un
	oscilloscope et on obtient le résultat suivant pour deux valeurs
	différentes de l'inductance \(L\). La résistance \(R\) vaut
	\(2\ \Omega\) et la résistance \(r\) vaut \(1\ \Omega\).

	FIGURE
}%

\QR{%
	Parmi les deux bobines d'inductance \(L_a\) et \(L_b\) , laquelle
	permet d'atteindre le régime stationnaire mentionné dans les questions
	8 à 10 ?
}{%
	solu
}%

\QR{%
	Retrouver, par lecture graphique, la valeur de \(L_a\).
}{%
	solu
}%

\QR{%
	Justifiez que \(L_b \gg L_a\), sans chercher à déterminer sa valeur.
}{%
	solu
}%

\QR{%
	Quelle est la valeur de l'inductance à retenir parmi \(L_a\) et
	\(L_b\) pour minimiser les variations de puissance reçue par la
	première guirlande?
}{%
	solu
}%

\end{document}%
