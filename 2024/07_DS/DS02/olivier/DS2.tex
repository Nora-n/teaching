Le devoir est composé de \textbf{4 parties indépendantes}:

\begin{itemize}
\item
  \textbf{Exercice:} Circuit de résistances
\item
  \textbf{Problème 1:} Étude d'une lampe de secours rechargeable
\item
  \textbf{Problème 2:} Guirlandes électriques
\item
  \textbf{Problème 3:} Étude d'une inductance
\end{itemize}

\begin{itemize}
\item
  La présentation, la lisibilité, l'orthographe, la qualité de la
  rédaction, la clarté et la précision des raisonnements entreront pour
  une part \emph{importante} dans l'appréciation des copies.
\item
  On portera une attention toute particulière à la vraisemblance des
  résultats obtenus : homogénéité des formules littérales, étude de cas
  particuliers, ordres de grandeur des valeurs numériques. \emph{Des
  points y seront attribués.}
\item
  Les réponses non justifiées et les applications numériques ne
  comportant pas d'unité ne donneront pas lieu à l'attribution de
  points.
\item
  Les résultats finaux \emph{non encadrés ou soulignés à la règle} ne
  seront tout simplement pas lus.
\item
  Si un candidat repère ce qui lui semble être une erreur d'énoncé, il
  le signale sur sa copie et poursuit sa composition en expliquant les
  raisons des initiatives qu'il est amené à prendre.
\end{itemize}

\hypertarget{exercice-circuit-de-ruxe9sistances}{%
\section*{Exercice: circuit de
résistances}\label{exercice-circuit-de-ruxe9sistances}}
\addcontentsline{toc}{section}{Exercice: circuit de résistances}

On considère le circuit ci-dessous.

\includegraphics[width=7.5cm,height=\textheight]{images-DS2/circuit-R.pdf}

\begin{enumerate}
\def\labelenumi{\arabic{enumi}.}
\item
  Déterminer parmi les 5 résistances, lesquelles sont en série et en
  parallèle.
\item
  En considérant que toute les résistances ont la même valeur
  \(R=1\ \mathrm{k}\Omega\), exprimer en fonction de \(R\) la résistance
  équivalente \(R_{AB}\): équivalente à toutes les résistances entre les
  points \(A\) et \(B\), soit \(R_1\), \(R_2\) et \(R_4\). Faire
  l'application numérique.
\item
  De même exprimer les résistances équivalentes \(R_{BC}\) et
  \(R_{AC}\). Faire les applications numériques.
\item
  Exprimer les tensions \(u_{AB}\) et \(u_{CB}\) en fonction de \(E\).
\item
  Exprimer les intensités \(I_1\) et \(I_4\) en fonction de \(I\).
\end{enumerate}

\newpage

\includegraphics[width=10cm,height=\textheight]{images-DS2/circuit-R-corr1.pdf}\\
\includegraphics[width=14cm,height=\textheight]{images-DS2/circuit-R-corr2.pdf}\\
\includegraphics[width=10cm,height=\textheight]{images-DS2/circuit-R-corr3.pdf}\\
\includegraphics[width=8cm,height=\textheight]{images-DS2/circuit-R-corr4.pdf}

\hypertarget{probluxe8me-1-etude-dune-lampe-de-secours-rechargeable}{%
\section*{Problème 1: Etude d'une lampe de secours
rechargeable}\label{probluxe8me-1-etude-dune-lampe-de-secours-rechargeable}}
\addcontentsline{toc}{section}{Problème 1: Etude d'une lampe de secours
rechargeable}

Il est recommandé d'avoir sur soi une lampe pour être vu en cas de
détresse ou tout simplement pour se déplacer par nuit noire. Pour ne pas
avoir à gérer des piles défaillantes ou des accumulateurs non chargés,
une ``lampe à secouer'' peut s'avérer utile. Un extrait d'une
description publicitaire de cet objet est rapporté ci-dessous.

En secouant la lampe 30 secondes (un peu comme une bombe de peinture),
de l'énergie électrique est produite et stockée dans un condensateur.
Vous obtenez alors environ 20 minutes d'une lumière produite par une DEL
(diode électroluminescente).

Si vous n'utilisez pas toute l'énergie produite, elle restera stockée
dans le condensateur pendant plusieurs semaines pour être ensuite
immédiatement disponible sur simple pression du bouton marche/arrêt.

On part d'une situation où on suppose que le condensateur vient d'être
chargé et que la tension à ses bornes est \(U_0 = 3,3\ \mathrm{V}\). On
cesse alors d'agiter la lampe et donc de recharger le condensateur.

Tout d'abord, on étudie la décharge de ce condensateur de capacité
\(C = 10\ \mathrm{F}\) (``supercondensateur'') dans un conducteur
ohmique de résistance \(R\) pouvant modéliser une lampe à incandescence.
Le circuit étudié est donc représenté par le schéma de la
{[}@fig:lampe1{]}. La partie de circuit utile lors de la phase de charge
du condensateur n'est pas représentée.

À l'instant initial \(t = 0\ \mathrm{s}\), on ferme l'interrupteur K et
la décharge commence.

\begin{figure}
\hypertarget{fig:lampe1}{%
\centering
\includegraphics[width=6cm,height=\textheight]{images-DS2/pb-lampe-fig1.png}
\caption{Circuit électrique équivalent lors de la phase de décharge du
condensateur}\label{fig:lampe1}
}
\end{figure}

\begin{enumerate}
\def\labelenumi{\arabic{enumi}.}
\tightlist
\item
  Établir l'équation différentielle vérifiée par \(u_c(t)\) pendant la
  décharge en faisant apparaître une constante de temps \(\tau\) dont on
  donnera l'expression. Puis déterminer l'expression littérale de la
  solution de cette équation différentielle.
\end{enumerate}

Au bout d'une durée environ égale à \(5\tau\) on peut considérer que la
décharge du condensateur est quasi-complète.

\begin{enumerate}
\def\labelenumi{\arabic{enumi}.}
\setcounter{enumi}{1}
\tightlist
\item
  Si l'on considère que cette durée est égale à 20 minutes (comme
  précisé dans le document fourni), déterminer la valeur de la
  résistance \(R\) du conducteur ohmique qu'il faut alors associer au
  condensateur de capacité \(C = 10\ \mathrm{F}\).
\end{enumerate}

Certains modèles électriques plus élaborés du ``supercondensateur''
utilisé ici permettent de traduire, plus fidèlement à la réalité, son
comportement réel dans un circuit. Un des modèles possibles fait
apparaître, autour de la capacité \(C\), une résistance \(R_f\) en
parallèle et une résistance série \(R_s\) conformément au schéma de la
{[}@fig:lampe2{]}.

\begin{figure}
\hypertarget{fig:lampe2}{%
\centering
\includegraphics[width=7cm,height=\textheight]{images-DS2/pb-lampe-fig2.pdf}
\caption{Modèle plus fidèle à la réalité pour le
``supercondensateur''}\label{fig:lampe2}
}
\end{figure}

\begin{enumerate}
\def\labelenumi{\arabic{enumi}.}
\setcounter{enumi}{2}
\tightlist
\item
  Pour quelles valeurs limites de \(R_s\) et \(R_f\) retrouve-t-on le
  modèle simple (\(C\) seul) du ``supercondensateur''?
\end{enumerate}

Pour la suite des questions, on revient au modèle simple (\(C\) seul)
pour le condensateur, toujours initialement chargé sous une tension
\(U_0 = 3,3\ \mathrm{V}\).

On remplace maintenant le conducteur ohmique de résistance \(R\) par une
DEL dont les caractéristiques sont les suivantes
({[}@fig:lampe3;@fig:lampe4{]}):

\begin{figure}
\hypertarget{fig:lampe3}{%
\centering
\includegraphics[width=12cm,height=\textheight]{images-DS2/pb-lampe-fig3.png}
\caption{Caractéristique \(i=f(u_d)\) et symbole pour la diode
électroluminescente DEL.}\label{fig:lampe3}
}
\end{figure}

\begin{figure}
\hypertarget{fig:lampe4}{%
\centering
\includegraphics[width=14cm,height=\textheight]{images-DS2/pb-lampe-fig4.pdf}
\caption{Electrical \& Optical Characteristics}\label{fig:lampe4}
}
\end{figure}

Pour cette diode, on appelle tension seuil, notée \(U_S\) la tension
minimale au-delà de laquelle la diode devient passante. On convient
alors que la diode électroluminescente cesse d'émettre suffisamment de
lumière dès que \(u_d < U_S + 0,1\ \mathrm{V}\).

\begin{enumerate}
\def\labelenumi{\arabic{enumi}.}
\setcounter{enumi}{3}
\item
  Comment se comporte la DEL lorsque la diode est bloquée :
  (\(u_d < 2,3\ \mathrm{V}\)). D'autre part, proposer un modèle
  électrique équivalent pour la DEL lorsqu'elle est passante
  (\(u_d > 2,3\ \mathrm{V}\)), sous forme d'un générateur de Thévenin
  (valeurs numériques attendues). On fera le schéma électrique
  correspondant en précisant bien les sens de l'intensité de de la
  tension \(u_d\).
\item
  Faire le schéma électrique de la LED modélisée et insérée dans le
  circuit précédent. Puis, montrer que la nouvelle équation
  différentielle régissant l'évolution de \(u_c(t)\) lorsque le
  condensateur se décharge dans la diode électroluminescente est
  \(\derive{u_c}{t} + \dfrac{u_c(t)}{\tau'} = \dfrac{U_S}{\tau'}\).
  Préciser l'expression de \(\tau'\).
\item
  Déterminer la solution \(u_c(t)\) de cette nouvelle équation
  différentielle, avec les mêmes conditions initiales que précédemment.
  Puis représenter graphiquement l'allure de son évolution en fonction
  du temps, en mettant en évidence les points importants du graphe
  (valeur et tangente à l'origine ainsi qu'une asymptote éventuelle).
\item
  Déterminer l'expression littérale de \(i(t)\). Puis représenter
  graphiquement l'allure de son évolution en fonction du temps, en
  mettant en évidence les points importants.
\item
  À l'aide des caractéristiques techniques fournies dans la
  {[}@fig:lampe4{]}, indiquer si le fonctionnement correct de la DEL est
  garanti sans dommage. Proposer une solution pour éventuellement
  remédier au problème rencontré (valeur numérique attendue).
\item
  Prévoir, sans la mise en œuvre de la solution précédente, la durée
  approximative d'éclairage de cette lampe notée \(T\). (on rappelle que
  \(\ln(10) \simeq 2,3\)). Conclure.
\item
  Exprimer, en fonction de \(U_0\) et de
  \(U_{Fi} = U_S + 0,1\ \mathrm{V}\), le pourcentage d'énergie restante
  dans le condensateur lorsque la DEL cesse d'émettre de la lumière par
  rapport à l'énergie initiale accumulée. (On ne cherche pas à la
  calculer, mais on estime ici ce pourcentage à environ 50 \%).
\end{enumerate}

\newpage

\includegraphics[width=17cm,height=\textheight]{images-DS2/pb-lampe-corr1.pdf}

\begin{enumerate}
\def\labelenumi{\arabic{enumi}.}
\setcounter{enumi}{2}
\tightlist
\item
  Avec \(R_f\to\infty\) (interrupteur ouvert) et \(R_s = 0\) (fil), le
  modèle devient équivalent à \(C\) seul.
\end{enumerate}

\includegraphics[width=17cm,height=\textheight]{images-DS2/pb-lampe-corr2.pdf}\\
\includegraphics[width=17cm,height=\textheight]{images-DS2/pb-lampe-corr3.pdf}

\hypertarget{probluxe8me-2-guirlandes-uxe9lectriques}{%
\section*{Problème 2: Guirlandes
électriques}\label{probluxe8me-2-guirlandes-uxe9lectriques}}
\addcontentsline{toc}{section}{Problème 2: Guirlandes électriques}

Dans ce problème, on cherche à optimiser l'alimentation électrique d'un
système comportant deux guirlande électriques \(G_1\) et \(G_2\),
chacune étant modélisée par un conducteur ohmique de résistance
identique \(R_1 = R_2 = R\).

La première guirlande est dédiée à un fonctionnement continu. La seconde
est associée avec un interrupteur \(S\) en série qui bascule de manière
périodique afin de produire un clignotement.

On supposera dans ce problème que la puissance lumineuse fournie par ces
guirlandes est proportionnelle à la puissance électrique qu'elles
reçoivent.

\hypertarget{systuxe8me-de-base}{%
\subsection*{Système de base}\label{systuxe8me-de-base}}
\addcontentsline{toc}{subsection}{Système de base}

On considère dans un premier temps le circuit ci-dessous alimenté par un
générateur réel de f.e.m. \(E\) et de résistance interne \(r\).
\textbf{Les expressions demandées ne feront intervenir que \(E,r\) et
\(R\).}

\emph{On considère que l'interrupteur \(S\) est ouvert.}

\begin{enumerate}
\def\labelenumi{\arabic{enumi}.}
\item
  Quelle est la puissance reçue \(\mathcal{P}_{2,o}\) par la seconde
  guirlande \(G_2\)?
\item
  Établir l'expression du courant \(i_o\) passant à travers le
  générateur puis l'expression de la puissance électrique
  \(\mathcal{P}_{1,o}\) reçue par la guirlande \(G_1\).
\end{enumerate}

\includegraphics[width=7.5cm,height=\textheight]{images-DS2/pb-guilandes-1.png}

\emph{On considère désormais que l'interrupteur \(S\) est fermé.}

\begin{enumerate}
\def\labelenumi{\arabic{enumi}.}
\setcounter{enumi}{2}
\item
  Établir l'expression du courant \(i_f\) passant à travers le
  générateur.
\item
  À l'aide d'un pont diviseur de courant, déterminer les expressions de
  \(i_{1,f}\) et \(i_{2,f}\).
\item
  Quelles sont alors les puissances \(\mathcal{P}_{1,f}\) et
  \(\mathcal{P}_{2,f}\) reçues par les deux guirlandes?
\end{enumerate}

\emph{Comparaisons des 2 situations.}

\begin{enumerate}
\def\labelenumi{\arabic{enumi}.}
\setcounter{enumi}{5}
\item
  La puissance reçue par la première guirlande est-elle identique dans
  les deux situations étudiées (\(S\) ouvert et fermé)? Sachant qu'elle
  ne doit pas clignoter, est-ce un problème? Expliquer.
\item
  Comment doit-on choisir \(r\) par rapport à \(R\) pour limiter le
  problème? Cette condition est-elle vérifiée pour
  \(r = R = 1\ \Omega\)?
\end{enumerate}

\hypertarget{systuxe8me-amuxe9lioruxe9}{%
\subsection*{Système amélioré}\label{systuxe8me-amuxe9lioruxe9}}
\addcontentsline{toc}{subsection}{Système amélioré}

On considère maintenant le circuit ci-dessous afin de limiter la
variation de puissance électrique reçue par la première guirlande, donc
la variation du courant \(i_1\).

Une bobine d'inductance \(L\) a donc été ajoutée en série avec la
première guirlande. L'interrupteur \(S\) est ouvert de manière
périodique pour \(t \in \left[0 ; \frac{T}{2} \right[\) et fermé pour
\(t \in \left[ \frac{T}{2} ; T\right[\).

\includegraphics[width=8cm,height=\textheight]{images-DS2/pb-guilandes-2.png}

\begin{enumerate}
\def\labelenumi{\arabic{enumi}.}
\setcounter{enumi}{7}
\tightlist
\item
  En régime stationnaire (permanent continu), donner le schéma
  équivalent du nouveau montage.
\end{enumerate}

\emph{On se place juste avant la fermeture de l'interrupteur},
c'est-à-dire en \(t = \frac{T}{2}^-\) , et on admet que le régime
stationnaire a été atteint.

\begin{enumerate}
\def\labelenumi{\arabic{enumi}.}
\setcounter{enumi}{8}
\item
  Déterminer la valeur de \(i_1\left(\frac{T}{2}^-\right)\). En déduire
  la valeur de \(i_1\left(\frac{T}{2}^+\right)\).
\item
  Déterminer les valeurs de \(i_2\left(\frac{T}{2}^-\right)\) et
  \(i_2\left(\frac{T}{2}^+\right)\).
\end{enumerate}

\emph{On considère l'intervalle \(\left[0 ; \frac{T}{2} \right[\),
lorsque l'interrupteur est ouvert.}

\begin{enumerate}
\def\labelenumi{\arabic{enumi}.}
\setcounter{enumi}{10}
\tightlist
\item
  Établir l'équation différentielle dont \(i_1\) est solution sur
  l'intervalle \(\left[0 ; \frac{T}{2} \right[\). On fera apparaître un
  temps caractéristique \(\tau_o\).
\end{enumerate}

\emph{On s'intéresse maintenant à l'intervalle
\(\left[ \frac{T}{2} ; T\right[\), lorsque l'interrupteur est fermé.}

\begin{enumerate}
\def\labelenumi{\arabic{enumi}.}
\setcounter{enumi}{11}
\item
  Montrer que \(i_1\) est solution de l'équation différentielle
  suivante:
  \[\derive{i_1}{t} + \dfrac{i_1}{\tau_f} = \dfrac{E}{L\left(1 + \dfrac{r}{R}\right)} \qquad \text{avec } \tau_f = \dfrac{L\left(1 + \dfrac{r}{R}\right)}{2r + R}\]
\item
  Donner la forme générale \(i_1(t)\) de la solution de cette équation
  différentielle. On ne cherchera pas à déterminer la constante
  intervenant dans la solution.
\end{enumerate}

On étudie expérimentalement les variations du courant \(i_1(t)\) en
mesurant la tension aux bornes de la guirlande \(G_1\) à l'aide d'un
oscilloscope et on obtient le résultat suivant pour deux valeurs
différentes de l'inductance \(L\). La résistance \(R\) vaut
\(2\ \Omega\) et la résistance \(r\) vaut \(1\ \Omega\).

\includegraphics[width=17cm,height=\textheight]{images-DS2/pb-guilandes-3.png}

\begin{enumerate}
\def\labelenumi{\arabic{enumi}.}
\setcounter{enumi}{13}
\item
  Parmi les deux bobines d'inductance \(L_a\) et \(L_b\) , laquelle
  permet d'atteindre le régime stationnaire mentionné dans les questions
  8 à 10 ?
\item
  Retrouver, par lecture graphique, la valeur de \(L_a\).
\item
  Justifiez que \(L_b \gg L_a\), sans chercher à déterminer sa valeur.
\item
  Quelle est la valeur de l'inductance à retenir parmi \(L_a\) et
  \(L_b\) pour minimiser les variations de puissance reçue par la
  première guirlande?
\end{enumerate}

\newpage

\includegraphics[width=18cm,height=\textheight]{images-DS2/pb-guilandes-corr1.png}\\
\includegraphics[width=18cm,height=\textheight]{images-DS2/pb-guilandes-corr2.png}\\
\includegraphics[width=18cm,height=\textheight]{images-DS2/pb-guilandes-corr3.png}

\hypertarget{probluxe8me-3-uxe9tude-dune-inductance}{%
\section*{Problème 3: Étude d'une
inductance}\label{probluxe8me-3-uxe9tude-dune-inductance}}
\addcontentsline{toc}{section}{Problème 3: Étude d'une inductance}

On considère le circuit ci-dessous, composé de 3 résistances
\(R_1 , R_2\) et \(R_3\), d'une bobine idéale d'inductance \(L\) et
d'une source idéale de force électromotrice \(E\). On suppose que
l'interrupteur \(K\) est ouvert depuis un temps très grand devant le
temps caractéristique d'évolution du circuit.

\includegraphics[width=8.5cm,height=\textheight]{images-DS2/pb-L.png}

En \(t = 0\), on ferme l'interrupteur.

\begin{enumerate}
\def\labelenumi{\arabic{enumi}.}
\item
  Dessiner le circuit équivalent en \(t = 0^-\) . En déduire \(i(0^-)\).
\item
  Déterminer la valeur de \(i(0^+)\).
\item
  Dessiner le circuit équivalent en \(t \to\infty\). En déduire
  \(i(+\infty)\), notée \(i_\infty\) dans la suite.
\item
  Déterminer l'équation différentielle vérifiée par \(i(t)\) pour des
  temps \(t > 0\). On pourra poser la constante de temps \(\tau\)
  suivante: \[\tau = \dfrac{L(R_2 + R_3)}{R_1R_2 + R_2R_3 + R_1R_3}\]
\item
  Déterminer la solution de cette équation différentielle. L'exprimer
  fonction de \(t, \tau\) et \(i_\infty\).
\end{enumerate}

On suppose dans la suite que l'interrupteur \(K\) est fermé depuis un
temps très grand devant le temps caractéristique d'évolution du circuit.
À \(t = 0\) (nouvelle origine des temps), on ouvre l'interrupteur.

\begin{enumerate}
\def\labelenumi{\arabic{enumi}.}
\setcounter{enumi}{5}
\item
  Déterminer l'équation différentielle vérifiée par \(i(t)\) pour
  \(t>0\).
\item
  Déterminer l'énergie initialement stockée dans la bobine (en
  \(t = 0\), lorsque l'on ouvre l'interrupteur), ainsi que l'énergie
  dissipée par effet Joule par l'ensemble des résistances entre
  \(t = 0\) et \(t \to\infty\).
\end{enumerate}

\includegraphics[width=18cm,height=\textheight]{images-DS2/pb-L-corr1.png}\\
\includegraphics[width=18cm,height=\textheight]{images-DS2/pb-L-corr2.png}
