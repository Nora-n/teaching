\documentclass[../DS03.tex]{subfiles}%
\graphicspath{{./figures/}}%

\begin{document}%
\section[31]"E"{Synthèse de l'ammoniac\ifcorrige{~\small\textit{(CCP TSI 2013)}}}

\enonce{%
	L’ammoniac \ce{NH_3(g)} est un intermédiaire important dans l’industrie
	chimique qui l’utilise comme précurseur pour la production d’engrais,
	d’explosifs et de polymères.
	\smallbreak
	En 2010, sa production mondiale était d’environ 130 millions de tonnes. La
	production de telles quantités de ce gaz a été rendue possible par
	l’apparition du procédé HaberBosch qui permet la synthèse de l’ammoniac à
	partir du diazote, présent en abondance dans l’atmosphère, et du dihydrogène,
	obtenu par reformage du méthane à la vapeur d’eau, selon la réaction :
	\begin{gather*}
		\ce{N_2_{\gaz} + 3 H_2_{\gaz} = 2 NH_3_{\gaz}}
		\tag*{\makebox[0pt][r]{$K^\circ(\SI{723}{K}) = \num{2.8e-5}$}}
	\end{gather*}
	Cette transformation chimique étant lente, on utilise un catalyseur à base de
	fer pour l’accélérer.
	\smallbreak
	Les réactifs de la synthèse, diazote et dihydrogène, sont introduits en
	proportions stœchiométriques dans le réacteur qui est maintenu, tout au
	long de la synthèse, à une pression totale $P = \num{300}P^\circ$ et à une
	température $T$ de \SI{723}{K}.
	\smallbreak
	On notera $n_0$ la quantité de matière initiale de diazote introduit dans le
	réacteur.
}

\QR[5]{%
  Réaliser un tableau d’avancement pour les lignes initiales et intermédiaires.
  Laissez la ligne équilibre vide pour la compléter après.
}{%
  On dresse le tableau d'avancement~:
  \begin{center}
    \def\rhgt{0.50}
    \centering
    \begin{tabularx}{.7\linewidth}{|l|c||YdYdY||Y|}
      \hline
      \multicolumn{2}{|c||}{
        $\xmathstrut{\rhgt}$
      \textbf{Équation}} \pt{1}+\pt{1} &
      $\ce{N_2_{\gaz}}$ & $+$       &
      $3\ce{H_2_{\gaz}}$ & $=$       &
      $2\ce{NH_3_{\gaz}}$ &
      $n\ind{tot,gaz}$
          \pt{1} \\
      \hline
      $\xmathstrut{\rhgt}$
      Initial & $\xi = 0$ & $n_0$ & \vline    & $3n_0$ & \vline    & $0$ &
      $4n_0$ 
          \tikzmark{MA} \\
      \hline
      $\xmathstrut{\rhgt}$
      Interm.             & $\xi$ & $n_0 - \xi$ & \vline    & $3n_0 - 3\xi$ &
      \vline    & $2\xi$ & $4n_0 - 2\xi$ 
          \tikzmark{MB} \\
      \hline
      $\xmathstrut{\rhgt}$
      Équilib.            & $\xi$ & $n_0(1-\rho)$ & \vline    & $3n_0(1-\rho)$ &
      \vline    & $2\rho n_0$ & $2n_0 (2-\rho)$
          \tikzmark{MC} \\
      \hline
    \end{tabularx}
  \end{center}
  \tikz[remember picture, overlay]
  \node[above right=-8pt and 35pt of pic cs:MA] {\pt{1}}
  ; \tikz[remember picture, overlay]
  \node[above right=-8pt and 24pt of pic cs:MB] {\pt{1}}
  ; \tikz[remember picture, overlay]
  \node[above right=-8pt and 18pt of pic cs:MC] {\fgr{+1 Q2}}
  ;
}

\QR[4]{%
	Rappeler la définition du rendement. Exprimer le rendement à l'équilibre
	$\rho$ de la synthèse en fonction de $n_0$ et $\xi\ind{eq}$.
	En déduire les expressions des quantités de matière en fonction de
	$n_0$ et $\rho$ en complétant le tableau précédent.
}{%
	On cherche $\xi\ind{max}$~: les réactifs étant introduits dans les proportions
	stœchiométriques, on trouve $\xi\ind{max}$ à partir de l'un des deux
	réactifs~:
	\begin{gather*}
		n_0 - \xi\ind{max} = 0 \Lra \boxed{\xi\ind{max} \stm[-1]{=} n_0}
		\\\beforetext{Ainsi,}
		\rho \stm{=} \frac{\xi\ind{eq}}{\xi\ind{max}}
		\Lra
		\boxed{\rho \stm{=} \frac{\xi\ind{eq}}{n_0}}
	\end{gather*}
  d'où la dernière ligne du tableau précédent.
}

% \QR{%
% }{%
% 	\begin{center}
% 		\def\rhgt{0.50}
% 		\centering
% 		\begin{tabularx}{.7\linewidth}{|l|c||YdYdY||Y|}
% 			\hline
% 			\multicolumn{2}{|c||}{
% 				$\xmathstrut{\rhgt}$
% 			\textbf{Équation}}  &
% 			$\ce{N_2_{\gaz}}$   & $+$       &
% 			$3\ce{H_2_{\gaz}}$  & $=$       &
% 			$2\ce{NH_3_{\gaz}}$ &
% 			$n\ind{tot,gaz}$              \\
% 			\hline
% 			$\xmathstrut{\rhgt}$
% 			Initial             & $\xi = 0$ &
% 			$n_0$               & \vline    &
% 			$3n_0$              & \vline    &
% 			$0$                 &
% 			$4n_0$                            \\
% 			\hline
% 			$\xmathstrut{\rhgt}$
% 			Équilib.            & $\xi$     &
% 			$n_0(1-\rho)$       & \vline    &
% 			$3n_0(1-\rho)$      & \vline    &
% 			$2\rho n_0$         & 
% 			$2n_0 (2-\rho)$
% 		      \tikzmark{MC}                                  \\
% 			\hline
% 		\end{tabularx}
% 	\end{center}
% }

\QR[5]{%
	Donner la définition du quotient réactionnel pour cette équation. Relier
	alors la constante d’équilibre $K^\circ$ aux pressions partielles à
	l’équilibre des différents constituants du système et à la pression standard
	$P^\circ$.
}{%
	\begin{DispWithArrows*}
		Q_r &\stm{=}
		\frac{a(\ce{NH3})^2}{a(\ce{N2})a(\ce{H2})^3}
		\Arrow{$K^\circ \stm[-1]{=} Q_{r,\ind{eq}}$}
		\\\Ra
		K^\circ &\stm{=}
		\frac{a(\ce{NH3})\ind{eq}^2}{a(\ce{N2})\ind{eq}a(\ce{H2})\ind{eq}^3}
		\Arrow{$a(X_{\gaz}) \stm{=} \frac{P_X}{P^\circ}$}
		\\\Lra
		\Aboxed{%
			K^\circ &\stm{=}
			\frac{%
        P(\ce{NH3})\ind{eq}^2(P^\circ)^2
      }{%
        P(\ce{N2})\ind{eq}P(\ce{H2})\ind{eq}^3}
		}
	\end{DispWithArrows*}
}

\QR[2]{%
	Relier la constante d’équilibre $K^\circ$ aux quantités de matière à
	l’équilibre des différents constituants du système, à la quantité de matière
	totale à l’équilibre $n\ind{tot}$, à la pression totale $P$ et à la pression
	standard $P^\circ$.
}{%
	\vspace{-25pt}
	\begin{gather*}
		\beforetext{Loi de \textsc{Dalton}~:}
		P(\ce{NH3})\ind{eq} \stm{=}
		\cfrac{n(\ce{NH3})\ind{eq}}{n\ind{tot}}\times P
		\\\Lra
		\boxed{%
			K^\circ \stm{=}
			\cfrac{%
				n(\ce{NH3})\ind{eq}^2n\ind{tot}^2
			}{%
				n(\ce{N2})\ind{eq}n(\ce{H2})\ind{eq}^3
			}
			\times
			\pa{\cfrac{P^\circ}{P}}^2
		}
	\end{gather*}
}

\QR[2]{%
	En déduire alors la relation entre $K^\circ$, $\rho$, $P$ et $P^\circ$.
}{%
	On injecte les expressions des quantités de matière à l'équilibre :
	\begin{align*}
		K^\circ & \stm{=}
		\cfrac{\pa{2n_0\rho}^2\pa{2n_0(2-\rho)}^2}{n_0(1-\rho)\times\pa{3n_0(1-\rho)}^3}\times \pa{\cfrac{P^\circ}{P}}^2
		\\\Lra
		\Aboxed{%
		K^\circ & \stm{=}
			\cfrac{16\rho^2(2-\rho)^2}{27(1-\rho)^4}\times \pa{\cfrac{P^\circ}{P}}^2
		}
	\end{align*}
}

\QR[10]{%
	Montrer que $\rho$ est solution d'un polynôme de degré 2, de la forme
	\[
		C\rho^{2} - 2C\rho + C-4 = 0
	\]
	avec $C$ une constante à exprimer en fonction de $K^\circ$
	\textbf{uniquement}. Donner alors l'expression analytique de $\rho$ en
	fonction de $C$. Application numérique.
}{%
		\begin{DispWithArrows*}[fleqn, mathindent=0pt, groups]
			K^\circ &=
			\cfrac{16\rho^2(2-\rho)^2}{27(1-\rho)^4}\times \pa{\cfrac{P^\circ}{P}}^2
			\CArrow{$\sqrt{(\cdot)}$}
			\\\Lra
			\sqrt{K^\circ} &\stm{=}
			\frac{4\rho(2-\rho)}{3 \sqrt{3}(1-\rho)^2}
			\underbracket[1pt]{\frac{P^\circ}{P}}_{=1/300}
			\Arrow{On rassemble}
			\\\Lra
			300 \sqrt{K^\circ}\cdot 3 \sqrt{3} (1-\rho)^2 &\stm{=} 4\rho(2-\rho)
			\Arrow{On développe}
			\\\Lra
			900 \sqrt{3K^\circ} (1-2\rho + \rho^2) &\stm{=} 8\rho - 4\rho^2
			\Arrow{On factorise}
			\\\Lra
			\rho^2 (900 \sqrt{3K^\circ} + 4) - 2\rho (900 \sqrt{3K^\circ}+4) + 900
			\sqrt{3K^\circ} &\stm{=} 0
			\Arrow{On identifie}
			\\\Lra
			\Aboxed{C\rho^{2} - 2C\rho + C-4 &= 0}
			\qav
			\boxed{C \stm{=} 900 \sqrt{3K^\circ} + 4}
      % \Arrow{Discriminant}
		\end{DispWithArrows*}
    \begin{isd}[lefthand ratio=.3]
      \vspace{-15pt}
      \begin{DispWithArrows*}[]
			\\\Ra
			\Delta &= 4C^2 - 4C(C-4)
			\\\Lra
			\Aboxed{\Delta &\stm{=} 16C}
      \end{DispWithArrows*}
      \tcblower
      \vspace{-15pt}
      \begin{DispWithArrows*}[]
              \\\qso
                  \rho_{\pm } &\stm{=} \frac{2C \pm 4 \sqrt{C}}{2C}
                  \\\Lra
                  \Aboxed{\rho_{\pm} &\stm{=} 1 \pm \frac{2}{\sqrt{C}}}
                       \qav
                           \left\{
                           \begin{array}{rcl}
                             C       & = & 900 \sqrt{3K^\circ} + 4
                             \\
                             K^\circ & = & \num{2.8e-5}
                           \end{array}
                           \right.
                     \\
                  \AN
                  \makebox[0pt][l]{$\xul{\phantom{\rho_+ = \num{1.57}}} \phantom{\qou} \xul{\phantom{\rho_- = \num{0.43}}}$}
                  \rho_+ &= \num{1.57} \stm{\qou} \rho_- = \num{0.43}
      \end{DispWithArrows*}
    \end{isd}
		Le rendement ne pouvant être supérieur à 1, on obtient alors $\boxed{\rho =
				\num{0.43}}$ \pt{1}
}%

\QR[3]{%
	Partant d'un état d'équilibre, on diminue la pression totale $P$ à température
	constante. Comment évolue le rendement $\rho$ ?
}{%
	En diminuant $P$, on augmente le quotient réactionnel \pt{1} par rapport à une
	situation d'équilibre où $Q=K^\circ$. La température étant constante, la
	constante d'équilibre reste la même \pt{1}. Donc $Q>K^\circ$, donc la réaction
	évolue dans le sens indirect ce qui a pour effet de diminuer le rendement.
	\pt{1}
}
\end{document}
