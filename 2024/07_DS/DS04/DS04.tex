\documentclass[a4paper, 10pt, garamond]{book}
\usepackage{cours-preambule}
\usepackage{tocloft}

\renewcommand{\mtcSfont}{\Large}
\setlength{\mtcindent}{5pt}
\mtcsetoffset{minitoc}{-5pt}
\addtolength{\cftsecnumwidth}{10pt}
\setcounter{minitocdepth}{1}

\def\lspace{25}

\dominitoc
\faketableofcontents

% \toggletrue{student}
\toggletrue{corrige}
% \renewcommand{\mycol}{black}
% \renewcommand{\mycol}{gray}

\makeatletter
\renewcommand{\@chapapp}{MPSI3 -- 13 décembre 2024 -- Devoir surveillé}
\makeatother

\setlist[blocQR,1]{leftmargin=10pt, label=\sqenumi}
\counterwithin*{equation}{section}

\hfuzz=5.003pt

\begin{document}
\setcounter{chapter}{3}

\settype{enon}
\settype{solu_prof}
\settype{solu_stud}

\chapter{\cswitch{Correction du DS}{Cinétique chimique et résonances}}
\label{ch:ds04}

\enonce{
\begin{center}
	\Large\bfseries
	Tout moyen de communication est interdit
	\smallbreak
	Les téléphones portables doivent être éteints et rangés dans les sacs
	\smallbreak
	\xul{Les calculatrices sont \textit{interdites}}
\end{center}
\begin{tcn}[cnt, bld](ror)<itc>""{Au programme}
	\large
	Transformations de la matière jusqu'à cinétique chimique, oscillateurs
	électriques et mécaniques en RSF et résonance, notions élémentaires de
	fonction de transfert.
\end{tcn}

{
\vfill
\let\item\olditem
\minitoc
\vfill
}

Les différentes questions peuvent être traitées dans l'ordre désiré.
\textbf{Cependant}, vous indiquerez le numéro correct de chaque question. Vous
prendrez soin d'indiquer sur votre copie si vous reprenez une question d'un
exercice plus loin dans la copie, sous peine qu'elle ne soit ni vue ni
corrigée.
\bigbreak
Vous porterez une attention particulière à la \textbf{qualité de rédaction}.
Vous énoncerez clairement les hypothèses, les lois et théorèmes utilisés. Les
relations mathématiques doivent être reliées par des connecteurs logiques.
\bigbreak
Vous prendrez soin de la \textbf{présentation} de votre copie, notamment au
niveau de l'écriture, de l'orthographe, des encadrements, de la marge et du
cadre laissé pour la note et le commentaire. Vous \textbf{encadrerez les
	expressions littérales}, sans faire apparaître les calculs. Vous ferez
apparaître cependant le détail des grandeurs avec leurs unités. Vous
\textbf{soulignerez les applications numériques}.
\bigbreak
Ainsi, l'étudiant-e s'expose aux malus suivants concernant la forme et le
fond~:
\begin{tcb}*(prop)"bomb"{Malus}
	\begin{minipage}[t]{0.50\linewidth}
		\begin{itemize}
			\item A~: application numérique mal faite~;
			\item N~: numéro de copie manquant~;
			\item P~: prénom manquant~;
			\item E~: manque d'encadrement des réponses~;
			\item M~: marge non laissée ou trop grande~;
			\item V~: confusion ou oubli de vecteurs~;
		\end{itemize}
	\end{minipage}
	\begin{minipage}[t]{0.50\linewidth}
		\begin{itemize}
			\item Q~: question mal ou non indiquée~;
			\item C~: copie grand carreaux~;
			\item U~: mauvaise unité (flagrante)~;
			\item H~: homogénéité non respectée~;
			\item S~: chiffres significatifs non cohérents~;
			\item $\f$~: loi physique fondamentale brisée.
		\end{itemize}
	\end{minipage}
\end{tcb}

% \begin{tcb}(impo){Exemple application numérique}
% 	\vspace*{-10pt}
% 	\begin{minipage}[c]{0.45\linewidth}
% 		\begin{gather*}
% 			\boxed{n = \frac{PV}{RT}}
% 			\qav
% 			\left\{
% 			\begin{array}{rcl}
% 				p & = & \SI{1.0e5}{Pa}                \\
% 				V & = & \SI{1.0e-3}{m^3}              \\
% 				R & = & \SI{8.314}{J.mol^{-1}.K^{-1}} \\
% 				T & = & \SI{300}{K}
% 			\end{array}
% 			\right.\\
% 			\mathrm{A.N.~:}\quad
% 			\xul{n = \SI{5.6e-4}{mol}}
% 		\end{gather*}
% 	\end{minipage}
% 	\hfill
% 	\cancel{\bcancel{
% 			\begin{minipage}[c]{0.45\linewidth}
% 				\begin{gather*}
% 					n = \frac{PV}{RT} = \frac{\num{e5}\cdot\num{1}}{8.32\cdot300}
% 					= 0.56
% 				\end{gather*}
% 			\end{minipage}
% 		}}
% \end{tcb}
\vfill

\begin{tcn}[cnt, bld](impo){Attention}
	\large
	Au moins un exercice de transformation de la matière et un exercice
	d'oscillateurs doivent être traités~!
\end{tcn}

\vfill

\newpage
}

\iftoggle{corrige}{}{
	\begin{center}
		\begin{framed}
			\Large\bfseries Remarques antérieures
		\end{framed}
	\end{center}
	\begin{enumerate}
		\item Inscrire dans le cadre attitré une remarque \textbf{pertinente} issue
		      du \textbf{DS04} de l'\textbf{année précédente}.
		\item De même avec une remarque pertinente du \textbf{DS03} de \textbf{cette
			      année}.
	\end{enumerate}
}


\setcounter{section}{0}
\resetQ

\subfile{E1/E1.tex}

\iftoggle{corrige}{}{%
	\newpage
}

\resetQ
\subfile{E2/E2.tex}

\iftoggle{corrige}{}{%
	\clearpage
}

\setcounter{section}{0}
\resetQ

\subfile{P1/P1.tex}

\iftoggle{corrige}{}{%
	\clearpage
}

\resetQ
\subfile{P2/P2.tex}

% \iftoggle{corrige}{}{%
% 	\clearpage
% }
%
% \resetQ
% \subfile{P3/P3.tex}

\end{document}
