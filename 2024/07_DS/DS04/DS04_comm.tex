\documentclass[a4paper, 10pt, final, garamond]{book}
\usepackage{cours-preambule}

\makeatletter
\renewcommand{\@chapapp}{Devoir surveill\'e -- num\'ero}
\makeatother

\begin{document}
\setcounter{chapter}{3}

\def\lspace{25}

\chapter{Commentaires sur le DS \oldno4}

\section{Commentaires généraux}
\subsection{Appréciation globale}
\subsection{Sur la forme}
Numérotez les \textbf{copies} et pas les pages, et numérotez les copies en
donnant le nombre de copie maximal~! Copie 1 $\Ra$ et alors~? Copie 1/2 $\Ra$ il
existe une autre copie. À savoir et à ne pas manquer.
\smallbreak
\textbf{Indiquez quand il y a des questions de l'autre côté d'une page
	pratiquement vide~!!} Un petit «~TSVP~» pour «~tournez s'il vous plaît~» fait
toute une différence sur le sentiment que vous faites un effort à alléger le
travail de correction, et ça ira très loin dans l'appréciation générale de votre
copie.
\smallbreak
Encore de bons efforts sur les commentaires antérieurs. Par contre, essayez de
vous \textbf{approprier} les commentaires. Je vous demanderai de ne \textbf{pas
	simplement me citer} mais de \textbf{reformuler avec vos mots} les points
importants que j'énonce. J'augmenterai les points bonus en conséquence. Si c'est
bien fait, ça pourra être vraiment bien récompensé.
\begin{tcn}(exem){Exemple}
	\begin{itemize}
		\item Premier exemple, commentaire DS04 2023~:
		      \begin{itemize}
			      \item Citation simple~: «~il faut connaître la différence entre
			            méthode intégrale/différentielle~» :
			            \pt{1} bonus actuellement.
			      \item Reformulation~: «~méthode différentielle = on trace $\ln (v)$
			            et la pente est l'ordre partiel~; méthode intégrale = on travaille
			            sur
			            $[\ce{A}](t)$ et le résultat dépend de l'ordre (0, 1 ou 2)~»
			            \pt{4}
			            facilement~!
		      \end{itemize}
		\item Second exemple, commentaire DS03 2024~:
		      \begin{itemize}
			      \item «~$\xi\ind{eq} \neq \xi\ind{max}$~» même en conditions
			            stœchiométriques~» \pt{1} ou peut-être \pt{2}
			      \item «~les conditions stœchiométriques portent sur les conditions
			            \textbf{initiales} de proportionnalité entre les réactifs,
			            mais ne dit
			            rien sur l'état final~: il n'est pas forcément maximal~!»
			            \pt{4}
			            facile~!
		      \end{itemize}
	\end{itemize}
\end{tcn}

Le but n'est pas que vous soyez des scribes, mais que vous preniez le temps de
vous poser des questions \textbf{et cherchiez les réponses} en lisant les
commentaires des années précédentes. Enfin, \textbf{n'en faites pas trop non
	plus}, une ou deux phrases suffisent, ne refaites pas de démonstration dans le
cadre remarque… Et vous pouvez continuer à ne faire que citer des portions des
commentaires, je ne vous pénaliserai pas, mais je veux que vous commenciez à
faire plus que ça.

\subsection{Commentaires principaux et récurrents}

% \begin{figure}[htbp!]
% 	\centering
% 	\includegraphics[width=\linewidth]{DS04_hist_all}
% 	\caption{Graphique des résultats}
% \end{figure}

\setcounter{section}{0}
\section[65]"E"{Étude d'un circuit RLC parallèle}
\begin{enumerate}
	\item[n]{3} % Q1
	      Ça ne sert à rien d'ajouter les admittances 2 par 2~! Si vous avez 3
	      impédances en série, vous faites $\Zu = \Zu_1+\Zu_2+\Zu_3$. Pour 3
	      impédances en parallèle, $\Yu = \Yu_1+\Yu_2+\Yu_3$~!
	\item[n]{5} % Q2
	      Ne sautez pas sur un pont diviseur de tension quand vous avec un circuit
	      avec une seule impédance équivalente. Choisissez votre méthode en fonction
	      de ce que vous avez.
	      \smallbreak
	      Répondez à la question~: donnez $\w_0$ et pas $\w_0{}^2$.
	      \textbf{Identifiez}. Gros problèmes d'identification…
	\item[n]{2} % Q3
	      N'oubliez pas la racine carrée~!
	\item[n]{8} % Q4
	      Résonance = amplitude max \textbf{pour $\boxed{\w \neq 0}$}~!
	\item[n]{8} % Q5
	      La bande passante c'est une différence de \textbf{pulsations} (ou
	      fréquences), donc vous ne pouvez pas écrire
	      \[
		      \Delta{\w} = \frac{U\ind{max}}{\sqrt{2}}
		      \quad \red{\circled{$-$H}}
	      \]
	      Il faut mettre les trinômes \textbf{sous forme de trinôme}~!
	      \[
		      \boxed{a x^2 + b x + c = 0}
		      \qMath{et pas}
		      \cancel{ax + \frac{b}{x} + c = 0}
	      \]
	\item[n]{7} % Q6
	      \textbf{Ne confondez pas équivalent asymptotique et limite~!}
	      {
		      \Large
		      \[
			      \Sim_{x \to 0} \neq \opto{}{x \to 0}
		      \]
	      }
	      Tracez, bon sang~!
	\item[n]{6} % Q7
	      \leavevmode\vspace*{-15pt}\relax
	      {\Large
		      \[
			      \phi(x) = \arg*{\boxed{\xul{U_0}}}
			      \qMath{et pas}
			      \phi(x) = \arg*{\abs{\xul{U_0}}}
		      \]
	      }
	      Sinon c'est l'argument d'un réel positif, c'est toujours nul…
	      \[
		      \boxed{\f \neq \phi \neq \Phi \neq \varnothing}
		      \quad \text{!}
	      \]
	      \smallbreak
	      On peut toujours composer par $\tan(\cdot)$. C'est pour composer par
	      $\boxed{\arctan(\cdot)}$ qu'il faut faire attention~!
	\item[n]{5} % Q8
	\item[n]{4} % Q9
	      Horrible, horrible représentation du courant $\eta$ sur le schéma. Il a
	      échappé à ma vigilance. Je vous présente mes excuses pour vos yeux
	      meurtris.
	\item[n]{5} % Q10
	\item[n]{5} % Q11
	\item[n]{2} % Q12
\end{enumerate}

\section[37]"E"{Monoxyde et dioxyde d'azote}
\begin{enumerate}
	\item[n]{6} % Q1
	      Il y a 4 fois plus de diazote que d'oxygène dans l'air. Cf.\ premier
	      exercice de TDTM2\_app.
	\item[n]{11} % Q2
	      Il faut voir que la réaction était quasi-nulle~!
	\item[n]{7} % Q3
	      Constante de réaction $K^\circ \neq k$ constante de vitesse…
	      Retour sur les confusions entre favorisé et sens de réaction. N'oubliez
	      pas la puissance sur la concentration initiale dans $k\ind{app}$.
	\item[n]{4} % Q4
	      Vous êtes tombé-es dans le panneau. On trace $\ln (v)$, ça na \textbf{rien à
		      voir} avec $\ln c(t)$. \textbf{La méthode différentielle} ($\ln (v)$) n'est
	      \textbf{pas la méthode intégrale} (régressions variées).
	\item[n]{4} % Q5
	      Les vitesses $v_1$ et $v_2$ sont différentes~! Ça se voit avec les
	      régressions. Ne partez pas d'une égalité clairement fausse.
	\item[n]{2} % Q6
	      Les proportions stœchiométriques n'ont \textbf{RÀV} avec le fait que les
	      ordres partiels soient ou non égaux aux coefficients stœchiométriques.
	\item[n]{3} % Q7
\end{enumerate}

\setcounter{section}{0}
\section[52]"P"{Suivi cinétique de la formation du dibrome}
\begin{enumerate}
	\item[n]{2} % Q1
	      Bien.
	\item[n]{8} % Q2
	      Faire un schéma pour montrer que chaque concentration est divisée par
	      2~! Cf.\ TP11… \textbf{Une seule personne a vu la dilution}~!
	      \smallbreak
	      Faites des tableaux d'avancement~!
	\item[n]{6} % Q3
	\item[n]{13} % Q4
	\item[n]{9} % Q5
	      \textbf{Tracez les données avec des croix}~! On se moque de la droite
	      toute seule. \textbf{N'invoquez pas le coefficient de corrélation} pour
	      justifier votre régression. Il faut que la droite passe par les points.
	      \textbf{Les coefficients directeurs et ordonnées à l'origine ont une
		      unité}~!
	\item[n]{2} % Q6
	      $t_{1/2}$ se trouvait directement dans le tableau~!
	\item[n]{9} % Q7
	\item[n]{3} % Q8
\end{enumerate}

\section[86]"P"{Résonance d'un verre}
\begin{enumerate}
	\item[n]{5} % Q1
	      Faites un effort sur les chiffres significatifs sur votre lecture…
	\item[n]{10} % Q2
	      Encore une fois, c'est $\ux$ et pas $\vv{x}$~!!
	      \smallbreak
	      Décomposez entièrement les forces sur les vecteurs de base $\ux$ et
	      $\uy$.
	      \smallbreak
	      \textbf{N'inventez pas des conditions initiales} si elles ne sont pas
	      données.
	      \smallbreak
	      Lisez bien l'énoncé~: $\ell_0 = 0$~! Même pas besoin de changement de
	      variable, $x(t)$ c'est déjà $\ell(t)$.
	      \smallbreak
	      Un axe c'est une droite, $Ox$ par exemple, mais $Ox$ a l'unité d'une
	      distance~; un vecteur de base c'est $\ux$, qui est \textbf{u}nitaire,
	      pas d'unité. Donc $\dcancel{\vv{Ox}}$~!
	      \smallbreak
	      \textbf{Faites un schéma~!}
	\item[n]{5} % Q3
	      \leavevmode\vspace*{-15pt}\relax
	      \begin{itemize}
		      \item Écrivez le PFD en version \textbf{vectorielle} avant toute
		            potentielle écriture en colonnes.
		      \item Quand vous \textbf{projetez} sur $\ux$, il ne reste \textbf{que
			            des scalaires}~! Vous ne pouvez pas écrire
		            \begin{gather*}
			            \beforetext{Sur $\ux$}
			            \vv{f} + \vv{F_r} = m \af
		            \end{gather*}
		            La relation vectorielle n'est vrai que pour la somme de tous les
		            vecteurs, vous ne pouvez pas extraire une partie de l'équation.
		            C'est comme si vous écriviez
		            \begin{gather*}
			            a+b = c+d
			            \\\beforetext{donc j'extrait}
			            b = d
		            \end{gather*}
		      \item Identifiez $Q$ et $\w_0$~!
	      \end{itemize}
	\item[n]{3} % Q4
	      $\w_0$ pulsation propre s'il n'y avait pas de frottements (oscillateur
	      harmonique), sinon c'est $\W \neq \w_0$.
	\item[n]{9} % Q5
	      Frottements faibles $\neq$ frottements nuls~! Frottements faibles $\Ra Q
		      \gg 1$~!
	\item[n]{4} % Q6
	\item[n]{4} % Q7
	\item[n]{2} % Q8
	\item[n]{5} % Q9
	\item[n]{4} % Q10
	\item[n]{2} % Q11
	\item[n]{4} % Q12
	\item[n]{3} % Q13
	\item[n]{10} % Q14
	\item[n]{2} % Q15
	\item[n]{2} % Q16
	\item[n]{7} % Q17
	\item[n]{5} % Q18
\end{enumerate}

\end{document}
