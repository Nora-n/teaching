\documentclass[../DS04.tex]{subfiles}
\graphicspath{{./figures/}}

% \subimport{/home/nora/Documents/Enseignement/Prepa/bpep/exercices/DS/chimie_NOx_thermo_cinetique/}{sujet.tex}

\begin{document}

\section[37]"E"{Monoxyde et dioxyde d'azote\ifcorrige{~\small\textit{(D'après PSI Centrale
		Supélec 2020)}}}

\enonce{%
	En consommant du kérosène, les moteurs d'avions entraînent le rejet de
	nombreux polluants parmi lesquels les monoxyde d'azote \ce{NO}
	et dioxyde d'azote \ce{NO2}, regroupés sous l'appellation NOx. Ces composés
	sont responsables d'une augmentation de la production d'ozone
	\ce{O3}, gaz à effet de serre, dans la basse atmosphère.
	\bigbreak
	Le monoxyde d'azote \ce{NO} est obtenu par oxydation du diazote \ce{N2} à
	haute température lors de la phase de combustion,
	modélisée par l'équation de réaction
	\begin{gather*}
		\ce{N2_{\gaz} + O2_{\gaz} = 2NO_{\gaz}}
		\tag*{\makebox[0pt][r]{$K^\circ_1 (\SI{1450}{K}) = \num{6.0e-6}$}}
	\end{gather*}
}

\QR[6]{%
	La réaction se produit dans l'air. Rappeler la fraction molaire du diazote et
	du dioxygène dans l'air. En déduire le tableau d'avancement de la réaction en
	notant $n_0$ la quantité de matière initiale en $\ce{O2_{\gaz}}$. Vous
	garderez une ligne libre pour la compléter ensuite.
}{%
	\vspace{-15pt}
	\[
		x_{\ce{O2}} = \num{0,2}
		\stm{\qet}
		x_{\ce{N2}} = \num{0,8} = 4 \cdot x_{\ce{O_2}}
	\]
	\begin{center}
		\def\rhgt{0.50}
		\centering
		\begin{tabularx}{.8\linewidth}{|l|c||YdYdY||Y|}
			\hline
			\multicolumn{2}{|c||}{
				$\xmathstrut{\rhgt}$
			\textbf{Équation}} \pt{1}+\pt{1} &
			$\ce{N_2_{\gaz}}$                & $+$                   &
			$\ce{O_2_{\gaz}}$                & $=$                   &
			$2\ce{NO_{\gaz}}$                &
			$n\ind{tot, gaz}$
			\makebox[0pt][l]{\pt{1}}                                   \\
			\hline
			$\xmathstrut{\rhgt}$
			Initial                          & $\xi = 0$             &
			$4n_0$                           & \vline                &
			$n_0$                            & \vline                &
			$0$                              &
			$5n_0$
			\tikzmark{MA}                                              \\
			\hline
			$\xmathstrut{\rhgt}$
			Équili.                          & $\xi\ind{eq}$         &
			$4n_0 - \xi\ind{eq}$             & \vline                &
			$n_0 - \xi\ind{eq}$              & \vline                &
			$2\xi\ind{eq}$                   &
			$5n_0$
			\tikzmark{MB}                                              \\
			\hline
			$\xmathstrut{\rhgt}$
			Final                            & $\xi\ind{eq} \ll n_0$ &
			$\approx 4n_0$                   & \vline                &
			$\approx n_0$                    & \vline                &
			$\num{4.9e-3}n_0$                &
			$5n_0$
			\tikzmark{MC}                                              \\
			\hline
		\end{tabularx}
	\end{center}
	\tikz[remember picture, overlay]
	\node[above right=-8pt and 35pt of pic cs:MA] {\pt{1}};
	\tikz[remember picture, overlay]
	\node[above right=-8pt and 35pt of pic cs:MB] {\pt{1}};
	\tikz[remember picture, overlay]
	\node[above right=-8pt and 30pt of pic cs:MC] {\fgr{+1 Q2}};
}

\QR[11]{%
	En faisant une hypothèse sur l'avancement à l'équilibre, évaluer la fraction
	molaire de monoxyde d'azote \ce{NO} présente à l'équilibre dans de l'air
	chauffé à \SI{1450}{\kelvin}. En déduire la quantité $n_{\ce{NO},\eql}$ à
	l'équilibre en fonction de $n_0$. Compléter alors la dernière ligne du tableau
	d'avancement précédent.
}{%
	~
	\vspace{-15pt}
	\smallbreak
	\begin{isd}[interior hidden, lefthand ratio=.45]
		Comme $K^\circ_1\ll 1$, on peut supposer la réaction quasi-nulle \pt{1}, soit
		\begin{gather*}
			\xi\ind{eq} \ll n_0
			\quad \stm{\Ra} \quad
			\left\{
			\begin{array}{ll}
				x_{\ce{O_2},\eql} & \approx x_{\ce{O_2},0}
				\\
				x_{\ce{N_2},\eql} & \approx x_{\ce{N_2},0}
			\end{array}
			\right.
			\\\Ra
			x_{\ce{NO},\eql} \stm{\ll} (x_{\ce{O_2},\eql}~;~x_{\ce{N_2},\eql})
		\end{gather*}
		D'après la loi d'action de masse et la loi de \textsc{Dalton} $p_{\ce{X}} =
			x_{\ce{X}}p\ind{tot}$ \pt{1},
		\tcblower
		\begin{gather*}
			K^\circ_1 \stm{=}
			\frac{p_{\ce{NO},\eql}^2}{p_{\ce{N_2},\eql}p_{\ce{O_2},\eql}} =
			\frac{x_{\ce{NO},\eql}^2}{x_{\ce{N_2},\eql}x_{\ce{O_2},\eql}}
			\\\Lra
			\boxed{%
				x_{\ce{NO},\eql} \stm{=}
				\sqrt{K_1^\circ \cdot x_{\ce{N_2},0} \cdot x_{\ce{O_2},0}}
			}
			\qav
			\left\{
			\begin{array}{rcl}
				K_1^\circ      & = & \num{6.0e-6}
				\\
				x_{\ce{N_2},0} & = & \num{0.8}
				\\
				x_{\ce{O_2},0} & = & \num{0.2}
			\end{array}
			\right.\\
			\AN
			\xul{
				x_{\ce{NO},\eql} \stm{=} \num{9.8e-4} \ll \num{0.2}
			}
		\end{gather*}
	\end{isd}
	\textbf{L'hypothèse est bien vérifiée} \pt{1}, et on trouve
	\[
		x_{\ce{NO},\eql} \stm{=} \frac{2\xi\ind{eq}}{5n_0}
		\Lra
		\boxed{2\xi\ind{eq} = 5x_{\ce{NO},\eql}n_0}
		\Ra
		\xul{n_{\ce{NO},\eql} \stm{\approx} \num{4.9e-3}n_0}
	\]
}

\enonce{%
	Dans un deuxième temps, il y a production de dioxyde d'azote \ce{NO2} à partir
	du monoxyde d'azote \ce{NO}, modélisée par l'équation de réaction
	\begin{gather*}
		\ce{2NO_{\gaz} + O_2_{\gaz} = 2NO_2_{\gaz}}
		\tag*{\makebox[0pt][r]{$K_2^\circ(\SI{400}{K}) = \num{2.0e7}$}}
	\end{gather*}
	La figure~\ref{fig:cinetique} fournit pour cette réaction les résultats d'une
	étude cinétique réalisée à \SI{400}{\kelvin}, où $v$ représente la dérivée
	temporelle de l'avancement volumique de la réaction. Les concentrations
	initiales utilisées dans cette étude sont
	\begin{itemize}
		\item[b]{Expérience 1} (tracé avec $\times$)~:
		      $[\ce{O_2}]_{01} = \SI{5.0e-3}{\mole\cdot\liter^{-1}}$,
		      $[\ce{NO}]_{01} = \SI{10}{\micro\mole\cdot\liter^{-1}}$,
		      $[\ce{NO2}]_{01} = 0$ ;
		\item[b]{Expérience 2} (tracé avec $+$)~:
		      $[\ce{O2}]_{02} = \SI{2,0e-3}{\mole\cdot\liter^{-1}}$,
		      $[\ce{NO}]_{02} = \SI{10}{\micro\mole\cdot\liter^{-1}}$,
		      $[\ce{NO2}]_{02} = 0$.
	\end{itemize}

	\begin{figure}[h]
		\centering
		\begin{tikzpicture}
			\begin{axis}[
					scale only axis,
					width=100mm, height=50mm,
					xlabel={$\ln\bigl([{\rm NO}]\bigr)$ (\si{\micro\mole\cdot\liter^{-1}})},
					xmin=0, xmax=2.5, xtick distance=0.5,
					ylabel=$\ln(v)$ (\si{\micro\mole\cdot\liter^{-1}\cdot\minute^{-1}}),
					ymin=-4, ymax=1, ytick distance=1,
					grid=major
				]
				\addplot [mark=x, only marks] coordinates {
						(1.46, -0.9) (1.67, -0.45) (1.87, -0.09) (2, 0.2) (2.14, 0.45) (2.22, 0.66) (2.265, 0.69) (2.3, 0.8)
					};
				\addplot [mark=+, only marks] coordinates {
						(0.46, -3.82) (0.79, -3.15) (1.16, -2.4) (1.42, -1.82) (1.77, -1.26) (2, -0.7) (2.14, -0.45) (2.3, -0.1)
					};

				\addplot [domain=1.4:2.4] {1.9871*x - 3.7841};
				\node at (5.5cm,4.15cm) [below]{$R^2 = \num{0,99975}$};
				\node at (5.5cm,4.15cm)[above] {$y = \num{1,9871} x - \num{3,7841}$};
				\addplot [domain=0.4:2.4] {1.9979*x - 4.723};
				\node at (7.5cm, 1.5cm) [below] {$R^2 = \num{0,99915}$};
				\node at (7.5cm, 1.5cm)  [above] {$y = \num{1,9979} x - \num{4,723}$};
			\end{axis}
		\end{tikzpicture}
		\caption{}
		\label{fig:cinetique}
	\end{figure}
}

\QR[7]{\label{q:kapp}%
	Commenter la valeur de la constante de réaction. Écrire l'expression de la
	vitesse de la réaction. On appelle $p$ l'ordre partiel sur $\ce{NO}$ et $q$
	l'ordre partiel sur $\ce{O_2}$, \textbf{supposés entiers}. En analysant les
	concentrations initiales des expériences proposées, simplifier la loi de
	vitesse en faisant apparaître une constante de vitesse apparente $k\ind{app}$.
	Comment s'appelle cette méthode~?
}{\label{q:kapp}%
	Cette réaction peut être considérée comme quasi-totale car $K^\circ_2\gg 1$.
	\pt{1} Avec les ordres partiels indiqués, on trouve
	\smallbreak
	\noindent
	\begin{isd}
		\vspace{-15pt}
		\begin{align*}
			v(t)          & \stm{=} k [\ce{NO}](t)^p \cdot [\ce{O_2}](t)^q
			\\
			\beforetext{Or, on a}
			[\ce{O_2}]_0  & \stm{\gg}{} [\ce{NO}]_0
			\\\Ra
			[\ce{O_2}](t) & \stm{\approx}{} [\ce{O_2}]_0
		\end{align*}
		\tcblower
		\vspace{-15pt}
		\begin{gather*}
			\Ra
			v(t)         = k [\ce{O_2}]_0^q \cdot [\ce{NO}](t)^p
			\\\Lra
			\boxed{v(t) \stm{=} k\ind{app} [\ce{NO}](t)^p}
			\qav \boxed{k\ind{app} \stm{=} k [\ce{O_2}]_0^q}
		\end{gather*}
		C'est la méthode de \textbf{dégénérescence de l'ordre}. \pt{1}
	\end{isd}
	\vspace{-15pt}
}

\QR[4]{%
	Expliquer l'intérêt de la régression linéaire effectuée. Comment s'appelle
	cette méthode~? Déterminer alors l'ordre partiel $p$.
}{%
	C'est la \textbf{méthode différentielle} \pt{1}. Il s'agit de tracer $\ln
		(v(t))$ en fonction de $\ln ([\ce{NO}](t))$. En effet, on a alors
	\begin{gather*}
		\ln (v(t)) \stm{=} q \cdot \ln ([\ce{NO}](t)) + \ln (k\ind{app})
		\Lra
		y          \stm{=} a \cdot x + b
	\end{gather*}
	c'est-à-dire que le coefficient directeur est l'ordre partiel $p$, et
	l'ordonnée à l'origine est $\ln (k\ind{app})$. En l'occurrence, pour les deux
	expériences on trouve
	\[
		a \approx 2
		\Ra
		\xul{p = 2} \pt{1}
	\]
}%

\QR[4]{%
	Toujours dans le cadre des expériences $1$ et $2$, montrer que
	\[
		q =
		\frac{%
			\ln (k\ind{app,1}) - \ln (k\ind{app,2})
		}{%
			\ln ([\ce{O_2}]_{01}) - \ln ([\ce{O_2}]_{02})
		}
	\]
	Calculer alors la valeur de $q$.
}{%
	D'après la question~\ref{q:kapp}, on a
	\smallbreak
	\noindent
	\begin{isd}[lefthand ratio=.35]
		\vspace{-15pt}
		\begin{align*}
			k\ind{app,1} = k [\ce{O_2}]_{01}^q
			\quad                                 & \stm{\text{et}} \quad
			k\ind{app,2} = k [\ce{O_2}]_{02}^q
			\\\Lra
			\frac{k\ind{app,1}}{k\ind{app,2}}     & =
			\frac{[\ce{O_2}]_{01}^q}{[\ce{O_2}]_{02}^q}
			\\\Lra
			\ln \frac{k\ind{app,1}}{k\ind{app,2}} & \stm{=}
			q \ln \frac{[\ce{O_2}]_{01}}{[\ce{O_2}]_{02}}
		\end{align*}
		\tcblower
		\vspace{-15pt}
		\begin{gather*}
			\Lra
			\boxed{
				q  \stm{=}
				\frac{%
					\ln (k\ind{app,1}) - \ln (k\ind{app,2})
				}{%
					\ln ([\ce{O_2}]_{01}) - \ln ([\ce{O_2}]_{02})
				}
			}
			\qav
			\left\{
			\begin{array}{rcl}
				\ln (k\ind{app,1}) & = & \SI{-3.7841}{\micro mol.L^{-1}.min^{-1}}
				\\
				\ln (k\ind{app,2}) & = & \SI{-4.723}{\micro mol.L^{-1}.min^{-1}}
				\\{}
				[\ce{O_2}]_{01}    & = & \SI{5.0e-3}{mol.L^{-1}}
				\\{}
				[\ce{O_2}]_{02}    & = & \SI{2.0e-3}{mol.L^{-1}}
			\end{array}
			\right.                          \\
			\AN
			% \makebox[0pt][l]{$\xul{\phantom{q = \num{1.02} \approx 1}}$}
			\xul{q \stm{=} \num{1.02} \approx 1}
		\end{gather*}
	\end{isd}
	\vspace{-15pt}
}%

\QR[2]{%
	Que peut-on déduire sur la nature de la réaction décrite par l'équation-bilan
	précédente~? Comment s'appelle la loi qui donne la loi de vitesse de ce type
	de réactions~?
}{%
	Les ordres partiels sont donc égaux aux coefficient stœchiométriques~: c'est
	une \textbf{réaction simple} \pt{1}, qui suit donc la \textbf{loi de
		\textsc{Van't Hoff}}. \pt{1}
}%

\QR[3]{%
	Calculer la valeur de la constante de vitesse. L'exprimer en utilisant des
	\si{mol}, des \si{L} et des \si{s}.
}{%
	Avec $q = 1$, on trouve $k$ à partir de la connaissance de $k\ind{app,1}$ ou
	$k\ind{app,2}$~:
	\begin{align*}
		\Aboxed{k & \stm{=} \frac{k\ind{app,1}}{[\ce{O_2}]_{01}}}
		\qav
		\left\{
		\begin{array}{rcl}
			\ln (k\ind{app,1}) & = & \SI{-3.784}{\micro mol.L^{-1}.min^{-1}}
			\\{}
			[\ce{O_2}]_{01}    & = & \SI{5.0e3}{\micro mol.L^{-1}}
		\end{array}
		\right.                                                         \\
		\AN
		\makebox[0pt][l]{$\xul{\phantom{k = \SI{4.5e-6}{\micro mol^{-2}.L^{2}.min^{-1}}}}$}
		k         & \stm{=} \SI{4.5e-6}{\micro mol^{-2}.L^{2}.min^{-1}}
		\Ra
		k \stm{=} \SI{7.5e10}{mol^{-2}.L^{2}.s^{-1}}
	\end{align*}
}%

\end{document}
