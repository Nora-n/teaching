\documentclass[../DS04.tex]{subfiles}
\graphicspath{{./figures/}}

% \subimport{/home/nora/Documents/Enseignement/Prepa/bpep/exercices/DS/chimie_NOx_thermo_cinetique/}{sujet.tex}

\begin{document}

\section{Monoxyde et dioxyde d'azote}

\enonce{%
	En consommant du kérosène, les moteurs d’avions entraînent le rejet de
	nombreux polluants parmi lesquels les monoxyde d’azote \ce{NO}
	et dioxyde d’azote \ce{NO2}, regroupés sous l’appellation NOx. Ces composés
	sont responsables d’une augmentation de la production d’ozone
	\ce{O3}, gaz à effet de serre, dans la basse atmosphère.
	\bigbreak
	Le monoxyde d’azote \ce{NO} est obtenu par oxydation du diazote \ce{N2} à
	haute température lors de la phase de combustion,
	modélisée par l’équation de réaction
	\begin{gather*}
		\ce{N_{2(g)} + O_{2(g)} = 2NO_{(g)}}
		\tag*{$K^\circ_1 (\SI{1450}{K}) = \num{6.0e-6}$}
	\end{gather*}
}

\QR{%
	La réaction se produit dans l'air. Rappeler la fraction molaire du diazote et
	du dioxygène dans l'air.
}{%
	$x(\ce{N2})=0,8$ et $x(\ce{O2})=0,2$
}

\QR{%
	Faire le tableau d'avancement de la réaction. On notera $n_0$ la quantité de matière initiale en \ce{O2}.
}{%
	\begin{center}
		\def\rhgt{0.50}
		\centering
		\begin{tabularx}{\linewidth}{|l|c||YdYdY||Y|}
			\hline
			\multicolumn{2}{|c||}{
				$\xmathstrut{\rhgt}$
			\textbf{Équation}} &
			$\ce{N_2}$         & $+$           &
			$\ce{O_2}$         & $\ra$         &
			$2\ce{NO}$         &
			$n_{\tot, gaz}$                      \\
			\hline
			$\xmathstrut{\rhgt}$
			Initial            & $\xi = 0$     &
			$4n_0$             & \vline        &
			$n_0$              & \vline        &
			$0$                &
			$5n_0$                               \\
			\hline
			$\xmathstrut{\rhgt}$
			Interm.            & $\xi$         &
			$4n_0 - \xi$       & \vline        &
			$n_0 - \xi$        & \vline        &
			$2\xi$             &
			$5n_0$                               \\
			\hline
			$\xmathstrut{\rhgt}$
			Final              & $\xi = \xi_f$ &
			$4n_0 - \xi_f$     & \vline        &
			$n_0 - \xi_f$      & \vline        &
			$0 + 2\xi_f$       &
			$5n_0$                               \\
			\hline
		\end{tabularx}
	\end{center}

	\begin{center}
		\begin{tabular}{*{6}{c}|c}
			         & \ce{N2}         & + & \ce{O2}        & = & \ce{2NO}    & $n_{tot}$ \\
			$t=0$    & $4n_0$          &   & $n_0$          &   & 0           & $5n_0$    \\
			$t_{eq}$ & $4n_0-\xi_{eq}$ &   & $n_0-\xi_{eq}$ &   & $2\xi_{eq}$ & $5n_0$    \\
			         & $\approx 4n_0$  &   & $\approx n_0$  &   &             &
		\end{tabular}
	\end{center}
}

\QR{%
	En faisant une hypothèse sur l'avancement à l'équilibre, évaluer la fraction molaire de monoxyde d’azote \ce{NO} présente à l’équilibre dans de l’air chauffé à \SI{1450}{\kelvin}.
}{%
	Comme $K^0_1\ll 1$, on peut supposer la réaction faiblement avancée, soit $\xi_{eq}\ll n_0$.

	D'après la loi d'action de masse,
	\eq[align*]{
		K^0_1&=\cfrac{P(\ce{NO})_{eq}^2}{P(\ce{N2})_{eq}P(\ce{O2})_{eq}}\\
		&=\cfrac{x(\ce{NO})_{eq}^2}{x(\ce{N2})_{eq}x(\ce{O2})_{eq}}\\
	}%

	On en déduit
	\eq{
		x(\ce{NO})=\sqrt{K^0_1x(\ce{N2})_{eq}x(\ce{O2})_{eq}}\approx \sqrt{K^0_1x(\ce{N2})_{0}x(\ce{O2})_{0}}=\cfrac{2}{5}\sqrt{K^0_1}
	}%

	AN : $\boxed{x(\ce{NO})=\num{9,8e-4}\ll 1}$

	On vérifie bien que $x(\ce{NO})\ll1$, donc que $\xi_{eq}\ll n_0$. L'hypothèse d'une réaction faiblement avancée est justifiée.

}

\enonce{
Dans un deuxième temps, il y a production de dioxyde d’azote \ce{NO2} à partir du monoxyde d’azote \ce{NO}, modélisée par l’équation de réaction

\eq{
\ce{2NO_{(g)}  +O_{2(g)} = 2NO_{2(g)}}\qquad K^0_2=\num{2.0e7}\qMath{à} \SI{400}{\kelvin}
}
}



\enonce{
	La figure~\ref{fig:cinetique} fournit pour cette réaction les résultats d'une étude cinétique réalisée à \SI{400}{\kelvin},
	où $v$ représente la dérivée temporelle de l'avancement volumique de la réaction. Les concentrations initiales utilisées dans cette étude sont

	\begin{itemize}
		\item  expérience 1 (tracé avec $\times$) : $[\ce{O_2}]_0 = \SI{5.0e-3}{\mole\cdot\liter^{-1}}$, $[\ce{NO}]_0 = \SI{10}{\micro\mole\cdot\liter^{-1}}$, $[\ce{NO2}]_0 = 0$ ;

		\item expérience 2 (tracé avec $+$) : $[\ce{O2}]_0 = \SI{2,0e-3}{\mole\cdot\liter^{-1}}$, $[\ce{NO}]_0 = \SI{10}{\micro\mole\cdot\liter^{-1}}$, $[\ce{NO2}]_0 = 0$.
	\end{itemize}

	\begin{figure}[h]
		\centering
		\begin{tikzpicture}
			\begin{axis}[
					scale only axis,
					width=100mm, height=50mm,
					xlabel={$\ln\bigl([{\rm NO}]\bigr)$ (\si{\micro\mole\cdot\liter^{-1}\cdot\minute^{-1}})},
					xmin=0, xmax=2.5, xtick distance=0.5,
					ylabel=$\ln(v)$ (\si{\micro\mole\cdot\liter^{-1}}),
					ymin=-4, ymax=1, ytick distance=1,
					grid=major
				]
				\addplot [mark=x, only marks] coordinates {
						(1.46, -0.9) (1.67, -0.45) (1.87, -0.09) (2, 0.2) (2.14, 0.45) (2.22, 0.66) (2.265, 0.69) (2.3, 0.8)
					};
				\addplot [mark=+, only marks] coordinates {
						(0.46, -3.82) (0.79, -3.15) (1.16, -2.4) (1.42, -1.82) (1.77, -1.26) (2, -0.7) (2.14, -0.45) (2.3, -0.1)
					};

				\addplot [domain=1.4:2.4] {1.9871*x - 3.7841};
				\node at (5.5cm,4.15cm) [below]{$R^2 = \num{0,99975}$};
				\node at (5.5cm,4.15cm)[above] {$y = \num{1,9871} x - \num{3,7841}$};
				\addplot [domain=0.4:2.4] {1.9979*x - 4.723};
				\node at (7.5cm, 1.5cm) [below] {$R^2 = \num{0,99915}$};
				\node at (7.5cm, 1.5cm)  [above] {$y = \num{1,9979} x - \num{4,723}$};
			\end{axis}
		\end{tikzpicture}
		\caption{}
		\label{fig:cinetique}
	\end{figure}
}

\QR{%
	La réaction peut-elle être considérée comme quasi-totale à cette température ?
}{%
	Oui car $K^0_2\gg 1$.
}


\QR{%
Utiliser les résultats précédents pour proposer une loi de vitesse pour la réaction.
Déterminer la valeur numérique de la constante de vitesse à la température considérée.
La forme obtenue pour la loi de vitesse était-elle prévisible ?
}{%

Chacune des études cinétiques est réalisée avec une concentration initiale de \ce{O2} largement supérieure à la concentration initiale de \ce{NO},
on peut faire l'hypothèse d'une concentration en dioxygène $\ce{O2}$ constante durant les expériences, ce qui permet d'étudier la cinétique par rapport au monoxyde d'azote $\ce{NO}$.

C'est la méthode de dégénérescence de l'ordre.

Le logarithme de la vitesse en fonction de la concentration $[\ce{NO}]$ est une droite de pente voisine de 2.
On peut proposer un ordre partiel 2 par rapport à $\ce{NO}$ : $v = k([\ce{O2}])\times [\ce{NO}]^2$.

Si la réaction présente un ordre $\alpha$ par rapport à $[\ce{O2}] = [\ce{O2}]_0$ constant au cours de l'expérience, on peut écrire $v = k[\ce{O2}]^\alpha [\ce{NO}]^2$,
soit pour une même concentration $[\ce{N2}]$ et deux concentrations différentes en dioxygène :
\eq{
	\ln\left(\frac{v_2}{v_1}\right) = \alpha\ln\left(\frac{[\ce{O2}]_{02}}{[\ce{O2}]_{01}}\right)
}

Les deux droites expérimentales sont parallèles, ce qui confirme l'indépendance du rapport $v_2/v_1$ vis à vis de la concentration $[\ce{O2}]$,
et en considérant les ordonnées à l'origine données par les régressions linéaires
\eq{
	\alpha = \frac{\ln\left(\frac{v_2}{v_1}\right)}{\ln\left(\frac{[\ce{O2}]_{02}}{[\ce{O2}]_{01}}\right)}= \cfrac{-4,723+3,7841}{\ln(2/5)}= 1,02\approx 1
}

Les résultats sont compatibles avec l'hypothèse d'une cinétique d'ordre partiel 1 par rapport au dioxygène, et on peut écrire
\eq{\boxed{v = k[\ce{O2}]\cdot[\ce{NO}]^2}}


Chacune des deux expériences permet de déterminer une constante de vitesse apparente $k\ind{app}$ pour \ce{NO} telle que
\eq{
	\ln(v) = \ln(k[\ce{O2}]) + 2 \ln[\ce{NO}] = \ln(k\ind{app}) + 2 \ln[\ce{NO}]~;~\ln(k\ind{app}) = \ln k + \ln[\ce{O2}]
}

Les valeurs expérimentales de $\ln(k\ind{app})$ sont les ordonnées à l'origine des droites de régression linéaire. On déduit des deux expériences les valeurs
\eq{\boxed{k = \frac{k\ind{app}}{[\ce{O2}]} =\cfrac{\exp(-3,7841)}{\num{5.0e3}}= \SI{4.5e-6}{\mu mol^{-2}.L^2.min^{-1}}}}

Attention à mettre $[\ce{O2}]$ en \si{\micro\mole\cdot\liter^{-1}}.
}

\end{document}
