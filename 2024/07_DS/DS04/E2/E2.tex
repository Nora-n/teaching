\documentclass[../DS04.tex]{subfiles}
\graphicspath{{./figures/}}

% \subimport{/home/nora/Documents/Enseignement/Prepa/bpep/exercices/TD/resonance_circuit_RLC_parallele_2/}{sujet.tex}

\begin{document}

\section[30]"E"{Étude d'un circuit RLC parallèle}

\enonce{%
	\noindent
	\begin{minipage}{0.5\linewidth}
		Le circuit ci-contre est constitué d'une source idéale de courant de c.e.m.
		$\eta(t) = \eta_0\cos(\w t)$. Cette source alimente une association parallèle
		constituée d'un condensateur, d'une bobine et d'une résistance. La tension
		aux bornes de cette association est $u(t)=U_0\cos(\w t+\phi)$. On note
		$\xul{U_0}=U_0e^{\jj\phi}$ l'amplitude complexe de $u(t)$.
	\end{minipage}
	\begin{minipage}{0.49\linewidth}
		\figsvg{fig1.pdf_tex}
	\end{minipage}
}

\subsection{Étude de l'amplitude et de la phase}

\QR{
	Exprimer l'impédance équivalente $\Zu$ du dipôle $AB$.
}{
	Dans le cas d'une association de dipôle en parallèle, on additionne les admittances :
	\eq{
		\Yu=\frac{1}{\Zu}=\frac{1}{R}+\frac{1}{\jlw}+\jcw=\frac{R(1-LC\w^2)+\jlw}{\jrlw}
	}
	\eq{
		\boxed{\Zu=\frac{\jrlw}{R(1-LC\w^2)+\jlw}}
	}
}


\QR{
	Montrer que l'amplitude complexe de la tension $u$ se met sous la forme :
	\eq{
		\xul{U_0}=\frac{R\eta_0}{1+jQ\left (x-\cfrac{1}{x}  \right )}\quad \mbox{avec }x=\w/\w_0
	}
	Exprimer $Q$ et $\w_0$ en fonction de $R$, $L$ et $C$. Comment s'appellent ces deux constantes~?
}{
	On applique la loi d'Ohm généralisée sur le dipôle équivalent $\Zu$, en utilisant les amplitudes complexes du courant et de la tension :
	\eq{
		\xul{U_0}=\Zu\eta_0=\frac{\jrlw}{R(1-LC\w^2)+\jlw}\eta_0
	}

	On divise par $\jlw$ et on trouve : $\boxed{\displaystyle{\xul{U_0}=\frac{R\eta_0}{1+jR\left (C\w-\frac{1}{L\w}  \right )}}}$

	Par identification on a $\displaystyle{RC=\frac{Q}{\w_0}\qquad \frac{R}{L}=Q\w_0}$.

	On en déduit $\boxed{\displaystyle{\w_0=\frac{1}{\sqrt{LC}}\qquad Q=R\sqrt{\frac{C}{L}}}}$.

	$\w_0$ est la pulsation propre du circuit et $Q$ le facteur de qualité.

}

\QR{
	Exprimer l'amplitude réelle $U_0$ de la tension $u$ en fonction de $R$, $\eta_0$, $Q$ et $x$.
}{
	$\boxed{\displaystyle{U_0=|\xul{U_0}|=\frac{R\eta_0}{\sqrt{1+Q^2\left (x-\frac{1}{x} \right )^2}}}}$
}

\QR{
	Y-a-t-il résonance en tension~? Si oui, préciser la valeur de $x$ à la résonance. En déduire la valeur de $\w$ à la résonance.
}{
	La résonance correspond à un maximum de la fonction $U_0$ à $x\neq 0$. $U_0$ est maximale si son dénominateur est minimal, soit quand $x-1/x=0$.

	Il y a toujours résonance en tension pour $x=1$, soit $\w=\w_0$.
}

\QR{
Comment définit-on la bande passante $\Delta\w$~? Montrer que $\Delta\w=\w_0/Q$.
}{
La bande passante $\Delta \w$ est l'ensemble des pulsations $\w$ vérifiant ${U_{max}/\sqrt{2}\leq U_0(\w)\leq U_{max}}$.

Soit $\Delta \w=[\w_1;\w_2]$, avec $\w_1$ et $\w_2$ solutions de l'équation $U_0(\w)=U_{max}/\sqrt{2}$, soit :
\eq{
	Q^2\left (x-\frac{1}{x}  \right )^2=1\quad\Leftrightarrow \quad x^2\pm\frac{x}{Q}-1=0
}
Parmi les 4 solutions de ces 2 équations polynomiales de degré 2, seules 2 solutions sont acceptables car donnant $x>0$ :
\eq{
	x_1=-\frac{1}{2Q}+\frac{1}{2Q}\sqrt{1+4Q^2}\qquad x_2=\frac{1}{2Q}+\frac{1}{2Q}\sqrt{1+4Q^2}
}
On obtient $\Delta x=x_2-x_1=1/Q$, soit $\Delta \w=\w_0\Delta x=\w_0/Q$
}

\QR{
	Faire l'étude asymptotique de la fonction $U_0(x)$. Tracer l'allure de $U_0$ en fonction de $x$.
}{
	\begin{itemize}
		\item si $x\rightarrow 0$ : $\displaystyle{\lim_{x\rightarrow 0}U_0(x)=\lim_{x\rightarrow 0}\frac{R\eta_0}{Q}x=0}$
		\item si $x\rightarrow +\infty$ : $\displaystyle{\lim_{x\rightarrow +\infty}U_0(x)=\lim_{x\rightarrow 0}\frac{R\eta_0}{Q}\frac{1}{x}=0}$
		\item si $x=1$, $U_0(1)=R\eta_0$
	\end{itemize}

	\figsvg{u0.pdf_tex}

}

\QR{
Exprimer la phase $\phi$ en fonction de $Q$ et $x$. Préciser le domaine de variation de $\phi$.
}{
$\displaystyle{\phi=-\arctan\left (Q(x-1/x))  \right )\in]-\pi/2;\pi/2[}$
}

\QR{
	Faire l'étude asymptotique de la fonction $\phi (x)$. Tracer l'allure de $\phi$ en fonction de $x$.
}{
	\begin{itemize}
		\item si $x\rightarrow 0$ : $\lim_{x\rightarrow 0}\phi(x)=\pi/2$
		\item si $x\rightarrow +\infty$ : $\lim_{x\rightarrow +\infty}\phi(x)=-\pi/2$
		\item si $x=1$ : $\phi(1)=0$
	\end{itemize}

	\figsvg{phase.pdf_tex}
}

\subsection{Expérience}

\enonce{
	\noindent
	\begin{minipage}{0.5\linewidth}
		Pour tracer les graphiques $U_0$ et $\phi$ en fonction de $\w$, il faut pouvoir observer simultanément le courant $\eta(t)$ et la tension $u(t)$. On ajoute une résistance  $r$ en série avec le générateur de courant afin de visualiser le courant $\eta(t)$ par l'intermédiaire de la tension $u_r(t)$. On propose le montage ci-contre.
	\end{minipage}
	\begin{minipage}{0.49\linewidth}
		\figsvg{rlcparmes.pdf_tex}
	\end{minipage}
}

\QR{
	Le montage proposé est-il valable ? Si oui, à quelle condition~?
}{
	Le montage est valable si le générateur de courant n'impose pas de masse, c'est-à-dire avec un générateur à masse flottante.
}

\QR{
	Quelle tension visualise-t-on sur la voie $A$~? sur la voie $B$~? Que faut-il faire pour visualiser $\eta(t)$ et $u(t)$~?
}{
	Sur la voie $A$, on visualise la tension $u_r(t)=r\eta(t)$. Donc il faut diviser par $r$ la voie $A$ pour visualiser $\eta(t)$.

	Sur la voie $B$, on visualise $-u(t)$. Donc il faut inverser la voie $B$ pour visualiser $u(t)$.

}


\enonce{
	La figure suivante montre une acquisition des tensions $u_r$ et $u$ faite pour
	une pulsation $\w$ donnée. Le calibre est de $\SI{1}{\volt}$ sur les deux
	voies.

	\figsvg{oscillo.pdf_tex}

}


\QR{
	La tension $u$ est-elle en avance ou en retard par rapport au courant $\eta$~?
}{
	La tension $u$ est en retard par rapport au courant $\eta$ car son maximum arrive après celui de la tension $u_r$.
}

\QR{
	Déterminer la valeur de la phase $\phi$ de la tension $u$ par rapport au courant $\eta$. On donnera sa valeur en degré.
}{
	Une période du signal $u_r$ (ou $u$) correspond à 10 carreaux. Donc un retard de 1 carreau correspond à un déphasage de $\SI{-36}{\degree}$.

	Ici, on a 2 carreaux de déphasage entre $u$ et $u_r$, donc $\boxed{\phi=-\SI{72}{\degree}}$
}

\QR{
	Que vaut l'amplitude $U_0$ de la tension $u$~?
}{
	L'amplitude de $u(t)$ correspond à 2 carreaux, donc $\boxed{U_0=\SI{2}{\volt}}$.
}

\QR{
	Définir mathématiquement la valeur efficace $s\ind{eff}$ d'un signal $s(t)$ périodique de période $T$.
}{
	$\displaystyle{s\ind{eff}=\sqrt{\frac{1}{T}\int_0^Ts^2(t)dt}}$
}

\QR{
	Soit un signal $s(t)$ sinusoïdal de période $T$, d'amplitude $S_0$ et de phase à l'origine nulle. Exprimer sa valeur efficace $s\ind{eff}$ en fonction de $S_0$. On étblira cette relation.
}{

	On écrit la fonction $s(t)$ : $s(t)=S_0\cos(2\pi t/T)$
	\eq{
		s^2\ind{eff}=\frac{1}{T}\int_0^T S_0^2\cos^2(2\pi t/T)dt
	}
	On linéarise le cosinus carré : $\displaystyle{\cos^2(2\pi t/T)=\frac{1+\cos(4\pi t/T)}{2}}$

	Puis on intègre, et on obtient :
	\eq{
		s^2\ind{eff}=\frac{S_0^2}{2T}\left [\int_0^T dt+\int_0^T \cos(4\pi t/T)dt \right ]=\frac{S_0^2}{2}
	}
	Car $\int_0^T \cos(4\pi t/T)dt=0$. On en déduit $\boxed{s\ind{eff}=S_0/\sqrt{2}}$

}


\QR{
	En déduire la valeur efficace de la tension $u(t)$. On donne $\sqrt{2}=1,4$.
}{
	$u\ind{eff}=2/\sqrt{2}=\sqrt{2}=\SI{1,4}{\volt}$
}

\end{document}
