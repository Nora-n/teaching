\documentclass[../DS04.tex]{subfiles}
\graphicspath{{./figures/}}

% \subimport{/home/nora/Documents/Enseignement/Prepa/bpep/exercices/TD/resonance_circuit_RLC_parallele_2/}{sujet.tex}

\begin{document}

\section[65]"E"{Étude d'un circuit RLC parallèle}

\enonce{%
	\noindent
	\begin{minipage}{0.5\linewidth}
		Le circuit ci-contre est constitué d'une source idéale de courant de c.e.m.
		$\eta(t) = \eta_0\cos(\w t)$. Cette source alimente une association parallèle
		constituée d'un condensateur, d'une bobine et d'une résistance. La tension
		aux bornes de cette association est $u(t)=U_0\cos(\w t+\phi)$. On note
		$\xul{U_0}=U_0e^{\jj\phi}$ l'amplitude complexe de $u(t)$.
	\end{minipage}
	\begin{minipage}{0.49\linewidth}
		\figsvg{fig1.pdf_tex}
	\end{minipage}
}

\subsection{Étude de l'amplitude et de la phase}

\QR[3]{%
	Exprimer l'impédance équivalente $\Zu$ du dipôle $AB$.
}{%
	Dans le cas d'une association de dipôle en parallèle, on additionne les
	admittances~:
	\begin{gather*}
		\Yu \stm{=}
		\frac{1}{\Zu} \stm{=}
		\frac{1}{R}+\frac{1}{\jlw}+\jcw
		\Lra
		\boxed{\Zu \stm{=} \frac{1}{\frac{1}{R}+\frac{1}{\jlw}+\jcw}}
		\qou
		\Zu = \frac{R}{1+\jj RC\w - \jj \frac{R}{L\w}}
	\end{gather*}
}

\QR[5]{%
	Montrer que l'amplitude complexe de la tension $u$ se met sous la forme~:
	\[
		\xul{U_0} =
		\frac{R\eta_0}{1+\jj Q \pa{x-\cfrac{1}{x}}}
		\qav
		x = \frac{\w}{\w_0}
	\]
	Exprimer $Q$ et $\w_0$ en fonction de $R$, $L$ et $C$. Comment s'appellent ces
	deux constantes~?
}{%
	On applique la loi d'\textsc{Ohm} généralisée sur le dipôle équivalent $\Zu$,
	en utilisant les amplitudes complexes du courant et de la tension~:
	\smallbreak
	\noindent
	\begin{isd}
		\vspace{-15pt}
		\begin{DispWithArrows*}[fleqn, mathindent=0pt]
			\xul{U_0} \stm{=} \Zu\eta_0
			\\\Lra
			\xul{U_0} \stm{=} \frac{R\eta_0}{1+\jj \pa{RC\w-\frac{R}{L\w}}}
		\end{DispWithArrows*}
		\tcblower
		\vspace{-15pt}
		\begin{DispWithArrows*}
			\\\text{Identification~:~}
			RC = \frac{Q}{\w_0}
			\quad &\stm{\text{et}} \quad
			\frac{R}{L} = Q \w_0
			\\\Lra
			\Aboxed{
				\w_0 = \frac{1}{\sqrt{LC}}
				\quad & \stm{\text{et}} \quad
				Q = R \sqrt{\frac{C}{L}}
			}
		\end{DispWithArrows*}
	\end{isd}
	avec $\w_0$ la pulsation propre du circuit et $Q$ le facteur de qualité. \pt{1}
}

\QR[2]{%
	Exprimer l'amplitude réelle $U_0$ de la tension $u$ en fonction de $R$,
	$\eta_0$, $Q$ et $x$.
}{%
	\sswitch{%
		\leavevmode\vspace*{-25pt}\relax
	}{%
		\leavevmode\vspace*{-15pt}\relax
	}%
	\[
		\boxed{%
			U_0 \stm{=}
			\abs{\xul{U_0}} \stm{=}
			\frac{R\eta_0}{\sqrt{1+Q^2\left (x-\frac{1}{x} \right )^2}}}
	\]
}

\QR[8]{%
	Définir ce qu'est la résonance. Peut-il toujours y avoir résonance en tension
	ici~? Si oui, préciser la valeur de $x$ puis de $\w$ à la résonance. Quelle
	est la valeur maximal $U\ind{max}$ de $U_0$~?
}{%
	La résonance correspond à un maximum de la fonction $U_0$ \pt{1} à $x\neq 0$
	\pt{1}. Ici, comme $R\eta_0 = \cte$ \pt{1}, on a $U_0$ maximale si son
	dénominateur est minimal \pt{1}, soit pour
	\begin{gather*}
		1 + \underbracket[1pt]{Q^2 \pa{x_r - \frac{1}{x_r}}^2}_{\geq 0} \quad
		\text{minimal}
		\Lra
		Q^2 \pa{x_r - \frac{1}{x_r}}^2 \stm{=} 0
		\Lra
		\boxed{x_r = 1}
		\stm{\Lra}
		\boxed{\w_r = \w_0}
	\end{gather*}
	Ainsi, il peut toujours y avoir résonance en tension ici \pt{1}, et on obtient
	$U\ind{max} = R \eta_0$. \pt{1}
}

\QR[8]{%
Comment définit-on la bande passante $\Delta\w$~? Montrer que $\Delta\w =
	\w_0/Q$.
}{%
La bande passante $\Delta \w$ est l'ensemble des pulsations $\w$ vérifiant
${U\ind{max}/\sqrt{2}\leq U_0(\w)\leq U\ind{max}}$, \pt{1} soit
$\Delta \w=[\w_1\,;\w_2]$, avec $\w_1$ et $\w_2$ solutions de l'équation
$U_0(\w_k) = U\ind{max}/\sqrt{2}$. En travaillant en pulsations réduites~:
\smallbreak
\noindent
\begin{isd}
	\begin{DispWithArrows*}[groups, fleqn]%
		U_0(x_k)
		&=
		\frac{U\ind{max}}{\sqrt{2}}
		\Arrow{On remplace}
		\\\Lra
		\frac{R\eta_0}{\sqrt{1 + Q^2 \pa{x_k - \frac{1}{x_k}} ^2}}
		&=
		\frac{R\eta_0}{\sqrt{2}}
		\Arrow{On isole}
		\\\Lra
		\Aboxed{
			Q^2 \pa{x_k - \frac{1}{x_r}}^2
			&\stm{=}
			1
		}
		\CArrow{$\sqrt{\cdot}$}
		\\\Lra
		Q \pa{x_k - \frac{1}{x_k}}
		&=
		\pm 1
		\Arrow{$\times x_k$}
		\\\Lra
		Qx_k{}^2 - Q
		&=
		\pm x_k
		\Arrow{$-\pm = \mp$}
		\\\Lra
		Q x_k{}^{2} \mp x_k - Q
		&\stm{=}
		0
	\end{DispWithArrows*}
	On a alors \textbf{deux trinômes}, soit \textbf{quatre racines possibles},
	avec
	\[
		\Delta
		\stm{=}
		1 + 4Q{}^{2}
	\]
	\tcblower
	\begin{DispWithArrows*}[groups]%
		\Ra
		x_{k,\pm,\pm}
		&\stm{=}
		\frac{\pm 1 \pm \sqrt{1+4Q{}^{2}}}{2Q}
	\end{DispWithArrows*}
	On ne garde que les racines positives, sachant que
	$\boxed{\sqrt{1+4Q{}^{2}} > 1}$~:
	\begin{DispWithArrows*}[]
		x_1 = x_{k,-,+} &= \frac{1}{2Q} \pa{-1 + \sqrt{1+4Q{}^{2}}}
		\\\stm{\text{et}} \quad
		x_2 = x_{k,+,+} &= \frac{1}{2Q} \pa{1 + \sqrt{1+4Q{}^{2}}}
	\end{DispWithArrows*}
	puis on obtient la bande passante en calculant la différence $\abs{x_2 -
			x_1}$~:
	\begin{align*}
		\Delta{x}         & = x_2 - x_1 =
		\frac{1+\cancel{\sqrt{1+4Q^2}} - \pa{-1 + \cancel{\sqrt{1+4Q^2}}}}{2Q}
		\\\Lra
		\Aboxed{\Delta{x} & \stm{=} \frac{1}{Q} \Lra \Delta{\w} \stm{=} \frac{\w_0}{Q}}
		\qed
	\end{align*}
\end{isd}
\vspace{-15pt}
}

\QR[7]{%
	Faire l'étude asymptotique de la fonction $U_0(x)$. Indiquer alors les limites
	de $U_0(x)$ en basses et hautes fréquences. Indiquer la valeur de
	$U_0(x = 1)$. Tracer l'allure de $U_0$ en fonction de $x$.
}{%
	~
	\vspace{-15pt}
	\smallbreak
	\noindent
	\begin{isd}[interior hidden, righthand ratio=.25]
		\begin{align*}
			U_0(x) & \stm{\Sim_{x \to 0}} \frac{R\eta_0}{Q}x \stm{\to} 0
			\\
			U_0(x) & \stm{\Sim_{x \to \infty}} \frac{R\eta_0}{Q} \frac{1}{x} \stm{\to} 0
			\\
			U_0(1) & \stm{=} R\eta_0
		\end{align*}
		\tcblower
		\def\svgwidth{.9\linewidth}
		\figsvg{u0.pdf_tex}
		\vspace{-15pt}
		\captionof{figure}{\protect\pt{1}+\protect\pt{1}}
	\end{isd}
	\vspace{-15pt}
}

\QR[6]{%
	Exprimer la phase $\phi$ en fonction de $Q$ et $x$. Préciser le domaine de
	variation de $\phi$.
}{%
	\leavevmode\vspace*{-15pt}\relax
	\begin{gather*}
		\phi \stm{=}
		\arg*{\xul{U_0}} \stm{=}
		\underbracket[1pt]{\cancel{\arg*{R\eta_0}}}_{=0}
		- \arg*{1 + \jj Q \left( x - \frac{1}{x} \right)}
		\\\Ra
		\tan(\phi) \stm{=}
		- \tan\/(%
		\arg{%
		\underbracket[1pt]{1}_{\mathclap{\Re > 0 \pt{1}}} +
		\jj Q\left( x - \frac{1}{x} \right)
		}
		)
		\Lra
		\boxed{\phi \stm{=} -\arctan(Q \pa{x - \frac{1}{x}})}
		\qavec
		\boxed{\phi \stm{\in} \left] - \frac{\pi}{2}\,; \frac{\pi}{2} \right[}
	\end{gather*}
}

\QR[7]{%
	Faire l'étude asymptotique de la fonction $\phi (x)$. Indiquer alors ses
	limites en hautes et basses fréquences. Indiquer l'allure de $\phi(x = 1)$.
	Tracer l'allure de $\phi$ en fonction de $x$.
}{%
	~
	\vspace{-15pt}
	\smallbreak
	\noindent
	\begin{isd}[interior hidden, righthand ratio=.25]
		\begin{align*}
			\phi(x) &
			\stm{\Sim_{x \to 0}}
			- \arctan(-\frac{Q}{x})
			\stm{\to}
			\frac{\pi}{2}
			\\
			\phi(x) &
			\stm{\Sim_{x \to \infty}}
			- \arctan(Qx)
			\stm{\to}
			- \frac{\pi}{2}
			\\
			\phi(1) & \stm{=} 0
		\end{align*}
		\tcblower
		\def\svgwidth{.9\linewidth}
		\figsvg{phase.pdf_tex}
		\vspace{-15pt}
		\captionof{figure}{\protect\pt{1}+\protect\pt{1}}
	\end{isd}
	\vspace{-15pt}
}

\subsection{Expérience}

\enonce{
	\noindent
	\begin{minipage}{0.5\linewidth}
		Pour tracer les graphiques $U_0$ et $\phi$ en fonction de $\w$, il faut
		pouvoir observer simultanément le courant $\eta(t)$ et la tension $u(t)$. On
		ajoute une résistance $r$ en série avec le générateur de courant afin de
		visualiser le courant $\eta(t)$ par l'intermédiaire de la tension $u_r(t)$.
		On propose le montage ci-contre.
	\end{minipage}
	\begin{minipage}{0.49\linewidth}
		\figsvg{rlcparmes.pdf_tex}
	\end{minipage}
}

\QR[5]{%
	À quelle condition le montage proposé est-il valable~? On suppose cette
	condition vérifiée dans la suite. Quelle tension visualise-t-on sur la voie
	$A$~? sur la voie $B$~? Que faut-il faire pour visualiser $\eta(t)$ et
	$u(t)$~?
}{%
	Le montage est valable si le générateur de courant n'impose pas de masse,
	c'est-à-dire avec un générateur à masse flottante. \pt{1}
	\begin{itemize}
		\item Sur la voie $A$, on visualise la tension $u_r(t)=r\eta(t)$. \pt{1}
		      Donc il faut diviser par $r$ la voie $A$ pour visualiser $\eta(t)$. \pt{1}
		\item Sur la voie $B$, on visualise $-u(t)$. \pt{1} Donc il faut inverser la
		      voie $B$ pour visualiser $u(t)$. \pt{1}
	\end{itemize}
}

\enonce{%
	La figure suivante montre une acquisition des tensions $u_r$ et $u$ faite pour
	une pulsation $\w$ donnée. Le calibre vertical est de $\SI{1}{\volt}$ sur les
	deux voies. \textbf{On ne connaît pas l'échelle de temps}.
	\figsvg{oscillo.pdf_tex}
}

\QR[4]{%
	La tension $u$ est-elle en avance ou en retard par rapport au courant $\eta$~?
	Justifier. En degrés, quel est la valeur de déphasage correspondant à un
	décalage d'une période~? Déterminer alors, toujours en degrés, la valeur du
	déphasage $\Delta{\f}_{u/i} = \phi$ de la tension $u$ par rapport au courant
	$\eta$.
}{%
	La tension $u$ est en retard \pt{1} par rapport au courant $\eta$ car son
	maximum arrive après celui de la tension $u_r$. \pt{1}
	\smallbreak
	Une période du signal $u_r$ (ou $u$) correspond à un déphasage de
	$\ang{360;;}$. \pt{1} Or ici, une période correspond à 10 carreaux, donc un
	retard de 1 carreau correspond à un déphasage de $\ang{-36;;}$, et on a 2
	carreaux de déphasage entre $u$ et $u_r$~; ainsi, $\xul{\phi = -\ang{72;;}}$.
	\pt{1}
}

\QR[5]{%
	Que vaut l'amplitude $U_0$ de la tension $u$~?
	Définir mathématiquement la valeur efficace $s\ind{eff}$ d'un signal $s(t)$
	périodique de période $T$. Que représentent physiquement $s\ind{eff}^2$ et $s
		\ind{eff}$~?
}{%
	L'amplitude de $u(t)$ correspond à 2 carreaux, donc $\xul{U_0 =
			\SI{2}{\volt}}$. \pt{1}
	\begin{gather*}
		\beforetext{Mathématiquement,}
		s \ind{eff} \stm{=}
		\sqrt{\moy{s^{2}(t)}} \stm{=}
		\sqrt{\frac{1}{T} \int_{0}^{T} s^2 (t) \dd{t}}
	\end{gather*}
	$s \ind{eff}^2$ représente l'énergie moyenne du signal \pt{1}~; ainsi, $s
		\ind{eff}$ correspond à l'amplitude constante qui porterait la même énergie
	moyenne que $s(t)$. \pt{1}
}

\QR[5]{%
	Soit un signal $s(t)$ sinusoïdal de période $T$, d'amplitude $S_0$ et de phase
	à l'origine nulle. Établir l'expression de sa valeur efficace $s\ind{eff}$ en
	fonction de $S_0$. En déduire la valeur efficace de la tension $u(t)$. On
	donne $\sqrt{2} = \num{1.4}$.
}{%
	~
	\vspace{-15pt}
	\smallbreak
	\begin{isd}[interior hidden]
		\begin{DispWithArrows*}[groups, fleqn, mathindent=0pt]
			s(t) &\stm{=} S_0\cos(\wt)
			\\\Ra
			\moy{s^{2}(t)} & =
			\frac{1}{T} \int_{0}^{T} \left( S_0^{2}\cos^{2}(\wt) \right) \dd{t}
			% \qor
			% \cos^2(\th) \stm{=} \frac{1}{2}\pa{\cos(2\th) + 1}
			% \Arrow{$\cos^2(\th) \stm{=} \frac{1}{2}\pa{\cos(2\th) + 1}$}
			\Arrow{Linéarisa$^\circ$}
			\\\Lra
			\moy{s^{2}(t)} & \stm{=}
			\frac{S_0^{2}}{2T}
			\left( \int_{0}^{T} \dd{t} + \int_{0}^{T} \cos(2\wt) \dd{t}\right)
		\end{DispWithArrows*}
		\tcblower
		\begin{DispWithArrows*}[groups]
			\Lra
			\moy{s^{2}(t)} & \stm{=}
			\frac{S_0^{2}}{2T}
			\Bigg(
			\underbracket[1pt]{\left[ t \right]_0^{T}}_{=T} +
			\underbracket[1pt]{\left[ \frac{S_0}{2\w} \sin(2\wt) \right]_0^{T}}_{=0}
			\Bigg)
			\\\Lra
			\Aboxed{s \ind{eff} & \stm{=} \frac{S_0}{\sqrt{2}}}
			\qMath{d'où}
			\boxed{u\ind{eff} = \frac{U_0}{\sqrt{2}}}
			\Ra
			\xul{u\ind{eff} = \SI{1.4}{V}} \pt{1}
		\end{DispWithArrows*}
	\end{isd}
	\vspace{-15pt}
}

\end{document}
