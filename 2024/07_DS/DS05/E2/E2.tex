\documentclass[../DS05.tex]{subfiles}
\graphicspath{{./figures/}}

\begin{document}

\exercice[33]{Chute d'une bille}
% \subimport{/home/nora/Documents/Enseignement/Prepa/bpep/exercices/TD/chute_de_bille/}{sujet.tex}
\enonce{
	\noindent On dispose du matériel suivant~:
	\begin{itemize}
		\item une bille de masse volumique $\rho_a=\SI{7900}{kg.m^{-3}}$, de rayon
		      $R=\SI{5}{mm}$~;
		\item une éprouvette graduée~;
		\item de la glycérine de masse volumique $\rho_g=\SI{1260}{kg.m^{-3}}$~;
		\item un dynamomètre, avec un point d'accroche permettant de mesurer une
		      force de traction~;
		\item trois béchers~;
		\item une boîte de masses variées.
	\end{itemize}
}

\QR[2]{%
	Donner l'expression générale de la poussée d'Archimède. Que devient son
	expression pour la bille dans la glycérine en fonction des données de
	l'énoncé et notamment de $R$~?
}{
	L'expression de la poussée d'Archimède est~:
	\[
		\boxed{\vv{\Pi} \stm[-1]{=} -\rho_{\rm fluide} V_{\rm immerge} \vv{g}}
		\Ra
		\boxed{\vv{\Pi} \stm[-1]{=} -\frac{4\rho_g \pi R^3}{3}\vv{g}}
	\]
}

\QR[4]{%
	Proposer un protocole expérimental permettant de vérifier l'expression de la
	poussée d'Archimède en utilisant le matériel listé.
}{
	On propose~:
	\begin{itemize}
		\litem{20pt}{\pt{1}}
		Remplir l'éprouvette graduée de glycérine à un volume connu~;
		\litem{20pt}{\pt{1}}
		Mesurer le poids des masses dans l'air à l'aide du dynamomètre~;
		\litem{20pt}{\pt{1}}
		Immerger les masses dans l'éprouvette, et relever le volume déplacé ainsi
		que la force indiquée par le dynamomètre~;
		\litem{20pt}{\pt{1}}
		Vérifier avec les données que la différence des forces est égale à
		$\norm{\vv{\Pi}}$.
	\end{itemize}
	\smallbreak
}

\enonce{
	La bille en acier tombe dans un tube rempli de glycérine. On considère que la
	force de frottement fluide exercée par la glycérine est $\vv{f}=-6\pi\eta R
		\vv{v}$ où $\eta$ est une constante appelée constante de viscosité dynamique
	de la glycérine. L'accélération de la pesanteur vaut $g = \SI{9.8}{m.s^{-2}}$.
}

\QR[7]{%
	Établir le système d'étude et faire un bilan des forces exercées sur la bille.
	On prendra un axe vertical descendant.
}{
	\noindent
	\begin{minipage}[t]{.70\linewidth}
		On établit le système d'étude~:
		\begin{itemize}
			\bitem{\ltm{20pt}{\pt{1}}Système}~:
			\{bille\} dans $\Rc\ind{labo}$ supposé galiléen
			\bitem{\ltm{20pt}{\pt{2}}Schéma}~: cf. Figure~\ref{fig:eprouv}.
			\bitem{\ltm{20pt}{\pt{1}}Modélisation}~:
			repère $(\Or, \uz)$,
			repérage~: $\OM = z\uz$, $\vf = \zp\uz$, $\af = \zpp\uz$.
			\bitem{Bilan des forces~:}
			\[
				\begin{array}{ll}
					\pt{1}\textbf{Poids}               &
					\Pf = mg\uz                                \\
					\pt{1}\textbf{Poussée d'Archimède} &
					\vv{\Pi} = -\frac{4\rho_g \pi R^3}{3}g \uz \\
					\pt{1}\textbf{Frottement fluide}   &
					\vv{f} = -6\pi\eta R v \uz
				\end{array}
			\]
		\end{itemize}
	\end{minipage}
	\hfill
	\begin{minipage}[t]{.25\linewidth}
		\vspace{0pt}
		\begin{center}
			\includegraphics[width=.7\linewidth]{schema_eprouv}
			\captionof{figure}{Schéma}
			\label{fig:eprouv}
		\end{center}
	\end{minipage}
}

\QR[3]{%
	Montrer que la somme de la poussée d'Archimède et du poids sur la bille est
	équivalent à considérer un nouveau poids $\Pf'$ sur une bille de masse
	volumique $\rho = \rho_a - \rho_g$ qui n'est pas soumise à la poussée
	d'Archimède. Attention la masse de la bille ne change pas.
}{
	On a~:
	\begin{gather*}
		\vv{P} + \vv{\Pi} \stm{=}
		\left( m - \frac{4\rho_g \pi R^3}{3} \right)\vv{g} \stm{=}
		m'\vv{g} = \vv{P'}
		\\\Lra
		m' = m - \frac{4\rho_g \pi R^3}{3}
		\Lra
		\boxed{m' \stm[-1]{=} \frac{4}{3}\pi R^{3}\left( \rho_a - \rho_g \right)}
	\end{gather*}
	d'où le résultat.
}

\QR[6]{%
	Établir l'équation différentielle vérifiée par $v$, la norme de la vitesse, en
	utilisant le résultat de la question précédente. La mettre sous forme
	canonique et en déduire la constante de temps $\tau$ caractéristique du régime
	transitoire, ainsi que la vitesse limite $v_l$ atteinte par la bille, en
	fonction des données du problème.
}{
	On applique le PFD à la bille~:
	\begin{DispWithArrows*}
		m\dv{\vv{v}}{t} &\stm{=} \vv{P} + \vv{\Pi} +\vv{f}
		\CArrow{$\cdot \uz$}
		\\\Ra
		\rho_a \frac{4}{3}\pi R^{3}\dv{v}{t} &\stm{=}
		\rho \frac{4}{3}\pi R^{3} g -6\pi\eta R v
		\Arrow{Forme canonique}
		\\\Lra
		\dv{v}{t} + \frac{9\eta}{2\rho_a R^2}v &\stm{=}
		\frac{\rho  g}{\rho_a}
		\Arrow{Constantes}
		\\\Lra
		\Aboxed{\dv{v}{t} + \frac{v}{\tau} &\stm[-1]{=} \frac{v_l}{\tau}}
	\end{DispWithArrows*}
	\begin{gather*}
		\beforetext{Ainsi,}
		\boxed{\tau \stm[-1]{=} \frac{2\rho_a R^2}{9\eta}}
		\qet
		v_l = \frac{\tau \rho g}{\rho_a} \stm{=} \boxed{\frac{2\rho g R^2}{9\eta}}
	\end{gather*}
}

\enonce{
	L'expérience est réalisée dans un tube vertical contenant de la glycérine. On
	lâche la bille à la surface du liquide choisie comme référence des altitudes,
	puis on mesure la durée $\Delta t=\SI{1.6}{s}$ mise pour passer de
	l'altitude $z_1 = \SI{40}{cm}$ à $z_2 = \SI{80}{cm}$.
}

\QR[5]{%
	En déduire la valeur de la vitesse limite, puis l'expression et la valeur de
	la viscosité $\eta$. L'exprimer en terme de pascals (Pa).
}{
	On suppose que le régime permanent est atteint (on vérifiera \textit{a
		posteriori} cette hypothèse)~:
	\smallbreak
	\begin{isd}
		\begin{gather*}
			\boxed{v_l \stm[-1]{=} \frac{\Delta z}{\Delta t}}
			\qav
			\left\{
			\begin{array}{rcl}
				\Delta{z} & = & \SI{40e-2}{m}
				\\
				\Delta{t} & = & \SI{1.6}{s}
			\end{array}
			\right.\\
			\AN
			\xul{
				v_l \stm{=} \SI{0.25}{m.s^{-1}}
			}
		\end{gather*}
		\tcblower
		\begin{gather*}
			\Ra
			\boxed{\eta \stm[-1]{=} \frac{2\rho g R^2}{9v_l}}
			\qav
			\left\{
			\begin{array}{rcl}
				\rho & = & \SI{6.640e3}{kg.m^{-3}}
				\\
				g    & = & \SI{9.8}{m.s^{-2}}
				\\
				R    & = & \SI{5e-3}{m}
				\\
				v_l  & = & \SI{0.25}{m.s^{-1}}
			\end{array}
			\right.\\
			\AN
			\xul{
			\eta \stm{=} \SI{1.45}{kg.m^{-1}.s^{-1}} \stm{=} \SI{1.45}{Pa.s}
			}
		\end{gather*}
	\end{isd}
}

\QR[1]{%
	Pourquoi ne pas avoir réalisé de mesure depuis la surface du liquide~?
}{
	Il faut attendre d'être sûr-e que la bille ait atteint le régime
	permanent\pt{1}.
}

\QR[3]{%
	Que vaut numériquement $\tau$~? Commenter.
}{
	\begin{gather*}
		\boxed{\tau \stm{=} \frac{2\rho_a R^2}{9\eta}}
		\qav
		\left\{
		\begin{array}{rcl}
			\rho_a & = & \SI{7900}{kg.m^{-3}}
			\\
			R      & = & \SI{5e-3}{m}
			\\
			\eta   & = & \SI{1.45}{kg.m^{-1}.s^{-1}}
		\end{array}
		\right.\\
		\AN
		\xul{
			\tau \stm{=} \SI{30e-3}{s}
		}
	\end{gather*}
	L'hypothèse de régime permanent est donc bien validée \pt{1} car $\tau \ll
		\Delta t$.
}

\QR[2]{%
	Pourquoi avoir choisi de la glycérine plutôt que de l'eau~?
}{
	La glycérine est plus visqueuse \pt{1} donc le régime permanent est atteint
	plus rapidement. Avec de l'eau ($\eta = \SI{e-3}{Pa.s}$), il n'est pas sûr que
	la bille puisse atteindre sa vitesse limite avant la fin de la chute \pt{1}.
}

\end{document}
