\documentclass[a4paper, 10pt, garamond, oneside]{book}
\usepackage{cours-preambule}

\toggletrue{student}
\HideSolutionstrue

\makeatletter
\renewcommand{\@chapapp}{MPSI3 -- 29 septembre 2023 -- Devoir surveillé}
\makeatother

\graphicspath{{./figures/}{./figures/E1}{./figures/E2}{./figures/P1}{./figures/P2}}

\setlist[enumerate]{resume}
\setlist[enumerate,1]{leftmargin=10pt, label=\sqenumi}
\setlist[enumerate,2]{leftmargin=20pt, label=\Alph*)}

\newcommand{\figsvg}[1]{
  \begin{center}
    \subimport{figures/}{#1}
  \end{center}
}
\newcommand{\figsvgCap}[2]{
  \begin{center}
    \subimport{figures/}{#1}
    \captionof{figure}{#2}
  \end{center}
}

\begin{document}
\setcounter{chapter}{0}
\chapter{Électrocinétique\ifprof{\!\!-- corrigé}}
\label{ch:ds02}

\ifstudent{
\begin{center}
	\Large\bfseries
	\xul{Tout moyen de communication est interdit}
	\smallbreak
	\xul{Les téléphones portables doivent être éteints et rangés dans les sacs}
	\smallbreak
	\xul{Les calculatrices sont \textit{autorisées}}
\end{center}
\begin{prgm}
	\vspace{6pt}
	\begin{tcb}*[cnt, bld, fontupper=\large](ror){}
		Toute l'optique géométrique de MPSI.
	\end{tcb}
	\vspace{-8pt}
\end{prgm}
Le devoir est composé des parties \textit{indépendantes} suivantes~:
{\Large
\begin{itemize}[label=$\diamond$]
	\bitem{Exercice 1}~: Étude de quelques lentilles minces
	\bitem{Exercice 2}~: Instruments d'optiques à l'infini -- deux parties
	indépendantes
	\bitem{Problème 1}~: Étude de pierres précieuses
	\bitem{Problème 2}~: Image au fond d'un gobelet
\end{itemize}
}

Les différentes questions peuvent être traitées dans l'ordre désiré.
\textbf{Cependant}, le numéro complet de la question doit être indiqué, et
\textbf{vous indiquerez si vous traitez la question d'un exercice sur une page
	complètement déconnectée}, sous peine de n'être ni vue ni corrigée.
\bigbreak
Une attention particulière sera portée à la \textbf{qualité de rédaction}. Les
hypothèses doivent être clairement énoncées, les propositions reliées entre
elles par des connecteurs logiques, les lois et théorèmes énoncés, sans pour
autant devenir une composition de français.
\bigbreak
De plus, la \textbf{présentation} de la copie sera prise en compte. Outre la
numérotation des questions, l'écriture, l'orthographe, les encadrements, la
marge, le cadre laissé pour la note et le commentaire font partie des points à
travailler. Il est notamment attendu que \textbf{les expressions littérales
	soient encadrées}, que \textbf{les calculs n'apparaissent pas} mais que le
détail des grandeurs avec leurs unités soit indiqué, et \textbf{les applications
	numériques soulignées}.
\bigbreak
Ainsi, l'étudiant-e s'expose aux malus suivants concernant la forme et le fond~:
\begin{tcb}*(prop)"bomb"{Malus}
	\begin{minipage}{0.50\linewidth}
		\begin{itemize}
			\item A~: application numérique mal faite~;
			\item N~: numéro de copie manquant~;
			\item P~: prénom manquant~;
			\item E~: manque d'encadrement des réponses~;
			\item M~: marge non laissée ou trop grande~;
			\item V~: confusion ou oubli de vecteurs~;
		\end{itemize}
	\end{minipage}
	\begin{minipage}{0.50\linewidth}
		\begin{itemize}
			\item Q~: question mal ou non indiquée~;
			\item C~: copie grand carreaux~;
			\item U~: mauvaise unité (flagrante)~;
			\item H~: homogénéité non respectée~;
			\item S~: chiffres significatifs non cohérents~;
			\item $\f$~: loi physique fondamentale brisée.
		\end{itemize}
	\end{minipage}
\end{tcb}

\begin{tcb}(impo){Exemple application numérique}
	\vspace*{-10pt}
	\begin{minipage}{0.45\linewidth}
		\begin{gather*}
			\boxed{n = \frac{PV}{RT}}
			\qav
			\left\{
			\begin{array}{rcl}
				p & = & \SI{1.0e5}{Pa}                \\
				V & = & \SI{1.0e-3}{m^3}              \\
				R & = & \SI{8.314}{J.mol^{-1}.K^{-1}} \\
				T & = & \SI{300}{K}
			\end{array}
			\right.\\
			\mathrm{A.N.~:}\quad
			\xul{n = \SI{5.6e-4}{mol}}
		\end{gather*}
	\end{minipage}
	\hfill
	\cancel{\bcancel{
			\begin{minipage}{0.45\linewidth}
				\begin{gather*}
					n = \frac{PV}{RT} = \frac{\num{e5}\cdot\num{1}}{8.32\cdot300}
					= 0.56
				\end{gather*}
			\end{minipage}
		}}
\end{tcb}
\newpage
}

\setcounter{section}{0}
\exercice[30]{Modélisation d'un dipôle linéaire}
\restartlist{enumerate}

\newpage

\setcounter{section}{0}
\prblm[30]{Charge d'un condensateur}
\restartlist{enumerate}
\ifstudent{
On réalise le circuit correspondant au schéma de la figure \ref{fig:elec1}.
\begin{figure}[htbp]
  \centering
  \includegraphics[scale=1]{elec_1}
  \caption{Circuit 1.}
  \label{fig:elec1}
\end{figure}
    Un dispositif d'acquisition de données relié à un ordinateur permet de
    suivre l'évolution de la tension aux bornes du condensateur en fonction du
    temps $t$.
	
    \begin{figure}[htbp]
      \centering
      \includegraphics[scale=1]{elec_2}
      \caption{Charge du condensateur.}
      \label{fig:elec2}
    \end{figure}
	
  On déclenche les acquisitions à la fermeture de l'interrupteur $K$, le
condensateur étant préalablement déchargé. L'ordinateur nous donne alors $u_c =
f(t)$, courbe  de la figure \ref{fig:elec2}. On donne $R = 50\si{\ohm}$. }

% \resetQ
% \subimport{/home/nora/Documents/Enseignement/Prepa/bpep/exercices/DS/balise_lumineuse/}{sujet.tex}
%
%
% \resetQ
% \subimport{/home/nora/Documents/Enseignement/Prepa/bpep/exercices/DS/circuit_RL_transitoire/}{sujet.tex}
%
%
% \resetQ
% \subimport{/home/nora/Documents/Enseignement/Prepa/bpep/exercices/DS/modelisation_dipole_lineaire/}{sujet.tex}
%
%
% \resetQ
% \subimport{/home/nora/Documents/Enseignement/Prepa/bpep/exercices/DS/point_fonctionnement_diode/}{sujet.tex}
%
%
% \resetQ
% \subimport{/home/nora/Documents/Enseignement/Prepa/bpep/exercices/DS/Station_TGV/}{sujet.tex}
%



\label{LastPage}
\end{document}
