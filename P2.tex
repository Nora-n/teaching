\documentclass[../DS05.tex]{subfiles}
\graphicspath{{./figures/}}

\begin{document}

\prblm[42]{Le bleu du ciel}
% \subimport{/home/nora/Documents/Enseignement/Prepa/bpep/exercices/DS/ciel_bleu/}{sujet.tex}
\enonce{
	Thomson a proposé un modèle d'atome dans lequel chaque électron
	$(M)$ est élastiquement lié à son noyau $(O)$~: il est soumis à une force de
	rappel $\vv{F}_R$ passant par le centre de l'atome. Dans tout l'exercice, on
	admettra que l'on peut se ramener à un problème selon une unique direction
	$(0,\ex)$, c'est-à-dire que $\vv{F}_R = -kx\ex$, où $x$ est la
	distance entre l'électron et l'atome.
	\smallbreak
	Nous supposerons que cet électron est freiné par une force de frottement de
	type fluide proportionnel à sa vitesse $\vv{F}_f = -h\vv{v} =
		-h\dv{x}{t}\ex$ et que le centre $O$ de l'atome est fixe dans le
	référentiel d'étude supposé galiléen.
	\smallbreak
	On admet qu'une onde lumineuse provenant du Soleil impose sur un électron de
	l'atmosphère, une force $\vv{F}_E=-eE_0\cos(\w t) \ex$.

	\paragraph*{Données.}
	masse d'une électron~: $m=\SI{9,1e-31}{\kilo\gram}$,
	charge élémentaire~: $e=\SI{1,6e-19}{\coulomb}$,
	célérité de la lumière dans le vide~: $c=\SI{3,00e8}{\metre\per\second}$,
	$k=\SI{500}{SI}$,
	$h=\SI{1e-20}{SI}$.
}

\QR{%
	Quelles sont les dimensions des grandeurs $k$ et $h$~? En quelles unités du
	système international les exprime-t-on~?
}{
	Par analyse dimensionnelle~:
	\[
		\dim(k) =
		\frac{{\rm force}}{{\rm longueur}} =
		\frac{MLT^{-2}}{L} \stm{=}
		\boxed{MT^{-2}}
		\quad~; \quad
		\dim(h) =
		\frac{{\rm force}}{{\rm vitesse}} =
		\frac{MLT^{-2}}{LT^{-1}} \stm{=}
		\boxed{MT^{-1}}
	\]
	Leurs unités en système international sont donc~:
	\[
		k~~\text{en}~~
		\boxed{\si{.\kilo\gram.\second^{-2}}} \stm{\qou} \boxed{\si{N.m^{-1}}}
		\qquad~; \qquad
		h~~\text{en}~~
		\boxed{\si{.\kilo\gram.\second^{-1}}} \pt{1}
	\]
}

\QR{%
	En utilisant le PFD, donner l'équation différentielle vérifiée par la position
	de l'électron $x(t)$.
}{
	D'après le PFD appliqué à l'électron dans le référentiel de l'atome considéré
	comme galiléen~:
	\[
		m\vv{a} \stm{=} \vv{F}_R + \vv{F}_f + \vv{F}_E
	\]
	En projetant cette relation sur l'axe $(O,\ex)$, on obtient~:
	\[
		\boxed{m\xpp \stm[-1]{=} -kx -h\xp -eE_0\cos(\w t) }
	\]
}

\QR{%
	Montrer qu'on peut l'exprimer sous la forme~:
	\[
		\dv[2]{x}{t}+\frac{\w_0}{Q}\dv{x}{t}+\w_0^2 x(t) = -\frac{e}{m}E_0\cos(\wt)
	\]
	On donnera les expressions de $\w_0$ et $Q$ en fonction des données.
}{
	Sous forme canonique, cette équation est~:
	\[
		\xpp + \frac{h}{m}\xp + \frac{k}{m} x \stm{=} -\frac{eE_0}{m}\cos(\wt)
	\]
	\begin{gather*}
		\beforetext{On en déduit que~:}
		\frac{\w_0}{Q} = \frac{h}{m}
		\stm{\qet}
		\w_0^2 = \frac{k}{m}
		\\
		\beforetext{On trouve alors~:}
		\boxed{\w_0 \stm[-1]{=} \sqrt{\frac{k}{m}}}
		\qet
		Q = \frac{m\w_0}{h} \Lra \boxed{Q \stm{=} \frac{\sqrt{mk}}{h}}
	\end{gather*}
}

% \enonce{
% 	On peut chercher les solutions de cette équation différentielle sous la forme~:
% 	\eq{
% 		x(t)=x_h(t)+x_p(t),
% 	}
%
% 	où $x_h(t)$ est une solution de l'équation homogène et $x_p(t)$ une solution particulière.
% }

\QR{%
	Calculer $Q$. Que peut-on en déduire sur le régime transitoire~?
}{
	On trouve~:
	\[
		\boxed{Q \stm[-1]{=} \num{2.1e6} > \frac{1}{2}}
	\]
	On en déduit que le régime transitoire est \textbf{pseudo-périodique}. \pt{1}
}

\QR{%
Montrer que le temps caractéristique du régime transitoire est $\tau=2Q/\w_0$,
et donnez l'expression de la pseudo-pulsation $\W$. Au bout de combien de
temps peut-on considérer le régime transitoire comme terminé~? Calculer
$\tau$. Peut-on considérer que l'électron est en régime permanent~?
}{
Le régime transitoire correspond à la solution homogène $x_h(t)$ telle que
\[
	\xpp + \frac{\w_0}{Q}\xp + \w_0{}^{2} x \stm{=} 0
\]
d'équation caractéristique
\[
	r^{2} + \frac{\w_0}{Q}r + \w_0{}^{2} \stm{=} 0
	\qav
	\Delta \stm{=} \frac{\w_0{}^{2}}{Q^{2}}\left( 1-4Q^{2} \right)
\]
Comme le régime est pseudo-périodique, on sait que les racines sont complexes,
et on aura
\[
	r_{\pm} \stm"2"{=}
	-\underbracket[1pt]{\frac{\w_0}{2Q}}_{1/\tau}
	\pm
	\jj \underbracket[1pt]{\frac{\w_0}{2Q}\sqrt{4Q^{2}-1}}_{\W}
\]
On a donc
\[
	\boxed{\tau \stm[-1]{=} \frac{2Q}{\w_0}}
	\qet
	\boxed{\Omega \stm[-1]{=} \frac{\w_0}{2Q}\sqrt{4Q^{2}-1}}
\]
Au bout de quelques $\tau$, on peut considérer que le régime transitoire est
nul. \pt{1}
\smallbreak
Par A.N., on trouve
\[
	\boxed{\tau \stm[-1]{=} \SI{1,8e-10}{\second}}.
\]
On suppose donc que l'électron est en régime permanent. \pt{1}
}

\QR[2]{%
	Pourquoi peut-on alors dire que $x(t) \approx X_m \cos(\w t +\f)$~?
}{
	On a
	\[
		x(t) \stm{=} x_h(t) + x_p(t)
	\]
	avec $x_h$ une solution homogène et $x_p$ la solution particulière, de même
	fréquence \pt{1} de l'excitation. Ainsi, pour des durées supérieures à quelques
	$\tau$, donc supérieures à $\SI{e-9}{s}$, on peut considérer que $x_h(t)=0$,
	soit $\boxed{x(t) \approx x_p(t)}$.
}

\QR{%
Exprimer $X_m$ en fonction de $\w_0$, de $Q$ et des données. On pourra utiliser la notation complexe.
}{
En notations complexes, on définit la représentation complexe $\xul{x}(t)=X_me^{j(\w t+\f)}$ et l'amplitude complexe $\xul{X_m}=X_me^{j\f}$.
On peut alors écrire~:
\[
	(\jw)^2 \xul{X_m} + \frac{(\jw) \w_0}{Q}\xul{X_m} +\w_0^2 \xul{X_m} \stm{=}
	\frac{-eE_0}{m}
	\quad \Rightarrow \quad
	\boxed{
		\xul{X_m} \stm[-1]{=}
		\frac{\frac{-eE_0}{m}}{-\w^2 + \frac{(\jw) \w_0}{Q} +\w_0^2}}
\]
On a alors~:
\eq{
	\boxed{X_m = |\xul{X_m}| = \frac{\frac{eE_0}{m}}{\sqrt{(\w_0^2 -\w^2)^2 + \left( \frac{\w \w_0}{Q} \right)^2} }
		=
		\frac{eE_0}{m\w_0^2}\frac{1}{\sqrt{\left(1-\frac{\w^2}{\w_0^2} \right)^2 + \frac{\w^2}{Q^2\w_0^2}}}
	}
}

}


\QR{%
	Exprimer $\f$ en fonction de $\w_0$ et de $Q$. On pourra également utiliser la notation complexe.
}{
	On peut réécrire l'amplitude complexe~:
	\eq{
		\begin{aligned}
			\xul{X_m} & = & \left( \frac{-eE_0}{m}\right)\times \left(\w_0^2-\w^2 \right)^{-1}\times  \left(1 + \frac{j}{Q} \frac{\w\w_0}{\w_0^2-\w^2}\right)^{-1}                                \\
			          & = & \left( \frac{eE_0}{m}\right)\times \left(\w^2-\w_0^2 \right)^{-1}\times  \left(1 + \frac{j}{Q} \frac{1}{\left(\frac{\w_0}{\w} - \frac{\w}{\w_0} \right)}\right)^{-1}.
		\end{aligned}
	}
	Finalement~:
	\[
		\f = \arg*{\xul{X_m}}
		=
		\arg*{\w_0} -
		\arg*{\w^2-\w_0^2} -
		\arg*{1 + \frac{j}{Q} \frac{1}{\frac{\w_0}{\w} - \frac{\w}{\w_0}} }
	\]
	On trouve alors~:
	\[
		\boxed{\f =
			- \arg*{\w^2-\w_0^2}
			- \arctan\frac{1}{Q\left(\frac{\w_0}{\w} - \frac{\w}{\w_0} \right)}}
	\]

	où $\arg*{ \w^2-\w_0^2  }$ est égal à $0$ si $\w >0$ ou $\pi$ sinon.
}

\enonce{
	Les longueurs d'ondes $\lambda$ du Soleil sont principalement incluses dans le domaine du visible, ainsi on considère que $\lambda \in [\lambda_b, \lambda_r]$, où $\lambda_b$ (resp. $\lambda_r$) est la longueur d'onde du rayonnement bleu (resp. rouge).
}


\QR{%
	Que valent $\lambda_b$ et $\lambda_r$~?
}{
	$\boxed{\lambda_b=\SI{400}{\nano\metre}}$ et $\boxed{\lambda_r=\SI{800}{\nano\metre}}$.
}


\QR{%
	En déduire que $\w \in[\w_r,\w_b]$. On donnera les valeurs littérales de $\w_r$ et $\w_b$ et on effectuera les applications numériques.
}{
	Le lien entre pulsation et longueur d'onde est~:
	\eq{
		\w = \frac{2\pi c}{\lambda}
	}

	Ainsi~:
	\eq{
		\w \in[\w_r,\w_b]
		\quad \text{avv} \quad
		\boxed{\w_r = \frac{2\pi c}{\lambda_r}=\SI{2,36e15}{\radian\per\second}}
		\quad~; \quad
		\boxed{\w_b = \frac{2\pi c}{\lambda_b}=\SI{4,71e15}{\radian\per\second}}
	}

}


\QR{%
	Calculer $\w_0$.
}{
	On trouve
	\eq{
		\boxed{\w_0 = \SI{2,34e16}{\radian\per\second}}
	}

}


\QR{%
	En déduire que~:
	\eq{
		X_m \approx \frac{eE_0}{m\w_0^2}.
	}

}{
	En comparant $\w$ et $\w_0$, on peut considérer que $\w_0 \gg \w$ (il y a au moins un facteur 5 entre les 2, c'est un peu juste). De plus, $Q\gg 1$. Ainsi on peut simplifier le dénominateur du $\xul{X_m}$ car
	\eq{
		\frac{\w\w_0}{Q} \ll \w^2 \ll \w_0^2.
	}

	Dans ce cas,
	\eq{
		\boxed{X_m \approx \frac{eE_0}{m\w_0^2}}.
	}

}

\enonce{
	Un électron diffuse dans toutes les directions un rayonnement dont la puissance moyenne $P$ est proportionnelle au carré de l'amplitude de son accélération.
}


\QR{%
	Montrer que~:
	\eq{
		P = K \left(  \frac{eE_0\w^2}{m\w_0^2}\right)^2,
	}

	où $K$ est une constante que l'on ne cherchera pas à exprimer.
}{
	En amplitude complexe, l'accélération est~:
	\eq{
		\xul{A_m} = (\jw)^2 \xul{X_m}
		\quad \Rightarrow \quad
		A_m = \frac{eE_0\w^2}{m\w_0^2}.
	}

	D'après le sujet, la puissance est proportionnelle au carré de l'amplitude de l'accélération, donc
	\eq{
		\boxed{P=KA_m^2 = K\left(\frac{eE_0\w^2}{m\w_0^2} \right)^2}.
	}

}


\QR{%
	Expliquer alors pourquoi le ciel est bleu.
}{
	On peut comparer la puissance diffusée pour un rayonnement bleu avv un rayonnement rouge~:
	\eq{
		\boxed{\frac{P_b}{P_r} = \frac{\w_b^2}{\w_r^2} = 4}
	}

	La puissance diffusée pour les rayonnements bleu est 4 fois plus importante que celle pour un rayonnement rouge, d'où la couleur du ciel.
}

\end{document}
