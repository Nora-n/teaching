\documentclass[a4paper, 12pt, final, garamond]{book}
\usepackage{cours-preambule}

\raggedbottom

\makeatletter
\renewcommand{\@chapapp}{Programme de kh\^olle -- semaine}
\makeatother

\begin{document}
\setcounter{chapter}{6}

\chapter{Du 13 au 16 novembre}

\section{Cours et exercices}

\section*{Électrocinétique ch.\ 4 -- Oscillateurs harmonique et amorti}
\begin{enumerate}[label=\Roman*]
	\bitem{Oscillateurs harmoniques}~:
	\begin{enumerate}[label=\Alph*]
		\bitem{Introduction harmonique}~: signal sinusoïdal, équation différentielle
		générale et solution, changement de variable, exemple expérimental LC.
		\bitem{Oscillateur harmonique LC libre}~: présentation, équation
		différentielle, unité de $\w_0$, solutions $u_C(t)$ et $i(t)$, graphique,
		bilan énergétique et graphique.
		\bitem{Ressort horizontal libre}~: force de rappel (de \textsc{Hooke}),
		schéma et situation initiale, équation différentielle et solution, analogie
		LC-ressort, bilan de puissance et conservation de l'énergie avec définition
		énergie potentielle élastique, tracé dans l'espace des phases.
		\bitem{Complément~: circuit LC montant}~: présentation, équation solutions
		et tracé, bilan d'énergie et tracé.
	\end{enumerate}
	\bitem{Oscillateurs amortis}~:
	\begin{enumerate}[label=\Alph*]
		\bitem{Introduction amorti}~: exemple expérimental RLC, vocabulaire, équation
		différentielle générale, dimension de $Q$, équation caractéristique et
		régimes de solutions, présentation des solutions générales.
		\bitem{Oscillateur amorti RLC libre}~: présentation, bilan de puissance,
		équation différentielle, et solutions dans tous les régimes avec tracé et
		transitoire à 95\%~; extrapolation à $Q \ra \infty$ et $Q \ra 0$~; tracés
		dans l'espace des phases.
		\bitem{Ressort horizontal amorti libre}~:
		schéma et situation initiale, équation différentielle, analogie RLC-ressort
		amorti, bilan de puissance.
	\end{enumerate}
\end{enumerate}

\section{Cours uniquement}
\section*{Chimie chapitre 1 -- Introduction}
\begin{enumerate}[label=\Roman*]
	\bitem{Vocabulaire général}~: atomes et molécules, classification par
	composition, états de la matière et systèmes physico-chimiques,
	transformations de la matière.
	\bitem{Quantification des systèmes}~: mole, masse molaire, fractions
	molaire et massique, masse volumique, concentrations molaire et
	massique, dilution~; pression d'un gaz, modèle du gaz parfait, volume
	molaire, pression partielle et loi de \textsc{Dalton}.
	%, intensivité et extensivité, activité d'un élément chimique.
\end{enumerate}

\section{Questions de cours possibles}
\begin{center}
	\begin{framed}
		\Large
		Pas de tableaux d'avancements pour cette semaine.
	\end{framed}
\end{center}

\newpage

\begin{enumerate}
	\item[] \textbf{Chapitre 4~: harmonique}
	\item Présenter le schéma et les conditions initiales, établir l'équation
	      différentielle, \textbf{justifier l'unité de $\w_0$}, établir les
	      solutions de $u_C(t)$ et $i(t)$ (ou $x(t)$ (ou $\ell(t)$) et $v(t)$)
	      et les tracer en fonction du temps \textbf{puis} dans l'espace des
	      phases sans tenir compte des constantes multiplicatives pour un des
	      systèmes suivants~:
	      \begin{tasks}[label=\protect\fbox{\Alph*}, label-width=4ex](3)
		      \task LC libre
		      \task LC montant
		      \task Ressort libre sans frottements
	      \end{tasks}

	\item Faire un \textbf{bilan de puissance} pour le circuit \textbf{LC libre}
	      \textit{et} le \textbf{ressort sans frottements}, démontrer la
	      conservation de l'énergie totale, tracer la forme des graphiques.

	\item Faire l'analogie complète entre les deux systèmes harmoniques LC libre
	      et ressort sans frottement~: présentation, conditions initiales,
	      équations différentielles \textbf{sans démonstration}, correspondance
	      entre les grandeurs, tracé de la solution \textbf{dans l'espace des
		      phases} sans résolution et commenter sur la conservation de l'énergie
	      visible dans le graphique.

	\item[] \textbf{Chapitre 4~: amorti}
	\item Présenter (schéma et conditions initiales),
	      donner et \textbf{démontrer} l'équation différentielle sous
	      forme canonique \textbf{qu'on ne cherchera pas à résoudre}, vérifier son
	      homogénéité, présenter les graphiques des solutions selon les valeurs de
	      $Q$ dans l'espace temporel \textbf{et} dans l'espace des phases en donnant
	      un approximation de la durée du régime transitoire à 95\% pour un des
	      systèmes suivants~:
	      \begin{tasks}[label=\protect\fbox{\Alph*}, label-width=4ex](2)
		      \task RLC libre
		      \task Ressort horizontal amorti
	      \end{tasks}

	\item Faire l'analogie complète entre les deux systèmes amortis RLC libre et
	      ressort avec frottement fluide~: présentation, conditions initiales,
	      équations différentielles \textbf{sans démonstration}, correspondance
	      entre les grandeurs, tracer de solutions \textbf{dans l'espace des
		      phases} selon différentes valeurs de $Q$ sans résolution et commenter
	      sur l'évolution de l'énergie visible dans le graphique.

	\item Résoudre l'équation différentielle d'un oscillateur amorti de conditions
	      initiales données par l'interrogataire pour l'un des trois régimes
	      possibles.

	\item Faire le \textbf{bilan de puissance} de l'oscillateur amorti électrique
	      \textbf{RLC} libre \textit{et} du \textbf{ressort horizontal avec
		      frottement fluide}, identifier les termes du bilan et expliciter la
	      signification physique de chacun des termes.
	\item[] \textbf{Chimie chapitre 1}
	\item L'air est constitué, en quantité de matière, à 80\% de diazote \ce{N2} et
	      à 20\% de dioxygène \ce{O2}.
	      \smallbreak
	      On a
	      $M(\ce{N_2}) = \SI{28.0}{g.mol^{-1}}$ et
	      $M(\ce{O_2}) = \SI{32.0}{g.mol^{-1}}$.
	      \smallbreak
	      En déduire les fractions molaires puis les fractions massiques.
	\item On dissout une masse $m = \SI{2.00}{g}$ de sel NaCl$\sol$ dans $V =
		      \SI{100}{mL}$ d'eau.
	      \smallbreak
	      On donne
	      $M(\ce{NaCl}) = \SI{58.44}{g.mol^{-1}}$ et
	      $M(\ce{Na}) = \SI{22.99}{g.mol^{-1}}$.
	      \smallbreak
	      Déterminer les concentrations molaire et massique en \ce{Na^{+}} dans la
	      solution.
	\item On considère une seringue cylindrique de \SI{10}{cm} le long et de
	      \SI{2.5}{cm} de diamètre, contenant \SI{0.250}{g} de diazote de masse
	      molaire $M({\rm N}_2) = \SI{28.01}{g.mol^{-1}}$ à la
	      température $T = \SI{20}{\degreeCelsius}$.
	      \begin{enumerate}
		      \item Calculer le volume de la seringue
		      \item Calculer la quantité de matière dans la seringue
		      \item Calculer la pression exercée par le diazote dans la seringue
	      \end{enumerate}
\end{enumerate}
\end{document}

