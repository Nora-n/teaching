\documentclass[a4paper, 12pt, final, garamond]{book}
\usepackage{cours-preambule}

\raggedbottom

\makeatletter
\renewcommand{\@chapapp}{Programme de kh\^olle -- semaine}
\makeatother

\begin{document}
\setcounter{chapter}{15}

\chapter{Du 29 janvier au 1\ier{} février}

\section{Cours et exercices}
\ssubsection{M2}{Dynamique du point}
\begin{enumerate}[label=\Roman*]
	\bitem{Introduction}~: inertie et quantité de mouvement, forces
	fondamentales.
	\bitem{Trois lois de \textsc{Newton}}~: principe d'inertie, principe
	fondamental de la mécanique, loi des actions réciproques.
	\bitem{Ensembles de points}~: centre d'inertie, quantité de mouvement
	d'un ensemble de points, théorème de la résultante cinétique, méthode
	générale de résolution.
	\bitem{Forces usuelles}~: poids, chute libre avec angle initial~; poussé
	d'\textsc{Archimède}~; frottements fluides, chute libre avec frottements
	linéaires et quadratique, résolution par adimensionnement~; frottements
	solides~; force de rappel d'un ressort et longueur d'équilibre vertical.
\end{enumerate}

\ssubsection{M3}{Mouvements courbes}
\begin{enumerate}[label=\Roman*]
	\bitem{Mouvement courbe dans le plan}~: position, vitesse,
	déplacement élémentaire, accélération en coordonnées polaires.
	\bitem{Exemples de mouvements plans}~: mouvement circulaire,
	circulaire uniforme, repère de \textsc{Frenet}.
	\bitem{Application~: pendule simple}~: tension d'un fil, pendule simple.
	\bitem{Mouvement courbe dans l'espace}~: coordonnées cylindriques,
	coordonnées sphériques.
\end{enumerate}

\section{Cours uniquement}
\ssubsection{M4}{Approche énergétique}
\begin{enumerate}[label=\Roman*]
	\bitem{Notions énergétiques}~: énergie, conservation, puissance.
	\bitem{Énergie cinétique et travail d'une force constante}~: définitions,
	exemples, travail du poids.
	\bitem{Puissance d'une force et travail élémentaire}~: définitions, TPC,
	TEC, et applications, comment choisir~?
	\bitem{Énergie potentielle}~: forces conservatives ou
	non, travail d'une force conservative, gradient d'une fonction scalaire,
	opérateur différentiel, lien à l'énergie potentielle
	\bitem{Énergie mécanique}~: définition, TEM et TPM et applications.
	\bitem{Énergie potentielle et équilibres}~: notion d'équilibre, lien
	avec $\Ec_p$, équilibres stables et instables, lien avec
	$\dv[2]{\Ec_p}{x}$
	%     , étude générale autour d'un point d'équilibre
	%     stable~: oscillateur harmonique.
	% \bitem{Énergie potentielle et trajectoire}~: détermination
	%     qualitative d'une trajectoire, état lié et diffusion~; cas du pendule
	%     simple, étude mouvement selon $\Ec_p$ et $\Ec_m$.
\end{enumerate}

\newpage
\section{Questions de cours possibles}
\ssubsection{M2}{Dynamique du point}
\begin{enumerate}
	\item
	      Présenter les lois du frottement de \textsc{Coulomb}, et refaire
	      l'exercice~:
	      \smallbreak
	      On considère un plan incliné d'un angle $\alpha$ par
	      rapport à l'horizontale. Une brique de masse $m$ est
	      lancée depuis le bas du plan vers le haut, avec une vitesse $v_0$.
	      On suppose qu'il existe des frottements solides,
	      avec $f$ le coefficient de frottements solides tel que $f =
		      \num{0.20}$.
	      \begin{enumerate}[label=\sqenumi]
		      \item Établir l'équation horaire du mouvement de la brique lors de
		            sa montée.
		      \item Déterminer la date à laquelle la brique s'arrête, ainsi que la
		            distance qu'elle aura parcourue. Commenter l'expression littérale.
	      \end{enumerate}
	\item
	      Position d'équilibre d'un ressort vertical~: présenter le système,
	      déterminer l'équation différentielle sur la position de la masse,
	      déterminer la longueur d'équilibre, solution pour des conditions
	      initiales données par l'interrogataire.
\end{enumerate}

\ssubsection{M3}{Mouvements courbes}
\begin{enumerate}[resume]
	\item
	      Présenter les coordonnées cylindriques avec un schéma introduisant la
	      base et indiquant les coordonnées, donner l'expression de $\OM$ dans
	      cette base, donner \textbf{et démontrer} l'expression de la vitesse, du
	      déplacement élémentaire et de l'accélération en coordonnées cylindriques.
	\item
	      Étude du pendule simple~: mise en situation, équation différentielle,
	      linéarisation, résolution. Que se passe-t-il pour de grands angles~? Tracer
	      l'allure de l'évolution de $T/T_0$ en fonction de $\tt_0$.
\end{enumerate}

\ssubsection{M4}{Approche énergétique}
\begin{enumerate}[resume]
	\item Définir la puissance d'une force, son travail élémentaire, ainsi que
	      son travail sur un chemin entre A et B.
	\item Énoncer et démontrer les théorèmes de la puissance cinétique et de
	      l'énergie cinétique.
	\item Retrouver les énergies potentielles de forces classiques (poids,
	      rappel élastique). Comment trouver une force à partir à partir de son
	      énergie potentielle~?
	\item Énoncer et démontrer les théorèmes de la puissance mécanique et de
	      l'énergie mécanique.
	\item Retrouver l'équation différentielle sur $\tt$ du pendule simple non
	      amorti à l'aide du TPC.
	\item Expliquer l'obtention des positions d'équilibre et leur
	      stabilité sur un graphique $\Ec_p(x)$. Démontrer l'équilibre et sa
	      stabilité en terme de conditions sur la dérivée première et seconde de
	      l'énergie potentielle.
	      % \item Savoir discuter le mouvement d'une particule en comparant son profil
	      %     d'énergie potentielle et son énergie mécanique~; état lié ou de
	      %     diffusion.
	      % \item Savoir réaliser l'approximation harmonique d'une cuvette de potentiel
	      %     par développement limité. En déduire que tout système décrit par une
	      %     énergie potentielle présentant un minimum local est assimilable à un
	      %     oscillateur harmonique.
\end{enumerate}

\end{document}
