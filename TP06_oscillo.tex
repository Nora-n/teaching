\documentclass[../main/main.tex]{subfiles}
\graphicspath{{./figures/}}

\makeatletter
\renewcommand{\@chapapp}{Travaux pratiques -- TP}
\makeatother

\toggletrue{student}
\HideSolutionstrue

\begin{document}
\setcounter{chapter}{4}

\chapter{Oscilloscope et trac\'e de caract\'eristiques}

\ifstudent{
	\begin{prgm}
		\begin{tcb}*(ror)"know"{Savoirs}
			\begin{itemize}[label=$\diamond$, leftmargin=10pt]
				\item Expliquer le lien entre résolution, calibre, nombre de points de
				      mesure~;
				\item Préciser la perturbation induite par l'appareil de mesure sur le
				      montage et ses limites (bande passante, résistance d'entrée)~;
				\item Définir la nature de la mesure effectuée (valeur efficace, valeur
				      moyenne, amplitude, valeur crête à crête, etc.)~;
				\item Gérer, dans un circuit électronique, les contraintes liées à la
				      liaison entre les masses.
			\end{itemize}
		\end{tcb}
		\begin{tcb}*(ror)"how"{Savoir-faire}
			\begin{itemize}[label=$\diamond$, leftmargin=10pt]
				\item Mesurer une tension au voltmètre ou à l'oscilloscope~;
				\item Mesurer une intensité à l'ampèrementre ou à l'oscilloscope aux
				      bornes d'une résistance adaptée.~;
				\item Mesurer une résistance à l'ohmmètre ou à l'oscilloscope ou au
				      voltmètre par diviseur de tension~;
			\end{itemize}
		\end{tcb}
	\end{prgm}
	\vspace{-10pt}
	\section{Objectifs}
	\begin{itemize}
		\item Se familiariser avec le GBF et l'oscilloscope numérique.
		\item Réaliser des montages simples d'électricité.
		\item Mesurer la résistance d'entrée $R_{e}$ d'un oscilloscope et la
		      résistance de sortie $R_{s}$ d'un GBF.
		\item Tracer une caractéristique de dipôle en utilisant un transformateur
		      d'isolement.
	\end{itemize}
}

\section{S'approprier}
\subsection{Résistances d'entrée et de sortie}

Nous avons vu en TD la méthode dite de la «~demi-tension~» qui permet de mesurer
la résistance d'entrée et de sortie d'un appareil (cf.\ exercice I TD 2, fait
\textit{via} la puissance)

\subsubsection{Résistance de sortie du générateur basse fréquence (GBF)}

\begin{wrapfigure}[6]{R}{0.35\textwidth}
	\vspace{-35pt}
	\begin{center}
		\includegraphics[width=0.35\textwidth]{res_entree}
	\end{center}
\end{wrapfigure}

Le GBF est un générateur réel pouvant être modélisé comme une association série
d'un générateur idéal de tension de force électromotrice $e$ associé à une
résistance de sortie $r_{s}$ (modèle de Thévenin). Comme vu en cours, on branche
le GBF sur une résistance variable $R'$ puis on mesure la tension $U_1$ aux
bornes de $R'$. Montrer que lorsque $U_1 = e/2$, alors $R' = r_{s}$. En déduire
une méthode simple de mesure expérimentale de $r_{s}$.

\medskip
\vspace{10pt}

\begin{tcb}(expe){Aide}
	Afin de mesurer $U_1$, l'oscilloscope se branche entre la masse (reliée à la
	borne noire de l'oscilloscope) et le nœud $Y_1$ (relié à la borne rouge de
	l'oscilloscope). Notez que dans un circuit, \textbf{la masse est un nœud
		commun à tous les appareils branchés}.
	Par conséquent, la borne noire du GBF ainsi que les deux bornes noires de
	l'oscilloscope doivent être impérativement reliées entre elles. Si ce n'est
	pas le cas, votre montage ne fonctionnera pas.
\end{tcb}

\subsubsection{Résistance d'entrée de l'oscilloscope}

\begin{wrapfigure}[10]{r}{0.4\textwidth}
	\vspace{-20pt}
	\begin{center}
		\includegraphics[width=0.4\textwidth]{res_sortie}
	\end{center}
	\vspace{-30pt}
\end{wrapfigure}

L'entrée d'un oscilloscope est assimilable à une résistance d'entrée $R_{e}$ en
dérivation avec une capacité $C_{e}$. À basse fréquence, le condensateur est
assimilable à un interrupteur ouvert si bien que l'on peut dans un tel régime
négligé sa présence. Montrer alors, en vous aidant du schéma, que la tension
$U_2$, mesurée par l'oscilloscope (modélisée par une résistance et une capacité
en parallèle), est égale à $U_{e}/2$ lorsque $R' = R_{e}$. En déduire une
méthode simple de mesure expérimentale de $R_{e}$. Remarquez que, contrairement
à ce qui est fait dans le cours, la résistance de sortie du GBF n'apparaît pas.
Elle est en réalité très faible devant les autres résistances $R_{e}$ et $R'$ et
sera donc négligée.

\subsection{Mesures avec un oscilloscope}

À partir du menu mesure, l'oscilloscope est capable de réaliser des mesures
automatiques des principales caractéristiques des signaux électriques. Vous
pourrez en particulier afficher~:

\begin{itemize}
	\item la période et la fréquence du signal~;
	\item la tension crête-crête $u_{\rm pp}$ du signal (valeur mesurée entre le
	      maximum et le minimum du signal)~;
	\item la tension efficace $u_{\rm eff}$ définie par
	      \encadre{$S_{\rm eff} = \sqrt{ \left\langle s^2(t) \right\rangle}
			      = \sqrt{\frac{1}{T}\int_{t_0}^{t_0+T} s^2(t) \dt}$}
\end{itemize}

\begin{minipage}{0.45\linewidth}
	L'amplitude $A$ d'un signal (qui intervient dans l'expression d'un signal
	sinusoïdal selon $s(t) = A \cos(\omega t + \varphi)$) est liée à $V_{\rm
				pp}$ selon

	\encadre{$A = u_{\rm pp}/2$}
\end{minipage}
\hfill
\begin{minipage}{0.45\linewidth}
	Par ailleurs, pour un signal sinusoïdal, \textbf{et uniquement pour un
		signal sinusoïdal} la tension efficace s'écrit~:

	\encadre{$u_{\rm eff} = u_{\rm pp}/\sqrt{2} = \sqrt{2} A$}
\end{minipage}

\begin{tcb}[width=\linewidth](ror){Attention}
	Pour toute mesure, vérifier que la source du menu mesure correspond bien à
	la courbe sur laquelle vous faites des mesures.
\end{tcb}

\subsection{Utilisation des oscilloscopes}
\subsubsection{Imprimer une courbe avec un oscilloscope \texttt{Rigol}}

\begin{enumerate}
	\item Allumer l'ordinateur et se connecter au réseau.
	\item Puis, programme, discipline, physique-chimie, physique, oscillo
	      \texttt{rigol}.
	\item \texttt{Tools}, \texttt{connect to oscillo}, puis \texttt{refresh}.
	\item Passer en noir et blanc (\texttt{B \& W}) et enfin \texttt{print}.
\end{enumerate}


\subsubsection{Imprimer une courbe avec un oscilloscope \texttt{Tektronix}}

\begin{enumerate}
	\item Ouvrir \texttt{Open Choice Desktop}. Sélectionner \texttt{instrument
		      USB} puis \texttt{Afficher écran}.
	\item Copier vers le presse-papier. Ouvrir \texttt{paint} et coller.
	\item Puis cliquer droit, inverser les couleurs.
	\item Sélection rectangulaire, pour ne garder que les oscillogrammes et les
	      réglages de l'oscilloscope.
	\item Copier~; Basculer dans \texttt{libre-office} ou \texttt{word} et
	      Coller~;
	\item Faire une belle mise en page et mettre des titres et commentaires
	      éventuels. Puis imprimer.
\end{enumerate}

\section{Réaliser~: Résistances d'entrée et de sortie}
\subsection{Mesure de la résistance de sortie du GBF}

\begin{enumerate}
	\item Réaliser le montage vu dans la partie \textit{S'approprier} pour une
	      fréquence d'environ $\SI{100}{Hz}$ et commencer avec $R'$ infinie, donc
	      débranchée.
	\item Mesurer alors $U_1$ grâce à l'oscilloscope, et régler le
	      \texttt{level} du GBF (bouton DC offset enfoncé) pour obtenir une
	      tension crête-crête de $\SI{2}{V}$. Cette tension correspond à la
	      tension à vide $e$ du générateur. En effet, à vide, c'est-à-dire pour
	      $R'$ infinie, le courant est nul et donc la tension relevée est
	      directement égale à $e$.
	\item Brancher la boîte de résistances variables $R'$ et l'ajuster pour
	      avoir $U_1 = e/2$.
	\item En déduire l'ordre de grandeur de la résistance de sortie $r_{s}$ du
	      GBF.
	\item Cette valeur est-elle cohérente avec les indications inscrites sur la
	      sortie du GBF~?
\end{enumerate}

\vspace{-0.6cm}

\subsection{Mesure de la résistance d'entrée de l'oscilloscope (modèle)}

\begin{enumerate}
	\item Prendre la notice de l'oscilloscope dont vous disposez, vérifier les
	      valeurs de $R_{e}$ et $C_{e}$ appelées \textit{input impedance} en
	      anglais.
	\item Mesurer d'abord la tension $U_{e}$ en connectant directement
	      l'oscilloscope au générateur (cela revient à prendre une résistance $R'$
	      nulle, assimilable à un fil donc).
	\item Réaliser ensuite le montage présenté dans la partie
	      \textit{S'approprier}.
	\item $U_{e}$ étant fixé ($\SI{2}{V}$ crête-crête), faire varier $R'$
	      jusqu'à ce que la tension aux bornes de l'oscilloscope ($U_2$) soit
	      égale à la moitié de la tension du générateur $U_{e}$. Vous pouvez
	      utiliser le menu mesure pour obtenir la valeur des deux tensions.
	\item En déduire l'ordre de grandeur de la résistance d'entrée $R_{e}$
	      expérimentale de l'oscilloscope.
\end{enumerate}

\section{Valider~: Effets des résistances d'entrée et de sortie}
\subsection{Effet de la résistance de sortie du GBF}

\begin{enumerate}
	\item Brancher l'oscilloscope aux bornes du GBF (toujours réglé à une
	      fréquence de $\SI{100}{Hz}$) et régler le \texttt{level} de celui-ci
	      pour obtenir une tension crête-crête de $\SI{2}{V}$. Comme précédemment,
	      nous mesurons ici la tension à vide $e$ du GBF.
	\item Brancher aux bornes du GBF deux résistances identiques de $R_1 = R_2 =
		      \SI{47}{\Omega}$ en série puis relever à l'aide de l'oscilloscope la
	      tension $U_1$ aux bornes de l'une d'elle.
	\item En appliquant le principe du pont diviseur de tension, que devrait
	      valoir $U_1$~? Est-ce la valeur que vous relevez~?
	\item Expliquez cet écart en considérant la résistance de sortie du GBF.
	\item Comment choisir $R_1$ et $R_2$ pour que l'on puisse négliger l'effet
	      de la résistance de sortie du GBF~? Reproduire le montage précédent en
	      utilisant désormais $R_1 = R_2 = \SI{10}{k\Omega}$. Montrer qu'alors
	      $U_1$ prend la valeur attendue.
\end{enumerate}

\subsection{Effet de la résistance d'entrée de l'oscilloscope}

\begin{enumerate}
	\item Brancher l'oscilloscope aux bornes du GBF (toujours réglé à une
	      fréquence de $\SI{100}{Hz}$) et régler le \texttt{level} de celui-ci
	      pour obtenir une tension crête-crête de $\SI{4}{V}$. Comme précédemment,
	      nous mesurons ici la tension à vide $e$ du GBF.
	\item Brancher aux bornes du GBF deux résistances identiques de $R_1 = R_2 =
		      \SI{1}{M\Omega}$ en série puis relever à l'aide de l'oscilloscope la
	      tension $U_1$ aux bornes de l'une d'elle.
	\item La tension $U_1$ obtenue est-elle conforme à vos attentes~? Expliquez
	      cet écart en tenant compte de la résistance d'entrée de l'oscilloscope.
	\item Comment choisir $R_1$ et $R_2$ pour que l'on puisse négliger l'effet
	      de la résistance d'entrée de l'oscilloscope~? Reproduire le montage
	      précédent en utilisant désormais $R_1 = R_2 = \SI{10}{k\Omega}$. Montrer
	      qu'alors $U_1$ prend la valeur attendue.
\end{enumerate}

\section{Conclure}

Résumer les recommandations pratiques que vous avez pu déduire de ce TP afin de
réaliser des mesures correctes en électricité.

\end{document}
