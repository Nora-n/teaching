\documentclass[a4paper, 10pt, garamond]{book}

\usepackage{cours-preambule}
\usepackage{paracol}

\begin{document}

\chapter{Test}

\begin{tcb*}[box](loi){Première loi}
	Or not?
	\[
		a = \frac{\pi}{2}
	\]
	\begin{isd}(loi)
		Left part
		\tcblower
		right part~?
	\end{isd}
\end{tcb*}

\begin{tcb}[lft](loi){Seconde}
	Seconde, pas stepped up
\end{tcb}

\begin{tcb*}[box](loi){Troisième, mais comptée comme 2}
	Parce que je suis trop forte~!
\end{tcb*}

\begin{tcb}[itc](prop)<aide>{Instruction}
	Ça marche~?
\end{tcb}

\begin{tcb}(rema)<dngr>{Remarque}
	Et oui~!
\end{tcb}

\begin{tcb}[box]{Définition}
	On appelle une \textbf{murge} une grosse cuite.
\end{tcb}

\begin{tcb*}[fil](prop){Première prop}
	Filled prop
\end{tcb*}

\begin{tcb}[fil](prop){Seconde prop}
	Filled prop
\end{tcb}

\begin{tcb}[box](ror){Side ration manuel}
	\lipsum[1]
	\smallbreak
	\begin{isd}(ror)<righthand ratio=.7>
		30\%
		\tcblower
		70\%
	\end{isd}
\end{tcb}

\begin{tcb*}[box](nota){Notation}
	\lipsum[1]
	\begin{isd*}
		Yes
		\tcblower
		No
	\end{isd*}
	\lipsum[2]
\end{tcb*}

\sde[box](prop)< right, >
{Super titre}{
	Rapide
}{
	$
		\begin{gathered}
			a + B = \frac{\pi}{2}
		\end{gathered}
	$
}


\sde*[box](defi)< left, dngr, csd>
{Titre}{
	Rapide
}{
	Éclair
}

\sde[lft](nota)<right, >{Titre génial}{
	\lipsum[2]
}{
	Of course.
}

\end{document}
