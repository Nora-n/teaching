\documentclass[a4paper, 11pt]{book}
\usepackage{cours-preambule}

\begin{document}

\begin{tcb*}(appl)<lftt>{Calorimétrie}
	Dans un calorimètre parfaitement isolé de masse en eau $m_0 = \SI{24}{g}$, on
	place $m_1 = \SI{150}{g}$ d'eau à $T_1 = \SI{298}{K}$. On ajoute $m_2 =
		\SI{100}{g}$ de cuivre à $T_2 = \SI{353}{K}$, avec $c_{\ce{Cu}} =
		\SI{385}{J.K^{-1}.kg^{-1}}$. On cherche la température d'équilibre $T_f$.
	\begin{enumerate}[label=\sqenumi]
		\item Exprimer $\Delta{H}\ind{eau}$ en fonction de $m_1$, $c\ind{eau}$,
		      $T_1$ et $T_f$.
		\item Exprimer $\Delta{H}_{\ce{Cu}}$ en fonction de $m_2$, $c_{\ce{Cu}}$,
		      $T_2$ et $T_f$.
		\item Exprimer $\Delta{H}\ind{calo}$ en fonction de $m_1$, $c\ind{eau}$,
		      $T_1$ et $T_f$.
		\item Justifier que $\Delta{H}\ind{tot} = 0$.
		\item En déduire $T_f$.
	\end{enumerate}
	\tcblower
	\begin{enumerate}[label=\sqenumi]
		\item[m]
			\psw{%
				\begin{gather*}
					\boxed{\Delta{H}\ind{eau} = m_1c\ind{eau} (T_f - T_1)}
				\end{gather*}
			}%
			\vspace{-25pt}
		\item[m]
			\psw{%
				\begin{gather*}
					\boxed{\Delta{H}_{\ce{Cu}} = m_2c_{\ce{Cu}} (T_f - T_2)}
				\end{gather*}
			}%
			\vspace{-25pt}
		\item[m]
			\psw{%
				\begin{gather*}
					\boxed{\Delta{H}\ind{calo} = m_0c\ind{eau} (T_f - T_1)}
				\end{gather*}
			}%
			\vspace{-25pt}
		\item \psw{%
			      Calorimètre isolé donc $Q = 0$, et pas de variation de volume donc
			      $W_p = 0$ et pas d'autres travaux donc $W_u = 0$~:
			      \[
				      \boxed{\Delta{H}\ind{tot} = 0}
			      \]
		      }%
		\item[m]
			\psw{%
				\begin{align*}
					(m_1 + m_0)c\ind{eau} (T_f - T_1) + m_2c_{\ce{Cu}}(T_f - T_2)
					    & = 0
					\\\Lra
					T_f \left( (m_1+m_0)c\ind{eau} + m_2c_{\ce{Cu}} \right)
					    & =
					T_1 \left( m_1+m_0 \right)c\ind{eau} + T_2m_2c_{\ce{Cu}}
					\\\Lra
					\Aboxed{
					T_f & =
					\frac{\left( m_1+m_0 \right)c\ind{eau}T_1 + m_2c_{\ce{Cu}}T_2}
					{(m_1+m_0)c\ind{eau} + m_2c_{\ce{Cu}}}
					}
					\\
					\makebox[0pt][l]{$\phantom{\AN}\xul{\phantom{T_f = \SI{301}{K}}}$}
					\AN
					T_f & = \SI{301}{K}
				\end{align*}
			}%
	\end{enumerate}
	\vspace{-25pt}
\end{tcb*}


\end{document}
