\documentclass[a4paper,11pt]{article}
\usepackage[svgnames, dvipsnames, table]{xcolor}
\usepackage{WriteOnGrid}
\usepackage{amsmath,amssymb}
\usepackage{lipsum}
\definecolor{myblue}{HTML}{0039a6}
\setlength{\jot}{.44cm}

\newcommand*\circled[1]{%
	\tikz[baseline=(char.base)]{%
		\node[shape=circle, draw, inner sep=1pt] (char) {#1};
	}%
}

\begin{document}
\pagestyle{empty}
\begin{PleinePageCinqCinq}[CouleurMarge=lightgray!50,NumLignes]
	\xdef\CCFullMargeG{3.7}
	\xdef\CCFullMargeH{1.3}
	\draw[red!75,thick]
	($(current page.north west)+(\CCFullMargeG,0)$) --
	($(current page.south west)+(\CCFullMargeG,0)$) ;
	\coordinate (CinqCinqOrigine) at
	($(current page.north west)+({\CCFullMargeG},{-\CCFullMargeH})$) ;

	\xdef\CCFullLargPap{24.7}
	\renewcommand\CadreNoteCinqCinq[2][3]{%on précise la {ligne de début} + [hauteur]
		%cadre de note
		\draw[thick, red]
		($(current page.north west)+(0,{(-#2+1)*0.5-\CCFullMargeH})$) --++
		({\CCFullLargPap-\CCFullMargeG},{0}) ;
		% \draw[thick, myblue]
		% ($(current page.north west)+(0,{(-#2+1)*0.5-\CCFullMargeH})$) --++
		% ({0},{-#1*0.5}) ;
		\draw[thick, red]
		($(current page.north west)+(0,{(-#2+1-#1)*0.5-\CCFullMargeH})$) --++
		({\CCFullLargPap-\CCFullMargeG},{0}) ;
		% \draw[thick, orange]
		% ($(current page.north west)+(0,{(-#2+1)*0.5-\CCFullMargeH})$) rectangle++
		% ({#1.*0.5},{-#1*0.5}) ;
		\draw[thick, red]
		($(current page.north west)+(0,{(-#2+1-#1)*0.5-\CCFullMargeH})$) --++
		({#1.*0.5+.2},{#1*0.5}) ;
	}
	\LignePapierCinqCinq[Echelle=1.25,Ligne=1,Largeur=17,Couleur=myblue]{%
		\hspace{-2.5cm}NICOLAS}
	\LignePapierCinqCinq[Echelle=1.25,Ligne=1,Largeur=17,Couleur=myblue]<center>{%
		\hspace{-1cm}{\Large TP 1 de Physique-Chimie}}
	\LignePapierCinqCinq[Echelle=1.25,Ligne=3,Largeur=17,Couleur=myblue]<center>{%
		\hspace{-1cm}{\Large Détermination de focales de lentilles}}
	\LignePapierCinqCinq[Echelle=1.25,Ligne=2,Largeur=17,Couleur=myblue]<right>{%
		Le 02/09/24}
	\LignePapierCinqCinq[Echelle=1.25,Ligne=2,Largeur=17,Couleur=myblue]{%
		\hspace{-2.5cm}Nora}
	\LignePapierCinqCinq[Echelle=1.25,Ligne=3,Largeur=17,Couleur=myblue]{%
		\hspace{-2.5cm}MPSI3}
	% \LignePapierCinqCinq[Echelle=1.25,Ligne=3,Couleur=red]<center>{\underline{\bfseries Devoir 2}}
	\CadreNoteCinqCinq[7]{5}
	\coordinate (CinqCinqOrigine) at
	($(current page.north west)+({\CCFullMargeG+0.5},{-\CCFullMargeH})$) ;
	\ParagraphePapierCinqCinq[Ligne=14,Couleur=myblue]{%
		\underline{\Large Objectifs}\\
		$\diamond$ Réaliser des alignements sur un banc optique~;\\
		$\diamond$ Reconnaître rapidement une lentille convergente et une lentille divergente~;\\
		$\diamond$ Déterminer une distance focale par différentes méthodes.\\[-.5cm]
		\hspace*{-0.52cm}\rule{\linewidth}{0.4pt}
	}
	\ParagraphePapierCinqCinq[Ligne=23,Couleur=myblue]{%
		\underline{\Large Observations et remarques personnelles}
		\\\\\\\\\\\hspace*{-0.52cm}\rule{\linewidth}{0.4pt}}
	%echelle de 1.25 et espacement de 10mm (2 lignes) := calcul 10/1.25 pour l'espacement
	% \ParagraphePapierCinqCinq[Ligne=16]{Remarques personnelles\\\rule{1.05\linewidth}{0.4pt}}
	% \ParagraphePapierCinqCinq[Ligne=38]
	% {%
	% 	On essaye avec des maths $1+\frac{1}{2}=\frac32$ en mode ligne avec des lignes assez longues pour voir
	% 	ce que ça peut donner\ldots Et une intégrale $\int_0^1 2x dx = 1$.\\
	% 	On essaye en passant à la ligne !!!
	% }
	\LignePapierCinqCinq[Ligne=35,Largeur=17,Couleur=myblue]{%
		\underline{\Large III) Analyser}}
	\ParagraphePapierCinqCinq[Ligne=37,Largeur=17,Couleur=myblue]{%
		\circled{1}
		Avec les notations de l'énoncé, la relation de Descartes devient
		\vspace*{-.3cm}
		\begin{align*}
			\frac{1}{f'}                 & = \frac{1}{D-x} - \frac{1}{-x} \\
			\Leftrightarrow \frac{1}{f'} & = \frac{x + D-x}{x(D-x)}       \\
			\Leftrightarrow f'           & = \ldots
		\end{align*}
	}
\end{PleinePageCinqCinq}
% \clearpage
% \pagestyle{empty}
% \begin{EnvQuadrillage}[%
% 		NbCarreaux=auto,
% 		% Elargir=10/0,
% 		Cadre=true,
% 		AffBarre=true, CouleurMarge=red!75, Marge=2
% 	]<\CoulSeyes>
% \end{EnvQuadrillage}
\end{document}
